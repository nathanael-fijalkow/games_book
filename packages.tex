%% better include hyperref early
\usepackage{url,hyperref}
\usepackage{pdfpages}

%% \let\qed\relax
%% \let\qedsymbol\relax
%% \let\smartqed\relax
%% \let\proof\relax
%% \let\endproof\relax
%% \let\theoremstyle\relax
%% \let\newtheoremstyle\relax
%% \usepackage{amsthm}
\usepackage{amsmath}
\usepackage{amssymb}
\usepackage{thmtools}

\newtheorem{convention}{Convention}

\usepackage{mathptmx}
%\usepackage{helvet}
\usepackage{courier}

\usepackage{type1cm}         

\usepackage{graphicx}       
\usepackage{xcolor}      

\usepackage[bottom]{footmisc}

\usepackage[obeyFinal,colorinlistoftodos]{todonotes}
%% note NM: \todo conflicts with \qed of package ntheorem (loaded by svmono)
%% also, the qed symbol of ntheorem is a bit complex to manipulate
%% (see https://tex.stackexchange.com/questions/197775/qed-sign-does-not-always-occur-with-package-ntheorem)
%% ntheorem definit des ``mark'' dans book.aux aux endroits ou il veut mettre de qed. Cf. option thmmarks.
%% voir https://tex.stackexchange.com/questions/296434/with-ntheorem-thmmarks-and-no-amsthm-defining-the-proof-environment-right

\usepackage{multirow,array,booktabs}
\usepackage{pdflscape}
\usepackage{afterpage}
%\usepackage{caption}
%\usepackage{subcaption}
%%%%%% Complexity package %%%%%%%%%%

\usepackage[small,full]{complexity}

%%%%%% Algorithm package %%%%%%%%%%

\usepackage[norelsize,ruled,algochapter,vlined]{algorithm2e}

%%%%%% Font package %%%%%%%%%%
\usepackage{dsfont}

%%%%%% References package %%%%%%%%%%

%\usepackage{url}
\usepackage{cleveref}

\crefname{algocf}{Algorithm}{Algorithms}
\Crefname{algocf}{Algorithm}{Algorithms}
\crefname{section}{Sect.}{Sects.}
\crefname{subsection}{Sect.}{Sects.}
\crefname{subsubsection}{Sect.}{Sects.}
\Crefname{section}{Section}{Sections}
\Crefname{subsection}{Section}{Sections}
\Crefname{subsubsection}{Section}{Sections}
\crefname{chapter}{Chap.}{Chaps.}
\Crefname{chapter}{Chapter}{Chapters}
\crefname{page}{p.}{pp.}
\Crefname{page}{Page}{Pages}
\crefname{figure}{Fig.}{Figs.}
\Crefname{figure}{Figure}{Figures}
\crefname{table}{Table}{Tables}
\Crefname{table}{Table}{Tables}
\crefname{equation}{}{}
\Crefname{equation}{Equation}{Equations}
\crefname{theorem}{Thm.}{Thms.}
\Crefname{theorem}{Theorem}{Theorems}
\crefname{problem}{Pb.}{Pbs.}
\Crefname{problem}{Problem}{Problems}
\crefname{lemma}{Lem.}{Lems.}
\Crefname{lemma}{Lemma}{Lemmata}
\crefname{corollary}{Cor.}{Cors.}
\Crefname{corollary}{Corollary}{Corollaries}
\crefname{claim}{Claim}{Claims}
\Crefname{claim}{Claim}{Claims}
\Crefname{corollary}{Corollary}{Corollaries}
\crefname{proposition}{Prop.}{Props.}
\Crefname{proposition}{Proposition}{Propositions}
\crefname{definition}{Def.}{Defs.}
\Crefname{definition}{Definition}{Definitions}
\crefname{claim}{Claim}{Claims}
\Crefname{claim}{Claim}{Claims}
\crefname{fact}{Fact}{Facts}
\Crefname{fact}{Fact}{Facts}
\crefname{example}{Ex.}{Exs.}
\Crefname{example}{Example}{Examples}
\crefname{remark}{Rmk.}{Rmks.}
\Crefname{remark}{Remark}{Remarks}
\crefname{algorithm}{Alg.}{Algs.}
\Crefname{algorithm}{Algorithm}{Algorithms}

\usepackage{wasysym} % for frownie/smiley (Chap. 6)
% !!! has to be loaded before knowledge !!!
\usepackage{bookmark}

%%%%%% Knowledge package %%%%%%%%%%

\usepackage{makeidx}         % allows index generation
\usepackage{multicol}        % used for the two-column index
\usepackage[hyperref,quotation,notion]{knowledge}
\def\knowledgeIntroIndexStyle#1{\textbf{#1}}

%%%%%% Local package %%%%%%%
%\usepackage{wasysym} % for frownie/smiley
%\def\smiley{+}
%\def\frownie{-}
%\def\wasyfamily{\fontencoding{U}\fontfamily{wasy}\selectfont}
%\def\smiley     {\mbox{\wasyfamily\char44}}
%\def\frownie    {\mbox{\wasyfamily\char47}}


%%%%%% Tikz package %%%%%%%%%%

\usepackage{tikz}
\usetikzlibrary{arrows}
\usetikzlibrary{automata}
\usetikzlibrary{shapes,snakes}
\usetikzlibrary{calc}
\usetikzlibrary{patterns}
\usetikzlibrary{shapes.geometric}
\usetikzlibrary{positioning}
\tikzstyle{every node}=[font=\small]
\tikzstyle{eve}=[circle,minimum size=.3cm,draw=gray!90,inner sep=1pt,fill=gray!20,very thick]
\tikzstyle{adam}=[rounded corners=.5,regular polygon,regular polygon
  sides=4,minimum size=.4cm,draw=gray!90,inner sep=1pt,fill=gray!20,very thick]
\tikzstyle{every edge}=[draw,>=stealth',shorten >=1pt]
\tikzstyle{win}=[fill=green!50,draw=green!70!black]
\tikzstyle{lose}=[fill=white,draw=red!70!black]

%% added by NM, used but apparently not defined elsewhere:
\tikzstyle{state}=[draw,circle,minimum size=5mm]
\tikzstyle{accepting}=[double]

%%%%%% This one must go after Knowledge %%%%%%%%

\usepackage{subcaption}

\usepackage{mathtools}

%\usepackage{tcolorbox}

