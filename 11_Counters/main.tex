

%*** General probabilistic notation ***

\newcommand{\expv}{\mathbf{E}} % EXP. VALUE
\newcommand{\discProbDist}{f} % Discrete prob distribution
\newcommand{\sampleSpace}{S} % Generic sample space
\newcommand{\sigmaAlg}{\mathcal{F}} % Generic sigma-algebra
\newcommand{\probm}{\mathbb{P}} % Generic probability measure, also prob. measure operator
\newcommand{\rvar}{X} % Generic random variable
%\newcommand{\dist}{\mathit{Dist}}

%*** MDP notation ***

\newcommand{\actions}{A} % The set of actions.
\newcommand{\colouring}{c} % the colouring function
\newcommand{\probTranFunc}{\Delta} % Transition function of an MDP
\newcommand{\edges}{E} % Set of edges in an MDP.
\newcommand{\colours}{C} % The set of colours in an MDP.
\newcommand{\mdp}{\mathcal{M}} % A generic MDP. 
\newcommand{\vinit}{v_0} % An initial vertex in an MDP.
\newcommand{\cylProb}{p} % Function assigning probabilities to cylinder sets in 
%the measure construction.
\newcommand{\emptyPlay}{\epsilon} %empty play
\newcommand{\objective}{\Omega} % Qualitative objective
\newcommand{\genColour}{\textsc{c}} % Generic colour
\newcommand{\quantObj}{f} % Generic quantitative objective
\newcommand{\indicator}[1]{\mathbf{1}_{#1}} % In.d RV
\newcommand{\eps}{\varepsilon} % Numerical epsilon
\newcommand{\maxc}{W} % Maximal abs. value of a colour

\newcommand{\winPos}{W_{>0}}
\newcommand{\winAS}{W_{=1}}
\newcommand{\cylinder}{\mathit{Cyl}}

\newcommand{\PrePos}{\text{Pre}_{>0}}
\newcommand{\PreAS}{\text{Pre}_{=1}}

\newcommand{\PreOPPos}{\mathcal{P}_{>0}}
\newcommand{\OPAS}{\mathcal{P}_{=1}}

\newcommand{\safeOP}{\mathit{Safe_{=1}}}
\newcommand{\closed}{\mathit{Cl}}

\newcommand{\reachOP}{\mathcal{V}}
\newcommand{\discOP}{\mathcal{D}}
\newcommand{\valsigma}{\vec{x}^{\sigma}}

\newcommand{\lp}{\mathcal{L}}
\newcommand{\lpdisc}{\lp_{\mathit{disc}}}
\newcommand{\lpmp}{\lp_{\mathit{mp}}}
\newcommand{\lpsol}[1]{\bar{#1}}
\newcommand{\lpmpdual}{\lpmp^{\mathit{dual}}}

\newcommand{\actevent}[3]{\actions^{#1}_{#2,#3}} % Returns #1-th action on the run 

\newcommand{\MeanPayoffSup}{\MeanPayoff^{+}}
\newcommand{\MeanPayoffInf}{\MeanPayoff^{-}}

\newcommand{\mcprob}{M}
\newcommand{\invdist}{\vec{z}}

\newcommand{\hittime}{T}


\knowledge{vector addition system with states}
  [Vector addition systems with states|vector addition systems with states]
  {notion,index={vector!addition system with states}}
\knowledge{Minsky machine}[Minsky machines]
  {notion,index={Minsky machine}}
\knowledge{deterministic Minsky machine}[deterministic Minsky machines]
  {notion,index={Minsky machine!deterministic}}
\knowledge{alternating Minsky machine}[alternating Minsky machines]
  {notion,index={Minsky machine!alternating}}
\knowledge{halting problem}{notion,index={Minsky machine!halting problem}}
\knowledge{vector game}
  [Vector games|vector games]
  {notion,index={vector!game}}
\knowledge{Petri net}[Petri nets]{notion,index={Petri net}}
\knowledge{one-counter game}[one-counter games|One-counter games]
  {notion,index={one-counter game}}
\knowledge{succinct}[succinct one-counter game|succinct one-counter games|Succinct one-counter games]
  {notion,index={one-counter game!succinct}}
\knowledge{countdown game}[Countdown games|countdown games]
  {notion,index={countdown!game}}
\knowledge{zero reachability}[Zero reachability]
  {notion,index={countdown!game!zero reachability}}
\knowledge{countdown system}[Countdown systems|countdown systems]
  {notion,index={countdown!system}}
\knowledge{vector system}
  [vector systems|Vector systems]
  {notion,index={vector!system}}
\knowledge{natural semantics}
  {notion,index={natural semantics}}
\knowledge{energy semantics}[Energy semantics]
  {notion,index={energy!semantics}}
\knowledge{integer semantics}
  {notion}
\knowledge{bounded semantics}
  {notion,index={bounded semantics}}
\knowledge{capped semantics}
  {notion,index={capped semantics}}
\knowledge{bounding game}[bounding games|Bounding games]
  {notion,index={bounding game}}
\knowledge{capping game}[capping games|Capping games]
  {notion,index={capping game}}
\knowledge{sink}
  {notion}
\knowledge{total}{notion}
\knowledge{configuration colouring}[configuration
  colourings|Configuration
  colourings]{notion,index={colouring!configuration}}
\knowledge{location colouring}[location colourings|Location
  colourings]{notion,index={colouring!location}}
\knowledge{configuration reachability}[Configuration
reachability|configuration reachability vector game]
                    {notion,index={vector!game!configuration
  reachability}}
\index{configuration reachability|see{vector game}}
\knowledge{coverability}[Coverability|coverability game|coverability
  vector game]{notion,index={vector!game!coverability}}
\index{coverability|see{vector game}}
\knowledge{parity vector game}[Parity vector games|parity vector games]
                    {notion,index={vector!game!parity}}
\knowledge{non-termination}[Non-termination|non-termination
  game|non-termination games|non-termination vector game]
                    {notion,index={vector!game!non-termination}}
\index{non-termination|see{vector game}}
\knowledge{given initial credit}[given]
                    {notion,index={initial credit!given}}
\knowledge{existential initial credit}[existential]
                    {notion,index={initial credit!existential}}
\knowledge{zero vector}{notion}
\knowledge{unit vector}[unit vectors]{notion}
\knowledge{infinity norm}[norm|norms]{notion}
\knowledge{zero test}[zero tests]{notion}
\knowledge{Pareto limit}[Pareto limits]{notion,index={Pareto!limit}}
\knowledge{well-quasi-order}[wqo|wqos|well-quasi-orders|Well-quasi-orders|well-quasi-ordered]
  {notion,index={well-quasi-order}}
\knowledge{energy game}[energy games|Energy
  games]{notion,index={energy!game}}
\knowledge{asymmetry}[asymmetric|Asymmetry|Asymmetric|asymmetric
vector system|asymmetric vector systems|Asymmetric vector
  systems|asymmetric vector game|asymmetric vector games|Asymmetric
  vector games|asymmetric games]
  {notion,index={vector!system!asymmetric}}
\knowledge{monotonic objective}[monotonic|monotonic objectives|Monotone
  objectives|monotonic vector game|monotonic vector games|Monotone
  vector games]
  {notion,index={vector!game!monotonic}}
\knowledge{hit-or-run game}[hit-or-run games]
  {notion,index={hit-or-run game}}
\knowledge{robot game}[robot games]
  {notion,index={robot game}}
\knowledge{quasi-order}[quasi-orders]
  {notion}
\knowledge{upward closure}[upward closures]
  {notion}
\knowledge{upwards closed}{notion}
\knowledge{downward closure}[downward closures]
  {notion}
\knowledge{downwards closed}{notion}
\knowledge{principal filter}[principal filters|filters|filter]{notion}
\knowledge{principal ideal}[principal ideals]{notion}
\knowledge{well-founded}{notion}
\knowledge{finite antichain condition}{notion}
\knowledge{ascending chain condition}{notion}
\knowledge{finite basis property}{notion}
%\knowledge{Pareto bound}[Pareto bounds]{notion,index={Pareto!bound}}
%\knowledge{viable}{notion}
\knowledge{bounding game}[bounding|Bounding games|bounding
games]{notion,index={bounding game}}
\knowledge{good sequence}[good]{notion}
\knowledge{self-covering tree}[self-covering trees|Self-covering trees]{notion}
\knowledge{return node}{notion}
%\knowledge{correct}{notion}
\knowledge{counterless}[counterless strategy|counterless
  strategies|Counterless
  strategies]{notion,index={strategy!counterless}}
\knowledge{simulate}[simulating|simulates]
  {notion}

Just like timed games arise from timed systems and pushdown games
from pushdown systems, counter games arise from (multi-)counter
systems.  Those are finite-state systems further endowed with a
finite number of counters whose values range over the natural numbers,
and are widely used to model and reason about systems handling
discrete resources.  Such resources include for instance money on a
bank account, items on a factory line, molecules in chemical
reactions, organisms in biological ones, replicated processes in
distributed computing, etc.  As with timed or pushdown systems,
counter systems give rise to infinite graphs that can be turned into
infinite game arenas.

\AP One could populate a zoo with the many variants of counter systems,
depending on the available counter operations.  One of the best known
specimens in this zoo are ""Minsky machines""~\cite{Minsky:1967},
where the operations are incrementing a counter, decrementing it, or
testing whether its value is zero.  "Minsky machines" are a universal
model of computation: their reachability problem is undecidable,
already with only two counters.  From the algorithmic perspective we
promote in this book, this means that the counter games arising from
"Minsky machines" are not going to be very interesting, unless perhaps
if we restrict ourselves to a single counter.  A more promising
species in our zoo are \emph{"vector addition systems with
  states"}~\cite{Greibach:1978,Hopcroft&Pansiot:1979}---or,
equivalently, ""Petri nets""~\cite{Petri:1962}---, where the only
available operations are increments and decrements.  "Vector addition
systems with states" enjoy a decidable reachability
problem~\cite{Mayr:1981,Kosaraju:1982,Lambert:1992,Leroux:2011}, which
makes them a much better candidate for studying the associated games.

In this chapter, we focus on "vector games", that is, on games defined
on arenas defined by "vector addition systems with states" with a
partition of states controlled by~Eve and Adam.  As we are going to
see in \cref{11-sec:counters}, those games turn out to be undecidable
already for quite restricted objectives and just two counters.  We
then investigate two restricted classes of "vector games".
\begin{enumerate}
\item In \cref{11-sec:dim1}, we consider \emph{"one-counter games"}.  These can
  be reduced to the pushdown games of \cref{10-chap:pushdown} and are
  therefore decidable.  Most of the section is thus devoted to proving
  sharp complexity lower bounds, already in the case of so-called
  \emph{"countdown games"}.
\item In \cref{11-sec:avag}, we turn our attention to the main results of
  this chapter.  By suitably restricting both
  %\begin{enumerate}
  %\item
  the systems, with an
  \emph{"asymmetry"} condition that forbids Adam to manipulate the
  counters, and
  %\item
  the "objective", with a \emph{"monotonicity@monotonic objective"}
    condition that ensures that Eve\'s winning region is "upwards
    closed"---meaning that larger counter values make it easier for
    her to win---,
  %\end{enumerate}
  one obtains a class of decidable "vector games" where "finite
  memory" strategies are sufficient.
  \begin{itemize}
  \item   This class is still rich enough to find many applications, and we
  zoom in on the connections with resource-conscious games like
  "\emph{energy} games" and "\emph{bounding} games" in
  \cref{11-sec:resource}---a subject that will be taken further in
  \cref{12-chap:multiobjective}.
  
  \item The computational complexity of "asymmetric" "monotonic@monotonic
  objective" "vector games" is now well-understood, and we devote
  \cref{11-sec:complexity} to the topic; \cref{11-tbl:cmplx} at the end of
  the chapter summarises these results.
  \end{itemize}
\end{enumerate}

\begin{acknowledgement}
  Thanks to ... Work partially funded by ANR-17-CE40-0028 \textsc{BraVAS}.
\end{acknowledgement}
\ifstandalone
\listoftodos[Todo List]
\tableofcontents
\fi

\section{Vector games}
\label{11-sec:counters}
\AP A ""vector system"" is a finite directed graph with a partition of
the vertices and weighted edges.  Formally, it is a tuple
$\?V=(\Loc,\Act,\Loc_\mEve,\Loc_\mAdam,\dd)$ where $\dd\in\+N$ is a
dimension, $\Loc$ is a finite set of locations partitioned into the
locations controlled by \Eve\ and \Adam, i.e.,
$\Loc=\Loc_\mEve\uplus \Loc_\mAdam$, and
$\Act\subseteq \Loc\times\+Z^\dd\times\Loc$ is a finite set of
weighted actions.  We write $\loc\step{\vec u}\loc'$
rather than $(\loc,\vec u,\loc')$ for actions in~$\Act$.  A
""vector addition system with states"" is a "vector system" where
$\Loc_\mAdam=\emptyset$, i.e., it corresponds to the one-player case.

\begin{example}[vector system]
\label{11-ex:mwg}
  \Cref{11-fig:mwg} presents a "vector system" of
  dimension two with locations $\{\loc,\loc'\}$ where~$\loc$ is
  controlled by \Eve\ and $\loc'$ by \Adam.% , and actions
  % $\{\loc\step{-1,-1}\loc,\loc\step{-1,0}\loc',\loc'\step{-1,0}\loc,\loc'\step{2,1}\loc\}$  
\end{example}
\begin{figure}[htbp]
  \centering
  \begin{tikzpicture}[auto,on grid,node distance=2.5cm]
    \node[s-eve](0){$\loc$};
    \node[s-adam,right=of 0](1){$\loc'$};
    \path[arrow,every node/.style={font=\footnotesize,inner sep=1}]
    (0) edge[loop left] node {$-1,-1$} ()
    (0) edge[bend right=10] node {$-1,0$} (1)
    (1) edge[bend left=30] node {$-1,0$} (0)
    (1) edge[bend right=30,swap] node {$2,1$} (0);
  \end{tikzpicture}
  \caption{A "vector system".}\label{11-fig:mwg}
\end{figure}

The intuition behind a "vector system" is that it
maintains~$\dd$ counters $\mathtt{c}_1,\dots,\mathtt{c}_\dd$ assigned
to integer values.  An action $\loc\step{\vec u}\loc'\in\Act$ then
updates each counter by adding the corresponding entry of~$\vec u$,
that is for all $1\leq j\leq\dd$, the action performs the update
$\mathtt{c}_j := \mathtt{c}_j+\vec u(j)$.

\medskip \AP Before we proceed any further, let us fix some notations
for vectors in $\+Z^\dd$.  We write `$\vec 0$' for the ""zero vector""
with $\vec 0(j)\eqdef 0$ for all $1\leq j\leq\dd$.  For all
$1\leq j\leq\dd$, we write `$\vec e_j$' for the ""unit vector"" with
$\vec e_j(j)\eqdef 1$ and $\vec e_{j}(j')\eqdef 0$ for all $j'\neq j$.
Addition and comparison are defined componentwise, so that for
instance $\vec u\leq\vec u'$ if and only if for all $1\leq j\leq\dd$,
$\vec u(j)\leq\vec u'(j)$.  We write
$\weight(\loc\step{\vec u}\loc')\eqdef\vec u$ for the weight of an
action and $\weight(\pi)\eqdef\sum_{1\leq j\leq |\pi|}\weight(\pi_j)$
for the cumulative weight of a finite sequence of actions
$\pi\in\Act^\ast$.  For a vector $\vec u\in\+Z^\dd$, we use its
""infinity norm"" $\|\vec u\|\eqdef\max_{1\leq j\leq\dd}|\vec u(j)|$,
hence $\|\vec 0\|=0$ and $\|\vec e_j\|=\|-\vec e_j\|=1$, and we let
$\|\loc\step{\vec u}\loc'\|\eqdef\|\weight(\loc\step{\vec
  u}\loc')\|=\|\vec u\|$
and $\|\Act\|\eqdef\max_{a\in\Act}\|\weight(a)\|$.  Unless stated
otherwise, we assume that all our vectors are represented in binary,
hence $\|\Act\|$ may be exponential in the size of~$\?V$.

\subsection{Arenas and Games}
%\paragraph{Natural Semantics}
\AP A "vector system" gives rise to an infinite graph
$G_\+N\eqdef(V,E)$ over the set of vertices
$V\eqdef(\Loc\times\+N^\dd)\uplus\{\sink\}$.  The vertices of the
graph are either \emph{configurations} $\loc(\vec v)$ consisting of a
location $\loc\in \Loc$ and a vector of non-negative integers
$\vec v\in\+N^\dd$---such a vector represents a valuation of the
counters $\mathtt{c}_1,\dots,\mathtt c_\dd$---, or the
""sink""~$\sink$.

\AP Consider an action in~$a=(\loc\step{\vec u}\loc')$ in~$\Act$: we
see it as a partial function
$a{:}\,\Loc\times\+N^\dd\,\pto \Loc\times\+N^\dd$ with domain
$\dom a\eqdef\{\loc(\vec v)\mid \vec v+\vec u\geq\vec 0\}$ and image
$a(\loc(\vec v))\eqdef \loc'(\vec v+\vec u)$; let also
$\dom\Act\eqdef\bigcup_{a\in\Act}\dom a$.  This allows us to define
the set~$E$ of edges as a set of pairs
\begin{align*}
  E&\eqdef\{(\loc(\vec v),a(\loc(\vec v)))\mid a\in\Act\text{ and
     }\loc(\vec v)\in\dom a\}\\%\loc'(\vec v+\vec u))\mid
                               %\loc\step{\vec u}\loc'\in\Act\}\\
  &\:\cup\:\{(\loc(\vec v),\sink)\mid\loc(\vec v)\not\in\dom\Act\}\cup\{(\sink,\sink)\}\;,
\end{align*}
where $\ing((v,v'))\eqdef v$ and $\out((v,v'))\eqdef v'$ for all
edges~$(v,v')\in E$.  There is therefore an edge $(v,v')$ between two
configurations $v=\loc(\vec v)$ and $v'=\loc'(\vec v')$ if there
exists an action $\loc\step{\vec u}\loc'\in\Act$ such that
$\vec v'=\vec v+\vec u$.  Note that, quite importantly,
$\vec v+\vec u$ must be non-negative on every coordinate for this edge
to exist.  If no action can be applied, there is an edge to the
"sink"~$\sink$, which ensures that $E$ is ""total"": for all $v\in V$,
there exists an edge $(v,v')\in E$ for some $v'$, and thus there are
no `deadlocks' in the graph.

The configurations are naturally partitioned into those in
$\VE\eqdef\Loc_\mEve\times\+N^\dd$ controlled by~\Eve\ and those in
$\VA\eqdef\Loc_\mAdam\times\+N^\dd$ controlled by~\Adam.  Regarding
the "sink", the only edge starting from~$\sink$ loops back
to it, and it does not matter who of \Eve\ or \Adam\ controls it.  This
gives rise to an infinite arena $\arena_\+N\eqdef(G_\+N,\VE,\VA)$ called
the ""natural semantics"" of~$\?V$.

\medskip Although we work in a turn-based setting with perfect
information, it is sometimes enlightening to consider the partial map
$\dest{:}\,V\times A\pto E$ defined by
$\dest(\loc(\vec v),a)\eqdef(\loc(\vec v),a(\loc(\vec v)))$ if
$\loc(\vec v)\in\dom a$ and
$\dest(\loc(\vec v),a)\eqdef(\loc(\vec v),\sink)$ if
$\loc(\vec v)\not\in\dom\Act$.  Note that a sequence~$\pi$ over $E$
that avoids the "sink" can also be described by an initial
configuration $\loc_0(\vec v_0)$ paired with a sequence
over~$\Act$.%   Because we work with qualitative objectives, we
% will rather use a vertex colouring $\col{:}\,V\to C$ and allow it to
% vary depending on the kind of game we are playing.

\begin{example}[natural semantics]
{11-ex:sem}
  \Cref{11-fig:sem} illustrates the "natural semantics" of the system
  of~\cref{11-fig:mwg}; observe that all the configurations $\loc(0,n)$
  for $n\in\+N$ lead to the "sink".
\end{example}

\begin{figure}[htbp]
  \centering\scalebox{.77}{
  \begin{tikzpicture}[auto,on grid,node distance=2.5cm]
    \draw[step=1,lightgray!50,dotted] (-5.7,0) grid (5.7,3.8);
    \node at (0,3.9) (sink) {\boldmath$\sink$};
    \draw[step=1,lightgray!50] (1,0) grid (5.5,3.5);
    \draw[step=1,lightgray!50] (-1,0) grid (-5.5,3.5);
    \draw[color=white](0,-.3) -- (0,3.8);
    \node at (0,0)[lightgray,font=\scriptsize,fill=white] {0};
    \node at (0,1)[lightgray,font=\scriptsize,fill=white] {1};
    \node at (0,2)[lightgray,font=\scriptsize,fill=white] {2};
    \node at (0,3)[lightgray,font=\scriptsize,fill=white] {3};
    \node at (1,3.9)[lightgray,font=\scriptsize,fill=white] {0};
    \node at (2,3.9)[lightgray,font=\scriptsize,fill=white] {1};
    \node at (3,3.9)[lightgray,font=\scriptsize,fill=white] {2};
    \node at (4,3.9)[lightgray,font=\scriptsize,fill=white] {3};
    \node at (5,3.9)[lightgray,font=\scriptsize,fill=white] {4};
    \node at (-1,3.9)[lightgray,font=\scriptsize,fill=white] {0};
    \node at (-2,3.9)[lightgray,font=\scriptsize,fill=white] {1};
    \node at (-3,3.9)[lightgray,font=\scriptsize,fill=white] {2};
    \node at (-4,3.9)[lightgray,font=\scriptsize,fill=white] {3};
    \node at (-5,3.9)[lightgray,font=\scriptsize,fill=white] {4};
    \node at (1,0)[s-eve-small] (e00) {};
    \node at (1,1)[s-adam-small](a01){};
    \node at (1,2)[s-eve-small] (e02){};
    \node at (1,3)[s-adam-small](a03){};
    \node at (2,0)[s-adam-small](a10){};
    \node at (2,1)[s-eve-small] (e11){};
    \node at (2,2)[s-adam-small](a12){};
    \node at (2,3)[s-eve-small] (e13){};
    \node at (3,0)[s-eve-small] (e20){};
    \node at (3,1)[s-adam-small](a21){};
    \node at (3,2)[s-eve-small] (e22){};
    \node at (3,3)[s-adam-small](a23){};
    \node at (4,0)[s-adam-small](a30){};
    \node at (4,1)[s-eve-small] (e31){};
    \node at (4,2)[s-adam-small](a32){};
    \node at (4,3)[s-eve-small] (e33){};
    \node at (5,0)[s-eve-small] (e40){};
    \node at (5,1)[s-adam-small](a41){};
    \node at (5,2)[s-eve-small] (e42){};
    \node at (5,3)[s-adam-small](a43){};
    \node at (-1,0)[s-adam-small](a00){};
    \node at (-1,1)[s-eve-small] (e01){};
    \node at (-1,2)[s-adam-small](a02){};
    \node at (-1,3)[s-eve-small] (e03){};
    \node at (-2,0)[s-eve-small] (e10){};
    \node at (-2,1)[s-adam-small](a11){};
    \node at (-2,2)[s-eve-small] (e12){};
    \node at (-2,3)[s-adam-small](a13){};
    \node at (-3,0)[s-adam-small](a20){};
    \node at (-3,1)[s-eve-small] (e21){};
    \node at (-3,2)[s-adam-small](a22){};
    \node at (-3,3)[s-eve-small] (e23){};
    \node at (-4,0)[s-eve-small] (e30){};
    \node at (-4,1)[s-adam-small](a31){};
    \node at (-4,2)[s-eve-small] (e32){};
    \node at (-4,3)[s-adam-small](a33){};
    \node at (-5,0)[s-adam-small](a40){};
    \node at (-5,1)[s-eve-small] (e41){};
    \node at (-5,2)[s-adam-small](a42){};
    \node at (-5,3)[s-eve-small] (e43){};
    \path[arrow] % l, -1,-1, l
    (e11) edge (e00)
    (e22) edge (e11)
    (e31) edge (e20)
    (e32) edge (e21)
    (e21) edge (e10)
    (e12) edge (e01)
    (e23) edge (e12)
    (e33) edge (e22)
    (e13) edge (e02)
    (e43) edge (e32)
    (e42) edge (e31)
    (e41) edge (e30);
    \path[arrow] % l, -1,0, l'
    (e11) edge (a01)
    (e20) edge (a10)
    (e22) edge (a12)
    (e31) edge (a21)
    (e32) edge (a22)
    (e21) edge (a11)
    (e12) edge (a02)
    (e30) edge (a20)
    (e10) edge (a00)
    (e13) edge (a03)
    (e23) edge (a13)
    (e33) edge (a23)
    (e43) edge (a33)
    (e42) edge (a32)
    (e41) edge (a31)
    (e40) edge (a30);
    \path[arrow] % l', -1,0, l
    (a11) edge (e01)
    (a20) edge (e10)
    (a22) edge (e12)
    (a31) edge (e21)
    (a32) edge (e22)
    (a21) edge (e11)
    (a12) edge (e02)
    (a30) edge (e20)
    (a10) edge (e00)
    (a33) edge (e23)
    (a23) edge (e13)
    (a13) edge (e03)
    (a43) edge (e33)
    (a42) edge (e32)
    (a41) edge (e31)
    (a40) edge (e30);
    \path[arrow] % l', 2,1, l
    (a01) edge (e22)
    (a10) edge (e31)
    (a11) edge (e32)
    (a00) edge (e21)
    (a02) edge (e23)
    (a12) edge (e33)
    (a22) edge (e43)
    (a21) edge (e42)
    (a20) edge (e41);
    \path[arrow] % dotted to Eve
    (-5.5,3.5) edge (e43)
    (5.5,2.5) edge (e42)
    (2.5,3.5) edge (e13)
    (5.5,0.5) edge (e40)
    (-5.5,1.5) edge (e41)
    (-3.5,3.5) edge (e23)
    (-1.5,3.5) edge (e03)
    (4.5,3.5) edge (e33)
    (5.5,0) edge (e40)
    (5.5,2) edge (e42)
    (-5.5,1) edge (e41)
    (-5.5,3) edge (e43);
    \path[dotted]
    (-5.7,3.7) edge (-5.5,3.5)
    (5.7,2.7) edge (5.5,2.5)
    (2.7,3.7) edge (2.5,3.5)
    (5.7,0.7) edge (5.5,0.5)
    (-3.7,3.7) edge (-3.5,3.5)
    (-1.7,3.7) edge (-1.5,3.5)
    (4.7,3.7) edge (4.5,3.5)
    (-5.7,1.7) edge (-5.5,1.5)
    (5.75,0) edge (5.5,0)
    (5.75,2) edge (5.5,2)
    (-5.75,1) edge (-5.5,1)
    (-5.75,3) edge (-5.5,3);
    \path[arrow]
    (5.5,1) edge (a41)
    (-5.5,2) edge (a42)
    (-5.5,0) edge (a40)
    (5.5,3) edge (a43);
    \path[dotted]
    (5.75,1) edge (5.5,1)
    (-5.75,2) edge (-5.5,2)
    (-5.75,0) edge (-5.5,0)
    (5.75,3) edge (5.5,3);
    \path[-]
    (a30) edge (5.5,.75)
    (a32) edge (5.5,2.75)
    (a31) edge (-5.5,1.75)
    (a23) edge (4,3.5)
    (a03) edge (2,3.5)
    (a13) edge (-3,3.5)
    (a33) edge (-5,3.5)
    (a43) edge (5.5,3.25)
    (a41) edge (5.5,1.25)
    (a40) edge (-5.5,0.25)
    (a42) edge (-5.5,2.25);
    \path[dotted]
    (5.5,.75) edge (5.8,.9)
    (5.5,2.75) edge (5.8,2.9)
    (-5.5,1.75) edge (-5.8,1.9)
    (4,3.5) edge (4.4,3.7)
    (2,3.5) edge (2.4,3.7)
    (-3,3.5) edge (-3.4,3.7)
    (-5,3.5) edge (-5.4,3.7)
    (5.5,3.25) edge (5.8,3.4)
    (5.5,1.25) edge (5.8,1.4)
    (-5.5,.25) edge (-5.8,0.4)
    (-5.5,2.25) edge (-5.8,2.4);
    \path[arrow]
    (sink) edge[loop left] ()
    (e00) edge[bend left=8] (sink)
    (e01) edge[bend right=8] (sink)
    (e02) edge[bend left=8] (sink)
    (e03) edge[bend right=8] (sink);
  \end{tikzpicture}}
  \caption{The "natural semantics" of the
    "vector system" of \cref{11-fig:mwg}: a circle (resp.\
    a square) at position $(i,j)$ of the grid denotes a configuration
    $\loc(i,j)$ (resp.\ $\loc'(i,j)$) controlled by~\Eve\ (resp.\
    \Adam).}
   \label{11-fig:sem}
\end{figure}

% \begin{remark}[Integer Semantics]Note that an alternative, ""integer
%   semantics"" would use instead $\Loc\times\+Z^\dd$ as set of vertices,
%   and
%   $\{(\loc(\vec v),\loc'(\vec v+\vec u))\mid \loc\step{\vec
%   u}\loc'\in\Act\}$
%   as set of edges.  This eschews the need of a "sink" vertex since there
%   are no deadlocks, but gives rise to somewhat less interesting games.
% \end{remark}

%\paragraph{Initial Credit}
\AP A "vector system" $\?V=(\Loc,\Act,\Loc_\mEve,\Loc_\mAdam,\dd)$, a
colouring~$\col{:}\,E\to C$, and an
objective~$\Omega\subseteq C^\omega$ together define a ""vector game""
$\game=(\natural(\?V),\col,\Omega)$.  Because $\natural(\?V)$ is an
infinite arena, we need to impose restrictions on our "colourings"
$\col{:}\,E\to C$ and the "qualitative
objectives"~$\Omega\subseteq C^\omega$; at the very least, they should
be recursive.

There are then two variants of the associated decision problem:
\begin{itemize}
\item\AP the ""given initial credit"" variant, where we are given $\?V$,
  $\col$, $\Omega$, a location $\loc_0\in\Loc$ and an initial credit
  $\vec v_0\in\+N^\dd$, and ask whether \Eve\ wins~$\game$ from the
  initial configuration~$\loc_0(\vec v_0)$;
\item\AP the ""existential initial credit"" variant, where we are given
  $\?V$, $\col$, $\Omega$, and a location $\loc_0\in\Loc$, and ask
  whether there exists an initial credit $\vec v_0\in\+N^\dd$ such
  that \Eve\ wins~$\game$ from the initial
  configuration~$\loc_0(\vec v_0)$.
\end{itemize}

%\paragraph{Reachability and Coverability}
Let us instantiate the previous abstract definition of "vector games".
We first consider two `"reachability"-like'
\index{reachability!\emph{see also} vector game\protect\mymoot|mymoot}
objectives, where $C\eqdef\{\varepsilon,\Win\}$ and
$\Omega\eqdef\Reach$, namely "configuration reachability" and
"coverability".  The difference between the two is that, in the
"configuration reachability" problem, a specific configuration
$\loc_f(\vec v_f)$ should be visited, whereas in the "coverability"
problem, \Eve\ attempts to visit $\loc_f(\vec v')$ for some
vector~$\vec v'$ componentwise larger or equal to
$\vec v_f$.\footnote{The name `"coverability"' comes from the the
  literature on "Petri nets" and "vector addition systems with
  states", because \Eve\ is attempting to \emph{cover}
  $\loc_f(\vec v_f)$, i.e., to reach a configuration $\loc_f(\vec v')$
  with $\vec v'\geq\vec v_f$.}

\decpb["configuration reachability vector game" with "given initial credit"]%
{\label{11-pb:reach} A "vector system"
  $\?V=(\Loc,\Act,\Loc_\mEve,\Loc_\mAdam,\dd)$, an initial location
  $\loc_0\in\Loc$, an initial credit $\vec v_0\in\+N^\dd$, and a
  target configuration $\loc_f(\vec v_f)\in\Loc\times\+N^\dd$.}%
{Does \Eve\ have a strategy to reach $\loc(\vec v)$ from
  $\loc_0(\vec v_0)$?\\That is, does she win the ""configuration
  reachability"" game $(\natural(\?V),\col,\Reach)$ from
  $\loc_0(\vec v_0)$, where $\col(e)= \Win$ if and only if
  $\ing(e)=\loc_f(\vec v_f)$?}

\decpb["coverability vector game" with "given initial credit"]%
{\label{11-pb:cov} A "vector system"
  $\?V=(\Loc,\Act,\Loc_\mEve,\Loc_\mAdam,\dd)$, an initial location
  $\loc_0\in\Loc$, an initial credit $\vec v_0\in\+N^\dd$, and a
  target configuration $\loc_f(\vec v_f)\in\Loc\times\+N^\dd$.}%
{Does \Eve\ have a strategy to reach $\loc(\vec v')$ for some
  $\vec v'\geq\vec v_f$ from $\loc_0(\vec v_0)$?\\That is, does she win
  the ""coverability"" game $(\natural(\?V),\col,\Reach)$ from
  $\loc_0(\vec v_0)$, where $\col(e)= \Win$ if and only if
  $\ing(e)=\loc_f(\vec v')$ for some $\vec v'\geq\vec v_f$?}

\begin{example}[Objectives]
\label{11-ex:cov}
  Consider the target configuration $\loc(2,2)$ in
  \cref{11-fig:mwg,11-fig:sem}.  \Eve's "winning region" in the
  "configuration reachability" "vector game" is
  $\WE=\{\loc(n+1,n+1)\mid n\in\+N\}\cup\{\loc'(0,1)\}$, displayed on the left
  in \cref{11-fig:cov}.  \Eve\ has indeed an obvious winning strategy
  from any configuration $\loc(n,n)$ with $n\geq 2$, which is to use
  the action $\loc\step{-1,-1}\loc$ until she reaches~$\loc(2,2)$.
  Furthermore, in $\loc'(0,1)$---due to the "natural semantics"---,
  \Adam\ has no choice but to use the action $\loc'\step{2,1}\loc$:
  therefore $\loc'(0,1)$ and $\loc(1,1)$ are also winning for \Eve.
\begin{figure}[htbp]
  \centering\scalebox{.48}{
  \begin{tikzpicture}[auto,on grid,node distance=2.5cm]
    \draw[step=1,lightgray!50,dotted] (-5.7,0) grid (5.7,3.8);
    \draw[color=white](0,-.3) -- (0,3.8);
    \node at (0,3.9) (sink) {\color{red!70!black}\boldmath$\sink$};
    \draw[step=1,lightgray!50] (1,0) grid (5.5,3.5);
    \draw[step=1,lightgray!50] (-1,0) grid (-5.5,3.5);
    \node at (0,0)[lightgray,font=\scriptsize,fill=white] {0};
    \node at (0,1)[lightgray,font=\scriptsize,fill=white] {1};
    \node at (0,2)[lightgray,font=\scriptsize,fill=white] {2};
    \node at (0,3)[lightgray,font=\scriptsize,fill=white] {3};
    \node at (1,3.9)[lightgray,font=\scriptsize,fill=white] {0};
    \node at (2,3.9)[lightgray,font=\scriptsize,fill=white] {1};
    \node at (3,3.9)[lightgray,font=\scriptsize,fill=white] {2};
    \node at (4,3.9)[lightgray,font=\scriptsize,fill=white] {3};
    \node at (5,3.9)[lightgray,font=\scriptsize,fill=white] {4};
    \node at (-1,3.9)[lightgray,font=\scriptsize,fill=white] {0};
    \node at (-2,3.9)[lightgray,font=\scriptsize,fill=white] {1};
    \node at (-3,3.9)[lightgray,font=\scriptsize,fill=white] {2};
    \node at (-4,3.9)[lightgray,font=\scriptsize,fill=white] {3};
    \node at (-5,3.9)[lightgray,font=\scriptsize,fill=white] {4};
    \node at (1,0)[s-eve-small,lose] (e00) {};
    \node at (1,1)[s-adam-small,win](a01){};
    \node at (1,2)[s-eve-small,lose] (e02){};
    \node at (1,3)[s-adam-small,lose](a03){};
    \node at (2,0)[s-adam-small,lose](a10){};
    \node at (2,1)[s-eve-small,win] (e11){};
    \node at (2,2)[s-adam-small,lose](a12){};
    \node at (2,3)[s-eve-small,lose] (e13){};
    \node at (3,0)[s-eve-small,lose] (e20){};
    \node at (3,1)[s-adam-small,lose](a21){};
    \node at (3,2)[s-eve-small,win] (e22){};
    \node at (3,3)[s-adam-small,lose](a23){};
    \node at (4,0)[s-adam-small,lose](a30){};
    \node at (4,1)[s-eve-small,lose] (e31){};
    \node at (4,2)[s-adam-small,lose](a32){};
    \node at (4,3)[s-eve-small,win] (e33){};
    \node at (5,0)[s-eve-small,lose] (e40){};
    \node at (5,1)[s-adam-small,lose](a41){};
    \node at (5,2)[s-eve-small,lose] (e42){};
    \node at (5,3)[s-adam-small,lose](a43){};
    \node at (-1,0)[s-adam-small,lose](a00){};
    \node at (-1,1)[s-eve-small,lose] (e01){};
    \node at (-1,2)[s-adam-small,lose](a02){};
    \node at (-1,3)[s-eve-small,lose] (e03){};
    \node at (-2,0)[s-eve-small,lose] (e10){};
    \node at (-2,1)[s-adam-small,lose](a11){};
    \node at (-2,2)[s-eve-small,lose] (e12){};
    \node at (-2,3)[s-adam-small,lose](a13){};
    \node at (-3,0)[s-adam-small,lose](a20){};
    \node at (-3,1)[s-eve-small,lose] (e21){};
    \node at (-3,2)[s-adam-small,lose](a22){};
    \node at (-3,3)[s-eve-small,lose] (e23){};
    \node at (-4,0)[s-eve-small,lose] (e30){};
    \node at (-4,1)[s-adam-small,lose](a31){};
    \node at (-4,2)[s-eve-small,lose] (e32){};
    \node at (-4,3)[s-adam-small,lose](a33){};
    \node at (-5,0)[s-adam-small,lose](a40){};
    \node at (-5,1)[s-eve-small,lose] (e41){};
    \node at (-5,2)[s-adam-small,lose](a42){};
    \node at (-5,3)[s-eve-small,lose] (e43){};
    \path[arrow] % l, -1,-1, l
    (e11) edge (e00)
    (e22) edge (e11)
    (e31) edge (e20)
    (e32) edge (e21)
    (e21) edge (e10)
    (e12) edge (e01)
    (e23) edge (e12)
    (e33) edge (e22)
    (e13) edge (e02)
    (e43) edge (e32)
    (e42) edge (e31)
    (e41) edge (e30);
    \path[arrow] % l, -1,0, l'
    (e11) edge (a01)
    (e20) edge (a10)
    (e22) edge (a12)
    (e31) edge (a21)
    (e32) edge (a22)
    (e21) edge (a11)
    (e12) edge (a02)
    (e30) edge (a20)
    (e10) edge (a00)
    (e13) edge (a03)
    (e23) edge (a13)
    (e33) edge (a23)
    (e43) edge (a33)
    (e42) edge (a32)
    (e41) edge (a31)
    (e40) edge (a30);
    \path[arrow] % l', -1,0, l
    (a11) edge (e01)
    (a20) edge (e10)
    (a22) edge (e12)
    (a31) edge (e21)
    (a32) edge (e22)
    (a21) edge (e11)
    (a12) edge (e02)
    (a30) edge (e20)
    (a10) edge (e00)
    (a33) edge (e23)
    (a23) edge (e13)
    (a13) edge (e03)
    (a43) edge (e33)
    (a42) edge (e32)
    (a41) edge (e31)
    (a40) edge (e30);
    \path[arrow] % l', 2,1, l
    (a01) edge (e22)
    (a10) edge (e31)
    (a11) edge (e32)
    (a00) edge (e21)
    (a02) edge (e23)
    (a12) edge (e33)
    (a22) edge (e43)
    (a21) edge (e42)
    (a20) edge (e41);
    \path[arrow] % dotted to Eve
    (-5.5,3.5) edge (e43)
    (5.5,2.5) edge (e42)
    (2.5,3.5) edge (e13)
    (5.5,0.5) edge (e40)
    (-5.5,1.5) edge (e41)
    (-3.5,3.5) edge (e23)
    (-1.5,3.5) edge (e03)
    (4.5,3.5) edge (e33)
    (5.5,0) edge (e40)
    (5.5,2) edge (e42)
    (-5.5,1) edge (e41)
    (-5.5,3) edge (e43);
    \path[dotted]
    (-5.7,3.7) edge (-5.5,3.5)
    (5.7,2.7) edge (5.5,2.5)
    (2.7,3.7) edge (2.5,3.5)
    (5.7,0.7) edge (5.5,0.5)
    (-3.7,3.7) edge (-3.5,3.5)
    (-1.7,3.7) edge (-1.5,3.5)
    (4.7,3.7) edge (4.5,3.5)
    (-5.7,1.7) edge (-5.5,1.5)
    (5.75,0) edge (5.5,0)
    (5.75,2) edge (5.5,2)
    (-5.75,1) edge (-5.5,1)
    (-5.75,3) edge (-5.5,3);
    \path[arrow]
    (5.5,1) edge (a41)
    (-5.5,2) edge (a42)
    (-5.5,0) edge (a40)
    (5.5,3) edge (a43);
    \path[dotted]
    (5.75,1) edge (5.5,1)
    (-5.75,2) edge (-5.5,2)
    (-5.75,0) edge (-5.5,0)
    (5.75,3) edge (5.5,3);
    \path[-]
    (a30) edge (5.5,.75)
    (a32) edge (5.5,2.75)
    (a31) edge (-5.5,1.75)
    (a23) edge (4,3.5)
    (a03) edge (2,3.5)
    (a13) edge (-3,3.5)
    (a33) edge (-5,3.5)
    (a43) edge (5.5,3.25)
    (a41) edge (5.5,1.25)
    (a40) edge (-5.5,0.25)
    (a42) edge (-5.5,2.25);
    \path[dotted]
    (5.5,.75) edge (5.8,.9)
    (5.5,2.75) edge (5.8,2.9)
    (-5.5,1.75) edge (-5.8,1.9)
    (4,3.5) edge (4.4,3.7)
    (2,3.5) edge (2.4,3.7)
    (-3,3.5) edge (-3.4,3.7)
    (-5,3.5) edge (-5.4,3.7)
    (5.5,3.25) edge (5.8,3.4)
    (5.5,1.25) edge (5.8,1.4)
    (-5.5,.25) edge (-5.8,0.4)
    (-5.5,2.25) edge (-5.8,2.4);
    \path[arrow]
    (sink) edge[loop left] ()
    (e00) edge[bend left=8] (sink)
    (e01) edge[bend right=8] (sink)
    (e02) edge[bend left=8] (sink)
    (e03) edge[bend right=8] (sink);
  \end{tikzpicture}}\quad~~\scalebox{.48}{
  \begin{tikzpicture}[auto,on grid,node distance=2.5cm]
    \draw[step=1,lightgray!50,dotted] (-5.7,0) grid (5.7,3.8);
    \draw[color=white](0,-.3) -- (0,3.8);
    \node at (0,3.9) (sink) {\color{red!70!black}\boldmath$\sink$};
    \draw[step=1,lightgray!50] (1,0) grid (5.5,3.5);
    \draw[step=1,lightgray!50] (-1,0) grid (-5.5,3.5);
    \node at (0,0)[lightgray,font=\scriptsize,fill=white] {0};
    \node at (0,1)[lightgray,font=\scriptsize,fill=white] {1};
    \node at (0,2)[lightgray,font=\scriptsize,fill=white] {2};
    \node at (0,3)[lightgray,font=\scriptsize,fill=white] {3};
    \node at (1,3.9)[lightgray,font=\scriptsize,fill=white] {0};
    \node at (2,3.9)[lightgray,font=\scriptsize,fill=white] {1};
    \node at (3,3.9)[lightgray,font=\scriptsize,fill=white] {2};
    \node at (4,3.9)[lightgray,font=\scriptsize,fill=white] {3};
    \node at (5,3.9)[lightgray,font=\scriptsize,fill=white] {4};
    \node at (-1,3.9)[lightgray,font=\scriptsize,fill=white] {0};
    \node at (-2,3.9)[lightgray,font=\scriptsize,fill=white] {1};
    \node at (-3,3.9)[lightgray,font=\scriptsize,fill=white] {2};
    \node at (-4,3.9)[lightgray,font=\scriptsize,fill=white] {3};
    \node at (-5,3.9)[lightgray,font=\scriptsize,fill=white] {4};
    \node at (1,0)[s-eve-small,lose] (e00) {};
    \node at (1,1)[s-adam-small,win](a01){};
    \node at (1,2)[s-eve-small,lose] (e02){};
    \node at (1,3)[s-adam-small,win](a03){};
    \node at (2,0)[s-adam-small,lose](a10){};
    \node at (2,1)[s-eve-small,win] (e11){};
    \node at (2,2)[s-adam-small,lose](a12){};
    \node at (2,3)[s-eve-small,win] (e13){};
    \node at (3,0)[s-eve-small,lose] (e20){};
    \node at (3,1)[s-adam-small,win](a21){};
    \node at (3,2)[s-eve-small,win] (e22){};
    \node at (3,3)[s-adam-small,win](a23){};
    \node at (4,0)[s-adam-small,lose](a30){};
    \node at (4,1)[s-eve-small,win] (e31){};
    \node at (4,2)[s-adam-small,win](a32){};
    \node at (4,3)[s-eve-small,win] (e33){};
    \node at (5,0)[s-eve-small,lose] (e40){};
    \node at (5,1)[s-adam-small,win](a41){};
    \node at (5,2)[s-eve-small,win] (e42){};
    \node at (5,3)[s-adam-small,win](a43){};
    \node at (-1,0)[s-adam-small,lose](a00){};
    \node at (-1,1)[s-eve-small,lose] (e01){};
    \node at (-1,2)[s-adam-small,win](a02){};
    \node at (-1,3)[s-eve-small,lose] (e03){};
    \node at (-2,0)[s-eve-small,lose] (e10){};
    \node at (-2,1)[s-adam-small,lose](a11){};
    \node at (-2,2)[s-eve-small,win] (e12){};
    \node at (-2,3)[s-adam-small,lose](a13){};
    \node at (-3,0)[s-adam-small,lose](a20){};
    \node at (-3,1)[s-eve-small,lose] (e21){};
    \node at (-3,2)[s-adam-small,win](a22){};
    \node at (-3,3)[s-eve-small,win] (e23){};
    \node at (-4,0)[s-eve-small,lose] (e30){};
    \node at (-4,1)[s-adam-small,lose](a31){};
    \node at (-4,2)[s-eve-small,win] (e32){};
    \node at (-4,3)[s-adam-small,win](a33){};
    \node at (-5,0)[s-adam-small,lose](a40){};
    \node at (-5,1)[s-eve-small,lose] (e41){};
    \node at (-5,2)[s-adam-small,win](a42){};
    \node at (-5,3)[s-eve-small,win] (e43){};
    \path[arrow] % l, -1,-1, l
    (e11) edge (e00)
    (e22) edge (e11)
    (e31) edge (e20)
    (e32) edge (e21)
    (e21) edge (e10)
    (e12) edge (e01)
    (e23) edge (e12)
    (e33) edge (e22)
    (e13) edge (e02)
    (e43) edge (e32)
    (e42) edge (e31)
    (e41) edge (e30);
    \path[arrow] % l, -1,0, l'
    (e11) edge (a01)
    (e20) edge (a10)
    (e22) edge (a12)
    (e31) edge (a21)
    (e32) edge (a22)
    (e21) edge (a11)
    (e12) edge (a02)
    (e30) edge (a20)
    (e10) edge (a00)
    (e13) edge (a03)
    (e23) edge (a13)
    (e33) edge (a23)
    (e43) edge (a33)
    (e42) edge (a32)
    (e41) edge (a31)
    (e40) edge (a30);
    \path[arrow] % l', -1,0, l
    (a11) edge (e01)
    (a20) edge (e10)
    (a22) edge (e12)
    (a31) edge (e21)
    (a32) edge (e22)
    (a21) edge (e11)
    (a12) edge (e02)
    (a30) edge (e20)
    (a10) edge (e00)
    (a33) edge (e23)
    (a23) edge (e13)
    (a13) edge (e03)
    (a43) edge (e33)
    (a42) edge (e32)
    (a41) edge (e31)
    (a40) edge (e30);
    \path[arrow] % l', 2,1, l
    (a01) edge (e22)
    (a10) edge (e31)
    (a11) edge (e32)
    (a00) edge (e21)
    (a02) edge (e23)
    (a12) edge (e33)
    (a22) edge (e43)
    (a21) edge (e42)
    (a20) edge (e41);
    \path[arrow] % dotted to Eve
    (-5.5,3.5) edge (e43)
    (5.5,2.5) edge (e42)
    (2.5,3.5) edge (e13)
    (5.5,0.5) edge (e40)
    (-5.5,1.5) edge (e41)
    (-3.5,3.5) edge (e23)
    (-1.5,3.5) edge (e03)
    (4.5,3.5) edge (e33)
    (5.5,0) edge (e40)
    (5.5,2) edge (e42)
    (-5.5,1) edge (e41)
    (-5.5,3) edge (e43);
    \path[dotted]
    (-5.7,3.7) edge (-5.5,3.5)
    (5.7,2.7) edge (5.5,2.5)
    (2.7,3.7) edge (2.5,3.5)
    (5.7,0.7) edge (5.5,0.5)
    (-3.7,3.7) edge (-3.5,3.5)
    (-1.7,3.7) edge (-1.5,3.5)
    (4.7,3.7) edge (4.5,3.5)
    (-5.7,1.7) edge (-5.5,1.5)
    (5.75,0) edge (5.5,0)
    (5.75,2) edge (5.5,2)
    (-5.75,1) edge (-5.5,1)
    (-5.75,3) edge (-5.5,3);
    \path[arrow]
    (5.5,1) edge (a41)
    (-5.5,2) edge (a42)
    (-5.5,0) edge (a40)
    (5.5,3) edge (a43);
    \path[dotted]
    (5.75,1) edge (5.5,1)
    (-5.75,2) edge (-5.5,2)
    (-5.75,0) edge (-5.5,0)
    (5.75,3) edge (5.5,3);
    \path[-]
    (a30) edge (5.5,.75)
    (a32) edge (5.5,2.75)
    (a31) edge (-5.5,1.75)
    (a23) edge (4,3.5)
    (a03) edge (2,3.5)
    (a13) edge (-3,3.5)
    (a33) edge (-5,3.5)
    (a43) edge (5.5,3.25)
    (a41) edge (5.5,1.25)
    (a40) edge (-5.5,0.25)
    (a42) edge (-5.5,2.25);
    \path[dotted]
    (5.5,.75) edge (5.8,.9)
    (5.5,2.75) edge (5.8,2.9)
    (-5.5,1.75) edge (-5.8,1.9)
    (4,3.5) edge (4.4,3.7)
    (2,3.5) edge (2.4,3.7)
    (-3,3.5) edge (-3.4,3.7)
    (-5,3.5) edge (-5.4,3.7)
    (5.5,3.25) edge (5.8,3.4)
    (5.5,1.25) edge (5.8,1.4)
    (-5.5,.25) edge (-5.8,0.4)
    (-5.5,2.25) edge (-5.8,2.4);
    \path[arrow]
    (sink) edge[loop left] ()
    (e00) edge[bend left=8] (sink)
    (e01) edge[bend right=8] (sink)
    (e02) edge[bend left=8] (sink)
    (e03) edge[bend right=8] (sink);
  \end{tikzpicture}}
  \caption{The "winning regions" of \Eve\ in the
    "configuration reachability" game (left) and the "coverability" game
    (right) on the graphs of \cref{11-fig:mwg,11-fig:sem} with target
    configuration~$\ell(2,2)$.  The winning vertices are in filled in
    green, while the losing ones are filled with white with a red
    border; the "sink" is always losing.}\label{11-fig:cov}
\end{figure}

In the "coverability" "vector game", \Eve's "winning region" is
$\WE=\{\loc(m+2,n+2),\loc'(m+2,n+2),\loc'(0,n+1),\loc(1,n+2),\loc'(2m+2,1),\loc(2m+3,1)\mid
m,n\in\+N\}$
displayed on the right in \cref{11-fig:cov}.  Observe in particular
that \Adam\ is forced to use the action $\ell'\step{2,1}\ell$ from
the configurations of the form $\loc'(0,n+1)$, which brings him to a
configuration $\ell(2,n+2)$ coloured~$\Win$ in the game, and thus the
configurations of the form $\loc(1,n+1)$ are also winning for \Eve\ 
since she can play $\loc\step{-1,0}\loc'$.  Thus the configurations of
the form $\loc(2m+3,n+1)$ are also winning for \Eve, as she can play
the action $\loc\step{-1,0}\loc'$ to a winning configuration
$\loc'(2m+2,n+1)$ where all the actions available to \Adam\ go into
her winning region.
\end{example}

\begin{remark}[Location reachability]
\label{11-rk:cov2cov}
  In both "configuration reachability" and "coverability", we can
  assume without loss of generality that $\loc_f\in\Loc_\mEve$ is
  controlled by \Eve\ and that $\vec v_f=\vec 0$ is the "zero vector".
  Thus "coverability" is equivalent to \emph{location reachability}:
  does \Eve\ have a strategy to reach $\loc_f(\vec v)$ for some~$\vec v$?
  
  There is indeed a \logspace\ reduction to that case.  If
  $\loc_0(\vec v_0)=\loc_f(\vec v_f)$ in the case of "configuration
  reachability", or $\loc_0=\loc_f$ and $\vec v_0\geq\vec v_f$ in the
  case of "coverability", the problem is trivial.
  %
  Otherwise, any winning play must use at least one action.  For
  each incoming action $a=(\loc\step{\vec u}\loc_f)$ of~$\loc_f$,
  create a new location~$\loc_a$ controlled by \Eve\ and replace~$a$ by
  $\loc\step{\vec u}\loc_a\step{\vec 0}\loc_f$, so that \Eve\ gains the
  control right before any play reaches~$\loc_f$.  Also add a new
  location~$\smiley$ controlled by \Eve\ with actions
  $\loc_a\step{-\vec v_f}\smiley$, and use $\smiley(\vec 0)$ as target
  configuration.
\end{remark}

\begin{remark}[Coverability to reachability]
\label{11-rk:cov2reach}
  There is a \logspace\ reduction from "coverability" to
  "configuration reachability".  By \cref{11-rk:cov2cov}, we can assume
  without loss of generality that $\loc_f\in\Loc_\mEve$ is controlled
  by \Eve\ and that $\vec v_f=\vec 0$ is the "zero vector". It suffices
  therefore to add an action $\loc_f\step{-\vec e_j}\loc_f$ for
  all $1\leq j\leq\dd$.
\end{remark}

Departing from "reachability" games, the
following is a very simple kind of "safety"
games where $C\eqdef\{\varepsilon,\Lose\}$ and $\Omega\eqdef\Safe$;
\cref{11-fig:nonterm} shows \Eve's "winning region" in the case of the
graphs of \cref{11-fig:mwg,11-fig:sem}.
\decpb["non-termination vector game" with "given initial credit"]%
{\label{11-pb:nonterm} A "vector system"
  $\?V=(\Loc,\Act,\Loc_\mEve,\Loc_\mAdam,\dd)$, an initial location
  $\loc_0\in\Loc$, and an initial credit $\vec v_0\in\+N^\dd$.}%
{Does \Eve\ have a strategy to avoid the "sink"~$\sink$ from
  $\loc_0(\vec v_0)$?\\That is, does she win the ""non-termination""
  game $(\natural(\?V),\col,\Safe)$ from $\loc_0(\vec v_0)$, where
  $\col(e)=\Lose$ if and only if $\ing(e)=\sink$?} 

  \begin{figure}[bhtp]
  \centering\scalebox{.48}{
  \begin{tikzpicture}[auto,on grid,node distance=2.5cm]
    \draw[step=1,lightgray!50,dotted] (-5.7,0) grid (5.7,3.8);
    \draw[color=white](0,-.3) -- (0,3.8);
    \node at (0,3.9) (sink) {\color{red!70!black}\boldmath$\sink$};
    \draw[step=1,lightgray!50] (1,0) grid (5.5,3.5);
    \draw[step=1,lightgray!50] (-1,0) grid (-5.5,3.5);
    \node at (0,0)[lightgray,font=\scriptsize,fill=white] {0};
    \node at (0,1)[lightgray,font=\scriptsize,fill=white] {1};
    \node at (0,2)[lightgray,font=\scriptsize,fill=white] {2};
    \node at (0,3)[lightgray,font=\scriptsize,fill=white] {3};
    \node at (1,3.9)[lightgray,font=\scriptsize,fill=white] {0};
    \node at (2,3.9)[lightgray,font=\scriptsize,fill=white] {1};
    \node at (3,3.9)[lightgray,font=\scriptsize,fill=white] {2};
    \node at (4,3.9)[lightgray,font=\scriptsize,fill=white] {3};
    \node at (5,3.9)[lightgray,font=\scriptsize,fill=white] {4};
    \node at (-1,3.9)[lightgray,font=\scriptsize,fill=white] {0};
    \node at (-2,3.9)[lightgray,font=\scriptsize,fill=white] {1};
    \node at (-3,3.9)[lightgray,font=\scriptsize,fill=white] {2};
    \node at (-4,3.9)[lightgray,font=\scriptsize,fill=white] {3};
    \node at (-5,3.9)[lightgray,font=\scriptsize,fill=white] {4};
    \node at (1,0)[s-eve-small,lose] (e00) {};
    \node at (1,1)[s-adam-small,win](a01){};
    \node at (1,2)[s-eve-small,lose] (e02){};
    \node at (1,3)[s-adam-small,win](a03){};
    \node at (2,0)[s-adam-small,lose](a10){};
    \node at (2,1)[s-eve-small,win] (e11){};
    \node at (2,2)[s-adam-small,lose](a12){};
    \node at (2,3)[s-eve-small,win] (e13){};
    \node at (3,0)[s-eve-small,lose] (e20){};
    \node at (3,1)[s-adam-small,win](a21){};
    \node at (3,2)[s-eve-small,win] (e22){};
    \node at (3,3)[s-adam-small,win](a23){};
    \node at (4,0)[s-adam-small,lose](a30){};
    \node at (4,1)[s-eve-small,win] (e31){};
    \node at (4,2)[s-adam-small,win](a32){};
    \node at (4,3)[s-eve-small,win] (e33){};
    \node at (5,0)[s-eve-small,lose] (e40){};
    \node at (5,1)[s-adam-small,win](a41){};
    \node at (5,2)[s-eve-small,win] (e42){};
    \node at (5,3)[s-adam-small,win](a43){};
    \node at (-1,0)[s-adam-small,win](a00){};
    \node at (-1,1)[s-eve-small,lose] (e01){};
    \node at (-1,2)[s-adam-small,win](a02){};
    \node at (-1,3)[s-eve-small,lose] (e03){};
    \node at (-2,0)[s-eve-small,win] (e10){};
    \node at (-2,1)[s-adam-small,lose](a11){};
    \node at (-2,2)[s-eve-small,win] (e12){};
    \node at (-2,3)[s-adam-small,lose](a13){};
    \node at (-3,0)[s-adam-small,win](a20){};
    \node at (-3,1)[s-eve-small,win] (e21){};
    \node at (-3,2)[s-adam-small,win](a22){};
    \node at (-3,3)[s-eve-small,win] (e23){};
    \node at (-4,0)[s-eve-small,win] (e30){};
    \node at (-4,1)[s-adam-small,win](a31){};
    \node at (-4,2)[s-eve-small,win] (e32){};
    \node at (-4,3)[s-adam-small,win](a33){};
    \node at (-5,0)[s-adam-small,win](a40){};
    \node at (-5,1)[s-eve-small,win] (e41){};
    \node at (-5,2)[s-adam-small,win](a42){};
    \node at (-5,3)[s-eve-small,win] (e43){};
    \path[arrow] % l, -1,-1, l
    (e11) edge (e00)
    (e22) edge (e11)
    (e31) edge (e20)
    (e32) edge (e21)
    (e21) edge (e10)
    (e12) edge (e01)
    (e23) edge (e12)
    (e33) edge (e22)
    (e13) edge (e02)
    (e43) edge (e32)
    (e42) edge (e31)
    (e41) edge (e30);
    \path[arrow] % l, -1,0, l'
    (e11) edge (a01)
    (e20) edge (a10)
    (e22) edge (a12)
    (e31) edge (a21)
    (e32) edge (a22)
    (e21) edge (a11)
    (e12) edge (a02)
    (e30) edge (a20)
    (e10) edge (a00)
    (e13) edge (a03)
    (e23) edge (a13)
    (e33) edge (a23)
    (e43) edge (a33)
    (e42) edge (a32)
    (e41) edge (a31)
    (e40) edge (a30);
    \path[arrow] % l', -1,0, l
    (a11) edge (e01)
    (a20) edge (e10)
    (a22) edge (e12)
    (a31) edge (e21)
    (a32) edge (e22)
    (a21) edge (e11)
    (a12) edge (e02)
    (a30) edge (e20)
    (a10) edge (e00)
    (a33) edge (e23)
    (a23) edge (e13)
    (a13) edge (e03)
    (a43) edge (e33)
    (a42) edge (e32)
    (a41) edge (e31)
    (a40) edge (e30);
    \path[arrow] % l', 2,1, l
    (a01) edge (e22)
    (a10) edge (e31)
    (a11) edge (e32)
    (a00) edge (e21)
    (a02) edge (e23)
    (a12) edge (e33)
    (a22) edge (e43)
    (a21) edge (e42)
    (a20) edge (e41);
    \path[arrow] % dotted to Eve
    (-5.5,3.5) edge (e43)
    (5.5,2.5) edge (e42)
    (2.5,3.5) edge (e13)
    (5.5,0.5) edge (e40)
    (-5.5,1.5) edge (e41)
    (-3.5,3.5) edge (e23)
    (-1.5,3.5) edge (e03)
    (4.5,3.5) edge (e33)
    (5.5,0) edge (e40)
    (5.5,2) edge (e42)
    (-5.5,1) edge (e41)
    (-5.5,3) edge (e43);
    \path[dotted]
    (-5.7,3.7) edge (-5.5,3.5)
    (5.7,2.7) edge (5.5,2.5)
    (2.7,3.7) edge (2.5,3.5)
    (5.7,0.7) edge (5.5,0.5)
    (-3.7,3.7) edge (-3.5,3.5)
    (-1.7,3.7) edge (-1.5,3.5)
    (4.7,3.7) edge (4.5,3.5)
    (-5.7,1.7) edge (-5.5,1.5)
    (5.75,0) edge (5.5,0)
    (5.75,2) edge (5.5,2)
    (-5.75,1) edge (-5.5,1)
    (-5.75,3) edge (-5.5,3);
    \path[arrow]
    (5.5,1) edge (a41)
    (-5.5,2) edge (a42)
    (-5.5,0) edge (a40)
    (5.5,3) edge (a43);
    \path[dotted]
    (5.75,1) edge (5.5,1)
    (-5.75,2) edge (-5.5,2)
    (-5.75,0) edge (-5.5,0)
    (5.75,3) edge (5.5,3);
    \path[-]
    (a30) edge (5.5,.75)
    (a32) edge (5.5,2.75)
    (a31) edge (-5.5,1.75)
    (a23) edge (4,3.5)
    (a03) edge (2,3.5)
    (a13) edge (-3,3.5)
    (a33) edge (-5,3.5)
    (a43) edge (5.5,3.25)
    (a41) edge (5.5,1.25)
    (a40) edge (-5.5,0.25)
    (a42) edge (-5.5,2.25);
    \path[dotted]
    (5.5,.75) edge (5.8,.9)
    (5.5,2.75) edge (5.8,2.9)
    (-5.5,1.75) edge (-5.8,1.9)
    (4,3.5) edge (4.4,3.7)
    (2,3.5) edge (2.4,3.7)
    (-3,3.5) edge (-3.4,3.7)
    (-5,3.5) edge (-5.4,3.7)
    (5.5,3.25) edge (5.8,3.4)
    (5.5,1.25) edge (5.8,1.4)
    (-5.5,.25) edge (-5.8,0.4)
    (-5.5,2.25) edge (-5.8,2.4);
    \path[arrow]
    (sink) edge[loop left] ()
    (e00) edge[bend left=8] (sink)
    (e01) edge[bend right=8] (sink)
    (e02) edge[bend left=8] (sink)
    (e03) edge[bend right=8] (sink);
  \end{tikzpicture}}
  \caption{The "winning region" of \Eve\ in the
    "non-termination" game on the graphs of \cref{11-fig:mwg,11-fig:sem}.}\label{11-fig:nonterm} 
\end{figure}

% \begin{example}
%   Consider the "non-termination" game on the graphs of
%   \cref{11-fig:mwg,11-fig:sem}.  \Eve's "winning region" is shown in
%   \cref{11-fig:nonterm}
% \end{example}

Finally, one of the most general "vector games" are "parity@parity
vector game" games, where $C\eqdef\{1,\dots,d\}$ and
$\Omega\eqdef\Parity$.  In order to define a colouring of the "natural
semantics", we assume that we are provided with a \emph{location
  colouring} $\lcol{:}\,\Loc\to\{1,\dots,d\}$.
\decpb["parity vector game" with "given initial credit"]%
{\label{11-pb:parity}A "vector system"
  $\?V=(\Loc,\Act,\Loc_\mEve,\Loc_\mAdam,\dd)$, an initial location
  $\loc_0\in\Loc$, an initial credit $\vec v_0\in\+N^\dd$, and a
  location colouring $\lcol{:}\,\Loc\to\{1,\dots,d\}$ for some $d>0$.}%
  {Does \Eve\ have a strategy to simultaneously avoid the
  "sink"~$\sink$ and fulfil the \index{parity!\emph{see also} vector
  game\protect\mymoot|mymoot}"parity" objective from $\loc_0(\vec
  v_0)$?\\That is, does she win the ""parity@parity vector game"" game
  $(\natural(\?V),\col,\Parity)$ from $\loc_0(\vec v_0)$, where
  $\col(e)\eqdef\lcol(\loc)$ if $\ing(e)=\loc(\vec v)$ for
  some~$\vec v\in\+N^\dd$, and $\col(e)\eqdef 1$ if $\ing(e)=\sink$?}

\begin{remark}[Non termination to parity]
\label{11-rk:nonterm2parity}
  There is a \logspace\ reduction from "non-termination" to
  "parity@parity vector game".
  Indeed, the two games coincide if we pick the constant location
  "colouring" defined by $\lcol(\loc)\eqdef 2$ for all $\loc\in\Loc$ in
  the parity game.
\end{remark}
\begin{remark}[Coverability to parity]
\label{11-rk:cov2parity}
  There is a \logspace\ reduction from "coverability" to
  "parity@parity vector game".  Indeed, by \cref{11-rk:cov2cov}, we can assume
  that $\loc_f\in\Loc_\mEve$ is controlled by \Eve\ and that the target
  credit is $\vec v_f=\vec 0$ the "zero vector".  It suffices
  therefore to add an action $\loc_f\step{\vec 0}\loc_f$ and to colour
  every location $\loc\neq\loc_f$ with $\lcol(\loc)\eqdef 1$ and
  to set $\lcol(\loc_f)\eqdef 2$.
\end{remark}

The "existential initial credit" variants of
\crefrange{11-pb:reach}{11-pb:parity} are defined similarly, where
$\vec v_0$ is not given as part of the input, but existentially
quantified in the question.

\subsection{Undecidability}
\label{11-sec:undec}
The bad news is that, although \crefrange{11-pb:reach}{11-pb:parity}
are all decidable in the one-player case---see
the \hyperref[11-sec:references]{bibliographic notes} at the end of the
chapter---, they become undecidable in the two-player setting.

\begin{theorem}[Undecidability of vector games]
\label{11-th:undec}
  "Configuration reachability", "coverability", "non-termination", and
  "parity@parity vector game" "vector games", both with "given" and
  with "existential initial credit", are undecidable in any dimension
  $\dd\geq 2$.
\end{theorem}
\begin{proof}
  By \cref{11-rk:cov2reach,11-rk:nonterm2parity}, it suffices to prove the
  undecidability of "coverability" and "non-termination".  For this,
  we exhibit reductions from the "halting problem" of "deterministic
  Minsky machines" with at least two counters.

  \AP Formally, a ""deterministic Minsky machine"" with $\dd$~counters
  $\?M=(\Loc,\Act,\dd)$ is defined similarly to a "vector addition
  system with states" with additional ""zero test"" actions
  $a=(\loc\step{i\eqby?0}\loc')$.  The set of locations contains a
  distinguished `halt' location~$\loc_\mathtt{halt}$, and for every
  $\loc\in\Loc$, exactly one of the following holds: either (i)
  $(\loc\step{\vec e_i}\loc')\in\Act$ for some $0<i\leq\dd$ and
  $\loc'\in\Loc$, or (ii) $(\loc\step{i\eqby?0}\loc')\in\Act$ and
  $(\loc\step{-\vec e_i}\loc'')\in\Act$ for some $0<i\leq\dd$ and
  $\loc',\loc''\in\Loc$, or (iii) $\loc=\loc_\mathtt{halt}$.  The
  semantics of~$\?M$ extends the "natural semantics" by
  handling "zero tests" actions $a=(\loc\step{i\eqby?0}\loc')$: we
  define the domain as $\dom a\eqdef\{\loc(\vec v)\mid \vec v(i)=0\}$
  and the image by $a(\loc(\vec v))\eqdef \loc(\vec v)$.  This
  semantics is deterministic, and from any starting vertex of $\natural(\?M)$,
  there is a unique "play", which either eventually visits
  $\loc_\mathtt{halt}$ and then the "sink" in the next step, or keeps
  avoiding both $\loc_\mathtt{halt}$ and the "sink"
  indefinitely. % defined by a step function defined over configurations in
  % $(\Loc\setminus\{\loc_\mathtt{halt}\})\times\+N^\dd$ by
  % \begin{equation*}
  %   \loc(\vec v)\mapsto\!\begin{cases}
  %     \loc'(\vec v+\vec e_i)
  %     &\text{if }\delta(\loc)=(\mathtt{c}_i\texttt{+\!+};\inst{goto}\loc'),\\
  %     \loc'(\vec v)
  %     &\text{if
  %     }\delta(\loc)=(\inst{if}\mathtt{c}_i\eqby?0\inst{goto}\loc'\inst{else}\mathtt{c}_i\textrm{-}\textrm{-};\inst{goto}\loc'')\text{
  %       and }\vec v(i)=0,\\
  %     \loc''(\vec v-\vec e_i)\!
  %     &\text{if }\delta(\loc)=(\inst{if}\mathtt{c}_i\eqby?0\inst{goto}\loc'\inst{else}\mathtt{c}_i\textrm{-}\textrm{-};\inst{goto}\loc'')\text{
  %       and }\vec v(i)>0.
  %   \end{cases}
  % \end{equation*}
  %\clearpage

  \AP The ""halting problem"" asks, given a "deterministic Minsky machine"
  and an initial location $\loc_0$, whether it halts, that is, whether
  $\loc_\mathtt{halt}(\vec v)$ is reachable for
  some~$\vec v\in\+N^\dd$ starting from $\loc_0(\vec 0)$.  The
  "halting problem" is undecidable in any dimension
  $\dd\geq 2$ \cite{Minsky:1967}.
  Thus the halting problem is akin to the "coverability" of
  $\loc_\mathtt{halt}(\vec 0)$ with "given initial credit"~$\vec 0$,
  but on the one hand there is only one player and on the other hand
  the machine can perform "zero tests".

  \begin{figure}[htbp]
    \centering
    \begin{tikzpicture}[auto,on grid,node distance=1.5cm]
      \node(to){$\mapsto$};
      \node[anchor=east,left=2.5cm of to](mm){"deterministic Minsky machine"};
      \node[anchor=west,right=2.5cm of to](mwg){"vector system"};
      \node[below=.7cm of to](map){$\rightsquigarrow$};
      \node[left=2.75cm of map](0){$\loc$};
      \node[right=of 0](1){$\loc'$};
      \node[right=1.25cm of map,s-eve](2){$\loc$};
      \node[right=of 2,s-eve](3){$\loc'$};
      \node[below=1.5cm of map](map2){$\rightsquigarrow$};
      \node[left=2.75cm of map2](4){$\loc$};
      \node[below right=.5cm and 1.5cm of 4](5){$\loc''$};
      \node[above right=.5cm and 1.5cm of 4](6){$\loc'$};
      \node[right=1.25cm of map2,s-eve](7){$\loc$};
      \node[below right=.5cm and 1.5cm of 7,s-eve](8){$\loc''$};
      \node[above right=.5cm and 1.5cm of 7,s-adam,inner sep=-1.5pt](9){$\loc'_{i\eqby?0}$};
      \node[below right=.5cm and 1.5cm of 9,s-eve](10){$\loc'$};
      \node[above right=.5cm and 1.5cm of 9,s-adam](11){$\frownie$};
      \path[arrow,every node/.style={font=\scriptsize}]
      (0) edge node{$\vec e_i$} (1)
      (2) edge node{$\vec e_i$} (3)
      (4) edge[swap] node{$-\vec e_i$} (5)
      (4) edge node{$i\eqby?0$} (6)
      (7) edge[swap] node{$-\vec e_i$} (8)
      (7) edge node{$\vec 0$} (9)
      (9) edge[swap] node{$\vec 0$} (10)
      (9) edge node{$-\vec e_i$} (11);
    \end{tikzpicture}
    \caption{Schema of the reduction in the proof of \cref{11-th:undec}.}\label{11-fig:undec}
  \end{figure}
  Here is now a reduction to
  \cref{11-pb:cov}.  Given an instance of the "halting problem", i.e.,
  given a "deterministic Minsky machine" $\?M=(\Loc,\Act,\dd)$ and an
  initial location $\loc_0$, we construct a "vector system"
  $\?V\eqdef(\Loc\uplus\Loc_{\eqby?0}\uplus\{\frownie\},\Act',\Loc,\Loc_{\eqby?0}\uplus\{\frownie\},\dd)$
  where all the original locations are controlled by~\Eve\ and
  $\Loc_{\eqby?0}\uplus\{\frownie\}$ is a set of new locations
  controlled by~\Adam.  We use $\Loc_{\eqby?0}$ as a set of
  locations defined by
  \begin{align*}
    \Loc_{\eqby?0}&\eqdef\{\loc'_{i\eqby?0}\mid\exists\loc\in\Loc\mathbin.(\loc\step{i\eqby?0}\loc')\in\Act\}\intertext{and
                   define the set of actions by (see \cref{11-fig:undec})}
    \Act'&\eqdef\{\loc\step{\vec
          e_i}\loc'\mid(\loc\step{\vec e_i}\loc')\in\Act\}\cup\{\loc\step{-\vec e_i}\loc''\mid(\loc\step{-\vec e_i}\loc'')\in\Act\}\\
    &\:\cup\:\{\loc\step{\vec
      0}\loc'_{i\eqby?0},\;\;\:\loc'_{i\eqby?0}\!\!\step{\vec 0}\loc',\;\;\:\loc'_{i\eqby?0}\!\!\step{-\vec e_i}\frownie\mid(\loc\step{i\eqby?0}\loc')\in\Act\}\;.
  \end{align*}
  We use $\loc_0(\vec 0)$ as initial configuration and
  $\loc_\mathtt{halt}(\vec 0)$ as target configuration for the
  constructed "coverability" instance.  Here is the crux of the
  argument why \Eve\ has a winning strategy to cover
  $\loc_\mathtt{halt}(\vec 0)$ from $\loc_0(\vec 0)$ if and only if
  the "Minsky machine@deterministic Minsky machine" halts.
  %
  Consider any configuration $\loc(\vec v)$.  If
  $(\loc\step{\vec e_i}\loc')\in\Act$, \Eve\ has no choice but to apply
  $\loc\step{\vec e_i}\loc'$ and go to the configuration
  $\loc'(\vec v+\vec e_i)$ also reached in one step in~$\?M$.  If
  $\{\loc\step{i\eqby?0}\loc',\loc\step{-\vec e_i}\loc''\}\in\Act$ and
  $\vec v(i)=0$, due to the "natural semantics", \Eve\ cannot use the
  action $\loc\step{-\vec e_i}\loc''$, thus she must use
  $\loc\step{\vec 0}\loc'_{i\eqby?0}$.  Still due to the "natural
  semantics", \Adam\ cannot use
  $\loc'_{i\eqby?0}\!\!\step{-\vec e_i}\frownie$, thus he must use
  $\loc'_{i\eqby?0}\!\!\step{\vec 0}\loc'$.  Hence \Eve\ regains the
  control in $\loc'(\vec v)$, which was also the configuration reached
  in one step in~$\?M$.  Finally, if
  $\{\loc\step{i\eqby?0}\loc',\loc\step{-\vec e_i}\loc''\}\in\Act$ and
  $\vec v(i)>0$, \Eve\ can choose: if she uses
  $\loc\step{-\vec e_i}\loc''$, she ends in the configuration
  $\loc''(\vec v-\vec e_i)$ also reached in one step in~$\?M$.  In
  fact, she should not use $\loc\step{\vec 0}\loc'_{i\eqby?0}$,
  because \Adam\ would then have the opportunity to apply
  $\loc'_{i\eqby?0}\!\!\step{-\vec e_i}\frownie$ and to win, as
  $\frownie$ is a deadlock location and all the subsequent moves end
  in the "sink".  Thus, if $\?M$ halts, then \Eve\ has a winning
  strategy that simply follows the unique "play" of~$\?M$, and
  conversely, if \Eve\ wins, then necessarily she had to follow the
  "play" of~$\?M$ and thus the machine halts.
    
  \medskip Further note that, in a "deterministic Minsky machine" the
  "halting problem" is similarly akin to the \emph{complement} of
  "non-termination" with "given initial credit"~$\vec 0$.  This means
  that, in the "vector system"
  $\?V=(\Loc\uplus\Loc_{\eqby?0}\uplus\{\frownie\},\Act',\Loc,\Loc_{\eqby?0}\uplus\{\frownie\},\dd)$
  defined earlier, \Eve\ has a winning strategy to avoid the "sink"
  from~$\loc_0(\vec 0)$ if and only if the given "Minsky
  machine@deterministic Minsky machine" does
  not halt from~$\loc_0(\vec 0)$, which shows the undecidability of
  \cref{11-pb:nonterm}.

  \medskip Finally, let us observe that both the "existential" and the
  universal initial credit variants of the "halting problem" are also
  undecidable.  Indeed, given an instance of the "halting problem",
  i.e., given a "deterministic Minsky machine" $\?M=(\Loc,\Act,\dd)$
  and an initial location $\loc_0$, we add $\dd$~new locations
  $\loc_\dd,\loc_{\dd-1},\dots,\loc_1$ with respective actions
  $\loc_j\step{-\vec e_j}\loc_j$ and $\loc_j\step{j\eqby?0}\loc_{j-1}$
  for all $\dd\geq j>0$.  This modified machine first resets all its
  counters to zero before reaching $\loc_0(\vec 0)$ and then performs
  the same execution as the original machine.  Thus there exists an
  initial credit~$\vec v$ such that the modified machine
  reaches~$\loc_\mathtt{halt}$ from $\loc_\dd(\vec v)$ if and only if
  for all initial credits~$\vec v$ the modified machine
  reaches~$\loc_\mathtt{halt}$ from $\loc_\dd(\vec v)$, if and only if
  $\loc_\mathtt{halt}$ was reachable from~$\loc_0(\vec 0)$ in the
  original machine.  The previous construction of a "vector system"
  applied to the modified machine then shows the undecidability of the
  "existential initial credit" variants of
  \cref{11-pb:cov,11-pb:nonterm}% ; note regarding the latter that \Eve\ 
  % wins the "non-termination" game with "existential initial credit" if
  % and only if there exists an initial credit such that the modified machine
  % does not halt, if and only if it is not the case that for all
  % initial credits the modified machine halts
  .
\end{proof}

% Local IspellDict: british


\section{Games in dimension one}
\label{11-sec:dim1}
\AP \Cref{11-th-undec} leaves open whether "vector games" might be
decidable in dimension one.  They are indeed decidable, and more
generally we learned in \cref{10-chap:pushdown} that "one-counter
games"---with the additional ability to test the counter for
zero---were decidable and in fact \PSPACE-complete.  This might seem
to settle the case of "vector games" in dimension one, except that the
"one-counter games" of \cref{10-chap:pushdown} only allow integer
weights in $\{-1,1\}$, whereas we allow arbitrary updates in~$\+Z$
with a binary encoding.  Hence the \PSPACE\ upper bound of
\cref{10-chap:pushdown} becomes an~\EXPSPACE\ one for ""succinct
one-counter games"".

\begin{corollary}\label{11-cor-dim1}
  "Configuration reachability", "coverability", "non-termination", and
  "parity@parity vector game" "vector games", both with "given" and with "existential
  initial credit", are in \EXPSPACE\ in dimension one.
\end{corollary}
\begin{proof}\hfill\\
  \TODO{proof of \cref{11-cor-dim1} depends on how \cref{10-sec:one-counter} is written}
\end{proof}

The goal of this section is therefore to establish that this
\EXPSPACE\ upper bound is tight (in most cases), by proving a matching
lower bound in \cref{11-one-counter}.  But first, we will study a
class of one-dimensional "vector games" of independent interest in
\cref{11-countdown}: "countdown games".

\subsection{Countdown Games}
\label{11-countdown}

\AP A one-dimensional "vector system"
$\?V=(\Loc,\Act,\Loc_\mEve,\Loc_\mAdam,1)$ is called a ""countdown
system"" if $\Act\subseteq\Loc\times\+Z_{<0}\times\Loc$, that is, if
for all $(\loc\step z\loc')\in\Act$, $z<0$.  We consider the games
defined by "countdown systems", both with "given" and with
"existential initial credit", and call the resulting games ""countdown
games"".

\begin{theorem}\label{11-countdown-given}
  "Configuration reachability" and "coverability" "countdown games"
  with "given initial credit" are \EXP-complete.
\end{theorem}
\begin{proof}
  For the upper bound, consider an instance, i.e., a "countdown
  system" $\?V=(\Loc,\Act,\Loc_\mEve,\Loc_\mAdam,1)$, an initial
  location $\loc_0\in\Loc$, an initial credit $n_0\in\+N$, and a
  target configuration $\loc_f(n_f)\in\Loc\times\+N$.  Because every
  action decreases strictly the counter value, the reachable part
  of the "natural semantics" of $\?V$ starting from $\loc_0(n_0)$ is
  finite and of size at most $1+|\Loc|\cdot (n_0+1)$, and because~$n_0$
  is encoded in binary, this is at most exponential in the size of the
  instance.  As seen in \cref{2-chap:regular}, such a "reachability"
  game can be solved in time polynomial in the size of the finite
  graph, thus in \EXP\ overall.

  \medskip For the lower bound, we start by considering a game played
  over an exponential-time Turing machine, before showing how to
  implement this game as a "countdown game".  Let us consider for this
  an arbitrary Turing machine~$\?M$ working in deterministic
  exponential time~$2^{p(n)}$ for some fixed polynomial~$p$ and an
  input word~$w=a_1\cdots a_n$ of length~$n$, which we assume to be
  positive.  Let $m\eqdef 2^{p(n)}\geq n$.  The computation of~$\?M$
  on~$w$ is a sequence of configurations $C_1,C_2,\dots,C_t$ of
  length~$t\leq m$.  Each configuration $C_i$ is of the form
  $\emkl \gamma_{i,1}\cdots\gamma_{i,m}\emkr$ where $\emkl$ and
  $\emkr$ are endmarkers and the symbols $\gamma_{i,j}$ are either
  taken from the finite tape alphabet~$\Gamma$ (which includes a blank
  symbol~$\blank$) or a pair $(q,a)$ of a state from~$Q$ and a tape
  symbol~$a$.  We assume that the set of states~$Q$ contains a single
  accepting state~$q_\mathrm{final}$.  The entire computation can be
  arranged over a $t\times m$ grid as shown in \cref{11-fig-exp}.

  \begin{figure}[htbp]
    \centering
    \hspace*{-.5ex}\begin{tikzpicture}[on grid,every node/.style={anchor=base}]
      \draw[step=1,lightgray!50,dotted] (-.5,-0.8) grid (10.5,-5.2);
      % rows
      %\draw[white](-.5,-3) -- (10.5,-3) (-.5,-6) -- (10.5,-6);
      \node[anchor=east] at (-.5,-5) {$C_1$};
      \node[anchor=east] at (-.5,-4) {$C_2$};
      \node[anchor=east] at (-.5,-3.4) {$\vdots~$};
      \node[anchor=east] at (-.5,-3) {$C_{i-1}$};
      \node[anchor=east] at (-.5,-2) {$C_i$};
      \node[anchor=east] at (-.5,-1.4) {$\vdots~$};
      \node[anchor=east] at (-.5,-1) {$C_t$};
      % columns
      \draw[color=white](4,-.5) -- (4,-5.2) (8,-.5) -- (8,-5.2);
      \node[lightgray] at (0,-.5) {$0$};
      \node[lightgray] at (1,-.5) {$1$};
      \node[lightgray] at (2,-.5) {$2$};
      \node[lightgray] at (3,-.5) {$3$};
      \node[lightgray] at (4,-.5) {$\cdots$};
      \node[lightgray] at (5,-.5) {$j-1$};
      \node[lightgray] at (6,-.5) {$j$};
      \node[lightgray] at (7,-.5) {$j+1$};
      \node[lightgray] at (8,-.5) {$\cdots$};
      \node[lightgray] at (9,-.5) {$m$};
      \node[lightgray] at (10,-.5) {$m+1$};
      % endmarkers
      \node at (0,-1.1) {$\emkl$};
      \node at (0,-2.1) {$\emkl$};
      \node at (0,-3.1) {$\emkl$};
      \node at (0,-4.1) {$\emkl$};
      \node at (0,-5.1) {$\emkl$};
      \node at (10,-1.1) {$\emkr$};
      \node at (10,-2.1) {$\emkr$};
      \node at (10,-3.1) {$\emkr$};
      \node at (10,-4.1) {$\emkr$};
      \node at (10,-5.1) {$\emkr$};
      % initial
      \node at (1,-5.1) {$q_0,a_1$};
      \node at (2,-5.1) {$a_2$};
      \node at (3,-5.1) {$a_3$};
      \node at (4,-5.1) {$\cdots$};
      \node at (5,-5.1) {$\blank$};
      \node at (6,-5.1) {$\blank$};
      \node at (7,-5.1) {$\blank$};
      \node at (8,-5.1) {$\cdots$};
      \node at (9,-5.1) {$\blank$};
      % second
      \node at (1,-4.1) {$a'_1$};
      \node at (2,-4.1) {$q_1,a_2$};
      \node at (3,-4.1) {$a_3$};
      \node at (4,-4.1) {$\cdots$};
      \node at (5,-4.1) {$\blank$};
      \node at (6,-4.1) {$\blank$};
      \node at (7,-4.1) {$\blank$};
      \node at (8,-4.1) {$\cdots$};
      \node at (9,-4.1) {$\blank$};
      % row i-1
      \node at (5,-3.7) {$\vdots$};
      \node at (6,-3.7) {$\vdots$};
      \node at (7,-3.7) {$\vdots$};
      \node at (4,-3.1) {$\cdots$};
      \node at (5,-3.1) {$\gamma_{i-1,j-1}$};
      \node at (6,-3.1) {$\gamma_{i-1,j}$};
      \node at (7,-3.1) {$\gamma_{i-1,j+1}$};
      \node at (8,-3.1) {$\cdots$};
      % row i
      \node at (5,-2.1) {$\cdots$};
      \node at (6,-2.1) {$\gamma_{i,j}$};
      \node at (7,-2.1) {$\cdots$};
      \node at (6,-1.7) {$\vdots$};
      % final
      \node at (1,-1.1) {$q_\mathrm{final},\blank$};
      \node at (2,-1.1) {$\blank$};
      \node at (3,-1.1) {$\blank$};
      \node at (4,-1.1) {$\cdots$};
      \node at (5,-1.1) {$\blank$};
      \node at (6,-1.1) {$\blank$};
      \node at (7,-1.1) {$\blank$};
      \node at (8,-1.1) {$\cdots$};
      \node at (9,-1.1) {$\blank$};      
      \end{tikzpicture}
    \caption{\label{11-fig-exp}The computation of~$\?M$ on
  input~$w=a_1\cdots a_n$.  This particular picture assumes~$\?M$
  starts by rewriting~$a_1$ into $a'_1$ and moving to the right in a
  state~$q_1$, and empties its tape before accepting its input by going
  to state~$q_\mathrm{final}$.}
  \end{figure}

  We now set up a two-player game where \Eve\ wants to prove that the
  input~$w$ is accepted.  Let
  $\Gamma'\eqdef \{\emkl,\emkr\}\cup\Gamma\cup(Q\times\Gamma)$. Rather
  than exhibiting the full computation from \cref{11-fig-exp}, the
  game will be played over positions $(i,j,\gamma_{i,j})$ where
  $0<i\leq m$, $0\leq j\leq m+1$, and $\gamma_{i,j}\in\Gamma'$.  \Eve
  wants to show that, in the computation of~$\?M$ over~$w$ as depicted
  in \cref{11-fig-exp}, the $j$th cell of the $i$th
  configuration~$C_i$ contains~$\gamma_{i,j}$.  In order to
  substantiate this claim, observe that the content of any cell
  $\gamma_{i,j}$ in the grid is determined by the actions of~$\?M$
  and the contents of (up to) three cells in the previous
  configuration.  Thus, if $i>1$ and $0<j<m+1$, \Eve\ provides a triple
  $(\gamma_{i-1,j-1},\gamma_{i-1,j},\gamma_{i-1,j+1})$ of symbols
  in~$\Gamma'$ that yield $\gamma_{i,j}$ according to the actions
  of~$\?M$, which we denote by
  $\gamma_{i-1,j-1},\gamma_{i-1,j},\gamma_{i-1,j+1}\vdash\gamma_{i,j}$,
  and \Adam\ chooses $j'\in\{j-1,j,j+1\}$ and returns the control
  to \Eve\ in position~$(i-1,j',\gamma_{i-1,j'})$.  Regarding the
  boundary cases where $i=0$ or $j=0$ or $j=m+1$, \Eve\ wins
  immediately if $j=0$ and $\gamma={\emkl}$, or if $j=m+1$ and
  $\gamma={\emkr}$, or if $i=0$ and $0<j\leq n$ and $\gamma=a_j$, or if
  $i=0$ and $n<j\leq m$ and $\gamma={\blank}$, and otherwise \Adam\ wins
  immediately.  The game starts in a position
  $(t,j,(q_\mathrm{final},a))$ for some $0<t\leq m$, $0< j\leq m$,
  and~$a\in\Gamma$ of \Eve's choosing.  It should be clear that \Eve
  has a winning strategy in this game if and only if~$w$ is accepted
  by~$\?M$.

  We now implement the previous game as a "coverability" game over a
  "countdown system" $\?V\eqdef(\Loc,\Act,\Loc_\mEve,\Loc_\mAdam,1)$.
  The idea is that the pair $(i,j)$ will be encoded as
  $(i-1)\cdot(m+2)+j+2$ in the counter value, while the
  symbol~$\gamma_{i,j}$ will be encoded in the location.  For
  instance, the endmarker $\emkl$ at position $(1,0)$ will be
  represented by configuration $\loc_{\emkl}(2)$, the first input
  $(q_0,a_1)$ at position~$(1,1)$ by $\loc_{(q_0,a_1)}(3)$, and the
  endmarker $\emkr$ at position $(m,m+1)$ by
  $\loc_{\emkr}(m\cdot(m+2)+1)$. The game starts from the initial
  configuration $\loc_0(n_0)$ where $n_0\eqdef m\cdot(m+2)+1$ and the
  target location is~$\smiley$.

  We define for this the sets of locations
  \begin{align*}
    \Loc_\mEve&\eqdef\{\loc_0,\smiley,\frownie\}
               \cup\{\loc_\gamma\mid\gamma\in\Gamma'\}\;,\\
    \Loc_\mAdam&\eqdef\{\loc_{(\gamma_1,\gamma_2,\gamma_3)}\mid\gamma_1,\gamma_2,\gamma_3\in\Gamma'\}
               \cup\{\loc_{=j}\mid 0<j\leq n\}
               \cup\{\loc_{1\leq?\leq m-n+1}\}\;.
  \end{align*}
  The intention behind the locations $\loc_{=j}\in\Loc_\mAdam$ is
  that \Eve\ can reach~$\smiley$ from a configuration $\loc_{=j}(c)$ if
  and only if $c=j$; we accordingly define~$\Act$ with the following
  actions, where~$\frownie$ is a deadlock location:
  \begin{align*}
    \loc_{=j}&\step{-j-1}\frownie\;,&\loc_{=j}&\step{-j}\smiley\;.
  \intertext{Similarly, \Eve\ should be able to reach~$\smiley$ from
  $\loc_{1\leq?\leq m-n+1}(c)$ if and only if $1\leq c\leq m-n+1$,
  which is implemented by the actions}
    \loc_{1\leq?\leq m-n+1}&\step{-m+n-2}\frownie\;,&
    \loc_{1\leq?\leq m-n+1}&\step{-1}\smiley\;,&
    \smiley&\step{-1}\smiley\;.
  \end{align*}
  Note this last action also ensures that \Eve\ can reach the
  location~$\smiley$ if and only if she can reach the configuration
  $\smiley(0)$, thus the game can equivalently be seen as a
  "configuration reachability" game.

  Regarding initialisation, \Eve\ can choose her initial position,
  which we implement by the actions
  \begin{align*}
    \loc_0 &\step{-1} \loc_0 & \loc_0 &\step{-1}\loc_{(q_\mathrm{final},a)}&&\text{for $a\in\Gamma$}\;.
    \intertext{Outside the boundary cases, the game is implemented by
    the following actions:}
    \loc_\gamma&\step{-m}\loc_{(\gamma_1,\gamma_2,\gamma_3)}&&&&\text{for
  $\gamma_1,\gamma_2,\gamma_3\vdash\gamma$}\;,\\ \loc_{(\gamma_1,\gamma_2,\gamma_3)}&\step{-k}\loc_{\gamma_k}&&&&\text{for
  $k\in\{1,2,3\}$}\;.
  \intertext{We handle the endmarker positions via the following
  actions, where \Eve\ proceeds along the left edge
  of \cref{11-fig-exp} until she reaches the initial left endmarker:}
   \loc_\emkl&\step{-m-2}\loc_\emkl\;,& \loc_\emkl&\step{-1}\loc_{=1}\;,& \loc_\emkr&\step{-m-1}\loc_\emkl\;.
  \intertext{For the positions inside the input word $w=a_1\cdots
  a_n$, we use the actions}
  \loc_{(q_0,a_1)}&\step{-2}\loc_{=1}\;,&\loc_{a_j}&\step{-2}\loc_{=j}&&\text{for
  $1<j\leq n$}\;.
  \intertext{Finally, for the blank symbols of~$C_1$, which should be
  associated with a counter value~$c$ such that $n+3\leq c\leq m+3$, we use the
  action}
  \loc_\blank&\step{-n-2}\loc_{1\leq?\leq m-n+1}\;.&&&&&\qedhere\hspace*{-1.5em}
  \end{align*}
\end{proof}


\begin{theorem}\label{11-countdown-exist}
  "Configuration reachability" and "coverability" "countdown games"
  with "existential initial credit" are \EXPSPACE-complete.
\end{theorem}
\begin{proof}
   For the upper bound, consider an instance, i.e., a "countdown
   system" $\?V=(\Loc,\Act,\Loc_\mEve,\Loc_\mAdam,1)$, an initial
   location~$\loc_0$, and a target configuration $\loc_f\in\Loc$.  We
   reduce this to an instance of "configuration reachability" with
   "given initial credit" in a one-dimensional "vector system" by
   adding a new location $\loc'_0$ controlled by~\Eve\ with actions
   $\loc'_0\step{1}\loc'_0$ and $\loc'_0\step 0\loc_0$, and asking
   whether \Eve\ has a winning strategy starting from $\loc'_0(0)$ in
   the new system.  By \cref{11-cor-dim1}, this "configuration
   reachability" game can be solved in \EXPSPACE.

   \medskip For the lower bound, we reduce from the acceptance problem
   of a deterministic Turing machine working in exponential space.
   The reduction is the same as in the proof
   of \cref{11-countdown-given}, except that now the length~$t$ of the
   computation is not bounded a priori, but this is compensated by the
   fact that we are playing the "existential initial credit" version
   of the "countdown game".  \qedhere%%   Indeed, \Eve\ wins the "countdown game"
   %% starting from configuration $\loc_{q_\mathrm{final},a}(n)$ if and
   %% only if she wins from position
   %% $(\lfloor(n-1)/(m+2)\rfloor+1,(n-1)\mathrel\mathrm{mod}(m-2),(q_\mathrm{final},a))$
   %% in the game played on the Turing machine.
\end{proof}

\medskip
Originally, "countdown games" were introduced with a slightly
different objective, which corresponds to the following decision
problem.
\AP\decpb["zero reachability" with "given initial credit"]
  {A "countdown system" $\?V=(\Loc,T,\Loc_\mEve,\Loc_\mAdam,1)$, an
  initial location $\loc_0\in\Loc$, and an initial credit
  $n_0\in\+N$.}
  {Does \Eve\ have a strategy to reach a configuration $\loc(0)$ for
  some $\loc\in\Loc$?\\That is, does she win the ""zero
  reachability""\index{zero reachability|see{countdown game}}
  game $(\?A_\+N(\?V),\col,\Reach)$ from $\loc_0(n_0)$, where
  $\col(e)=\Win$ if and only if $\ing(e)=\loc(0)$ for some $\loc\in\Loc$?}
\begin{theorem}\label{11-countdown-zero}
  "Zero reachability" "countdown games" with "given initial credit"
  are \EXP-complete.
\end{theorem}
\begin{proof}
  The upper bound of \cref{11-countdown-given} applies in the same
  way.  Regarding the lower bound, we modify the lower bound
  construction of \cref{11-countdown-given} in the following way: we
  use $\loc_0(2\cdot n_0+1)$ as initial configuration, multiply all
  the action weights in~$\Act$ by two, and add a new
  location~$\loc_\mathrm{zero}$ with an action
  $\smiley\step{-1}\loc_\mathrm{zero}$.  Because all the counter
  values in the new game are odd unless we reach $\loc_\mathrm{zero}$,
  the only way for \Eve\ to bring the counter to zero in this new game
  is to first reach $\smiley(1)$, which occurs if and only if she
  could reach $\smiley(0)$ in the original game.
\end{proof}

\subsection{Vector Games in Dimension One}
\label{11-one-counter}

"Countdown games" are frequently employed to prove complexity lower
bounds.  Here, we use them to show that the \EXPSPACE\ upper bounds
from \cref{11-cor-dim1} are tight in most cases.
\begin{theorem}\label{11-th-dim1}
  "Configuration reachability", "coverability", and "parity@parity
  vector game" "vector games", both with "given" and with "existential
  initial credit", are \EXPSPACE-complete in dimension one;
  "non-termination" "vector games" in dimension one are \EXP-hard with
  "given initial credit" and \EXPSPACE-complete with "existential
  initial credit".
\end{theorem}
\begin{proof}
  By \cref{11-countdown-exist}, "configuration reachability" and
  "coverability" "vector games" with existential initial credit
  are \EXPSPACE-hard in dimension one.
  Furthermore, \cref{11-cov2parity} allows to deduce that
  "parity@parity vector game" is also \EXPSPACE-hard.  Finally, we can
  argue as in the upper bound proof of \cref{11-countdown-exist} that
  all these games are also hard with "given initial credit": we add a
  new initial location $\loc'_0$ controlled by \Eve\ with actions
  $\loc'_0\step 1\loc'_0$ and $\loc'_0\step 0\loc_0$ and play the game
  starting from $\loc'_0(0)$.

  Regarding "non-termination", we can add a self loop $\smiley\step
  0\smiley$ to the construction
  of \cref{11-countdown-given,12-countdown-exist}: then the only way
  to build an infinite play that avoids the "sink" is to reach the
  target location $\smiley$.  This shows that the games are \EXP-hard
  with "given initial credit" and \EXPSPACE-hard with "existential
  initial credit".  Note that the trick of reducing "existential" to
  "given initial credit" with an initial incrementing loop $\loc'_0\step
  1\loc'_0$ does not work, because \Eve\ would have a trivial winning
  strategy that consists in just playing this loop forever.
\end{proof}

%% To the best of my knowledge, there is a complexity gap for 
%% "non-termination" "vector games" with "given initial credit" in
%% dimension one, between \EXP-hardness and \EXPSPACE-easiness.

% Local IspellDict: british


\section{Asymmetric games}
\label{11-sec:avag}
\Cref{11-th:undec} shows that "vector games" are too powerful to be
algorithmically relevant, except in dimension one where
\cref{11-th:dim1} applies.  This prompts the study of restricted kinds
of "vector games", which might be decidable in arbitrary dimension.
This section introduces one such restriction, called
\emph{"asymmetry"}, which turns out to be very fruitful: it yields
decidable games (see \cref{11-sec:complexity}), and is
related to another class of games on counter systems called "energy
games" (see \cref{11-sec:resource}).

\paragraph{Asymmetric Games} A "vector system"
$\?V=(\Loc,\Act,\Loc_\mEve,\Loc_\mAdam,\dd)$ is
""asymmetric""\index{asymmetry|see{vector system}} if, for all
locations $\loc\in\Loc_\mAdam$ controlled by Adam and all actions
$(\loc\step{\vec u}\loc')\in\Act$ originating from those,
$\vec u=\vec 0$ the "zero vector".  In other words, Adam may only
change the current location, and cannot interact directly with the
counters.

\begin{example}[Asymmetric vector system]
\label{11-ex:avg}
  \Cref{11-fig:avg} presents an "asymmetric vector system" of
  dimension two with locations partitioned as $\Loc_\mEve=\{\loc,\loc_{2,1},\loc_{\text-1,0}\}$ and $\Loc_\mAdam=\{\loc'\}$% , and actions
  %  $\{\loc\step{-1,-1}\loc,\loc\step{-1,0}\loc',\loc'\step{-1,0}\loc,\loc'\step{2,1}\loc\}$
  .  We omit the labels on the actions originating from Adam\'s
  locations, since those are necessarily the "zero vector".  It is
  worth observing that this "vector system" behaves quite differently
  from the one of \cref{11-ex:mwg} on \cpageref{11-ex:mwg}: for
  instance, in $\loc'(0,1)$, Adam can now ensure that the "sink" will
  be reached by playing the action $\loc'\step{0,0}\loc_{\text-1,0}$,
  whereas in \cref{11-ex:mwg}, the action $\loc'\step{-1,0}\loc$
  was just inhibited by the "natural semantics".
\end{example}
\begin{figure}[htbp]
  \centering
  \begin{tikzpicture}[auto,on grid,node distance=2.5cm]
    \node[s-eve,inner sep=3pt](0){$\loc$};
    \node[s-adam,right=of 0,inner sep=2pt](1){$\loc'$};
    \node[s-eve,above left=1cm and 1.2cm of 1](2){$\loc_{2,1}$};
    \node[s-eve,below left=1cm and 1.2cm of 1](3){$\loc_{\text-1,0}$};
    %\node[eve,left=of 0](4){$\loc_1$};
    \path[arrow,every node/.style={font=\footnotesize,inner sep=1}]
    (0) edge[loop left] node {$-1,-1$} ()
    (0) edge[bend right=10] node {$-1,0$} (1)
    (1) edge[bend right=10]  (2)
    (1) edge[bend left=10] (3)
    (2) edge[swap,bend right=10] node{$2,1$} (0)
    (3) edge[bend left=10] node{$-1,0$} (0);
  \end{tikzpicture}
  \caption{An "asymmetric vector system".}\label{11-fig:avg}
\end{figure}

\subsection{The Case of Configuration Reachability}
\label{11-sec:reach}

In spite of the restriction to "asymmetric" "vector systems",
"configuration reachability" remains undecidable.
\begin{theorem}[Reachability in asymmetric vector games is undecidable]
\label{11-th:asym-undec}
  "Configuration reachability" "asymmetric vector games", both with
  "given" and with "existential initial credit", are undecidable in
  any dimension $\dd\geq 2$.
\end{theorem}
\begin{proof}
  We first reduce from the "halting problem" of "deterministic Minsky
  machines" to "configuration reachability" with "given initial
  credit".  Given an instance of the "halting problem", i.e., given
  $\?M=(\Loc,\Act,\dd)$ and an initial location $\loc_0$ where we
  assume without loss of generality that $\?M$ checks that all its
  counters are zero before going to $\loc_\mathtt{halt}$, we construct
  an "asymmetric vector system"
  $\?V\eqdef(\Loc\uplus\Loc_{\eqby?0}\uplus\Loc_{\dd},\Act',\Loc\uplus\Loc'_{\eqby?0},\Loc_{\eqby?0},\dd)$
  where all the original locations and $\Loc_{\dd}$ are
  controlled by~Eve and $\Loc_{\eqby?0}$ is controlled by Adam.

  \begin{figure}[htbp]
    \centering
    \begin{tikzpicture}[auto,on grid,node distance=1.5cm]
      \node(to){$\mapsto$};
      \node[anchor=east,left=2.5cm of to](mm){"deterministic Minsky machine"};
      \node[anchor=west,right=2.5cm of to](mwg){"asymmetric vector system"};
      \node[below=.7cm of to](map){$\rightsquigarrow$};
      \node[left=2.75cm of map](0){$\loc$};
      \node[right=of 0](1){$\loc'$};
      \node[right=1.25cm of map,s-eve](2){$\loc$};
      \node[right=of 2,s-eve](3){$\loc'$};
      \node[below=2.5cm of map](map2){$\rightsquigarrow$};
      \node[left=2.75cm of map2](4){$\loc$};
      \node[below right=.5cm and 1.5cm of 4](5){$\loc''$};
      \node[above right=.5cm and 1.5cm of 4](6){$\loc'$};
      \node[right=1.25cm of map2,s-eve](7){$\loc$};
      \node[below right=.5cm and 1.5cm of 7,s-eve](8){$\loc''$};
      \node[above right=.5cm and 1.5cm of 7,s-adam,inner sep=-1.5pt](9){$\loc'_{i\eqby?0}$};
      \node[below right=.5cm and 1.5cm of 9,s-eve](10){$\loc'$};
      \node[above right=.5cm and 1.5cm of 9,s-eve](11){$\loc_{i}$};
      \node[right=of 11,s-eve,inner sep=0pt](12){$\loc_{\mathtt{halt}}$};
      \path[arrow,every node/.style={font=\scriptsize}]
      (0) edge node{$\vec e_i$} (1)
      (2) edge node{$\vec e_i$} (3)
      (4) edge[swap] node{$-\vec e_i$} (5)
      (4) edge node{$i\eqby?0$} (6)
      (7) edge[swap] node{$-\vec e_i$} (8)
      (7) edge node{$\vec 0$} (9)
      (9) edge[swap] node{$\vec 0$} (10)
      (9) edge node{$\vec 0$} (11)
      (11) edge node{$\vec 0$} (12)
      (11) edge[loop above] node{$\forall j\neq i\mathbin.-\vec e_j$}();
    \end{tikzpicture}
    \caption{Schema of the reduction in the proof of \cref{11-th:asym-undec}.}\label{11-fig:asym-undec}
  \end{figure}

  We
  use $\Loc_{\eqby?0}$ and $\Loc_{\dd}$ as two sets of locations disjoint from~$\Loc$ defined by
  \begin{align*}
    \Loc_{\eqby?0}&\eqdef\{\loc'_{i\eqby?0}\in\Loc\times\{1,\dots,\dd\}\mid\exists\loc\in\Loc\mathbin.(\loc\step{i\eqby?0}\loc')\in\Act\}\\
    \Loc_{\dd}&\eqdef\{\loc_{i}\mid 1\leq i\leq \dd\}
    \intertext{and define the set of actions by (see \cref{11-fig:asym-undec})}
    \Act'&\eqdef\{\loc\step{\vec
          e_i}\loc'\mid(\loc\step{\vec e_i}\loc')\in\Act\}\cup\{\loc\step{-\vec e_i}\loc''\mid(\loc\step{-\vec e_i}\loc'')\in\Act\}\\
    &\:\cup\:\{\loc\step{\vec
      0}\loc'_{i\eqby?0},\;\;\:\loc'_{i\eqby?0}\!\!\step{\vec
      0}\loc',\;\;\:\loc'_{i\eqby?0}\!\!\step{\vec 0}\loc_{i}\mid
      (\loc\step{i\eqby?0}\loc')\in\Act\}\\
    &\:\cup\:\{\loc_i\!\step{-\vec e_j}\loc_{i},\;\;\:\loc_{i}\!\step{\vec
      0}\loc_\mathtt{halt}\mid 1\leq i,j\leq\dd, j\neq i\}\;.
  \end{align*}
  We use $\loc_0(\vec 0)$ as initial configuration and
  $\loc_\mathtt{halt}(\vec 0)$ as target configuration for the
  constructed "configuration reachability" instance.  Here is the crux
  of the argument why Eve has a winning strategy to reach
  $\loc_\mathtt{halt}(\vec 0)$ from $\loc_0(\vec 0)$ if and only if
  the "Minsky machine@deterministic Minsky machine" halts, i.e., if
  and only if the "Minsky machine@deterministic Minsky machine"
  reaches $\loc_\mathtt{halt}(\vec 0)$.
  %
  Consider any configuration $\loc(\vec v)$.  If
  $(\loc\step{\vec e_i}\loc')\in\Act$, Eve has no choice but to apply
  $\loc\step{\vec e_i}\loc'$ and go to the configuration
  $\loc'(\vec v+\vec e_i)$ also reached in one step in~$\?M$.  If
  $\{\loc\step{i\eqby?0}\loc',\loc\step{-\vec e_i}\loc''\}\in\Act$ and
  $\vec v(i)=0$, due to the "natural semantics", Eve cannot use the
  action $\loc\step{-\vec e_i}\loc''$, thus she must use
  $\loc\step{\vec 0}\loc'_{i\eqby?0}$.  Then, either Adam plays
  $\loc'_{i\eqby?0}\!\!\step{\vec 0}\loc'$ and Eve regains the
  control in $\loc'(\vec v)$, which was also the configuration reached
  in one step in~$\?M$, or Adam plays
  $\loc'_{i\eqby?0}\!\!\step{\vec 0}\loc_{i}$ and Eve 
  regains the control in $\loc_{i}(\vec v)$ with
  $\vec v(i)=0$.  Using the actions
  $\loc_{i}\!\step{-\vec e_j}\loc_{i}$ for
  $j\neq i$, Eve can then reach $\loc_{i}(\vec 0)$ and move
  to $\loc_\mathtt{halt}(\vec 0)$.  Finally, if
  $\{\loc\step{i\eqby?0}\loc',\loc\step{-\vec e_i}\loc''\}\in\Act$ and
  $\vec v(i)>0$, Eve can choose: if she uses
  $\loc\step{-\vec e_i}\loc''$, she ends in the configuration
  $\loc''(\vec v-\vec e_i)$ also reached in one step in~$\?M$.  In
  fact, she should not use $\loc\step{\vec 0}\loc'_{i\eqby?0}$,
  because Adam would then have the opportunity to apply
  $\loc'_{i\eqby?0}\!\!\step{\vec 0}\loc_{i}$, and in
  $\loc_{i}(\vec v)$ with $\vec v(i)>0$, there is no way to
  reach a configuration with an empty $i$th component, let alone to
  reach $\loc_\mathtt{halt}(\vec 0)$.  Thus, if $\?M$ halts, then Eve 
  has a winning strategy that mimics the unique "play"
  of~$\?M$, and conversely, if Eve wins, then necessarily she had to
  follow the "play" of~$\?M$ and thus the machine halts.

  \medskip Finally, regarding the "existential initial credit"
  variant, the arguments used in the proof of \cref{11-th:undec} apply
  similarly to show that it is also undecidable.
\end{proof}

In dimension~one, \cref{11-th:dim1} applies, thus "configuration
reachability" is decidable in \EXPSPACE.  This bound is actually
tight.
\begin{theorem}[Asymmetric vector games are \EXPSPACE-complete in dimension~one]
\label{11-th:asym-dim1}
  "Configuration reachability" "asymmetric vector games", both with
  "given" and with "existential initial credit",
  are \EXPSPACE-complete in dimension~one.
\end{theorem}
\begin{proof}
  Let us first consider the "existential initial credit" variant.  We
  proceed as in \cref{11-th:countdown-given,11-th:countdown-exist} and
  reduce from the acceptance problem for a deterministic Turing
  machine working in exponential space $m=2^{p(n)}$.  The reduction is
  mostly the same as in \cref{11-th:countdown-given}, with a few changes.
  Consider the integer $m-n$ from that reduction.  While this is an
  exponential value, it can be written as $m-n=\sum_{0\leq e\leq
  p(n)}2^{e}\cdot b_e$ for a polynomial number of bits $b_0,\dots,b_{p(n)}$.
  For all $0\leq d\leq p(n)$, we define $m_d\eqdef \sum_{0\leq e\leq
  d}2^{e}\cdot b_e$; thus $m-n+1=m_{p(n)}+1$.

  We define now the sets of
  locations
  \begin{align*}
    \Loc_\mEve&\eqdef\{\loc_0,\smiley\}
      \cup\{\loc_\gamma\mid\gamma\in\Gamma'\}
      \cup\{\loc_\gamma^k\mid 1\leq k\leq 3\}
      \cup\{\loc_{=j}\mid 0<j\leq n\}\\
      &\:\cup\:\{\loc_{1\leq?\leq m_d+1}\mid 0\leq d\leq
  p(n)\}\cup\{\loc_{1\leq?\leq 2^d}\mid 1\leq d\leq p(n)\}\;,\\
    \Loc_\mAdam&\eqdef\{\loc_{(\gamma_1,\gamma_2,\gamma_3)}\mid\gamma_1,\gamma_2,\gamma_3\in\Gamma'\}\;.
  \end{align*}
  The intention behind the locations $\loc_{=j}\in\Loc_\mEve$ is
  that Eve can reach~$\smiley(0)$ from a configuration $\loc_{=j}(c)$ if
  and only if $c=j$; we define accordingly~$\Act$ with the
  action $\loc_{=j}\step{-j}\smiley$.
  Similarly, Eve should be able to reach~$\smiley(0)$ from
  $\loc_{1\leq?\leq m_d+1}(c)$ for $0\leq d\leq p(n)$ if and only if
  $1\leq c\leq m_d+1$,
  which is implemented by the following actions: if $b_{d+1}=1$, then
  \begin{align*}
    \loc_{1\leq?\leq m_{d+1}+1}&\step{0}\loc_{1\leq?\leq 2^{d+1}}\;,&
    \loc_{1\leq?\leq m_{d+1}+1}&\step{-2^{d+1}}\loc_{1\leq ?\leq m_{d}+1}\;,
    \intertext{and if $b_{d+1}=0$,}
    \loc_{1\leq?\leq m_{d+1}+1}&\step{0}\loc_{1\leq ?\leq m_{d}+1}\;,
    \intertext{and finally}
    \loc_{1\leq?\leq m_0+1}&\step{-b_0}\loc_{=1}\;,&\loc_{1\leq?\leq m_0+1}&\step{0}\loc_{=1}\;,
    \intertext{where for all $1\leq d\leq p(n)$, $\loc_{1\leq?\leq 2^d}(c)$ allows to
    reach $\smiley(0)$ if and only if $1\leq c\leq 2^d$:}
    \loc_{1\leq?\leq 2^{d+1}}&\step{-2^{d}}\loc_{1\leq?\leq
                               2^d}\;,&\loc_{1\leq?\leq
                                        2^{d+1}}&\step{0}\loc_{1\leq?\leq
                                                  2^d}\;,\\\loc_{1\leq?\leq
    2^1}&\step{-1}\loc_{=1}\;,&\loc_{1\leq?\leq 2^1}&\step{0}\loc_{=1}\;.
  \end{align*}

  The remainder of the reduction is now very similar to the reduction shown
  in \cref{11-th:countdown-given}.
  Regarding initialisation, Eve can choose her initial position,
  which we implement by the actions
  \begin{align*}
    \loc_0 &\step{-1} \loc_0 & \loc_0 &\step{-1}\loc_{(q_\mathrm{final},a)}&&\text{for $a\in\Gamma$}\;.
    \intertext{Outside the boundary cases, the game is implemented by
    the following actions:}
    \loc_\gamma&\step{-m}\loc_{(\gamma_1,\gamma_2,\gamma_3)}&&&&\text{for
  $\gamma_1,\gamma_2,\gamma_3\vdash\gamma$}\;,\\ \loc_{(\gamma_1,\gamma_2,\gamma_3)}&\step{0}\loc^k_{\gamma_k}&\loc^k_{\gamma_k}&\step{-k}\loc_{\gamma_k}&&\text{for
  $k\in\{1,2,3\}$}\;.
  \intertext{We handle the endmarker positions via the following
  actions, where Eve proceeds along the left edge
  of \cref{11-fig:exp} until she reaches the initial left endmarker:}
   \loc_\emkl&\step{-m-2}\loc_\emkl\;,& \loc_\emkl&\step{-1}\loc_{=1}\;,& \loc_\emkr&\step{-m-1}\loc_\emkl\;.
  \intertext{For the positions inside the input word $w=a_1\cdots
  a_n$, we use the actions}
  \loc_{(q_0,a_1)}&\step{-2}\loc_{=1}\;,&\loc_{a_j}&\step{-2}\loc_{=j}&&\text{for
  $1<j\leq n$}\;.
  \intertext{Finally, for the blank symbols of~$C_1$, which should be
  associated with a counter value~$c$ such that $n+3\leq c\leq m+3$,
  i.e., such that $1\leq c-n-2\leq m-n+1=m_{p(n)}+1$, we use the
  action}
  \loc_\blank&\step{-n-2}\loc_{1\leq?\leq m_{p(n)}+1}\;.
  \end{align*}

  Regarding the "given initial credit" variant, we add a new location
  $\loc'_0$ controlled by Eve and let her choose her initial credit
  when starting from $\loc'_0(0)$ by using the new actions
  $\loc'_0\step{1}\loc'_0$ and $\loc'_0\step{0}\loc_0$.
\end{proof}

\subsection{Asymmetric Monotone Games}
\label{11-sec:mono}


The results on "configuration reachability" might give the impression
that "asymmetry" does not help much for solving "vector games": we
obtained in \cref{11-sec:reach} exactly the same results as in the
general case.  Thankfully, the situation changes drastically if we
consider the other types of "vector games": "coverability",
"non-termination", and "parity@parity vector games" become decidable
in "asymmetric vector games".  The main rationale for this comes from
order theory, which prompts the following definitions.

\paragraph{Quasi-orders}\AP A ""quasi-order"" $(X,{\leq})$ is a
set~$X$ together with a reflexive and transitive
relation~${\leq}\subseteq X\times X$.  Two elements $x,y\in X$ are
incomparable if $x\not\leq y$ and $y\not\leq x$, and they are
equivalent if $x\leq y$ and $y\leq x$.  The associated strict relation
$x<y$ holds if $x\leq y$ and $y\not\leq x$.

The ""upward closure"" of a subset $S\subseteq X$ is the set of
elements greater or equal to the elements of S:
${\uparrow}S\eqdef\{x\in X\mid\exists y\in S\mathbin.y\leq x\}$.  A
subset $U\subseteq X$ is ""upwards closed"" if ${\uparrow}U=U$.  When
$S=\{x\}$ is a singleton, we write more simply ${\uparrow}x$ for its
upward closure and call the resulting "upwards closed" set a
""principal filter"".  Dually, the ""downward closure"" of~$S$ is
${\downarrow}S\eqdef\{x\in X\mid\exists y\in S\mathbin.x\leq y\}$, a
""downwards closed"" set is a subset $D\subseteq X$ such that
$D={\downarrow}D$, and ${\downarrow}x$ is called a ""principal
ideal"".  Note that the complement $X\setminus U$ of an upwards closed
set~$U$ is downwards closed and vice versa.


\paragraph{Monotone Games}\AP
Let us consider again the "natural semantics" $\natural(\?V)$ of a
"vector system".  The set of vertices
$V=\Loc\times\+N^\dd\cup\{\sink\}$ is naturally equipped with a
partial ordering: $v\leq v'$ if either $v=v'=\sink$, or $v=\loc(\vec
v)$ and $v'=\loc(\vec v')$ are two configurations that share the same
location and satisfy $\vec v(i)\leq\vec v'(i)$ for all $1\leq
i\leq\dd$, i.e., if $\vec v\leq\vec v'$ for the componentwise
ordering.

Consider a set of colours $C$ and a vertex colouring $\vcol{:}\,V\to C$
of the "natural semantics" $\natural(\?V)$ of a "vector system", which
defines a colouring $\col{:}\,E\to C$ where
$\col(e)\eqdef\vcol(\ing(e))$.  We
say that the "colouring"~$\vcol$ is ""monotonic"" if $C$ is finite and,
for every colour $p\in C$, the set $\vcol^{-1}(p)$ of vertices coloured
by~$p$ is "upwards closed" with respect to ${\leq}$.  Clearly, the
"colourings" of "coverability", "non-termination", and "parity@parity
vector games" "vector games" are "monotonic", whereas those of
"configuration reachability" "vector games" are not.  By extension, we
call a "vector game" \emph{"monotonic"} if its underlying "colouring"
is "monotonic".

\begin{lemma}[Simulation]
\label{11-lem:mono}
  In a "monotonic" "asymmetric vector game", if Eve wins from a
  vertex~$v_0$, then she also wins from~$v'_0$ for all $v'_0\geq
  v_0$.
\end{lemma}
\begin{proof}
  It suffices for this to check that, for all $v_1\leq v_2$ in $V$,
  \begin{description}
  \item[(colours)] $\vcol(v_1)=\vcol(v_2)$ since $\vcol$ is "monotonic";
  \item[(zig Eve)] if $v_1,v_2\in V_\mEve$, $a\in\Act$, and
    $\dest(v_1,a)=v'_1\neq\sink$ is defined, then
    $v'_2\eqdef\dest(v_2,a)$ is such that $v'_2\geq v'_1$: indeed,
    $v'_1\neq\sink$ entails that $v_1$ is a configuration
    $\loc(\vec v_1)$ and $v'_1=\loc'(\vec v_1+\vec u)$ for the action
    $a=(\loc\step{\vec u}\loc')\in\Act$, but then $v_2=\loc(\vec v_2)$
    for some $\vec v_2\geq\vec v_1$ and
    $v'_2=\loc'(\vec v_2+\vec u)\geq v'_1$;
  \item[(zig Adam)] if $v_1,v_2\in V_\mAdam$, $a\in\Act$, and
    $\dest(v_2,a)=v'_2$ is defined, then
    $v'_1\eqdef\dest(v_1,a)\leq v'_2$: indeed, either $v'_2=\sink$ and
    then $v'_1=\sink$, or $v'_2\neq\sink$, thus
    $v_2=\loc(\vec v_2)$, $v'_2=\loc'(\vec v_2)$, and
    $a=(\loc\step{\vec 0}\loc')\in\Act$ (recall that the game is
    "asymmetric"), but then $v_1=\loc(\vec v_1)$ for some
    $\vec v_1\leq\vec v_2$ and thus $v'_1=\loc'(\vec v_1)\leq v'_2$.
  \end{description}
  The above conditions show that, if $\sigma{:}\,E^\ast\to\Act$ is a
  strategy of Eve that wins from~$v_0$, then by
  ""simulating""~$\sigma$ starting from~$v'_0$---i.e., by applying the
  same actions when given a pointwise larger or equal history---she
  will also win.\todoquestion{Is that clear?}
\end{proof}

Note that \cref{11-lem:mono} implies that $\WE$ is "upwards closed":
$v_0\in\WE$ and $v_0\leq v'_0$ imply $v_0'\in\WE$.  \Cref{11-lem:mono}
does not necessarily hold in "vector games" without the "asymmetry"
condition.  For instance, in both \cref{11-fig:cov,11-fig:nonterm} on
\cpageref{11-fig:cov}, $\loc'(0,1)\in\WE$ but $\loc'(1,2)\in\WA$ for
the "coverability" and "non-termination" objectives.  This is due to
the fact that the action $\loc'\step{-1,0}\loc$ is available
in~$\loc'(1,2)$ but not in~$\loc'(0,1)$.


\paragraph{Well-quasi-orders}\AP What makes "monotonic" "vector games" so
interesting is that the partial order $(V,{\leq})$ associated with the
"natural semantics" of a "vector system" is a ""well-quasi-order"".  A
"quasi-order" $(X,{\leq})$ is "well@well-quasi-order" (a \emph{"wqo"})
if any of the following equivalent characterisations
hold~\cite{Kruskal:1972,Schmitz&Schnoebelen:2012}:
\begin{itemize}
  % \item\AP in any infinite sequence $x_0,x_1,\dots$ of elements of~$X$,
  %   there exist indices $i<j$ such that $x_i\leq x_j$---infinite sequences in $X$ are ""good""---,
  \item\AP in any infinite sequence $x_0,x_1,\cdots$ of elements
    of~$X$, there exists an infinite sequence of indices
    $n_0<n_1<\cdots$ such that $x_{n_0}\leq
    x_{n_1}\leq\cdots$---infinite sequences in $X$ are ""good""---,
  \item\AP any strictly ascending sequence $U_0\subsetneq
    U_1\subsetneq\cdots$ of "upwards closed" sets $U_i\subseteq X$ is
    finite---$X$ has the ""ascending chain condition""---,
  \item\AP any non-empty "upwards closed" $U\subseteq X$ has at least
    one, and at most finitely many minimal elements up to equivalence;
    therefore any "upwards closed" $U\subseteq X$ is a finite union
    $U=\bigcup_{1\leq j\leq n}{\uparrow}x_j$ of finitely many
    "principal filters"~${\uparrow}x_j$---$X$ has the ""finite basis
    property"".
\end{itemize}

The fact that $(V,{\leq})$ satisfies all of the above is an easy
consequence of \emph{Dickson's Lemma}~\cite{Dickson:1913}.

% In a "monotonic" "vector game", by the "finite basis property", each
% set $\col^{-1}(d)$ for a colour $d\in C$ has finitely many minimal
% elements $\min\col^{-1}(d)$, thus there exists a ""viable"" natural
% number
% $B\geq \max_{d\in C}\max_{\vec v\in\min\col^{-1}(d)}\|\vec v\|$.  In
% \cref{11-pb:cov}, $\|\vec v\|$ is "viable" if $\loc(\vec v)$ is the
% target configuration, while for \cref{11-pb:nonterm,11-pb:parity},
% $0$~is "viable".

\paragraph{Pareto Limits}\AP By the "finite basis property" of
$(V,{\leq})$ and \cref{11-lem:mono}, in a "monotonic" "asymmetric
vector game", $\WE=\bigcup_{1\leq j\leq n}{\uparrow}\loc_j(\vec v_j)$
is a finite union of "principal filters".  The set
$\mathsf{Pareto}\eqdef\{\loc_1(\vec v_1),\dots,\loc_n(\vec v_n)\}$ is
called the ""Pareto limit"" or \emph{Pareto frontier} of the game.
Both the "existential" and the "given initial credit" variants of the
game can be reduced to computing this "Pareto limit": with
"existential initial credit" and an initial location $\loc_0$, check
whether $\loc_0(\vec v)\in\mathsf{Pareto}$ for some $\vec v$, and with
"given initial credit" and an initial configuration $\loc_0(\vec v_0)$, check
whether $\loc_0(\vec v)\in\mathsf{Pareto}$ for some $\vec v\leq\vec
v_0$.
\begin{example}[Pareto limit]
  Consider the "asymmetric vector system" from \cref{11-fig:avg} on
  \cpageref{11-fig:avg}.  For the "coverability game" with target
  configuration $\loc(2,2)$, the "Pareto limit" is
  $\mathsf{Pareto}=\{\loc(2,2),\loc'(3,2),\loc_{2,1}(0,1),\loc_{\text-1,0}(3,2)\}$,
  while for the "non-termination game", $\mathsf{Pareto}=\emptyset$:
  Eve loses from all the vertices.  Observe that this is consistent
  with Eve\'s "winning region" in the "coverability" "energy game"
  shown in \cref{11-fig:cov-nrg}.
\end{example}

% \paragraph{Pareto Bounds}
% \AP Consider a "monotonic" "asymmetric" "vector game"
% $\?G=(\natural(\?V),\col,\Omega)$ with "Pareto limit"
% $\mathsf{Pareto}=\{\loc_1(\vec v_1),\dots,\loc_n(\vec v_n)\}$.  Since
% this "Pareto limit" is finite, there exists a number
% $\mathsf{ParetoBound}\eqdef\max_{\loc(\vec v)\in\mathsf{Pareto}}\|\vec
% v\|$ bounding the necessary "given initial credit" for Eve to win the
% game from any initial location.  Another bound of interest is
% $\mathsf{ExistentialParetoBound}\eqdef\max_{\loc\in\Loc}\min_{\loc(\vec
% v)\in\mathsf{Pareto}}\|\vec v\|$, which bounds the "existential
% initial credit" necessary for Eve to win the game from any initial
% location.  Note that
% $\mathsf{ParetoBound}\geq \mathsf{ExistentialParetoBound}$ and that
% the two values always coincide in dimension $\dd=1$.  We call a number
% $B\geq\mathsf{ParetoBound}$ a ""Pareto bound"" and a number
% $B\geq\mathsf{ExistentialParetoBound}$ an ""existential Pareto
% bound"".

\begin{example}[Doubly exponential Pareto limit]
\label{11-ex:pareto}
  Consider the one-player "vector system" of \cref{11-fig:pareto},
  where the "meta-decrement" from~$\loc_0$ to~$\loc_1$ can be
  implemented using $O(n)$ additional counters and a set~$\Loc'$ of
  $O(n)$ additional locations by the arguments of the
  forthcoming \cref{11-th:avag-hard}.
  
  \begin{figure}[htbp]
    \centering
  \begin{tikzpicture}[auto,on grid,node distance=2.5cm]
    \node[s-eve](0){$\loc_0$};
    \node[s-eve,right=of 0](1){$\loc_1$};
    \node[s-eve,below right=1.5 and 1.25 of 0](2){$\loc_f$};
    \path[arrow,every node/.style={font=\footnotesize,inner sep=1}]
    (0) edge node {$-2^{2^n}\cdot\vec e_1$} (1)
    (0) edge[bend right=10,swap] node {$-\vec e_2$} (2)
    (1) edge[bend left=10] node {$\vec 0$} (2);
  \end{tikzpicture}
  \caption{A one-player "vector system"
  with a large "Pareto limit".}\label{11-fig:pareto}
  \end{figure}
  For the "coverability game" with target
  configuration~$\loc_f(\vec 0)$, if $\loc_0$ is the initial location
  and we are "given initial credit" $m\cdot\vec e_1$, Eve wins if and
  only if $m\geq 2^{2^n}$, but with "existential initial credit" she
  can start from $\loc_0(\vec e_2)$ instead.  We have indeed
  $\mathsf{Pareto}\cap(\{\loc_0,\loc_1,\loc_f\}\times\+N^\dd)=\{\loc_0(\vec
  e_2),\loc_0(2^{2^n}\cdot\vec e_1),\loc_1(\vec 0),\loc_f(\vec 0)\}$.
  Looking more in-depth into the construction of \cref{11-th:avag-hard},
  there is also an at least double exponential number of distinct
  minimal configurations in~$\mathsf{Pareto}$.
\end{example}



\paragraph{Finite Memory} 
Besides having a finitely represented "winning region", Eve also has
finite memory strategies in "asymmetric vector games" with "parity"
objectives; the following argument is straightforward to adapt to
the other regular objectives from \cref{2-chap:regular}.
\begin{lemma}[Finite memory suffices in parity asymmetric vector games]
\label{11-lem:finmem}
  If Eve has a "strategy" winning from some vertex~$v_0$ in a
  "parity@parity vector game" "asymmetric vector game", then she has a
  "finite-memory" one.
\end{lemma}
\begin{proof}
  Assume~$\sigma$ is a winning strategy from~$v_0$.  Consider the tree
  of vertices visited by plays consistent with~$\sigma$: each branch
  is an infinite sequence $v_0,v_1,\dots$ of elements of~$V$ where the
  maximal priority occuring infinitely often is some even number~$p$.
  Since $(V,{\leq})$ is a "wqo", this is a "good sequence": there
  exists infinitely many indices $n_0<n_1<\cdots$ such that
  $v_{n_0}\leq v_{n_1}\leq\cdots$.  There exists $i<j$ such
  that~$p=\max_{n_i\leq n<n_j}\vcol(v_n)$ is the maximal priority
  occurring in some interval $v_{n_i},v_{n_{i+1}},\dots,v_{n_{j-1}}$.
  Then Eve can play in~$v_{n_j}$ as if she were in~$v_{n_i}$, in
  $v_{n_j+1}$ as if she were in $v_{n_i+1}$ and so on, and we prune
  the tree at index~$n_j$ along this branch so that $v_{n_j}$ is a
  leaf, and we call~$v_{n_i}$ the ""return node"" of that leaf.  We
  therefore obtain a finitely branching tree with finite branches,
  which by K\H{o}nig's Lemma is finite.

  The finite tree we obtain this way is sometimes called a
  ""self-covering tree"".  %   It is labelled by configurations
  % in~$\Loc\times\+N^\dd$ and
  % \begin{enumerate}
  % \item its root label is~$\loc_0(\vec v_0)$,
  % \item if an internal node is labelled by $\loc(\vec v)$ with
  %   $\loc\in\Loc_\mEve$, then it has exactly one child labelled by
  %   some~$\loc'(\vec v+\vec u)$,
  % \item if an internal node is labelled by $\loc(\vec v)$ with
  %   $\loc\in\Loc_\mAdam$, then it has one child labelled~$\loc'(\vec
  %   v)$ for each action $\loc\step{\vec 0}\loc'\in\Act$,
  % \item\label{11-wt:self-even} if a leaf is labelled by $\loc(\vec v)$,
  %   consider the branch
  %   $\loc_0(\vec v_0),\loc_1(\vec v_1),\dots,\loc_n(\vec v_n)=\loc(\vec
  %   v)$ that reaches the leaf: then
  %   \begin{enumerate}
  %   \item\label{11-wt:self}there exists~$i<n$ such that
  %     $\loc_i=\loc$ and $\vec v_i\leq\vec v$---we call the node labelled by
  %     $\loc_i(\vec v_i)$ the ""return point"" of the leaf---, and
  %   \item\label{11-wt:even} the maximal priority observed between the two
  %     nodes, i.e., $\max_{i\leq j<n}\col(v_j)$, is even.
  %   \end{enumerate}
  % \end{enumerate}
  It is relatively straightforward to construct a finite "memory
  structure"~$(M,m_0,\delta)$ (as defined in \cref{1-sec:memory}) from a
  "self-covering tree", using its internal nodes as memory states plus
  an additional sink memory state~$m_\bot$; the initial memory
  state~$m_0$ is the root of the tree.  In a node~$m$ labelled by
  $\loc(\vec v)$, given an edge
  $e=(\loc(\vec v'),\loc'(\vec v'+\vec u))$ arising from an
  action~$\loc\step{\vec u}\loc'\in\Act$, if $\vec v'\geq\vec v$ and
  $m$~has a child~$m'$ labelled by $\loc'(\vec v+\vec u)$ in the
  "self-covering tree", then either~$m'$ is a leaf with "return
  node"~$m''$ and we set $\delta(m,e)\eqdef m''$, or $m'$~is an
  internal node and we set $\delta(m,e)\eqdef m'$; in all the other
  cases, $\delta(m,e)\eqdef m_\bot$. % The strategy
  % $\sigma'{:}\,V\times M\to\Act$ associated with the memory structure
  % picks an arbitrary action except if $v=\loc(\vec v')$ is a
  % configuration with $\loc\in\Loc_\mEve$ and $m$~is an internal node
  % of the "self-covering tree" labelled by $\loc(\vec v)$ for some
  % $\vec v\leq\vec v'$, in which case~$m$ has a single child labelled
  % by $\loc'(\vec v+\vec u)$ in the "self-covering tree", corresponding
  % to an action $a=(\loc\step{\vec u}\loc')\in\Act$, and we set
  % $\sigma'(v,m)\eqdef a$.
  \todoquestion{Is that reasonably clear?  A
    bit heavy no?}
\end{proof}

\TODO{Provide an additional example of a "self-covering tree".}

\begin{example}[doubly exponential memory]
  Consider the one-player "vector system" of \cref{11-fig:finitemem},
  where the "meta-decrement" from~$\loc_1$ to~$\loc_0$ can be
  implemented using $O(n)$ additional counters and $O(n)$ additional
  locations by the arguments of the forthcoming \cref{11-th:avag-hard}
  on \cpageref{11-th:avag-hard}.
  
  \begin{figure}[htbp]
    \centering
  \begin{tikzpicture}[auto,on grid,node distance=2.5cm]
    \node[s-eve](0){$\loc_0$};
    \node[s-eve,right=of 0](1){$\loc_1$};
    \node[black!50,above=.5 of 0,font=\scriptsize]{$2$};
    \node[black!50,above=.5 of 1,font=\scriptsize]{$1$};
    \path[arrow,every node/.style={font=\footnotesize,inner sep=2}]
    (1) edge[bend left=15] node {$-2^{2^n}\cdot\vec e_1$} (0)
    (0) edge[bend left=15] node {$\vec 0$} (1)
    (1) edge[loop right] node{$\vec e_1$} ();
  \end{tikzpicture}
  \caption{A one-player "vector system"
  witnessing the need for double exponential memory.}\label{11-fig:finitemem}
  \end{figure}

  For the "parity@parity vector game" game with location colouring
  $\lcol(\loc_0)\eqdef 2$ and $\lcol(\loc_1)\eqdef 1$, note that Eve 
  must visit $\loc_0$ infinitely often in order to fulfil the parity
  requirements.  Starting from the initial
  configuration~$\loc_0(\vec 0)$, any winning play of Eve begins
  by \begin{equation*} \loc_0(\vec 0)\step{0}\loc_1(\vec 0)\step{\vec
      e_1}\loc_1(\vec e_1)\step{\vec e_1}\cdots\step{\vec
      e_1}\loc_1(m\cdot\vec
    e_1)\mstep{-2^{2^n}}\loc_0((m-2^{2^n})\cdot\vec
    e_1) \end{equation*} for some~$m\geq 2^{2^n}$ before she visits
  again a
  configuration---namely~$\loc_0((m-2^{2^n})\cdot\vec e_1)$---greater
  or equal than a previous configuration---namely
  $\loc_0(\vec 0)$---\emph{and} witnesses a maximal even parity in the
  meantime.  She then has a winning strategy that simply repeats this
  sequence of actions, allowing her to visit successively
  $\loc_0(2(m-2^{2^n})\cdot\vec e_1)$,
  $\loc_0(3(m-2^{2^n})\cdot\vec e_1)$, etc.  In this example, she
  needs at least $2^{2^n}$ memory to remember how many times the
  $\loc_1\step{\vec e_1}\loc_1$ loop should be taken.
\end{example}

\subsubsection{Attractor Computation for Coverability}
\label{11-sec:attr}
So far, we have not seen how to compute the "Pareto limit" derived
from \cref{11-lem:mono} nor the finite "memory structure" derived
from \cref{11-lem:finmem}.  These objects are not merely finite but
also computable.  The simplest case is the one of "coverability"
"asymmetric" "monotonic vector games": the fixed-point computation of
\cref{2-sec:attractors} for "reachability" objectives can be turned into
an algorithm computing the "Pareto limit" of the game.

\begin{fact}[Computable Pareto limit]
\label{11-fact:pareto-cov}
  The "Pareto limit" of a "coverability" "asymmetric vector game" is
  computable.
\end{fact}
\begin{proof}
Let $\loc_f(\vec v_f)$ be the target configuration.  We define a
chain $U_0\subseteq U_1\subseteq\cdots$ of sets $U_i\subseteq V$ by
\begin{align*}
  U_0&\eqdef{\uparrow}\loc_f(\vec v_f)\;,&
  U_{i+1}&\eqdef U_i\cup\mathrm{Pre}(U_i)\;.
\end{align*}
Observe that for all~$i$, $U_i$ is "upwards closed".  This can be
checked by induction over~$i$: it holds initially in~$U_0$, and for
the induction step, if $v\in U_{i+1}$ and $v'\geq v$, then either
\begin{itemize}
\item $v=\loc(\vec v)\in\mathrm{Pre}(U_i)\cap\VE$ thanks to some
  $\loc\step{\vec u}\loc'\in\Act$ such that
  $\loc'(\vec v+\vec u)\in U_i$; therefore $v'=\loc(\vec v')$ for some
  $\vec v'\geq \vec v$ is such that $\loc'(\vec v'+\vec u)\in U_i$ as
  well, thus $v'\in \mathrm{Pre}(U_i)\subseteq U_{i+1}$, or
\item $v=\loc(\vec v)\in\mathrm{Pre}(U_i)\cap\VA$ because for all
  $\loc\step{\vec 0}\loc'\in\Act$, $\loc'(\vec v)\in U_i$; therefore
  $v'=\loc(\vec v')$ for some $\vec v'\geq \vec v$ is such that
  $\loc'(\vec v')\in U_i$ as well, thus
  $v'\in \mathrm{Pre}(U_i)\subseteq U_{i+1}$, or
\item $v\in U_i$ and therefore $v'\in U_i\subseteq U_{i+1}$.
\end{itemize}

By the "ascending chain condition", there is a finite rank~$i$ such
that $U_{i+1}\subseteq U_i$ and then $\WE=U_i$.  Thus the
"Pareto limit" is obtained after finitely many steps.
In order to turn this idea into an algorithm, we need a way of
representing those infinite "upwards closed" sets $U_i$.  Thankfully,
by the "finite basis property", each $U_i$ has a finite basis $B_i$
such that ${\uparrow}B_i=U_i$.  We therefore compute the following
sequence of sets
\begin{align*}
  B_0&\eqdef\{\loc_f(\vec v_f)\}&B_{i+1}&\eqdef
                                       B_i\cup\min\mathrm{Pre}({\uparrow}B_i)\;.
\end{align*}
Indeed, given a finite basis~$B_i$ for~$U_i$, it is straightforward to
compute a finite basis for the "upwards closed" $\mathrm{Pre}(U_i)$.
This results in \cref{11-algo:cov} below.
\end{proof}

\begin{algorithm}
 \KwData{A "vector system" and a target configuration $\loc_f(\vec v_f)$}

$B_0 \leftarrow \{\loc_f(\vec v_f)\}$ ;

$i \leftarrow 0$ ;
     
\Repeat{${\uparrow}B_i \supseteq B_{i+1}$}{

$B_{i+1} \leftarrow B_i \cup \min\mathrm{Pre}({\uparrow}B_i)$ ;

$i \leftarrow i + 1$ ;}

\Return{$\min B_i = \mathsf{Pareto}(\game)$}
\caption{Fixed point algorithm for "coverability" in "asymmetric" "vector
  games".}
\label{11-algo:cov}
\end{algorithm}

While this algorithm terminates thanks to the "ascending chain
condition", it may take quite long.  For instance, in
\cref{11-ex:pareto}, it requires at least~$2^{2^n}$ steps before it
reaches its fixed point.  This is a worst-case instance, as it turns
out that this algorithm works in \kEXP[2]; see the bibliographic notes
at the end of the chapter.  Note that such a
fixed-point computation does not work directly for "non-termination"
or "parity vector games", due to the need for greatest fixed-points.

% Local IspellDict: british


\section{Resource-conscious games}
\label{11-sec:resource}
"Vector games" are very well suited for reasoning about systems
manipulating discrete resources, modelled as counters.  However, in
the "natural semantics", actions that would deplete some resource,
i.e., that would make some counter go negative, are simply inhibited.
In models of real-world systems monitoring resources like a gas
tank or a battery, a depleted resource would be considered as a system
failure.  In the "energy games" of \cref{11-energy}, those situations
are accordingly considered as winning for \Adam.  Moreover, if we are
modelling systems with a bounded capacity for storing resources, a
counter exceeding some bound might also be considered as a failure,
which will be considered with "bounding games" in \cref{11-bounding}.

These resource-conscious games can be seen as providing alternative
semantics for "vector systems".  They will also be instrumental in
establishing complexity upper bounds for "monotonic" "asymmetric vector
games" later in \cref{11-sec:complexity}, and are strongly related to
"multidimensional" "mean-payoff" games, as will be explained in
\cref{13-sec:MPEG} of \cref{12-chap:multiobjective}.

\subsection{Energy Semantics}
\label{11-energy}

"Energy games" model systems where the depletion of a resource
allows \Adam\ to win.  This is captured by an ""energy semantics""
$\energy(\?V)\eqdef(V,E_\+E,\VE,\VA)$ associated with a "vector
system" $\?V$: we let as before
$V\eqdef(\Loc\times\+N^\dd)\uplus\{\sink\}$, but define instead
\begin{align*}
  E_\+E&\eqdef \{(\loc(\vec v), \loc'(\vec v+\vec u)\mid
         \loc\step{\vec u}\loc'\in\Act\text{
      and }\vec v+\vec u\geq\vec 0\}\\
    &\:\cup\:\{(\loc(\vec v),\sink)\mid\forall\loc\step{\vec
      u}\loc'\in\Act\mathbin.\vec v+\vec u\not\geq\vec 0\}
    \cup\{(\sink,\sink)\}\;.
\end{align*}
In the "energy semantics", moves that would result in a negative
component lead to the "sink" instead of being inhibited.

\begin{example}\label{11-ex-nrg}
  \Cref{11-fig-nrg} illustrates the "energy semantics" of the vector
  system depicted in~\cref{11-fig-mwg} on \cpageref{11-fig-mwg}.  Observe that,
  by contrast with the "natural semantics" of the same system depicted
  in \cref{11-fig-sem}, all the configurations $\loc'(0,n)$ controlled
  by \Adam\ can now move to the "sink".
\end{example}
\begin{figure}[thbp]
  \centering\scalebox{.77}{
  \begin{tikzpicture}[auto,on grid,node distance=2.5cm]
    \draw[step=1,lightgray!50,dotted] (-5.7,0) grid (5.7,3.8);
    \draw[color=white](0,-.3) -- (0,3.8);
    \node at (0,3.9) (sink) {\boldmath$\sink$};
    \draw[step=1,lightgray!50] (1,0) grid (5.5,3.5);
    \draw[step=1,lightgray!50] (-1,0) grid (-5.5,3.5);
    \node at (0,0)[lightgray,font=\scriptsize,fill=white] {0};
    \node at (0,1)[lightgray,font=\scriptsize,fill=white] {1};
    \node at (0,2)[lightgray,font=\scriptsize,fill=white] {2};
    \node at (0,3)[lightgray,font=\scriptsize,fill=white] {3};
    \node at (1,3.9)[lightgray,font=\scriptsize,fill=white] {0};
    \node at (2,3.9)[lightgray,font=\scriptsize,fill=white] {1};
    \node at (3,3.9)[lightgray,font=\scriptsize,fill=white] {2};
    \node at (4,3.9)[lightgray,font=\scriptsize,fill=white] {3};
    \node at (5,3.9)[lightgray,font=\scriptsize,fill=white] {4};
    \node at (-1,3.9)[lightgray,font=\scriptsize,fill=white] {0};
    \node at (-2,3.9)[lightgray,font=\scriptsize,fill=white] {1};
    \node at (-3,3.9)[lightgray,font=\scriptsize,fill=white] {2};
    \node at (-4,3.9)[lightgray,font=\scriptsize,fill=white] {3};
    \node at (-5,3.9)[lightgray,font=\scriptsize,fill=white] {4};
    \node at (1,0)[s-eve-small] (e00) {};
    \node at (1,1)[s-adam-small](a01){};
    \node at (1,2)[s-eve-small] (e02){};
    \node at (1,3)[s-adam-small](a03){};
    \node at (2,0)[s-adam-small](a10){};
    \node at (2,1)[s-eve-small] (e11){};
    \node at (2,2)[s-adam-small](a12){};
    \node at (2,3)[s-eve-small] (e13){};
    \node at (3,0)[s-eve-small] (e20){};
    \node at (3,1)[s-adam-small](a21){};
    \node at (3,2)[s-eve-small] (e22){};
    \node at (3,3)[s-adam-small](a23){};
    \node at (4,0)[s-adam-small](a30){};
    \node at (4,1)[s-eve-small] (e31){};
    \node at (4,2)[s-adam-small](a32){};
    \node at (4,3)[s-eve-small] (e33){};
    \node at (5,0)[s-eve-small] (e40){};
    \node at (5,1)[s-adam-small](a41){};
    \node at (5,2)[s-eve-small] (e42){};
    \node at (5,3)[s-adam-small](a43){};
    \node at (-1,0)[s-adam-small](a00){};
    \node at (-1,1)[s-eve-small] (e01){};
    \node at (-1,2)[s-adam-small](a02){};
    \node at (-1,3)[s-eve-small] (e03){};
    \node at (-2,0)[s-eve-small] (e10){};
    \node at (-2,1)[s-adam-small](a11){};
    \node at (-2,2)[s-eve-small] (e12){};
    \node at (-2,3)[s-adam-small](a13){};
    \node at (-3,0)[s-adam-small](a20){};
    \node at (-3,1)[s-eve-small] (e21){};
    \node at (-3,2)[s-adam-small](a22){};
    \node at (-3,3)[s-eve-small] (e23){};
    \node at (-4,0)[s-eve-small] (e30){};
    \node at (-4,1)[s-adam-small](a31){};
    \node at (-4,2)[s-eve-small] (e32){};
    \node at (-4,3)[s-adam-small](a33){};
    \node at (-5,0)[s-adam-small](a40){};
    \node at (-5,1)[s-eve-small] (e41){};
    \node at (-5,2)[s-adam-small](a42){};
    \node at (-5,3)[s-eve-small] (e43){};
    \path[arrow] % l, -1,-1, l
    (e11) edge (e00)
    (e22) edge (e11)
    (e31) edge (e20)
    (e32) edge (e21)
    (e21) edge (e10)
    (e12) edge (e01)
    (e23) edge (e12)
    (e33) edge (e22)
    (e13) edge (e02)
    (e43) edge (e32)
    (e42) edge (e31)
    (e41) edge (e30);
    \path[arrow] % l, -1,0, l'
    (e11) edge (a01)
    (e20) edge (a10)
    (e22) edge (a12)
    (e31) edge (a21)
    (e32) edge (a22)
    (e21) edge (a11)
    (e12) edge (a02)
    (e30) edge (a20)
    (e10) edge (a00)
    (e13) edge (a03)
    (e23) edge (a13)
    (e33) edge (a23)
    (e43) edge (a33)
    (e42) edge (a32)
    (e41) edge (a31)
    (e40) edge (a30);
    \path[arrow] % l', -1,0, l
    (a11) edge (e01)
    (a20) edge (e10)
    (a22) edge (e12)
    (a31) edge (e21)
    (a32) edge (e22)
    (a21) edge (e11)
    (a12) edge (e02)
    (a30) edge (e20)
    (a10) edge (e00)
    (a33) edge (e23)
    (a23) edge (e13)
    (a13) edge (e03)
    (a43) edge (e33)
    (a42) edge (e32)
    (a41) edge (e31)
    (a40) edge (e30);
    \path[arrow] % l', 2,1, l
    (a01) edge (e22)
    (a10) edge (e31)
    (a11) edge (e32)
    (a00) edge (e21)
    (a02) edge (e23)
    (a12) edge (e33)
    (a22) edge (e43)
    (a21) edge (e42)
    (a20) edge (e41);
    \path[arrow] % dotted to Eve
    (-5.5,3.5) edge (e43)
    (5.5,2.5) edge (e42)
    (2.5,3.5) edge (e13)
    (5.5,0.5) edge (e40)
    (-5.5,1.5) edge (e41)
    (-3.5,3.5) edge (e23)
    (-1.5,3.5) edge (e03)
    (4.5,3.5) edge (e33)
    (5.5,0) edge (e40)
    (5.5,2) edge (e42)
    (-5.5,1) edge (e41)
    (-5.5,3) edge (e43);
    \path[dotted]
    (-5.7,3.7) edge (-5.5,3.5)
    (5.7,2.7) edge (5.5,2.5)
    (2.7,3.7) edge (2.5,3.5)
    (5.7,0.7) edge (5.5,0.5)
    (-3.7,3.7) edge (-3.5,3.5)
    (-1.7,3.7) edge (-1.5,3.5)
    (4.7,3.7) edge (4.5,3.5)
    (-5.7,1.7) edge (-5.5,1.5)
    (5.75,0) edge (5.5,0)
    (5.75,2) edge (5.5,2)
    (-5.75,1) edge (-5.5,1)
    (-5.75,3) edge (-5.5,3);
    \path[arrow]
    (5.5,1) edge (a41)
    (-5.5,2) edge (a42)
    (-5.5,0) edge (a40)
    (5.5,3) edge (a43);
    \path[dotted]
    (5.75,1) edge (5.5,1)
    (-5.75,2) edge (-5.5,2)
    (-5.75,0) edge (-5.5,0)
    (5.75,3) edge (5.5,3);
    \path[-]
    (a30) edge (5.5,.75)
    (a32) edge (5.5,2.75)
    (a31) edge (-5.5,1.75)
    (a23) edge (4,3.5)
    (a03) edge (2,3.5)
    (a13) edge (-3,3.5)
    (a33) edge (-5,3.5)
    (a43) edge (5.5,3.25)
    (a41) edge (5.5,1.25)
    (a40) edge (-5.5,0.25)
    (a42) edge (-5.5,2.25);
    \path[dotted]
    (5.5,.75) edge (5.8,.9)
    (5.5,2.75) edge (5.8,2.9)
    (-5.5,1.75) edge (-5.8,1.9)
    (4,3.5) edge (4.4,3.7)
    (2,3.5) edge (2.4,3.7)
    (-3,3.5) edge (-3.4,3.7)
    (-5,3.5) edge (-5.4,3.7)
    (5.5,3.25) edge (5.8,3.4)
    (5.5,1.25) edge (5.8,1.4)
    (-5.5,.25) edge (-5.8,0.4)
    (-5.5,2.25) edge (-5.8,2.4);
    \path[arrow]
    (sink) edge[loop left] ()
    (e00) edge[bend left=8] (sink)
    (e01) edge[bend right=8] (sink)
    (e02) edge[bend left=8] (sink)
    (e03) edge[bend right=8] (sink)
    (a00) edge[bend right=8] (sink)
    (a01) edge[bend left=8] (sink)
    (a02) edge[bend right=8] (sink)
    (a03) edge[bend left=8] (sink);
  \end{tikzpicture}}
  \caption{The "energy semantics" of the
    "vector system" of \cref{11-fig-mwg}: a circle (resp.\
    a square) at position $(i,j)$ of the grid denotes a configuration
    $\loc(i,j)$ (resp.\ $\loc'(i,j)$) controlled by~\Eve\ (resp.\
    \Adam).}\label{11-fig-nrg}
\end{figure}

Given a "colouring" $\col{:}\,E\to C$ and an objective~$\Omega$, we
call the resulting game $(\energy(\?V),\col,\Omega)$ an ""energy
game"".  In particular, we shall speak of "configuration
reachability", "coverability", "non-termination", and "parity@parity
vector game" "energy games" when replacing $\natural(\?V)$ by
$\energy(\?V)$ in \crefrange{11-pb-reach}{11-pb-parity}; the
"existential initial credit" variants are defined similarly.

\begin{example}\label{11-ex-cov-nrg}
  Consider the target configuration $\loc(2,2)$ in
  \cref{11-fig-mwg,12-fig-nrg}.  \Eve's "winning region" in the
  "configuration reachability" "energy game" is
  $\WE=\{\loc(n+2,n+2)\mid n\in\+N\}$, displayed on the left in
  \cref{11-fig-cov-nrg}.  In the "coverability" "energy game", \Eve's
  "winning region" is
  $\WE=\{\loc(m+2,n+2),\loc'(m+3,n+2)\mid m,n\in\+N\}$ displayed on
  the right in \cref{11-fig-cov-nrg}.
\end{example}
\begin{figure}[htbp]
  \centering\scalebox{.48}{
  \begin{tikzpicture}[auto,on grid,node distance=2.5cm]
    \draw[step=1,lightgray!50,dotted] (-5.7,0) grid (5.7,3.8);
    \draw[color=white](0,-.3) -- (0,3.8);
    \node at (0,3.9) (sink) {\color{red!70!black}\boldmath$\sink$};
    \draw[step=1,lightgray!50] (1,0) grid (5.5,3.5);
    \draw[step=1,lightgray!50] (-1,0) grid (-5.5,3.5);
    \node at (0,0)[lightgray,font=\scriptsize,fill=white] {0};
    \node at (0,1)[lightgray,font=\scriptsize,fill=white] {1};
    \node at (0,2)[lightgray,font=\scriptsize,fill=white] {2};
    \node at (0,3)[lightgray,font=\scriptsize,fill=white] {3};
    \node at (1,3.9)[lightgray,font=\scriptsize,fill=white] {0};
    \node at (2,3.9)[lightgray,font=\scriptsize,fill=white] {1};
    \node at (3,3.9)[lightgray,font=\scriptsize,fill=white] {2};
    \node at (4,3.9)[lightgray,font=\scriptsize,fill=white] {3};
    \node at (5,3.9)[lightgray,font=\scriptsize,fill=white] {4};
    \node at (-1,3.9)[lightgray,font=\scriptsize,fill=white] {0};
    \node at (-2,3.9)[lightgray,font=\scriptsize,fill=white] {1};
    \node at (-3,3.9)[lightgray,font=\scriptsize,fill=white] {2};
    \node at (-4,3.9)[lightgray,font=\scriptsize,fill=white] {3};
    \node at (-5,3.9)[lightgray,font=\scriptsize,fill=white] {4};
    \node at (1,0)[s-eve-small,lose] (e00) {};
    \node at (1,1)[s-adam-small,lose](a01){};
    \node at (1,2)[s-eve-small,lose] (e02){};
    \node at (1,3)[s-adam-small,lose](a03){};
    \node at (2,0)[s-adam-small,lose](a10){};
    \node at (2,1)[s-eve-small,lose] (e11){};
    \node at (2,2)[s-adam-small,lose](a12){};
    \node at (2,3)[s-eve-small,lose] (e13){};
    \node at (3,0)[s-eve-small,lose] (e20){};
    \node at (3,1)[s-adam-small,lose](a21){};
    \node at (3,2)[s-eve-small,win] (e22){};
    \node at (3,3)[s-adam-small,lose](a23){};
    \node at (4,0)[s-adam-small,lose](a30){};
    \node at (4,1)[s-eve-small,lose] (e31){};
    \node at (4,2)[s-adam-small,lose](a32){};
    \node at (4,3)[s-eve-small,win] (e33){};
    \node at (5,0)[s-eve-small,lose] (e40){};
    \node at (5,1)[s-adam-small,lose](a41){};
    \node at (5,2)[s-eve-small,lose] (e42){};
    \node at (5,3)[s-adam-small,lose](a43){};
    \node at (-1,0)[s-adam-small,lose](a00){};
    \node at (-1,1)[s-eve-small,lose] (e01){};
    \node at (-1,2)[s-adam-small,lose](a02){};
    \node at (-1,3)[s-eve-small,lose] (e03){};
    \node at (-2,0)[s-eve-small,lose] (e10){};
    \node at (-2,1)[s-adam-small,lose](a11){};
    \node at (-2,2)[s-eve-small,lose] (e12){};
    \node at (-2,3)[s-adam-small,lose](a13){};
    \node at (-3,0)[s-adam-small,lose](a20){};
    \node at (-3,1)[s-eve-small,lose] (e21){};
    \node at (-3,2)[s-adam-small,lose](a22){};
    \node at (-3,3)[s-eve-small,lose] (e23){};
    \node at (-4,0)[s-eve-small,lose] (e30){};
    \node at (-4,1)[s-adam-small,lose](a31){};
    \node at (-4,2)[s-eve-small,lose] (e32){};
    \node at (-4,3)[s-adam-small,lose](a33){};
    \node at (-5,0)[s-adam-small,lose](a40){};
    \node at (-5,1)[s-eve-small,lose] (e41){};
    \node at (-5,2)[s-adam-small,lose](a42){};
    \node at (-5,3)[s-eve-small,lose] (e43){};
    \path[arrow] % l, -1,-1, l
    (e11) edge (e00)
    (e22) edge (e11)
    (e31) edge (e20)
    (e32) edge (e21)
    (e21) edge (e10)
    (e12) edge (e01)
    (e23) edge (e12)
    (e33) edge (e22)
    (e13) edge (e02)
    (e43) edge (e32)
    (e42) edge (e31)
    (e41) edge (e30);
    \path[arrow] % l, -1,0, l'
    (e11) edge (a01)
    (e20) edge (a10)
    (e22) edge (a12)
    (e31) edge (a21)
    (e32) edge (a22)
    (e21) edge (a11)
    (e12) edge (a02)
    (e30) edge (a20)
    (e10) edge (a00)
    (e13) edge (a03)
    (e23) edge (a13)
    (e33) edge (a23)
    (e43) edge (a33)
    (e42) edge (a32)
    (e41) edge (a31)
    (e40) edge (a30);
    \path[arrow] % l', -1,0, l
    (a11) edge (e01)
    (a20) edge (e10)
    (a22) edge (e12)
    (a31) edge (e21)
    (a32) edge (e22)
    (a21) edge (e11)
    (a12) edge (e02)
    (a30) edge (e20)
    (a10) edge (e00)
    (a33) edge (e23)
    (a23) edge (e13)
    (a13) edge (e03)
    (a43) edge (e33)
    (a42) edge (e32)
    (a41) edge (e31)
    (a40) edge (e30);
    \path[arrow] % l', 2,1, l
    (a01) edge (e22)
    (a10) edge (e31)
    (a11) edge (e32)
    (a00) edge (e21)
    (a02) edge (e23)
    (a12) edge (e33)
    (a22) edge (e43)
    (a21) edge (e42)
    (a20) edge (e41);
    \path[arrow] % dotted to Eve
    (-5.5,3.5) edge (e43)
    (5.5,2.5) edge (e42)
    (2.5,3.5) edge (e13)
    (5.5,0.5) edge (e40)
    (-5.5,1.5) edge (e41)
    (-3.5,3.5) edge (e23)
    (-1.5,3.5) edge (e03)
    (4.5,3.5) edge (e33)
    (5.5,0) edge (e40)
    (5.5,2) edge (e42)
    (-5.5,1) edge (e41)
    (-5.5,3) edge (e43);
    \path[dotted]
    (-5.7,3.7) edge (-5.5,3.5)
    (5.7,2.7) edge (5.5,2.5)
    (2.7,3.7) edge (2.5,3.5)
    (5.7,0.7) edge (5.5,0.5)
    (-3.7,3.7) edge (-3.5,3.5)
    (-1.7,3.7) edge (-1.5,3.5)
    (4.7,3.7) edge (4.5,3.5)
    (-5.7,1.7) edge (-5.5,1.5)
    (5.75,0) edge (5.5,0)
    (5.75,2) edge (5.5,2)
    (-5.75,1) edge (-5.5,1)
    (-5.75,3) edge (-5.5,3);
    \path[arrow]
    (5.5,1) edge (a41)
    (-5.5,2) edge (a42)
    (-5.5,0) edge (a40)
    (5.5,3) edge (a43);
    \path[dotted]
    (5.75,1) edge (5.5,1)
    (-5.75,2) edge (-5.5,2)
    (-5.75,0) edge (-5.5,0)
    (5.75,3) edge (5.5,3);
    \path[-]
    (a30) edge (5.5,.75)
    (a32) edge (5.5,2.75)
    (a31) edge (-5.5,1.75)
    (a23) edge (4,3.5)
    (a03) edge (2,3.5)
    (a13) edge (-3,3.5)
    (a33) edge (-5,3.5)
    (a43) edge (5.5,3.25)
    (a41) edge (5.5,1.25)
    (a40) edge (-5.5,0.25)
    (a42) edge (-5.5,2.25);
    \path[dotted]
    (5.5,.75) edge (5.8,.9)
    (5.5,2.75) edge (5.8,2.9)
    (-5.5,1.75) edge (-5.8,1.9)
    (4,3.5) edge (4.4,3.7)
    (2,3.5) edge (2.4,3.7)
    (-3,3.5) edge (-3.4,3.7)
    (-5,3.5) edge (-5.4,3.7)
    (5.5,3.25) edge (5.8,3.4)
    (5.5,1.25) edge (5.8,1.4)
    (-5.5,.25) edge (-5.8,0.4)
    (-5.5,2.25) edge (-5.8,2.4);
    \path[arrow]
    (sink) edge[loop left] ()
    (e00) edge[bend left=8] (sink)
    (e01) edge[bend right=8] (sink)
    (e02) edge[bend left=8] (sink)
    (e03) edge[bend right=8] (sink)
    (a00) edge[bend right=8] (sink)
    (a01) edge[bend left=8] (sink)
    (a02) edge[bend right=8] (sink)
    (a03) edge[bend left=8] (sink);
  \end{tikzpicture}}\quad~~\scalebox{.48}{
  \begin{tikzpicture}[auto,on grid,node distance=2.5cm]
    \draw[step=1,lightgray!50,dotted] (-5.7,0) grid (5.7,3.8);
    \draw[color=white](0,-.3) -- (0,3.8);
    \node at (0,3.9) (sink) {\color{red!70!black}\boldmath$\sink$};
    \draw[step=1,lightgray!50] (1,0) grid (5.5,3.5);
    \draw[step=1,lightgray!50] (-1,0) grid (-5.5,3.5);
    \node at (0,0)[lightgray,font=\scriptsize,fill=white] {0};
    \node at (0,1)[lightgray,font=\scriptsize,fill=white] {1};
    \node at (0,2)[lightgray,font=\scriptsize,fill=white] {2};
    \node at (0,3)[lightgray,font=\scriptsize,fill=white] {3};
    \node at (1,3.9)[lightgray,font=\scriptsize,fill=white] {0};
    \node at (2,3.9)[lightgray,font=\scriptsize,fill=white] {1};
    \node at (3,3.9)[lightgray,font=\scriptsize,fill=white] {2};
    \node at (4,3.9)[lightgray,font=\scriptsize,fill=white] {3};
    \node at (5,3.9)[lightgray,font=\scriptsize,fill=white] {4};
    \node at (-1,3.9)[lightgray,font=\scriptsize,fill=white] {0};
    \node at (-2,3.9)[lightgray,font=\scriptsize,fill=white] {1};
    \node at (-3,3.9)[lightgray,font=\scriptsize,fill=white] {2};
    \node at (-4,3.9)[lightgray,font=\scriptsize,fill=white] {3};
    \node at (-5,3.9)[lightgray,font=\scriptsize,fill=white] {4};
    \node at (1,0)[s-eve-small,lose] (e00) {};
    \node at (1,1)[s-adam-small,lose](a01){};
    \node at (1,2)[s-eve-small,lose] (e02){};
    \node at (1,3)[s-adam-small,lose](a03){};
    \node at (2,0)[s-adam-small,lose](a10){};
    \node at (2,1)[s-eve-small,lose] (e11){};
    \node at (2,2)[s-adam-small,lose](a12){};
    \node at (2,3)[s-eve-small,lose] (e13){};
    \node at (3,0)[s-eve-small,lose] (e20){};
    \node at (3,1)[s-adam-small,lose](a21){};
    \node at (3,2)[s-eve-small,win] (e22){};
    \node at (3,3)[s-adam-small,lose](a23){};
    \node at (4,0)[s-adam-small,lose](a30){};
    \node at (4,1)[s-eve-small,lose] (e31){};
    \node at (4,2)[s-adam-small,win](a32){};
    \node at (4,3)[s-eve-small,win] (e33){};
    \node at (5,0)[s-eve-small,lose] (e40){};
    \node at (5,1)[s-adam-small,lose](a41){};
    \node at (5,2)[s-eve-small,win] (e42){};
    \node at (5,3)[s-adam-small,win](a43){};
    \node at (-1,0)[s-adam-small,lose](a00){};
    \node at (-1,1)[s-eve-small,lose] (e01){};
    \node at (-1,2)[s-adam-small,lose](a02){};
    \node at (-1,3)[s-eve-small,lose] (e03){};
    \node at (-2,0)[s-eve-small,lose] (e10){};
    \node at (-2,1)[s-adam-small,lose](a11){};
    \node at (-2,2)[s-eve-small,lose] (e12){};
    \node at (-2,3)[s-adam-small,lose](a13){};
    \node at (-3,0)[s-adam-small,lose](a20){};
    \node at (-3,1)[s-eve-small,lose] (e21){};
    \node at (-3,2)[s-adam-small,lose](a22){};
    \node at (-3,3)[s-eve-small,win] (e23){};
    \node at (-4,0)[s-eve-small,lose] (e30){};
    \node at (-4,1)[s-adam-small,lose](a31){};
    \node at (-4,2)[s-eve-small,win] (e32){};
    \node at (-4,3)[s-adam-small,win](a33){};
    \node at (-5,0)[s-adam-small,lose](a40){};
    \node at (-5,1)[s-eve-small,lose] (e41){};
    \node at (-5,2)[s-adam-small,win](a42){};
    \node at (-5,3)[s-eve-small,win] (e43){};
    \path[arrow] % l, -1,-1, l
    (e11) edge (e00)
    (e22) edge (e11)
    (e31) edge (e20)
    (e32) edge (e21)
    (e21) edge (e10)
    (e12) edge (e01)
    (e23) edge (e12)
    (e33) edge (e22)
    (e13) edge (e02)
    (e43) edge (e32)
    (e42) edge (e31)
    (e41) edge (e30);
    \path[arrow] % l, -1,0, l'
    (e11) edge (a01)
    (e20) edge (a10)
    (e22) edge (a12)
    (e31) edge (a21)
    (e32) edge (a22)
    (e21) edge (a11)
    (e12) edge (a02)
    (e30) edge (a20)
    (e10) edge (a00)
    (e13) edge (a03)
    (e23) edge (a13)
    (e33) edge (a23)
    (e43) edge (a33)
    (e42) edge (a32)
    (e41) edge (a31)
    (e40) edge (a30);
    \path[arrow] % l', -1,0, l
    (a11) edge (e01)
    (a20) edge (e10)
    (a22) edge (e12)
    (a31) edge (e21)
    (a32) edge (e22)
    (a21) edge (e11)
    (a12) edge (e02)
    (a30) edge (e20)
    (a10) edge (e00)
    (a33) edge (e23)
    (a23) edge (e13)
    (a13) edge (e03)
    (a43) edge (e33)
    (a42) edge (e32)
    (a41) edge (e31)
    (a40) edge (e30);
    \path[arrow] % l', 2,1, l
    (a01) edge (e22)
    (a10) edge (e31)
    (a11) edge (e32)
    (a00) edge (e21)
    (a02) edge (e23)
    (a12) edge (e33)
    (a22) edge (e43)
    (a21) edge (e42)
    (a20) edge (e41);
    \path[arrow] % dotted to Eve
    (-5.5,3.5) edge (e43)
    (5.5,2.5) edge (e42)
    (2.5,3.5) edge (e13)
    (5.5,0.5) edge (e40)
    (-5.5,1.5) edge (e41)
    (-3.5,3.5) edge (e23)
    (-1.5,3.5) edge (e03)
    (4.5,3.5) edge (e33)
    (5.5,0) edge (e40)
    (5.5,2) edge (e42)
    (-5.5,1) edge (e41)
    (-5.5,3) edge (e43);
    \path[dotted]
    (-5.7,3.7) edge (-5.5,3.5)
    (5.7,2.7) edge (5.5,2.5)
    (2.7,3.7) edge (2.5,3.5)
    (5.7,0.7) edge (5.5,0.5)
    (-3.7,3.7) edge (-3.5,3.5)
    (-1.7,3.7) edge (-1.5,3.5)
    (4.7,3.7) edge (4.5,3.5)
    (-5.7,1.7) edge (-5.5,1.5)
    (5.75,0) edge (5.5,0)
    (5.75,2) edge (5.5,2)
    (-5.75,1) edge (-5.5,1)
    (-5.75,3) edge (-5.5,3);
    \path[arrow]
    (5.5,1) edge (a41)
    (-5.5,2) edge (a42)
    (-5.5,0) edge (a40)
    (5.5,3) edge (a43);
    \path[dotted]
    (5.75,1) edge (5.5,1)
    (-5.75,2) edge (-5.5,2)
    (-5.75,0) edge (-5.5,0)
    (5.75,3) edge (5.5,3);
    \path[-]
    (a30) edge (5.5,.75)
    (a32) edge (5.5,2.75)
    (a31) edge (-5.5,1.75)
    (a23) edge (4,3.5)
    (a03) edge (2,3.5)
    (a13) edge (-3,3.5)
    (a33) edge (-5,3.5)
    (a43) edge (5.5,3.25)
    (a41) edge (5.5,1.25)
    (a40) edge (-5.5,0.25)
    (a42) edge (-5.5,2.25);
    \path[dotted]
    (5.5,.75) edge (5.8,.9)
    (5.5,2.75) edge (5.8,2.9)
    (-5.5,1.75) edge (-5.8,1.9)
    (4,3.5) edge (4.4,3.7)
    (2,3.5) edge (2.4,3.7)
    (-3,3.5) edge (-3.4,3.7)
    (-5,3.5) edge (-5.4,3.7)
    (5.5,3.25) edge (5.8,3.4)
    (5.5,1.25) edge (5.8,1.4)
    (-5.5,.25) edge (-5.8,0.4)
    (-5.5,2.25) edge (-5.8,2.4);
    \path[arrow]
    (sink) edge[loop left] ()
    (e00) edge[bend left=8] (sink)
    (e01) edge[bend right=8] (sink)
    (e02) edge[bend left=8] (sink)
    (e03) edge[bend right=8] (sink)
    (a00) edge[bend right=8] (sink)
    (a01) edge[bend left=8] (sink)
    (a02) edge[bend right=8] (sink)
    (a03) edge[bend left=8] (sink);
  \end{tikzpicture}}
  \caption{The "winning regions" of \Eve\ in the
    "configuration reachability" "energy game" (left) and the
    "coverability" "energy game"
    (right) on the graphs of \cref{11-fig-mwg,12-fig-nrg} with target
    configuration~$\ell(2,2)$.  The winning vertices are in filled in
    green, while the losing ones are filled with white with a red
    border; the "sink" is always losing.}\label{11-fig-cov-nrg}
\end{figure}

The reader might have noticed that the "natural semantics" of the
"asymmetric" system of \cref{11-fig-avg} and the "energy semantics" of
the system of \cref{11-fig-mwg} are essentially the same.  This
correspondence is quite general.
\begin{lemma}\label{11-fact-nrg}
  "Energy games" and "asymmetric" "vector games" are
  \logspace-equivalent for "configuration reachability",
  "coverability", "non-termination", and "parity@parity vector games",
  both with "given" and with "existential initial credit".
\end{lemma}
\begin{proof}
  Let us first reduce "asymmetric vector games" to "energy games".
  Given $\?V$, $\col$, and $\Omega$ where $\?V$ is "asymmetric" and
  $\Eve$ loses if the play ever visits the "sink"~$\sink$, we see that
  $\Eve$ wins $(\natural(\?V),\col,\Omega)$ from some $v\in V$ if and
  only if she wins $(\energy(\?V),\col,\Omega)$ from $v$.  Of course,
  this might not be true if~$\?V$ is not "asymmetric", as seen for
  instance in \cref{11-ex-cov,12-ex-cov-nrg}.
  % It suffices for this to observe
  % that, for all $v\in V$,
  % \begin{description}
  %   \item[(colours)] $\col(v)$ is the same in both games;
  %   \item[(zig \Eve)] if $v\in\VE$ and $(v,v')\in E$ with $v'\neq\sink$,
  %     then $(v,v')\in E_\+E$: indeed, $v'\neq\sink$ entails that
  %     $v$ is a configuration $\loc(\vec v)$ and $v'=\loc'(\vec v+\vec
  %     u)$ for some action $\loc\step{\vec u}\loc'\in\Act$ with
  %     $\vec v+\vec u\geq\vec 0$, but then $(v,v')\in E_\+E$;
  %   \item[(zag \Eve)] if $v\in\VE$ and $(v,v')\in E_\+E$ with $v'\neq\sink$,
  %     then $(v,v')\in E$: by the same reasoning;
  %   \item[(zig \Adam)] if $v\in\VA$ and $(v,v')\in E_\+E$,
  %     then $(v,v')\in E$: indeed, either $v=\sink$ and then $(v,v')\in
  %     E$, or $v=\loc(\vec v)$ and $v'=\sink$ and then there are no
  %     outgoing actions from~$\loc$ in $\Act$ thus $(v,v')\in
  %     E$, or $v=\loc(\vec v)$ and $v'=\loc'(\vec v)$ for some
  %     action $\loc\step{\vec 0}\loc'\in\Act$ (recall that~$\?V$ is
  %     assumed to be asymmetric) and then $(v,v')\in E$;
  %   \item[(zag \Adam)] if $v\in\VA$ and $(v,v')\in E$,
  %     then $(v,v')\in E_\+E$: by the same reasoning.
  %   \end{description}

  \medskip Conversely, let us reduce "energy games" to "asymmetric
  vector games".  Consider
  $\?V=(\Loc,\Act,\Loc_\mEve,\Loc_\mAdam,\dd)$, a colouring $\col$
  defined from a vertex colouring $\vcol$ by
  $\col(e)\eqdef\vcol(\ing(e))$, and an objective $\Omega$, where
  $\vcol$ and $\Omega$ are such that $\Eve$ loses if the play ever
  visits the "sink"~$\sink$ and such that, for all $\pi\in C^\ast$,
  $p\in C$, and $\pi'\in C^\omega$, $\pi p\pi'\in\Omega$ if and only
  if $\pi pp\pi'\in\Omega$ (we shall call $\Omega$
  \emph{stutter-invariant}, and the objectives in the statement are
  indeed stutter-invariant).  We construct an "asymmetric vector
  system"
  $\?V'\eqdef(\Loc\uplus\Loc_\Act,\Act',\Loc_\mEve\uplus\Loc_\Act,\Loc_\mAdam,\dd)$
  where we add the following locations controlled by~\Eve:
    \begin{align*}
      \Loc_\Act&\eqdef\{\loc_a\mid a=(\loc\step{\vec
                 u}\loc')\in\Act\text{ and }\loc\in\Loc_\mAdam\}\;.
      \intertext{We also modify the set of actions:}
      \Act'&\eqdef\{\loc\step{\vec u}\loc'\mid \loc\step{\vec
             u}\loc'\in\Act\text{ and }\loc\in\Loc_\mEve\}\\
      &\:\cup\:\{\loc\step{\vec 0}\loc_a,\;\loc_a\step{\vec u}\loc'\mid a=(\loc\step{\vec u}\loc')\in\Act\text{ and }\loc\in\Loc_\mAdam\}\;.
    \end{align*}
    \Cref{11-fig-avg} presents the result of this reduction on the
    system of \cref{11-fig-mwg}.  We define a vertex colouring
    $\vcol'$ of $\arena_\+N(\?V')$ with $\vcol'(v)\eqdef\vcol(v)$ for
    all $v\in \Loc\times\+N^\dd\uplus\{\sink\}$ and
    $\vcol'(\loc_a(\vec v))\eqdef\vcol(\loc(\vec v))$ if
    $a=(\loc\step{\vec u}\loc')\in\Act$.  Then, for all vertices
    $v\in V$, \Eve\ wins from~$v$ in the "energy game"
    $(\energy(\?V),\col,\Omega)$ if and only if she wins from~$v$ in
    the "vector game" $(\natural(\?V'),\col',\Omega)$.  The crux of
    the argument is that, in a configuration $\loc(\vec v)$ where
    $\loc\in\Loc_\mAdam$, if $a=(\loc\step{\vec u}\loc')\in\Act$ is an
    action with $\vec v+\vec u\not\geq\vec 0$, in the "energy
    semantics", \Adam\ can force the play into the "sink" by
    playing~$a$; the same occurs in $\?V'$ with the "natural
    semantics", as \Adam\ can now choose to play
    $\loc\step{\vec 0}\loc_a$ where \Eve\ has only
    $\loc_a\step{\vec u}\loc'$ at her disposal, which leads to the
    sink.\todoquestion{Is that clear?}
    % First note that the available edges for \Eve\ that avoid the "sink"
    % are the same in both games.  Furthermore, if $(v,v')$ is an edge
    % in the "energy game" from $v\in\VA$, then either $v'$ is a
    % configuration $v'=\loc'(\vec v+\vec u)$ for some
    % $t=(\loc\step{\vec u}\loc')\in\Act$ such that $v=\loc(\vec v)$,
    % and in that case the sequence of moves
    % $(v,\loc_{t)(\vec v))(\loc_t(\vec v),v')$ is forced upon \Eve\ in
    %   the "vector game".
\end{proof}

In turn, "energy games" with "existential initial credit" are related
to the "multi-dimensional mean-payoff games" of
\cref{12-chap:multiobjective}.% We shall see the
% case of dimension~one in \cref{11-subsec:mono-dim1}.


%\subsubsection{Capped Semantics}
\label{11-capping}

In resource-conscious systems, there is often a maximal capacity for
resources: for instance, a battery cannot be charged any further once
full.  We can model this by providing an alternative semantics
for vector systems.  Given a capacity $C\in\+N$, for a vector~$\vec v$
in~$\+N^\dd$, let us write $\capp v$ for the vector `capped' at~$C$:
for all~$1\leq i\leq\dd$, $\capp v(i)\eqdef\vec v(i)$ if $\vec v(i)<C$
and $\capp v\eqdef C$ if $\vec v(i)\geq C$.  We define for a
capacity~$C\in\+N$ the ""capped semantics""
$\capped(\?V)=(V^C,\overline{E}^C,\VE^C,\VA^C)$ of a "vector system"~$\?V$ by
\begin{align*}
  V^C&\eqdef\{\loc(\vec v)\mid\loc\in\Loc\text{ and }\|\vec v\|<C\}\;,\\
  \overline{E}^C&\eqdef \{(\loc(\vec v),\loc'(\capp{v+\vec u})\mid\loc\step{\vec
       u}\loc'\in\Act\text{ and }\vec v+\vec u\geq\vec 0\}\\
     &\:\cup\:\{(\loc(\vec v),\sink)\mid\forall\loc\step{\vec
               u}\loc'\in\Act\mathbin.\vec v+\vec u\not\geq\vec
               0\}
     \cup\{(\sink,\sink)\}\;.
\end{align*}
As usual, $\VE^C\eqdef V^C\cap\Loc_\mEve\times\+N^\dd$ and
$\VA^C\eqdef V^C\cap\Loc_\mAdam\times\+N^\dd$.  This is the same as
the "energy semantics" where we cap the vectors resulting from actions
in~$\Act$.  All the configurations in this arena have "norm" less
than~$C$, thus $|V^C|=|\Loc| C^\dd+1$, and the qualitative games of
\cref{chap:regular} are decidable over this "arena".


% Local IspellDict: british


\subsection{Bounded Semantics}
\label{11-bounding}

While \Adam\ wins immediately in an "energy game" if a resource gets
depleted, he also wins in a "bounding game" if a resource reaches a
certain bound~$B$.  %By contrast with "capped semantics", t
This is
a \emph{hard upper bound}, allowing to model systems where exceeding a
capacity results in failure, like a dam that overflows and floods the
area.  We define for a bound~$B\in\+N$ the ""bounded semantics""
$\bounded(\?V)=(V^B,E^B,\VE^B,\VA^B)$ of a "vector system"~$\?V$ by
\begin{align*}
  V^B&\eqdef\{\loc(\vec v)\mid\loc\in\Loc\text{ and }\|\vec v\|<B\}\;,\\
  E^B&\eqdef \{(\loc(\vec v),\loc'(\vec v+\vec u))\mid\loc\step{\vec
       u}\loc'\in\Act,\vec v+\vec u\geq\vec 0,\text{ and }\|\vec
       v+\vec u\|<B\}\\
     &\:\cup\:\{(\loc(\vec v),\sink)\mid\forall\loc\step{\vec
               u}\loc'\in\Act\mathbin.\vec v+\vec u\not\geq\vec
               0\text{ or }\|\vec v+\vec u\|\geq B\}
     \cup\{(\sink,\sink)\}\;.
\end{align*}
As usual, $\VE^B\eqdef V^B\cap\Loc_\mEve\times\+N^\dd$ and
$\VA^B\eqdef V^B\cap\Loc_\mAdam\times\+N^\dd$.  Any edge from the
"energy semantics" that would bring to a configuration $\loc(\vec v)$
with $\vec v(i)\geq B$ for some $1\leq i\leq\dd$ leads instead to the
sink.  All the configurations in this arena have "norm" less than~$B$,
thus $|V^B|=|\Loc| B^\dd+1$, and the qualitative games of
\cref{2-chap:regular} are decidable over this "arena".

Our focus here is on "non-termination" games played on the "bounded
semantics" where~$B$ is not given as part of the input, but quantified
existentially.  As usual, the "existential initial credit" variant
of \cref{11-pb-bounding} is obtained by quantifying~$\vec v_0$
existentially in the question.
\decpb["bounding game" with "given initial credit"]%
{\label{11-pb-bounding} A "vector system"
  $\?V=(\Loc,\Act,\Loc_\mEve,\Loc_\mAdam,\dd)$, an initial location
  $\loc_0\in\Loc$, and an initial credit $\vec v_0\in\+N^\dd$.}%
  {Does there exist $B\in\+N$ such that \Eve\ has a strategy to avoid the
  "sink"~$\sink$ from $\loc_0(\vec v_0)$ in the "bounded
  semantics"?  That is, does there exist $B\in\+N$ such that she wins
  the ""bounding"" game $(\bounded(\?V),\col,\Safe)$ from
  $\loc_0(\vec v_0)$, where $\col(e)\eqdef\Lose$ if and only if $\ing(e)=\sink$?}

\begin{lemma}\label{11-parity2bounding}
  There is a \logspace\ reduction from "parity@parity vector games"
  "asymmetric" "vector games" to "bounding games", both with "given"
  and with "existential initial credit".
\end{lemma}
\begin{proof}
  Given an "asymmetric vector system"
  $\?V=(\Loc,\Act,\Loc_\mEve,\Loc_\mAdam,\dd)$, a location colouring
  $\lcol{:}\,\Loc\to\{1,\dots,2d\}$, and an initial location
  $\loc_0\in\Loc$, we construct a "vector system" $\?V'$ of dimension
  $\dd'\eqdef\dd+d$ as described in \cref{11-fig-bounding}, where the
  priorities in~$\?V$ for $p\in\{1,\dots,d\}$ are indicated above the
  corresponding locations.
  
  \begin{figure}[htbp]
    \centering
    \begin{tikzpicture}[auto,on grid,node distance=1.5cm]
      \node(to){$\mapsto$};
      \node[anchor=east,left=2.5cm of to](mm){"asymmetric vector system"~$\?V$};
      \node[anchor=west,right=2.5cm of to](mwg){"vector system"~$\?V'$};
      % Eve location, even parity
      \node[below=1.3cm of to](imap){$\rightsquigarrow$};
      \node[s-eve,left=2.75cm of imap](i0){$\loc$};
      \node[black!50,above=.4 of i0,font=\scriptsize]{$2p$};
      \node[right=of i0](i1){$\loc'$};
      \node[right=1.25cm of imap,s-eve](i2){$\loc$};
      \node[right=1.8 of i2,s-eve-small](i3){};      
      \node[right=1.8 of i3](i4){$\loc'$};      
      \path[arrow,every node/.style={font=\footnotesize}]
      (i0) edge node{$\vec u$} (i1)
      (i2) edge[loop above] node{$\forall 1\leq i\leq\dd\mathbin.-\vec e_i$} ()
      (i2) edge node{$\vec u$} (i3)
      (i3) edge[loop below] node{$\forall 1\leq j\leq p\mathbin.\vec e_{\dd+j}$} ()
      (i3) edge node{$\vec 0$} (i4);
      % Eve location, odd parity
      \node[below=2cm of imap](dmap){$\rightsquigarrow$};
      \node[s-eve,left=2.75cm of dmap](d0){$\loc$};
      \node[black!50,above=.4 of d0,font=\scriptsize]{$2p-1$};
      \node[right=of d0](d1){$\loc'$};
      \node[right=1.25cm of dmap,s-eve](d2){$\loc$};
      \node[right=2 of d2](d3){$\loc'$};
      \path[arrow,every node/.style={font=\footnotesize}]
      (d0) edge node{$\vec u$} (d1)
      (d2) edge[loop above] node{$\forall 1\leq i\leq\dd\mathbin.-\vec e_i$} ()
      (d2) edge node{$\vec u-\vec e_{\dd+p}$} (d3);
      % Adam location, even parity
      \node[below=1.1cm of dmap](zmap){$\rightsquigarrow$};
      \node[s-adam,left=2.75cm of zmap](z0){$\loc$};
      \node[black!50,above=.4 of z0,font=\scriptsize]{$2p$};
      \node[right=of z0](z1){$\loc'$};
      \node[right=1.25cm of zmap,s-adam](z2){$\loc$};
      \node[right=of z2,s-eve-small](z3){};
      \node[right=of z3](z4){$\loc'$};
      \path[arrow,every node/.style={font=\footnotesize}]
      (z0) edge node{$\vec 0$} (z1)
      (z2) edge node{$\vec 0$} (z3)
      (z3) edge node{$\vec 0$} (z4)
      (z3) edge[loop below] node{$\forall 1\leq j\leq p\mathbin.\vec e_{\dd+j}$} ();
      % Adam location, odd parity
      \node[below=1.6cm of zmap](amap){$\rightsquigarrow$};
      \node[s-adam,left=2.75cm of amap](a0){$\loc$};
      \node[black!50,above=.4 of a0,font=\scriptsize]{$2p-1$};
      \node[right=of a0](a1){$\loc'$};
      \node[right=1.25cm of amap,s-adam](a2){$\loc$};
      \node[right=2 of a2](a3){$\loc'$};
      \path[arrow,every node/.style={font=\footnotesize}]
      (a0) edge node{$\vec 0$} (a1)
      (a2) edge node{$-\vec e_{\dd+p}$} (a3);
    \end{tikzpicture}
    \caption{Schema of the reduction to
      "bounding games" in the proof of \cref{11-parity2bounding}.}\label{11-fig-bounding}
  \end{figure}
  
  If \Eve\ wins the "bounding game" played over $\?V'$ from some
  configuration $\loc_0(\vec v_0)$, then she also wins the "parity
  vector game" played over~$\?V$ from the configuration $\loc_0(\vec
  v'_0)$ where $\vec v'_0$ is the projection of $\vec v_0$
  to~$\+N^\dd$.  Indeed, she can play essentially the same strategy:
  by \cref{11-fact-mono} she can simply ignore the new decrement
  self loops, while the actions on the components in
  $\{\dd+1,\dots,\dd+d\}$ ensure that the maximal priority visited
  infinitely often is even---otherwise some decrement $-\vec
  e_{\dd+p}$ would be played infinitely often but the increment $\vec
  e_{\dd+p}$ only finitely often.\todoquestion{Should I provide more details?}
  

  \medskip
  Conversely, consider the "parity@parity vector game" game~$\game$ played over
  $\natural(\?V)$ with the colouring defined by~$\lcol$.  Then the
  "Pareto limit" of the game is finite, thus there exists a natural
  number
  \begin{equation}\label{11-b0}
    B_0\eqdef 1+\max_{\loc_0(\vec v_0)\in\mathsf{Pareto}(\?G)}\|\vec
  v_0\|
  \end{equation} bounding the "norms" of the minimal winning configurations.
  For a vector~$\vec v$ in~$\+N^\dd$, let us write $\capp[B_0]v$ for
  the vector `capped' at~$B$: for all~$1\leq i\leq\dd$,
  $\capp[B_0]v(i)\eqdef\vec v(i)$ if $\vec v(i)<B_0$ and
  $\capp[B_0]v\eqdef B_0$ if $\vec v(i)\geq B_0$.

  %% \begin{claim}\label{11-cl-capped}
  %%   If $\loc(\vec v)\in\WE(\game)$, then $\loc(\capp[B_0]v)\in\WE(\game)$.
  %% \end{claim}
  %% Indeed, by definition of the "Pareto limit"~$\mathsf{Pareto}(\game)$,
  %% if $\loc(\vec v)\in\WE(\game)$, then there exists~$\vec v'\leq\vec v$
  %% such that $\loc(\vec v')\in\mathsf{Pareto}(\game)$.  By definition of
  %% the bound~$B_0$, $\|\vec v'\|<B_0$.  Thus $\vec v'\leq\capp[B_0]v$.
  %% Thus $\loc(\capp[B_0]v)\in\WE(\game)$.

  Consider now some configuration $\loc_0(\vec
  v_0)\in\mathsf{Pareto}(\game)$.  As seen in \cref{11-fact-finmem},
  since $\loc_0(\vec v_0)\in\WE(\game)$, there is a finite
  "self-covering tree" witnessing the fact, and an associated winning
  strategy.  Let $H(\loc_0(\vec v_0))$ denote the height of this
  "self-covering tree" and observe that all the configurations in this
  tree have norm bounded by $\|\vec v_0\|+\|\Act\|\cdot H(\loc_0(\vec
  v_0))$.
  Let us define
  \begin{equation}\label{11-b}
   B\eqdef B_0+(\|\Act\|+1)\cdot \max_{\loc_0(\vec
  v_0)\in\mathsf{Pareto}(\?G)}H(\loc_0(\vec v_0))\;.
  \end{equation}
  This is a bound on the norm of the configurations appearing on the
  (finitely many) self-covering trees spawned by the elements
  of~$\mathsf{Pareto}(\game)$.  Note that $B\geq B_0+(\|\Act\|+1)$ since
  a self-covering tree has height at least~one.

  Consider the "non-termination" game
  $\game_B\eqdef(\bounded(\?V'),\col',\Safe)$ played over the
  "bounded semantics" defined by~$B$, where $\col'(e)=\Lose$ if and
  only if $\ing(e)=\sink$.  Let $\vec b\eqdef\sum_{1\leq p\leq
  d}(B-1)\cdot\vec e_{\dd+p}$.
  {\renewcommand{\qedsymbol}{}
  \begin{claim}\label{11-cl-parity2bounding} If $\loc_0(\vec
    v)\in\WE(\game)$, then
    $\loc_0(\capp[B_0]{v}+\vec b)\in\WE(\game_B)$.
  \end{claim}}
  Indeed, by definition of the "Pareto
  limit"~$\mathsf{Pareto}(\game)$, if $\loc_0(\vec v)\in\WE(\game)$,
  then there exists~$\vec v_0\leq\vec v$ such that $\loc_0(\vec
  v_0)\in\mathsf{Pareto}(\game)$.  By definition of the bound~$B_0$,
  $\|\vec v_0\|<B_0$, thus $\vec v_0\leq\capp[B_0]v$.  Consider the
  "self-covering tree" of height~$H(\loc_0(\vec v_0))$ associated
  to~$\loc_0(\vec v_0)$, and the strategy~$\sigma'$ defined by the
  memory structure from the
  proof of \cref{11-fact-finmem}.  This is a winning strategy for \Eve
  in $\game$ starting from $\loc_0(\vec v_0)$, and
  by \cref{11-fact-mono}, it is also winning
  from~$\loc_0(\capp[B_0]v)$.
    
  Here is how \Eve\ wins $\game_B$ from~$\loc_0(\capp[B_0]v+\vec b)$.
  She essentially follows the strategy~$\sigma'$, with two
  modifications.  First, whenever $\sigma'$ goes to a "return node"
  $\loc(\vec v)$ instead of a leaf $\loc(\vec v')$---thus $\vec
  v\leq\vec v'$---, the next time \Eve\ has the control, she uses the
  self loops to decrement the current configuration by $\vec v'-\vec
  v$.  This ensures that any play consistent with the modified
  strategy remains between zero and $B-1$ on the components
  in~$\{1,\dots,\dd\}$.  (Note that if she never regains the control,
  the current vector never changes any more since~$\?V$ is
  "asymmetric".)

  Second, whenever a play in~$\game$ visits a location with even
  parity~$2p$ for some~$p$ in~$\{1,\dots,d\}$, \Eve\ has the opportunity
  to increase the coordinates in~$\{\dd+1,\dots,\dd+p\}$ in~$\game_B$.
  She does so and increments until all these components reach~$B-1$.
  This ensures that any play consistent with the modified strategy
  remains between zero and $B-1$ on the components
  in~$\{\dd+1,\dots,\dd+p\}$.  Indeed, $\sigma'$ guarantees that the
  longest sequence of moves before a play visits a location with
  maximal even priority is bounded by $H(\loc_0(\vec v_0))$, thus the
  decrements $-\vec e_{\dd+p}$ introduced in~$\game_B$ by the
  locations from~$\game$ with odd parity~$2p-1$ will never force the
  play to go negative.\todoquestion{Is that clear enough?}
\end{proof}

The bound~$B$ defined in~\cref{11-b} in the previous proof is not
constructive, and possibly much larger than really required.
Nevertheless, one can sometimes show that an explicit~$B$ suffices in
a "bounding game".
A simple example is provided by the "coverability" "asymmetric"
"vector games" with "existential initial credit" arising from
\cref{11-cov2parity}, i.e., where the objective is to reach some
location~$\loc_f$.  Indeed, it is rather straightforward that there
exists a suitable initial credit such that \Eve\ wins the game if and
only if she wins the finite reachability game played over the
underlying directed graph over~$\Loc$ where we ignore the counters.
Thus, for an initial location~$\loc_0$, $B_0=|\Loc|\cdot\|\Act\|+1$
bounds the norm of the necessary initial credit, while a simple path
may visit at most~$|\Loc|$ locations, thus
$B=B_0+|\Loc|\cdot\|\Act\|$ suffices for \Eve\ to win the constructed
"bounding game".

In the general case of "bounding games" with "existential initial
credit", an explicit bound can be established.  The proof goes
along very different lines and is too involved to fit in this chapter,
but we refer the reader
to \cite{Jurdzinski&Lazic&Schmitz:2015,Colcombet&Jurdzinski&Lazic&Schmitz:2017}
for details.
\begin{theorem}\label{11-th-bounding}
  If \Eve\ wins a "bounding game" with "existential initial credit"
  defined by a "vector
  system"~$\?V=(\Loc,\Act,\Loc_\mEve,\Loc_\mAdam,\dd)$, then an
  initial credit $\vec v_0$ with $\|\vec
  v_0\|=(4|\Loc|\cdot\|\Act\|)^{2(\dd+2)^3}$ and a bound
  $B=2(4|\Loc|\cdot\|\Act\|)^{2(\dd+2)^3}+1$ suffice for this.
\end{theorem}

\Cref{11-th-bounding} also yields a way of handling "bounding games"
with "given initial credit".  
\TODO{Last missing bit regarding complexity upper bounds.}
  
% Local IspellDict: british



\section{The complexity of asymmetric monotone games}
\label{11-sec:complexity}
Unlike general "vector games" and "configuration reachability"
"asymmetric" ones, "coverability", "non-termination", and
"parity@parity vector game" "asymmetric vector games" are decidable.
We survey in this section the best known complexity bounds for every
case; see \cref{11-tbl:cmplx} at the end of the chapter for a summary.

\subsection{Upper Bounds}
\label{11-sec:up}
We begin with complexity upper bounds.  The main results are that
"parity@parity vector game" games with "existential initial credit"
can be solved in \coNP, but are in \kEXP[2] with "given initial
credit".  In both cases however, the complexity is pseudo-polynomial
if both the dimension~$\dd$ and the number of priorities~$d$ are
fixed, which is rather good news: one can hope that, in practice, both
the number of different resources (encoded by the counters) and the
complexity of the functional specification (encoded by the parity
condition) are tiny compared to the size of the system.

%\subsubsection{Existential Initial Credit}
\label{11-subsec:up-exist}

\paragraph{Counterless Strategies}
Consider a "strategy"~$\tau$ of \Adam in a "vector game".  In all the
games we consider, "uniform" "positional" strategies suffice over the
infinite "arena" $\natural(\?V)=(V,E,\VE,\VA)$: $\tau$ maps vertices
in~$V$ to edges in~$E$.  We call~$\tau$ ""counterless"" if, for all
locations $\loc\in\Loc_\mAdam$ and all vectors
$\vec v,\vec v'\in\+N^\dd$, $\tau(\loc(\vec v))=\tau(\loc(\vec v'))$.
A "counterless" strategy thus only considers the current location of
the play.
\begin{lemma}\label{11-counterless}
  Let $\?V=(\Loc,\Act,\Loc_\mEve,\Loc_\mAdam,\dd)$ be an "asymmetric
  vector system", $\loc_0\in\Loc$ be a location, and
  $\lcol{:}\,\Loc\to\{1,\dots,d\}$ be a location colouring.  If \Adam
  wins from $\loc_0(\vec v)$ for every initial credit~$\vec v$ in the
  "parity@parity vector game" game played over $\?V$ with~$\lcol$, then
  he has a single "counterless strategy" such that he wins from
  $\loc_0(\vec v)$ for every initial credit~$\vec v$.
\end{lemma}
\begin{proof}
  Let $\Act_\mAdam\eqdef\{(\loc\step{\vec
    u}\loc')\in\Act\mid\loc\in\Loc_\mAdam\}$ be the set of actions
  controlled by \Adam.  We assume without loss of generality that
  every location $\loc\in\Loc_\mAdam$ has either one or two outgoing
  actions, thus $|\Loc_\mAdam|\leq|\Act_\mAdam|\leq
  2|\Loc_\mAdam|$.  We proceed by induction over $|\Act_\mAdam|$.  For
  the base case, if $|\Act_\mAdam|=|\Loc_\mAdam|$ then every location
  controlled by \Adam has a single outgoing action, thus any
  strategy for \Adam is trivially "counterless".

  For the induction step, consider some location
  $\hat\loc\in\Loc_\mAdam$ with two outgoing actions
  $a_l\eqdef\hat\loc\step{\vec 0}\loc_l$ and
  $a_r\eqdef\hat\loc\step{\vec 0}\loc_r$.  Let $\?V_l$ and $\?V_r$ be
  the "vector systems" obtained from~$\?V$ by removing
  respectively~$a_r$ and~$a_l$ from~$\Act$, i.e., by using
  $\Act_l\eqdef\Act\setminus\{a_r\}$ and
  $\Act_r\eqdef\Act\setminus\{a_l\}$.  If $\Adam$ wins the
  "parity@parity vector game" game from $\loc(\vec v)$ for every
  initial credit~$\vec v$ in either $\?V_l$ or $\?V_r$, then by
  induction hypothesis he has a "counterless" winning strategy winning
  from $\loc(\vec v)$ for every initial credit~$\vec v$, and the same
  strategy is winning in~$\?V$ from $\loc(\vec v)$ for every initial
  credit~$\vec v$.

  In order to conclude the proof, we show that, if \Adam loses in
  $\?V_l$ from $\loc_0(\vec v_l)$ for some $\vec v_l\in\+N^\dd$ and in
  $\?V_r$ from $\loc_0(\vec v_r)$ for some $\vec v_r\in\+N^\dd$, then
  there exists $\vec v_0\in\+N^\dd$ such that \Eve wins from
  $\loc_0(\vec v_0)$ in~$\?V$.  Let $\sigma_l$ and $\sigma_r$ denote
  \Eve's winning strategies in the two games.  By a slight abuse of
  notations (justified by the fact that we are only interested in a
  few initial vertices), we see plays as sequences of actions and
  strategies as maps $\Act^\ast\to\Act$.\todoquestion{I hope this is not too messy}  Consider the set of
  plays consistent with~$\sigma_r$ starting from $\loc_0(\vec v_r)$.
  If none of those plays visits $\hat\loc$, then $\Eve$ wins in $\?V$
  from $\loc_0(\vec v_r)$ and we conclude.  Otherwise, there is some
  finite prefix~$\hat\pi$ of a play that
  visits~$\hat\loc(\hat{\vec v})$ for some vector
  $\hat{\vec v}=\vec v_r+\weight(\hat\pi)$.  We let
  $\vec v_0\eqdef\vec v_l+\hat{\vec v}$ and show that \Eve wins from
  $\loc_0(\vec v_0)$.

  \begin{scope}\knowledge{mode}{notion}
    We define now a strategy $\sigma$ for $\Eve$ over~$\?V$ that
    switches between applying~$\sigma_l$ and~$\sigma_r$ each time
    $a_r$ is used and switches back each time~$a_l$ is used.  More
    precisely, given a finite or infinite sequence~$\pi$ of actions,
    we decompose $\pi$ as $\pi_1 a_1 \pi_2 a_2 \pi_3\cdots$ where each
    segment $\pi_j\in(\Act\setminus\{a_l,a_r\})^\ast$ does not use
    either~$a_l$ nor~$a_r$ and each $a_j\in\{a_l,a_r\}$.  The
    associated ""mode"" $m(j)\in\{l,r\}$ of a segment~$\pi_j$
    is~$m(1)\eqdef l$ for the initial segment and otherwise
    $m(j)\eqdef l$ if $e_{j-1}=a_l$ and $m(j)\eqdef r$ otherwise.  The
    $l$-subsequence associated with $\pi$ is the sequence of segments
    $\pi(l)\eqdef\pi_{l_1}a_{l_2-1}\pi_{l_2}a_{l_3-1}\pi_{l_3}\cdots$
    with "mode"~$m(l_i)=l$, while the $r$-subsequence is the sequence
    $\pi(r)\eqdef\hat\pi a_{r_1-1}\pi_{r_1}a_{r_2-1}\pi_{r_2}\cdots$
    with "mode"~$m(r_i)=r$ prefixed by~$\hat\pi$.  Then we let
    $\sigma(\pi)\eqdef\sigma_{m}(\pi(m))$ where $m\in\{l,r\}$ is the
    "mode" of the last segment of~$\pi$.

    Consider an infinite play $\pi$ consistent with~$\sigma$ starting
    from~$\loc_0(\vec v_0)$.  Since $\vec v_0\geq\vec v_l$ and
    $\vec v_0\geq \vec v_r+\weight(\hat\pi)$, $\pi(l)$ and $\pi(r)$
    starting from~$\loc_0(\vec v_0)$ are consistent with
    "simulating"---in the sense of \cref{11-fact-mono}---$\sigma_l$
    from $\loc_0(\vec v_l)$ and $\sigma_r$ from $\loc_0(\vec v_r)$.
    Let $\pi'$ be a finite prefix of~$\pi$.  Then
    $\weight(\pi')=\weight(\pi'(l))+\weight(\pi'(r))$ where $\pi'(l)$
    is a prefix of~$\pi(l)$ and $\pi'(r)$ of~$\pi(r)$, thus
    $\weight(\pi'(l))\leq\vec v_l$ and
    $\weight(\pi'(r))\leq\vec v_r+\weight(\hat\pi)$, thus
    $\weight(\pi')\leq\vec v_0$: the play~$\pi$ avoids the "sink".
    Furthermore, the maximal priority seen infinitely often along
    $\pi(l)$ and $\pi(r)$ is even (note that one of~$\pi(l)$
    and~$\pi(r)$ might not be infinite), thus the maximal priority
    seen infinitely often along~$\pi$ is also even.  This shows
    that~$\sigma$ is winning for \Eve from $\loc_0(\vec v_0)$.\todoquestion{Is
    that clear?}
  \end{scope}
\end{proof}

We are going to exploit \cref{11-counterless} in \cref{11-exist-easy}
in order to prove a~\coNP\ upper bound for "asymmetric games" with
"existential initial credit": it suffices in order to decide those
games to guess a "counterless" winning strategy~$\tau$ for \Adam and
check that it is indeed winning by checking that \Eve loses the
one-player game arising from~$\tau$.  This last step requires an
algorithmic result of independent interest.

\paragraph{One-player Case}
Let $\?V=(\Loc,\Act,\dd)$ be a "vector addition system with states",
$\lcol{:}\,\Loc\to\{1,\dots,d\}$ a location colouring, and
$\loc_0\in\Loc$ an initial location.  Then \Eve wins the
"parity@parity vector game" one-player game from~$\loc_0(\vec v_0)$
for some initial credit~$\vec v_0$ if and only if there exists some
location such that
\begin{itemize}
\item $\loc$ is reachable from~$\loc_0$ in the directed graph
  underlying~$\?V$ and
\item there is a cycle~$\pi\in\Act^\ast$ from $\loc$ to itself such
  that $\weight(\pi)\geq 0$ and the maximal priority occurring
  along~$\pi$ is even.
\end{itemize}
Indeed, assume we can find such a location~$\loc$.  Let
$\hat\pi\in\Act^\ast$ be a path from~$\loc_0$ to~$\loc$ and $\vec
v_0(i)\eqdef\max\{\|\weight(\pi')\|\mid\pi'\text{ is a prefix of
}\hat\pi\pi\}$ for all $1\leq i\leq\dd$.  Then $\loc_0(\vec v_0)$ can
reach $\loc(\vec v_0+\weight(\hat\pi))$ in the "natural semantics"
of~$\?V$ by following~$\hat\pi$, and then $\loc(\vec v_0+\vec
W(\hat\pi)+n\weight(\pi))\geq \loc(\vec v_0+\weight(\hat\pi))$ after
$n$~repetitions of the cycle~$\pi$.  The infinite play arising from
this strategy has an even maximal priority.

Conversely, if \Eve wins, then there is a winning play
$\pi\in\Act^\omega$ from $\loc_0(\vec v_0)$ for some $\vec v_0$.
Recall that $(V,{\leq})$ is a "wqo", and we argue as in
\cref{11-fact-finmem} that there is indeed such a location~$\loc$.

\medskip
Therefore, solving one-player "parity vector games" boils down to
determining the existence of a cycle with non-negative effect and even
maximal priority.  We shall use linear programming techniques in order
to check the existence of such a cycle in polynomial
time~\cite{Kosaraju&Sullivan:1988}.

\medskip
\begin{scope}
\knowledge{non-negative}{notion}
\knowledge{multi-cycle}[multi-cycles]{notion}
\knowledge{suitable}{notion}
Let us start with a relaxed problem: we call a
""multi-cycle"" a non-empty finite set of cycles~$\Pi$ and let
$\weight(\Pi)\eqdef\sum_{\pi\in\Pi}\weight(\pi)$ be its weight; we write
$t\in\Pi$ if~$t\in\pi$ for some $\pi\in\Pi$.
Let $M\in 2^{\Act}$ be a set of `mandatory' subsets of actions and
$F\subseteq\Act$ a set of `forbidden' actions.  Then we say that
$\Pi$ is ""non-negative"" if $\weight(\Pi)\geq\vec 0$, and that it is
""suitable"" for~$(M,F)$ if for all $\Act'\in M$ there exists
$t\in\Act'$ such that $t\in\Pi$, and if for all $t\in F$,
$t\not\in\Pi$.  We use the same terminology for a single cycle~$\pi$.

\begin{lemma}\label{11-lem-zmulticycle}
  Let $\?V$ be a "vector addition system with states",
  $M\in 2^{\Act}$, and $F\subseteq\Act$.  We can check in polynomial
  time whether~$\?V$ contains a "non-negative" "multi-cycle"~$\Pi$
  "suitable" for~$(M,F)$.
\end{lemma}
\begin{proof}
  We reduce the problem to solving a linear program.  For a
  location~$\loc$, let
  $\mathrm{in}(\loc)\eqdef\{(\loc'\step{\vec u}\loc)\in\Act\mid
  \loc'\in\Loc\}$
  and
  $\mathrm{out}(\loc)\eqdef\{(\loc\step{\vec u}\loc')\in\Act\mid
  \loc'\in\Loc\}$ be its sets of incoming and outgoing actions.  The
  linear program has a variable $x_a$ for each action $a\in\Act$,
  which represents the number of times the action~$a$ occurs in
  the "multi-cycle".  It consists of the following contraints:
  \begin{align*}
    \forall\loc&\in\Loc,&\sum_{a\in\mathrm{in}(\loc)}x_a&=\sum_{a\in\mathrm{out}(\loc)}x_a\;,\tag{"multi-cycle"}\\
    \forall a&\in\Act,&x_a&\geq 0\;,\tag{non-negative uses}\\
    \forall i&\in\{1,\dots,\dd\},&\sum_{a\in\Act} x_a\cdot\weight(t)(i)&\geq
                                            0\;,\tag{"non-negative" weight}\\
    &&\sum_{a\in\Act}x_a&\geq 0\tag{non empty}\\
    \forall \Act'&\in M,&\sum_{a\in\Act'}x_a&\geq 0\;,\tag{every subset
                                               in~$M$ is used}\\
    \forall a&\in F,&x_a&= 0\;.\tag{no forbidden actions}
  \end{align*}
  As solving a linear program is in polynomial time~\cite{}\todoquestion{agree
  on a ref with \cref{chap:signal}?}, the result follows.
\end{proof}

Of course, what we are aiming for is finding a "non-negative"
\emph{cycle} "suitable" for $(M,F)$ rather than a "multi-cycle".
Let us define for this the relation $\loc\sim\loc'$ over~$\Loc$ if
$\loc=\loc'$ or if there exists a "non-negative" "multi-cycle"~$\Pi$
"suitable" for~$(M,F)$ such that~$\loc$ and~$\loc'$ belong to some
cycle~$\pi\in\Pi$.
\begin{claim}\label{11-cl-sim} The relation~$\sim$ is an equivalence
  relation.\end{claim}
\begin{proof}
  Symmetry and reflexivity are trivial, and if $\loc\sim\loc'$ and
  $\loc'\sim\loc''$ because~$\loc$ and~$\loc'$ appear in some cycle
  $\pi\in\Pi$ and $\loc'$ and~$\loc''$ in some cycle $\pi'\in\Pi'$ for
  two "non-negative" "multi-cycles"~$\Pi$ and~$\Pi'$ "suitable"
  for~$(M,F)$, then up to a circular shift $\pi$ and~$\pi'$ can be
  assumed to start and end with $\loc'$, and then
  $(\Pi\setminus\{\pi\})\cup(\Pi'\setminus\{\pi'\})\cup\{\pi\pi'\}$ is
  also a "non-negative" "multi-cycle" "suitable" for~$(M,F)$.
\end{proof}

Thus~$\sim$ defines a partition~$\Loc/{\sim}$ of~$\Loc$.
In order to find a "non-negative" cycle~$\pi$ "suitable" for~$(M,F)$,
we are going to compute the partition~$\Loc/{\sim}$ of~$\Loc$
according to~$\sim$.  If we obtain a partition with a single
equivalence class, we are done: there exists such a cycle.  Otherwise,
such a cycle if it exists must be included in one of the subsystems
$(P,\Act\cap(P\times\+Z^\dd\times P),\dd)$ induced by the equivalence
classes $P\in\Loc/{\sim}$.  This yields \cref{11-algo:zcycle}, which
assumes that we know how to compute the partition~$\Loc/{\sim}$.  Note
that the depth of the recursion in \cref{11-algo:zcycle} is bounded
by~$|\Loc|$ and that recursive calls operate over disjoint subsets
of~$\Loc$, thus assuming that we can compute the partition in
polynomial time, then \cref{11-algo:zcycle} also works in polynomial
time.

\begin{algorithm}
 \KwData{A "vector addition system with states"
   $\?V=(\Loc,\Act,\dd)$, $M\in 2^\Act$, $F\subseteq\Act$}

\If{$|\Loc|=1$}
  {\If{$\?V$ has a "non-negative" "multi-cycle" "suitable" for~$(M,F)$}
    {\Return{true}}}

$\Loc/{\sim} \leftarrow \mathrm{partition}(\?V,M,F)$ ;

\If{$|\Loc/{\sim}|=1$}{\Return{true}}

\ForEach{$P\in\Loc/{\sim}$}{\If{$\mathrm{cycle}((P,\Act\cap(P\times\+Z^\dd\times
    P),\dd),M,F)$}{\Return{true}}}

\Return{false}
\caption{$\text{cycle}(\?V,M,F)$}
\label{11-algo:zcycle}
\end{algorithm}

It remains to see how to compute the partition $\Loc/{\sim}$. Consider
for this the set of actions
$\Act'\eqdef\{a\mid\exists\Pi\text{ a "non-negative" "multi-cycle"
  "suitable" for $(M,F)$ with $a\in\Pi$}\}$ and
$\?V'=(\Loc',\Act',\dd)$ the subsystem induced by $\Act'$.
\begin{claim}\label{11-cl-part}
  There exists a path from~$\loc$ to~$\loc'$ in $\?V'$
  if and only if $\loc\sim\loc'$.
\end{claim}
\begin{proof}
  If $\loc\sim\loc'$, then either $\loc=\loc'$ and there is an empty
  path, or there exist~$\Pi$ and~$\pi\in\Pi$ such that $\loc$
  and~$\loc'$ belong to~$\pi$ and $\Pi$ is a "non-negative"
  "multi-cycle" "suitable" for $(M,F)$, thus every action of~$\pi$ is
  in~$\Act'$ and there is a path in~$\?V'$.  

  Conversely, if there is a path $\pi\in{\Act'}^\ast$ from~$\loc$
  to~$\loc'$, then $\loc\sim\loc'$ by induction on~$\pi$.  Indeed, if
  $|\pi|=0$ then $\loc=\loc'$.  For the induction step, $\pi=\pi' a$
  with $\pi'\in{\Act'}^\ast$ a path from $\loc$ to $\loc''$ and
  $a=(\loc''\step{\vec u}\loc')\in\Act'$ for some~$\vec u$.  By
  induction hypothesis, $\loc\sim\loc''$ and since $a\in\Act'$,
  $\loc''\sim\loc'$, thus $\loc\sim\loc'$ by transitivity shown
  in~\cref{11-cl-sim}. 
\end{proof}

By \cref{11-cl-part}, the equivalence classes of~$\sim$ are the
strongly connected components of~$\?V'$.  This yields the following
polynomial time algorithm for computing~$\Loc/{\sim}$.

\begin{algorithm}
 \KwData{A "vector addition system with states"
   $\?V=(\Loc,\Act,\dd)$, $M\in 2^\Act$, $F\subseteq\Act$}

$\Act'\leftarrow\emptyset$;

\ForEach{$a\in\Act$}{\If{$\?V$ has a "non-negative" "multi-cycle"
    "suitable"
    for~$(M\cup\{\{a\}\},F)$}{$\Act'\leftarrow\Act'\cup\{a\}$}}

$\?V'\leftarrow \text{subsystem induced by~$\Act'$}$ ;

\Return{$\mathrm{SCC}(\?V')$}
\caption{$\text{partition}(\?V,M,F)$}
\label{11-algo:part}
\end{algorithm}

Together, \cref{11-lem-zmulticycle}
and \cref{11-algo:part,12-algo-zcycle} yield the following.

\begin{lemma}\label{11-lem-zcycle}
  Let $\?V$ be a "vector addition system with states",
  $M\in 2^{\Act}$, and $F\subseteq\Act$.  We can check in polynomial
  time whether~$\?V$ contains a "non-negative" cycle~$\pi$
  "suitable" for~$(M,F)$.
\end{lemma}

Finally, we obtain the desired polynomial time upper bound for
"parity@parity vector games" in "vector addition systems with states".
\begin{theorem}\label{11-thm-zcycle}
  Whether \Eve wins a one-player "parity vector game" with
  "existential initial credit" is in~\P.
\end{theorem}
\begin{proof}
  Let $\?V=(\Loc,\Act,\dd)$ be a "vector addition system with states",
  $\lcol{:}\,\Loc\to\{1,\dots,$ $d\}$ a location colouring, and
  $\loc_0\in\Loc$ an initial location.  We start by trimming~$\?V$ to
  only keep the locations reachable from~$\loc_0$ in the underlying
  directed graph.  Then, for every even priority $p\in\{1,\dots,d\}$,
  we use \cref{11-lem-zcycle} to check for the existence of a
  "non-negative" cycle with maximal priority~$p$: it suffices for this
  to set $M\eqdef\{\lcol^{-1}(p)\}$ and
  $F\eqdef\lcol^{-1}(\{p+1,\dots,d\})$.
\end{proof}
\end{scope}

\paragraph{Upper Bounds}
We are now equipped to prove our upper bounds.  We begin with a nearly
trivial case.  In a "coverability" "asymmetric vector game" with
"existential initial credit", the counters play no role at all: \Eve
has a winning strategy for some initial credit in the "vector game" if
and only if she has one to reach the target location~$\loc_f$ in the
finite game played over~$\Loc$ and edges~$(\loc,\loc')$ whenever
$\loc\step{\vec u}\loc'\in\Act$ for some~$\vec u$.  This entails that
"coverability" "asymmetric vector games" are quite easy to solve.

\begin{theorem}\label{11-cov-exist-P}
  "Coverability" "asymmetric" "vector games" with "existential initial
  credit" are \P-complete.
\end{theorem}

Regarding "non-termination" and "parity@parity vector game", we
exploit \cref{11-counterless,12-thm-zcycle}.

\begin{theorem}\label{11-exist-easy}
  "Non-termination" and "parity@parity vector game" "asymmetric"
  "vector games" with "existential initial credit" are in~\coNP.
\end{theorem}
\begin{proof}
  By \cref{11-nonterm2parity}, it suffices to prove the statement for
  "parity@parity vector games" games.  By \cref{11-counterless},
  if \Adam wins the game, we can guess a "counterless" winning
  strategy~$\tau$ telling which action to choose for every location.
  This strategy yields a one-player game, and by \cref{11-thm-zcycle}
  we can check in polynomial time that~$\tau$ was indeed winning
  for~\Adam.
\end{proof}

Finally, in fixed dimension and with a fixed number of priorities, we
can simply apply the results of \cref{11-bounding}.
\begin{corollary}\label{11-exist-pseudop}
  "Parity@parity vector game" "asymmetric" "vector games" with
  "existential initial credit" are in pseudo-polynomial time if the
  dimension and the number of priorities are fixed.
\end{corollary}
\begin{proof}
  Consider an "asymmetric vector system"
  $\?V=(\Loc,\Act,\Loc_\mEve,\Loc_\mAdam,\dd)$ and a location
  colouring $\lcol{:}\,\Loc\to\{1,\dots,2d\}$.
  By \cref{11-parity2bounding}, the "parity vector game" with
  "existential initial credit" over~$\?V$ problem reduces to a
  "bounding game" with "existential initial credit" over a "vector
  system"~$\?V'=(\Loc',\Act',\Loc'_\mEve,\Loc'_\mAdam,\dd+d)$ where
  $\Loc'\in O(|\Loc|)$ and $\|\Act'\|=\|\Act\|$.
  By \cref{11-th-bounding}, it suffices to consider the case of a
  "non-termination" game with "existential initial credit" played over
  the "bounded semantics" $\bounded(\?V')$ where $B$ is in
  $(|\Loc'|\cdot\|\Act'\|)^{O(\dd+d)^3}$.  Such a game can be solved in
  linear time in the size of the bounded arena using attractor
  techniques, thus in $O(|\Loc|\cdot B)^{\dd+d}$, which is in
  $(|\Loc|\cdot\|\Act\|)^{O(\dd+d)^4}$ in terms of the original instance.
\end{proof}

\subsubsection{Given Initial Credit}
\label{11-subsec:up-given}
\TODO{\Cref{11-subsec:up-given}}

\begin{theorem}\label{11-avag-easy}
  "Coverability", "non-termination", and "parity@parity vector game"
  "asymmetric" "vector games" with "given initial credit" are in
  \kEXP[2].  If the dimension is fixed, they are in \EXP, and if the
  number of priorities is also fixed, they are in pseudo-polynomial
  time.
\end{theorem}

% Local IspellDict: british

\subsubsection{Existential Initial Credit}
\label{11-sec:up-exist}

\paragraph{Counterless Strategies}
Consider a "strategy"~$\tau$ of \Adam\ in a "vector game".  In all the
games we consider, "uniform" "positional" strategies suffice over the
infinite "arena" $\natural(\?V)=(V,E,\VE,\VA)$: $\tau$ maps vertices
in~$V$ to edges in~$E$.  We call~$\tau$ ""counterless"" if, for all
locations $\loc\in\Loc_\mAdam$ and all vectors
$\vec v,\vec v'\in\+N^\dd$, $\tau(\loc(\vec v))=\tau(\loc(\vec v'))$.
A "counterless" strategy thus only considers the current location of
the play.
\begin{lemma}[Counterless strategies]
\label{11-lem:counterless}
  Let $\?V=(\Loc,\Act,\Loc_\mEve,\Loc_\mAdam,\dd)$ be an "asymmetric
  vector system", $\loc_0\in\Loc$ be a location, and
  $\lcol{:}\,\Loc\to\{1,\dots,d\}$ be a location colouring.  If \Adam\ 
  wins from $\loc_0(\vec v)$ for every initial credit~$\vec v$ in the
  "parity@parity vector game" game played over $\?V$ with~$\lcol$, then
  he has a single "counterless strategy" such that he wins from
  $\loc_0(\vec v)$ for every initial credit~$\vec v$.
\end{lemma}
\begin{proof}
  Let $\Act_\mAdam\eqdef\{(\loc\step{\vec
    u}\loc')\in\Act\mid\loc\in\Loc_\mAdam\}$ be the set of actions
  controlled by \Adam.  We assume without loss of generality that
  every location $\loc\in\Loc_\mAdam$ has either one or two outgoing
  actions, thus $|\Loc_\mAdam|\leq|\Act_\mAdam|\leq
  2|\Loc_\mAdam|$.  We proceed by induction over $|\Act_\mAdam|$.  For
  the base case, if $|\Act_\mAdam|=|\Loc_\mAdam|$ then every location
  controlled by \Adam\ has a single outgoing action, thus any
  strategy for \Adam\ is trivially "counterless".

  For the induction step, consider some location
  $\hat\loc\in\Loc_\mAdam$ with two outgoing actions
  $a_l\eqdef\hat\loc\step{\vec 0}\loc_l$ and
  $a_r\eqdef\hat\loc\step{\vec 0}\loc_r$.  Let $\?V_l$ and $\?V_r$ be
  the "vector systems" obtained from~$\?V$ by removing
  respectively~$a_r$ and~$a_l$ from~$\Act$, i.e., by using
  $\Act_l\eqdef\Act\setminus\{a_r\}$ and
  $\Act_r\eqdef\Act\setminus\{a_l\}$.  If $\Adam$ wins the
  "parity@parity vector game" game from $\loc(\vec v)$ for every
  initial credit~$\vec v$ in either $\?V_l$ or $\?V_r$, then by
  induction hypothesis he has a "counterless" winning strategy winning
  from $\loc(\vec v)$ for every initial credit~$\vec v$, and the same
  strategy is winning in~$\?V$ from $\loc(\vec v)$ for every initial
  credit~$\vec v$.

  In order to conclude the proof, we show that, if \Adam\ loses in
  $\?V_l$ from $\loc_0(\vec v_l)$ for some $\vec v_l\in\+N^\dd$ and in
  $\?V_r$ from $\loc_0(\vec v_r)$ for some $\vec v_r\in\+N^\dd$, then
  there exists $\vec v_0\in\+N^\dd$ such that \Eve\ wins from
  $\loc_0(\vec v_0)$ in~$\?V$.  Let $\sigma_l$ and $\sigma_r$ denote
  \Eve's winning strategies in the two games.  By a slight abuse of
  notations (justified by the fact that we are only interested in a
  few initial vertices), we see plays as sequences of actions and
  strategies as maps $\Act^\ast\to\Act$.\todoquestion{I hope this is not too messy}  Consider the set of
  plays consistent with~$\sigma_r$ starting from $\loc_0(\vec v_r)$.
  If none of those plays visits $\hat\loc$, then $\Eve$ wins in $\?V$
  from $\loc_0(\vec v_r)$ and we conclude.  Otherwise, there is some
  finite prefix~$\hat\pi$ of a play that
  visits~$\hat\loc(\hat{\vec v})$ for some vector
  $\hat{\vec v}=\vec v_r+\weight(\hat\pi)$.  We let
  $\vec v_0\eqdef\vec v_l+\hat{\vec v}$ and show that \Eve\ wins from
  $\loc_0(\vec v_0)$.

  \begin{scope}\knowledge{mode}{notion}
    We define now a strategy $\sigma$ for $\Eve$ over~$\?V$ that
    switches between applying~$\sigma_l$ and~$\sigma_r$ each time
    $a_r$ is used and switches back each time~$a_l$ is used.  More
    precisely, given a finite or infinite sequence~$\pi$ of actions,
    we decompose $\pi$ as $\pi_1 a_1 \pi_2 a_2 \pi_3\cdots$ where each
    segment $\pi_j\in(\Act\setminus\{a_l,a_r\})^\ast$ does not use
    either~$a_l$ nor~$a_r$ and each $a_j\in\{a_l,a_r\}$.  The
    associated ""mode"" $m(j)\in\{l,r\}$ of a segment~$\pi_j$
    is~$m(1)\eqdef l$ for the initial segment and otherwise
    $m(j)\eqdef l$ if $e_{j-1}=a_l$ and $m(j)\eqdef r$ otherwise.  The
    $l$-subsequence associated with $\pi$ is the sequence of segments
    $\pi(l)\eqdef\pi_{l_1}a_{l_2-1}\pi_{l_2}a_{l_3-1}\pi_{l_3}\cdots$
    with "mode"~$m(l_i)=l$, while the $r$-subsequence is the sequence
    $\pi(r)\eqdef\hat\pi a_{r_1-1}\pi_{r_1}a_{r_2-1}\pi_{r_2}\cdots$
    with "mode"~$m(r_i)=r$ prefixed by~$\hat\pi$.  Then we let
    $\sigma(\pi)\eqdef\sigma_{m}(\pi(m))$ where $m\in\{l,r\}$ is the
    "mode" of the last segment of~$\pi$.

    Consider an infinite play $\pi$ consistent with~$\sigma$ starting
    from~$\loc_0(\vec v_0)$.  Since $\vec v_0\geq\vec v_l$ and
    $\vec v_0\geq \vec v_r+\weight(\hat\pi)$, $\pi(l)$ and $\pi(r)$
    starting from~$\loc_0(\vec v_0)$ are consistent with
    "simulating"---in the sense of \cref{11-lem:mono}---$\sigma_l$
    from $\loc_0(\vec v_l)$ and $\sigma_r$ from $\loc_0(\vec v_r)$.
    Let $\pi'$ be a finite prefix of~$\pi$.  Then
    $\weight(\pi')=\weight(\pi'(l))+\weight(\pi'(r))$ where $\pi'(l)$
    is a prefix of~$\pi(l)$ and $\pi'(r)$ of~$\pi(r)$, thus
    $\weight(\pi'(l))\leq\vec v_l$ and
    $\weight(\pi'(r))\leq\vec v_r+\weight(\hat\pi)$, thus
    $\weight(\pi')\leq\vec v_0$: the play~$\pi$ avoids the "sink".
    Furthermore, the maximal priority seen infinitely often along
    $\pi(l)$ and $\pi(r)$ is even (note that one of~$\pi(l)$
    and~$\pi(r)$ might not be infinite), thus the maximal priority
    seen infinitely often along~$\pi$ is also even.  This shows
    that~$\sigma$ is winning for \Eve\ from $\loc_0(\vec v_0)$.\todoquestion{Is
    that clear?}
  \end{scope}
\end{proof}

We are going to exploit \cref{11-lem:counterless}
in \cref{11-th:exist-easy} in order to prove a~\coNP\ upper bound for
"asymmetric games" with "existential initial credit": it suffices in
order to decide those games to guess a "counterless" winning
strategy~$\tau$ for \Adam\ and check that it is indeed winning by
checking that \Eve\ loses the one-player game arising from~$\tau$.
This last step requires an algorithmic result of independent interest.

\paragraph{One-player Case}
Let $\?V=(\Loc,\Act,\dd)$ be a "vector addition system with states",
$\lcol{:}\,\Loc\to\{1,\dots,d\}$ a location colouring, and
$\loc_0\in\Loc$ an initial location.  Then \Eve\ wins the
"parity@parity vector game" one-player game from~$\loc_0(\vec v_0)$
for some initial credit~$\vec v_0$ if and only if there exists some
location such that
\begin{itemize}
\item $\loc$ is reachable from~$\loc_0$ in the directed graph
  underlying~$\?V$ and
\item there is a cycle~$\pi\in\Act^\ast$ from $\loc$ to itself such
  that $\weight(\pi)\geq 0$ and the maximal priority occurring
  along~$\pi$ is even.
\end{itemize}
Indeed, assume we can find such a location~$\loc$.  Let
$\hat\pi\in\Act^\ast$ be a path from~$\loc_0$ to~$\loc$ and $\vec
v_0(i)\eqdef\max\{\|\weight(\pi')\|\mid\pi'\text{ is a prefix of
}\hat\pi\pi\}$ for all $1\leq i\leq\dd$.  Then $\loc_0(\vec v_0)$ can
reach $\loc(\vec v_0+\weight(\hat\pi))$ in the "natural semantics"
of~$\?V$ by following~$\hat\pi$, and then $\loc(\vec v_0+\vec
W(\hat\pi)+n\weight(\pi))\geq \loc(\vec v_0+\weight(\hat\pi))$ after
$n$~repetitions of the cycle~$\pi$.  The infinite play arising from
this strategy has an even maximal priority.

Conversely, if \Eve\ wins, then there is a winning play
$\pi\in\Act^\omega$ from $\loc_0(\vec v_0)$ for some $\vec v_0$.
Recall that $(V,{\leq})$ is a "wqo", and we argue as in
\cref{11-lem:finmem} that there is indeed such a location~$\loc$.

\medskip
Therefore, solving one-player "parity vector games" boils down to
determining the existence of a cycle with non-negative effect and even
maximal priority.  We shall use linear programming techniques in order
to check the existence of such a cycle in polynomial
time~\cite{Kosaraju&Sullivan:1988}.

\medskip
\begin{scope}
\knowledge{non-negative}{notion}
\knowledge{multi-cycle}[multi-cycles]{notion}
\knowledge{suitable}{notion}
Let us start with a relaxed problem: we call a
""multi-cycle"" a non-empty finite set of cycles~$\Pi$ and let
$\weight(\Pi)\eqdef\sum_{\pi\in\Pi}\weight(\pi)$ be its weight; we write
$t\in\Pi$ if~$t\in\pi$ for some $\pi\in\Pi$.
Let $M\in 2^{\Act}$ be a set of `mandatory' subsets of actions and
$F\subseteq\Act$ a set of `forbidden' actions.  Then we say that
$\Pi$ is ""non-negative"" if $\weight(\Pi)\geq\vec 0$, and that it is
""suitable"" for~$(M,F)$ if for all $\Act'\in M$ there exists
$t\in\Act'$ such that $t\in\Pi$, and if for all $t\in F$,
$t\not\in\Pi$.  We use the same terminology for a single cycle~$\pi$.

\begin{lemma}[Linear programs for suitable non-negative multi-cycles]
\label{11-lem:zmulticycle}
  Let $\?V$ be a "vector addition system with states",
  $M\in 2^{\Act}$, and $F\subseteq\Act$.  We can check in polynomial
  time whether~$\?V$ contains a "non-negative" "multi-cycle"~$\Pi$
  "suitable" for~$(M,F)$.
\end{lemma}
\begin{proof}
  We reduce the problem to solving a linear program.  For a
  location~$\loc$, let
  $\mathrm{in}(\loc)\eqdef\{(\loc'\step{\vec u}\loc)\in\Act\mid
  \loc'\in\Loc\}$
  and
  $\mathrm{out}(\loc)\eqdef\{(\loc\step{\vec u}\loc')\in\Act\mid
  \loc'\in\Loc\}$ be its sets of incoming and outgoing actions.  The
  linear program has a variable $x_a$ for each action $a\in\Act$,
  which represents the number of times the action~$a$ occurs in
  the "multi-cycle".  It consists of the following contraints:
  \begin{align*}
    \forall\loc&\in\Loc,&\sum_{a\in\mathrm{in}(\loc)}x_a&=\sum_{a\in\mathrm{out}(\loc)}x_a\;,\tag{"multi-cycle"}\\
    \forall a&\in\Act,&x_a&\geq 0\;,\tag{non-negative uses}\\
    \forall i&\in\{1,\dots,\dd\},&\sum_{a\in\Act} x_a\cdot\weight(t)(i)&\geq
                                            0\;,\tag{"non-negative" weight}\\
    &&\sum_{a\in\Act}x_a&\geq 0\tag{non empty}\\
    \forall \Act'&\in M,&\sum_{a\in\Act'}x_a&\geq 0\;,\tag{every subset
                                               in~$M$ is used}\\
    \forall a&\in F,&x_a&= 0\;.\tag{no forbidden actions}
  \end{align*}
  As solving a linear program is in polynomial time~\cite{}\todoquestion{agree
  on a ref with \cref{8-chap:signal}?}, the result follows.
\end{proof}

Of course, what we are aiming for is finding a "non-negative"
\emph{cycle} "suitable" for $(M,F)$ rather than a "multi-cycle".
Let us define for this the relation $\loc\sim\loc'$ over~$\Loc$ if
$\loc=\loc'$ or if there exists a "non-negative" "multi-cycle"~$\Pi$
"suitable" for~$(M,F)$ such that~$\loc$ and~$\loc'$ belong to some
cycle~$\pi\in\Pi$.
\begin{claim}\label{11-cl:sim} The relation~$\sim$ is an equivalence
  relation.\end{claim}
\begin{proof}
  Symmetry and reflexivity are trivial, and if $\loc\sim\loc'$ and
  $\loc'\sim\loc''$ because~$\loc$ and~$\loc'$ appear in some cycle
  $\pi\in\Pi$ and $\loc'$ and~$\loc''$ in some cycle $\pi'\in\Pi'$ for
  two "non-negative" "multi-cycles"~$\Pi$ and~$\Pi'$ "suitable"
  for~$(M,F)$, then up to a circular shift $\pi$ and~$\pi'$ can be
  assumed to start and end with $\loc'$, and then
  $(\Pi\setminus\{\pi\})\cup(\Pi'\setminus\{\pi'\})\cup\{\pi\pi'\}$ is
  also a "non-negative" "multi-cycle" "suitable" for~$(M,F)$.
\end{proof}

Thus~$\sim$ defines a partition~$\Loc/{\sim}$ of~$\Loc$.
In order to find a "non-negative" cycle~$\pi$ "suitable" for~$(M,F)$,
we are going to compute the partition~$\Loc/{\sim}$ of~$\Loc$
according to~$\sim$.  If we obtain a partition with a single
equivalence class, we are done: there exists such a cycle.  Otherwise,
such a cycle if it exists must be included in one of the subsystems
$(P,\Act\cap(P\times\+Z^\dd\times P),\dd)$ induced by the equivalence
classes $P\in\Loc/{\sim}$.  This yields \cref{11-algo:zcycle}, which
assumes that we know how to compute the partition~$\Loc/{\sim}$.  Note
that the depth of the recursion in \cref{11-algo:zcycle} is bounded
by~$|\Loc|$ and that recursive calls operate over disjoint subsets
of~$\Loc$, thus assuming that we can compute the partition in
polynomial time, then \cref{11-algo:zcycle} also works in polynomial
time.

\begin{algorithm}
 \KwData{A "vector addition system with states"
   $\?V=(\Loc,\Act,\dd)$, $M\in 2^\Act$, $F\subseteq\Act$}

\If{$|\Loc|=1$}
  {\If{$\?V$ has a "non-negative" "multi-cycle" "suitable" for~$(M,F)$}
    {\Return{true}}}

$\Loc/{\sim} \leftarrow \mathrm{partition}(\?V,M,F)$ ;

\If{$|\Loc/{\sim}|=1$}{\Return{true}}

\ForEach{$P\in\Loc/{\sim}$}{\If{$\mathrm{cycle}((P,\Act\cap(P\times\+Z^\dd\times
    P),\dd),M,F)$}{\Return{true}}}

\Return{false}
\caption{$\text{cycle}(\?V,M,F)$}
\label{11-algo:zcycle}
\end{algorithm}

It remains to see how to compute the partition $\Loc/{\sim}$. Consider
for this the set of actions
$\Act'\eqdef\{a\mid\exists\Pi\text{ a "non-negative" "multi-cycle"
  "suitable" for $(M,F)$ with $a\in\Pi$}\}$ and
$\?V'=(\Loc',\Act',\dd)$ the subsystem induced by $\Act'$.
\begin{claim}
\label{11-cl:part}
  There exists a path from~$\loc$ to~$\loc'$ in $\?V'$
  if and only if $\loc\sim\loc'$.
\end{claim}
\begin{proof}
  If $\loc\sim\loc'$, then either $\loc=\loc'$ and there is an empty
  path, or there exist~$\Pi$ and~$\pi\in\Pi$ such that $\loc$
  and~$\loc'$ belong to~$\pi$ and $\Pi$ is a "non-negative"
  "multi-cycle" "suitable" for $(M,F)$, thus every action of~$\pi$ is
  in~$\Act'$ and there is a path in~$\?V'$.  

  Conversely, if there is a path $\pi\in{\Act'}^\ast$ from~$\loc$
  to~$\loc'$, then $\loc\sim\loc'$ by induction on~$\pi$.  Indeed, if
  $|\pi|=0$ then $\loc=\loc'$.  For the induction step, $\pi=\pi' a$
  with $\pi'\in{\Act'}^\ast$ a path from $\loc$ to $\loc''$ and
  $a=(\loc''\step{\vec u}\loc')\in\Act'$ for some~$\vec u$.  By
  induction hypothesis, $\loc\sim\loc''$ and since $a\in\Act'$,
  $\loc''\sim\loc'$, thus $\loc\sim\loc'$ by transitivity shown
  in~\cref{11-cl:sim}. 
\end{proof}

By \cref{11-cl:part}, the equivalence classes of~$\sim$ are the
strongly connected components of~$\?V'$.  This yields the following
polynomial time algorithm for computing~$\Loc/{\sim}$.

\begin{algorithm}
 \KwData{A "vector addition system with states"
   $\?V=(\Loc,\Act,\dd)$, $M\in 2^\Act$, $F\subseteq\Act$}

$\Act'\leftarrow\emptyset$;

\ForEach{$a\in\Act$}{\If{$\?V$ has a "non-negative" "multi-cycle"
    "suitable"
    for~$(M\cup\{\{a\}\},F)$}{$\Act'\leftarrow\Act'\cup\{a\}$}}

$\?V'\leftarrow \text{subsystem induced by~$\Act'$}$ ;

\Return{$\mathrm{SCC}(\?V')$}
\caption{$\text{partition}(\?V,M,F)$}
\label{11-algo:part}
\end{algorithm}

Together, \cref{11-lem:zmulticycle}
and \cref{11-algo:part,11-algo:zcycle} yield the following.

\begin{lemma}[Polynomial-time detection of suitable non-negative cycles]
\label{11-lem:zcycle}
  Let $\?V$ be a "vector addition system with states",
  $M\in 2^{\Act}$, and $F\subseteq\Act$.  We can check in polynomial
  time whether~$\?V$ contains a "non-negative" cycle~$\pi$
  "suitable" for~$(M,F)$.
\end{lemma}

Finally, we obtain the desired polynomial time upper bound for
"parity@parity vector games" in "vector addition systems with states".
\begin{theorem}[Existential one-player parity vector games are in~\P]
\label{11-thm:zcycle}
  Whether \Eve\ wins a one-player "parity vector game" with
  "existential initial credit" is in~\P.
\end{theorem}
\begin{proof}
  Let $\?V=(\Loc,\Act,\dd)$ be a "vector addition system with states",
  $\lcol{:}\,\Loc\to\{1,\dots,$ $d\}$ a location colouring, and
  $\loc_0\in\Loc$ an initial location.  We start by trimming~$\?V$ to
  only keep the locations reachable from~$\loc_0$ in the underlying
  directed graph.  Then, for every even priority $p\in\{1,\dots,d\}$,
  we use \cref{11-lem:zcycle} to check for the existence of a
  "non-negative" cycle with maximal priority~$p$: it suffices for this
  to set $M\eqdef\{\lcol^{-1}(p)\}$ and
  $F\eqdef\lcol^{-1}(\{p+1,\dots,d\})$.
\end{proof}
\end{scope}

\paragraph{Upper Bounds}
We are now equipped to prove our upper bounds.  We begin with a nearly
trivial case.  In a "coverability" "asymmetric vector game" with
"existential initial credit", the counters play no role at all: \Eve\ 
has a winning strategy for some initial credit in the "vector game" if
and only if she has one to reach the target location~$\loc_f$ in the
finite game played over~$\Loc$ and edges~$(\loc,\loc')$ whenever
$\loc\step{\vec u}\loc'\in\Act$ for some~$\vec u$.  This entails that
"coverability" "asymmetric vector games" are quite easy to solve.

\begin{theorem}[Existential coverability asymmetric vector games are in~\P]
\label{11-th:cov-exist-P}
  "Coverability" "asymmetric" "vector games" with "existential initial
  credit" are \P-complete.
\end{theorem}

Regarding "non-termination" and "parity@parity vector game", we
exploit \cref{11-lem:counterless,11-thm:zcycle}.

\begin{theorem}[Existential parity asymmetric vector games are in~\coNP]
\label{11-th:exist-easy}
  "Non-termination" and "parity@parity vector game" "asymmetric"
  "vector games" with "existential initial credit" are in~\coNP.
\end{theorem}
\begin{proof}
  By \cref{11-rk:nonterm2parity}, it suffices to prove the statement for
  "parity@parity vector games" games.  By \cref{11-lem:counterless},
  if \Adam\ wins the game, we can guess a "counterless" winning
  strategy~$\tau$ telling which action to choose for every location.
  This strategy yields a one-player game, and by \cref{11-thm:zcycle}
  we can check in polynomial time that~$\tau$ was indeed winning
  for~\Adam.
\end{proof}

Finally, in fixed dimension and with a fixed number of priorities, we
can simply apply the results of \cref{11-sec:bounding}.
\begin{corollary}[Existential fixed-dimensional parity asymmetric vector games are pseudo-polynomial]
\label{11-cor:exist-pseudop}
  "Parity@parity vector game" "asymmetric" "vector games" with
  "existential initial credit" are in pseudo-polynomial time if the
  dimension and the number of priorities are fixed.
\end{corollary}
\begin{proof}
  Consider an "asymmetric vector system"
  $\?V=(\Loc,\Act,\Loc_\mEve,\Loc_\mAdam,\dd)$ and a location
  colouring $\lcol{:}\,\Loc\to\{1,\dots,2d\}$.
  By \cref{11-lem:parity2bounding}, the "parity vector game" with
  "existential initial credit" over~$\?V$ problem reduces to a
  "bounding game" with "existential initial credit" over a "vector
  system"~$\?V'=(\Loc',\Act',\Loc'_\mEve,\Loc'_\mAdam,\dd+d)$ where
  $\Loc'\in O(|\Loc|)$ and $\|\Act'\|=\|\Act\|$.
  By \cref{11-th:bounding}, it suffices to consider the case of a
  "non-termination" game with "existential initial credit" played over
  the "bounded semantics" $\bounded(\?V')$ where $B$ is in
  $(|\Loc'|\cdot\|\Act'\|)^{O(\dd+d)^3}$.  Such a game can be solved in
  linear time in the size of the bounded arena using attractor
  techniques, thus in $O(|\Loc|\cdot B)^{\dd+d}$, which is in
  $(|\Loc|\cdot\|\Act\|)^{O(\dd+d)^4}$ in terms of the original instance.
\end{proof}

\subsubsection{Given Initial Credit}
\label{11-sec:up-given}
\TODO{\Cref{11-sec:up-given}}

\begin{theorem}[Upper bounds for asymmetric vector games]
\label{11-th:avag-easy}
  "Coverability", "non-termination", and "parity@parity vector game"
  "asymmetric" "vector games" with "given initial credit" are in
  \kEXP[2].  If the dimension is fixed, they are in \EXP, and if the
  number of priorities is also fixed, they are in pseudo-polynomial
  time.
\end{theorem}

% Local IspellDict: british

\subsection{Lower Bounds}
\label{11-sec:low}
Let us turn our attention to complexity lower bounds for "monotonic"
"asymmetric vector games".  It turns out that most of the upper bounds
shown in \cref{11-sec:up} are tight.
%\subsubsection{Existential Initial Credit}
In the "existential initial credit" variant of our games, we have the
following lower bound matching \cref{11-exist-easy}, already with a
unary encoding.

\begin{theorem}\label{11-exist-hard}
  "Non-termination", and "parity@parity vector game"
  "asymmetric" "vector games" with "existential initial credit" are
  \coNP-hard.% in any dimension~$\dd\geq 2$.
\end{theorem}
\begin{proof}
  By \cref{11-nonterm2parity}, it suffices to show hardness for
  "non-termination games".  We reduce from the \lang{3SAT} problem:
  given a formula $\varphi=\bigwedge_{1\leq i\leq m}C_i$ where each
  clause $C_i$ is a disjonction of the form
  $\litt_{i,1}\vee\litt_{i,2}\vee\litt_{i,3}$ of literals taken from
  $X=\{x_1,\neg x_1,x_2,$ $\neg x_2,\dots,x_k,\neg x_k\}$, we construct
  an "asymmetric" "vector system" $\?V$ where Eve wins the
  "non-termination game" with "existential initial credit" if and only
  if~$\varphi$ is not satisfiable; since the game is determined, we
  actually show that Adam wins the game if and only if~$\varphi$ is
  satisfiable.

  Our "vector system" has dimension~$2k$, and for a literal
  $\litt\in X$, we define the vector
  \begin{equation*}
    \vec u_\litt\eqdef\begin{cases}
      \vec e_{2n-1}-\vec e_{2n}&\text{if }\litt=x_n\;,\\
      \vec e_{2n}-\vec e_{2n-1}&\text{if }\litt=\neg x_n\;.
    \end{cases}
  \end{equation*}
  We define $\?V\eqdef(\Loc,\Act,\Loc_\mEve,\Loc_\mAdam,2k)$ where
  \begin{align*}
    \Loc_\mEve&\eqdef\{\varphi\}\cup\{\litt_{i,j}\mid 1\leq i\leq m,1\leq j\leq
                3\}\;,\\
    \Loc_\mAdam&\eqdef\{C_i\mid 1\leq i\leq m\}\;,\\
    \Act&\eqdef\{\varphi\step{\vec 0}C_i\mid 1\leq i\leq m\}\cup\{C_i\step{\vec 0}\litt_{i,j},\;\;\litt_{i,j}\xrightarrow{\vec u_{\litt_{i,j}}}\varphi\mid 1\leq i\leq m,1\leq j\leq 3\}\;.
  \end{align*}
  \begin{scope}
    We use~$\varphi$ as our initial location.
    \knowledge{literal assignment}{notion}
    \knowledge{conflicting}{notion}
    %
    Let us call a map $v{:}\,X\to\{0,1\}$ a ""literal assignment""; we
    call it ""conflicting"" if there exists $1\leq n\leq k$ such that
    $v(x_n)=v(\neg x_n)$.

    Assume that~$\varphi$ is satisfiable.  Then there exists a
    non-"conflicting" "literal assignment"~$v$ that satisfies all the
    clauses: for each $1\leq i\leq m$, there exists $1\leq j\leq 3$
    such that $v(\litt_{i,j})=1$; this yields a "counterless" strategy
    for Adam, which selects $(C_i,\litt_{i,j})$ for each
    $1\leq i\leq m$.  Consider any infinite "play" consistent with
    this strategy.  This "play" only visits literals $\litt$ where
    $v(\litt)=1$.  There exists a literal $\litt\in X$ that is visited
    infinitely often along the "play", say $\litt=x_n$.  Because~$v$ is
    non-"conflicting", $v(\neg x_n)=0$, thus the location $\neg x_n$
    is never visited.  Thus the play uses the action
    $\litt\step{\vec e_{2n-1}-\vec e_{2n}}\varphi$ infinitely often,
    and never uses any action with a positive effect on
    component~$2n$.  Hence the play is losing from any initial credit.

    Conversely, assume that~$\varphi$ is not satisfiable.  By
    contradiction, assume that Adam wins the game for all initial
    credits.  By \cref{11-counterless}, he has a "counterless" winning
    strategy~$\tau$ that selects a literal in every clause.  Consider
    a "literal assignment" that maps each one of the selected literals
    to~$1$ and the remaining ones in a non-conflicting manner.  By
    definition, this "literal assignment" satisfies all the clauses,
    but because~$\varphi$ is not satisfiable, it is "conflicting":
    necessarily, there exist $1\leq n\leq k$ and $1\leq i,i'\leq m$,
    such that $\tau$ selects $x_n$ in $C_i$ and $\neg x_n$ in
    $C_{i'}$.  But this yields a winning strategy for Eve, which
    alternates in the initial location $\varphi$ between $C_{i}$
    and $C_{i'}$, and for which an initial credit
    $\vec e_{2n-1}+\vec e_{2n}$ suffices: a contradiction.
  \end{scope}
  % We reduce from the \lang{Partition} problem: given a finite set
  % $S=\{w_0,\dots,w_{n-1}\}\subseteq\+N$ (encoded in binary), does
  % there exist a partition $S_1,S_2$ of~$S$ such that
  % $\sum_{w\in S_1}w=\sum_{w\in S_2}w$?  From such an instance, we
  % construct an "asymmetric vector system" where Eve wins the
  % "non-termination game" with "existential initial credit" if and only
  % if there is no such partition.  By \cref{11-nonterm2parity}, this
  % also entails the \coNP-hardness of "parity@parity vector games"
  % "asymmetric vector games".
  % \begin{figure}[htbp]
  % \centering
  % \begin{tikzpicture}[auto,on grid,node distance=1.8cm]
  %   \node[adam,inner sep=2.1](0){$\loc_0$};
  %   \node[eve,above right=1 and 1 of 0,inner sep=2.2](01){$\loc_{0,1}$};
  %   \node[eve,below right=1 and 1 of 0,inner sep=2.2](02){$\loc_{0,2}$};
  %   \node[adam,below right=1 and 2 of 01,inner sep=2.1](1){$\loc_1$};
  %   \node[eve,above right=1 and 1 of 1,inner sep=2.2](11){$\loc_{1,1}$};
  %   \node[eve,below right=1 and 1 of 1,inner sep=2.2](12){$\loc_{1,2}$};
  %   \node[adam,below right=1 and 2 of 11,inner sep=2.1](2){$\loc_2$};
  %   \node[right=1 of 2](dots){$\Huge\cdots$};
  %   \node[adam,right=1 of dots,inner sep=-1.8](nm){$\loc_{n-1}$};
  %   \node[eve,above right=1 and 1 of nm,inner sep=0](nm1){$\loc_{n-1,1}$};
  %   \node[eve,below right=1 and 1 of nm,inner sep=0](nm2){$\loc_{n-1,2}$};
  %   \node[eve,below right=1 and 2 of nm1,inner sep=2.2](n){$\loc_n$};    
  %   \path[->,every node/.style={font=\footnotesize,inner sep=.1}]
  %   (0) edge node {$\vec 0$} (01)
  %   (0) edge[swap] node {$\vec 0$} (02)
  %   (01) edge[near start] node {$\!-w_0\cdot\vec e_1$} (1)
  %   (02) edge[swap,near start] node {$\!-w_0\cdot\vec e_2$} (1)
  %   (1) edge node {$\vec 0$} (11)
  %   (1) edge[swap] node {$\vec 0$} (12)
  %   (11) edge[near start] node {$\!-w_1\cdot\vec e_1$} (2)
  %   (12) edge[swap,near start] node {$\!-w_1\cdot\vec e_2$} (2)
  %   (nm) edge node {$\vec 0$} (nm1)
  %   (nm) edge[swap] node {$\vec 0$} (nm2)
  %   (nm1) edge[near start] node {$\!\!-w_{n-1}\cdot\vec e_1$} (n)
  %   (nm2) edge[swap,near start] node {$\!\!-w_{n-1}\cdot\vec e_2$} (n);
  %   \draw[->,rounded corners=20pt,>=stealth',shorten >=1pt] (n) -- (10.9,1.6) -- (0.1,1.6) -- (0);
  %   \draw[->,rounded corners=20pt,>=stealth',shorten >=1pt] (n) -- (10.9,-1.6) -- (0.1,-1.6) --
  %   (0);
  %   \node[font=\footnotesize] at (5.5,1.8) {$((W/2)-1)\cdot\vec e_1$};
  %   \node[font=\footnotesize] at (5.5,-1.85) {$((W/2)-1)\cdot\vec e_2$};
  % \end{tikzpicture}
  % \caption{\label{11-fig-part} The "vector system" built from a
  %   \lang{Partition} instance.}
  % \end{figure}
  % We may assume that~$\sum_{w\in S}w$ is even, as otherwise trivially
  % no partition exists; let $W\eqdef(\sum_{w\in S}w)/2$.  We define
  % $\?V\eqdef(\Loc,\Act,\Loc_\mEve,\Loc_\mAdam,2)$ where
  % \begin{align*}
  %   \Loc_\mEve&\eqdef\{\loc_{i,j}\mid 0\leq i<n,1\leq 2\leq
  %               j\}\cup\{\loc_n\}\;,\\
  %   \Loc_\mAdam&\eqdef\{\loc_i\mid 0\leq i<n\}\;,\\
  %   \Act&\eqdef\{\loc_i\step{\vec 0}\loc_{i,j}\mid 0\leq i<n,1\leq 2\leq
  %               j\}\\
  %   &\:\cup\:\{\loc_{i,j}\step{-w_i\cdot\vec e_j}\loc_{i+1}\mid 0\leq i<n,1\leq 2\leq
  %               j\}\\
  %   &\:\cup\:\{\loc_n\step{(W-1)\cdot\vec e_j+W\cdot\vec e_{3-j}}\loc_0\mid 1\leq 2\leq
  %               j\}\;;
  % \end{align*}
  % see \cref{11-fig-part} for a depiction of~$\?V$.  We use~$\loc_0$ as
  % our initial location.
  % Let us show that Adam wins if and only if the \lang{Partition}
  % instance was positive.  If there exist $S_1,S_2\subseteq S$ with
  % $S_1\cap S_2$ and $\sum_{w\in S_1}w=\sum_{w\in S_2}w=W/2$, then
  % Adam has a winning "counterless" strategy where, in
  % location~$\loc_i$ for $0\leq i<n$, he goes to $\loc_{i,1}$ if and
  % only if $w_i\in S_1$.  Then, if the game begins in
  % $\loc_0(c_1,c_2)$, it reaches $\loc_n(c_1-W/2,c_2-W/2)$ after
  % visiting each $\loc_i$~once, and 
\end{proof}

Note that \cref{11-exist-hard} does not apply to fixed dimensions
$\dd\geq 2$.  We know by \cref{11-exist-pseudop} that those games can
be solved in pseudo-polynomial time if the number of priorities is
fixed, and by \cref{11-exist-easy} that they are in \coNP.

\subsubsection{Given Initial Credit}
With "given initial credit", we have a lower bound matching the
\kEXP[2] upper bound of \cref{11-avag-easy}, already with a unary
encoding.  The proof itself is an adaptation of the proof by
\citem[Lipton]{Lipton:1976} of $\EXPSPACE$-hardness of "coverability" in
the one-player case.

\begin{theorem}\label{11-avag-hard}
  "Coverability", "non-termination", and "parity@parity vector game"
  "asymmetric" "vector games" with "given initial credit" are
  \kEXP[2]-hard.
\end{theorem}
\begin{proof}
  We reduce from the "halting problem" of an ""alternating Minsky
  machine"" $\?M=(\Loc,\Act,\Loc_\mEve,\Loc_\mAdam,\dd)$ with counters
  bounded by $B\eqdef 2^{2^n}$ for $n\eqdef|\?M|$.  Such a machine is
  similar to an "asymmetric" "vector system" with increments
  $\loc\step{\vec e_i}\loc'$, decrements $\loc\step{-\vec e_i}\loc'$,
  and "zero test" actions $\loc\step{i\eqby{?0}}\loc'$, all
  restricted to locations $\loc\in\Loc_\mEve$; the only actions
  available to Adam are actions $\loc\step{\vec 0}\loc'$.  The
  set of locations contains a distinguished `halt' location
  $\loc_\mathtt{halt}\in\Loc$ with no outgoing action.  The
  machine comes with the promise that, along any "play", the norm of
  all the visited configurations $\loc(\vec v)$ satisfies
  $\|\vec v\|<B$.  The "halting problem" asks, given an initial
  location $\loc_0\in\Loc$, whether Eve has a winning strategy to
  visit $\loc_\mathtt{halt}(\vec v)$ for some $\vec v\in\+N^\dd$ from
  the initial configuration $\loc_0(\vec 0)$.  This problem is
  \kEXP[2]-complete if $\dd\geq 3$ by standard
  arguments~\cite{Fischer&Meyer&Rosenberg:1968}.

  %%%%%\begin{scope}
    \knowledge{meta-increment}[meta-increments]{notion}
    \knowledge{meta-decrement}[meta-decrements]{notion} Let us start
    by a quick refresher on Lipton's construction~\cite{Lipton:1976};
    see also~\cite{Esparza:1996} for a nice exposition.  At the heart
    of the construction lies a collection of one-player gadgets
    implementing \emph{level~$j$} ""meta-increments""
    $\loc\mstep{2^{2^j}\cdot\vec c}\loc'$ and \emph{level~$j$}
    ""meta-decrements"" $\loc\mstep{-2^{2^j}\cdot\vec c}\loc'$ for
    some "unit vector"~$\vec c$ using $O(j)$ auxiliary counters and
    $\poly(j)$ actions, with precondition that the auxiliary counters
    are initially empty in~$\loc$ and postrelation that they are empty
    again in~$\loc'$.  The construction is by induction over~$j$; let
    us first see a naive implementation for "meta-increments".  For
    the base case~$j=0$, this is just a standard action
    $\loc\step{2\vec c}\loc'$.  For the induction step $j+1$, we use
    the gadget of \cref{11-fig-meta-incr} below, where
    $\vec x_{j},\bar{\vec x}_{j},\vec z_{j},\bar{\vec z}_{j}$ are
    distinct fresh "unit vectors": the gadget performs two nested
    loops, each of $2^{2^j}$ iterations, thus iterates the unit
    increment of~$\vec c$ a total of $\big(2^{2^j}\big)^2=2^{2^{j+1}}$
    times.  A "meta-decrement" is obtained similarly.

    \begin{figure}[htbp]
      \centering
      \begin{tikzpicture}[auto,on grid,node distance=1.55cm]
      \node[s-eve](0){$\loc$};
      \node[s-eve-small,right=of 0](1){};
      \node[s-eve-small,right=of 1](2){};
      \node[s-eve-small,right=of 2](3){};
      \node[s-eve-small,right=of 3](4){};
      \node[s-eve-small,right=of 4](5){};
      \node[s-eve-small,right=of 5](6){};
      \node[s-eve,right=of 6](7){$\loc'$};
      \path[arrow,every node/.style={font=\footnotesize,inner sep=2pt}]
      (0) edge node{$2^{2^j}\cdot\vec x_{j}$} (1)
      (1) edge node{$2^{2^j}\cdot\vec z_{j}$} (2)
      (2) edge node{$\bar{\vec x}_{j}-\vec x_{j}$} (3)
      (3) edge node{$\bar{\vec z}_{j}-\vec z_{j}$} (4)
      (4) edge node{$\vec c$} (5)
      (5) edge node{$-2^{2^j}\cdot\bar{\vec z}_{j}$} (6)
      (6) edge node{$-2^{2^j}\cdot\bar{\vec x}_{j}$} (7);
      \draw[->,rounded corners=10pt,>=stealth'] (5) --
      (7.4,.65) -- (5,.65) -- (3);
      \node[font=\footnotesize,inner sep=2pt] at (6.2,.75) {$\vec 0$};
      \draw[->,rounded corners=10pt,>=stealth'] (6) --
      (8.95,1.25) -- (1.9,1.25) -- (1);
      \node[font=\footnotesize,inner sep=2pt] at (5.43,1.35) {$\vec 0$};
    \end{tikzpicture}
    \caption{\label{11-fig-meta-incr} A naive implementation of the
      "meta-increment" $\loc\mstep{2^{2^{j+1}}\cdot\vec c}\loc'$.}
  \end{figure}
  
  Note that this level~$(j+1)$ gadget contains two copies of the
  level~$j$ "meta-increment" and two of the level~$j$
  "meta-decrement", hence this naive implementation has
  size~$\mathsf{exp}(j)$.  In order to obtain a polynomial size, we would like
  to use a single \emph{shared} level~$j$ gadget for each~$j$, instead
  of hard-wiring multiple copies.  The idea is to use a `dispatch
  mechanism,' using extra counters, to encode the choice of "unit
  vector"~$\vec c$ and of return location~$\loc'$.  Let us see how to
  do this in the case of the return location~$\loc'$; the mechanism
  for the vector~$\vec c$ is similar.  We enumerate the (finitely many)
  possible return locations~$\loc_0,\dots,\loc_{m-1}$ of the gadget
  implementing $\loc\mstep{2^{2^{j+1}}\cdot\vec c}\loc'$.  We use two
  auxiliary counters with "unit vectors" $\vec r_j$
  and~$\bar{\vec r}_j$ to encode the return location.  Assume $\loc'$
  is the $i$th possible return location, i.e., $\loc'=\loc_i$ in our
  enumeration: before entering the shared gadget implementation, we
  initialise~$\vec r_j$ and~$\bar{\vec r}_j$ by performing the action
  $\loc\step{i\cdot\vec r_j+(m-i)\cdot\bar{\vec r}_j}\cdots$.  Then,
  where we would simply go to~$\loc'$ in \cref{11-fig-meta-incr} at
  the end of the gadget, the shared gadget has a final action
  $\cdots\step{\vec 0}\loc_{\mathrm{return}_j}$ leading to a dispatch
  location for returns: for all $0\leq i<m$, we have an action
  $\loc_{\mathrm{return}_j}\step{-i\cdot\vec r_j-(m-i)\cdot\bar{\vec
      r}_j}\loc_i$
  that leads to the desired return location.\todoquestion{Is that
    clear enough?}
  

  \bigskip Let us return to the proof.  Consider an instance of the
  "halting problem".  We first exhibit a reduction to "coverability";
  by \cref{11-cov2parity}, this will also entail the \kEXP[2]-hardness
  of "parity@parity vector game" "asymmetric" "vector games".  We
  build an "asymmetric vector system"
  $\?V=(\Loc',\Act',\Loc'_\mEve,\Loc_\mAdam,\dd')$ with
  $\dd'=2\dd+O(n)$.  Each of the counters~$\mathtt{c}_i$ of $\?M$ is
  paired with a \emph{complementary} counter~$\bar{\mathtt{c}_i}$ such
  that their sum is~$B$ throughout the simulation of~$\?M$.  We
  denote by $\vec c_i$ and $\bar{\vec c}_i$ the corresponding "unit
  vectors" for $1\leq i\leq\dd$.  The "vector system"~$\?V$ starts by
  initialising the counters $\bar{\mathtt{c}}_i$ to~$B$ by a sequence
  of "meta-increments"
  $\loc'_{i-1}\mstep{2^{2^n}\cdot\bar{\vec c}_i}\loc'_i$ for
  $1\leq i\leq\dd$, before starting the simulation by an action
  $\loc'_\dd\step{\vec 0}\loc_0$.  The simulation of~$\?M$ uses the
  actions depicted in \cref{11-fig-lipton}.  Those maintain the
  invariant on the complement counters.  Regarding "zero tests", Eve
  yields the control to Adam, who has a choice between performing a
  "meta-decrement" that will fail if $\bar{\mathtt c}_i< 2^{2^n}$,
  which by the invariant is if and only if $\mathtt{c}_i>0$, or going
  to~$\loc'$.

  \begin{figure}[htbp]
    \centering
    \begin{tikzpicture}[auto,on grid,node distance=1.5cm]
      \node(to){$\mapsto$};
      \node[anchor=east,left=2.5cm of to](mm){"alternating Minsky machine"};
      \node[anchor=west,right=2.5cm of to](mwg){"asymmetric vector system"};
      % increment of Eve
      \node[below=.7cm of to](imap){$\rightsquigarrow$};
      \node[s-eve,left=2.75cm of imap](i0){$\loc$};
      \node[right=of i0](i1){$\loc'$};
      \node[right=1.25cm of imap,s-eve](i2){$\loc$};
      \node[right=1.8 of i2](i3){$\loc'$};      
      \path[arrow,every node/.style={font=\footnotesize}]
      (i0) edge node{$\vec e_i$} (i1)
      (i2) edge node{$\vec c_i-\bar{\vec c}_i$} (i3);
      % decrement of Eve
      \node[below=1cm of imap](dmap){$\rightsquigarrow$};
      \node[s-eve,left=2.75cm of dmap](d0){$\loc$};
      \node[right=of d0](d1){$\loc'$};
      \node[right=1.25cm of dmap,s-eve](d2){$\loc$};
      \node[right=1.8 of d2](d3){$\loc'$};
      \path[arrow,every node/.style={font=\footnotesize}]
      (d0) edge node{$-\vec e_i$} (d1)
      (d2) edge node{$-\vec c_i+\bar{\vec c}_i$} (d3);
      % zero test of Eve
      \node[below=1.5cm of dmap](zmap){$\rightsquigarrow$};
      \node[s-eve,left=2.75cm of zmap](z0){$\loc$};
      \node[right=of z0](z1){$\loc'$};
      \node[right=1.25cm of zmap,s-eve](z2){$\loc$};
      \node[right=of z2,s-adam-small](z3){};
      \node[above right=.8 and 1.1 of z3,s-eve-small](z4){};
      \node[below right=.8 and 1.1 of z3,inner sep=0pt](z5){$\loc'$};
      \node[right=1.8 of z4](z6){$\loc_\mathtt{halt}$};
      \path[arrow,every node/.style={font=\footnotesize}]
      (z0) edge node{$i\eqby{?0}$} (z1)
      (z2) edge node{$\vec 0$} (z3)
      (z3) edge node{$\vec 0$} (z4)
      (z3) edge[swap] node{$\vec 0$} (z5)
      (z4) edge node{$-2^{2^n}\cdot\bar{\vec c}_i$} (z6);
      % action of Adam
      \node[below=1.5cm of zmap](amap){$\rightsquigarrow$};
      \node[s-adam,left=2.75cm of amap](a0){$\loc$};
      \node[right=of a0](a1){$\loc'$};
      \node[right=1.25cm of amap,s-adam](a2){$\loc$};
      \node[right=of a2](a3){$\loc'$};
      \path[arrow,every node/.style={font=\footnotesize}]
      (a0) edge node{$\vec 0$} (a1)
      (a2) edge node{$\vec 0$} (a3);
    \end{tikzpicture}
    \caption{\label{11-fig-lipton}Schema of the reduction to
      "coverability" in the proof of \cref{11-avag-hard}.}
  \end{figure}
  
  It is hopefully clear that Eve wins the "coverability game" played
  on~$\?V$ starting from $\loc'_0(\vec 0)$ and with target
  configuration $\loc_\mathtt{halt}(\vec 0)$ if and only if the
  "alternating Minsky machine" halts.

  \medskip Regarding "non-termination" games, we use essentially the
  same reduction.  First observe that, if Eve can ensure reaching
  $\loc_\mathtt{halt}$ in the "alternating Minsky machine", then she
  can do so after at most $|\Loc|B^\dd$ steps.  We therefore use a
  `time budget': this is an additional component in $\?V$ with
  associated "unit vector"~$\vec t$.  This component is initialised to
  $|\Loc|B^\dd=|\Loc|2^{\dd 2^n}$ before the simulation, and decreases
  by~one at every step; see \cref{11-fig-lipton-nonterm}.  We also add
  a self loop $\loc_\mathtt{halt}\step{\vec 0}\loc_\mathtt{halt}$.
  Then the only way to avoid the "sink" and thus to win the
  "non-termination" game is to reach $\loc_\mathtt{halt}$.
  \begin{figure}[htbp]
    \centering
    \begin{tikzpicture}[auto,on grid,node distance=1.5cm]
      \node(to){$\mapsto$};
      \node[anchor=east,left=2.5cm of to](mm){"alternating Minsky machine"};
      \node[anchor=west,right=2.5cm of to](mwg){"asymmetric vector system"};
      % increment of Eve
      \node[below=.7cm of to](imap){$\rightsquigarrow$};
      \node[s-eve,left=2.75cm of imap](i0){$\loc$};
      \node[right=of i0](i1){$\loc'$};
      \node[right=1.25cm of imap,s-eve](i2){$\loc$};
      \node[right=1.8 of i2](i3){$\loc'$};      
      \path[arrow,every node/.style={font=\footnotesize}]
      (i0) edge node{$\vec e_i$} (i1)
      (i2) edge node{$\vec c_i-\bar{\vec c}_i-\vec t$} (i3);
      % decrement of Eve
      \node[below=1cm of imap](dmap){$\rightsquigarrow$};
      \node[s-eve,left=2.75cm of dmap](d0){$\loc$};
      \node[right=of d0](d1){$\loc'$};
      \node[right=1.25cm of dmap,s-eve](d2){$\loc$};
      \node[right=1.8 of d2](d3){$\loc'$};
      \path[arrow,every node/.style={font=\footnotesize}]
      (d0) edge node{$-\vec e_i$} (d1)
      (d2) edge node{$-\vec c_i+\bar{\vec c}_i-\vec t$} (d3);
      % zero test of Eve
      \node[below=1.5cm of dmap](zmap){$\rightsquigarrow$};
      \node[s-eve,left=2.75cm of zmap](z0){$\loc$};
      \node[right=of z0](z1){$\loc'$};
      \node[right=1.25cm of zmap,s-eve](z2){$\loc$};
      \node[right=of z2,s-adam-small](z3){};
      \node[above right=.8 and 1.1 of z3,s-eve-small](z4){};
      \node[below right=.8 and 1.1 of z3,inner sep=0pt](z5){$\loc'$};
      \node[right=1.8 of z4](z6){$\loc_\mathtt{halt}$};
      \path[arrow,every node/.style={font=\footnotesize}]
      (z0) edge node{$i\eqby{?0}$} (z1)
      (z2) edge node{$-\vec t$} (z3)
      (z3) edge node{$\vec 0$} (z4)
      (z3) edge[swap] node{$\vec 0$} (z5)
      (z4) edge node{$-2^{2^n}\cdot\bar{\vec c}_i$} (z6);
      % action of Adam
      \node[below=1.5cm of zmap](amap){$\rightsquigarrow$};
      \node[s-adam,left=2.75cm of amap](a0){$\loc$};
      \node[right=of a0](a1){$\loc'$};
      \node[right=1.25cm of amap,s-adam](a2){$\loc$};
      \node[right=of a2,s-eve-small](a3){};
      \node[right=of a3](a4){$\loc'$};
      \path[arrow,every node/.style={font=\footnotesize}]
      (a0) edge node{$\vec 0$} (a1)
      (a2) edge node{$\vec 0$} (a3)
      (a3) edge node{$-\vec t$} (a4);
    \end{tikzpicture}
    \caption{\label{11-fig-lipton-nonterm}Schema of the reduction to
      "non-termination" in the proof of \cref{11-avag-hard}.}
  \end{figure}

  We still need to extend our initialisation phase.  It suffices for
  this to implement a gadget for $\dd$-"meta-increments"
  $\loc\mstep{2^{\dd 2^j}\cdot\vec c}\loc'$ and $\dd$-"meta-decrements"
  $\loc\mstep{-2^{\dd 2^j}\cdot\vec c}\loc'$; this is the same argument as
  in Lipton's construction, with a base case $\loc\mstep{2^\dd}\loc'$
  for $j=0$.  Then we initialise our time budget through $|\Loc|$
  successive $\dd$-"meta-increments"
  $\loc\mstep{2^{\dd 2^n}\cdot\vec t}\loc'$.
  %%\end{scope}
\end{proof}

The proof of \cref{11-avag-hard} relies crucially on the fact that the
dimension is not fixed: although $\dd\geq 3$ suffices in the
"alternating Minsky machine", we need $O(|\?M|)$ additional counters
to carry out the reduction.  A separate argument is thus needed in
order to match the \EXP\ upper bound of \cref{11-avag-easy} in fixed
dimension.

\begin{theorem}\label{11-avag-two}
  "Coverability", "non-termination", and "parity@parity vector game"
  "asymmetric" "vector games" with "given initial credit" are
  \EXP-hard in dimension $\dd\geq 2$.
\end{theorem}
\begin{proof}
  We exhibit a reduction from "countdown games" with "given initial
  credit", which are \EXP-complete by \cref{11-countdown-given}.
  Consider an instance of a "configuration reachability" countdown
  game: a "countdown system"
  $\?V=(\Loc,\Act,\Loc_\mEve,\Loc_\mAdam,1)$ with initial
  configuration $\loc_0(n_0)$ and target
  configuration~$\smiley(0)$---as seen in the proof
  of \cref{11-countdown-given}, we can indeed assume that the target
  credit is zero; we will also assume that Eve controls~$\smiley$ and
  that the only action available in~$\smiley$ is
  $\smiley\step{-1}\smiley$.  We construct an "asymmetric" "vector
  system" $\?V'$ of dimension~2 such that Eve can ensure
  reaching~$\smiley(0,n_0)$ from $\loc_0(n_0,0)$ in~$\?V'$ if and only
  if she could ensure reaching $\smiley(0)$ from $\loc_0(n_0)$
  in~$\?V$.  The translation is depicted in \cref{11-fig-dim2}.
  
  \begin{figure}[htbp]
    \centering
    \begin{tikzpicture}[auto,on grid,node distance=1.5cm]
      \node(to){$\mapsto$};
      \node[anchor=east,left=2.5cm of to](mm){"countdown system"};
      \node[anchor=west,right=2.5cm of to](mwg){"asymmetric vector system"};
      % action of Eve
      \node[below=.7cm of to](imap){$\rightsquigarrow$};
      \node[s-eve,left=2.75cm of imap](i0){$\loc$};
      \node[right=of i0](i1){$\loc'$};
      \node[right=1.25cm of imap,s-eve](i2){$\loc$};
      \node[right=1.8 of i2](i3){$\loc'$};      
      \path[arrow,every node/.style={font=\footnotesize,inner sep=1pt}]
      (i0) edge node{$-n$} (i1)
      (i2) edge node{$-n,n$} (i3);
      % minimal action of Adam
      \node[below=1cm of imap](dmap){$\rightsquigarrow$};
      \node[s-adam,left=2.75cm of dmap](d0){$\loc$};
      \node[right=of d0](d1){$\loc'$};
      \node[below=.5 of d0]{$n=\min\{n'\mid\exists\loc''\in\Loc\mathbin.\loc\step{-n'}\loc''\in\Act\}$};
      \node[right=1.25cm of dmap,s-adam](d2){$\loc$};
      \node[right=1.8 of d2,s-eve-small](d3){};
      \node[right=1.8 of d3](d4){$\loc'$};
      \path[arrow,every node/.style={font=\footnotesize,inner sep=1pt}]
      (d0) edge node{$-n$} (d1)
      (d2) edge node{$0,0$} (d3)
      (d3) edge node{$-n,n$} (d4);
      % non-minimal action of Adam
      \node[below=1.5cm of dmap](zmap){$\rightsquigarrow$};
      \node[s-adam,left=2.75cm of zmap](z0){$\loc$};
      \node[right=of z0](z1){$\loc'$};
      \node[below=.5 of z0]{$n\neq\min\{n'\mid\exists\loc''\in\Loc\mathbin.\loc\step{-n'}\loc''\in\Act\}$};
      \node[right=1.25cm of zmap,s-adam](z2){$\loc$};
      \node[right=of z2,s-eve-small](z3){};
      \node[above right=.8 and 2.1 of z3](z4){$\loc'$};
      \node[below right=.8 and 2.1 of z3,s-eve](z5){$\smiley$};
      \path[arrow,every node/.style={font=\footnotesize,inner sep=1pt}]
      (z0) edge node{$-n$} (z1)
      (z2) edge node{$0,0$} (z3)
      (z3) edge[bend left=8] node{$-n,n$} (z4)
      (z3) edge[swap,bend right=8] node{$n_0-n+1,-n_0+n-1$} (z5)
      (z5) edge[loop above] node{$-1,1$} ()
      (z5) edge[loop right] node{$\,0,0$} ();
    \end{tikzpicture}
    \caption{\label{11-fig-dim2}Schema of the reduction in the proof
    of \cref{11-avag-two}.}
  \end{figure}
    
  The idea behind this translation is that a configuration $\loc(c)$
  of~$\?V$ is simulated by a configuration $\loc(c,n_0-c)$ in~$\?V'$.
  The crucial point is how to handle Adam\'s moves.  In a
  configuration $\loc(c,n_0-c)$ with $\loc\in\Loc_\mAdam$, according
  to the "natural semantics" of $\?V$, Adam should be able to
  simulate an action $\loc\step{-n}\loc'$ if and only if $c\geq n$.
  Observe that otherwise if $c<n$ and thus $n_0-c>n_0-n$, Eve can
  play to reach~$\smiley$ and win immediately.  An exception to the
  above is if $n$ is minimal among the decrements in~$\loc$, because
  according to the "natural semantics" of~$\?V$, if $c<n$ there should
  be an edge to the "sink", and this is handled in the second line
  of \cref{11-fig-dim2}.

  Then Eve can reach $\smiley(0,n_0)$ if and only if she can cover
  $\smiley(0,n_0)$, if and only if she can avoid the "sink" thanks to
  the self loop $\smiley\step{0,0}\smiley$.  This
  shows the \EXP-hardness of "coverability" and "non-termination"
  "asymmetric" "vector games" in dimension~two; the hardness of
  "parity@parity vector game" follows
  from \cref{11-cov2parity,12-nonterm2parity}.
\end{proof}



% Local IspellDict: british

\subsubsection{Existential Initial Credit}
In the "existential initial credit" variant of our games, we have the
following lower bound matching \cref{11-th:exist-easy}, already with a
unary encoding.

\begin{theorem}[Existential non-termination asymmetric vector games are \coNP-hard]
\label{11-th:exist-hard}
  "Non-termination", and "parity@parity vector game"
  "asymmetric" "vector games" with "existential initial credit" are
  \coNP-hard.% in any dimension~$\dd\geq 2$.
\end{theorem}
\begin{proof}
  By \cref{11-rk:nonterm2parity}, it suffices to show hardness for
  "non-termination games".  We reduce from the \lang{3SAT} problem:
  given a formula $\varphi=\bigwedge_{1\leq i\leq m}C_i$ where each
  clause $C_i$ is a disjonction of the form
  $\litt_{i,1}\vee\litt_{i,2}\vee\litt_{i,3}$ of literals taken from
  $X=\{x_1,\neg x_1,x_2,$ $\neg x_2,\dots,x_k,\neg x_k\}$, we construct
  an "asymmetric" "vector system" $\?V$ where \Eve\ wins the
  "non-termination game" with "existential initial credit" if and only
  if~$\varphi$ is not satisfiable; since the game is determined, we
  actually show that \Adam\ wins the game if and only if~$\varphi$ is
  satisfiable.

  Our "vector system" has dimension~$2k$, and for a literal
  $\litt\in X$, we define the vector
  \begin{equation*}
    \vec u_\litt\eqdef\begin{cases}
      \vec e_{2n-1}-\vec e_{2n}&\text{if }\litt=x_n\;,\\
      \vec e_{2n}-\vec e_{2n-1}&\text{if }\litt=\neg x_n\;.
    \end{cases}
  \end{equation*}
  We define $\?V\eqdef(\Loc,\Act,\Loc_\mEve,\Loc_\mAdam,2k)$ where
  \begin{align*}
    \Loc_\mEve&\eqdef\{\varphi\}\cup\{\litt_{i,j}\mid 1\leq i\leq m,1\leq j\leq
                3\}\;,\\
    \Loc_\mAdam&\eqdef\{C_i\mid 1\leq i\leq m\}\;,\\
    \Act&\eqdef\{\varphi\step{\vec 0}C_i\mid 1\leq i\leq m\}\cup\{C_i\step{\vec 0}\litt_{i,j},\;\;\litt_{i,j}\xrightarrow{\vec u_{\litt_{i,j}}}\varphi\mid 1\leq i\leq m,1\leq j\leq 3\}\;.
  \end{align*}
  \begin{scope}
    We use~$\varphi$ as our initial location.
    \knowledge{literal assignment}{notion}
    \knowledge{conflicting}{notion}
    %
    Let us call a map $v{:}\,X\to\{0,1\}$ a ""literal assignment""; we
    call it ""conflicting"" if there exists $1\leq n\leq k$ such that
    $v(x_n)=v(\neg x_n)$.

    Assume that~$\varphi$ is satisfiable.  Then there exists a
    non-"conflicting" "literal assignment"~$v$ that satisfies all the
    clauses: for each $1\leq i\leq m$, there exists $1\leq j\leq 3$
    such that $v(\litt_{i,j})=1$; this yields a "counterless" strategy
    for \Adam, which selects $(C_i,\litt_{i,j})$ for each
    $1\leq i\leq m$.  Consider any infinite "play" consistent with
    this strategy.  This "play" only visits literals $\litt$ where
    $v(\litt)=1$.  There exists a literal $\litt\in X$ that is visited
    infinitely often along the "play", say $\litt=x_n$.  Because~$v$ is
    non-"conflicting", $v(\neg x_n)=0$, thus the location $\neg x_n$
    is never visited.  Thus the play uses the action
    $\litt\step{\vec e_{2n-1}-\vec e_{2n}}\varphi$ infinitely often,
    and never uses any action with a positive effect on
    component~$2n$.  Hence the play is losing from any initial credit.

    Conversely, assume that~$\varphi$ is not satisfiable.  By
    contradiction, assume that \Adam\ wins the game for all initial
    credits.  By \cref{11-lem:counterless}, he has a "counterless" winning
    strategy~$\tau$ that selects a literal in every clause.  Consider
    a "literal assignment" that maps each one of the selected literals
    to~$1$ and the remaining ones in a non-conflicting manner.  By
    definition, this "literal assignment" satisfies all the clauses,
    but because~$\varphi$ is not satisfiable, it is "conflicting":
    necessarily, there exist $1\leq n\leq k$ and $1\leq i,i'\leq m$,
    such that $\tau$ selects $x_n$ in $C_i$ and $\neg x_n$ in
    $C_{i'}$.  But this yields a winning strategy for \Eve, which
    alternates in the initial location $\varphi$ between $C_{i}$
    and $C_{i'}$, and for which an initial credit
    $\vec e_{2n-1}+\vec e_{2n}$ suffices: a contradiction.
  \end{scope}
  % We reduce from the \lang{Partition} problem: given a finite set
  % $S=\{w_0,\dots,w_{n-1}\}\subseteq\+N$ (encoded in binary), does
  % there exist a partition $S_1,S_2$ of~$S$ such that
  % $\sum_{w\in S_1}w=\sum_{w\in S_2}w$?  From such an instance, we
  % construct an "asymmetric vector system" where \Eve\ wins the
  % "non-termination game" with "existential initial credit" if and only
  % if there is no such partition.  By \cref{11-rk:nonterm2parity}, this
  % also entails the \coNP-hardness of "parity@parity vector games"
  % "asymmetric vector games".
  % \begin{figure}[htbp]
  % \centering
  % \begin{tikzpicture}[auto,on grid,node distance=1.8cm]
  %   \node[adam,inner sep=2.1](0){$\loc_0$};
  %   \node[eve,above right=1 and 1 of 0,inner sep=2.2](01){$\loc_{0,1}$};
  %   \node[eve,below right=1 and 1 of 0,inner sep=2.2](02){$\loc_{0,2}$};
  %   \node[adam,below right=1 and 2 of 01,inner sep=2.1](1){$\loc_1$};
  %   \node[eve,above right=1 and 1 of 1,inner sep=2.2](11){$\loc_{1,1}$};
  %   \node[eve,below right=1 and 1 of 1,inner sep=2.2](12){$\loc_{1,2}$};
  %   \node[adam,below right=1 and 2 of 11,inner sep=2.1](2){$\loc_2$};
  %   \node[right=1 of 2](dots){$\Huge\cdots$};
  %   \node[adam,right=1 of dots,inner sep=-1.8](nm){$\loc_{n-1}$};
  %   \node[eve,above right=1 and 1 of nm,inner sep=0](nm1){$\loc_{n-1,1}$};
  %   \node[eve,below right=1 and 1 of nm,inner sep=0](nm2){$\loc_{n-1,2}$};
  %   \node[eve,below right=1 and 2 of nm1,inner sep=2.2](n){$\loc_n$};    
  %   \path[->,every node/.style={font=\footnotesize,inner sep=.1}]
  %   (0) edge node {$\vec 0$} (01)
  %   (0) edge[swap] node {$\vec 0$} (02)
  %   (01) edge[near start] node {$\!-w_0\cdot\vec e_1$} (1)
  %   (02) edge[swap,near start] node {$\!-w_0\cdot\vec e_2$} (1)
  %   (1) edge node {$\vec 0$} (11)
  %   (1) edge[swap] node {$\vec 0$} (12)
  %   (11) edge[near start] node {$\!-w_1\cdot\vec e_1$} (2)
  %   (12) edge[swap,near start] node {$\!-w_1\cdot\vec e_2$} (2)
  %   (nm) edge node {$\vec 0$} (nm1)
  %   (nm) edge[swap] node {$\vec 0$} (nm2)
  %   (nm1) edge[near start] node {$\!\!-w_{n-1}\cdot\vec e_1$} (n)
  %   (nm2) edge[swap,near start] node {$\!\!-w_{n-1}\cdot\vec e_2$} (n);
  %   \draw[->,rounded corners=20pt,>=stealth',shorten >=1pt] (n) -- (10.9,1.6) -- (0.1,1.6) -- (0);
  %   \draw[->,rounded corners=20pt,>=stealth',shorten >=1pt] (n) -- (10.9,-1.6) -- (0.1,-1.6) --
  %   (0);
  %   \node[font=\footnotesize] at (5.5,1.8) {$((W/2)-1)\cdot\vec e_1$};
  %   \node[font=\footnotesize] at (5.5,-1.85) {$((W/2)-1)\cdot\vec e_2$};
  % \end{tikzpicture}
  % \caption{\label{11-fig:part} The "vector system" built from a
  %   \lang{Partition} instance.}
  % \end{figure}
  % We may assume that~$\sum_{w\in S}w$ is even, as otherwise trivially
  % no partition exists; let $W\eqdef(\sum_{w\in S}w)/2$.  We define
  % $\?V\eqdef(\Loc,\Act,\Loc_\mEve,\Loc_\mAdam,2)$ where
  % \begin{align*}
  %   \Loc_\mEve&\eqdef\{\loc_{i,j}\mid 0\leq i<n,1\leq 2\leq
  %               j\}\cup\{\loc_n\}\;,\\
  %   \Loc_\mAdam&\eqdef\{\loc_i\mid 0\leq i<n\}\;,\\
  %   \Act&\eqdef\{\loc_i\step{\vec 0}\loc_{i,j}\mid 0\leq i<n,1\leq 2\leq
  %               j\}\\
  %   &\:\cup\:\{\loc_{i,j}\step{-w_i\cdot\vec e_j}\loc_{i+1}\mid 0\leq i<n,1\leq 2\leq
  %               j\}\\
  %   &\:\cup\:\{\loc_n\step{(W-1)\cdot\vec e_j+W\cdot\vec e_{3-j}}\loc_0\mid 1\leq 2\leq
  %               j\}\;;
  % \end{align*}
  % see \cref{11-fig:part} for a depiction of~$\?V$.  We use~$\loc_0$ as
  % our initial location.
  % Let us show that \Adam\ wins if and only if the \lang{Partition}
  % instance was positive.  If there exist $S_1,S_2\subseteq S$ with
  % $S_1\cap S_2$ and $\sum_{w\in S_1}w=\sum_{w\in S_2}w=W/2$, then
  % \Adam\ has a winning "counterless" strategy where, in
  % location~$\loc_i$ for $0\leq i<n$, he goes to $\loc_{i,1}$ if and
  % only if $w_i\in S_1$.  Then, if the game begins in
  % $\loc_0(c_1,c_2)$, it reaches $\loc_n(c_1-W/2,c_2-W/2)$ after
  % visiting each $\loc_i$~once, and 
\end{proof}

Note that \cref{11-th:exist-hard} does not apply to fixed dimensions
$\dd\geq 2$.  We know by \cref{11-cor:exist-pseudop} that those games can
be solved in pseudo-polynomial time if the number of priorities is
fixed, and by \cref{11-th:exist-easy} that they are in \coNP.

\subsubsection{Given Initial Credit}
With "given initial credit", we have a lower bound matching the
\kEXP[2] upper bound of \cref{11-th:avag-easy}, already with a unary
encoding.  The proof itself is an adaptation of the proof by
\citem[Lipton]{Lipton:1976} of \EXPSPACE-hardness of "coverability" in
the one-player case.

\begin{theorem}[Coverability and non-termination asymmetric vector games are {\kEXP[2]-hard}]
\label{11-th:avag-hard}
  "Coverability", "non-termination", and "parity@parity vector game"
  "asymmetric" "vector games" with "given initial credit" are
  \kEXP[2]-hard.
\end{theorem}
\begin{proof}
  We reduce from the "halting problem" of an ""alternating Minsky
  machine"" $\?M=(\Loc,\Act,\Loc_\mEve,\Loc_\mAdam,\dd)$ with counters
  bounded by $B\eqdef 2^{2^n}$ for $n\eqdef|\?M|$.  Such a machine is
  similar to an "asymmetric" "vector system" with increments
  $\loc\step{\vec e_i}\loc'$, decrements $\loc\step{-\vec e_i}\loc'$,
  and "zero test" actions $\loc\step{i\eqby?0}\loc'$, all
  restricted to locations $\loc\in\Loc_\mEve$; the only actions
  available to \Adam\ are actions $\loc\step{\vec 0}\loc'$.  The
  set of locations contains a distinguished `halt' location
  $\loc_\mathtt{halt}\in\Loc$ with no outgoing action.  The
  machine comes with the promise that, along any "play", the norm of
  all the visited configurations $\loc(\vec v)$ satisfies
  $\|\vec v\|<B$.  The "halting problem" asks, given an initial
  location $\loc_0\in\Loc$, whether \Eve\ has a winning strategy to
  visit $\loc_\mathtt{halt}(\vec v)$ for some $\vec v\in\+N^\dd$ from
  the initial configuration $\loc_0(\vec 0)$.  This problem is
  \kEXP[2]-complete if $\dd\geq 3$ by standard
  arguments~\cite{Fischer&Meyer&Rosenberg:1968}.

  %%%%%\begin{scope}
    \knowledge{meta-increment}[meta-increments]{notion}
    \knowledge{meta-decrement}[meta-decrements]{notion} Let us start
    by a quick refresher on Lipton's construction~\cite{Lipton:1976};
    see also~\cite{Esparza:1996} for a nice exposition.  At the heart
    of the construction lies a collection of one-player gadgets
    implementing \emph{level~$j$} ""meta-increments""
    $\loc\mstep{2^{2^j}\cdot\vec c}\loc'$ and \emph{level~$j$}
    ""meta-decrements"" $\loc\mstep{-2^{2^j}\cdot\vec c}\loc'$ for
    some "unit vector"~$\vec c$ using $O(j)$ auxiliary counters and
    $\poly(j)$ actions, with precondition that the auxiliary counters
    are initially empty in~$\loc$ and postrelation that they are empty
    again in~$\loc'$.  The construction is by induction over~$j$; let
    us first see a naive implementation for "meta-increments".  For
    the base case~$j=0$, this is just a standard action
    $\loc\step{2\vec c}\loc'$.  For the induction step $j+1$, we use
    the gadget of \cref{11-fig:meta-incr} below, where
    $\vec x_{j},\bar{\vec x}_{j},\vec z_{j},\bar{\vec z}_{j}$ are
    distinct fresh "unit vectors": the gadget performs two nested
    loops, each of $2^{2^j}$ iterations, thus iterates the unit
    increment of~$\vec c$ a total of $\big(2^{2^j}\big)^2=2^{2^{j+1}}$
    times.  A "meta-decrement" is obtained similarly.

    \begin{figure}[htbp]
      \centering
      \begin{tikzpicture}[auto,on grid,node distance=1.55cm]
      \node[s-eve](0){$\loc$};
      \node[s-eve-small,right=of 0](1){};
      \node[s-eve-small,right=of 1](2){};
      \node[s-eve-small,right=of 2](3){};
      \node[s-eve-small,right=of 3](4){};
      \node[s-eve-small,right=of 4](5){};
      \node[s-eve-small,right=of 5](6){};
      \node[s-eve,right=of 6](7){$\loc'$};
      \path[arrow,every node/.style={font=\footnotesize,inner sep=2pt}]
      (0) edge node{$2^{2^j}\cdot\vec x_{j}$} (1)
      (1) edge node{$2^{2^j}\cdot\vec z_{j}$} (2)
      (2) edge node{$\bar{\vec x}_{j}-\vec x_{j}$} (3)
      (3) edge node{$\bar{\vec z}_{j}-\vec z_{j}$} (4)
      (4) edge node{$\vec c$} (5)
      (5) edge node{$-2^{2^j}\cdot\bar{\vec z}_{j}$} (6)
      (6) edge node{$-2^{2^j}\cdot\bar{\vec x}_{j}$} (7);
      \draw[->,rounded corners=10pt,>=stealth'] (5) --
      (7.4,.65) -- (5,.65) -- (3);
      \node[font=\footnotesize,inner sep=2pt] at (6.2,.75) {$\vec 0$};
      \draw[->,rounded corners=10pt,>=stealth'] (6) --
      (8.95,1.25) -- (1.9,1.25) -- (1);
      \node[font=\footnotesize,inner sep=2pt] at (5.43,1.35) {$\vec 0$};
    \end{tikzpicture}
    \caption{A naive implementation of the
      "meta-increment" $\loc\mstep{2^{2^{j+1}}\cdot\vec c}\loc'$.}\label{11-fig:meta-incr}
  \end{figure}
  
  Note that this level~$(j+1)$ gadget contains two copies of the
  level~$j$ "meta-increment" and two of the level~$j$
  "meta-decrement", hence this naive implementation has
  size~$\mathsf{exp}(j)$.  In order to obtain a polynomial size, we would like
  to use a single \emph{shared} level~$j$ gadget for each~$j$, instead
  of hard-wiring multiple copies.  The idea is to use a `dispatch
  mechanism,' using extra counters, to encode the choice of "unit
  vector"~$\vec c$ and of return location~$\loc'$.  Let us see how to
  do this in the case of the return location~$\loc'$; the mechanism
  for the vector~$\vec c$ is similar.  We enumerate the (finitely many)
  possible return locations~$\loc_0,\dots,\loc_{m-1}$ of the gadget
  implementing $\loc\mstep{2^{2^{j+1}}\cdot\vec c}\loc'$.  We use two
  auxiliary counters with "unit vectors" $\vec r_j$
  and~$\bar{\vec r}_j$ to encode the return location.  Assume $\loc'$
  is the $i$th possible return location, i.e., $\loc'=\loc_i$ in our
  enumeration: before entering the shared gadget implementation, we
  initialise~$\vec r_j$ and~$\bar{\vec r}_j$ by performing the action
  $\loc\step{i\cdot\vec r_j+(m-i)\cdot\bar{\vec r}_j}\cdots$.  Then,
  where we would simply go to~$\loc'$ in \cref{11-fig:meta-incr} at
  the end of the gadget, the shared gadget has a final action
  $\cdots\step{\vec 0}\loc_{\mathrm{return}_j}$ leading to a dispatch
  location for returns: for all $0\leq i<m$, we have an action
  $\loc_{\mathrm{return}_j}\step{-i\cdot\vec r_j-(m-i)\cdot\bar{\vec
      r}_j}\loc_i$
  that leads to the desired return location.\todoquestion{Is that
    clear enough?}
  

  \bigskip Let us return to the proof.  Consider an instance of the
  "halting problem".  We first exhibit a reduction to "coverability";
  by \cref{11-rk:cov2parity}, this will also entail the \kEXP[2]-hardness
  of "parity@parity vector game" "asymmetric" "vector games".  We
  build an "asymmetric vector system"
  $\?V=(\Loc',\Act',\Loc'_\mEve,\Loc_\mAdam,\dd')$ with
  $\dd'=2\dd+O(n)$.  Each of the counters~$\mathtt{c}_i$ of $\?M$ is
  paired with a \emph{complementary} counter~$\bar{\mathtt{c}_i}$ such
  that their sum is~$B$ throughout the simulation of~$\?M$.  We
  denote by $\vec c_i$ and $\bar{\vec c}_i$ the corresponding "unit
  vectors" for $1\leq i\leq\dd$.  The "vector system"~$\?V$ starts by
  initialising the counters $\bar{\mathtt{c}}_i$ to~$B$ by a sequence
  of "meta-increments"
  $\loc'_{i-1}\mstep{2^{2^n}\cdot\bar{\vec c}_i}\loc'_i$ for
  $1\leq i\leq\dd$, before starting the simulation by an action
  $\loc'_\dd\step{\vec 0}\loc_0$.  The simulation of~$\?M$ uses the
  actions depicted in \cref{11-fig:lipton}.  Those maintain the
  invariant on the complement counters.  Regarding "zero tests", \Eve\ 
  yields the control to \Adam, who has a choice between performing a
  "meta-decrement" that will fail if $\bar{\mathtt c}_i< 2^{2^n}$,
  which by the invariant is if and only if $\mathtt{c}_i>0$, or going
  to~$\loc'$.

  \begin{figure}[htbp]
    \centering
    \begin{tikzpicture}[auto,on grid,node distance=1.5cm]
      \node(to){$\mapsto$};
      \node[anchor=east,left=2.5cm of to](mm){"alternating Minsky machine"};
      \node[anchor=west,right=2.5cm of to](mwg){"asymmetric vector system"};
      % increment of Eve
      \node[below=.7cm of to](imap){$\rightsquigarrow$};
      \node[s-eve,left=2.75cm of imap](i0){$\loc$};
      \node[right=of i0](i1){$\loc'$};
      \node[right=1.25cm of imap,s-eve](i2){$\loc$};
      \node[right=1.8 of i2](i3){$\loc'$};      
      \path[arrow,every node/.style={font=\footnotesize}]
      (i0) edge node{$\vec e_i$} (i1)
      (i2) edge node{$\vec c_i-\bar{\vec c}_i$} (i3);
      % decrement of Eve
      \node[below=1cm of imap](dmap){$\rightsquigarrow$};
      \node[s-eve,left=2.75cm of dmap](d0){$\loc$};
      \node[right=of d0](d1){$\loc'$};
      \node[right=1.25cm of dmap,s-eve](d2){$\loc$};
      \node[right=1.8 of d2](d3){$\loc'$};
      \path[arrow,every node/.style={font=\footnotesize}]
      (d0) edge node{$-\vec e_i$} (d1)
      (d2) edge node{$-\vec c_i+\bar{\vec c}_i$} (d3);
      % zero test of Eve
      \node[below=1.5cm of dmap](zmap){$\rightsquigarrow$};
      \node[s-eve,left=2.75cm of zmap](z0){$\loc$};
      \node[right=of z0](z1){$\loc'$};
      \node[right=1.25cm of zmap,s-eve](z2){$\loc$};
      \node[right=of z2,s-adam-small](z3){};
      \node[above right=.8 and 1.1 of z3,s-eve-small](z4){};
      \node[below right=.8 and 1.1 of z3,inner sep=0pt](z5){$\loc'$};
      \node[right=1.8 of z4](z6){$\loc_\mathtt{halt}$};
      \path[arrow,every node/.style={font=\footnotesize}]
      (z0) edge node{$i\eqby?0$} (z1)
      (z2) edge node{$\vec 0$} (z3)
      (z3) edge node{$\vec 0$} (z4)
      (z3) edge[swap] node{$\vec 0$} (z5)
      (z4) edge node{$-2^{2^n}\cdot\bar{\vec c}_i$} (z6);
      % action of Adam
      \node[below=1.5cm of zmap](amap){$\rightsquigarrow$};
      \node[s-adam,left=2.75cm of amap](a0){$\loc$};
      \node[right=of a0](a1){$\loc'$};
      \node[right=1.25cm of amap,s-adam](a2){$\loc$};
      \node[right=of a2](a3){$\loc'$};
      \path[arrow,every node/.style={font=\footnotesize}]
      (a0) edge node{$\vec 0$} (a1)
      (a2) edge node{$\vec 0$} (a3);
    \end{tikzpicture}
    \caption{Schema of the reduction to
      "coverability" in the proof of \cref{11-th:avag-hard}.}\label{11-fig:lipton}
  \end{figure}
  
  It is hopefully clear that \Eve\ wins the "coverability game" played
  on~$\?V$ starting from $\loc'_0(\vec 0)$ and with target
  configuration $\loc_\mathtt{halt}(\vec 0)$ if and only if the
  "alternating Minsky machine" halts.

  \medskip Regarding "non-termination" games, we use essentially the
  same reduction.  First observe that, if \Eve\ can ensure reaching
  $\loc_\mathtt{halt}$ in the "alternating Minsky machine", then she
  can do so after at most $|\Loc|B^\dd$ steps.  We therefore use a
  `time budget': this is an additional component in $\?V$ with
  associated "unit vector"~$\vec t$.  This component is initialised to
  $|\Loc|B^\dd=|\Loc|2^{\dd 2^n}$ before the simulation, and decreases
  by~one at every step; see \cref{11-fig:lipton-nonterm}.  We also add
  a self loop $\loc_\mathtt{halt}\step{\vec 0}\loc_\mathtt{halt}$.
  Then the only way to avoid the "sink" and thus to win the
  "non-termination" game is to reach $\loc_\mathtt{halt}$.
  \begin{figure}[htbp]
    \centering
    \begin{tikzpicture}[auto,on grid,node distance=1.5cm]
      \node(to){$\mapsto$};
      \node[anchor=east,left=2.5cm of to](mm){"alternating Minsky machine"};
      \node[anchor=west,right=2.5cm of to](mwg){"asymmetric vector system"};
      % increment of Eve
      \node[below=.7cm of to](imap){$\rightsquigarrow$};
      \node[s-eve,left=2.75cm of imap](i0){$\loc$};
      \node[right=of i0](i1){$\loc'$};
      \node[right=1.25cm of imap,s-eve](i2){$\loc$};
      \node[right=1.8 of i2](i3){$\loc'$};      
      \path[arrow,every node/.style={font=\footnotesize}]
      (i0) edge node{$\vec e_i$} (i1)
      (i2) edge node{$\vec c_i-\bar{\vec c}_i-\vec t$} (i3);
      % decrement of Eve
      \node[below=1cm of imap](dmap){$\rightsquigarrow$};
      \node[s-eve,left=2.75cm of dmap](d0){$\loc$};
      \node[right=of d0](d1){$\loc'$};
      \node[right=1.25cm of dmap,s-eve](d2){$\loc$};
      \node[right=1.8 of d2](d3){$\loc'$};
      \path[arrow,every node/.style={font=\footnotesize}]
      (d0) edge node{$-\vec e_i$} (d1)
      (d2) edge node{$-\vec c_i+\bar{\vec c}_i-\vec t$} (d3);
      % zero test of Eve
      \node[below=1.5cm of dmap](zmap){$\rightsquigarrow$};
      \node[s-eve,left=2.75cm of zmap](z0){$\loc$};
      \node[right=of z0](z1){$\loc'$};
      \node[right=1.25cm of zmap,s-eve](z2){$\loc$};
      \node[right=of z2,s-adam-small](z3){};
      \node[above right=.8 and 1.1 of z3,s-eve-small](z4){};
      \node[below right=.8 and 1.1 of z3,inner sep=0pt](z5){$\loc'$};
      \node[right=1.8 of z4](z6){$\loc_\mathtt{halt}$};
      \path[arrow,every node/.style={font=\footnotesize}]
      (z0) edge node{$i\eqby?0$} (z1)
      (z2) edge node{$-\vec t$} (z3)
      (z3) edge node{$\vec 0$} (z4)
      (z3) edge[swap] node{$\vec 0$} (z5)
      (z4) edge node{$-2^{2^n}\cdot\bar{\vec c}_i$} (z6);
      % action of Adam
      \node[below=1.5cm of zmap](amap){$\rightsquigarrow$};
      \node[s-adam,left=2.75cm of amap](a0){$\loc$};
      \node[right=of a0](a1){$\loc'$};
      \node[right=1.25cm of amap,s-adam](a2){$\loc$};
      \node[right=of a2,s-eve-small](a3){};
      \node[right=of a3](a4){$\loc'$};
      \path[arrow,every node/.style={font=\footnotesize}]
      (a0) edge node{$\vec 0$} (a1)
      (a2) edge node{$\vec 0$} (a3)
      (a3) edge node{$-\vec t$} (a4);
    \end{tikzpicture}
    \caption{Schema of the reduction to
      "non-termination" in the proof of \cref{11-th:avag-hard}.}\label{11-fig:lipton-nonterm}
  \end{figure}

  We still need to extend our initialisation phase.  It suffices for
  this to implement a gadget for $\dd$-"meta-increments"
  $\loc\mstep{2^{\dd 2^j}\cdot\vec c}\loc'$ and $\dd$-"meta-decrements"
  $\loc\mstep{-2^{\dd 2^j}\cdot\vec c}\loc'$; this is the same argument as
  in Lipton's construction, with a base case $\loc\mstep{2^\dd}\loc'$
  for $j=0$.  Then we initialise our time budget through $|\Loc|$
  successive $\dd$-"meta-increments"
  $\loc\mstep{2^{\dd 2^n}\cdot\vec t}\loc'$.
  %%\end{scope}
\end{proof}

The proof of \cref{11-th:avag-hard} relies crucially on the fact that the
dimension is not fixed: although $\dd\geq 3$ suffices in the
"alternating Minsky machine", we need $O(|\?M|)$ additional counters
to carry out the reduction.  A separate argument is thus needed in
order to match the \EXP\ upper bound of \cref{11-th:avag-easy} in fixed
dimension.

\begin{theorem}[Fixed-dimensional coverability and non-termination asymmetric vector games are \EXP-hard]
\label{11-th:avag-two}
  "Coverability", "non-termination", and "parity@parity vector game"
  "asymmetric" "vector games" with "given initial credit" are
  \EXP-hard in dimension $\dd\geq 2$.
\end{theorem}
\begin{proof}
  We exhibit a reduction from "countdown games" with "given initial
  credit", which are \EXP-complete by \cref{11-th:countdown-given}.
  Consider an instance of a "configuration reachability" countdown
  game: a "countdown system"
  $\?V=(\Loc,\Act,\Loc_\mEve,\Loc_\mAdam,1)$ with initial
  configuration $\loc_0(n_0)$ and target
  configuration~$\smiley(0)$---as seen in the proof
  of \cref{11-th:countdown-given}, we can indeed assume that the target
  credit is zero; we will also assume that \Eve\ controls~$\smiley$ and
  that the only action available in~$\smiley$ is
  $\smiley\step{-1}\smiley$.  We construct an "asymmetric" "vector
  system" $\?V'$ of dimension~2 such that \Eve\ can ensure
  reaching~$\smiley(0,n_0)$ from $\loc_0(n_0,0)$ in~$\?V'$ if and only
  if she could ensure reaching $\smiley(0)$ from $\loc_0(n_0)$
  in~$\?V$.  The translation is depicted in \cref{11-fig:dim2}.
  
  \begin{figure}[htbp]
    \centering
    \begin{tikzpicture}[auto,on grid,node distance=1.5cm]
      \node(to){$\mapsto$};
      \node[anchor=east,left=2.5cm of to](mm){"countdown system"};
      \node[anchor=west,right=2.5cm of to](mwg){"asymmetric vector system"};
      % action of Eve
      \node[below=.7cm of to](imap){$\rightsquigarrow$};
      \node[s-eve,left=2.75cm of imap](i0){$\loc$};
      \node[right=of i0](i1){$\loc'$};
      \node[right=1.25cm of imap,s-eve](i2){$\loc$};
      \node[right=1.8 of i2](i3){$\loc'$};      
      \path[arrow,every node/.style={font=\footnotesize,inner sep=1pt}]
      (i0) edge node{$-n$} (i1)
      (i2) edge node{$-n,n$} (i3);
      % minimal action of Adam
      \node[below=1cm of imap](dmap){$\rightsquigarrow$};
      \node[s-adam,left=2.75cm of dmap](d0){$\loc$};
      \node[right=of d0](d1){$\loc'$};
      \node[below=.5 of d0]{$n=\min\{n'\mid\exists\loc''\in\Loc\mathbin.\loc\step{-n'}\loc''\in\Act\}$};
      \node[right=1.25cm of dmap,s-adam](d2){$\loc$};
      \node[right=1.8 of d2,s-eve-small](d3){};
      \node[right=1.8 of d3](d4){$\loc'$};
      \path[arrow,every node/.style={font=\footnotesize,inner sep=1pt}]
      (d0) edge node{$-n$} (d1)
      (d2) edge node{$0,0$} (d3)
      (d3) edge node{$-n,n$} (d4);
      % non-minimal action of Adam
      \node[below=1.5cm of dmap](zmap){$\rightsquigarrow$};
      \node[s-adam,left=2.75cm of zmap](z0){$\loc$};
      \node[right=of z0](z1){$\loc'$};
      \node[below=.5 of z0]{$n\neq\min\{n'\mid\exists\loc''\in\Loc\mathbin.\loc\step{-n'}\loc''\in\Act\}$};
      \node[right=1.25cm of zmap,s-adam](z2){$\loc$};
      \node[right=of z2,s-eve-small](z3){};
      \node[above right=.8 and 2.1 of z3](z4){$\loc'$};
      \node[below right=.8 and 2.1 of z3,s-eve](z5){$\smiley$};
      \path[arrow,every node/.style={font=\footnotesize,inner sep=1pt}]
      (z0) edge node{$-n$} (z1)
      (z2) edge node{$0,0$} (z3)
      (z3) edge[bend left=8] node{$-n,n$} (z4)
      (z3) edge[swap,bend right=8] node{$n_0-n+1,-n_0+n-1$} (z5)
      (z5) edge[loop above] node{$-1,1$} ()
      (z5) edge[loop right] node{$\,0,0$} ();
    \end{tikzpicture}
    \caption{Schema of the reduction in the proof
    of \cref{11-th:avag-two}.}\label{11-fig:dim2}
  \end{figure}
    
  The idea behind this translation is that a configuration $\loc(c)$
  of~$\?V$ is simulated by a configuration $\loc(c,n_0-c)$ in~$\?V'$.
  The crucial point is how to handle \Adam's moves.  In a
  configuration $\loc(c,n_0-c)$ with $\loc\in\Loc_\mAdam$, according
  to the "natural semantics" of $\?V$, \Adam\ should be able to
  simulate an action $\loc\step{-n}\loc'$ if and only if $c\geq n$.
  Observe that otherwise if $c<n$ and thus $n_0-c>n_0-n$, \Eve\ can
  play to reach~$\smiley$ and win immediately.  An exception to the
  above is if $n$ is minimal among the decrements in~$\loc$, because
  according to the "natural semantics" of~$\?V$, if $c<n$ there should
  be an edge to the "sink", and this is handled in the second line
  of \cref{11-fig:dim2}.

  Then \Eve\ can reach $\smiley(0,n_0)$ if and only if she can cover
  $\smiley(0,n_0)$, if and only if she can avoid the "sink" thanks to
  the self loop $\smiley\step{0,0}\smiley$.  This
  shows the \EXP-hardness of "coverability" and "non-termination"
  "asymmetric" "vector games" in dimension~two; the hardness of
  "parity@parity vector game" follows
  from \cref{11-rk:cov2parity,11-rk:nonterm2parity}.
\end{proof}

  
\subsection{Dimension One}
\label{11-sec:mono-dim1}
%\TODO{contents of \cref{11-subsec:mono-dim1} depend on what goes into \cref{chap:multiobjective}}

% Local IspellDict: british

\TODO{contents of \cref{11-sec:mono-dim1} depend on what goes into \cref{12-chap:multiobjective}}


\section*{Bibliographic references}
\label{11-sec:references}
We refer to~\cref{2-sec:references} for the role of parity objectives and how they emerged in automata theory as a subclass of Muller objectives.
Another related motivation comes from the works of Emerson, Jutla, and Sistla~\cite{Emerson&Jutla&Sistla:1993},
who showed that solving parity games is linear-time equivalent to the model-checking problem for modal $\mu$-calculus.
This logical formalism is an established tool in program verification, and a common denominator to a wide range of modal, temporal and fixpoint logics used in various fields.

\vskip1em
Let us discuss the progress obtained over the years for each of the three families of algorithms.

\vskip1em
\textit{Value iteration algorithms and separating automata}.
The heart of value iteration algorithms is the value function, which in the context of parity games and related developments for automata
have been studied under the name progress measures or signatures.
They appear naturally in the context of fixed point computations so it is hard to determine who first introduced them.
Streett and Emerson~\cite{Streett&Emerson:1984,Streett&Emerson:1989} defined signatures for the study of the modal $\mu$-calculus,
and Stirling and Walker~\cite{Stirling&Walker:1989} later developped the notion.
Both the proofs of Emerson and Jutla~\cite{Emerson&Jutla:1991} and of Walukiewicz~\cite{Walukiewicz:1996} use signatures to show the positionality of parity games over infinite games.

Jurdzi{\'n}ski~\cite{Jurdzinski:2000} used this notion to give the first value iteration algorithm for parity games, 
with running time $O(m n^{d/2})$.
The algorithm is called ``small progress measures'' and is an instance of the class of value iteration algorithms we construct 
in~\cref{3-sec:value_iteration} by considering the universal tree of size $n^h$.
Bernet, Janin, and Walukiewicz~\cite{Bernet&Janin&Walukiewicz:2002} investigated reductions from parity games to safety games
through the notion of permissive strategies, and constructed a separating automaton\footnote{We note that the general framework of separating automata came later, introduced by Boja{\'n}czyk and Czerwi{\'n}ski~\cite{Bojanczyk&Czerwinski:2018}.} corresponding to the universal tree of size $n^h$.

The new era for parity games started in 2017 when Calude, Jain, Khoussainov, Li, and Stephan~\cite{Calude&Jain&al:2017} constructed a quasipolynomial time algorithm. 
Our presentation follows the technical developments of the subsequent paper by Fearnley, Jain, Schewe, Stephan, and Wojtczak~\cite{Fearnley&Jain&al:2017} which recasts the algorithm as a value iteration algorithm.
Boja{\'n}czyk and Czerwi{\'n}ski~\cite{Bojanczyk&Czerwinski:2018} introduce the separation framework to better understand the original algorithm.

Soon after two other quasipolynomial time algorithms emerged.
Jurdzi{\'n}ski and Lazi{\'c}~\cite{Jurdzinski&Lazic:2017} showed that the small progress measure algorithm can be adapted to a ``succinct progress measure'' algorithm, matching (and slightly improving) the quasipolynomial time complexity.
The presentation using universal tree that we follow in~\cref{3-sec:value_iteration} and an almost matching lower bound on their sizes is due to Fijalkow~\cite{Fijalkow:2018}.
The connection between separating automata and universal trees was shown by Czerwi{\'n}ski, Daviaud, Fijalkow, Jurdzi{\'n}ski, Lazi{\'c}, and Parys~\cite{Czerwinski&Daviaud&al:2018}. 

The third quasipolynomial time algorithm is due to Lehtinen~\cite{Lehtinen:2018}.
The original algorithm has a slightly worse complexity ($n^{O(\log(n))}$ instead of $n^{O(\log(d))}$),
but Parys~\cite{Parys:2020} later improved the construction to (essentially) match the complexity of the previous two algorithms.
Although not explicitly, the algorithm constructs an automaton with similar properties as a separating automaton,
but the automaton is non-deterministic.
Colcombet and Fijalkow~\cite{Colcombet&Fijalkow:2019} revisited the link between separating automata and universal trees
and proposed the notion of good for small games automata, capturing the automaton defined by Lehtinen's algorithm.
The equivalence result between separating automata, good for small games automata, and universal graphs, holds for any positionally determined objective, giving a strong theoretical foundation for the family of value iteration algorithms.

\vskip1em
\textit{Attractor decomposition algorithms}.
The McNaughton Zielonka's algorithm has complexity $O(m n^d)$.
Parys~\cite{Parys:2019} constructed the fourth quasipolynomial time algorithm as an improved take over McNaughton Zielonka's algorithm.
As for Lehtinen's algorithm, the original algorithm has a slightly worse complexity ($n^{O(\log(n))}$ instead of $n^{O(\log(d))}$).
Lehtinen, Schewe, and Wojtczak~\cite{Lehtinen&Schewe&Wojtczak:2019} later improved the construction.
As discussed in~\cref{3-sec:relationships} the complexity of this algorithm is quasipolynomial and of the form $n^{O(\log(d))}$,
but a bit worse than the three previous algorithms since the algorithm is symmetric and has a recursion depth of $d$,
while the value iteration algorithms only consider odd priorities hence replace $d$ by $d/2$.

Jurdzi{\'n}ski and Morvan~\cite{Jurdzinksi&Morvan:2020} constructed a generic McNaughton Zielonka's algorithm parameterised by the choice of two universal trees, one for each player.
\mynote{CONTINUE}


\vskip1em
\textit{Strategy improvement algorithms}.
As we will see in~\cref{4-chap:payoff}, parity games can be reduced to mean payoff games,
so any algorithm for solving mean payoff games can be used for solving parity games.
In particular, the existing strategy improvement algorithm for mean payoff games can be run on parity games. 
V{\"o}ge and Jurdzin{\'n}ski~\cite{Voge&Jurdzinski:2000} introduced the first discrete strategy improvement for parity games,
running in exponential time.
For some time there was some hope that the strategy improvement algorithm, for some well chosen policy on switching edges,
solves parity games in polynomial time.
Friedmann~\cite{Friedmann:2011} cast some serious doubts by constructing numerous exponential lower bounds applying to different variants of the algorithm.
Fearnley~\cite{Fearnley:2017} investigated efficient implementations of the algorithm, focussing on the cost of computing and updating the value function for a given strategy.
Our proof of correctness is original. \mynote{SAY MORE?}

The complexity was reduced to subexponential with randomised algorithms 
by Jurdzin{\'n}ski, Paterson, and Zwick~\cite{Jurdzinski&Paterson&Zwick:2008}.
A natural question is whether there exists a quasipolynomial strategy improvement algorithm; 
as discussed in~\cref{3-sec:relationships} the notion of universal trees cannot be used to achieve this,
and the question remains to this day open.


\markright{Bibliographic Notes}
\ifstandalone\addcontentsline{toc}{section}{Bibliographic Notes}\fi%


% Local IspellDict: british
