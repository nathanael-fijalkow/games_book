\paragraph{Vector Addition Systems with States}
In their one-player version, i.e.\ in "vector addition systems with
states", all the games presented in \cref{11-sec:counters} are decidable.
With "given initial credit", "configuration reachability" is simply
called `reachability' and was first shown decidable by
\citem[Mayr]{Mayr:1981} (with simpler proofs
in~\cite{Kosaraju:1982,Lambert:1992,Leroux:2011}) and recently shown
to be of non-elementary complexity~\cite{Czerwinski&Lasota&Lazic&Leroux&Mazowiecki:2019}.  "Coverability" and
"non-termination" are considerably easier, as they are
\EXPSPACE-complete~\cite{Lipton:1976,Rackoff:1978} and so is
"parity@parity vector game"~\cite{Habermehl:1997}.  With "existential
initial credit", the problems are markedly simpler: "configuration
reachability" becomes \EXPSPACE-complete, while "coverability" is in
\NL\ and "non-termination" and "parity" can be solved in polynomial
time by~\cref{11-thm:zcycle} using linear programming
techniques~\cite{Kosaraju&Sullivan:1988}.

\paragraph{Undecidability of Vector Games}
The undecidability results of \cref{11-sec:undec} are folklore.  One
can find undecidability proofs
in~\cite{Abdulla&Bouajjani&dOrso:2003,Raskin&Samuelides&VanBegin:2005};
"non-termination" was called `deadlock-freedom' by \citem[Raskin et
al.]{Raskin&Samuelides&VanBegin:2005}. "Configuration reachability" is
undecidable even in very restricted cases, like the ""robot games""
of~\citet{Niskanen&Potapov&Reichert:2016}.

\paragraph{Succinct One-Counter Games}
One-dimensional "vector systems" are often called \emph{one-counter
nets} in the literature, by contrast with \emph{one-counter automata}
where zero tests are allowed.  The \EXPSPACE-completeness of "succinct
one-counter games" was shown by~\citem[Hunter]{Hunter:2015}.  "Countdown games"
were originally defined with "given initial credit" and a "zero
reachability" objective, and shown
\EXP-complete in \cite{Jurdzinski&Laroussinie&Sproston:2008}; see also
\citet{Kiefer:2013} for a variant called ""hit-or-run games"".  The
hardness proofs for \cref{11-th:countdown-given,11-th:countdown-exist} are
adapted from~\citet{Jancar&Osicka&Sawa:2018}, where "countdown games"
with "existential initial credit" were first introduced.

\paragraph{Asymmetric Vector Games}
The "asymmetric" "vector games" of \cref{11-sec:avag} appear
under many guises in the litterature: as `and-branching' "vector
addition systems with states"
in~\cite{Lincoln&Mitchell&Scedrov&Shankar:1992}, as `vector games'
in~\cite{Kanovich:1995}, as `B-games'
in~\cite{Raskin&Samuelides&VanBegin:2005}, as `single sided' vector
addition games in~\cite{Abdulla&al:2013}, and as `alternating'
"vector addition systems with states" in~\cite{Courtois&Schmitz:2014}.

The undecidability of "configuration reachability" shown
in~\cref{11-sec:reach} was already proven by \citem[Lincoln et
al.]{Lincoln&Mitchell&Scedrov&Shankar:1992} and used to show the
undecidability of propositional linear logic;
\citem[Kanovich]{Kanovich:1995,Kanovich:2016} refines this result to
show the undecidability of the $(!,\oplus)$-Horn fragment of linear
logic.  Similar proof ideas are used for Boolean BI and separation
logic in~\cite{Larchey&Galmiche:2013,Brotherston&Kanovich:2014}.

\paragraph{Asymmetric Monotone Vector Games}
The notion of "asymmetric" infinite games over a "well-quasi-ordered"
"arena" constitutes a natural extension of the notion of
""well-structured systems"" of
\citet{Abdulla&Cerans&Jonsson&Tsay:2000} and
\citet{Finkel&Schnoebelen:2001}, and was undertaken
in~\cite{Abdulla&Bouajjani&dOrso:2003,Raskin&Samuelides&VanBegin:2005}.
The decidability of "coverability" and "non-termination" through "wqo"
arguments like those of \cref{11-fact:pareto-cov} was shown
by~\citem[Raskin et al.]{Raskin&Samuelides&VanBegin:2005}.  More
advanced "wqo" techniques were needed for the first decidability proof
of "parity@parity vector game" in~\cite{Abdulla&al:2013}.  See
also \cite{Schmitz&Schnoebelen:2012} for more on the algorithmic uses of
"wqos".

By analysing the attractor computation of \cref{11-sec:attr}, one can
show that \cref{11-algo:cov} works in \kEXP[2], thus matching the
optimal upper bound from~\cref{11-th:avag-easy}: this can be done using
the Rackoff-style argument of \citet{Courtois&Schmitz:2014} and the
analysis of \citet{Bozzelli&Ganty:2011}, or by a direct analysis of the
attractor computation algorithm~\cite{Lazic&Schmitz:2019}.

\paragraph{Energy Games}
\AP An alternative take on "energy games" is to see a "vector system"
$\?V=(\Loc,\Act,\Loc_\mEve,\Loc_\mAdam,\dd)$ as a finite "arena" with
edges $\loc\step{\vec u}\loc'$ coloured by $\vec u$, thus with set of
colours $C\eqdef\+Z^\dd$.  For an initial credit $\vec v_0\in\+N^\dd$
and $1\leq i\leq\dd$, the associated ""energy objective"" is then
defined as\todoquestion{is that the right place for this?}
\begin{equation*}
  \mathsf{Energy}_{\vec v_0}(i)\eqdef\left\{\pi\in E^\omega\;\middle|\;\forall
  n\in\+N\mathbin.\left(\vec v_0(i)+\sum_{0\leq j\leq n}c(\pi)(i)\right)\geq 0\right\}\;,
\end{equation*}%\todo{we should decide: is $\Omega\subseteq C^\omega$,
%  or $\Omega\subseteq E^\omega$?}
that is, $\pi$ is winning if the successive sums of weights on
coordinate~$i$ are always non-negative.
\AP The ""multi-energy objective"" then asks for the "play"~$\pi$ to
belong simultaneously to $\mathsf{Energy}_{\vec v_0}(i)$ for all
$1\leq i\leq\dd$.  This is a multiobjective in the sense of the
forthcoming \cref{12-chap:multiobjective}.  "Multi-energy games" are
equivalent to "non-termination" games played on the arena
$\energy(\?V)$ defined by the "energy semantics".
%% As we are going to see in
%% \cref{11-fact:nrg}, solving a multi-energy game is equivalent to
%% solving a "non-termination" "asymmetric vector game", and solving a
%% multi-energy parity game to solving a "parity@parity vector game"
%% "asymmetric vector game".

\TODO{energy games with weak or hard upper bounds, bounding games}

\TODO{references for energy games}
\cite{Chakrabarti&deAlfaro&Henzinger&Stoelinga:2003,Bouyer&Fahrenberg&Larsen&Markey&Srba:2008}
... in~\cite{Abdulla&al:2013}, where the
relationship with "energy games" was also first observed.  The
equivalence with "mean-payoff games" in dimension~one was first
noticed by~\citem[Bouyer et
al.]{Bouyer&Fahrenberg&Larsen&Markey&Srba:2008}.  A similar connection
in the multi-dimensional case was established
in~\cite{Chatterjee&Doyen&Henzinger&Raskin:2010,Velner&al:2015} and
will be discussed in~\cref{12-chap:multiobjective}.

\paragraph{Complexity} \Cref{11-tbl:cmplx} summarises the complexity
results for "asymmetric vector games".  For the upper bounds with
"existential initial credit" of \cref{11-sec:up-exist}, the existence
of "counterless" winning strategies for \Adam\ was originally shown by
\citem[Br\'azdil et al.]{Brazdil&Jancar&Kucera:2010} in the case of
"non-termination games"; the proof of \cref{11-lem:counterless} is a
straightforward adaptation using
ideas from \cite{Chatterjee&Doyen:2012} to handle "parities@parity
vector game".  An alternative proof through "bounding games" is
presented in~\cite{Colcombet&Jurdzinski&Lazic&Schmitz:2017}.

The \coNP\ upper of \cref{11-th:exist-easy} was shown soon after
Br\'azdil et al.'s work by
\citem[Chatterjee et al.]{Chatterjee&Doyen&Henzinger&Raskin:2010} in
the case of "non-termination games".  The extension of
\cref{11-th:exist-easy} to "parity@parity vector game" was shown
by~\cite{Chatterjee&Randour&Raskin:2014} by a reduction from
"parity@parity vector games" to "non-termination games" somewhat
reminiscent of \cref{?}.  The proof of
\cref{11-th:exist-easy} takes a slightly different approach using
\cref{11-lem:zcycle} for finding non-negative cycles, which is a
trivial adaptation of a result by \citem[Kosaraju and
Sullivan]{Kosaraju&Sullivan:1988}.  The pseudo-polynomial bound
of~\cref{11-cor:exist-pseudop} is taken
from~\cite{Colcombet&Jurdzinski&Lazic&Schmitz:2017}.

For the upper bounds with "given initial credit" of
\cref{11-sec:up-given}, regarding "coverability", the \kEXP[2] upper
bound of \cref{11-th:avag-easy} was first shown by~\citem[Courtois and
Schmitz]{Courtois&Schmitz:2014} by adapting Rackoff's technique for
"vector addition systems with states"~\cite{Rackoff:1978}.  Regarding
"non-termination", the first complexity upper bounds were shown
by~\citem[Br\'azdil et al.]{Brazdil&Jancar&Kucera:2010} and were
in \kEXP, thus non-elementary in the size of the input.  Very
roughly, \todo{This is a leftover from a previous write-up}
their argument went as follows: one can extract a pseudo-polynomial
"existential Pareto bound"~$B$ in the one-player case from the proof
of \cref{11-thm:zcycle}, from which the proof of \cref{11-lem:counterless}
yields a $2^{|\Act|}(B+|\Loc|)$ "existential Pareto bound" in the
two-player case, and finally by arguments similar to \cref{?} a tower
of~$\dd$ exponentials on the "given initial credit" problem.  The
two-dimensional case with a unary encoding was shown a bit later to be
in~\P\ by~\citem[Chaloupka]{Chaloupka:2013}.  Finally, a
matching \kEXP[2] upper bound (and pseudo-polynomial in any fixed
dimension) was obtained by~\citem[Jurdzi\'nski et
al.]{Jurdzinski&Lazic&Schmitz:2015}.  Regarding "parity@parity vector
game", \citem[Jan\v{c}ar]{Jancar:2015} showed how to obtain
non-elementary upper bounds by reducing to the case
of~\citet{Brazdil&Jancar&Kucera:2010}, before a tight \kEXP[2] upper
bound (and pseudo-polynomial in fixed dimension with a fixed number of
priorities) was shown
in~\cite{Colcombet&Jurdzinski&Lazic&Schmitz:2017}.

The \coNP\ hardness with "existential initial credit" in
\cref{11-th:exist-hard} originates from
\citet{Chatterjee&Doyen&Henzinger&Raskin:2010}.  The \kEXP[2]-hardness
of both "coverability" and "non-termination" games with "given initial
credit" from \cref{11-th:avag-hard} was shown
in~\cite{Courtois&Schmitz:2014} by adapting Lipton's construction for
"vector addition systems with states"~\cite{Lipton:1976}; similar
proofs can be found for instance
in~\cite{Demri&Jurdzinski&Lachish&Lazic:2012,Berard&Haddad&Sassolas&Sznajder:2012}.
The hardness for \EXP-hardness in dimension two was first shown by
\cite{Fahrenberg&Juhl&Larsen&Srba:2011}; the direct proof in
\cref{11-th:avag-two} by a reduction from "countdown games" was suggested
by~\citet{Mazowiecki&Perez:2017}.

The $\NP\cap\coNP$ upper bounds in dimension~one from
\cref{11-sec:mono-dim1} are due to \citem[Bouyer et
al.]{Bouyer&Fahrenberg&Larsen&Markey&Srba:2008} for "given
initial credit" and \citem[Chatterjee
and Doyen]{Chatterjee&Doyen:2012} for "existential initial credit".

\paragraph{Some Applications}
Besides their many algorithmic applications for solving various types
of games, "vector games" have been employed in several fields to prove
decidability and complexity results, for instance for linear,
relevance, or separation
logics~\cite{Lincoln&Mitchell&Scedrov&Shankar:1992,Kanovich:1995,Urquhart:1999,Larchey&Galmiche:2013,Brotherston&Kanovich:2014,Kanovich:2016},
resource-bounded logics~\cite{Alechina&al:2018}, simulation and
bisimulation
problems~\cite{Kiefer:2013,Abdulla&al:2013,Courtois&Schmitz:2014,Jancar&Osicka&Sawa:2018},
orchestration synthesis~\cite{DeGiacomo&Vardi&Felli&Alechina&Logan:2018},
and model-checking probabilistic timed
automata~\cite{Jurdzinski&Laroussinie&Sproston:2008}.

%\afterpage{%
%    \clearpage% Flush earlier floats (otherwise order might not be correct)
    %\thispagestyle{empty}% empty page style (?)
    \begin{landscape}% Landscape page
      \centering
      \captionof{table}{\label{12-tbl-cmplx}The complexity of
        "asymmetric vector games".}
  \bigskip

  \begingroup
  \catcode`\&=12
  \catcode`!=4
  \ifstandalone
  \setlength{\tabcolsep}{3pt}
  \begin{tabular}{p{11em}cccc}
  \toprule 
  !!\multicolumn{3}{c}{Dimension}\\
  \cmidrule(l){3-5}
  Game ! Initial credit ! Fixed $\dd=1$ ! Fixed $\dd\geq 2$ ! Arbitrary\\
  \midrule
    configuration reachability%
    ! both
    ! \EXPSPACE-complete            
    !\multicolumn{2}{c}{undecidable}
  \\[-.5em]
    % reachability                                     
    ! % both 
    ! {\tiny\cref{12-th-asym-dim1}} 
    !\multicolumn{2}{c}{\tiny\cref{12-th-asym-undec}~\cite{Lincoln&Mitchell&Scedrov&Shankar:1992}}%
  \\
  \addlinespace
  \multirow{3}{*}{"coverability"}               
    ! "existential"
    ! \multicolumn{3}{c}{\P-complete}
  \\[-.5em]
    % coverability
    ! % existential 
    ! \multicolumn{3}{c}{\tiny\cref{12-cov-exist-P}}
  \\
    % coverability
    ! "given"      
    ! in $\NP\cap\coNP$             
    ! \EXP-complete
    ! \kEXP[2]-complete
  \\[-.5em]
    % coverability
    ! % given
    ! {\tiny\cref{?}}                
    ! {\tiny\cref{12-avag-two,12-avag-easy}}
    ! {\tiny\cref{12-avag-hard,12-avag-easy}}
  \\[-.7em]
    % coverability
    ! % given
    !
    !{\tiny\cite{Fahrenberg&Juhl&Larsen&Srba:2011,Courtois&Schmitz:2014}}
    !{\tiny\cite{Courtois&Schmitz:2014}}
  \\
  \addlinespace
  \multirow{3}{*}{"non-termination"}
    ! "existential"
    ! in $\NP\cap\coNP$             
    ! in \coNP
    ! \coNP-complete
  \\[-.5em]
    % non-termination
    ! % existential
    ! {\tiny\cite{Chatterjee&Doyen:2012}?}
    !
    !{\tiny\cref{12-exist-hard,12-exist-easy}~\cite{Chatterjee&Doyen&Henzinger&Raskin:2010}}
  \\
    % non-termination
    ! "given"
    ! in $\NP\cap\coNP$
    ! \EXP-complete
    ! \kEXP[2]-complete
  \\[-.5em]
    % non-termination
    ! % given
    ! {\tiny\cite{Bouyer&Fahrenberg&Larsen&Markey&Srba:2008}}
    ! {\tiny\cref{12-avag-two,12-avag-easy}}
    ! {\tiny\cref{12-avag-hard,12-avag-easy}}
  \\[-.7em]
    % non-termination
    ! % given
    !
    !{\tiny\cite{Fahrenberg&Juhl&Larsen&Srba:2011,Jurdzinski&Lazic&Schmitz:2015}}
    !{\tiny\cite{Courtois&Schmitz:2014,Jurdzinski&Lazic&Schmitz:2015}}
  \\
  \addlinespace
  \multirow{3}{*}{"parity@parity vector game"}
    ! "existential"
    ! in $\NP\cap\coNP$
    ! in \coNP
    ! \coNP-complete
  \\[-.5em]
    % parity
    ! % existential
    ! {\tiny\cite{Chatterjee&Doyen:2012}}
    ! 
    !{\tiny\cref{12-exist-hard,12-exist-easy}~\cite{Chatterjee&Doyen&Henzinger&Raskin:2010,Chatterjee&Randour&Raskin:2014}}
  \\
    % parity                                 
    ! "given"
    ! in \EXP?%in $\NP\cap\coNP$ 
    ! \EXP-complete
    ! \kEXP[2]-complete
  \\[-.5em]
    % parity
    ! % given
    ! %{\tiny\cref{?} \cite{Chatterjee&Doyen:2012}}
    ! {\tiny\cref{12-avag-two,12-avag-easy}}
    ! {\tiny\cref{12-avag-hard,12-avag-easy}}
  \\[-.7em]
    % parity
    ! % given
    !
    !{\tiny\cite{Fahrenberg&Juhl&Larsen&Srba:2011,Colcombet&Jurdzinski&Lazic&Schmitz:2017}}
    !{\tiny\cite{Courtois&Schmitz:2014,Colcombet&Jurdzinski&Lazic&Schmitz:2017}}
  \\
  \bottomrule  
  \end{tabular}
  \else
  \setlength{\tabcolsep}{7pt}
  \begin{tabular}{p{12em}cccc}
  \toprule 
  !!\multicolumn{3}{c}{Dimension}\\
  \cmidrule(l){3-5}
  Game ! Initial credit ! Fixed $\dd=1$ ! Fixed $\dd\geq 2$ ! Arbitrary\\
  \midrule
  configuration reachability 
  ! -
  ! \EXPSPACE-complete
  !\multicolumn{2}{c}{undecidable}
  \\[-.5em]
  % reachability
  ! % -
  ! {\tiny\cref{12-th-asym-dim1}} 
  !\multicolumn{2}{c}{{\tiny\cref{12-th-asym-undec}~\cite{Lincoln&Mitchell&Scedrov&Shankar:1992}}}\\
  \addlinespace
  \multirow{3}{*}{"coverability"}
  ! "existential"
  ! \multicolumn{3}{c}{\P-complete}
  \\[-.5em]
  % coverability
  ! % existential
  ! \multicolumn{3}{c}{\tiny\cref{12-cov-exist-P}}
  \\
  % coverability
  ! "given"
  ! in $\NP\cap\coNP$ 
  ! \EXP-complete
  ! \kEXP[2]-complete
  \\[-.5em]
  % coverability
  ! % given
  ! {\tiny\cref{?}}                
  !
    {\tiny\cref{12-avag-two,12-avag-easy}~\cite{Fahrenberg&Juhl&Larsen&Srba:2011,Courtois&Schmitz:2014}}
  ! {\tiny\cref{12-avag-hard,12-avag-easy} \cite{Courtois&Schmitz:2014}}\\
  \addlinespace
  \multirow{3}{*}{"non-termination"}
  ! "existential"
  ! in $\NP\cap\coNP$
  ! in \coNP
  ! \coNP-complete
  \\[-.5em]
  % non-termination
  ! % existential
  ! {\tiny\cite{Chatterjee&Doyen:2012}?}
  !
  !{\tiny\cref{12-exist-hard,12-exist-easy}~\cite{Chatterjee&Doyen&Henzinger&Raskin:2010}}
  \\
  % non-termination  
  ! "given"
  ! in $\NP\cap\coNP$ 
  ! \EXP-complete
  ! \kEXP[2]-complete
  \\[-.5em]
  % non-termination
  ! % given
  ! {\tiny\cite{Bouyer&Fahrenberg&Larsen&Markey&Srba:2008}}
  ! {\tiny\cref{12-avag-two,12-avag-easy}~\cite{Fahrenberg&Juhl&Larsen&Srba:2011,Jurdzinski&Lazic&Schmitz:2015}}
  !{\tiny\cref{12-avag-hard,12-avag-easy}~\cite{Courtois&Schmitz:2014,Jurdzinski&Lazic&Schmitz:2015}}
  \\
  \addlinespace
  \multirow{3}{*}{"parity@parity vector game"}
  ! "existential"
  ! in $\NP\cap\coNP$
  ! in \coNP 
  ! \coNP-complete
  \\[-.5em]
  % parity
  ! % existential
  ! {\tiny\cite{Chatterjee&Doyen:2012}?}
  ! 
  ! {\tiny\cref{12-exist-hard,12-exist-easy}~\cite{Chatterjee&Doyen&Henzinger&Raskin:2010,Chatterjee&Randour&Raskin:2014}}
  \\
  % parity
  ! "given"
  ! in \EXP?%in $\NP\cap\coNP$
  ! \EXP-complete
  ! \kEXP[2]-complete
  \\[-.5em]
  % parity
  ! % given    
  ! {\tiny}
  ! {\tiny\cref{12-avag-two,12-avag-easy}~\cite{Fahrenberg&Juhl&Larsen&Srba:2011,Colcombet&Jurdzinski&Lazic&Schmitz:2017}} 
  ! {\tiny\cref{12-avag-hard,12-avag-easy}~\cite{Courtois&Schmitz:2014,Colcombet&Jurdzinski&Lazic&Schmitz:2017}}\\
  \bottomrule  
  \end{tabular}
  \fi
  \endgroup
    \end{landscape}
%    \clearpage% Flush page
%}

% Local IspellDict: british 
