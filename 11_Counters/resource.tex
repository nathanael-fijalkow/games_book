"Vector games" are very well suited for reasoning about systems
manipulating discrete resources, modelled as counters.  However, in
the "natural semantics", actions that would deplete some resource,
i.e., that would make some counter go negative, are simply inhibited.
In models of real-world systems monitoring resources like a gas
tank or a battery, a depleted resource would be considered as a system
failure.  In the "energy games" of \cref{11-sec:energy}, those situations
are accordingly considered as winning for \Adam.  Moreover, if we are
modelling systems with a bounded capacity for storing resources, a
counter exceeding some bound might also be considered as a failure,
which will be considered with "bounding games" in \cref{11-sec:bounding}.

These resource-conscious games can be seen as providing alternative
semantics for "vector systems".  They will also be instrumental in
establishing complexity upper bounds for "monotonic" "asymmetric vector
games" later in \cref{11-sec:complexity}, and are strongly related to
"multidimensional" "mean-payoff" games, as will be explained in
\cref{12-sec:MPEG} of \cref{12-chap:multiobjective}.

\subsection{Energy Semantics}
\label{11-sec:energy}

"Energy games" model systems where the depletion of a resource
allows \Adam\ to win.  This is captured by an ""energy semantics""
$\energy(\?V)\eqdef(V,E_\+E,\VE,\VA)$ associated with a "vector
system" $\?V$: we let as before
$V\eqdef(\Loc\times\+N^\dd)\uplus\{\sink\}$, but define instead
\begin{align*}
  E_\+E&\eqdef \{(\loc(\vec v), \loc'(\vec v+\vec u)\mid
         \loc\step{\vec u}\loc'\in\Act\text{
      and }\vec v+\vec u\geq\vec 0\}\\
    &\:\cup\:\{(\loc(\vec v),\sink)\mid\forall\loc\step{\vec
      u}\loc'\in\Act\mathbin.\vec v+\vec u\not\geq\vec 0\}
    \cup\{(\sink,\sink)\}\;.
\end{align*}
In the "energy semantics", moves that would result in a negative
component lead to the "sink" instead of being inhibited.

\begin{example}[Energy semantics]
\label{11-ex:nrg}
  \Cref{11-fig:nrg} illustrates the "energy semantics" of the vector
  system depicted in~\cref{11-fig:mwg} on \cpageref{11-fig:mwg}.  Observe that,
  by contrast with the "natural semantics" of the same system depicted
  in \cref{11-fig:sem}, all the configurations $\loc'(0,n)$ controlled
  by \Adam\ can now move to the "sink".
\end{example}
\begin{figure}[thbp]
  \centering\scalebox{.77}{
  \begin{tikzpicture}[auto,on grid,node distance=2.5cm]
    \draw[step=1,lightgray!50,dotted] (-5.7,0) grid (5.7,3.8);
    \draw[color=white](0,-.3) -- (0,3.8);
    \node at (0,3.9) (sink) {\boldmath$\sink$};
    \draw[step=1,lightgray!50] (1,0) grid (5.5,3.5);
    \draw[step=1,lightgray!50] (-1,0) grid (-5.5,3.5);
    \node at (0,0)[lightgray,font=\scriptsize,fill=white] {0};
    \node at (0,1)[lightgray,font=\scriptsize,fill=white] {1};
    \node at (0,2)[lightgray,font=\scriptsize,fill=white] {2};
    \node at (0,3)[lightgray,font=\scriptsize,fill=white] {3};
    \node at (1,3.9)[lightgray,font=\scriptsize,fill=white] {0};
    \node at (2,3.9)[lightgray,font=\scriptsize,fill=white] {1};
    \node at (3,3.9)[lightgray,font=\scriptsize,fill=white] {2};
    \node at (4,3.9)[lightgray,font=\scriptsize,fill=white] {3};
    \node at (5,3.9)[lightgray,font=\scriptsize,fill=white] {4};
    \node at (-1,3.9)[lightgray,font=\scriptsize,fill=white] {0};
    \node at (-2,3.9)[lightgray,font=\scriptsize,fill=white] {1};
    \node at (-3,3.9)[lightgray,font=\scriptsize,fill=white] {2};
    \node at (-4,3.9)[lightgray,font=\scriptsize,fill=white] {3};
    \node at (-5,3.9)[lightgray,font=\scriptsize,fill=white] {4};
    \node at (1,0)[s-eve-small] (e00) {};
    \node at (1,1)[s-adam-small](a01){};
    \node at (1,2)[s-eve-small] (e02){};
    \node at (1,3)[s-adam-small](a03){};
    \node at (2,0)[s-adam-small](a10){};
    \node at (2,1)[s-eve-small] (e11){};
    \node at (2,2)[s-adam-small](a12){};
    \node at (2,3)[s-eve-small] (e13){};
    \node at (3,0)[s-eve-small] (e20){};
    \node at (3,1)[s-adam-small](a21){};
    \node at (3,2)[s-eve-small] (e22){};
    \node at (3,3)[s-adam-small](a23){};
    \node at (4,0)[s-adam-small](a30){};
    \node at (4,1)[s-eve-small] (e31){};
    \node at (4,2)[s-adam-small](a32){};
    \node at (4,3)[s-eve-small] (e33){};
    \node at (5,0)[s-eve-small] (e40){};
    \node at (5,1)[s-adam-small](a41){};
    \node at (5,2)[s-eve-small] (e42){};
    \node at (5,3)[s-adam-small](a43){};
    \node at (-1,0)[s-adam-small](a00){};
    \node at (-1,1)[s-eve-small] (e01){};
    \node at (-1,2)[s-adam-small](a02){};
    \node at (-1,3)[s-eve-small] (e03){};
    \node at (-2,0)[s-eve-small] (e10){};
    \node at (-2,1)[s-adam-small](a11){};
    \node at (-2,2)[s-eve-small] (e12){};
    \node at (-2,3)[s-adam-small](a13){};
    \node at (-3,0)[s-adam-small](a20){};
    \node at (-3,1)[s-eve-small] (e21){};
    \node at (-3,2)[s-adam-small](a22){};
    \node at (-3,3)[s-eve-small] (e23){};
    \node at (-4,0)[s-eve-small] (e30){};
    \node at (-4,1)[s-adam-small](a31){};
    \node at (-4,2)[s-eve-small] (e32){};
    \node at (-4,3)[s-adam-small](a33){};
    \node at (-5,0)[s-adam-small](a40){};
    \node at (-5,1)[s-eve-small] (e41){};
    \node at (-5,2)[s-adam-small](a42){};
    \node at (-5,3)[s-eve-small] (e43){};
    \path[arrow] % l, -1,-1, l
    (e11) edge (e00)
    (e22) edge (e11)
    (e31) edge (e20)
    (e32) edge (e21)
    (e21) edge (e10)
    (e12) edge (e01)
    (e23) edge (e12)
    (e33) edge (e22)
    (e13) edge (e02)
    (e43) edge (e32)
    (e42) edge (e31)
    (e41) edge (e30);
    \path[arrow] % l, -1,0, l'
    (e11) edge (a01)
    (e20) edge (a10)
    (e22) edge (a12)
    (e31) edge (a21)
    (e32) edge (a22)
    (e21) edge (a11)
    (e12) edge (a02)
    (e30) edge (a20)
    (e10) edge (a00)
    (e13) edge (a03)
    (e23) edge (a13)
    (e33) edge (a23)
    (e43) edge (a33)
    (e42) edge (a32)
    (e41) edge (a31)
    (e40) edge (a30);
    \path[arrow] % l', -1,0, l
    (a11) edge (e01)
    (a20) edge (e10)
    (a22) edge (e12)
    (a31) edge (e21)
    (a32) edge (e22)
    (a21) edge (e11)
    (a12) edge (e02)
    (a30) edge (e20)
    (a10) edge (e00)
    (a33) edge (e23)
    (a23) edge (e13)
    (a13) edge (e03)
    (a43) edge (e33)
    (a42) edge (e32)
    (a41) edge (e31)
    (a40) edge (e30);
    \path[arrow] % l', 2,1, l
    (a01) edge (e22)
    (a10) edge (e31)
    (a11) edge (e32)
    (a00) edge (e21)
    (a02) edge (e23)
    (a12) edge (e33)
    (a22) edge (e43)
    (a21) edge (e42)
    (a20) edge (e41);
    \path[arrow] % dotted to Eve
    (-5.5,3.5) edge (e43)
    (5.5,2.5) edge (e42)
    (2.5,3.5) edge (e13)
    (5.5,0.5) edge (e40)
    (-5.5,1.5) edge (e41)
    (-3.5,3.5) edge (e23)
    (-1.5,3.5) edge (e03)
    (4.5,3.5) edge (e33)
    (5.5,0) edge (e40)
    (5.5,2) edge (e42)
    (-5.5,1) edge (e41)
    (-5.5,3) edge (e43);
    \path[dotted]
    (-5.7,3.7) edge (-5.5,3.5)
    (5.7,2.7) edge (5.5,2.5)
    (2.7,3.7) edge (2.5,3.5)
    (5.7,0.7) edge (5.5,0.5)
    (-3.7,3.7) edge (-3.5,3.5)
    (-1.7,3.7) edge (-1.5,3.5)
    (4.7,3.7) edge (4.5,3.5)
    (-5.7,1.7) edge (-5.5,1.5)
    (5.75,0) edge (5.5,0)
    (5.75,2) edge (5.5,2)
    (-5.75,1) edge (-5.5,1)
    (-5.75,3) edge (-5.5,3);
    \path[arrow]
    (5.5,1) edge (a41)
    (-5.5,2) edge (a42)
    (-5.5,0) edge (a40)
    (5.5,3) edge (a43);
    \path[dotted]
    (5.75,1) edge (5.5,1)
    (-5.75,2) edge (-5.5,2)
    (-5.75,0) edge (-5.5,0)
    (5.75,3) edge (5.5,3);
    \path[-]
    (a30) edge (5.5,.75)
    (a32) edge (5.5,2.75)
    (a31) edge (-5.5,1.75)
    (a23) edge (4,3.5)
    (a03) edge (2,3.5)
    (a13) edge (-3,3.5)
    (a33) edge (-5,3.5)
    (a43) edge (5.5,3.25)
    (a41) edge (5.5,1.25)
    (a40) edge (-5.5,0.25)
    (a42) edge (-5.5,2.25);
    \path[dotted]
    (5.5,.75) edge (5.8,.9)
    (5.5,2.75) edge (5.8,2.9)
    (-5.5,1.75) edge (-5.8,1.9)
    (4,3.5) edge (4.4,3.7)
    (2,3.5) edge (2.4,3.7)
    (-3,3.5) edge (-3.4,3.7)
    (-5,3.5) edge (-5.4,3.7)
    (5.5,3.25) edge (5.8,3.4)
    (5.5,1.25) edge (5.8,1.4)
    (-5.5,.25) edge (-5.8,0.4)
    (-5.5,2.25) edge (-5.8,2.4);
    \path[arrow]
    (sink) edge[loop left] ()
    (e00) edge[bend left=8] (sink)
    (e01) edge[bend right=8] (sink)
    (e02) edge[bend left=8] (sink)
    (e03) edge[bend right=8] (sink)
    (a00) edge[bend right=8] (sink)
    (a01) edge[bend left=8] (sink)
    (a02) edge[bend right=8] (sink)
    (a03) edge[bend left=8] (sink);
  \end{tikzpicture}}
  \caption{The "energy semantics" of the
    "vector system" of \cref{11-fig:mwg}: a circle (resp.\
    a square) at position $(i,j)$ of the grid denotes a configuration
    $\loc(i,j)$ (resp.\ $\loc'(i,j)$) controlled by~\Eve\ (resp.\
    \Adam).}\label{11-fig:nrg}
\end{figure}

Given a "colouring" $\col{:}\,E\to C$ and an objective~$\Omega$, we
call the resulting game $(\energy(\?V),\col,\Omega)$ an ""energy
game"".  In particular, we shall speak of "configuration
reachability", "coverability", "non-termination", and "parity@parity
vector game" "energy games" when replacing $\natural(\?V)$ by
$\energy(\?V)$ in \crefrange{11-pb:reach}{11-pb:parity}; the
"existential initial credit" variants are defined similarly.

\begin{example}[Energy games]
\label{11-ex:cov-nrg}
  Consider the target configuration $\loc(2,2)$ in
  \cref{11-fig:mwg,11-fig:nrg}.  \Eve's "winning region" in the
  "configuration reachability" "energy game" is
  $\WE=\{\loc(n+2,n+2)\mid n\in\+N\}$, displayed on the left in
  \cref{11-fig:cov-nrg}.  In the "coverability" "energy game", \Eve's
  "winning region" is
  $\WE=\{\loc(m+2,n+2),\loc'(m+3,n+2)\mid m,n\in\+N\}$ displayed on
  the right in \cref{11-fig:cov-nrg}.
\end{example}
\begin{figure}[htbp]
  \centering\scalebox{.48}{
  \begin{tikzpicture}[auto,on grid,node distance=2.5cm]
    \draw[step=1,lightgray!50,dotted] (-5.7,0) grid (5.7,3.8);
    \draw[color=white](0,-.3) -- (0,3.8);
    \node at (0,3.9) (sink) {\color{red!70!black}\boldmath$\sink$};
    \draw[step=1,lightgray!50] (1,0) grid (5.5,3.5);
    \draw[step=1,lightgray!50] (-1,0) grid (-5.5,3.5);
    \node at (0,0)[lightgray,font=\scriptsize,fill=white] {0};
    \node at (0,1)[lightgray,font=\scriptsize,fill=white] {1};
    \node at (0,2)[lightgray,font=\scriptsize,fill=white] {2};
    \node at (0,3)[lightgray,font=\scriptsize,fill=white] {3};
    \node at (1,3.9)[lightgray,font=\scriptsize,fill=white] {0};
    \node at (2,3.9)[lightgray,font=\scriptsize,fill=white] {1};
    \node at (3,3.9)[lightgray,font=\scriptsize,fill=white] {2};
    \node at (4,3.9)[lightgray,font=\scriptsize,fill=white] {3};
    \node at (5,3.9)[lightgray,font=\scriptsize,fill=white] {4};
    \node at (-1,3.9)[lightgray,font=\scriptsize,fill=white] {0};
    \node at (-2,3.9)[lightgray,font=\scriptsize,fill=white] {1};
    \node at (-3,3.9)[lightgray,font=\scriptsize,fill=white] {2};
    \node at (-4,3.9)[lightgray,font=\scriptsize,fill=white] {3};
    \node at (-5,3.9)[lightgray,font=\scriptsize,fill=white] {4};
    \node at (1,0)[s-eve-small,lose] (e00) {};
    \node at (1,1)[s-adam-small,lose](a01){};
    \node at (1,2)[s-eve-small,lose] (e02){};
    \node at (1,3)[s-adam-small,lose](a03){};
    \node at (2,0)[s-adam-small,lose](a10){};
    \node at (2,1)[s-eve-small,lose] (e11){};
    \node at (2,2)[s-adam-small,lose](a12){};
    \node at (2,3)[s-eve-small,lose] (e13){};
    \node at (3,0)[s-eve-small,lose] (e20){};
    \node at (3,1)[s-adam-small,lose](a21){};
    \node at (3,2)[s-eve-small,win] (e22){};
    \node at (3,3)[s-adam-small,lose](a23){};
    \node at (4,0)[s-adam-small,lose](a30){};
    \node at (4,1)[s-eve-small,lose] (e31){};
    \node at (4,2)[s-adam-small,lose](a32){};
    \node at (4,3)[s-eve-small,win] (e33){};
    \node at (5,0)[s-eve-small,lose] (e40){};
    \node at (5,1)[s-adam-small,lose](a41){};
    \node at (5,2)[s-eve-small,lose] (e42){};
    \node at (5,3)[s-adam-small,lose](a43){};
    \node at (-1,0)[s-adam-small,lose](a00){};
    \node at (-1,1)[s-eve-small,lose] (e01){};
    \node at (-1,2)[s-adam-small,lose](a02){};
    \node at (-1,3)[s-eve-small,lose] (e03){};
    \node at (-2,0)[s-eve-small,lose] (e10){};
    \node at (-2,1)[s-adam-small,lose](a11){};
    \node at (-2,2)[s-eve-small,lose] (e12){};
    \node at (-2,3)[s-adam-small,lose](a13){};
    \node at (-3,0)[s-adam-small,lose](a20){};
    \node at (-3,1)[s-eve-small,lose] (e21){};
    \node at (-3,2)[s-adam-small,lose](a22){};
    \node at (-3,3)[s-eve-small,lose] (e23){};
    \node at (-4,0)[s-eve-small,lose] (e30){};
    \node at (-4,1)[s-adam-small,lose](a31){};
    \node at (-4,2)[s-eve-small,lose] (e32){};
    \node at (-4,3)[s-adam-small,lose](a33){};
    \node at (-5,0)[s-adam-small,lose](a40){};
    \node at (-5,1)[s-eve-small,lose] (e41){};
    \node at (-5,2)[s-adam-small,lose](a42){};
    \node at (-5,3)[s-eve-small,lose] (e43){};
    \path[arrow] % l, -1,-1, l
    (e11) edge (e00)
    (e22) edge (e11)
    (e31) edge (e20)
    (e32) edge (e21)
    (e21) edge (e10)
    (e12) edge (e01)
    (e23) edge (e12)
    (e33) edge (e22)
    (e13) edge (e02)
    (e43) edge (e32)
    (e42) edge (e31)
    (e41) edge (e30);
    \path[arrow] % l, -1,0, l'
    (e11) edge (a01)
    (e20) edge (a10)
    (e22) edge (a12)
    (e31) edge (a21)
    (e32) edge (a22)
    (e21) edge (a11)
    (e12) edge (a02)
    (e30) edge (a20)
    (e10) edge (a00)
    (e13) edge (a03)
    (e23) edge (a13)
    (e33) edge (a23)
    (e43) edge (a33)
    (e42) edge (a32)
    (e41) edge (a31)
    (e40) edge (a30);
    \path[arrow] % l', -1,0, l
    (a11) edge (e01)
    (a20) edge (e10)
    (a22) edge (e12)
    (a31) edge (e21)
    (a32) edge (e22)
    (a21) edge (e11)
    (a12) edge (e02)
    (a30) edge (e20)
    (a10) edge (e00)
    (a33) edge (e23)
    (a23) edge (e13)
    (a13) edge (e03)
    (a43) edge (e33)
    (a42) edge (e32)
    (a41) edge (e31)
    (a40) edge (e30);
    \path[arrow] % l', 2,1, l
    (a01) edge (e22)
    (a10) edge (e31)
    (a11) edge (e32)
    (a00) edge (e21)
    (a02) edge (e23)
    (a12) edge (e33)
    (a22) edge (e43)
    (a21) edge (e42)
    (a20) edge (e41);
    \path[arrow] % dotted to Eve
    (-5.5,3.5) edge (e43)
    (5.5,2.5) edge (e42)
    (2.5,3.5) edge (e13)
    (5.5,0.5) edge (e40)
    (-5.5,1.5) edge (e41)
    (-3.5,3.5) edge (e23)
    (-1.5,3.5) edge (e03)
    (4.5,3.5) edge (e33)
    (5.5,0) edge (e40)
    (5.5,2) edge (e42)
    (-5.5,1) edge (e41)
    (-5.5,3) edge (e43);
    \path[dotted]
    (-5.7,3.7) edge (-5.5,3.5)
    (5.7,2.7) edge (5.5,2.5)
    (2.7,3.7) edge (2.5,3.5)
    (5.7,0.7) edge (5.5,0.5)
    (-3.7,3.7) edge (-3.5,3.5)
    (-1.7,3.7) edge (-1.5,3.5)
    (4.7,3.7) edge (4.5,3.5)
    (-5.7,1.7) edge (-5.5,1.5)
    (5.75,0) edge (5.5,0)
    (5.75,2) edge (5.5,2)
    (-5.75,1) edge (-5.5,1)
    (-5.75,3) edge (-5.5,3);
    \path[arrow]
    (5.5,1) edge (a41)
    (-5.5,2) edge (a42)
    (-5.5,0) edge (a40)
    (5.5,3) edge (a43);
    \path[dotted]
    (5.75,1) edge (5.5,1)
    (-5.75,2) edge (-5.5,2)
    (-5.75,0) edge (-5.5,0)
    (5.75,3) edge (5.5,3);
    \path[-]
    (a30) edge (5.5,.75)
    (a32) edge (5.5,2.75)
    (a31) edge (-5.5,1.75)
    (a23) edge (4,3.5)
    (a03) edge (2,3.5)
    (a13) edge (-3,3.5)
    (a33) edge (-5,3.5)
    (a43) edge (5.5,3.25)
    (a41) edge (5.5,1.25)
    (a40) edge (-5.5,0.25)
    (a42) edge (-5.5,2.25);
    \path[dotted]
    (5.5,.75) edge (5.8,.9)
    (5.5,2.75) edge (5.8,2.9)
    (-5.5,1.75) edge (-5.8,1.9)
    (4,3.5) edge (4.4,3.7)
    (2,3.5) edge (2.4,3.7)
    (-3,3.5) edge (-3.4,3.7)
    (-5,3.5) edge (-5.4,3.7)
    (5.5,3.25) edge (5.8,3.4)
    (5.5,1.25) edge (5.8,1.4)
    (-5.5,.25) edge (-5.8,0.4)
    (-5.5,2.25) edge (-5.8,2.4);
    \path[arrow]
    (sink) edge[loop left] ()
    (e00) edge[bend left=8] (sink)
    (e01) edge[bend right=8] (sink)
    (e02) edge[bend left=8] (sink)
    (e03) edge[bend right=8] (sink)
    (a00) edge[bend right=8] (sink)
    (a01) edge[bend left=8] (sink)
    (a02) edge[bend right=8] (sink)
    (a03) edge[bend left=8] (sink);
  \end{tikzpicture}}\quad~~\scalebox{.48}{
  \begin{tikzpicture}[auto,on grid,node distance=2.5cm]
    \draw[step=1,lightgray!50,dotted] (-5.7,0) grid (5.7,3.8);
    \draw[color=white](0,-.3) -- (0,3.8);
    \node at (0,3.9) (sink) {\color{red!70!black}\boldmath$\sink$};
    \draw[step=1,lightgray!50] (1,0) grid (5.5,3.5);
    \draw[step=1,lightgray!50] (-1,0) grid (-5.5,3.5);
    \node at (0,0)[lightgray,font=\scriptsize,fill=white] {0};
    \node at (0,1)[lightgray,font=\scriptsize,fill=white] {1};
    \node at (0,2)[lightgray,font=\scriptsize,fill=white] {2};
    \node at (0,3)[lightgray,font=\scriptsize,fill=white] {3};
    \node at (1,3.9)[lightgray,font=\scriptsize,fill=white] {0};
    \node at (2,3.9)[lightgray,font=\scriptsize,fill=white] {1};
    \node at (3,3.9)[lightgray,font=\scriptsize,fill=white] {2};
    \node at (4,3.9)[lightgray,font=\scriptsize,fill=white] {3};
    \node at (5,3.9)[lightgray,font=\scriptsize,fill=white] {4};
    \node at (-1,3.9)[lightgray,font=\scriptsize,fill=white] {0};
    \node at (-2,3.9)[lightgray,font=\scriptsize,fill=white] {1};
    \node at (-3,3.9)[lightgray,font=\scriptsize,fill=white] {2};
    \node at (-4,3.9)[lightgray,font=\scriptsize,fill=white] {3};
    \node at (-5,3.9)[lightgray,font=\scriptsize,fill=white] {4};
    \node at (1,0)[s-eve-small,lose] (e00) {};
    \node at (1,1)[s-adam-small,lose](a01){};
    \node at (1,2)[s-eve-small,lose] (e02){};
    \node at (1,3)[s-adam-small,lose](a03){};
    \node at (2,0)[s-adam-small,lose](a10){};
    \node at (2,1)[s-eve-small,lose] (e11){};
    \node at (2,2)[s-adam-small,lose](a12){};
    \node at (2,3)[s-eve-small,lose] (e13){};
    \node at (3,0)[s-eve-small,lose] (e20){};
    \node at (3,1)[s-adam-small,lose](a21){};
    \node at (3,2)[s-eve-small,win] (e22){};
    \node at (3,3)[s-adam-small,lose](a23){};
    \node at (4,0)[s-adam-small,lose](a30){};
    \node at (4,1)[s-eve-small,lose] (e31){};
    \node at (4,2)[s-adam-small,win](a32){};
    \node at (4,3)[s-eve-small,win] (e33){};
    \node at (5,0)[s-eve-small,lose] (e40){};
    \node at (5,1)[s-adam-small,lose](a41){};
    \node at (5,2)[s-eve-small,win] (e42){};
    \node at (5,3)[s-adam-small,win](a43){};
    \node at (-1,0)[s-adam-small,lose](a00){};
    \node at (-1,1)[s-eve-small,lose] (e01){};
    \node at (-1,2)[s-adam-small,lose](a02){};
    \node at (-1,3)[s-eve-small,lose] (e03){};
    \node at (-2,0)[s-eve-small,lose] (e10){};
    \node at (-2,1)[s-adam-small,lose](a11){};
    \node at (-2,2)[s-eve-small,lose] (e12){};
    \node at (-2,3)[s-adam-small,lose](a13){};
    \node at (-3,0)[s-adam-small,lose](a20){};
    \node at (-3,1)[s-eve-small,lose] (e21){};
    \node at (-3,2)[s-adam-small,lose](a22){};
    \node at (-3,3)[s-eve-small,win] (e23){};
    \node at (-4,0)[s-eve-small,lose] (e30){};
    \node at (-4,1)[s-adam-small,lose](a31){};
    \node at (-4,2)[s-eve-small,win] (e32){};
    \node at (-4,3)[s-adam-small,win](a33){};
    \node at (-5,0)[s-adam-small,lose](a40){};
    \node at (-5,1)[s-eve-small,lose] (e41){};
    \node at (-5,2)[s-adam-small,win](a42){};
    \node at (-5,3)[s-eve-small,win] (e43){};
    \path[arrow] % l, -1,-1, l
    (e11) edge (e00)
    (e22) edge (e11)
    (e31) edge (e20)
    (e32) edge (e21)
    (e21) edge (e10)
    (e12) edge (e01)
    (e23) edge (e12)
    (e33) edge (e22)
    (e13) edge (e02)
    (e43) edge (e32)
    (e42) edge (e31)
    (e41) edge (e30);
    \path[arrow] % l, -1,0, l'
    (e11) edge (a01)
    (e20) edge (a10)
    (e22) edge (a12)
    (e31) edge (a21)
    (e32) edge (a22)
    (e21) edge (a11)
    (e12) edge (a02)
    (e30) edge (a20)
    (e10) edge (a00)
    (e13) edge (a03)
    (e23) edge (a13)
    (e33) edge (a23)
    (e43) edge (a33)
    (e42) edge (a32)
    (e41) edge (a31)
    (e40) edge (a30);
    \path[arrow] % l', -1,0, l
    (a11) edge (e01)
    (a20) edge (e10)
    (a22) edge (e12)
    (a31) edge (e21)
    (a32) edge (e22)
    (a21) edge (e11)
    (a12) edge (e02)
    (a30) edge (e20)
    (a10) edge (e00)
    (a33) edge (e23)
    (a23) edge (e13)
    (a13) edge (e03)
    (a43) edge (e33)
    (a42) edge (e32)
    (a41) edge (e31)
    (a40) edge (e30);
    \path[arrow] % l', 2,1, l
    (a01) edge (e22)
    (a10) edge (e31)
    (a11) edge (e32)
    (a00) edge (e21)
    (a02) edge (e23)
    (a12) edge (e33)
    (a22) edge (e43)
    (a21) edge (e42)
    (a20) edge (e41);
    \path[arrow] % dotted to Eve
    (-5.5,3.5) edge (e43)
    (5.5,2.5) edge (e42)
    (2.5,3.5) edge (e13)
    (5.5,0.5) edge (e40)
    (-5.5,1.5) edge (e41)
    (-3.5,3.5) edge (e23)
    (-1.5,3.5) edge (e03)
    (4.5,3.5) edge (e33)
    (5.5,0) edge (e40)
    (5.5,2) edge (e42)
    (-5.5,1) edge (e41)
    (-5.5,3) edge (e43);
    \path[dotted]
    (-5.7,3.7) edge (-5.5,3.5)
    (5.7,2.7) edge (5.5,2.5)
    (2.7,3.7) edge (2.5,3.5)
    (5.7,0.7) edge (5.5,0.5)
    (-3.7,3.7) edge (-3.5,3.5)
    (-1.7,3.7) edge (-1.5,3.5)
    (4.7,3.7) edge (4.5,3.5)
    (-5.7,1.7) edge (-5.5,1.5)
    (5.75,0) edge (5.5,0)
    (5.75,2) edge (5.5,2)
    (-5.75,1) edge (-5.5,1)
    (-5.75,3) edge (-5.5,3);
    \path[arrow]
    (5.5,1) edge (a41)
    (-5.5,2) edge (a42)
    (-5.5,0) edge (a40)
    (5.5,3) edge (a43);
    \path[dotted]
    (5.75,1) edge (5.5,1)
    (-5.75,2) edge (-5.5,2)
    (-5.75,0) edge (-5.5,0)
    (5.75,3) edge (5.5,3);
    \path[-]
    (a30) edge (5.5,.75)
    (a32) edge (5.5,2.75)
    (a31) edge (-5.5,1.75)
    (a23) edge (4,3.5)
    (a03) edge (2,3.5)
    (a13) edge (-3,3.5)
    (a33) edge (-5,3.5)
    (a43) edge (5.5,3.25)
    (a41) edge (5.5,1.25)
    (a40) edge (-5.5,0.25)
    (a42) edge (-5.5,2.25);
    \path[dotted]
    (5.5,.75) edge (5.8,.9)
    (5.5,2.75) edge (5.8,2.9)
    (-5.5,1.75) edge (-5.8,1.9)
    (4,3.5) edge (4.4,3.7)
    (2,3.5) edge (2.4,3.7)
    (-3,3.5) edge (-3.4,3.7)
    (-5,3.5) edge (-5.4,3.7)
    (5.5,3.25) edge (5.8,3.4)
    (5.5,1.25) edge (5.8,1.4)
    (-5.5,.25) edge (-5.8,0.4)
    (-5.5,2.25) edge (-5.8,2.4);
    \path[arrow]
    (sink) edge[loop left] ()
    (e00) edge[bend left=8] (sink)
    (e01) edge[bend right=8] (sink)
    (e02) edge[bend left=8] (sink)
    (e03) edge[bend right=8] (sink)
    (a00) edge[bend right=8] (sink)
    (a01) edge[bend left=8] (sink)
    (a02) edge[bend right=8] (sink)
    (a03) edge[bend left=8] (sink);
  \end{tikzpicture}}
  \caption{The "winning regions" of \Eve\ in the
    "configuration reachability" "energy game" (left) and the
    "coverability" "energy game"
    (right) on the graphs of \cref{11-fig:mwg,11-fig:nrg} with target
    configuration~$\ell(2,2)$.  The winning vertices are in filled in
    green, while the losing ones are filled with white with a red
    border; the "sink" is always losing.}\label{11-fig:cov-nrg}
\end{figure}

The reader might have noticed that the "natural semantics" of the
"asymmetric" system of \cref{11-fig:avg} and the "energy semantics" of
the system of \cref{11-fig:mwg} are essentially the same.  This
correspondence is quite general.
\begin{lemma}[Energy vs.\ asymmetric vector games]
\label{11-lem:nrg}
  "Energy games" and "asymmetric" "vector games" are
  \logspace-equivalent for "configuration reachability",
  "coverability", "non-termination", and "parity@parity vector games",
  both with "given" and with "existential initial credit".
\end{lemma}
\begin{proof}
  Let us first reduce "asymmetric vector games" to "energy games".
  Given $\?V$, $\col$, and $\Omega$ where $\?V$ is "asymmetric" and
  $\Eve$ loses if the play ever visits the "sink"~$\sink$, we see that
  $\Eve$ wins $(\natural(\?V),\col,\Omega)$ from some $v\in V$ if and
  only if she wins $(\energy(\?V),\col,\Omega)$ from $v$.  Of course,
  this might not be true if~$\?V$ is not "asymmetric", as seen for
  instance in \cref{11-ex:cov,11-ex:cov-nrg}.
  % It suffices for this to observe
  % that, for all $v\in V$,
  % \begin{description}
  %   \item[(colours)] $\col(v)$ is the same in both games;
  %   \item[(zig \Eve)] if $v\in\VE$ and $(v,v')\in E$ with $v'\neq\sink$,
  %     then $(v,v')\in E_\+E$: indeed, $v'\neq\sink$ entails that
  %     $v$ is a configuration $\loc(\vec v)$ and $v'=\loc'(\vec v+\vec
  %     u)$ for some action $\loc\step{\vec u}\loc'\in\Act$ with
  %     $\vec v+\vec u\geq\vec 0$, but then $(v,v')\in E_\+E$;
  %   \item[(zag \Eve)] if $v\in\VE$ and $(v,v')\in E_\+E$ with $v'\neq\sink$,
  %     then $(v,v')\in E$: by the same reasoning;
  %   \item[(zig \Adam)] if $v\in\VA$ and $(v,v')\in E_\+E$,
  %     then $(v,v')\in E$: indeed, either $v=\sink$ and then $(v,v')\in
  %     E$, or $v=\loc(\vec v)$ and $v'=\sink$ and then there are no
  %     outgoing actions from~$\loc$ in $\Act$ thus $(v,v')\in
  %     E$, or $v=\loc(\vec v)$ and $v'=\loc'(\vec v)$ for some
  %     action $\loc\step{\vec 0}\loc'\in\Act$ (recall that~$\?V$ is
  %     assumed to be asymmetric) and then $(v,v')\in E$;
  %   \item[(zag \Adam)] if $v\in\VA$ and $(v,v')\in E$,
  %     then $(v,v')\in E_\+E$: by the same reasoning.
  %   \end{description}

  \medskip Conversely, let us reduce "energy games" to "asymmetric
  vector games".  Consider
  $\?V=(\Loc,\Act,\Loc_\mEve,\Loc_\mAdam,\dd)$, a colouring $\col$
  defined from a vertex colouring $\vcol$ by
  $\col(e)\eqdef\vcol(\ing(e))$, and an objective $\Omega$, where
  $\vcol$ and $\Omega$ are such that $\Eve$ loses if the play ever
  visits the "sink"~$\sink$ and such that, for all $\pi\in C^\ast$,
  $p\in C$, and $\pi'\in C^\omega$, $\pi p\pi'\in\Omega$ if and only
  if $\pi pp\pi'\in\Omega$ (we shall call $\Omega$
  \emph{stutter-invariant}, and the objectives in the statement are
  indeed stutter-invariant).  We construct an "asymmetric vector
  system"
  $\?V'\eqdef(\Loc\uplus\Loc_\Act,\Act',\Loc_\mEve\uplus\Loc_\Act,\Loc_\mAdam,\dd)$
  where we add the following locations controlled by~\Eve:
    \begin{align*}
      \Loc_\Act&\eqdef\{\loc_a\mid a=(\loc\step{\vec
                 u}\loc')\in\Act\text{ and }\loc\in\Loc_\mAdam\}\;.
      \intertext{We also modify the set of actions:}
      \Act'&\eqdef\{\loc\step{\vec u}\loc'\mid \loc\step{\vec
             u}\loc'\in\Act\text{ and }\loc\in\Loc_\mEve\}\\
      &\:\cup\:\{\loc\step{\vec 0}\loc_a,\;\loc_a\step{\vec u}\loc'\mid a=(\loc\step{\vec u}\loc')\in\Act\text{ and }\loc\in\Loc_\mAdam\}\;.
    \end{align*}
    \Cref{11-fig:avg} presents the result of this reduction on the
    system of \cref{11-fig:mwg}.  We define a vertex colouring
    $\vcol'$ of $\arena_\+N(\?V')$ with $\vcol'(v)\eqdef\vcol(v)$ for
    all $v\in \Loc\times\+N^\dd\uplus\{\sink\}$ and
    $\vcol'(\loc_a(\vec v))\eqdef\vcol(\loc(\vec v))$ if
    $a=(\loc\step{\vec u}\loc')\in\Act$.  Then, for all vertices
    $v\in V$, \Eve\ wins from~$v$ in the "energy game"
    $(\energy(\?V),\col,\Omega)$ if and only if she wins from~$v$ in
    the "vector game" $(\natural(\?V'),\col',\Omega)$.  The crux of
    the argument is that, in a configuration $\loc(\vec v)$ where
    $\loc\in\Loc_\mAdam$, if $a=(\loc\step{\vec u}\loc')\in\Act$ is an
    action with $\vec v+\vec u\not\geq\vec 0$, in the "energy
    semantics", \Adam\ can force the play into the "sink" by
    playing~$a$; the same occurs in $\?V'$ with the "natural
    semantics", as \Adam\ can now choose to play
    $\loc\step{\vec 0}\loc_a$ where \Eve\ has only
    $\loc_a\step{\vec u}\loc'$ at her disposal, which leads to the
    sink.\todoquestion{Is that clear?}
    % First note that the available edges for \Eve\ that avoid the "sink"
    % are the same in both games.  Furthermore, if $(v,v')$ is an edge
    % in the "energy game" from $v\in\VA$, then either $v'$ is a
    % configuration $v'=\loc'(\vec v+\vec u)$ for some
    % $t=(\loc\step{\vec u}\loc')\in\Act$ such that $v=\loc(\vec v)$,
    % and in that case the sequence of moves
    % $(v,\loc_{t)(\vec v))(\loc_t(\vec v),v')$ is forced upon \Eve\ in
    %   the "vector game".
\end{proof}

In turn, "energy games" with "existential initial credit" are related
to the "multi-dimensional mean-payoff games" of
\cref{12-chap:multiobjective}.% We shall see the
% case of dimension~one in \cref{11-subsec:mono-dim1}.


%\subsubsection{Capped Semantics}
\label{11-capping}

In resource-conscious systems, there is often a maximal capacity for
resources: for instance, a battery cannot be charged any further once
full.  We can model this by providing an alternative semantics
for vector systems.  Given a capacity $C\in\+N$, for a vector~$\vec v$
in~$\+N^\dd$, let us write $\capp v$ for the vector `capped' at~$C$:
for all~$1\leq i\leq\dd$, $\capp v(i)\eqdef\vec v(i)$ if $\vec v(i)<C$
and $\capp v\eqdef C$ if $\vec v(i)\geq C$.  We define for a
capacity~$C\in\+N$ the ""capped semantics""
$\capped(\?V)=(V^C,\overline{E}^C,\VE^C,\VA^C)$ of a "vector system"~$\?V$ by
\begin{align*}
  V^C&\eqdef\{\loc(\vec v)\mid\loc\in\Loc\text{ and }\|\vec v\|<C\}\;,\\
  \overline{E}^C&\eqdef \{(\loc(\vec v),\loc'(\capp{v+\vec u})\mid\loc\step{\vec
       u}\loc'\in\Act\text{ and }\vec v+\vec u\geq\vec 0\}\\
     &\:\cup\:\{(\loc(\vec v),\sink)\mid\forall\loc\step{\vec
               u}\loc'\in\Act\mathbin.\vec v+\vec u\not\geq\vec
               0\}
     \cup\{(\sink,\sink)\}\;.
\end{align*}
As usual, $\VE^C\eqdef V^C\cap\Loc_\mEve\times\+N^\dd$ and
$\VA^C\eqdef V^C\cap\Loc_\mAdam\times\+N^\dd$.  This is the same as
the "energy semantics" where we cap the vectors resulting from actions
in~$\Act$.  All the configurations in this arena have "norm" less
than~$C$, thus $|V^C|=|\Loc| C^\dd+1$, and the qualitative games of
\cref{chap:regular} are decidable over this "arena".


% Local IspellDict: british


\subsection{Bounded Semantics}
\label{11-sec:bounding}

While \Adam\ wins immediately in an "energy game" if a resource gets
depleted, he also wins in a "bounding game" if a resource reaches a
certain bound~$B$.  %By contrast with "capped semantics", t
This is
a \emph{hard upper bound}, allowing to model systems where exceeding a
capacity results in failure, like a dam that overflows and floods the
area.  We define for a bound~$B\in\+N$ the ""bounded semantics""
$\bounded(\?V)=(V^B,E^B,\VE^B,\VA^B)$ of a "vector system"~$\?V$ by
\begin{align*}
  V^B&\eqdef\{\loc(\vec v)\mid\loc\in\Loc\text{ and }\|\vec v\|<B\}\;,\\
  E^B&\eqdef \{(\loc(\vec v),\loc'(\vec v+\vec u))\mid\loc\step{\vec
       u}\loc'\in\Act,\vec v+\vec u\geq\vec 0,\text{ and }\|\vec
       v+\vec u\|<B\}\\
     &\:\cup\:\{(\loc(\vec v),\sink)\mid\forall\loc\step{\vec
               u}\loc'\in\Act\mathbin.\vec v+\vec u\not\geq\vec
               0\text{ or }\|\vec v+\vec u\|\geq B\}
     \cup\{(\sink,\sink)\}\;.
\end{align*}
As usual, $\VE^B\eqdef V^B\cap\Loc_\mEve\times\+N^\dd$ and
$\VA^B\eqdef V^B\cap\Loc_\mAdam\times\+N^\dd$.  Any edge from the
"energy semantics" that would bring to a configuration $\loc(\vec v)$
with $\vec v(i)\geq B$ for some $1\leq i\leq\dd$ leads instead to the
sink.  All the configurations in this arena have "norm" less than~$B$,
thus $|V^B|=|\Loc| B^\dd+1$, and the qualitative games of
\cref{2-chap:regular} are decidable over this "arena".

Our focus here is on "non-termination" games played on the "bounded
semantics" where~$B$ is not given as part of the input, but quantified
existentially.  As usual, the "existential initial credit" variant
of \cref{11-pb:bounding} is obtained by quantifying~$\vec v_0$
existentially in the question.
\decpb["bounding game" with "given initial credit"]%
{\label{11-pb:bounding} A "vector system"
  $\?V=(\Loc,\Act,\Loc_\mEve,\Loc_\mAdam,\dd)$, an initial location
  $\loc_0\in\Loc$, and an initial credit $\vec v_0\in\+N^\dd$.}%
  {Does there exist $B\in\+N$ such that \Eve\ has a strategy to avoid the
  "sink"~$\sink$ from $\loc_0(\vec v_0)$ in the "bounded
  semantics"?  That is, does there exist $B\in\+N$ such that she wins
  the ""bounding"" game $(\bounded(\?V),\col,\Safe)$ from
  $\loc_0(\vec v_0)$, where $\col(e)\eqdef\Lose$ if and only if $\ing(e)=\sink$?}

\begin{lemma}\label{11-lem:parity2bounding}
  There is a \logspace\ reduction from "parity@parity vector games"
  "asymmetric" "vector games" to "bounding games", both with "given"
  and with "existential initial credit".
\end{lemma}
\begin{proof}
  Given an "asymmetric vector system"
  $\?V=(\Loc,\Act,\Loc_\mEve,\Loc_\mAdam,\dd)$, a location colouring
  $\lcol{:}\,\Loc\to\{1,\dots,2d\}$, and an initial location
  $\loc_0\in\Loc$, we construct a "vector system" $\?V'$ of dimension
  $\dd'\eqdef\dd+d$ as described in \cref{11-fig:bounding}, where the
  priorities in~$\?V$ for $p\in\{1,\dots,d\}$ are indicated above the
  corresponding locations.
  
  \begin{figure}[htbp]
    \centering
    \begin{tikzpicture}[auto,on grid,node distance=1.5cm]
      \node(to){$\mapsto$};
      \node[anchor=east,left=2.5cm of to](mm){"asymmetric vector system"~$\?V$};
      \node[anchor=west,right=2.5cm of to](mwg){"vector system"~$\?V'$};
      % Eve location, even parity
      \node[below=1.3cm of to](imap){$\rightsquigarrow$};
      \node[s-eve,left=2.75cm of imap](i0){$\loc$};
      \node[black!50,above=.4 of i0,font=\scriptsize]{$2p$};
      \node[right=of i0](i1){$\loc'$};
      \node[right=1.25cm of imap,s-eve](i2){$\loc$};
      \node[right=1.8 of i2,s-eve-small](i3){};      
      \node[right=1.8 of i3](i4){$\loc'$};      
      \path[arrow,every node/.style={font=\footnotesize}]
      (i0) edge node{$\vec u$} (i1)
      (i2) edge[loop above] node{$\forall 1\leq i\leq\dd\mathbin.-\vec e_i$} ()
      (i2) edge node{$\vec u$} (i3)
      (i3) edge[loop below] node{$\forall 1\leq j\leq p\mathbin.\vec e_{\dd+j}$} ()
      (i3) edge node{$\vec 0$} (i4);
      % Eve location, odd parity
      \node[below=2cm of imap](dmap){$\rightsquigarrow$};
      \node[s-eve,left=2.75cm of dmap](d0){$\loc$};
      \node[black!50,above=.4 of d0,font=\scriptsize]{$2p-1$};
      \node[right=of d0](d1){$\loc'$};
      \node[right=1.25cm of dmap,s-eve](d2){$\loc$};
      \node[right=2 of d2](d3){$\loc'$};
      \path[arrow,every node/.style={font=\footnotesize}]
      (d0) edge node{$\vec u$} (d1)
      (d2) edge[loop above] node{$\forall 1\leq i\leq\dd\mathbin.-\vec e_i$} ()
      (d2) edge node{$\vec u-\vec e_{\dd+p}$} (d3);
      % Adam location, even parity
      \node[below=1.1cm of dmap](zmap){$\rightsquigarrow$};
      \node[s-adam,left=2.75cm of zmap](z0){$\loc$};
      \node[black!50,above=.4 of z0,font=\scriptsize]{$2p$};
      \node[right=of z0](z1){$\loc'$};
      \node[right=1.25cm of zmap,s-adam](z2){$\loc$};
      \node[right=of z2,s-eve-small](z3){};
      \node[right=of z3](z4){$\loc'$};
      \path[arrow,every node/.style={font=\footnotesize}]
      (z0) edge node{$\vec 0$} (z1)
      (z2) edge node{$\vec 0$} (z3)
      (z3) edge node{$\vec 0$} (z4)
      (z3) edge[loop below] node{$\forall 1\leq j\leq p\mathbin.\vec e_{\dd+j}$} ();
      % Adam location, odd parity
      \node[below=1.6cm of zmap](amap){$\rightsquigarrow$};
      \node[s-adam,left=2.75cm of amap](a0){$\loc$};
      \node[black!50,above=.4 of a0,font=\scriptsize]{$2p-1$};
      \node[right=of a0](a1){$\loc'$};
      \node[right=1.25cm of amap,s-adam](a2){$\loc$};
      \node[right=2 of a2](a3){$\loc'$};
      \path[arrow,every node/.style={font=\footnotesize}]
      (a0) edge node{$\vec 0$} (a1)
      (a2) edge node{$-\vec e_{\dd+p}$} (a3);
    \end{tikzpicture}
    \caption{Schema of the reduction to
      "bounding games" in the proof of \cref{11-lem:parity2bounding}.}\label{11-fig:bounding}
  \end{figure}
  
  If \Eve\ wins the "bounding game" played over $\?V'$ from some
  configuration $\loc_0(\vec v_0)$, then she also wins the "parity
  vector game" played over~$\?V$ from the configuration $\loc_0(\vec
  v'_0)$ where $\vec v'_0$ is the projection of $\vec v_0$
  to~$\+N^\dd$.  Indeed, she can play essentially the same strategy:
  by \cref{11-lem:mono} she can simply ignore the new decrement
  self loops, while the actions on the components in
  $\{\dd+1,\dots,\dd+d\}$ ensure that the maximal priority visited
  infinitely often is even---otherwise some decrement $-\vec
  e_{\dd+p}$ would be played infinitely often but the increment $\vec
  e_{\dd+p}$ only finitely often.\todoquestion{Should I provide more details?}
  

  \medskip
  Conversely, consider the "parity@parity vector game" game~$\game$ played over
  $\natural(\?V)$ with the colouring defined by~$\lcol$.  Then the
  "Pareto limit" of the game is finite, thus there exists a natural
  number
  \begin{equation}\label{11-eq:b0}
    B_0\eqdef 1+\max_{\loc_0(\vec v_0)\in\mathsf{Pareto}(\?G)}\|\vec
  v_0\|
  \end{equation} bounding the "norms" of the minimal winning configurations.
  For a vector~$\vec v$ in~$\+N^\dd$, let us write $\capp[B_0]v$ for
  the vector `capped' at~$B$: for all~$1\leq i\leq\dd$,
  $\capp[B_0]v(i)\eqdef\vec v(i)$ if $\vec v(i)<B_0$ and
  $\capp[B_0]v\eqdef B_0$ if $\vec v(i)\geq B_0$.

  %% \begin{claim}\label{11-cl:capped}
  %%   If $\loc(\vec v)\in\WE(\game)$, then $\loc(\capp[B_0]v)\in\WE(\game)$.
  %% \end{claim}
  %% Indeed, by definition of the "Pareto limit"~$\mathsf{Pareto}(\game)$,
  %% if $\loc(\vec v)\in\WE(\game)$, then there exists~$\vec v'\leq\vec v$
  %% such that $\loc(\vec v')\in\mathsf{Pareto}(\game)$.  By definition of
  %% the bound~$B_0$, $\|\vec v'\|<B_0$.  Thus $\vec v'\leq\capp[B_0]v$.
  %% Thus $\loc(\capp[B_0]v)\in\WE(\game)$.

  Consider now some configuration $\loc_0(\vec
  v_0)\in\mathsf{Pareto}(\game)$.  As seen in \cref{11-lem:finmem},
  since $\loc_0(\vec v_0)\in\WE(\game)$, there is a finite
  "self-covering tree" witnessing the fact, and an associated winning
  strategy.  Let $H(\loc_0(\vec v_0))$ denote the height of this
  "self-covering tree" and observe that all the configurations in this
  tree have norm bounded by $\|\vec v_0\|+\|\Act\|\cdot H(\loc_0(\vec
  v_0))$.
  Let us define
  \begin{equation}\label{11-eq:b}
   B\eqdef B_0+(\|\Act\|+1)\cdot \max_{\loc_0(\vec
  v_0)\in\mathsf{Pareto}(\?G)}H(\loc_0(\vec v_0))\;.
  \end{equation}
  This is a bound on the norm of the configurations appearing on the
  (finitely many) self-covering trees spawned by the elements
  of~$\mathsf{Pareto}(\game)$.  Note that $B\geq B_0+(\|\Act\|+1)$ since
  a self-covering tree has height at least~one.

  Consider the "non-termination" game
  $\game_B\eqdef(\bounded(\?V'),\col',\Safe)$ played over the
  "bounded semantics" defined by~$B$, where $\col'(e)=\Lose$ if and
  only if $\ing(e)=\sink$.  Let $\vec b\eqdef\sum_{1\leq p\leq
  d}(B-1)\cdot\vec e_{\dd+p}$.
  {\renewcommand{\qedsymbol}{}
  \begin{claim}\label{11-cl:parity2bounding} If $\loc_0(\vec
    v)\in\WE(\game)$, then
    $\loc_0(\capp[B_0]{v}+\vec b)\in\WE(\game_B)$.
  \end{claim}}
  Indeed, by definition of the "Pareto
  limit"~$\mathsf{Pareto}(\game)$, if $\loc_0(\vec v)\in\WE(\game)$,
  then there exists~$\vec v_0\leq\vec v$ such that $\loc_0(\vec
  v_0)\in\mathsf{Pareto}(\game)$.  By definition of the bound~$B_0$,
  $\|\vec v_0\|<B_0$, thus $\vec v_0\leq\capp[B_0]v$.  Consider the
  "self-covering tree" of height~$H(\loc_0(\vec v_0))$ associated
  to~$\loc_0(\vec v_0)$, and the strategy~$\sigma'$ defined by the
  memory structure from the
  proof of \cref{11-lem:finmem}.  This is a winning strategy for \Eve\ 
  in $\game$ starting from $\loc_0(\vec v_0)$, and
  by \cref{11-lem:mono}, it is also winning
  from~$\loc_0(\capp[B_0]v)$.
    
  Here is how \Eve\ wins $\game_B$ from~$\loc_0(\capp[B_0]v+\vec b)$.
  She essentially follows the strategy~$\sigma'$, with two
  modifications.  First, whenever $\sigma'$ goes to a "return node"
  $\loc(\vec v)$ instead of a leaf $\loc(\vec v')$---thus $\vec
  v\leq\vec v'$---, the next time \Eve\ has the control, she uses the
  self loops to decrement the current configuration by $\vec v'-\vec
  v$.  This ensures that any play consistent with the modified
  strategy remains between zero and $B-1$ on the components
  in~$\{1,\dots,\dd\}$.  (Note that if she never regains the control,
  the current vector never changes any more since~$\?V$ is
  "asymmetric".)

  Second, whenever a play in~$\game$ visits a location with even
  parity~$2p$ for some~$p$ in~$\{1,\dots,d\}$, \Eve\ has the opportunity
  to increase the coordinates in~$\{\dd+1,\dots,\dd+p\}$ in~$\game_B$.
  She does so and increments until all these components reach~$B-1$.
  This ensures that any play consistent with the modified strategy
  remains between zero and $B-1$ on the components
  in~$\{\dd+1,\dots,\dd+p\}$.  Indeed, $\sigma'$ guarantees that the
  longest sequence of moves before a play visits a location with
  maximal even priority is bounded by $H(\loc_0(\vec v_0))$, thus the
  decrements $-\vec e_{\dd+p}$ introduced in~$\game_B$ by the
  locations from~$\game$ with odd parity~$2p-1$ will never force the
  play to go negative.\todoquestion{Is that clear enough?}
\end{proof}

The bound~$B$ defined in~\cref{11-eq:b} in the previous proof is not
constructive, and possibly much larger than really required.
Nevertheless, one can sometimes show that an explicit~$B$ suffices in
a "bounding game".
A simple example is provided by the "coverability" "asymmetric"
"vector games" with "existential initial credit" arising from
\cref{11-rk:cov2parity}, i.e., where the objective is to reach some
location~$\loc_f$.  Indeed, it is rather straightforward that there
exists a suitable initial credit such that \Eve\ wins the game if and
only if she wins the finite reachability game played over the
underlying directed graph over~$\Loc$ where we ignore the counters.
Thus, for an initial location~$\loc_0$, $B_0=|\Loc|\cdot\|\Act\|+1$
bounds the norm of the necessary initial credit, while a simple path
may visit at most~$|\Loc|$ locations, thus
$B=B_0+|\Loc|\cdot\|\Act\|$ suffices for \Eve\ to win the constructed
"bounding game".

In the general case of "bounding games" with "existential initial
credit", an explicit bound can be established.  The proof goes
along very different lines and is too involved to fit in this chapter,
but we refer the reader
to \cite{Jurdzinski&Lazic&Schmitz:2015,Colcombet&Jurdzinski&Lazic&Schmitz:2017}
for details.
\begin{theorem}[Bounds on bounding]
\label{11-th:bounding}
  If \Eve\ wins a "bounding game" with "existential initial credit"
  defined by a "vector
  system"~$\?V=(\Loc,\Act,\Loc_\mEve,\Loc_\mAdam,\dd)$, then an
  initial credit $\vec v_0$ with $\|\vec
  v_0\|=(4|\Loc|\cdot\|\Act\|)^{2(\dd+2)^3}$ and a bound
  $B=2(4|\Loc|\cdot\|\Act\|)^{2(\dd+2)^3}+1$ suffice for this.
\end{theorem}

\Cref{11-th:bounding} also yields a way of handling "bounding games"
with "given initial credit".  
\TODO{Last missing bit regarding complexity upper bounds.}
  
% Local IspellDict: british

