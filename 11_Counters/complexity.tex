Unlike general "vector games" and "configuration reachability"
"asymmetric" ones, "coverability", "non-termination", and
"parity@parity vector game" "asymmetric vector games" are decidable.
We survey in this section the best known complexity bounds for every
case; see \cref{11-tbl:cmplx} at the end of the chapter for a summary.

\subsection{Upper Bounds}
\label{11-sec:up}
We begin with complexity upper bounds.  The main results are that
"parity@parity vector game" games with "existential initial credit"
can be solved in \coNP, but are in \kEXP[2] with "given initial
credit".  In both cases however, the complexity is pseudo-polynomial
if both the dimension~$\dd$ and the number of priorities~$d$ are
fixed, which is rather good news: one can hope that, in practice, both
the number of different resources (encoded by the counters) and the
complexity of the functional specification (encoded by the parity
condition) are tiny compared to the size of the system.

%\subsubsection{Existential Initial Credit}
\label{12-sub-up-exist}

\paragraph{Counterless Strategies}
Consider a "strategy"~$\tau$ of \Adam in a "vector game".  In all the
games we consider, "uniform" "positional" strategies suffice over the
infinite "arena" $\natural(\?V)=(V,E,\VE,\VA)$: $\tau$ maps vertices
in~$V$ to edges in~$E$.  We call~$\tau$ ""counterless"" if, for all
locations $\loc\in\Loc_\mAdam$ and all vectors
$\vec v,\vec v'\in\+N^\dd$, $\tau(\loc(\vec v))=\tau(\loc(\vec v'))$.
A "counterless" strategy thus only considers the current location of
the play.
\begin{lemma}\label{12-counterless}
  Let $\?V=(\Loc,\Act,\Loc_\mEve,\Loc_\mAdam,\dd)$ be an "asymmetric
  vector system", $\loc_0\in\Loc$ be a location, and
  $\lcol{:}\,\Loc\to\{1,\dots,d\}$ be a location colouring.  If \Adam
  wins from $\loc_0(\vec v)$ for every initial credit~$\vec v$ in the
  "parity@parity vector game" game played over $\?V$ with~$\lcol$, then
  he has a single "counterless strategy" such that he wins from
  $\loc_0(\vec v)$ for every initial credit~$\vec v$.
\end{lemma}
\begin{proof}
  Let $\Act_\mAdam\eqdef\{(\loc\step{\vec
    u}\loc')\in\Act\mid\loc\in\Loc_\mAdam\}$ be the set of actions
  controlled by \Adam.  We assume without loss of generality that
  every location $\loc\in\Loc_\mAdam$ has either one or two outgoing
  actions, thus $|\Loc_\mAdam|\leq|\Act_\mAdam|\leq
  2|\Loc_\mAdam|$.  We proceed by induction over $|\Act_\mAdam|$.  For
  the base case, if $|\Act_\mAdam|=|\Loc_\mAdam|$ then every location
  controlled by \Adam has a single outgoing action, thus any
  strategy for \Adam is trivially "counterless".

  For the induction step, consider some location
  $\hat\loc\in\Loc_\mAdam$ with two outgoing actions
  $a_l\eqdef\hat\loc\step{\vec 0}\loc_l$ and
  $a_r\eqdef\hat\loc\step{\vec 0}\loc_r$.  Let $\?V_l$ and $\?V_r$ be
  the "vector systems" obtained from~$\?V$ by removing
  respectively~$a_r$ and~$a_l$ from~$\Act$, i.e., by using
  $\Act_l\eqdef\Act\setminus\{a_r\}$ and
  $\Act_r\eqdef\Act\setminus\{a_l\}$.  If $\Adam$ wins the
  "parity@parity vector game" game from $\loc(\vec v)$ for every
  initial credit~$\vec v$ in either $\?V_l$ or $\?V_r$, then by
  induction hypothesis he has a "counterless" winning strategy winning
  from $\loc(\vec v)$ for every initial credit~$\vec v$, and the same
  strategy is winning in~$\?V$ from $\loc(\vec v)$ for every initial
  credit~$\vec v$.

  In order to conclude the proof, we show that, if \Adam loses in
  $\?V_l$ from $\loc_0(\vec v_l)$ for some $\vec v_l\in\+N^\dd$ and in
  $\?V_r$ from $\loc_0(\vec v_r)$ for some $\vec v_r\in\+N^\dd$, then
  there exists $\vec v_0\in\+N^\dd$ such that \Eve wins from
  $\loc_0(\vec v_0)$ in~$\?V$.  Let $\sigma_l$ and $\sigma_r$ denote
  \Eve's winning strategies in the two games.  By a slight abuse of
  notations (justified by the fact that we are only interested in a
  few initial vertices), we see plays as sequences of actions and
  strategies as maps $\Act^\ast\to\Act$.\todoquestion{I hope this is not too messy}  Consider the set of
  plays consistent with~$\sigma_r$ starting from $\loc_0(\vec v_r)$.
  If none of those plays visits $\hat\loc$, then $\Eve$ wins in $\?V$
  from $\loc_0(\vec v_r)$ and we conclude.  Otherwise, there is some
  finite prefix~$\hat\pi$ of a play that
  visits~$\hat\loc(\hat{\vec v})$ for some vector
  $\hat{\vec v}=\vec v_r+\weight(\hat\pi)$.  We let
  $\vec v_0\eqdef\vec v_l+\hat{\vec v}$ and show that \Eve wins from
  $\loc_0(\vec v_0)$.

  \begin{scope}\knowledge{mode}{notion}
    We define now a strategy $\sigma$ for $\Eve$ over~$\?V$ that
    switches between applying~$\sigma_l$ and~$\sigma_r$ each time
    $a_r$ is used and switches back each time~$a_l$ is used.  More
    precisely, given a finite or infinite sequence~$\pi$ of actions,
    we decompose $\pi$ as $\pi_1 a_1 \pi_2 a_2 \pi_3\cdots$ where each
    segment $\pi_j\in(\Act\setminus\{a_l,a_r\})^\ast$ does not use
    either~$a_l$ nor~$a_r$ and each $a_j\in\{a_l,a_r\}$.  The
    associated ""mode"" $m(j)\in\{l,r\}$ of a segment~$\pi_j$
    is~$m(1)\eqdef l$ for the initial segment and otherwise
    $m(j)\eqdef l$ if $e_{j-1}=a_l$ and $m(j)\eqdef r$ otherwise.  The
    $l$-subsequence associated with $\pi$ is the sequence of segments
    $\pi(l)\eqdef\pi_{l_1}a_{l_2-1}\pi_{l_2}a_{l_3-1}\pi_{l_3}\cdots$
    with "mode"~$m(l_i)=l$, while the $r$-subsequence is the sequence
    $\pi(r)\eqdef\hat\pi a_{r_1-1}\pi_{r_1}a_{r_2-1}\pi_{r_2}\cdots$
    with "mode"~$m(r_i)=r$ prefixed by~$\hat\pi$.  Then we let
    $\sigma(\pi)\eqdef\sigma_{m}(\pi(m))$ where $m\in\{l,r\}$ is the
    "mode" of the last segment of~$\pi$.

    Consider an infinite play $\pi$ consistent with~$\sigma$ starting
    from~$\loc_0(\vec v_0)$.  Since $\vec v_0\geq\vec v_l$ and
    $\vec v_0\geq \vec v_r+\weight(\hat\pi)$, $\pi(l)$ and $\pi(r)$
    starting from~$\loc_0(\vec v_0)$ are consistent with
    "simulating"---in the sense of \cref{12-fact-mono}---$\sigma_l$
    from $\loc_0(\vec v_l)$ and $\sigma_r$ from $\loc_0(\vec v_r)$.
    Let $\pi'$ be a finite prefix of~$\pi$.  Then
    $\weight(\pi')=\weight(\pi'(l))+\weight(\pi'(r))$ where $\pi'(l)$
    is a prefix of~$\pi(l)$ and $\pi'(r)$ of~$\pi(r)$, thus
    $\weight(\pi'(l))\leq\vec v_l$ and
    $\weight(\pi'(r))\leq\vec v_r+\weight(\hat\pi)$, thus
    $\weight(\pi')\leq\vec v_0$: the play~$\pi$ avoids the "sink".
    Furthermore, the maximal priority seen infinitely often along
    $\pi(l)$ and $\pi(r)$ is even (note that one of~$\pi(l)$
    and~$\pi(r)$ might not be infinite), thus the maximal priority
    seen infinitely often along~$\pi$ is also even.  This shows
    that~$\sigma$ is winning for \Eve from $\loc_0(\vec v_0)$.\todoquestion{Is
    that clear?}
  \end{scope}
\end{proof}

We are going to exploit \cref{12-counterless} in \cref{12-exist-easy}
in order to prove a~\coNP\ upper bound for "asymmetric games" with
"existential initial credit": it suffices in order to decide those
games to guess a "counterless" winning strategy~$\tau$ for \Adam and
check that it is indeed winning by checking that \Eve loses the
one-player game arising from~$\tau$.  This last step requires an
algorithmic result of independent interest.

\paragraph{One-player Case}
Let $\?V=(\Loc,\Act,\dd)$ be a "vector addition system with states",
$\lcol{:}\,\Loc\to\{1,\dots,d\}$ a location colouring, and
$\loc_0\in\Loc$ an initial location.  Then \Eve wins the
"parity@parity vector game" one-player game from~$\loc_0(\vec v_0)$
for some initial credit~$\vec v_0$ if and only if there exists some
location such that
\begin{itemize}
\item $\loc$ is reachable from~$\loc_0$ in the directed graph
  underlying~$\?V$ and
\item there is a cycle~$\pi\in\Act^\ast$ from $\loc$ to itself such
  that $\weight(\pi)\geq 0$ and the maximal priority occurring
  along~$\pi$ is even.
\end{itemize}
Indeed, assume we can find such a location~$\loc$.  Let
$\hat\pi\in\Act^\ast$ be a path from~$\loc_0$ to~$\loc$ and $\vec
v_0(i)\eqdef\max\{\|\weight(\pi')\|\mid\pi'\text{ is a prefix of
}\hat\pi\pi\}$ for all $1\leq i\leq\dd$.  Then $\loc_0(\vec v_0)$ can
reach $\loc(\vec v_0+\weight(\hat\pi))$ in the "natural semantics"
of~$\?V$ by following~$\hat\pi$, and then $\loc(\vec v_0+\vec
W(\hat\pi)+n\weight(\pi))\geq \loc(\vec v_0+\weight(\hat\pi))$ after
$n$~repetitions of the cycle~$\pi$.  The infinite play arising from
this strategy has an even maximal priority.

Conversely, if \Eve wins, then there is a winning play
$\pi\in\Act^\omega$ from $\loc_0(\vec v_0)$ for some $\vec v_0$.
Recall that $(V,{\leq})$ is a "wqo", and we argue as in
\cref{12-fact-finmem} that there is indeed such a location~$\loc$.

\medskip
Therefore, solving one-player "parity vector games" boils down to
determining the existence of a cycle with non-negative effect and even
maximal priority.  We shall use linear programming techniques in order
to check the existence of such a cycle in polynomial
time~\cite{Kosaraju&Sullivan:1988}.

\medskip
\begin{scope}
\knowledge{non-negative}{notion}
\knowledge{multi-cycle}[multi-cycles]{notion}
\knowledge{suitable}{notion}
Let us start with a relaxed problem: we call a
""multi-cycle"" a non-empty finite set of cycles~$\Pi$ and let
$\weight(\Pi)\eqdef\sum_{\pi\in\Pi}\weight(\pi)$ be its weight; we write
$t\in\Pi$ if~$t\in\pi$ for some $\pi\in\Pi$.
Let $M\in 2^{\Act}$ be a set of `mandatory' subsets of actions and
$F\subseteq\Act$ a set of `forbidden' actions.  Then we say that
$\Pi$ is ""non-negative"" if $\weight(\Pi)\geq\vec 0$, and that it is
""suitable"" for~$(M,F)$ if for all $\Act'\in M$ there exists
$t\in\Act'$ such that $t\in\Pi$, and if for all $t\in F$,
$t\not\in\Pi$.  We use the same terminology for a single cycle~$\pi$.

\begin{lemma}\label{12-lem-zmulticycle}
  Let $\?V$ be a "vector addition system with states",
  $M\in 2^{\Act}$, and $F\subseteq\Act$.  We can check in polynomial
  time whether~$\?V$ contains a "non-negative" "multi-cycle"~$\Pi$
  "suitable" for~$(M,F)$.
\end{lemma}
\begin{proof}
  We reduce the problem to solving a linear program.  For a
  location~$\loc$, let
  $\mathrm{in}(\loc)\eqdef\{(\loc'\step{\vec u}\loc)\in\Act\mid
  \loc'\in\Loc\}$
  and
  $\mathrm{out}(\loc)\eqdef\{(\loc\step{\vec u}\loc')\in\Act\mid
  \loc'\in\Loc\}$ be its sets of incoming and outgoing actions.  The
  linear program has a variable $x_a$ for each action $a\in\Act$,
  which represents the number of times the action~$a$ occurs in
  the "multi-cycle".  It consists of the following contraints:
  \begin{align*}
    \forall\loc&\in\Loc,&\sum_{a\in\mathrm{in}(\loc)}x_a&=\sum_{a\in\mathrm{out}(\loc)}x_a\;,\tag{"multi-cycle"}\\
    \forall a&\in\Act,&x_a&\geq 0\;,\tag{non-negative uses}\\
    \forall i&\in\{1,\dots,\dd\},&\sum_{a\in\Act} x_a\cdot\weight(t)(i)&\geq
                                            0\;,\tag{"non-negative" weight}\\
    &&\sum_{a\in\Act}x_a&\geq 0\tag{non empty}\\
    \forall \Act'&\in M,&\sum_{a\in\Act'}x_a&\geq 0\;,\tag{every subset
                                               in~$M$ is used}\\
    \forall a&\in F,&x_a&= 0\;.\tag{no forbidden actions}
  \end{align*}
  As solving a linear program is in polynomial time~\cite{}\todoquestion{agree
  on a ref with \cref{chap:signal}?}, the result follows.
\end{proof}

Of course, what we are aiming for is finding a "non-negative"
\emph{cycle} "suitable" for $(M,F)$ rather than a "multi-cycle".
Let us define for this the relation $\loc\sim\loc'$ over~$\Loc$ if
$\loc=\loc'$ or if there exists a "non-negative" "multi-cycle"~$\Pi$
"suitable" for~$(M,F)$ such that~$\loc$ and~$\loc'$ belong to some
cycle~$\pi\in\Pi$.
\begin{claim}\label{12-cl-sim} The relation~$\sim$ is an equivalence
  relation.\end{claim}
\begin{proof}
  Symmetry and reflexivity are trivial, and if $\loc\sim\loc'$ and
  $\loc'\sim\loc''$ because~$\loc$ and~$\loc'$ appear in some cycle
  $\pi\in\Pi$ and $\loc'$ and~$\loc''$ in some cycle $\pi'\in\Pi'$ for
  two "non-negative" "multi-cycles"~$\Pi$ and~$\Pi'$ "suitable"
  for~$(M,F)$, then up to a circular shift $\pi$ and~$\pi'$ can be
  assumed to start and end with $\loc'$, and then
  $(\Pi\setminus\{\pi\})\cup(\Pi'\setminus\{\pi'\})\cup\{\pi\pi'\}$ is
  also a "non-negative" "multi-cycle" "suitable" for~$(M,F)$.
\end{proof}

Thus~$\sim$ defines a partition~$\Loc/{\sim}$ of~$\Loc$.
In order to find a "non-negative" cycle~$\pi$ "suitable" for~$(M,F)$,
we are going to compute the partition~$\Loc/{\sim}$ of~$\Loc$
according to~$\sim$.  If we obtain a partition with a single
equivalence class, we are done: there exists such a cycle.  Otherwise,
such a cycle if it exists must be included in one of the subsystems
$(P,\Act\cap(P\times\+Z^\dd\times P),\dd)$ induced by the equivalence
classes $P\in\Loc/{\sim}$.  This yields \cref{12-algo-zcycle}, which
assumes that we know how to compute the partition~$\Loc/{\sim}$.  Note
that the depth of the recursion in \cref{12-algo-zcycle} is bounded
by~$|\Loc|$ and that recursive calls operate over disjoint subsets
of~$\Loc$, thus assuming that we can compute the partition in
polynomial time, then \cref{12-algo-zcycle} also works in polynomial
time.

\begin{algorithm}
 \KwData{A "vector addition system with states"
   $\?V=(\Loc,\Act,\dd)$, $M\in 2^\Act$, $F\subseteq\Act$}

\If{$|\Loc|=1$}
  {\If{$\?V$ has a "non-negative" "multi-cycle" "suitable" for~$(M,F)$}
    {\Return{true}}}

$\Loc/{\sim} \leftarrow \mathrm{partition}(\?V,M,F)$ ;

\If{$|\Loc/{\sim}|=1$}{\Return{true}}

\ForEach{$P\in\Loc/{\sim}$}{\If{$\mathrm{cycle}((P,\Act\cap(P\times\+Z^\dd\times
    P),\dd),M,F)$}{\Return{true}}}

\Return{false}
\caption{$\text{cycle}(\?V,M,F)$}
\label{12-algo-zcycle}
\end{algorithm}

It remains to see how to compute the partition $\Loc/{\sim}$. Consider
for this the set of actions
$\Act'\eqdef\{a\mid\exists\Pi\text{ a "non-negative" "multi-cycle"
  "suitable" for $(M,F)$ with $a\in\Pi$}\}$ and
$\?V'=(\Loc',\Act',\dd)$ the subsystem induced by $\Act'$.
\begin{claim}\label{12-cl-part}
  There exists a path from~$\loc$ to~$\loc'$ in $\?V'$
  if and only if $\loc\sim\loc'$.
\end{claim}
\begin{proof}
  If $\loc\sim\loc'$, then either $\loc=\loc'$ and there is an empty
  path, or there exist~$\Pi$ and~$\pi\in\Pi$ such that $\loc$
  and~$\loc'$ belong to~$\pi$ and $\Pi$ is a "non-negative"
  "multi-cycle" "suitable" for $(M,F)$, thus every action of~$\pi$ is
  in~$\Act'$ and there is a path in~$\?V'$.  

  Conversely, if there is a path $\pi\in{\Act'}^\ast$ from~$\loc$
  to~$\loc'$, then $\loc\sim\loc'$ by induction on~$\pi$.  Indeed, if
  $|\pi|=0$ then $\loc=\loc'$.  For the induction step, $\pi=\pi' a$
  with $\pi'\in{\Act'}^\ast$ a path from $\loc$ to $\loc''$ and
  $a=(\loc''\step{\vec u}\loc')\in\Act'$ for some~$\vec u$.  By
  induction hypothesis, $\loc\sim\loc''$ and since $a\in\Act'$,
  $\loc''\sim\loc'$, thus $\loc\sim\loc'$ by transitivity shown
  in~\cref{12-cl-sim}. 
\end{proof}

By \cref{12-cl-part}, the equivalence classes of~$\sim$ are the
strongly connected components of~$\?V'$.  This yields the following
polynomial time algorithm for computing~$\Loc/{\sim}$.

\begin{algorithm}
 \KwData{A "vector addition system with states"
   $\?V=(\Loc,\Act,\dd)$, $M\in 2^\Act$, $F\subseteq\Act$}

$\Act'\leftarrow\emptyset$;

\ForEach{$a\in\Act$}{\If{$\?V$ has a "non-negative" "multi-cycle"
    "suitable"
    for~$(M\cup\{\{a\}\},F)$}{$\Act'\leftarrow\Act'\cup\{a\}$}}

$\?V'\leftarrow \text{subsystem induced by~$\Act'$}$ ;

\Return{$\mathrm{SCC}(\?V')$}
\caption{$\text{partition}(\?V,M,F)$}
\label{12-algo-part}
\end{algorithm}

Together, \cref{12-lem-zmulticycle}
and \cref{12-algo-part,12-algo-zcycle} yield the following.

\begin{lemma}\label{12-lem-zcycle}
  Let $\?V$ be a "vector addition system with states",
  $M\in 2^{\Act}$, and $F\subseteq\Act$.  We can check in polynomial
  time whether~$\?V$ contains a "non-negative" cycle~$\pi$
  "suitable" for~$(M,F)$.
\end{lemma}

Finally, we obtain the desired polynomial time upper bound for
"parity@parity vector games" in "vector addition systems with states".
\begin{theorem}\label{12-thm-zcycle}
  Whether \Eve wins a one-player "parity vector game" with
  "existential initial credit" is in~\P.
\end{theorem}
\begin{proof}
  Let $\?V=(\Loc,\Act,\dd)$ be a "vector addition system with states",
  $\lcol{:}\,\Loc\to\{1,\dots,$ $d\}$ a location colouring, and
  $\loc_0\in\Loc$ an initial location.  We start by trimming~$\?V$ to
  only keep the locations reachable from~$\loc_0$ in the underlying
  directed graph.  Then, for every even priority $p\in\{1,\dots,d\}$,
  we use \cref{12-lem-zcycle} to check for the existence of a
  "non-negative" cycle with maximal priority~$p$: it suffices for this
  to set $M\eqdef\{\lcol^{-1}(p)\}$ and
  $F\eqdef\lcol^{-1}(\{p+1,\dots,d\})$.
\end{proof}
\end{scope}

\paragraph{Upper Bounds}
We are now equipped to prove our upper bounds.  We begin with a nearly
trivial case.  In a "coverability" "asymmetric vector game" with
"existential initial credit", the counters play no role at all: \Eve
has a winning strategy for some initial credit in the "vector game" if
and only if she has one to reach the target location~$\loc_f$ in the
finite game played over~$\Loc$ and edges~$(\loc,\loc')$ whenever
$\loc\step{\vec u}\loc'\in\Act$ for some~$\vec u$.  This entails that
"coverability" "asymmetric vector games" are quite easy to solve.

\begin{theorem}\label{12-cov-exist-P}
  "Coverability" "asymmetric" "vector games" with "existential initial
  credit" are \P-complete.
\end{theorem}

Regarding "non-termination" and "parity@parity vector game", we
exploit \cref{12-counterless,12-thm-zcycle}.

\begin{theorem}\label{12-exist-easy}
  "Non-termination" and "parity@parity vector game" "asymmetric"
  "vector games" with "existential initial credit" are in~\coNP.
\end{theorem}
\begin{proof}
  By \cref{12-nonterm2parity}, it suffices to prove the statement for
  "parity@parity vector games" games.  By \cref{12-counterless},
  if \Adam wins the game, we can guess a "counterless" winning
  strategy~$\tau$ telling which action to choose for every location.
  This strategy yields a one-player game, and by \cref{12-thm-zcycle}
  we can check in polynomial time that~$\tau$ was indeed winning
  for~\Adam.
\end{proof}

Finally, in fixed dimension and with a fixed number of priorities, we
can simply apply the results of \cref{12-bounding}.
\begin{corollary}\label{12-exist-pseudop}
  "Parity@parity vector game" "asymmetric" "vector games" with
  "existential initial credit" are in pseudo-polynomial time if the
  dimension and the number of priorities are fixed.
\end{corollary}
\begin{proof}
  Consider an "asymmetric vector system"
  $\?V=(\Loc,\Act,\Loc_\mEve,\Loc_\mAdam,\dd)$ and a location
  colouring $\lcol{:}\,\Loc\to\{1,\dots,2d\}$.
  By \cref{12-parity2bounding}, the "parity vector game" with
  "existential initial credit" over~$\?V$ problem reduces to a
  "bounding game" with "existential initial credit" over a "vector
  system"~$\?V'=(\Loc',\Act',\Loc'_\mEve,\Loc'_\mAdam,\dd+d)$ where
  $\Loc'\in O(|\Loc|)$ and $\|\Act'\|=\|\Act\|$.
  By \cref{12-th-bounding}, it suffices to consider the case of a
  "non-termination" game with "existential initial credit" played over
  the "bounded semantics" $\bounded(\?V')$ where $B$ is in
  $(|\Loc'|\cdot\|\Act'\|)^{O(\dd+d)^3}$.  Such a game can be solved in
  linear time in the size of the bounded arena using attractor
  techniques, thus in $O(|\Loc|\cdot B)^{\dd+d}$, which is in
  $(|\Loc|\cdot\|\Act\|)^{O(\dd+d)^4}$ in terms of the original instance.
\end{proof}

\subsubsection{Given Initial Credit}
\label{12-sub-up-given}
\TODO{\Cref{12-sub-up-given}}

\begin{theorem}\label{12-avag-easy}
  "Coverability", "non-termination", and "parity@parity vector game"
  "asymmetric" "vector games" with "given initial credit" are in
  \kEXP[2].  If the dimension is fixed, they are in \EXP, and if the
  number of priorities is also fixed, they are in pseudo-polynomial
  time.
\end{theorem}

% Local IspellDict: british

\subsubsection{Existential Initial Credit}
\label{11-sec:up-exist}

\paragraph{Counterless Strategies}
Consider a "strategy"~$\tau$ of \Adam\ in a "vector game".  In all the
games we consider, "uniform" "positional" strategies suffice over the
infinite "arena" $\natural(\?V)=(V,E,\VE,\VA)$: $\tau$ maps vertices
in~$V$ to edges in~$E$.  We call~$\tau$ ""counterless"" if, for all
locations $\loc\in\Loc_\mAdam$ and all vectors
$\vec v,\vec v'\in\+N^\dd$, $\tau(\loc(\vec v))=\tau(\loc(\vec v'))$.
A "counterless" strategy thus only considers the current location of
the play.
\begin{lemma}[Counterless strategies]
\label{11-lem:counterless}
  Let $\?V=(\Loc,\Act,\Loc_\mEve,\Loc_\mAdam,\dd)$ be an "asymmetric
  vector system", $\loc_0\in\Loc$ be a location, and
  $\lcol{:}\,\Loc\to\{1,\dots,d\}$ be a location colouring.  If \Adam
  wins from $\loc_0(\vec v)$ for every initial credit~$\vec v$ in the
  "parity@parity vector game" game played over $\?V$ with~$\lcol$, then
  he has a single "counterless strategy" such that he wins from
  $\loc_0(\vec v)$ for every initial credit~$\vec v$.
\end{lemma}
\begin{proof}
  Let $\Act_\mAdam\eqdef\{(\loc\step{\vec
    u}\loc')\in\Act\mid\loc\in\Loc_\mAdam\}$ be the set of actions
  controlled by \Adam.  We assume without loss of generality that
  every location $\loc\in\Loc_\mAdam$ has either one or two outgoing
  actions, thus $|\Loc_\mAdam|\leq|\Act_\mAdam|\leq
  2|\Loc_\mAdam|$.  We proceed by induction over $|\Act_\mAdam|$.  For
  the base case, if $|\Act_\mAdam|=|\Loc_\mAdam|$ then every location
  controlled by \Adam\ has a single outgoing action, thus any
  strategy for \Adam\ is trivially "counterless".

  For the induction step, consider some location
  $\hat\loc\in\Loc_\mAdam$ with two outgoing actions
  $a_l\eqdef\hat\loc\step{\vec 0}\loc_l$ and
  $a_r\eqdef\hat\loc\step{\vec 0}\loc_r$.  Let $\?V_l$ and $\?V_r$ be
  the "vector systems" obtained from~$\?V$ by removing
  respectively~$a_r$ and~$a_l$ from~$\Act$, i.e., by using
  $\Act_l\eqdef\Act\setminus\{a_r\}$ and
  $\Act_r\eqdef\Act\setminus\{a_l\}$.  If $\Adam$ wins the
  "parity@parity vector game" game from $\loc(\vec v)$ for every
  initial credit~$\vec v$ in either $\?V_l$ or $\?V_r$, then by
  induction hypothesis he has a "counterless" winning strategy winning
  from $\loc(\vec v)$ for every initial credit~$\vec v$, and the same
  strategy is winning in~$\?V$ from $\loc(\vec v)$ for every initial
  credit~$\vec v$.

  In order to conclude the proof, we show that, if \Adam\ loses in
  $\?V_l$ from $\loc_0(\vec v_l)$ for some $\vec v_l\in\+N^\dd$ and in
  $\?V_r$ from $\loc_0(\vec v_r)$ for some $\vec v_r\in\+N^\dd$, then
  there exists $\vec v_0\in\+N^\dd$ such that \Eve\ wins from
  $\loc_0(\vec v_0)$ in~$\?V$.  Let $\sigma_l$ and $\sigma_r$ denote
  \Eve's winning strategies in the two games.  By a slight abuse of
  notations (justified by the fact that we are only interested in a
  few initial vertices), we see plays as sequences of actions and
  strategies as maps $\Act^\ast\to\Act$.\todoquestion{I hope this is not too messy}  Consider the set of
  plays consistent with~$\sigma_r$ starting from $\loc_0(\vec v_r)$.
  If none of those plays visits $\hat\loc$, then $\Eve$ wins in $\?V$
  from $\loc_0(\vec v_r)$ and we conclude.  Otherwise, there is some
  finite prefix~$\hat\pi$ of a play that
  visits~$\hat\loc(\hat{\vec v})$ for some vector
  $\hat{\vec v}=\vec v_r+\weight(\hat\pi)$.  We let
  $\vec v_0\eqdef\vec v_l+\hat{\vec v}$ and show that \Eve\ wins from
  $\loc_0(\vec v_0)$.

  \begin{scope}\knowledge{mode}{notion}
    We define now a strategy $\sigma$ for $\Eve$ over~$\?V$ that
    switches between applying~$\sigma_l$ and~$\sigma_r$ each time
    $a_r$ is used and switches back each time~$a_l$ is used.  More
    precisely, given a finite or infinite sequence~$\pi$ of actions,
    we decompose $\pi$ as $\pi_1 a_1 \pi_2 a_2 \pi_3\cdots$ where each
    segment $\pi_j\in(\Act\setminus\{a_l,a_r\})^\ast$ does not use
    either~$a_l$ nor~$a_r$ and each $a_j\in\{a_l,a_r\}$.  The
    associated ""mode"" $m(j)\in\{l,r\}$ of a segment~$\pi_j$
    is~$m(1)\eqdef l$ for the initial segment and otherwise
    $m(j)\eqdef l$ if $e_{j-1}=a_l$ and $m(j)\eqdef r$ otherwise.  The
    $l$-subsequence associated with $\pi$ is the sequence of segments
    $\pi(l)\eqdef\pi_{l_1}a_{l_2-1}\pi_{l_2}a_{l_3-1}\pi_{l_3}\cdots$
    with "mode"~$m(l_i)=l$, while the $r$-subsequence is the sequence
    $\pi(r)\eqdef\hat\pi a_{r_1-1}\pi_{r_1}a_{r_2-1}\pi_{r_2}\cdots$
    with "mode"~$m(r_i)=r$ prefixed by~$\hat\pi$.  Then we let
    $\sigma(\pi)\eqdef\sigma_{m}(\pi(m))$ where $m\in\{l,r\}$ is the
    "mode" of the last segment of~$\pi$.

    Consider an infinite play $\pi$ consistent with~$\sigma$ starting
    from~$\loc_0(\vec v_0)$.  Since $\vec v_0\geq\vec v_l$ and
    $\vec v_0\geq \vec v_r+\weight(\hat\pi)$, $\pi(l)$ and $\pi(r)$
    starting from~$\loc_0(\vec v_0)$ are consistent with
    "simulating"---in the sense of \cref{11-lem:mono}---$\sigma_l$
    from $\loc_0(\vec v_l)$ and $\sigma_r$ from $\loc_0(\vec v_r)$.
    Let $\pi'$ be a finite prefix of~$\pi$.  Then
    $\weight(\pi')=\weight(\pi'(l))+\weight(\pi'(r))$ where $\pi'(l)$
    is a prefix of~$\pi(l)$ and $\pi'(r)$ of~$\pi(r)$, thus
    $\weight(\pi'(l))\leq\vec v_l$ and
    $\weight(\pi'(r))\leq\vec v_r+\weight(\hat\pi)$, thus
    $\weight(\pi')\leq\vec v_0$: the play~$\pi$ avoids the "sink".
    Furthermore, the maximal priority seen infinitely often along
    $\pi(l)$ and $\pi(r)$ is even (note that one of~$\pi(l)$
    and~$\pi(r)$ might not be infinite), thus the maximal priority
    seen infinitely often along~$\pi$ is also even.  This shows
    that~$\sigma$ is winning for \Eve\ from $\loc_0(\vec v_0)$.\todoquestion{Is
    that clear?}
  \end{scope}
\end{proof}

We are going to exploit \cref{11-lem:counterless}
in \cref{11-th:exist-easy} in order to prove a~\coNP\ upper bound for
"asymmetric games" with "existential initial credit": it suffices in
order to decide those games to guess a "counterless" winning
strategy~$\tau$ for \Adam\ and check that it is indeed winning by
checking that \Eve\ loses the one-player game arising from~$\tau$.
This last step requires an algorithmic result of independent interest.

\paragraph{One-player Case}
Let $\?V=(\Loc,\Act,\dd)$ be a "vector addition system with states",
$\lcol{:}\,\Loc\to\{1,\dots,d\}$ a location colouring, and
$\loc_0\in\Loc$ an initial location.  Then \Eve\ wins the
"parity@parity vector game" one-player game from~$\loc_0(\vec v_0)$
for some initial credit~$\vec v_0$ if and only if there exists some
location such that
\begin{itemize}
\item $\loc$ is reachable from~$\loc_0$ in the directed graph
  underlying~$\?V$ and
\item there is a cycle~$\pi\in\Act^\ast$ from $\loc$ to itself such
  that $\weight(\pi)\geq 0$ and the maximal priority occurring
  along~$\pi$ is even.
\end{itemize}
Indeed, assume we can find such a location~$\loc$.  Let
$\hat\pi\in\Act^\ast$ be a path from~$\loc_0$ to~$\loc$ and $\vec
v_0(i)\eqdef\max\{\|\weight(\pi')\|\mid\pi'\text{ is a prefix of
}\hat\pi\pi\}$ for all $1\leq i\leq\dd$.  Then $\loc_0(\vec v_0)$ can
reach $\loc(\vec v_0+\weight(\hat\pi))$ in the "natural semantics"
of~$\?V$ by following~$\hat\pi$, and then $\loc(\vec v_0+\vec
W(\hat\pi)+n\weight(\pi))\geq \loc(\vec v_0+\weight(\hat\pi))$ after
$n$~repetitions of the cycle~$\pi$.  The infinite play arising from
this strategy has an even maximal priority.

Conversely, if \Eve\ wins, then there is a winning play
$\pi\in\Act^\omega$ from $\loc_0(\vec v_0)$ for some $\vec v_0$.
Recall that $(V,{\leq})$ is a "wqo", and we argue as in
\cref{11-lem:finmem} that there is indeed such a location~$\loc$.

\medskip
Therefore, solving one-player "parity vector games" boils down to
determining the existence of a cycle with non-negative effect and even
maximal priority.  We shall use linear programming techniques in order
to check the existence of such a cycle in polynomial
time~\cite{Kosaraju&Sullivan:1988}.

\medskip
\begin{scope}
\knowledge{non-negative}{notion}
\knowledge{multi-cycle}[multi-cycles]{notion}
\knowledge{suitable}{notion}
Let us start with a relaxed problem: we call a
""multi-cycle"" a non-empty finite set of cycles~$\Pi$ and let
$\weight(\Pi)\eqdef\sum_{\pi\in\Pi}\weight(\pi)$ be its weight; we write
$t\in\Pi$ if~$t\in\pi$ for some $\pi\in\Pi$.
Let $M\in 2^{\Act}$ be a set of `mandatory' subsets of actions and
$F\subseteq\Act$ a set of `forbidden' actions.  Then we say that
$\Pi$ is ""non-negative"" if $\weight(\Pi)\geq\vec 0$, and that it is
""suitable"" for~$(M,F)$ if for all $\Act'\in M$ there exists
$t\in\Act'$ such that $t\in\Pi$, and if for all $t\in F$,
$t\not\in\Pi$.  We use the same terminology for a single cycle~$\pi$.

\begin{lemma}[Linear programs for suitable non-negative multi-cycles]
\label{11-lem:zmulticycle}
  Let $\?V$ be a "vector addition system with states",
  $M\in 2^{\Act}$, and $F\subseteq\Act$.  We can check in polynomial
  time whether~$\?V$ contains a "non-negative" "multi-cycle"~$\Pi$
  "suitable" for~$(M,F)$.
\end{lemma}
\begin{proof}
  We reduce the problem to solving a linear program.  For a
  location~$\loc$, let
  $\mathrm{in}(\loc)\eqdef\{(\loc'\step{\vec u}\loc)\in\Act\mid
  \loc'\in\Loc\}$
  and
  $\mathrm{out}(\loc)\eqdef\{(\loc\step{\vec u}\loc')\in\Act\mid
  \loc'\in\Loc\}$ be its sets of incoming and outgoing actions.  The
  linear program has a variable $x_a$ for each action $a\in\Act$,
  which represents the number of times the action~$a$ occurs in
  the "multi-cycle".  It consists of the following contraints:
  \begin{align*}
    \forall\loc&\in\Loc,&\sum_{a\in\mathrm{in}(\loc)}x_a&=\sum_{a\in\mathrm{out}(\loc)}x_a\;,\tag{"multi-cycle"}\\
    \forall a&\in\Act,&x_a&\geq 0\;,\tag{non-negative uses}\\
    \forall i&\in\{1,\dots,\dd\},&\sum_{a\in\Act} x_a\cdot\weight(t)(i)&\geq
                                            0\;,\tag{"non-negative" weight}\\
    &&\sum_{a\in\Act}x_a&\geq 0\tag{non empty}\\
    \forall \Act'&\in M,&\sum_{a\in\Act'}x_a&\geq 0\;,\tag{every subset
                                               in~$M$ is used}\\
    \forall a&\in F,&x_a&= 0\;.\tag{no forbidden actions}
  \end{align*}
  As solving a linear program is in polynomial time~\cite{}\todoquestion{agree
  on a ref with \cref{8-chap:signal}?}, the result follows.
\end{proof}

Of course, what we are aiming for is finding a "non-negative"
\emph{cycle} "suitable" for $(M,F)$ rather than a "multi-cycle".
Let us define for this the relation $\loc\sim\loc'$ over~$\Loc$ if
$\loc=\loc'$ or if there exists a "non-negative" "multi-cycle"~$\Pi$
"suitable" for~$(M,F)$ such that~$\loc$ and~$\loc'$ belong to some
cycle~$\pi\in\Pi$.
\begin{claim}\label{11-cl:sim} The relation~$\sim$ is an equivalence
  relation.\end{claim}
\begin{proof}
  Symmetry and reflexivity are trivial, and if $\loc\sim\loc'$ and
  $\loc'\sim\loc''$ because~$\loc$ and~$\loc'$ appear in some cycle
  $\pi\in\Pi$ and $\loc'$ and~$\loc''$ in some cycle $\pi'\in\Pi'$ for
  two "non-negative" "multi-cycles"~$\Pi$ and~$\Pi'$ "suitable"
  for~$(M,F)$, then up to a circular shift $\pi$ and~$\pi'$ can be
  assumed to start and end with $\loc'$, and then
  $(\Pi\setminus\{\pi\})\cup(\Pi'\setminus\{\pi'\})\cup\{\pi\pi'\}$ is
  also a "non-negative" "multi-cycle" "suitable" for~$(M,F)$.
\end{proof}

Thus~$\sim$ defines a partition~$\Loc/{\sim}$ of~$\Loc$.
In order to find a "non-negative" cycle~$\pi$ "suitable" for~$(M,F)$,
we are going to compute the partition~$\Loc/{\sim}$ of~$\Loc$
according to~$\sim$.  If we obtain a partition with a single
equivalence class, we are done: there exists such a cycle.  Otherwise,
such a cycle if it exists must be included in one of the subsystems
$(P,\Act\cap(P\times\+Z^\dd\times P),\dd)$ induced by the equivalence
classes $P\in\Loc/{\sim}$.  This yields \cref{11-algo:zcycle}, which
assumes that we know how to compute the partition~$\Loc/{\sim}$.  Note
that the depth of the recursion in \cref{11-algo:zcycle} is bounded
by~$|\Loc|$ and that recursive calls operate over disjoint subsets
of~$\Loc$, thus assuming that we can compute the partition in
polynomial time, then \cref{11-algo:zcycle} also works in polynomial
time.

\begin{algorithm}
 \KwData{A "vector addition system with states"
   $\?V=(\Loc,\Act,\dd)$, $M\in 2^\Act$, $F\subseteq\Act$}

\If{$|\Loc|=1$}
  {\If{$\?V$ has a "non-negative" "multi-cycle" "suitable" for~$(M,F)$}
    {\Return{true}}}

$\Loc/{\sim} \leftarrow \mathrm{partition}(\?V,M,F)$ ;

\If{$|\Loc/{\sim}|=1$}{\Return{true}}

\ForEach{$P\in\Loc/{\sim}$}{\If{$\mathrm{cycle}((P,\Act\cap(P\times\+Z^\dd\times
    P),\dd),M,F)$}{\Return{true}}}

\Return{false}
\caption{$\text{cycle}(\?V,M,F)$}
\label{11-algo:zcycle}
\end{algorithm}

It remains to see how to compute the partition $\Loc/{\sim}$. Consider
for this the set of actions
$\Act'\eqdef\{a\mid\exists\Pi\text{ a "non-negative" "multi-cycle"
  "suitable" for $(M,F)$ with $a\in\Pi$}\}$ and
$\?V'=(\Loc',\Act',\dd)$ the subsystem induced by $\Act'$.
\begin{claim}
\label{11-cl:part}
  There exists a path from~$\loc$ to~$\loc'$ in $\?V'$
  if and only if $\loc\sim\loc'$.
\end{claim}
\begin{proof}
  If $\loc\sim\loc'$, then either $\loc=\loc'$ and there is an empty
  path, or there exist~$\Pi$ and~$\pi\in\Pi$ such that $\loc$
  and~$\loc'$ belong to~$\pi$ and $\Pi$ is a "non-negative"
  "multi-cycle" "suitable" for $(M,F)$, thus every action of~$\pi$ is
  in~$\Act'$ and there is a path in~$\?V'$.  

  Conversely, if there is a path $\pi\in{\Act'}^\ast$ from~$\loc$
  to~$\loc'$, then $\loc\sim\loc'$ by induction on~$\pi$.  Indeed, if
  $|\pi|=0$ then $\loc=\loc'$.  For the induction step, $\pi=\pi' a$
  with $\pi'\in{\Act'}^\ast$ a path from $\loc$ to $\loc''$ and
  $a=(\loc''\step{\vec u}\loc')\in\Act'$ for some~$\vec u$.  By
  induction hypothesis, $\loc\sim\loc''$ and since $a\in\Act'$,
  $\loc''\sim\loc'$, thus $\loc\sim\loc'$ by transitivity shown
  in~\cref{11-cl:sim}. 
\end{proof}

By \cref{11-cl:part}, the equivalence classes of~$\sim$ are the
strongly connected components of~$\?V'$.  This yields the following
polynomial time algorithm for computing~$\Loc/{\sim}$.

\begin{algorithm}
 \KwData{A "vector addition system with states"
   $\?V=(\Loc,\Act,\dd)$, $M\in 2^\Act$, $F\subseteq\Act$}

$\Act'\leftarrow\emptyset$;

\ForEach{$a\in\Act$}{\If{$\?V$ has a "non-negative" "multi-cycle"
    "suitable"
    for~$(M\cup\{\{a\}\},F)$}{$\Act'\leftarrow\Act'\cup\{a\}$}}

$\?V'\leftarrow \text{subsystem induced by~$\Act'$}$ ;

\Return{$\mathrm{SCC}(\?V')$}
\caption{$\text{partition}(\?V,M,F)$}
\label{11-algo:part}
\end{algorithm}

Together, \cref{11-lem:zmulticycle}
and \cref{11-algo:part,11-algo:zcycle} yield the following.

\begin{lemma}[Polynomial-time detection of suitable non-negative cycles]
\label{11-lem:zcycle}
  Let $\?V$ be a "vector addition system with states",
  $M\in 2^{\Act}$, and $F\subseteq\Act$.  We can check in polynomial
  time whether~$\?V$ contains a "non-negative" cycle~$\pi$
  "suitable" for~$(M,F)$.
\end{lemma}

Finally, we obtain the desired polynomial time upper bound for
"parity@parity vector games" in "vector addition systems with states".
\begin{theorem}[Existential one-player parity vector games are in~\P]
\label{11-thm:zcycle}
  Whether \Eve\ wins a one-player "parity vector game" with
  "existential initial credit" is in~\P.
\end{theorem}
\begin{proof}
  Let $\?V=(\Loc,\Act,\dd)$ be a "vector addition system with states",
  $\lcol{:}\,\Loc\to\{1,\dots,$ $d\}$ a location colouring, and
  $\loc_0\in\Loc$ an initial location.  We start by trimming~$\?V$ to
  only keep the locations reachable from~$\loc_0$ in the underlying
  directed graph.  Then, for every even priority $p\in\{1,\dots,d\}$,
  we use \cref{11-lem:zcycle} to check for the existence of a
  "non-negative" cycle with maximal priority~$p$: it suffices for this
  to set $M\eqdef\{\lcol^{-1}(p)\}$ and
  $F\eqdef\lcol^{-1}(\{p+1,\dots,d\})$.
\end{proof}
\end{scope}

\paragraph{Upper Bounds}
We are now equipped to prove our upper bounds.  We begin with a nearly
trivial case.  In a "coverability" "asymmetric vector game" with
"existential initial credit", the counters play no role at all: \Eve
has a winning strategy for some initial credit in the "vector game" if
and only if she has one to reach the target location~$\loc_f$ in the
finite game played over~$\Loc$ and edges~$(\loc,\loc')$ whenever
$\loc\step{\vec u}\loc'\in\Act$ for some~$\vec u$.  This entails that
"coverability" "asymmetric vector games" are quite easy to solve.

\begin{theorem}[Existential coverability asymmetric vector games are in~\P]
\label{11-th:cov-exist-P}
  "Coverability" "asymmetric" "vector games" with "existential initial
  credit" are \P-complete.
\end{theorem}

Regarding "non-termination" and "parity@parity vector game", we
exploit \cref{11-lem:counterless,11-thm:zcycle}.

\begin{theorem}[Existential parity asymmetric vector games are in~\coNP]
\label{11-th:exist-easy}
  "Non-termination" and "parity@parity vector game" "asymmetric"
  "vector games" with "existential initial credit" are in~\coNP.
\end{theorem}
\begin{proof}
  By \cref{11-rk:nonterm2parity}, it suffices to prove the statement for
  "parity@parity vector games" games.  By \cref{11-lem:counterless},
  if \Adam\ wins the game, we can guess a "counterless" winning
  strategy~$\tau$ telling which action to choose for every location.
  This strategy yields a one-player game, and by \cref{11-thm:zcycle}
  we can check in polynomial time that~$\tau$ was indeed winning
  for~\Adam.
\end{proof}

Finally, in fixed dimension and with a fixed number of priorities, we
can simply apply the results of \cref{11-sec:bounding}.
\begin{corollary}[Existential fixed-dimensional parity asymmetric vector games are pseudo-polynomial]
\label{11-cor:exist-pseudop}
  "Parity@parity vector game" "asymmetric" "vector games" with
  "existential initial credit" are in pseudo-polynomial time if the
  dimension and the number of priorities are fixed.
\end{corollary}
\begin{proof}
  Consider an "asymmetric vector system"
  $\?V=(\Loc,\Act,\Loc_\mEve,\Loc_\mAdam,\dd)$ and a location
  colouring $\lcol{:}\,\Loc\to\{1,\dots,2d\}$.
  By \cref{11-lem:parity2bounding}, the "parity vector game" with
  "existential initial credit" over~$\?V$ problem reduces to a
  "bounding game" with "existential initial credit" over a "vector
  system"~$\?V'=(\Loc',\Act',\Loc'_\mEve,\Loc'_\mAdam,\dd+d)$ where
  $\Loc'\in O(|\Loc|)$ and $\|\Act'\|=\|\Act\|$.
  By \cref{11-th:bounding}, it suffices to consider the case of a
  "non-termination" game with "existential initial credit" played over
  the "bounded semantics" $\bounded(\?V')$ where $B$ is in
  $(|\Loc'|\cdot\|\Act'\|)^{O(\dd+d)^3}$.  Such a game can be solved in
  linear time in the size of the bounded arena using attractor
  techniques, thus in $O(|\Loc|\cdot B)^{\dd+d}$, which is in
  $(|\Loc|\cdot\|\Act\|)^{O(\dd+d)^4}$ in terms of the original instance.
\end{proof}

\subsubsection{Given Initial Credit}
\label{11-sec:up-given}
\TODO{\Cref{11-sec:up-given}}

\begin{theorem}[Upper bounds for asymmetric vector games]
\label{11-th:avag-easy}
  "Coverability", "non-termination", and "parity@parity vector game"
  "asymmetric" "vector games" with "given initial credit" are in
  \kEXP[2].  If the dimension is fixed, they are in \EXP, and if the
  number of priorities is also fixed, they are in pseudo-polynomial
  time.
\end{theorem}

% Local IspellDict: british

\subsection{Lower Bounds}
\label{11-sec:low}
Let us turn our attention to complexity lower bounds for "monotonic"
"asymmetric vector games".  It turns out that most of the upper bounds
shown in \cref{11-sec:up} are tight.
%\subsubsection{Existential Initial Credit}
In the "existential initial credit" variant of our games, we have the
following lower bound matching \cref{11-exist-easy}, already with a
unary encoding.

\begin{theorem}\label{11-exist-hard}
  "Non-termination", and "parity@parity vector game"
  "asymmetric" "vector games" with "existential initial credit" are
  \coNP-hard.% in any dimension~$\dd\geq 2$.
\end{theorem}
\begin{proof}
  By \cref{11-nonterm2parity}, it suffices to show hardness for
  "non-termination games".  We reduce from the \lang{3SAT} problem:
  given a formula $\varphi=\bigwedge_{1\leq i\leq m}C_i$ where each
  clause $C_i$ is a disjonction of the form
  $\litt_{i,1}\vee\litt_{i,2}\vee\litt_{i,3}$ of literals taken from
  $X=\{x_1,\neg x_1,x_2,$ $\neg x_2,\dots,x_k,\neg x_k\}$, we construct
  an "asymmetric" "vector system" $\?V$ where Eve wins the
  "non-termination game" with "existential initial credit" if and only
  if~$\varphi$ is not satisfiable; since the game is determined, we
  actually show that Adam wins the game if and only if~$\varphi$ is
  satisfiable.

  Our "vector system" has dimension~$2k$, and for a literal
  $\litt\in X$, we define the vector
  \begin{equation*}
    \vec u_\litt\eqdef\begin{cases}
      \vec e_{2n-1}-\vec e_{2n}&\text{if }\litt=x_n\;,\\
      \vec e_{2n}-\vec e_{2n-1}&\text{if }\litt=\neg x_n\;.
    \end{cases}
  \end{equation*}
  We define $\?V\eqdef(\Loc,\Act,\Loc_\mEve,\Loc_\mAdam,2k)$ where
  \begin{align*}
    \Loc_\mEve&\eqdef\{\varphi\}\cup\{\litt_{i,j}\mid 1\leq i\leq m,1\leq j\leq
                3\}\;,\\
    \Loc_\mAdam&\eqdef\{C_i\mid 1\leq i\leq m\}\;,\\
    \Act&\eqdef\{\varphi\step{\vec 0}C_i\mid 1\leq i\leq m\}\cup\{C_i\step{\vec 0}\litt_{i,j},\;\;\litt_{i,j}\xrightarrow{\vec u_{\litt_{i,j}}}\varphi\mid 1\leq i\leq m,1\leq j\leq 3\}\;.
  \end{align*}
  \begin{scope}
    We use~$\varphi$ as our initial location.
    \knowledge{literal assignment}{notion}
    \knowledge{conflicting}{notion}
    %
    Let us call a map $v{:}\,X\to\{0,1\}$ a ""literal assignment""; we
    call it ""conflicting"" if there exists $1\leq n\leq k$ such that
    $v(x_n)=v(\neg x_n)$.

    Assume that~$\varphi$ is satisfiable.  Then there exists a
    non-"conflicting" "literal assignment"~$v$ that satisfies all the
    clauses: for each $1\leq i\leq m$, there exists $1\leq j\leq 3$
    such that $v(\litt_{i,j})=1$; this yields a "counterless" strategy
    for Adam, which selects $(C_i,\litt_{i,j})$ for each
    $1\leq i\leq m$.  Consider any infinite "play" consistent with
    this strategy.  This "play" only visits literals $\litt$ where
    $v(\litt)=1$.  There exists a literal $\litt\in X$ that is visited
    infinitely often along the "play", say $\litt=x_n$.  Because~$v$ is
    non-"conflicting", $v(\neg x_n)=0$, thus the location $\neg x_n$
    is never visited.  Thus the play uses the action
    $\litt\step{\vec e_{2n-1}-\vec e_{2n}}\varphi$ infinitely often,
    and never uses any action with a positive effect on
    component~$2n$.  Hence the play is losing from any initial credit.

    Conversely, assume that~$\varphi$ is not satisfiable.  By
    contradiction, assume that Adam wins the game for all initial
    credits.  By \cref{11-counterless}, he has a "counterless" winning
    strategy~$\tau$ that selects a literal in every clause.  Consider
    a "literal assignment" that maps each one of the selected literals
    to~$1$ and the remaining ones in a non-conflicting manner.  By
    definition, this "literal assignment" satisfies all the clauses,
    but because~$\varphi$ is not satisfiable, it is "conflicting":
    necessarily, there exist $1\leq n\leq k$ and $1\leq i,i'\leq m$,
    such that $\tau$ selects $x_n$ in $C_i$ and $\neg x_n$ in
    $C_{i'}$.  But this yields a winning strategy for Eve, which
    alternates in the initial location $\varphi$ between $C_{i}$
    and $C_{i'}$, and for which an initial credit
    $\vec e_{2n-1}+\vec e_{2n}$ suffices: a contradiction.
  \end{scope}
  % We reduce from the \lang{Partition} problem: given a finite set
  % $S=\{w_0,\dots,w_{n-1}\}\subseteq\+N$ (encoded in binary), does
  % there exist a partition $S_1,S_2$ of~$S$ such that
  % $\sum_{w\in S_1}w=\sum_{w\in S_2}w$?  From such an instance, we
  % construct an "asymmetric vector system" where Eve wins the
  % "non-termination game" with "existential initial credit" if and only
  % if there is no such partition.  By \cref{11-nonterm2parity}, this
  % also entails the \coNP-hardness of "parity@parity vector games"
  % "asymmetric vector games".
  % \begin{figure}[htbp]
  % \centering
  % \begin{tikzpicture}[auto,on grid,node distance=1.8cm]
  %   \node[adam,inner sep=2.1](0){$\loc_0$};
  %   \node[eve,above right=1 and 1 of 0,inner sep=2.2](01){$\loc_{0,1}$};
  %   \node[eve,below right=1 and 1 of 0,inner sep=2.2](02){$\loc_{0,2}$};
  %   \node[adam,below right=1 and 2 of 01,inner sep=2.1](1){$\loc_1$};
  %   \node[eve,above right=1 and 1 of 1,inner sep=2.2](11){$\loc_{1,1}$};
  %   \node[eve,below right=1 and 1 of 1,inner sep=2.2](12){$\loc_{1,2}$};
  %   \node[adam,below right=1 and 2 of 11,inner sep=2.1](2){$\loc_2$};
  %   \node[right=1 of 2](dots){$\Huge\cdots$};
  %   \node[adam,right=1 of dots,inner sep=-1.8](nm){$\loc_{n-1}$};
  %   \node[eve,above right=1 and 1 of nm,inner sep=0](nm1){$\loc_{n-1,1}$};
  %   \node[eve,below right=1 and 1 of nm,inner sep=0](nm2){$\loc_{n-1,2}$};
  %   \node[eve,below right=1 and 2 of nm1,inner sep=2.2](n){$\loc_n$};    
  %   \path[->,every node/.style={font=\footnotesize,inner sep=.1}]
  %   (0) edge node {$\vec 0$} (01)
  %   (0) edge[swap] node {$\vec 0$} (02)
  %   (01) edge[near start] node {$\!-w_0\cdot\vec e_1$} (1)
  %   (02) edge[swap,near start] node {$\!-w_0\cdot\vec e_2$} (1)
  %   (1) edge node {$\vec 0$} (11)
  %   (1) edge[swap] node {$\vec 0$} (12)
  %   (11) edge[near start] node {$\!-w_1\cdot\vec e_1$} (2)
  %   (12) edge[swap,near start] node {$\!-w_1\cdot\vec e_2$} (2)
  %   (nm) edge node {$\vec 0$} (nm1)
  %   (nm) edge[swap] node {$\vec 0$} (nm2)
  %   (nm1) edge[near start] node {$\!\!-w_{n-1}\cdot\vec e_1$} (n)
  %   (nm2) edge[swap,near start] node {$\!\!-w_{n-1}\cdot\vec e_2$} (n);
  %   \draw[->,rounded corners=20pt,>=stealth',shorten >=1pt] (n) -- (10.9,1.6) -- (0.1,1.6) -- (0);
  %   \draw[->,rounded corners=20pt,>=stealth',shorten >=1pt] (n) -- (10.9,-1.6) -- (0.1,-1.6) --
  %   (0);
  %   \node[font=\footnotesize] at (5.5,1.8) {$((W/2)-1)\cdot\vec e_1$};
  %   \node[font=\footnotesize] at (5.5,-1.85) {$((W/2)-1)\cdot\vec e_2$};
  % \end{tikzpicture}
  % \caption{\label{11-fig-part} The "vector system" built from a
  %   \lang{Partition} instance.}
  % \end{figure}
  % We may assume that~$\sum_{w\in S}w$ is even, as otherwise trivially
  % no partition exists; let $W\eqdef(\sum_{w\in S}w)/2$.  We define
  % $\?V\eqdef(\Loc,\Act,\Loc_\mEve,\Loc_\mAdam,2)$ where
  % \begin{align*}
  %   \Loc_\mEve&\eqdef\{\loc_{i,j}\mid 0\leq i<n,1\leq 2\leq
  %               j\}\cup\{\loc_n\}\;,\\
  %   \Loc_\mAdam&\eqdef\{\loc_i\mid 0\leq i<n\}\;,\\
  %   \Act&\eqdef\{\loc_i\step{\vec 0}\loc_{i,j}\mid 0\leq i<n,1\leq 2\leq
  %               j\}\\
  %   &\:\cup\:\{\loc_{i,j}\step{-w_i\cdot\vec e_j}\loc_{i+1}\mid 0\leq i<n,1\leq 2\leq
  %               j\}\\
  %   &\:\cup\:\{\loc_n\step{(W-1)\cdot\vec e_j+W\cdot\vec e_{3-j}}\loc_0\mid 1\leq 2\leq
  %               j\}\;;
  % \end{align*}
  % see \cref{11-fig-part} for a depiction of~$\?V$.  We use~$\loc_0$ as
  % our initial location.
  % Let us show that Adam wins if and only if the \lang{Partition}
  % instance was positive.  If there exist $S_1,S_2\subseteq S$ with
  % $S_1\cap S_2$ and $\sum_{w\in S_1}w=\sum_{w\in S_2}w=W/2$, then
  % Adam has a winning "counterless" strategy where, in
  % location~$\loc_i$ for $0\leq i<n$, he goes to $\loc_{i,1}$ if and
  % only if $w_i\in S_1$.  Then, if the game begins in
  % $\loc_0(c_1,c_2)$, it reaches $\loc_n(c_1-W/2,c_2-W/2)$ after
  % visiting each $\loc_i$~once, and 
\end{proof}

Note that \cref{11-exist-hard} does not apply to fixed dimensions
$\dd\geq 2$.  We know by \cref{11-exist-pseudop} that those games can
be solved in pseudo-polynomial time if the number of priorities is
fixed, and by \cref{11-exist-easy} that they are in \coNP.

\subsubsection{Given Initial Credit}
With "given initial credit", we have a lower bound matching the
\kEXP[2] upper bound of \cref{11-avag-easy}, already with a unary
encoding.  The proof itself is an adaptation of the proof by
\citem[Lipton]{Lipton:1976} of $\EXPSPACE$-hardness of "coverability" in
the one-player case.

\begin{theorem}\label{11-avag-hard}
  "Coverability", "non-termination", and "parity@parity vector game"
  "asymmetric" "vector games" with "given initial credit" are
  \kEXP[2]-hard.
\end{theorem}
\begin{proof}
  We reduce from the "halting problem" of an ""alternating Minsky
  machine"" $\?M=(\Loc,\Act,\Loc_\mEve,\Loc_\mAdam,\dd)$ with counters
  bounded by $B\eqdef 2^{2^n}$ for $n\eqdef|\?M|$.  Such a machine is
  similar to an "asymmetric" "vector system" with increments
  $\loc\step{\vec e_i}\loc'$, decrements $\loc\step{-\vec e_i}\loc'$,
  and "zero test" actions $\loc\step{i\eqby{?0}}\loc'$, all
  restricted to locations $\loc\in\Loc_\mEve$; the only actions
  available to Adam are actions $\loc\step{\vec 0}\loc'$.  The
  set of locations contains a distinguished `halt' location
  $\loc_\mathtt{halt}\in\Loc$ with no outgoing action.  The
  machine comes with the promise that, along any "play", the norm of
  all the visited configurations $\loc(\vec v)$ satisfies
  $\|\vec v\|<B$.  The "halting problem" asks, given an initial
  location $\loc_0\in\Loc$, whether Eve has a winning strategy to
  visit $\loc_\mathtt{halt}(\vec v)$ for some $\vec v\in\+N^\dd$ from
  the initial configuration $\loc_0(\vec 0)$.  This problem is
  \kEXP[2]-complete if $\dd\geq 3$ by standard
  arguments~\cite{Fischer&Meyer&Rosenberg:1968}.

  %%%%%\begin{scope}
    \knowledge{meta-increment}[meta-increments]{notion}
    \knowledge{meta-decrement}[meta-decrements]{notion} Let us start
    by a quick refresher on Lipton's construction~\cite{Lipton:1976};
    see also~\cite{Esparza:1996} for a nice exposition.  At the heart
    of the construction lies a collection of one-player gadgets
    implementing \emph{level~$j$} ""meta-increments""
    $\loc\mstep{2^{2^j}\cdot\vec c}\loc'$ and \emph{level~$j$}
    ""meta-decrements"" $\loc\mstep{-2^{2^j}\cdot\vec c}\loc'$ for
    some "unit vector"~$\vec c$ using $O(j)$ auxiliary counters and
    $\poly(j)$ actions, with precondition that the auxiliary counters
    are initially empty in~$\loc$ and postrelation that they are empty
    again in~$\loc'$.  The construction is by induction over~$j$; let
    us first see a naive implementation for "meta-increments".  For
    the base case~$j=0$, this is just a standard action
    $\loc\step{2\vec c}\loc'$.  For the induction step $j+1$, we use
    the gadget of \cref{11-fig-meta-incr} below, where
    $\vec x_{j},\bar{\vec x}_{j},\vec z_{j},\bar{\vec z}_{j}$ are
    distinct fresh "unit vectors": the gadget performs two nested
    loops, each of $2^{2^j}$ iterations, thus iterates the unit
    increment of~$\vec c$ a total of $\big(2^{2^j}\big)^2=2^{2^{j+1}}$
    times.  A "meta-decrement" is obtained similarly.

    \begin{figure}[htbp]
      \centering
      \begin{tikzpicture}[auto,on grid,node distance=1.55cm]
      \node[s-eve](0){$\loc$};
      \node[s-eve-small,right=of 0](1){};
      \node[s-eve-small,right=of 1](2){};
      \node[s-eve-small,right=of 2](3){};
      \node[s-eve-small,right=of 3](4){};
      \node[s-eve-small,right=of 4](5){};
      \node[s-eve-small,right=of 5](6){};
      \node[s-eve,right=of 6](7){$\loc'$};
      \path[arrow,every node/.style={font=\footnotesize,inner sep=2pt}]
      (0) edge node{$2^{2^j}\cdot\vec x_{j}$} (1)
      (1) edge node{$2^{2^j}\cdot\vec z_{j}$} (2)
      (2) edge node{$\bar{\vec x}_{j}-\vec x_{j}$} (3)
      (3) edge node{$\bar{\vec z}_{j}-\vec z_{j}$} (4)
      (4) edge node{$\vec c$} (5)
      (5) edge node{$-2^{2^j}\cdot\bar{\vec z}_{j}$} (6)
      (6) edge node{$-2^{2^j}\cdot\bar{\vec x}_{j}$} (7);
      \draw[->,rounded corners=10pt,>=stealth'] (5) --
      (7.4,.65) -- (5,.65) -- (3);
      \node[font=\footnotesize,inner sep=2pt] at (6.2,.75) {$\vec 0$};
      \draw[->,rounded corners=10pt,>=stealth'] (6) --
      (8.95,1.25) -- (1.9,1.25) -- (1);
      \node[font=\footnotesize,inner sep=2pt] at (5.43,1.35) {$\vec 0$};
    \end{tikzpicture}
    \caption{\label{11-fig-meta-incr} A naive implementation of the
      "meta-increment" $\loc\mstep{2^{2^{j+1}}\cdot\vec c}\loc'$.}
  \end{figure}
  
  Note that this level~$(j+1)$ gadget contains two copies of the
  level~$j$ "meta-increment" and two of the level~$j$
  "meta-decrement", hence this naive implementation has
  size~$\mathsf{exp}(j)$.  In order to obtain a polynomial size, we would like
  to use a single \emph{shared} level~$j$ gadget for each~$j$, instead
  of hard-wiring multiple copies.  The idea is to use a `dispatch
  mechanism,' using extra counters, to encode the choice of "unit
  vector"~$\vec c$ and of return location~$\loc'$.  Let us see how to
  do this in the case of the return location~$\loc'$; the mechanism
  for the vector~$\vec c$ is similar.  We enumerate the (finitely many)
  possible return locations~$\loc_0,\dots,\loc_{m-1}$ of the gadget
  implementing $\loc\mstep{2^{2^{j+1}}\cdot\vec c}\loc'$.  We use two
  auxiliary counters with "unit vectors" $\vec r_j$
  and~$\bar{\vec r}_j$ to encode the return location.  Assume $\loc'$
  is the $i$th possible return location, i.e., $\loc'=\loc_i$ in our
  enumeration: before entering the shared gadget implementation, we
  initialise~$\vec r_j$ and~$\bar{\vec r}_j$ by performing the action
  $\loc\step{i\cdot\vec r_j+(m-i)\cdot\bar{\vec r}_j}\cdots$.  Then,
  where we would simply go to~$\loc'$ in \cref{11-fig-meta-incr} at
  the end of the gadget, the shared gadget has a final action
  $\cdots\step{\vec 0}\loc_{\mathrm{return}_j}$ leading to a dispatch
  location for returns: for all $0\leq i<m$, we have an action
  $\loc_{\mathrm{return}_j}\step{-i\cdot\vec r_j-(m-i)\cdot\bar{\vec
      r}_j}\loc_i$
  that leads to the desired return location.\todoquestion{Is that
    clear enough?}
  

  \bigskip Let us return to the proof.  Consider an instance of the
  "halting problem".  We first exhibit a reduction to "coverability";
  by \cref{11-cov2parity}, this will also entail the \kEXP[2]-hardness
  of "parity@parity vector game" "asymmetric" "vector games".  We
  build an "asymmetric vector system"
  $\?V=(\Loc',\Act',\Loc'_\mEve,\Loc_\mAdam,\dd')$ with
  $\dd'=2\dd+O(n)$.  Each of the counters~$\mathtt{c}_i$ of $\?M$ is
  paired with a \emph{complementary} counter~$\bar{\mathtt{c}_i}$ such
  that their sum is~$B$ throughout the simulation of~$\?M$.  We
  denote by $\vec c_i$ and $\bar{\vec c}_i$ the corresponding "unit
  vectors" for $1\leq i\leq\dd$.  The "vector system"~$\?V$ starts by
  initialising the counters $\bar{\mathtt{c}}_i$ to~$B$ by a sequence
  of "meta-increments"
  $\loc'_{i-1}\mstep{2^{2^n}\cdot\bar{\vec c}_i}\loc'_i$ for
  $1\leq i\leq\dd$, before starting the simulation by an action
  $\loc'_\dd\step{\vec 0}\loc_0$.  The simulation of~$\?M$ uses the
  actions depicted in \cref{11-fig-lipton}.  Those maintain the
  invariant on the complement counters.  Regarding "zero tests", Eve
  yields the control to Adam, who has a choice between performing a
  "meta-decrement" that will fail if $\bar{\mathtt c}_i< 2^{2^n}$,
  which by the invariant is if and only if $\mathtt{c}_i>0$, or going
  to~$\loc'$.

  \begin{figure}[htbp]
    \centering
    \begin{tikzpicture}[auto,on grid,node distance=1.5cm]
      \node(to){$\mapsto$};
      \node[anchor=east,left=2.5cm of to](mm){"alternating Minsky machine"};
      \node[anchor=west,right=2.5cm of to](mwg){"asymmetric vector system"};
      % increment of Eve
      \node[below=.7cm of to](imap){$\rightsquigarrow$};
      \node[s-eve,left=2.75cm of imap](i0){$\loc$};
      \node[right=of i0](i1){$\loc'$};
      \node[right=1.25cm of imap,s-eve](i2){$\loc$};
      \node[right=1.8 of i2](i3){$\loc'$};      
      \path[arrow,every node/.style={font=\footnotesize}]
      (i0) edge node{$\vec e_i$} (i1)
      (i2) edge node{$\vec c_i-\bar{\vec c}_i$} (i3);
      % decrement of Eve
      \node[below=1cm of imap](dmap){$\rightsquigarrow$};
      \node[s-eve,left=2.75cm of dmap](d0){$\loc$};
      \node[right=of d0](d1){$\loc'$};
      \node[right=1.25cm of dmap,s-eve](d2){$\loc$};
      \node[right=1.8 of d2](d3){$\loc'$};
      \path[arrow,every node/.style={font=\footnotesize}]
      (d0) edge node{$-\vec e_i$} (d1)
      (d2) edge node{$-\vec c_i+\bar{\vec c}_i$} (d3);
      % zero test of Eve
      \node[below=1.5cm of dmap](zmap){$\rightsquigarrow$};
      \node[s-eve,left=2.75cm of zmap](z0){$\loc$};
      \node[right=of z0](z1){$\loc'$};
      \node[right=1.25cm of zmap,s-eve](z2){$\loc$};
      \node[right=of z2,s-adam-small](z3){};
      \node[above right=.8 and 1.1 of z3,s-eve-small](z4){};
      \node[below right=.8 and 1.1 of z3,inner sep=0pt](z5){$\loc'$};
      \node[right=1.8 of z4](z6){$\loc_\mathtt{halt}$};
      \path[arrow,every node/.style={font=\footnotesize}]
      (z0) edge node{$i\eqby{?0}$} (z1)
      (z2) edge node{$\vec 0$} (z3)
      (z3) edge node{$\vec 0$} (z4)
      (z3) edge[swap] node{$\vec 0$} (z5)
      (z4) edge node{$-2^{2^n}\cdot\bar{\vec c}_i$} (z6);
      % action of Adam
      \node[below=1.5cm of zmap](amap){$\rightsquigarrow$};
      \node[s-adam,left=2.75cm of amap](a0){$\loc$};
      \node[right=of a0](a1){$\loc'$};
      \node[right=1.25cm of amap,s-adam](a2){$\loc$};
      \node[right=of a2](a3){$\loc'$};
      \path[arrow,every node/.style={font=\footnotesize}]
      (a0) edge node{$\vec 0$} (a1)
      (a2) edge node{$\vec 0$} (a3);
    \end{tikzpicture}
    \caption{\label{11-fig-lipton}Schema of the reduction to
      "coverability" in the proof of \cref{11-avag-hard}.}
  \end{figure}
  
  It is hopefully clear that Eve wins the "coverability game" played
  on~$\?V$ starting from $\loc'_0(\vec 0)$ and with target
  configuration $\loc_\mathtt{halt}(\vec 0)$ if and only if the
  "alternating Minsky machine" halts.

  \medskip Regarding "non-termination" games, we use essentially the
  same reduction.  First observe that, if Eve can ensure reaching
  $\loc_\mathtt{halt}$ in the "alternating Minsky machine", then she
  can do so after at most $|\Loc|B^\dd$ steps.  We therefore use a
  `time budget': this is an additional component in $\?V$ with
  associated "unit vector"~$\vec t$.  This component is initialised to
  $|\Loc|B^\dd=|\Loc|2^{\dd 2^n}$ before the simulation, and decreases
  by~one at every step; see \cref{11-fig-lipton-nonterm}.  We also add
  a self loop $\loc_\mathtt{halt}\step{\vec 0}\loc_\mathtt{halt}$.
  Then the only way to avoid the "sink" and thus to win the
  "non-termination" game is to reach $\loc_\mathtt{halt}$.
  \begin{figure}[htbp]
    \centering
    \begin{tikzpicture}[auto,on grid,node distance=1.5cm]
      \node(to){$\mapsto$};
      \node[anchor=east,left=2.5cm of to](mm){"alternating Minsky machine"};
      \node[anchor=west,right=2.5cm of to](mwg){"asymmetric vector system"};
      % increment of Eve
      \node[below=.7cm of to](imap){$\rightsquigarrow$};
      \node[s-eve,left=2.75cm of imap](i0){$\loc$};
      \node[right=of i0](i1){$\loc'$};
      \node[right=1.25cm of imap,s-eve](i2){$\loc$};
      \node[right=1.8 of i2](i3){$\loc'$};      
      \path[arrow,every node/.style={font=\footnotesize}]
      (i0) edge node{$\vec e_i$} (i1)
      (i2) edge node{$\vec c_i-\bar{\vec c}_i-\vec t$} (i3);
      % decrement of Eve
      \node[below=1cm of imap](dmap){$\rightsquigarrow$};
      \node[s-eve,left=2.75cm of dmap](d0){$\loc$};
      \node[right=of d0](d1){$\loc'$};
      \node[right=1.25cm of dmap,s-eve](d2){$\loc$};
      \node[right=1.8 of d2](d3){$\loc'$};
      \path[arrow,every node/.style={font=\footnotesize}]
      (d0) edge node{$-\vec e_i$} (d1)
      (d2) edge node{$-\vec c_i+\bar{\vec c}_i-\vec t$} (d3);
      % zero test of Eve
      \node[below=1.5cm of dmap](zmap){$\rightsquigarrow$};
      \node[s-eve,left=2.75cm of zmap](z0){$\loc$};
      \node[right=of z0](z1){$\loc'$};
      \node[right=1.25cm of zmap,s-eve](z2){$\loc$};
      \node[right=of z2,s-adam-small](z3){};
      \node[above right=.8 and 1.1 of z3,s-eve-small](z4){};
      \node[below right=.8 and 1.1 of z3,inner sep=0pt](z5){$\loc'$};
      \node[right=1.8 of z4](z6){$\loc_\mathtt{halt}$};
      \path[arrow,every node/.style={font=\footnotesize}]
      (z0) edge node{$i\eqby{?0}$} (z1)
      (z2) edge node{$-\vec t$} (z3)
      (z3) edge node{$\vec 0$} (z4)
      (z3) edge[swap] node{$\vec 0$} (z5)
      (z4) edge node{$-2^{2^n}\cdot\bar{\vec c}_i$} (z6);
      % action of Adam
      \node[below=1.5cm of zmap](amap){$\rightsquigarrow$};
      \node[s-adam,left=2.75cm of amap](a0){$\loc$};
      \node[right=of a0](a1){$\loc'$};
      \node[right=1.25cm of amap,s-adam](a2){$\loc$};
      \node[right=of a2,s-eve-small](a3){};
      \node[right=of a3](a4){$\loc'$};
      \path[arrow,every node/.style={font=\footnotesize}]
      (a0) edge node{$\vec 0$} (a1)
      (a2) edge node{$\vec 0$} (a3)
      (a3) edge node{$-\vec t$} (a4);
    \end{tikzpicture}
    \caption{\label{11-fig-lipton-nonterm}Schema of the reduction to
      "non-termination" in the proof of \cref{11-avag-hard}.}
  \end{figure}

  We still need to extend our initialisation phase.  It suffices for
  this to implement a gadget for $\dd$-"meta-increments"
  $\loc\mstep{2^{\dd 2^j}\cdot\vec c}\loc'$ and $\dd$-"meta-decrements"
  $\loc\mstep{-2^{\dd 2^j}\cdot\vec c}\loc'$; this is the same argument as
  in Lipton's construction, with a base case $\loc\mstep{2^\dd}\loc'$
  for $j=0$.  Then we initialise our time budget through $|\Loc|$
  successive $\dd$-"meta-increments"
  $\loc\mstep{2^{\dd 2^n}\cdot\vec t}\loc'$.
  %%\end{scope}
\end{proof}

The proof of \cref{11-avag-hard} relies crucially on the fact that the
dimension is not fixed: although $\dd\geq 3$ suffices in the
"alternating Minsky machine", we need $O(|\?M|)$ additional counters
to carry out the reduction.  A separate argument is thus needed in
order to match the \EXP\ upper bound of \cref{11-avag-easy} in fixed
dimension.

\begin{theorem}\label{11-avag-two}
  "Coverability", "non-termination", and "parity@parity vector game"
  "asymmetric" "vector games" with "given initial credit" are
  \EXP-hard in dimension $\dd\geq 2$.
\end{theorem}
\begin{proof}
  We exhibit a reduction from "countdown games" with "given initial
  credit", which are \EXP-complete by \cref{11-countdown-given}.
  Consider an instance of a "configuration reachability" countdown
  game: a "countdown system"
  $\?V=(\Loc,\Act,\Loc_\mEve,\Loc_\mAdam,1)$ with initial
  configuration $\loc_0(n_0)$ and target
  configuration~$\smiley(0)$---as seen in the proof
  of \cref{11-countdown-given}, we can indeed assume that the target
  credit is zero; we will also assume that Eve controls~$\smiley$ and
  that the only action available in~$\smiley$ is
  $\smiley\step{-1}\smiley$.  We construct an "asymmetric" "vector
  system" $\?V'$ of dimension~2 such that Eve can ensure
  reaching~$\smiley(0,n_0)$ from $\loc_0(n_0,0)$ in~$\?V'$ if and only
  if she could ensure reaching $\smiley(0)$ from $\loc_0(n_0)$
  in~$\?V$.  The translation is depicted in \cref{11-fig-dim2}.
  
  \begin{figure}[htbp]
    \centering
    \begin{tikzpicture}[auto,on grid,node distance=1.5cm]
      \node(to){$\mapsto$};
      \node[anchor=east,left=2.5cm of to](mm){"countdown system"};
      \node[anchor=west,right=2.5cm of to](mwg){"asymmetric vector system"};
      % action of Eve
      \node[below=.7cm of to](imap){$\rightsquigarrow$};
      \node[s-eve,left=2.75cm of imap](i0){$\loc$};
      \node[right=of i0](i1){$\loc'$};
      \node[right=1.25cm of imap,s-eve](i2){$\loc$};
      \node[right=1.8 of i2](i3){$\loc'$};      
      \path[arrow,every node/.style={font=\footnotesize,inner sep=1pt}]
      (i0) edge node{$-n$} (i1)
      (i2) edge node{$-n,n$} (i3);
      % minimal action of Adam
      \node[below=1cm of imap](dmap){$\rightsquigarrow$};
      \node[s-adam,left=2.75cm of dmap](d0){$\loc$};
      \node[right=of d0](d1){$\loc'$};
      \node[below=.5 of d0]{$n=\min\{n'\mid\exists\loc''\in\Loc\mathbin.\loc\step{-n'}\loc''\in\Act\}$};
      \node[right=1.25cm of dmap,s-adam](d2){$\loc$};
      \node[right=1.8 of d2,s-eve-small](d3){};
      \node[right=1.8 of d3](d4){$\loc'$};
      \path[arrow,every node/.style={font=\footnotesize,inner sep=1pt}]
      (d0) edge node{$-n$} (d1)
      (d2) edge node{$0,0$} (d3)
      (d3) edge node{$-n,n$} (d4);
      % non-minimal action of Adam
      \node[below=1.5cm of dmap](zmap){$\rightsquigarrow$};
      \node[s-adam,left=2.75cm of zmap](z0){$\loc$};
      \node[right=of z0](z1){$\loc'$};
      \node[below=.5 of z0]{$n\neq\min\{n'\mid\exists\loc''\in\Loc\mathbin.\loc\step{-n'}\loc''\in\Act\}$};
      \node[right=1.25cm of zmap,s-adam](z2){$\loc$};
      \node[right=of z2,s-eve-small](z3){};
      \node[above right=.8 and 2.1 of z3](z4){$\loc'$};
      \node[below right=.8 and 2.1 of z3,s-eve](z5){$\smiley$};
      \path[arrow,every node/.style={font=\footnotesize,inner sep=1pt}]
      (z0) edge node{$-n$} (z1)
      (z2) edge node{$0,0$} (z3)
      (z3) edge[bend left=8] node{$-n,n$} (z4)
      (z3) edge[swap,bend right=8] node{$n_0-n+1,-n_0+n-1$} (z5)
      (z5) edge[loop above] node{$-1,1$} ()
      (z5) edge[loop right] node{$\,0,0$} ();
    \end{tikzpicture}
    \caption{\label{11-fig-dim2}Schema of the reduction in the proof
    of \cref{11-avag-two}.}
  \end{figure}
    
  The idea behind this translation is that a configuration $\loc(c)$
  of~$\?V$ is simulated by a configuration $\loc(c,n_0-c)$ in~$\?V'$.
  The crucial point is how to handle Adam\'s moves.  In a
  configuration $\loc(c,n_0-c)$ with $\loc\in\Loc_\mAdam$, according
  to the "natural semantics" of $\?V$, Adam should be able to
  simulate an action $\loc\step{-n}\loc'$ if and only if $c\geq n$.
  Observe that otherwise if $c<n$ and thus $n_0-c>n_0-n$, Eve can
  play to reach~$\smiley$ and win immediately.  An exception to the
  above is if $n$ is minimal among the decrements in~$\loc$, because
  according to the "natural semantics" of~$\?V$, if $c<n$ there should
  be an edge to the "sink", and this is handled in the second line
  of \cref{11-fig-dim2}.

  Then Eve can reach $\smiley(0,n_0)$ if and only if she can cover
  $\smiley(0,n_0)$, if and only if she can avoid the "sink" thanks to
  the self loop $\smiley\step{0,0}\smiley$.  This
  shows the \EXP-hardness of "coverability" and "non-termination"
  "asymmetric" "vector games" in dimension~two; the hardness of
  "parity@parity vector game" follows
  from \cref{11-cov2parity,12-nonterm2parity}.
\end{proof}



% Local IspellDict: british

\subsubsection{Existential Initial Credit}
In the "existential initial credit" variant of our games, we have the
following lower bound matching \cref{11-th:exist-easy}, already with a
unary encoding.

\begin{theorem}[Existential non-termination asymmetric vector games are \coNP-hard]
\label{11-th:exist-hard}
  "Non-termination", and "parity@parity vector game"
  "asymmetric" "vector games" with "existential initial credit" are
  \coNP-hard.% in any dimension~$\dd\geq 2$.
\end{theorem}
\begin{proof}
  By \cref{11-rk:nonterm2parity}, it suffices to show hardness for
  "non-termination games".  We reduce from the \lang{3SAT} problem:
  given a formula $\varphi=\bigwedge_{1\leq i\leq m}C_i$ where each
  clause $C_i$ is a disjonction of the form
  $\litt_{i,1}\vee\litt_{i,2}\vee\litt_{i,3}$ of literals taken from
  $X=\{x_1,\neg x_1,x_2,$ $\neg x_2,\dots,x_k,\neg x_k\}$, we construct
  an "asymmetric" "vector system" $\?V$ where \Eve\ wins the
  "non-termination game" with "existential initial credit" if and only
  if~$\varphi$ is not satisfiable; since the game is determined, we
  actually show that \Adam\ wins the game if and only if~$\varphi$ is
  satisfiable.

  Our "vector system" has dimension~$2k$, and for a literal
  $\litt\in X$, we define the vector
  \begin{equation*}
    \vec u_\litt\eqdef\begin{cases}
      \vec e_{2n-1}-\vec e_{2n}&\text{if }\litt=x_n\;,\\
      \vec e_{2n}-\vec e_{2n-1}&\text{if }\litt=\neg x_n\;.
    \end{cases}
  \end{equation*}
  We define $\?V\eqdef(\Loc,\Act,\Loc_\mEve,\Loc_\mAdam,2k)$ where
  \begin{align*}
    \Loc_\mEve&\eqdef\{\varphi\}\cup\{\litt_{i,j}\mid 1\leq i\leq m,1\leq j\leq
                3\}\;,\\
    \Loc_\mAdam&\eqdef\{C_i\mid 1\leq i\leq m\}\;,\\
    \Act&\eqdef\{\varphi\step{\vec 0}C_i\mid 1\leq i\leq m\}\cup\{C_i\step{\vec 0}\litt_{i,j},\;\;\litt_{i,j}\xrightarrow{\vec u_{\litt_{i,j}}}\varphi\mid 1\leq i\leq m,1\leq j\leq 3\}\;.
  \end{align*}
  \begin{scope}
    We use~$\varphi$ as our initial location.
    \knowledge{literal assignment}{notion}
    \knowledge{conflicting}{notion}
    %
    Let us call a map $v{:}\,X\to\{0,1\}$ a ""literal assignment""; we
    call it ""conflicting"" if there exists $1\leq n\leq k$ such that
    $v(x_n)=v(\neg x_n)$.

    Assume that~$\varphi$ is satisfiable.  Then there exists a
    non-"conflicting" "literal assignment"~$v$ that satisfies all the
    clauses: for each $1\leq i\leq m$, there exists $1\leq j\leq 3$
    such that $v(\litt_{i,j})=1$; this yields a "counterless" strategy
    for \Adam, which selects $(C_i,\litt_{i,j})$ for each
    $1\leq i\leq m$.  Consider any infinite "play" consistent with
    this strategy.  This "play" only visits literals $\litt$ where
    $v(\litt)=1$.  There exists a literal $\litt\in X$ that is visited
    infinitely often along the "play", say $\litt=x_n$.  Because~$v$ is
    non-"conflicting", $v(\neg x_n)=0$, thus the location $\neg x_n$
    is never visited.  Thus the play uses the action
    $\litt\step{\vec e_{2n-1}-\vec e_{2n}}\varphi$ infinitely often,
    and never uses any action with a positive effect on
    component~$2n$.  Hence the play is losing from any initial credit.

    Conversely, assume that~$\varphi$ is not satisfiable.  By
    contradiction, assume that \Adam\ wins the game for all initial
    credits.  By \cref{11-lem:counterless}, he has a "counterless" winning
    strategy~$\tau$ that selects a literal in every clause.  Consider
    a "literal assignment" that maps each one of the selected literals
    to~$1$ and the remaining ones in a non-conflicting manner.  By
    definition, this "literal assignment" satisfies all the clauses,
    but because~$\varphi$ is not satisfiable, it is "conflicting":
    necessarily, there exist $1\leq n\leq k$ and $1\leq i,i'\leq m$,
    such that $\tau$ selects $x_n$ in $C_i$ and $\neg x_n$ in
    $C_{i'}$.  But this yields a winning strategy for \Eve, which
    alternates in the initial location $\varphi$ between $C_{i}$
    and $C_{i'}$, and for which an initial credit
    $\vec e_{2n-1}+\vec e_{2n}$ suffices: a contradiction.
  \end{scope}
  % We reduce from the \lang{Partition} problem: given a finite set
  % $S=\{w_0,\dots,w_{n-1}\}\subseteq\+N$ (encoded in binary), does
  % there exist a partition $S_1,S_2$ of~$S$ such that
  % $\sum_{w\in S_1}w=\sum_{w\in S_2}w$?  From such an instance, we
  % construct an "asymmetric vector system" where \Eve\ wins the
  % "non-termination game" with "existential initial credit" if and only
  % if there is no such partition.  By \cref{11-rk:nonterm2parity}, this
  % also entails the \coNP-hardness of "parity@parity vector games"
  % "asymmetric vector games".
  % \begin{figure}[htbp]
  % \centering
  % \begin{tikzpicture}[auto,on grid,node distance=1.8cm]
  %   \node[adam,inner sep=2.1](0){$\loc_0$};
  %   \node[eve,above right=1 and 1 of 0,inner sep=2.2](01){$\loc_{0,1}$};
  %   \node[eve,below right=1 and 1 of 0,inner sep=2.2](02){$\loc_{0,2}$};
  %   \node[adam,below right=1 and 2 of 01,inner sep=2.1](1){$\loc_1$};
  %   \node[eve,above right=1 and 1 of 1,inner sep=2.2](11){$\loc_{1,1}$};
  %   \node[eve,below right=1 and 1 of 1,inner sep=2.2](12){$\loc_{1,2}$};
  %   \node[adam,below right=1 and 2 of 11,inner sep=2.1](2){$\loc_2$};
  %   \node[right=1 of 2](dots){$\Huge\cdots$};
  %   \node[adam,right=1 of dots,inner sep=-1.8](nm){$\loc_{n-1}$};
  %   \node[eve,above right=1 and 1 of nm,inner sep=0](nm1){$\loc_{n-1,1}$};
  %   \node[eve,below right=1 and 1 of nm,inner sep=0](nm2){$\loc_{n-1,2}$};
  %   \node[eve,below right=1 and 2 of nm1,inner sep=2.2](n){$\loc_n$};    
  %   \path[->,every node/.style={font=\footnotesize,inner sep=.1}]
  %   (0) edge node {$\vec 0$} (01)
  %   (0) edge[swap] node {$\vec 0$} (02)
  %   (01) edge[near start] node {$\!-w_0\cdot\vec e_1$} (1)
  %   (02) edge[swap,near start] node {$\!-w_0\cdot\vec e_2$} (1)
  %   (1) edge node {$\vec 0$} (11)
  %   (1) edge[swap] node {$\vec 0$} (12)
  %   (11) edge[near start] node {$\!-w_1\cdot\vec e_1$} (2)
  %   (12) edge[swap,near start] node {$\!-w_1\cdot\vec e_2$} (2)
  %   (nm) edge node {$\vec 0$} (nm1)
  %   (nm) edge[swap] node {$\vec 0$} (nm2)
  %   (nm1) edge[near start] node {$\!\!-w_{n-1}\cdot\vec e_1$} (n)
  %   (nm2) edge[swap,near start] node {$\!\!-w_{n-1}\cdot\vec e_2$} (n);
  %   \draw[->,rounded corners=20pt,>=stealth',shorten >=1pt] (n) -- (10.9,1.6) -- (0.1,1.6) -- (0);
  %   \draw[->,rounded corners=20pt,>=stealth',shorten >=1pt] (n) -- (10.9,-1.6) -- (0.1,-1.6) --
  %   (0);
  %   \node[font=\footnotesize] at (5.5,1.8) {$((W/2)-1)\cdot\vec e_1$};
  %   \node[font=\footnotesize] at (5.5,-1.85) {$((W/2)-1)\cdot\vec e_2$};
  % \end{tikzpicture}
  % \caption{\label{11-fig:part} The "vector system" built from a
  %   \lang{Partition} instance.}
  % \end{figure}
  % We may assume that~$\sum_{w\in S}w$ is even, as otherwise trivially
  % no partition exists; let $W\eqdef(\sum_{w\in S}w)/2$.  We define
  % $\?V\eqdef(\Loc,\Act,\Loc_\mEve,\Loc_\mAdam,2)$ where
  % \begin{align*}
  %   \Loc_\mEve&\eqdef\{\loc_{i,j}\mid 0\leq i<n,1\leq 2\leq
  %               j\}\cup\{\loc_n\}\;,\\
  %   \Loc_\mAdam&\eqdef\{\loc_i\mid 0\leq i<n\}\;,\\
  %   \Act&\eqdef\{\loc_i\step{\vec 0}\loc_{i,j}\mid 0\leq i<n,1\leq 2\leq
  %               j\}\\
  %   &\:\cup\:\{\loc_{i,j}\step{-w_i\cdot\vec e_j}\loc_{i+1}\mid 0\leq i<n,1\leq 2\leq
  %               j\}\\
  %   &\:\cup\:\{\loc_n\step{(W-1)\cdot\vec e_j+W\cdot\vec e_{3-j}}\loc_0\mid 1\leq 2\leq
  %               j\}\;;
  % \end{align*}
  % see \cref{11-fig:part} for a depiction of~$\?V$.  We use~$\loc_0$ as
  % our initial location.
  % Let us show that \Adam\ wins if and only if the \lang{Partition}
  % instance was positive.  If there exist $S_1,S_2\subseteq S$ with
  % $S_1\cap S_2$ and $\sum_{w\in S_1}w=\sum_{w\in S_2}w=W/2$, then
  % \Adam\ has a winning "counterless" strategy where, in
  % location~$\loc_i$ for $0\leq i<n$, he goes to $\loc_{i,1}$ if and
  % only if $w_i\in S_1$.  Then, if the game begins in
  % $\loc_0(c_1,c_2)$, it reaches $\loc_n(c_1-W/2,c_2-W/2)$ after
  % visiting each $\loc_i$~once, and 
\end{proof}

Note that \cref{11-th:exist-hard} does not apply to fixed dimensions
$\dd\geq 2$.  We know by \cref{11-cor:exist-pseudop} that those games can
be solved in pseudo-polynomial time if the number of priorities is
fixed, and by \cref{11-th:exist-easy} that they are in \coNP.

\subsubsection{Given Initial Credit}
With "given initial credit", we have a lower bound matching the
\kEXP[2] upper bound of \cref{11-th:avag-easy}, already with a unary
encoding.  The proof itself is an adaptation of the proof by
\citem[Lipton]{Lipton:1976} of \EXPSPACE-hardness of "coverability" in
the one-player case.

\begin{theorem}[Coverability and non-termination asymmetric vector games are {\kEXP[2]-hard}]
\label{11-th:avag-hard}
  "Coverability", "non-termination", and "parity@parity vector game"
  "asymmetric" "vector games" with "given initial credit" are
  \kEXP[2]-hard.
\end{theorem}
\begin{proof}
  We reduce from the "halting problem" of an ""alternating Minsky
  machine"" $\?M=(\Loc,\Act,\Loc_\mEve,\Loc_\mAdam,\dd)$ with counters
  bounded by $B\eqdef 2^{2^n}$ for $n\eqdef|\?M|$.  Such a machine is
  similar to an "asymmetric" "vector system" with increments
  $\loc\step{\vec e_i}\loc'$, decrements $\loc\step{-\vec e_i}\loc'$,
  and "zero test" actions $\loc\step{i\eqby?0}\loc'$, all
  restricted to locations $\loc\in\Loc_\mEve$; the only actions
  available to \Adam\ are actions $\loc\step{\vec 0}\loc'$.  The
  set of locations contains a distinguished `halt' location
  $\loc_\mathtt{halt}\in\Loc$ with no outgoing action.  The
  machine comes with the promise that, along any "play", the norm of
  all the visited configurations $\loc(\vec v)$ satisfies
  $\|\vec v\|<B$.  The "halting problem" asks, given an initial
  location $\loc_0\in\Loc$, whether \Eve\ has a winning strategy to
  visit $\loc_\mathtt{halt}(\vec v)$ for some $\vec v\in\+N^\dd$ from
  the initial configuration $\loc_0(\vec 0)$.  This problem is
  \kEXP[2]-complete if $\dd\geq 3$ by standard
  arguments~\cite{Fischer&Meyer&Rosenberg:1968}.

  %%%%%\begin{scope}
    \knowledge{meta-increment}[meta-increments]{notion}
    \knowledge{meta-decrement}[meta-decrements]{notion} Let us start
    by a quick refresher on Lipton's construction~\cite{Lipton:1976};
    see also~\cite{Esparza:1996} for a nice exposition.  At the heart
    of the construction lies a collection of one-player gadgets
    implementing \emph{level~$j$} ""meta-increments""
    $\loc\mstep{2^{2^j}\cdot\vec c}\loc'$ and \emph{level~$j$}
    ""meta-decrements"" $\loc\mstep{-2^{2^j}\cdot\vec c}\loc'$ for
    some "unit vector"~$\vec c$ using $O(j)$ auxiliary counters and
    $\poly(j)$ actions, with precondition that the auxiliary counters
    are initially empty in~$\loc$ and postrelation that they are empty
    again in~$\loc'$.  The construction is by induction over~$j$; let
    us first see a naive implementation for "meta-increments".  For
    the base case~$j=0$, this is just a standard action
    $\loc\step{2\vec c}\loc'$.  For the induction step $j+1$, we use
    the gadget of \cref{11-fig:meta-incr} below, where
    $\vec x_{j},\bar{\vec x}_{j},\vec z_{j},\bar{\vec z}_{j}$ are
    distinct fresh "unit vectors": the gadget performs two nested
    loops, each of $2^{2^j}$ iterations, thus iterates the unit
    increment of~$\vec c$ a total of $\big(2^{2^j}\big)^2=2^{2^{j+1}}$
    times.  A "meta-decrement" is obtained similarly.

    \begin{figure}[htbp]
      \centering
      \begin{tikzpicture}[auto,on grid,node distance=1.55cm]
      \node[s-eve](0){$\loc$};
      \node[s-eve-small,right=of 0](1){};
      \node[s-eve-small,right=of 1](2){};
      \node[s-eve-small,right=of 2](3){};
      \node[s-eve-small,right=of 3](4){};
      \node[s-eve-small,right=of 4](5){};
      \node[s-eve-small,right=of 5](6){};
      \node[s-eve,right=of 6](7){$\loc'$};
      \path[arrow,every node/.style={font=\footnotesize,inner sep=2pt}]
      (0) edge node{$2^{2^j}\cdot\vec x_{j}$} (1)
      (1) edge node{$2^{2^j}\cdot\vec z_{j}$} (2)
      (2) edge node{$\bar{\vec x}_{j}-\vec x_{j}$} (3)
      (3) edge node{$\bar{\vec z}_{j}-\vec z_{j}$} (4)
      (4) edge node{$\vec c$} (5)
      (5) edge node{$-2^{2^j}\cdot\bar{\vec z}_{j}$} (6)
      (6) edge node{$-2^{2^j}\cdot\bar{\vec x}_{j}$} (7);
      \draw[->,rounded corners=10pt,>=stealth'] (5) --
      (7.4,.65) -- (5,.65) -- (3);
      \node[font=\footnotesize,inner sep=2pt] at (6.2,.75) {$\vec 0$};
      \draw[->,rounded corners=10pt,>=stealth'] (6) --
      (8.95,1.25) -- (1.9,1.25) -- (1);
      \node[font=\footnotesize,inner sep=2pt] at (5.43,1.35) {$\vec 0$};
    \end{tikzpicture}
    \caption{A naive implementation of the
      "meta-increment" $\loc\mstep{2^{2^{j+1}}\cdot\vec c}\loc'$.}\label{11-fig:meta-incr}
  \end{figure}
  
  Note that this level~$(j+1)$ gadget contains two copies of the
  level~$j$ "meta-increment" and two of the level~$j$
  "meta-decrement", hence this naive implementation has
  size~$\mathsf{exp}(j)$.  In order to obtain a polynomial size, we would like
  to use a single \emph{shared} level~$j$ gadget for each~$j$, instead
  of hard-wiring multiple copies.  The idea is to use a `dispatch
  mechanism,' using extra counters, to encode the choice of "unit
  vector"~$\vec c$ and of return location~$\loc'$.  Let us see how to
  do this in the case of the return location~$\loc'$; the mechanism
  for the vector~$\vec c$ is similar.  We enumerate the (finitely many)
  possible return locations~$\loc_0,\dots,\loc_{m-1}$ of the gadget
  implementing $\loc\mstep{2^{2^{j+1}}\cdot\vec c}\loc'$.  We use two
  auxiliary counters with "unit vectors" $\vec r_j$
  and~$\bar{\vec r}_j$ to encode the return location.  Assume $\loc'$
  is the $i$th possible return location, i.e., $\loc'=\loc_i$ in our
  enumeration: before entering the shared gadget implementation, we
  initialise~$\vec r_j$ and~$\bar{\vec r}_j$ by performing the action
  $\loc\step{i\cdot\vec r_j+(m-i)\cdot\bar{\vec r}_j}\cdots$.  Then,
  where we would simply go to~$\loc'$ in \cref{11-fig:meta-incr} at
  the end of the gadget, the shared gadget has a final action
  $\cdots\step{\vec 0}\loc_{\mathrm{return}_j}$ leading to a dispatch
  location for returns: for all $0\leq i<m$, we have an action
  $\loc_{\mathrm{return}_j}\step{-i\cdot\vec r_j-(m-i)\cdot\bar{\vec
      r}_j}\loc_i$
  that leads to the desired return location.\todoquestion{Is that
    clear enough?}
  

  \bigskip Let us return to the proof.  Consider an instance of the
  "halting problem".  We first exhibit a reduction to "coverability";
  by \cref{11-rk:cov2parity}, this will also entail the \kEXP[2]-hardness
  of "parity@parity vector game" "asymmetric" "vector games".  We
  build an "asymmetric vector system"
  $\?V=(\Loc',\Act',\Loc'_\mEve,\Loc_\mAdam,\dd')$ with
  $\dd'=2\dd+O(n)$.  Each of the counters~$\mathtt{c}_i$ of $\?M$ is
  paired with a \emph{complementary} counter~$\bar{\mathtt{c}_i}$ such
  that their sum is~$B$ throughout the simulation of~$\?M$.  We
  denote by $\vec c_i$ and $\bar{\vec c}_i$ the corresponding "unit
  vectors" for $1\leq i\leq\dd$.  The "vector system"~$\?V$ starts by
  initialising the counters $\bar{\mathtt{c}}_i$ to~$B$ by a sequence
  of "meta-increments"
  $\loc'_{i-1}\mstep{2^{2^n}\cdot\bar{\vec c}_i}\loc'_i$ for
  $1\leq i\leq\dd$, before starting the simulation by an action
  $\loc'_\dd\step{\vec 0}\loc_0$.  The simulation of~$\?M$ uses the
  actions depicted in \cref{11-fig:lipton}.  Those maintain the
  invariant on the complement counters.  Regarding "zero tests", \Eve
  yields the control to \Adam, who has a choice between performing a
  "meta-decrement" that will fail if $\bar{\mathtt c}_i< 2^{2^n}$,
  which by the invariant is if and only if $\mathtt{c}_i>0$, or going
  to~$\loc'$.

  \begin{figure}[htbp]
    \centering
    \begin{tikzpicture}[auto,on grid,node distance=1.5cm]
      \node(to){$\mapsto$};
      \node[anchor=east,left=2.5cm of to](mm){"alternating Minsky machine"};
      \node[anchor=west,right=2.5cm of to](mwg){"asymmetric vector system"};
      % increment of Eve
      \node[below=.7cm of to](imap){$\rightsquigarrow$};
      \node[s-eve,left=2.75cm of imap](i0){$\loc$};
      \node[right=of i0](i1){$\loc'$};
      \node[right=1.25cm of imap,s-eve](i2){$\loc$};
      \node[right=1.8 of i2](i3){$\loc'$};      
      \path[arrow,every node/.style={font=\footnotesize}]
      (i0) edge node{$\vec e_i$} (i1)
      (i2) edge node{$\vec c_i-\bar{\vec c}_i$} (i3);
      % decrement of Eve
      \node[below=1cm of imap](dmap){$\rightsquigarrow$};
      \node[s-eve,left=2.75cm of dmap](d0){$\loc$};
      \node[right=of d0](d1){$\loc'$};
      \node[right=1.25cm of dmap,s-eve](d2){$\loc$};
      \node[right=1.8 of d2](d3){$\loc'$};
      \path[arrow,every node/.style={font=\footnotesize}]
      (d0) edge node{$-\vec e_i$} (d1)
      (d2) edge node{$-\vec c_i+\bar{\vec c}_i$} (d3);
      % zero test of Eve
      \node[below=1.5cm of dmap](zmap){$\rightsquigarrow$};
      \node[s-eve,left=2.75cm of zmap](z0){$\loc$};
      \node[right=of z0](z1){$\loc'$};
      \node[right=1.25cm of zmap,s-eve](z2){$\loc$};
      \node[right=of z2,s-adam-small](z3){};
      \node[above right=.8 and 1.1 of z3,s-eve-small](z4){};
      \node[below right=.8 and 1.1 of z3,inner sep=0pt](z5){$\loc'$};
      \node[right=1.8 of z4](z6){$\loc_\mathtt{halt}$};
      \path[arrow,every node/.style={font=\footnotesize}]
      (z0) edge node{$i\eqby?0$} (z1)
      (z2) edge node{$\vec 0$} (z3)
      (z3) edge node{$\vec 0$} (z4)
      (z3) edge[swap] node{$\vec 0$} (z5)
      (z4) edge node{$-2^{2^n}\cdot\bar{\vec c}_i$} (z6);
      % action of Adam
      \node[below=1.5cm of zmap](amap){$\rightsquigarrow$};
      \node[s-adam,left=2.75cm of amap](a0){$\loc$};
      \node[right=of a0](a1){$\loc'$};
      \node[right=1.25cm of amap,s-adam](a2){$\loc$};
      \node[right=of a2](a3){$\loc'$};
      \path[arrow,every node/.style={font=\footnotesize}]
      (a0) edge node{$\vec 0$} (a1)
      (a2) edge node{$\vec 0$} (a3);
    \end{tikzpicture}
    \caption{Schema of the reduction to
      "coverability" in the proof of \cref{11-th:avag-hard}.}\label{11-fig:lipton}
  \end{figure}
  
  It is hopefully clear that \Eve\ wins the "coverability game" played
  on~$\?V$ starting from $\loc'_0(\vec 0)$ and with target
  configuration $\loc_\mathtt{halt}(\vec 0)$ if and only if the
  "alternating Minsky machine" halts.

  \medskip Regarding "non-termination" games, we use essentially the
  same reduction.  First observe that, if \Eve\ can ensure reaching
  $\loc_\mathtt{halt}$ in the "alternating Minsky machine", then she
  can do so after at most $|\Loc|B^\dd$ steps.  We therefore use a
  `time budget': this is an additional component in $\?V$ with
  associated "unit vector"~$\vec t$.  This component is initialised to
  $|\Loc|B^\dd=|\Loc|2^{\dd 2^n}$ before the simulation, and decreases
  by~one at every step; see \cref{11-fig:lipton-nonterm}.  We also add
  a self loop $\loc_\mathtt{halt}\step{\vec 0}\loc_\mathtt{halt}$.
  Then the only way to avoid the "sink" and thus to win the
  "non-termination" game is to reach $\loc_\mathtt{halt}$.
  \begin{figure}[htbp]
    \centering
    \begin{tikzpicture}[auto,on grid,node distance=1.5cm]
      \node(to){$\mapsto$};
      \node[anchor=east,left=2.5cm of to](mm){"alternating Minsky machine"};
      \node[anchor=west,right=2.5cm of to](mwg){"asymmetric vector system"};
      % increment of Eve
      \node[below=.7cm of to](imap){$\rightsquigarrow$};
      \node[s-eve,left=2.75cm of imap](i0){$\loc$};
      \node[right=of i0](i1){$\loc'$};
      \node[right=1.25cm of imap,s-eve](i2){$\loc$};
      \node[right=1.8 of i2](i3){$\loc'$};      
      \path[arrow,every node/.style={font=\footnotesize}]
      (i0) edge node{$\vec e_i$} (i1)
      (i2) edge node{$\vec c_i-\bar{\vec c}_i-\vec t$} (i3);
      % decrement of Eve
      \node[below=1cm of imap](dmap){$\rightsquigarrow$};
      \node[s-eve,left=2.75cm of dmap](d0){$\loc$};
      \node[right=of d0](d1){$\loc'$};
      \node[right=1.25cm of dmap,s-eve](d2){$\loc$};
      \node[right=1.8 of d2](d3){$\loc'$};
      \path[arrow,every node/.style={font=\footnotesize}]
      (d0) edge node{$-\vec e_i$} (d1)
      (d2) edge node{$-\vec c_i+\bar{\vec c}_i-\vec t$} (d3);
      % zero test of Eve
      \node[below=1.5cm of dmap](zmap){$\rightsquigarrow$};
      \node[s-eve,left=2.75cm of zmap](z0){$\loc$};
      \node[right=of z0](z1){$\loc'$};
      \node[right=1.25cm of zmap,s-eve](z2){$\loc$};
      \node[right=of z2,s-adam-small](z3){};
      \node[above right=.8 and 1.1 of z3,s-eve-small](z4){};
      \node[below right=.8 and 1.1 of z3,inner sep=0pt](z5){$\loc'$};
      \node[right=1.8 of z4](z6){$\loc_\mathtt{halt}$};
      \path[arrow,every node/.style={font=\footnotesize}]
      (z0) edge node{$i\eqby?0$} (z1)
      (z2) edge node{$-\vec t$} (z3)
      (z3) edge node{$\vec 0$} (z4)
      (z3) edge[swap] node{$\vec 0$} (z5)
      (z4) edge node{$-2^{2^n}\cdot\bar{\vec c}_i$} (z6);
      % action of Adam
      \node[below=1.5cm of zmap](amap){$\rightsquigarrow$};
      \node[s-adam,left=2.75cm of amap](a0){$\loc$};
      \node[right=of a0](a1){$\loc'$};
      \node[right=1.25cm of amap,s-adam](a2){$\loc$};
      \node[right=of a2,s-eve-small](a3){};
      \node[right=of a3](a4){$\loc'$};
      \path[arrow,every node/.style={font=\footnotesize}]
      (a0) edge node{$\vec 0$} (a1)
      (a2) edge node{$\vec 0$} (a3)
      (a3) edge node{$-\vec t$} (a4);
    \end{tikzpicture}
    \caption{Schema of the reduction to
      "non-termination" in the proof of \cref{11-th:avag-hard}.}\label{11-fig:lipton-nonterm}
  \end{figure}

  We still need to extend our initialisation phase.  It suffices for
  this to implement a gadget for $\dd$-"meta-increments"
  $\loc\mstep{2^{\dd 2^j}\cdot\vec c}\loc'$ and $\dd$-"meta-decrements"
  $\loc\mstep{-2^{\dd 2^j}\cdot\vec c}\loc'$; this is the same argument as
  in Lipton's construction, with a base case $\loc\mstep{2^\dd}\loc'$
  for $j=0$.  Then we initialise our time budget through $|\Loc|$
  successive $\dd$-"meta-increments"
  $\loc\mstep{2^{\dd 2^n}\cdot\vec t}\loc'$.
  %%\end{scope}
\end{proof}

The proof of \cref{11-th:avag-hard} relies crucially on the fact that the
dimension is not fixed: although $\dd\geq 3$ suffices in the
"alternating Minsky machine", we need $O(|\?M|)$ additional counters
to carry out the reduction.  A separate argument is thus needed in
order to match the \EXP\ upper bound of \cref{11-th:avag-easy} in fixed
dimension.

\begin{theorem}[Fixed-dimensional coverability and non-termination asymmetric vector games are \EXP-hard]
\label{11-th:avag-two}
  "Coverability", "non-termination", and "parity@parity vector game"
  "asymmetric" "vector games" with "given initial credit" are
  \EXP-hard in dimension $\dd\geq 2$.
\end{theorem}
\begin{proof}
  We exhibit a reduction from "countdown games" with "given initial
  credit", which are \EXP-complete by \cref{11-th:countdown-given}.
  Consider an instance of a "configuration reachability" countdown
  game: a "countdown system"
  $\?V=(\Loc,\Act,\Loc_\mEve,\Loc_\mAdam,1)$ with initial
  configuration $\loc_0(n_0)$ and target
  configuration~$\smiley(0)$---as seen in the proof
  of \cref{11-th:countdown-given}, we can indeed assume that the target
  credit is zero; we will also assume that \Eve\ controls~$\smiley$ and
  that the only action available in~$\smiley$ is
  $\smiley\step{-1}\smiley$.  We construct an "asymmetric" "vector
  system" $\?V'$ of dimension~2 such that \Eve\ can ensure
  reaching~$\smiley(0,n_0)$ from $\loc_0(n_0,0)$ in~$\?V'$ if and only
  if she could ensure reaching $\smiley(0)$ from $\loc_0(n_0)$
  in~$\?V$.  The translation is depicted in \cref{11-fig:dim2}.
  
  \begin{figure}[htbp]
    \centering
    \begin{tikzpicture}[auto,on grid,node distance=1.5cm]
      \node(to){$\mapsto$};
      \node[anchor=east,left=2.5cm of to](mm){"countdown system"};
      \node[anchor=west,right=2.5cm of to](mwg){"asymmetric vector system"};
      % action of Eve
      \node[below=.7cm of to](imap){$\rightsquigarrow$};
      \node[s-eve,left=2.75cm of imap](i0){$\loc$};
      \node[right=of i0](i1){$\loc'$};
      \node[right=1.25cm of imap,s-eve](i2){$\loc$};
      \node[right=1.8 of i2](i3){$\loc'$};      
      \path[arrow,every node/.style={font=\footnotesize,inner sep=1pt}]
      (i0) edge node{$-n$} (i1)
      (i2) edge node{$-n,n$} (i3);
      % minimal action of Adam
      \node[below=1cm of imap](dmap){$\rightsquigarrow$};
      \node[s-adam,left=2.75cm of dmap](d0){$\loc$};
      \node[right=of d0](d1){$\loc'$};
      \node[below=.5 of d0]{$n=\min\{n'\mid\exists\loc''\in\Loc\mathbin.\loc\step{-n'}\loc''\in\Act\}$};
      \node[right=1.25cm of dmap,s-adam](d2){$\loc$};
      \node[right=1.8 of d2,s-eve-small](d3){};
      \node[right=1.8 of d3](d4){$\loc'$};
      \path[arrow,every node/.style={font=\footnotesize,inner sep=1pt}]
      (d0) edge node{$-n$} (d1)
      (d2) edge node{$0,0$} (d3)
      (d3) edge node{$-n,n$} (d4);
      % non-minimal action of Adam
      \node[below=1.5cm of dmap](zmap){$\rightsquigarrow$};
      \node[s-adam,left=2.75cm of zmap](z0){$\loc$};
      \node[right=of z0](z1){$\loc'$};
      \node[below=.5 of z0]{$n\neq\min\{n'\mid\exists\loc''\in\Loc\mathbin.\loc\step{-n'}\loc''\in\Act\}$};
      \node[right=1.25cm of zmap,s-adam](z2){$\loc$};
      \node[right=of z2,s-eve-small](z3){};
      \node[above right=.8 and 2.1 of z3](z4){$\loc'$};
      \node[below right=.8 and 2.1 of z3,s-eve](z5){$\smiley$};
      \path[arrow,every node/.style={font=\footnotesize,inner sep=1pt}]
      (z0) edge node{$-n$} (z1)
      (z2) edge node{$0,0$} (z3)
      (z3) edge[bend left=8] node{$-n,n$} (z4)
      (z3) edge[swap,bend right=8] node{$n_0-n+1,-n_0+n-1$} (z5)
      (z5) edge[loop above] node{$-1,1$} ()
      (z5) edge[loop right] node{$\,0,0$} ();
    \end{tikzpicture}
    \caption{Schema of the reduction in the proof
    of \cref{11-th:avag-two}.}\label{11-fig:dim2}
  \end{figure}
    
  The idea behind this translation is that a configuration $\loc(c)$
  of~$\?V$ is simulated by a configuration $\loc(c,n_0-c)$ in~$\?V'$.
  The crucial point is how to handle \Adam's moves.  In a
  configuration $\loc(c,n_0-c)$ with $\loc\in\Loc_\mAdam$, according
  to the "natural semantics" of $\?V$, \Adam\ should be able to
  simulate an action $\loc\step{-n}\loc'$ if and only if $c\geq n$.
  Observe that otherwise if $c<n$ and thus $n_0-c>n_0-n$, \Eve\ can
  play to reach~$\smiley$ and win immediately.  An exception to the
  above is if $n$ is minimal among the decrements in~$\loc$, because
  according to the "natural semantics" of~$\?V$, if $c<n$ there should
  be an edge to the "sink", and this is handled in the second line
  of \cref{11-fig:dim2}.

  Then \Eve\ can reach $\smiley(0,n_0)$ if and only if she can cover
  $\smiley(0,n_0)$, if and only if she can avoid the "sink" thanks to
  the self loop $\smiley\step{0,0}\smiley$.  This
  shows the \EXP-hardness of "coverability" and "non-termination"
  "asymmetric" "vector games" in dimension~two; the hardness of
  "parity@parity vector game" follows
  from \cref{11-rk:cov2parity,11-rk:nonterm2parity}.
\end{proof}

  
\subsection{Dimension One}
\label{11-sec:mono-dim1}
%\TODO{contents of \cref{11-subsec:mono-dim1} depend on what goes into \cref{chap:multiobjective}}

% Local IspellDict: british

\TODO{contents of \cref{11-sec:mono-dim1} depend on what goes into \cref{12-chap:multiobjective}}
