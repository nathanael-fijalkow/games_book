
\paragraph{Witness Trees}
We introduce now a concept that will also be used in \cref{11-subsec:up}
for solving games with "given initial credit".  As seen in the proof
of \cref{11-fact-finmem}, if \Eve wins a "parity@parity vector game"
"asymmetric vector game" from some vertex~$v_0$, then there is a
finite tree that witnesses the fact.

For two vectors $\vec v$ and $\vec v'$ in $\+Z^\dd$, we write $\vec
v\sqsubseteq \vec v'$ if, for all $1\leq i\leq\dd$, either $\vec
v(i)\geq 0$ and then $\vec v(i)\leq\vec v'(i)$, or $\vec v(i)<0$ and
$\vec v(i)=\vec v'(i)$.

\begin{definition}\label{11-def-wt}
  Given a "vector system"
  $\?V=(\Loc,\Act,\Loc_\mEve,\Loc_\mAdam,\dd)$, and a location
  colouring $\col{:}\,\Loc\to\{1,\dots,d\}$, a ""witness tree"" for a
  pair $\loc_0(\vec v_0)\in\Loc\times\+Z^\dd$ is a finite tree~$t$
  labelled by pairs in~$\Loc\times\+Z^\dd$ such that
\begin{enumerate}
\item its root label is~$\loc_0(\vec v_0)$,
\item if an internal node is labelled by $\loc(\vec v)$ with
  $\loc\in\Loc_\mEve$, then it has exactly one child labelled by
  some~$\loc'(\vec v')$ such that $\loc\step{\vec u}\loc'\in\Act$ and
  $\vec v'\sqsupseteq\vec v+\vec u$,
\item if an internal node is labelled by $\loc(\vec v)$ with
  $\loc\in\Loc_\mAdam$, then it has one child labelled~$\loc'(\vec
  v')$ for some $\vec v'\sqsupseteq\vec v+\vec u$ for each action
  $\loc\step{\vec u}\loc'\in\Act$,
\item\label{11-wt-self-even} if a leaf is labelled by $\loc(\vec v)$,
  consider the branch
  $\loc_0(\vec v_0),\loc_1(\vec v_1),\dots,\loc_n(\vec v_n)=\loc(\vec
  v)$ that reaches the leaf: then
  \begin{enumerate}
  \item\label{11-wt-self}there exists~$i<n$ such that
  $\loc_i=\loc$ and $\vec v_i\leq\vec v$---we call the node labelled by
  $\loc_i(\vec v_i)$ the ""return point"" of the leaf---, and
  \item\label{11-wt-even} the maximal priority observed between the two
  nodes, i.e., $\max_{i\leq j<n}\col(v_j)$, is even.
  \end{enumerate}
\end{enumerate}
\AP Let $I\subseteq \{1,\dots,\dd\}$.  A "witness tree" is
\emph{$I$-correct} if all its labels $\loc(\vec v)$ are such that
$\vec v(i)\geq 0$ for all $i\in I$; we call it ""correct"" if it is
$\{1,\dots,\dd\}$-correct.  An $\emptyset$-"correct" "witness tree" is
just a "witness tree".
\end{definition}

\begin{remark}\label{11-rk-wt}
  By \cref{11-nonterm2parity,12-cov2parity}, some simplifications of
  \cref{11-def-wt} are possible when dealing with "non-termination" or
  "coverability". In the case of "non-termination",
  condition~\ref{11-wt-self-even} is simplified to only require
  condition~\ref{11-wt-self}.  In the case of "coverability" with a
  target configuration $\loc_f(\vec v_f)$, condition~\ref{11-wt-self}
  is changed to require $\loc=\loc_f$ and $\vec v\geq\vec v_f$
  instead.
\end{remark}

If~$t$ is an $I$-"correct" "witness tree" and $\vec u\in\+Z^\dd$ is a
vector of integers, we write $t+\vec u$ for the tree where
every label $\loc(\vec v)$ is replaced by $\loc(\vec v+\vec u)$.
Observe that $t+\vec u$ is still a "witness tree", and that if $\vec
u(i)\geq 0$ for all $i\in I$, then it is still $I$-"correct".

\begin{lemma}[Witness Tree]\label{11-lem-wt}
  \Eve wins a "parity" "asymmetric vector game" with "existential
  initial credit" from an initial location $\loc_0$ if and only if
  there exists a "witness tree" for $\loc_0(\vec 0)$.
\end{lemma}
\begin{proof}
  If \Eve wins the game, then there exists an initial credit
  $\vec v_0\in\+N^\dd$ such that she wins the game from
  $\loc_0(\vec v_0)$.  The proof of~\cref{11-fact-finmem} shows that
  there exists a "correct" "witness tree"~$t$ for $\loc_0(\vec v_0)$.
  Then the tree $t-\vec v_0$ is an $\emptyset$-"correct" "witness
  tree" for $\loc_0(\vec 0)$.

  Conversely, if there exists a $\emptyset$-"correct" "witness tree"
  rooted by $\loc_0(\vec 0)$, let
  $z\eqdef\max\{|\vec v(i)|\mid \exists\loc\in\Loc\mathbin.\loc(\vec
  v)\text{ labels a node of }t,1\leq i\leq \dd,\text{ and }\vec v(i)<0\}$
  be the maximal negative value appearing in~$t$.  Let $\vec z$ be
  defined by $\vec z(i)\eqdef z$ for all $1\leq i\leq\dd$.  Then
  $t+\vec z$ is rooted by $\loc_0(\vec z)$ and is "correct".  As
  in~\cref{11-fact-finmem}, $t+\vec z$ describes a winning strategy
  for \Eve starting from~$\loc_0(\vec z)$.
\end{proof}

Let us denote by $h(t)$ the height of a tree~$t$.  We are interested
in bounding the height of "witness trees".  Fix an "asymmetric vector
system" $\?V$ and a location colouring~$\col$.  For a subset
$I\subseteq\{1,\dots,\dd\}$, we let
\begin{align}
  H_I&\eqdef \sup_{\loc_0(\vec v_0)\in\Loc\times\+Z^\dd}\min\{h(t)\mid t\text{
    is an $I$-"correct" "witness tree" for }\loc_0(\vec v_0)\}\;,
  \intertext{%
  be the height of minimal $I$-"correct" "witness trees", independently
  of their root label (where $H_I=0$ if no $I$-"correct" "witness tree"
  exists).  We also denote by $\Effect(t)$ the maximal effect of a "witness
    tree" with root label $\loc_0(\vec v_0)$, defined by}
  \Effect(t)&\eqdef\max\{|(\vec v_0-\vec v)(i)|\mid\exists\loc\in\Loc\mathbin.\loc(\vec
v)\text{ labels some node of }t\text{ and }1\leq i\leq\dd\}\;.\notag
  \intertext{Then}
  \Effect_I&\eqdef\sup_{\loc_0(\vec v_0)\in\Loc\times\+Z^\dd}\min\{\Effect(t)\mid t\text{
    is an $I$-"correct" "witness tree" for }\loc_0(\vec v_0)\}
  \intertext{bounds the effect of $I$-"correct" "witness trees".
    Observe that}
  \Effect_I&\leq H_I\cdot\|\Act\|\;.
\end{align}
In the following, we are going to provide upper bounds on
$\Delta_\emptyset$ depending on the underlying "asymmetric vector
system" and location colouring.

\paragraph{Coverability}
Let us begin with "coverability" of a target configuration
$\loc_f(\vec v_f)$.  Observe that, if two labels $\loc(\vec v)$ and
$\loc(\vec v')$ appear twice along a branch of a "witness tree"~$t$
for "coverability" as defined in \cref{11-rk-wt}, i.e., if
$t=C[C'[t']]$ for some context~$C$, some non-empty context~$C'$ with
root label~$\loc(\vec v)$, and some tree $t'$ with root label
$\loc(\vec v')$, then the ""shrunk"" tree $(C+(\vec v'-\vec v))[t']$,
where we have relabelled $C$ accordingly, is still a "coverability"
"witness tree".

\begin{claim}\label{11-wt-cov-empty}
  Minimal "coverability" "witness trees" have height at
  most~$H_\emptyset\leq |\Loc|$, thus $\Effect_\emptyset\leq|\Loc|\|\Act\|$.
\end{claim}

What \cref{11-wt-cov-empty} captures is the fact that, in a
"coverability" "asymmetric vector game" with "existential initial
credit", the counters play no role at all: \Eve has a winning strategy
for some initial credit in the "vector game" if and only if she has
one to reach the target location~$\loc_f$ in the finite game played
over~$\Loc$ and edges~$(\loc,\loc')$ whenever $\loc\step{\vec
u}\loc'\in\Act$ for some~$\vec u$.  This entails that "coverability"
"asymmetric vector games" are quite easy to solve.

\begin{theorem}\label{11-cov-exist-P}
  "Coverability" "asymmetric vector games" with "existential initial
  credit" are \P-complete.
\end{theorem}
\begin{proof}
  For the upper bound, by \cref{11-lem-wt}, \Eve wins the game if and
  only if there exists a "coverability" "witness tree".  As we have
  just discussed, an alternating Turing machine can guess and check
  the existence of such a tree using logarithmic space: it only has to
  remember the current location.  The lower bound already holds for
  finite reachability games.
\end{proof}

\paragraph{Non-termination}
Let us turn to "witness trees" for "non-termination".  An analogue
of \cref{11-wt-cov-empty} exists for that case, but its proof is much
too difficult to be included in this chapter.
\begin{claim}
  "Non-termination" "witness trees" of minimal effect have effect at most
  $\Effect_\emptyset\leq(|\Loc|\|\Act\|)^{\poly(d)}$.
\end{claim}
\begin{proof}
  By~\cite{Jurdzinski&Lazic&Schmitz:2015}, there is bound of
  $B_\emptyset\eqdef(4|\Loc|\|\Act\|)^{2(\dd+2)}$ on the norms of the
  labels in a "witness tree" for "non-termination".  Let $H'\eqdef
  |\Loc|\|\Act\| (B_\emptyset+1)^{\dd}$ and assume~$t$ is a witness
  tree of minimal height.  Consider a leaf along with its "return
  point".  Then the length of the segment from the return point to the
  leaf cannot exceed~$H'$, as otherwise some node label would repeat
  within the segment, and the tree could be "shrunk"
\end{proof}
