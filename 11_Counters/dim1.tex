\AP \Cref{11-th:undec} leaves open whether "vector games" might be
decidable in dimension one.  They are indeed decidable, and more
generally we learned in \cref{10-chap:pushdown} that "one-counter
games"---with the additional ability to test the counter for
zero---were decidable and in fact \PSPACE-complete.  This might seem
to settle the case of "vector games" in dimension one, except that the
"one-counter games" of \cref{10-chap:pushdown} only allow integer
weights in $\{-1,1\}$, whereas we allow arbitrary updates in~$\+Z$
with a binary encoding.  Hence the \PSPACE\ upper bound of
\cref{10-chap:pushdown} becomes an~\EXPSPACE\ one for ""succinct
one-counter games"".

\begin{corollary}[One-dimensional vector games are in \EXPSPACE]
\label{11-cor:dim1}
  "Configuration reachability", "coverability", "non-termination", and
  "parity@parity vector game" "vector games", both with "given" and with "existential
  initial credit", are in \EXPSPACE\ in dimension one.
\end{corollary}
\begin{proof}\hfill\\
  \TODO{proof of \cref{11-cor:dim1} depends on how \cref{10-sec:one-counter} is written}
\end{proof}

The goal of this section is therefore to establish that this
\EXPSPACE\ upper bound is tight (in most cases), by proving a matching
lower bound in \cref{11-sec:one-counter}.  But first, we will study a
class of one-dimensional "vector games" of independent interest in
\cref{11-sec:countdown}: "countdown games".

\subsection{Countdown Games}
\label{11-sec:countdown}

\AP A one-dimensional "vector system"
$\?V=(\Loc,\Act,\Loc_\mEve,\Loc_\mAdam,1)$ is called a ""countdown
system"" if $\Act\subseteq\Loc\times\+Z_{<0}\times\Loc$, that is, if
for all $(\loc\step z\loc')\in\Act$, $z<0$.  We consider the games
defined by "countdown systems", both with "given" and with
"existential initial credit", and call the resulting games ""countdown
games"".

\begin{theorem}[Countdown games are \EXP-complete]
\label{11-th:countdown-given}
  "Configuration reachability" and "coverability" "countdown games"
  with "given initial credit" are \EXP-complete.
\end{theorem}
\begin{proof}
  For the upper bound, consider an instance, i.e., a "countdown
  system" $\?V=(\Loc,\Act,\Loc_\mEve,\Loc_\mAdam,1)$, an initial
  location $\loc_0\in\Loc$, an initial credit $n_0\in\+N$, and a
  target configuration $\loc_f(n_f)\in\Loc\times\+N$.  Because every
  action decreases strictly the counter value, the reachable part
  of the "natural semantics" of $\?V$ starting from $\loc_0(n_0)$ is
  finite and of size at most $1+|\Loc|\cdot (n_0+1)$, and because~$n_0$
  is encoded in binary, this is at most exponential in the size of the
  instance.  As seen in \cref{2-chap:regular}, such a "reachability"
  game can be solved in time polynomial in the size of the finite
  graph, thus in \EXP\ overall.

  \medskip For the lower bound, we start by considering a game played
  over an exponential-time Turing machine, before showing how to
  implement this game as a "countdown game".  Let us consider for this
  an arbitrary Turing machine~$\?M$ working in deterministic
  exponential time~$2^{p(n)}$ for some fixed polynomial~$p$ and an
  input word~$w=a_1\cdots a_n$ of length~$n$, which we assume to be
  positive.  Let $m\eqdef 2^{p(n)}\geq n$.  The computation of~$\?M$
  on~$w$ is a sequence of configurations $C_1,C_2,\dots,C_t$ of
  length~$t\leq m$.  Each configuration $C_i$ is of the form
  $\emkl \gamma_{i,1}\cdots\gamma_{i,m}\emkr$ where $\emkl$ and
  $\emkr$ are endmarkers and the symbols $\gamma_{i,j}$ are either
  taken from the finite tape alphabet~$\Gamma$ (which includes a blank
  symbol~$\blank$) or a pair $(q,a)$ of a state from~$Q$ and a tape
  symbol~$a$.  We assume that the set of states~$Q$ contains a single
  accepting state~$q_\mathrm{final}$.  The entire computation can be
  arranged over a $t\times m$ grid where each line corresponds to a
  configuration~$C_i$, as shown in \cref{11-fig:exp}.

  \begin{figure}[htbp]
    \centering
    \hspace*{-.5ex}\begin{tikzpicture}[on grid,every node/.style={anchor=base}]
      \draw[step=1,lightgray!50,dotted] (-.5,-0.8) grid (10.5,-5.2);
      % rows
      %\draw[white](-.5,-3) -- (10.5,-3) (-.5,-6) -- (10.5,-6);
      \node[anchor=east] at (-.5,-5) {$C_1$};
      \node[anchor=east] at (-.5,-4) {$C_2$};
      \node[anchor=east] at (-.5,-3.4) {$\vdots~$};
      \node[anchor=east] at (-.5,-3) {$C_{i-1}$};
      \node[anchor=east] at (-.5,-2) {$C_i$};
      \node[anchor=east] at (-.5,-1.4) {$\vdots~$};
      \node[anchor=east] at (-.5,-1) {$C_t$};
      % columns
      \draw[color=white](4,-.5) -- (4,-5.2) (8,-.5) -- (8,-5.2);
      \node[lightgray] at (0,-.5) {$0$};
      \node[lightgray] at (1,-.5) {$1$};
      \node[lightgray] at (2,-.5) {$2$};
      \node[lightgray] at (3,-.5) {$3$};
      \node[lightgray] at (4,-.5) {$\cdots$};
      \node[lightgray] at (5,-.5) {$j-1$};
      \node[lightgray] at (6,-.5) {$j$};
      \node[lightgray] at (7,-.5) {$j+1$};
      \node[lightgray] at (8,-.5) {$\cdots$};
      \node[lightgray] at (9,-.5) {$m$};
      \node[lightgray] at (10,-.5) {$m+1$};
      % endmarkers
      \node at (0,-1.1) {$\emkl$};
      \node at (0,-2.1) {$\emkl$};
      \node at (0,-3.1) {$\emkl$};
      \node at (0,-4.1) {$\emkl$};
      \node at (0,-5.1) {$\emkl$};
      \node at (10,-1.1) {$\emkr$};
      \node at (10,-2.1) {$\emkr$};
      \node at (10,-3.1) {$\emkr$};
      \node at (10,-4.1) {$\emkr$};
      \node at (10,-5.1) {$\emkr$};
      % initial
      \node at (1,-5.1) {$q_0,a_1$};
      \node at (2,-5.1) {$a_2$};
      \node at (3,-5.1) {$a_3$};
      \node at (4,-5.1) {$\cdots$};
      \node at (5,-5.1) {$\blank$};
      \node at (6,-5.1) {$\blank$};
      \node at (7,-5.1) {$\blank$};
      \node at (8,-5.1) {$\cdots$};
      \node at (9,-5.1) {$\blank$};
      % second
      \node at (1,-4.1) {$a'_1$};
      \node at (2,-4.1) {$q_1,a_2$};
      \node at (3,-4.1) {$a_3$};
      \node at (4,-4.1) {$\cdots$};
      \node at (5,-4.1) {$\blank$};
      \node at (6,-4.1) {$\blank$};
      \node at (7,-4.1) {$\blank$};
      \node at (8,-4.1) {$\cdots$};
      \node at (9,-4.1) {$\blank$};
      % row i-1
      \node at (5,-3.7) {$\vdots$};
      \node at (6,-3.7) {$\vdots$};
      \node at (7,-3.7) {$\vdots$};
      \node at (4,-3.1) {$\cdots$};
      \node at (5,-3.1) {$\gamma_{i-1,j-1}$};
      \node at (6,-3.1) {$\gamma_{i-1,j}$};
      \node at (7,-3.1) {$\gamma_{i-1,j+1}$};
      \node at (8,-3.1) {$\cdots$};
      % row i
      \node at (5,-2.1) {$\cdots$};
      \node at (6,-2.1) {$\gamma_{i,j}$};
      \node at (7,-2.1) {$\cdots$};
      \node at (6,-1.7) {$\vdots$};
      % final
      \node at (1,-1.1) {$q_\mathrm{final},\blank$};
      \node at (2,-1.1) {$\blank$};
      \node at (3,-1.1) {$\blank$};
      \node at (4,-1.1) {$\cdots$};
      \node at (5,-1.1) {$\blank$};
      \node at (6,-1.1) {$\blank$};
      \node at (7,-1.1) {$\blank$};
      \node at (8,-1.1) {$\cdots$};
      \node at (9,-1.1) {$\blank$};      
      \end{tikzpicture}
    \caption{The computation of~$\?M$ on
  input~$w=a_1\cdots a_n$.  This particular picture assumes~$\?M$
  starts by rewriting~$a_1$ into $a'_1$ and moving to the right in a
  state~$q_1$, and empties its tape before accepting its input by going
  to state~$q_\mathrm{final}$.}\label{11-fig:exp}
  \end{figure}

  We now set up a two-player game where \Eve\ wants to prove that the
  input~$w$ is accepted.  Let
  $\Gamma'\eqdef \{\emkl,\emkr\}\cup\Gamma\cup(Q\times\Gamma)$. Rather
  than exhibiting the full computation from \cref{11-fig:exp}, the
  game will be played over positions $(i,j,\gamma_{i,j})$ where
  $0<i\leq m$, $0\leq j\leq m+1$, and $\gamma_{i,j}\in\Gamma'$.  \Eve
  wants to show that, in the computation of~$\?M$ over~$w$ as depicted
  in \cref{11-fig:exp}, the $j$th cell of the $i$th
  configuration~$C_i$ contains~$\gamma_{i,j}$.  In order to
  substantiate this claim, observe that the content of any cell
  $\gamma_{i,j}$ in the grid is determined by the actions of~$\?M$
  and the contents of (up to) three cells in the previous
  configuration.  Thus, if $i>1$ and $0<j<m+1$, \Eve\ provides a triple
  $(\gamma_{i-1,j-1},\gamma_{i-1,j},\gamma_{i-1,j+1})$ of symbols
  in~$\Gamma'$ that yield $\gamma_{i,j}$ according to the actions
  of~$\?M$, which we denote by
  $\gamma_{i-1,j-1},\gamma_{i-1,j},\gamma_{i-1,j+1}\vdash\gamma_{i,j}$,
  and \Adam\ chooses $j'\in\{j-1,j,j+1\}$ and returns the control
  to \Eve\ in position~$(i-1,j',\gamma_{i-1,j'})$.  Regarding the
  boundary cases where $i=0$ or $j=0$ or $j=m+1$, \Eve\ wins
  immediately if $j=0$ and $\gamma={\emkl}$, or if $j=m+1$ and
  $\gamma={\emkr}$, or if $i=0$ and $0<j\leq n$ and $\gamma=a_j$, or if
  $i=0$ and $n<j\leq m$ and $\gamma={\blank}$, and otherwise \Adam\ wins
  immediately.  The game starts in a position
  $(t,j,(q_\mathrm{final},a))$ for some $0<t\leq m$, $0< j\leq m$,
  and~$a\in\Gamma$ of \Eve's choosing.  It should be clear that \Eve
  has a winning strategy in this game if and only if~$w$ is accepted
  by~$\?M$.

  We now implement the previous game as a "coverability" game over a
  "countdown system" $\?V\eqdef(\Loc,\Act,\Loc_\mEve,\Loc_\mAdam,1)$.
  The idea is that the pair $(i,j)$ will be encoded as
  $(i-1)\cdot(m+2)+j+2$ in the counter value, while the
  symbol~$\gamma_{i,j}$ will be encoded in the location.  For
  instance, the endmarker $\emkl$ at position $(1,0)$ will be
  represented by configuration $\loc_{\emkl}(2)$, the first input
  $(q_0,a_1)$ at position~$(1,1)$ by $\loc_{(q_0,a_1)}(3)$, and the
  endmarker $\emkr$ at position $(m,m+1)$ by
  $\loc_{\emkr}(m\cdot(m+2)+1)$. The game starts from the initial
  configuration $\loc_0(n_0)$ where $n_0\eqdef m\cdot(m+2)+1$ and the
  target location is~$\smiley$.

  We define for this the sets of locations
  \begin{align*}
    \Loc_\mEve&\eqdef\{\loc_0,\smiley,\frownie\}
               \cup\{\loc_\gamma\mid\gamma\in\Gamma'\}\;,\\
    \Loc_\mAdam&\eqdef\{\loc_{(\gamma_1,\gamma_2,\gamma_3)}\mid\gamma_1,\gamma_2,\gamma_3\in\Gamma'\}
               \cup\{\loc_{=j}\mid 0<j\leq n\}
               \cup\{\loc_{1\leq?\leq m-n+1}\}\;.
  \end{align*}
  The intention behind the locations $\loc_{=j}\in\Loc_\mAdam$ is
  that \Eve\ can reach~$\smiley$ from a configuration $\loc_{=j}(c)$ if
  and only if $c=j$; we accordingly define~$\Act$ with the following
  actions, where~$\frownie$ is a deadlock location:
  \begin{align*}
    \loc_{=j}&\step{-j-1}\frownie\;,&\loc_{=j}&\step{-j}\smiley\;.
  \intertext{Similarly, \Eve\ should be able to reach~$\smiley$ from
  $\loc_{1\leq?\leq m-n+1}(c)$ if and only if $1\leq c\leq m-n+1$,
  which is implemented by the actions}
    \loc_{1\leq?\leq m-n+1}&\step{-m+n-2}\frownie\;,&
    \loc_{1\leq?\leq m-n+1}&\step{-1}\smiley\;,&
    \smiley&\step{-1}\smiley\;.
  \end{align*}
  Note this last action also ensures that \Eve\ can reach the
  location~$\smiley$ if and only if she can reach the configuration
  $\smiley(0)$, thus the game can equivalently be seen as a
  "configuration reachability" game.

  Regarding initialisation, \Eve\ can choose her initial position,
  which we implement by the actions
  \begin{align*}
    \loc_0 &\step{-1} \loc_0 & \loc_0 &\step{-1}\loc_{(q_\mathrm{final},a)}&&\text{for $a\in\Gamma$}\;.
    \intertext{Outside the boundary cases, the game is implemented by
    the following actions:}
    \loc_\gamma&\step{-m}\loc_{(\gamma_1,\gamma_2,\gamma_3)}&&&&\text{for
  $\gamma_1,\gamma_2,\gamma_3\vdash\gamma$}\;,\\ \loc_{(\gamma_1,\gamma_2,\gamma_3)}&\step{-k}\loc_{\gamma_k}&&&&\text{for
  $k\in\{1,2,3\}$}\;.
  \intertext{We handle the endmarker positions via the following
  actions, where \Eve\ proceeds along the left edge
  of \cref{11-fig:exp} until she reaches the initial left endmarker:}
   \loc_\emkl&\step{-m-2}\loc_\emkl\;,& \loc_\emkl&\step{-1}\loc_{=1}\;,& \loc_\emkr&\step{-m-1}\loc_\emkl\;.
  \intertext{For the positions inside the input word $w=a_1\cdots
  a_n$, we use the actions}
  \loc_{(q_0,a_1)}&\step{-2}\loc_{=1}\;,&\loc_{a_j}&\step{-2}\loc_{=j}&&\text{for
  $1<j\leq n$}\;.
  \intertext{Finally, for the blank symbols of~$C_1$, which should be
  associated with a counter value~$c$ such that $n+3\leq c\leq m+3$, we use the
  action}
  \loc_\blank&\step{-n-2}\loc_{1\leq?\leq m-n+1}\;.&&&&&\qedhere\hspace*{-1.5em}
  \end{align*}
\end{proof}


\begin{theorem}[Existential countdown games are \EXPSPACE-complete]
\label{11-th:countdown-exist}
  "Configuration reachability" and "coverability" "countdown games"
  with "existential initial credit" are \EXPSPACE-complete.
\end{theorem}
\begin{proof}
   For the upper bound, consider an instance, i.e., a "countdown
   system" $\?V=(\Loc,\Act,\Loc_\mEve,\Loc_\mAdam,1)$, an initial
   location~$\loc_0$, and a target configuration $\loc_f\in\Loc$.  We
   reduce this to an instance of "configuration reachability" with
   "given initial credit" in a one-dimensional "vector system" by
   adding a new location $\loc'_0$ controlled by~\Eve\ with actions
   $\loc'_0\step{1}\loc'_0$ and $\loc'_0\step 0\loc_0$, and asking
   whether \Eve\ has a winning strategy starting from $\loc'_0(0)$ in
   the new system.  By \cref{11-cor:dim1}, this "configuration
   reachability" game can be solved in \EXPSPACE.

   \medskip For the lower bound, we reduce from the acceptance problem
   of a deterministic Turing machine working in exponential space.
   The reduction is the same as in the proof
   of \cref{11-th:countdown-given}, except that now the length~$t$ of the
   computation is not bounded a priori, but this is compensated by the
   fact that we are playing the "existential initial credit" version
   of the "countdown game".  \qedhere%%   Indeed, \Eve\ wins the "countdown game"
   %% starting from configuration $\loc_{q_\mathrm{final},a}(n)$ if and
   %% only if she wins from position
   %% $(\lfloor(n-1)/(m+2)\rfloor+1,(n-1)\mathrel\mathrm{mod}(m-2),(q_\mathrm{final},a))$
   %% in the game played on the Turing machine.
\end{proof}

\medskip
Originally, "countdown games" were introduced with a slightly
different objective, which corresponds to the following decision
problem.
\AP\decpb["zero reachability" with "given initial credit"]
  {A "countdown system" $\?V=(\Loc,T,\Loc_\mEve,\Loc_\mAdam,1)$, an
  initial location $\loc_0\in\Loc$, and an initial credit
  $n_0\in\+N$.}
  {Does \Eve\ have a strategy to reach a configuration $\loc(0)$ for
  some $\loc\in\Loc$?\\That is, does she win the ""zero
  reachability""\index{zero reachability|see{countdown game}}
  game $(\?A_\+N(\?V),\col,\Reach)$ from $\loc_0(n_0)$, where
  $\col(e)=\Win$ if and only if $\ing(e)=\loc(0)$ for some $\loc\in\Loc$?}
\begin{theorem}[Countdown to zero games are \EXP-complete]
\label{11-th:countdown-zero}
  "Zero reachability" "countdown games" with "given initial credit"
  are \EXP-complete.
\end{theorem}
\begin{proof}
  The upper bound of \cref{11-th:countdown-given} applies in the same
  way.  Regarding the lower bound, we modify the lower bound
  construction of \cref{11-th:countdown-given} in the following way: we
  use $\loc_0(2\cdot n_0+1)$ as initial configuration, multiply all
  the action weights in~$\Act$ by two, and add a new
  location~$\loc_\mathrm{zero}$ with an action
  $\smiley\step{-1}\loc_\mathrm{zero}$.  Because all the counter
  values in the new game are odd unless we reach $\loc_\mathrm{zero}$,
  the only way for \Eve\ to bring the counter to zero in this new game
  is to first reach $\smiley(1)$, which occurs if and only if she
  could reach $\smiley(0)$ in the original game.
\end{proof}

\subsection{Vector Games in Dimension One}
\label{11-sec:one-counter}

"Countdown games" are frequently employed to prove complexity lower
bounds.  Here, we use them to show that the \EXPSPACE\ upper bounds
from \cref{11-cor:dim1} are tight in most cases.
\begin{theorem}[The complexity of vector games in dimension one]
\label{11-th:dim1}
  "Configuration reachability", "coverability", and "parity@parity
  vector game" "vector games", both with "given" and with "existential
  initial credit", are \EXPSPACE-complete in dimension one;
  "non-termination" "vector games" in dimension one are \EXP-hard with
  "given initial credit" and \EXPSPACE-complete with "existential
  initial credit".
\end{theorem}
\begin{proof}
  By \cref{11-th:countdown-exist}, "configuration reachability" and
  "coverability" "vector games" with existential initial credit
  are \EXPSPACE-hard in dimension one.
  Furthermore, \cref{11-rk:cov2parity} allows to deduce that
  "parity@parity vector game" is also \EXPSPACE-hard.  Finally, we can
  argue as in the upper bound proof of \cref{11-th:countdown-exist} that
  all these games are also hard with "given initial credit": we add a
  new initial location $\loc'_0$ controlled by \Eve\ with actions
  $\loc'_0\step 1\loc'_0$ and $\loc'_0\step 0\loc_0$ and play the game
  starting from $\loc'_0(0)$.

  Regarding "non-termination", we can add a self loop $\smiley\step
  0\smiley$ to the construction
  of \cref{11-th:countdown-given,11-th:countdown-exist}: then the only way
  to build an infinite play that avoids the "sink" is to reach the
  target location $\smiley$.  This shows that the games are \EXP-hard
  with "given initial credit" and \EXPSPACE-hard with "existential
  initial credit".  Note that the trick of reducing "existential" to
  "given initial credit" with an initial incrementing loop $\loc'_0\step
  1\loc'_0$ does not work, because \Eve\ would have a trivial winning
  strategy that consists in just playing this loop forever.
\end{proof}

%% To the best of my knowledge, there is a complexity gap for 
%% "non-termination" "vector games" with "given initial credit" in
%% dimension one, between \EXP-hardness and \EXPSPACE-easiness.

% Local IspellDict: british
