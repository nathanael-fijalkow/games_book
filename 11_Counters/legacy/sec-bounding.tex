\subsection{Bounded Semantics}
\label{11-bounding}

While Adam wins immediately in an "energy game" if a resource gets
depleted, he also wins in a "bounding game" if a resource reaches a
certain bound~$B$.  %By contrast with "capped semantics", t
This is
a \emph{hard upper bound}, allowing to model systems where exceeding a
capacity results in failure, like a dam that overflows and floods the
area.  We define for a bound~$B\in\+N$ the ""bounded semantics""
$\bounded(\?V)=(V^B,E^B,\VE^B,\VA^B)$ of a "vector system"~$\?V$ by
\begin{align*}
  V^B&\eqdef\{\loc(\vec v)\mid\loc\in\Loc\text{ and }\|\vec v\|<B\}\;,\\
  E^B&\eqdef \{(\loc(\vec v),\loc'(\vec v+\vec u))\mid\loc\step{\vec
       u}\loc'\in\Act,\vec v+\vec u\geq\vec 0,\text{ and }\|\vec
       v+\vec u\|<B\}\\
     &\:\cup\:\{(\loc(\vec v),\sink)\mid\forall\loc\step{\vec
               u}\loc'\in\Act\mathbin.\vec v+\vec u\not\geq\vec
               0\text{ or }\|\vec v+\vec u\|\geq B\}
     \cup\{(\sink,\sink)\}\;.
\end{align*}
As usual, $\VE^B\eqdef V^B\cap\Loc_\mEve\times\+N^\dd$ and
$\VA^B\eqdef V^B\cap\Loc_\mAdam\times\+N^\dd$.  Any edge from the
"energy semantics" that would bring to a configuration $\loc(\vec v)$
with $\vec v(i)\geq B$ for some $1\leq i\leq\dd$ leads instead to the
sink.  All the configurations in this arena have "norm" less than~$B$,
thus $|V^B|=|\Loc| B^\dd+1$, and the qualitative games of
\cref{chap:regular} are decidable over this "arena".

Our focus here is on "non-termination" games played on the "bounded
semantics" where~$B$ is not given as part of the input, but quantified
existentially.  As usual, the "existential initial credit" variant
of \cref{11-pb-bounding} is obtained by quantifying~$\vec v_0$
existentially in the question.
\decpb["bounding game" with "given initial credit"]%
{\label{11-pb-bounding} A "vector system"
  $\?V=(\Loc,\Act,\Loc_\mEve,\Loc_\mAdam,\dd)$, an initial location
  $\loc_0\in\Loc$, and an initial credit $\vec v_0\in\+N^\dd$.}%
  {Does there exist $B\in\+N$ such that Eve has a strategy to avoid the
  "sink"~$\sink$ from $\loc_0(\vec v_0)$ in the "bounded
  semantics"?  That is, does there exist $B\in\+N$ such that she wins
  the ""bounding"" game $(\bounded(\?V),\col,\Safe)$ from
  $\loc_0(\vec v_0)$, where $\col(e)\eqdef\Lose$ if and only if $\ing(e)=\sink$?}

\begin{lemma}\label{11-parity2bounding}
  There is a \logspace\ reduction from "parity@parity vector games"
  "asymmetric" "vector games" to "bounding games", both with "given"
  and with "existential initial credit".
\end{lemma}
\begin{proof}
  Given an "asymmetric vector system"
  $\?V=(\Loc,\Act,\Loc_\mEve,\Loc_\mAdam,\dd)$, a location colouring
  $\lcol{:}\,\Loc\to\{1,\dots,2d\}$, and an initial location
  $\loc_0\in\Loc$, we construct a "vector system" $\?V'$ of dimension
  $\dd'\eqdef\dd+d$ as described in \cref{11-fig-bounding}, where the
  priorities in~$\?V$ for $p\in\{1,\dots,d\}$ are indicated above the
  corresponding locations.
  
  \begin{figure}[htbp]
    \centering
    \begin{tikzpicture}[auto,on grid,node distance=1.5cm]
      \node(to){$\mapsto$};
      \node[anchor=east,left=2.5cm of to](mm){"asymmetric vector system"~$\?V$};
      \node[anchor=west,right=2.5cm of to](mwg){"vector system"~$\?V'$};
      % Eve location, even parity
      \node[below=1.3cm of to](imap){$\rightsquigarrow$};
      \node[s-eve,left=2.75cm of imap](i0){$\loc$};
      \node[black!50,above=.4 of i0,font=\scriptsize]{$2p$};
      \node[right=of i0](i1){$\loc'$};
      \node[right=1.25cm of imap,s-eve](i2){$\loc$};
      \node[right=1.8 of i2,s-eve-small](i3){};      
      \node[right=1.8 of i3](i4){$\loc'$};      
      \path[arrow,every node/.style={font=\footnotesize}]
      (i0) edge node{$\vec u$} (i1)
      (i2) edge[loop above] node{$\forall 1\leq i\leq\dd\mathbin.-\vec e_i$} ()
      (i2) edge node{$\vec u$} (i3)
      (i3) edge[loop below] node{$\forall 1\leq j\leq p\mathbin.\vec e_{\dd+j}$} ()
      (i3) edge node{$\vec 0$} (i4);
      % Eve location, odd parity
      \node[below=2cm of imap](dmap){$\rightsquigarrow$};
      \node[s-eve,left=2.75cm of dmap](d0){$\loc$};
      \node[black!50,above=.4 of d0,font=\scriptsize]{$2p-1$};
      \node[right=of d0](d1){$\loc'$};
      \node[right=1.25cm of dmap,s-eve](d2){$\loc$};
      \node[right=2 of d2](d3){$\loc'$};
      \path[arrow,every node/.style={font=\footnotesize}]
      (d0) edge node{$\vec u$} (d1)
      (d2) edge[loop above] node{$\forall 1\leq i\leq\dd\mathbin.-\vec e_i$} ()
      (d2) edge node{$\vec u-\vec e_{\dd+p}$} (d3);
      % Adam location, even parity
      \node[below=1.1cm of dmap](zmap){$\rightsquigarrow$};
      \node[s-adam,left=2.75cm of zmap](z0){$\loc$};
      \node[black!50,above=.4 of z0,font=\scriptsize]{$2p$};
      \node[right=of z0](z1){$\loc'$};
      \node[right=1.25cm of zmap,s-adam](z2){$\loc$};
      \node[right=of z2,s-eve-small](z3){};
      \node[right=of z3](z4){$\loc'$};
      \path[arrow,every node/.style={font=\footnotesize}]
      (z0) edge node{$\vec 0$} (z1)
      (z2) edge node{$\vec 0$} (z3)
      (z3) edge node{$\vec 0$} (z4)
      (z3) edge[loop below] node{$\forall 1\leq j\leq p\mathbin.\vec e_{\dd+j}$} ();
      % Adam location, odd parity
      \node[below=1.6cm of zmap](amap){$\rightsquigarrow$};
      \node[s-adam,left=2.75cm of amap](a0){$\loc$};
      \node[black!50,above=.4 of a0,font=\scriptsize]{$2p-1$};
      \node[right=of a0](a1){$\loc'$};
      \node[right=1.25cm of amap,s-adam](a2){$\loc$};
      \node[right=2 of a2](a3){$\loc'$};
      \path[arrow,every node/.style={font=\footnotesize}]
      (a0) edge node{$\vec 0$} (a1)
      (a2) edge node{$-\vec e_{\dd+p}$} (a3);
    \end{tikzpicture}
    \caption{\label{11-fig-bounding}Schema of the reduction to
      "bounding games" in the proof of \cref{11-parity2bounding}.}
  \end{figure}
  
  If Eve wins the "bounding game" played over $\?V'$ from some
  configuration $\loc_0(\vec v_0)$, then she also wins the "parity
  vector game" played over~$\?V$ from the configuration $\loc_0(\vec
  v'_0)$ where $\vec v'_0$ is the projection of $\vec v_0$
  to~$\+N^\dd$.  Indeed, she can play essentially the same strategy:
  by \cref{11-fact-mono} she can simply ignore the new decrement
  self loops, while the actions on the components in
  $\{\dd+1,\dots,\dd+d\}$ ensure that the maximal priority visited
  infinitely often is even---otherwise some decrement $-\vec
  e_{\dd+p}$ would be played infinitely often but the increment $\vec
  e_{\dd+p}$ only finitely often.\todoquestion{Should I provide more details?}
  

  \medskip
  Conversely, consider the "parity@parity vector game" game~$\game$ played over
  $\natural(\?V)$ with the colouring defined by~$\lcol$.  Then the
  "Pareto limit" of the game is finite, thus there exists a natural
  number
  \begin{equation}\label{11-b0}
    B_0\eqdef 1+\max_{\loc_0(\vec v_0)\in\mathsf{Pareto}(\?G)}\|\vec
  v_0\|
  \end{equation} bounding the "norms" of the minimal winning configurations.
  For a vector~$\vec v$ in~$\+N^\dd$, let us write $\capp[B_0]v$ for
  the vector `capped' at~$B$: for all~$1\leq i\leq\dd$,
  $\capp[B_0]v(i)\eqdef\vec v(i)$ if $\vec v(i)<B_0$ and
  $\capp[B_0]v\eqdef B_0$ if $\vec v(i)\geq B_0$.

  %% \begin{claim}\label{11-cl-capped}
  %%   If $\loc(\vec v)\in\WE(\game)$, then $\loc(\capp[B_0]v)\in\WE(\game)$.
  %% \end{claim}
  %% Indeed, by definition of the "Pareto limit"~$\mathsf{Pareto}(\game)$,
  %% if $\loc(\vec v)\in\WE(\game)$, then there exists~$\vec v'\leq\vec v$
  %% such that $\loc(\vec v')\in\mathsf{Pareto}(\game)$.  By definition of
  %% the bound~$B_0$, $\|\vec v'\|<B_0$.  Thus $\vec v'\leq\capp[B_0]v$.
  %% Thus $\loc(\capp[B_0]v)\in\WE(\game)$.

  Consider now some configuration $\loc_0(\vec
  v_0)\in\mathsf{Pareto}(\game)$.  As seen in \cref{11-fact-finmem},
  since $\loc_0(\vec v_0)\in\WE(\game)$, there is a finite
  "self-covering tree" witnessing the fact, and an associated winning
  strategy.  Let $H(\loc_0(\vec v_0))$ denote the height of this
  "self-covering tree" and observe that all the configurations in this
  tree have norm bounded by $\|\vec v_0\|+\|\Act\|\cdot H(\loc_0(\vec
  v_0))$.
  Let us define
  \begin{equation}\label{11-b}
   B\eqdef B_0+(\|\Act\|+1)\cdot \max_{\loc_0(\vec
  v_0)\in\mathsf{Pareto}(\?G)}H(\loc_0(\vec v_0))\;.
  \end{equation}
  This is a bound on the norm of the configurations appearing on the
  (finitely many) self-covering trees spawned by the elements
  of~$\mathsf{Pareto}(\game)$.  Note that $B\geq B_0+(\|\Act\|+1)$ since
  a self-covering tree has height at least~one.

  Consider the "non-termination" game
  $\game_B\eqdef(\bounded(\?V'),\col',\Safe)$ played over the
  "bounded semantics" defined by~$B$, where $\col'(e)=\Lose$ if and
  only if $\ing(e)=\sink$.  Let $\vec b\eqdef\sum_{1\leq p\leq
  d}(B-1)\cdot\vec e_{\dd+p}$.
  {\renewcommand{\qedsymbol}{}
  \begin{claim}\label{11-cl-parity2bounding} If $\loc_0(\vec
    v)\in\WE(\game)$, then
    $\loc_0(\capp[B_0]{v}+\vec b)\in\WE(\game_B)$.
  \end{claim}}
  Indeed, by definition of the "Pareto
  limit"~$\mathsf{Pareto}(\game)$, if $\loc_0(\vec v)\in\WE(\game)$,
  then there exists~$\vec v_0\leq\vec v$ such that $\loc_0(\vec
  v_0)\in\mathsf{Pareto}(\game)$.  By definition of the bound~$B_0$,
  $\|\vec v_0\|<B_0$, thus $\vec v_0\leq\capp[B_0]v$.  Consider the
  "self-covering tree" of height~$H(\loc_0(\vec v_0))$ associated
  to~$\loc_0(\vec v_0)$, and the strategy~$\sigma'$ defined by the
  memory structure from the
  proof of \cref{11-fact-finmem}.  This is a winning strategy for Eve
  in $\game$ starting from $\loc_0(\vec v_0)$, and
  by \cref{11-fact-mono}, it is also winning
  from~$\loc_0(\capp[B_0]v)$.
    
  Here is how Eve wins $\game_B$ from~$\loc_0(\capp[B_0]v+\vec b)$.
  She essentially follows the strategy~$\sigma'$, with two
  modifications.  First, whenever $\sigma'$ goes to a "return node"
  $\loc(\vec v)$ instead of a leaf $\loc(\vec v')$---thus $\vec
  v\leq\vec v'$---, the next time Eve has the control, she uses the
  self loops to decrement the current configuration by $\vec v'-\vec
  v$.  This ensures that any play consistent with the modified
  strategy remains between zero and $B-1$ on the components
  in~$\{1,\dots,\dd\}$.  (Note that if she never regains the control,
  the current vector never changes any more since~$\?V$ is
  "asymmetric".)

  Second, whenever a play in~$\game$ visits a location with even
  parity~$2p$ for some~$p$ in~$\{1,\dots,d\}$, Eve has the opportunity
  to increase the coordinates in~$\{\dd+1,\dots,\dd+p\}$ in~$\game_B$.
  She does so and increments until all these components reach~$B-1$.
  This ensures that any play consistent with the modified strategy
  remains between zero and $B-1$ on the components
  in~$\{\dd+1,\dots,\dd+p\}$.  Indeed, $\sigma'$ guarantees that the
  longest sequence of moves before a play visits a location with
  maximal even priority is bounded by $H(\loc_0(\vec v_0))$, thus the
  decrements $-\vec e_{\dd+p}$ introduced in~$\game_B$ by the
  locations from~$\game$ with odd parity~$2p-1$ will never force the
  play to go negative.\todoquestion{Is that clear enough?}
\end{proof}

The bound~$B$ defined in~\cref{11-b} in the previous proof is not
constructive, and possibly much larger than really required.
Nevertheless, one can sometimes show that an explicit~$B$ suffices in
a "bounding game".
A simple example is provided by the "coverability" "asymmetric"
"vector games" with "existential initial credit" arising from
\cref{11-cov2parity}, i.e., where the objective is to reach some
location~$\loc_f$.  Indeed, it is rather straightforward that there
exists a suitable initial credit such that Eve wins the game if and
only if she wins the finite reachability game played over the
underlying directed graph over~$\Loc$ where we ignore the counters.
Thus, for an initial location~$\loc_0$, $B_0=|\Loc|\cdot\|\Act\|+1$
bounds the norm of the necessary initial credit, while a simple path
may visit at most~$|\Loc|$ locations, thus
$B=B_0+|\Loc|\cdot\|\Act\|$ suffices for Eve to win the constructed
"bounding game".

In the general case of "bounding games" with "existential initial
credit", an explicit bound can be established.  The proof goes
along very different lines and is too involved to fit in this chapter,
but we refer the reader
to \cite{Jurdzinski&Lazic&Schmitz:2015,Colcombet&Jurdzinski&Lazic&Schmitz:2017}
for details.
\begin{theorem}\label{11-th-bounding}
  If Eve wins a "bounding game" with "existential initial credit"
  defined by a "vector
  system"~$\?V=(\Loc,\Act,\Loc_\mEve,\Loc_\mAdam,\dd)$, then an
  initial credit $\vec v_0$ with $\|\vec
  v_0\|=(4|\Loc|\cdot\|\Act\|)^{2(\dd+2)^3}$ and a bound
  $B=2(4|\Loc|\cdot\|\Act\|)^{2(\dd+2)^3}+1$ suffice for this.
\end{theorem}

\Cref{11-th-bounding} also yields a way of handling "bounding games"
with "given initial credit".  
\TODO{Last missing bit regarding complexity upper bounds.}
  
% Local IspellDict: british
