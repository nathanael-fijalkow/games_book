
The results on "configuration reachability" might give the impression
that "asymmetry" does not help much for solving "vector games": we
obtained in \cref{11-subsec:reach} exactly the same results as in the
general case.  Thankfully, the situation changes drastically if we
consider the other types of "vector games": "coverability",
"non-termination", and "parity@parity vector games" become decidable
in "asymmetric vector games".  The main rationale for this comes from
order theory, which prompts the following definitions.

\paragraph{Quasi-orders}\AP A ""quasi-order"" $(X,{\leq})$ is a
set~$X$ together with a reflexive and transitive
relation~${\leq}\subseteq X\times X$.  Two elements $x,y\in X$ are
incomparable if $x\not\leq y$ and $y\not\leq x$, and they are
equivalent if $x\leq y$ and $y\leq x$.  The associated strict relation
$x<y$ holds if $x\leq y$ and $y\not\leq x$.

The ""upward closure"" of a subset $S\subseteq X$ is the set of
elements greater or equal to the elements of S:
${\uparrow}S\eqdef\{x\in X\mid\exists y\in S\mathbin.y\leq x\}$.  A
subset $U\subseteq X$ is ""upwards closed"" if ${\uparrow}U=U$.  When
$S=\{x\}$ is a singleton, we write more simply ${\uparrow}x$ for its
upward closure and call the resulting "upwards closed" set a
""principal filter"".  Dually, the ""downward closure"" of~$S$ is
${\downarrow}S\eqdef\{x\in X\mid\exists y\in S\mathbin.x\leq y\}$, a
""downwards closed"" set is a subset $D\subseteq X$ such that
$D={\downarrow}D$, and ${\downarrow}x$ is called a ""principal
ideal"".  Note that the complement $X\setminus U$ of an upwards closed
set~$U$ is downwards closed and vice versa.


\paragraph{Monotone Games}\AP
Let us consider again the "natural semantics" $\natural(\?V)$ of a
"vector system".  The set of vertices
$V=\Loc\times\+N^\dd\cup\{\sink\}$ is naturally equipped with a
partial ordering: $v\leq v'$ if either $v=v'=\sink$, or $v=\loc(\vec
v)$ and $v'=\loc(\vec v')$ are two configurations that share the same
location and satisfy $\vec v(i)\leq\vec v'(i)$ for all $1\leq
i\leq\dd$, i.e., if $\vec v\leq\vec v'$ for the componentwise
ordering.

Consider a set of colours $C$ and a vertex colouring $\vcol{:}\,V\to C$
of the "natural semantics" $\natural(\?V)$ of a "vector system", which
defines a colouring $\col{:}\,E\to C$ where
$\col(e)\eqdef\vcol(\ing(e))$.  We
say that the "colouring"~$\vcol$ is ""monotonic"" if $C$ is finite and,
for every colour $p\in C$, the set $\vcol^{-1}(p)$ of vertices coloured
by~$p$ is "upwards closed" with respect to ${\leq}$.  Clearly, the
"colourings" of "coverability", "non-termination", and "parity@parity
vector games" "vector games" are "monotonic", whereas those of
"configuration reachability" "vector games" are not.  By extension, we
call a "vector game" \emph{"monotonic"} if its underlying "colouring"
is "monotonic".

\begin{lemma}[Simulation]\label{11-fact-mono}
  In a "monotonic" "asymmetric vector game", if Eve wins from a
  vertex~$v_0$, then she also wins from~$v'_0$ for all $v'_0\geq
  v_0$.
\end{lemma}
\begin{proof}
  It suffices for this to check that, for all $v_1\leq v_2$ in $V$,
  \begin{description}
  \item[(colours)] $\vcol(v_1)=\vcol(v_2)$ since $\vcol$ is "monotonic";
  \item[(zig Eve)] if $v_1,v_2\in V_\mEve$, $a\in\Act$, and
    $\dest(v_1,a)=v'_1\neq\sink$ is defined, then
    $v'_2\eqdef\dest(v_2,a)$ is such that $v'_2\geq v'_1$: indeed,
    $v'_1\neq\sink$ entails that $v_1$ is a configuration
    $\loc(\vec v_1)$ and $v'_1=\loc'(\vec v_1+\vec u)$ for the action
    $a=(\loc\step{\vec u}\loc')\in\Act$, but then $v_2=\loc(\vec v_2)$
    for some $\vec v_2\geq\vec v_1$ and
    $v'_2=\loc'(\vec v_2+\vec u)\geq v'_1$;
  \item[(zig Adam)] if $v_1,v_2\in V_\mAdam$, $a\in\Act$, and
    $\dest(v_2,a)=v'_2$ is defined, then
    $v'_1\eqdef\dest(v_1,a)\leq v'_2$: indeed, either $v'_2=\sink$ and
    then $v'_1=\sink$, or $v'_2\neq\sink$, thus
    $v_2=\loc(\vec v_2)$, $v'_2=\loc'(\vec v_2)$, and
    $a=(\loc\step{\vec 0}\loc')\in\Act$ (recall that the game is
    "asymmetric"), but then $v_1=\loc(\vec v_1)$ for some
    $\vec v_1\leq\vec v_2$ and thus $v'_1=\loc'(\vec v_1)\leq v'_2$.
  \end{description}
  The above conditions show that, if $\sigma{:}\,E^\ast\to\Act$ is a
  strategy of Eve that wins from~$v_0$, then by
  ""simulating""~$\sigma$ starting from~$v'_0$---i.e., by applying the
  same actions when given a pointwise larger or equal history---she
  will also win.\todoquestion{Is that clear?}
\end{proof}

Note that \cref{11-fact-mono} implies that $\WE$ is "upwards closed":
$v_0\in\WE$ and $v_0\leq v'_0$ imply $v_0'\in\WE$.  \Cref{11-fact-mono}
does not necessarily hold in "vector games" without the "asymmetry"
condition.  For instance, in both \cref{11-fig-cov,12-fig-nonterm} on
\cpageref{11-fig-cov}, $\loc'(0,1)\in\WE$ but $\loc'(1,2)\in\WA$ for
the "coverability" and "non-termination" objectives.  This is due to
the fact that the action $\loc'\step{-1,0}\loc$ is available
in~$\loc'(1,2)$ but not in~$\loc'(0,1)$.


\paragraph{Well-quasi-orders}\AP What makes "monotonic" "vector games" so
interesting is that the partial order $(V,{\leq})$ associated with the
"natural semantics" of a "vector system" is a ""well-quasi-order"".  A
"quasi-order" $(X,{\leq})$ is "well@well-quasi-order" (a \emph{"wqo"})
if any of the following equivalent characterisations
hold~\cite{Kruskal:1972,Schmitz&Schnoebelen:2012}:
\begin{itemize}
  % \item\AP in any infinite sequence $x_0,x_1,\dots$ of elements of~$X$,
  %   there exist indices $i<j$ such that $x_i\leq x_j$---infinite sequences in $X$ are ""good""---,
  \item\AP in any infinite sequence $x_0,x_1,\cdots$ of elements
    of~$X$, there exists an infinite sequence of indices
    $n_0<n_1<\cdots$ such that $x_{n_0}\leq
    x_{n_1}\leq\cdots$---infinite sequences in $X$ are ""good""---,
  \item\AP any strictly ascending sequence $U_0\subsetneq
    U_1\subsetneq\cdots$ of "upwards closed" sets $U_i\subseteq X$ is
    finite---$X$ has the ""ascending chain condition""---,
  \item\AP any non-empty "upwards closed" $U\subseteq X$ has at least
    one, and at most finitely many minimal elements up to equivalence;
    therefore any "upwards closed" $U\subseteq X$ is a finite union
    $U=\bigcup_{1\leq j\leq n}{\uparrow}x_j$ of finitely many
    "principal filters"~${\uparrow}x_j$---$X$ has the ""finite basis
    property"".
\end{itemize}

The fact that $(V,{\leq})$ satisfies all of the above is an easy
consequence of \emph{Dickson's Lemma}~\cite{Dickson:1913}.

% In a "monotonic" "vector game", by the "finite basis property", each
% set $\col^{-1}(d)$ for a colour $d\in C$ has finitely many minimal
% elements $\min\col^{-1}(d)$, thus there exists a ""viable"" natural
% number
% $B\geq \max_{d\in C}\max_{\vec v\in\min\col^{-1}(d)}\|\vec v\|$.  In
% \cref{11-pb-cov}, $\|\vec v\|$ is "viable" if $\loc(\vec v)$ is the
% target configuration, while for \cref{11-pb-nonterm,12-pb-parity},
% $0$~is "viable".

\paragraph{Pareto Limits}\AP By the "finite basis property" of
$(V,{\leq})$ and \cref{11-fact-mono}, in a "monotonic" "asymmetric
vector game", $\WE=\bigcup_{1\leq j\leq n}{\uparrow}\loc_j(\vec v_j)$
is a finite union of "principal filters".  The set
$\mathsf{Pareto}\eqdef\{\loc_1(\vec v_1),\dots,\loc_n(\vec v_n)\}$ is
called the ""Pareto limit"" or \emph{Pareto frontier} of the game.
Both the "existential" and the "given initial credit" variants of the
game can be reduced to computing this "Pareto limit": with
"existential initial credit" and an initial location $\loc_0$, check
whether $\loc_0(\vec v)\in\mathsf{Pareto}$ for some $\vec v$, and with
"given initial credit" and an initial configuration $\loc_0(\vec v_0)$, check
whether $\loc_0(\vec v)\in\mathsf{Pareto}$ for some $\vec v\leq\vec
v_0$.
\begin{example}
  Consider the "asymmetric vector system" from \cref{11-fig-avg} on
  \cpageref{11-fig-avg}.  For the "coverability game" with target
  configuration $\loc(2,2)$, the "Pareto limit" is
  $\mathsf{Pareto}=\{\loc(2,2),\loc'(3,2),\loc_{2,1}(0,1),\loc_{\text-1,0}(3,2)\}$,
  while for the "non-termination game", $\mathsf{Pareto}=\emptyset$:
  Eve loses from all the vertices.  Observe that this is consistent
  with Eve\'s "winning region" in the "coverability" "energy game"
  shown in \cref{11-fig-cov-nrg}.
\end{example}

% \paragraph{Pareto Bounds}
% \AP Consider a "monotonic" "asymmetric" "vector game"
% $\?G=(\natural(\?V),\col,\Omega)$ with "Pareto limit"
% $\mathsf{Pareto}=\{\loc_1(\vec v_1),\dots,\loc_n(\vec v_n)\}$.  Since
% this "Pareto limit" is finite, there exists a number
% $\mathsf{ParetoBound}\eqdef\max_{\loc(\vec v)\in\mathsf{Pareto}}\|\vec
% v\|$ bounding the necessary "given initial credit" for Eve to win the
% game from any initial location.  Another bound of interest is
% $\mathsf{ExistentialParetoBound}\eqdef\max_{\loc\in\Loc}\min_{\loc(\vec
% v)\in\mathsf{Pareto}}\|\vec v\|$, which bounds the "existential
% initial credit" necessary for Eve to win the game from any initial
% location.  Note that
% $\mathsf{ParetoBound}\geq \mathsf{ExistentialParetoBound}$ and that
% the two values always coincide in dimension $\dd=1$.  We call a number
% $B\geq\mathsf{ParetoBound}$ a ""Pareto bound"" and a number
% $B\geq\mathsf{ExistentialParetoBound}$ an ""existential Pareto
% bound"".

\begin{example}\label{11-ex-pareto}
  Consider the one-player "vector system" of \cref{11-fig-pareto},
  where the "meta-decrement" from~$\loc_0$ to~$\loc_1$ can be
  implemented using $O(n)$ additional counters and a set~$\Loc'$ of
  $O(n)$ additional locations by the arguments of the
  forthcoming \cref{11-avag-hard}.
  
  \begin{figure}[htbp]
    \centering
  \begin{tikzpicture}[auto,on grid,node distance=2.5cm]
    \node[s-eve](0){$\loc_0$};
    \node[s-eve,right=of 0](1){$\loc_1$};
    \node[s-eve,below right=1.5 and 1.25 of 0](2){$\loc_f$};
    \path[arrow,every node/.style={font=\footnotesize,inner sep=1}]
    (0) edge node {$-2^{2^n}\cdot\vec e_1$} (1)
    (0) edge[bend right=10,swap] node {$-\vec e_2$} (2)
    (1) edge[bend left=10] node {$\vec 0$} (2);
  \end{tikzpicture}
  \caption{\label{11-fig-pareto} A one-player "vector system"
  with a large "Pareto limit".}
  \end{figure}
  For the "coverability game" with target
  configuration~$\loc_f(\vec 0)$, if $\loc_0$ is the initial location
  and we are "given initial credit" $m\cdot\vec e_1$, Eve wins if and
  only if $m\geq 2^{2^n}$, but with "existential initial credit" she
  can start from $\loc_0(\vec e_2)$ instead.  We have indeed
  $\mathsf{Pareto}\cap(\{\loc_0,\loc_1,\loc_f\}\times\+N^\dd)=\{\loc_0(\vec
  e_2),\loc_0(2^{2^n}\cdot\vec e_1),\loc_1(\vec 0),\loc_f(\vec 0)\}$.
  Looking more in-depth into the construction of \cref{11-avag-hard},
  there is also an at least double exponential number of distinct
  minimal configurations in~$\mathsf{Pareto}$.
\end{example}



\paragraph{Finite Memory} 
Besides having a finitely represented "winning region", Eve also has
finite memory strategies in "asymmetric vector games" with "parity"
objectives; the following argument is straightforward to adapt to
the other regular objectives from \cref{chap:regular}.
\begin{lemma}\label{11-fact-finmem}
  If Eve has a "strategy" winning from some vertex~$v_0$ in a
  "parity@parity vector game" "asymmetric vector game", then she has a
  "finite-memory" one.
\end{lemma}
\begin{proof}
  Assume~$\sigma$ is a winning strategy from~$v_0$.  Consider the tree
  of vertices visited by plays consistent with~$\sigma$: each branch
  is an infinite sequence $v_0,v_1,\dots$ of elements of~$V$ where the
  maximal priority occuring infinitely often is some even number~$p$.
  Since $(V,{\leq})$ is a "wqo", this is a "good sequence": there
  exists infinitely many indices $n_0<n_1<\cdots$ such that
  $v_{n_0}\leq v_{n_1}\leq\cdots$.  There exists $i<j$ such
  that~$p=\max_{n_i\leq n<n_j}\vcol(v_n)$ is the maximal priority
  occurring in some interval $v_{n_i},v_{n_{i+1}},\dots,v_{n_{j-1}}$.
  Then Eve can play in~$v_{n_j}$ as if she were in~$v_{n_i}$, in
  $v_{n_j+1}$ as if she were in $v_{n_i+1}$ and so on, and we prune
  the tree at index~$n_j$ along this branch so that $v_{n_j}$ is a
  leaf, and we call~$v_{n_i}$ the ""return node"" of that leaf.  We
  therefore obtain a finitely branching tree with finite branches,
  which by K\H{o}nig's Lemma is finite.

  The finite tree we obtain this way is sometimes called a
  ""self-covering tree"".  %   It is labelled by configurations
  % in~$\Loc\times\+N^\dd$ and
  % \begin{enumerate}
  % \item its root label is~$\loc_0(\vec v_0)$,
  % \item if an internal node is labelled by $\loc(\vec v)$ with
  %   $\loc\in\Loc_\mEve$, then it has exactly one child labelled by
  %   some~$\loc'(\vec v+\vec u)$,
  % \item if an internal node is labelled by $\loc(\vec v)$ with
  %   $\loc\in\Loc_\mAdam$, then it has one child labelled~$\loc'(\vec
  %   v)$ for each action $\loc\step{\vec 0}\loc'\in\Act$,
  % \item\label{11-wt-self-even} if a leaf is labelled by $\loc(\vec v)$,
  %   consider the branch
  %   $\loc_0(\vec v_0),\loc_1(\vec v_1),\dots,\loc_n(\vec v_n)=\loc(\vec
  %   v)$ that reaches the leaf: then
  %   \begin{enumerate}
  %   \item\label{11-wt-self}there exists~$i<n$ such that
  %     $\loc_i=\loc$ and $\vec v_i\leq\vec v$---we call the node labelled by
  %     $\loc_i(\vec v_i)$ the ""return point"" of the leaf---, and
  %   \item\label{11-wt-even} the maximal priority observed between the two
  %     nodes, i.e., $\max_{i\leq j<n}\col(v_j)$, is even.
  %   \end{enumerate}
  % \end{enumerate}
  It is relatively straightforward to construct a finite "memory
  structure"~$(M,m_0,\delta)$ (as defined in \cref{1-finite-memory}) from a
  "self-covering tree", using its internal nodes as memory states plus
  an additional sink memory state~$m_\bot$; the initial memory
  state~$m_0$ is the root of the tree.  In a node~$m$ labelled by
  $\loc(\vec v)$, given an edge
  $e=(\loc(\vec v'),\loc'(\vec v'+\vec u))$ arising from an
  action~$\loc\step{\vec u}\loc'\in\Act$, if $\vec v'\geq\vec v$ and
  $m$~has a child~$m'$ labelled by $\loc'(\vec v+\vec u)$ in the
  "self-covering tree", then either~$m'$ is a leaf with "return
  node"~$m''$ and we set $\delta(m,e)\eqdef m''$, or $m'$~is an
  internal node and we set $\delta(m,e)\eqdef m'$; in all the other
  cases, $\delta(m,e)\eqdef m_\bot$. % The strategy
  % $\sigma'{:}\,V\times M\to\Act$ associated with the memory structure
  % picks an arbitrary action except if $v=\loc(\vec v')$ is a
  % configuration with $\loc\in\Loc_\mEve$ and $m$~is an internal node
  % of the "self-covering tree" labelled by $\loc(\vec v)$ for some
  % $\vec v\leq\vec v'$, in which case~$m$ has a single child labelled
  % by $\loc'(\vec v+\vec u)$ in the "self-covering tree", corresponding
  % to an action $a=(\loc\step{\vec u}\loc')\in\Act$, and we set
  % $\sigma'(v,m)\eqdef a$.
  \todoquestion{Is that reasonably clear?  A
    bit heavy no?}
\end{proof}

\TODO{Provide an additional example of a "self-covering tree".}

\begin{example}
  Consider the one-player "vector system" of \cref{11-fig-finitemem},
  where the "meta-decrement" from~$\loc_1$ to~$\loc_0$ can be
  implemented using $O(n)$ additional counters and $O(n)$ additional
  locations by the arguments of the forthcoming \cref{11-avag-hard}
  on \cpageref{11-avag-hard}.
  
  \begin{figure}[htbp]
    \centering
  \begin{tikzpicture}[auto,on grid,node distance=2.5cm]
    \node[s-eve](0){$\loc_0$};
    \node[s-eve,right=of 0](1){$\loc_1$};
    \node[black!50,above=.5 of 0,font=\scriptsize]{$2$};
    \node[black!50,above=.5 of 1,font=\scriptsize]{$1$};
    \path[arrow,every node/.style={font=\footnotesize,inner sep=2}]
    (1) edge[bend left=15] node {$-2^{2^n}\cdot\vec e_1$} (0)
    (0) edge[bend left=15] node {$\vec 0$} (1)
    (1) edge[loop right] node{$\vec e_1$} ();
  \end{tikzpicture}
  \caption{\label{11-fig-finitemem} A one-player "vector system"
  witnessing the need for double exponential memory.}
  \end{figure}

  For the "parity@parity vector game" game with location colouring
  $\lcol(\loc_0)\eqdef 2$ and $\lcol(\loc_1)\eqdef 1$, note that Eve
  must visit $\loc_0$ infinitely often in order to fulfil the parity
  requirements.  Starting from the initial
  configuration~$\loc_0(\vec 0)$, any winning play of Eve begins
  by \begin{equation*} \loc_0(\vec 0)\step{0}\loc_1(\vec 0)\step{\vec
      e_1}\loc_1(\vec e_1)\step{\vec e_1}\cdots\step{\vec
      e_1}\loc_1(m\cdot\vec
    e_1)\mstep{-2^{2^n}}\loc_0((m-2^{2^n})\cdot\vec
    e_1) \end{equation*} for some~$m\geq 2^{2^n}$ before she visits
  again a
  configuration---namely~$\loc_0((m-2^{2^n})\cdot\vec e_1)$---greater
  or equal than a previous configuration---namely
  $\loc_0(\vec 0)$---\emph{and} witnesses a maximal even parity in the
  meantime.  She then has a winning strategy that simply repeats this
  sequence of actions, allowing her to visit successively
  $\loc_0(2(m-2^{2^n})\cdot\vec e_1)$,
  $\loc_0(3(m-2^{2^n})\cdot\vec e_1)$, etc.  In this example, she
  needs at least $2^{2^n}$ memory to remember how many times the
  $\loc_1\step{\vec e_1}\loc_1$ loop should be taken.
\end{example}

\subsubsection{Attractor Computation for Coverability}
\label{11-subsec:attr}
So far, we have not seen how to compute the "Pareto limit" derived
from \cref{11-fact-mono} nor the finite "memory structure" derived
from \cref{11-fact-finmem}.  These objects are not merely finite but
also computable.  The simplest case is the one of "coverability"
"asymmetric" "monotonic vector games": the fixed-point computation of
\cref{chap:regular} for "reachability" objectives can be turned into
an algorithm computing the "Pareto limit" of the game.

\begin{fact}\label{11-pareto-cov}
  The "Pareto limit" of a "coverability" "asymmetric vector game" is
  computable.
\end{fact}
\begin{proof}
Let $\loc_f(\vec v_f)$ be the target configuration.  We define a
chain $U_0\subseteq U_1\subseteq\cdots$ of sets $U_i\subseteq V$ by
\begin{align*}
  U_0&\eqdef{\uparrow}\loc_f(\vec v_f)\;,&
  U_{i+1}&\eqdef U_i\cup\mathrm{Pre}(U_i)\;.
\end{align*}
Observe that for all~$i$, $U_i$ is "upwards closed".  This can be
checked by induction over~$i$: it holds initially in~$U_0$, and for
the induction step, if $v\in U_{i+1}$ and $v'\geq v$, then either
\begin{itemize}
\item $v=\loc(\vec v)\in\mathrm{Pre}(U_i)\cap\VE$ thanks to some
  $\loc\step{\vec u}\loc'\in\Act$ such that
  $\loc'(\vec v+\vec u)\in U_i$; therefore $v'=\loc(\vec v')$ for some
  $\vec v'\geq \vec v$ is such that $\loc'(\vec v'+\vec u)\in U_i$ as
  well, thus $v'\in \mathrm{Pre}(U_i)\subseteq U_{i+1}$, or
\item $v=\loc(\vec v)\in\mathrm{Pre}(U_i)\cap\VA$ because for all
  $\loc\step{\vec 0}\loc'\in\Act$, $\loc'(\vec v)\in U_i$; therefore
  $v'=\loc(\vec v')$ for some $\vec v'\geq \vec v$ is such that
  $\loc'(\vec v')\in U_i$ as well, thus
  $v'\in \mathrm{Pre}(U_i)\subseteq U_{i+1}$, or
\item $v\in U_i$ and therefore $v'\in U_i\subseteq U_{i+1}$.
\end{itemize}

By the "ascending chain condition", there is a finite rank~$i$ such
that $U_{i+1}\subseteq U_i$ and then $\WE=U_i$.  Thus the
"Pareto limit" is obtained after finitely many steps.
In order to turn this idea into an algorithm, we need a way of
representing those infinite "upwards closed" sets $U_i$.  Thankfully,
by the "finite basis property", each $U_i$ has a finite basis $B_i$
such that ${\uparrow}B_i=U_i$.  We therefore compute the following
sequence of sets
\begin{align*}
  B_0&\eqdef\{\loc_f(\vec v_f)\}&B_{i+1}&\eqdef
                                       B_i\cup\min\mathrm{Pre}({\uparrow}B_i)\;.
\end{align*}
Indeed, given a finite basis~$B_i$ for~$U_i$, it is straightforward to
compute a finite basis for the "upwards closed" $\mathrm{Pre}(U_i)$.
This results in \cref{11-algo:cov} below.
\end{proof}

\begin{algorithm}
 \KwData{A "vector system" and a target configuration $\loc_f(\vec v_f)$}

$B_0 \leftarrow \{\loc_f(\vec v_f)\}$ ;

$i \leftarrow 0$ ;
     
\Repeat{${\uparrow}B_i \supseteq B_{i+1}$}{

$B_{i+1} \leftarrow B_i \cup \min\mathrm{Pre}({\uparrow}B_i)$ ;

$i \leftarrow i + 1$ ;}

\Return{$\min B_i = \mathsf{Pareto}(\game)$}
\caption{Fixed point algorithm for "coverability" "asymmetric" "vector
  games".}
\label{11-algo:cov}
\end{algorithm}

While this algorithm terminates thanks to the "ascending chain
condition", it may take quite long.  For instance, in
\cref{11-ex-pareto}, it requires at least~$2^{2^n}$ steps before it
reaches its fixed point.  This is a worst-case instance, as it turns
out that this algorithm works in \kEXP[2]; see the bibliographic notes
at the end of the chapter.  Note that such a
fixed-point computation does not work directly for "non-termination"
or "parity vector games", due to the need for greatest fixed-points.

% Local IspellDict: british
