\subsubsection{Existential Initial Credit}
In the "existential initial credit" variant of our games, we have the
following lower bound matching \cref{11-exist-easy}, already with a
unary encoding.

\begin{theorem}\label{11-exist-hard}
  "Non-termination", and "parity@parity vector game"
  "asymmetric" "vector games" with "existential initial credit" are
  \coNP-hard.% in any dimension~$\dd\geq 2$.
\end{theorem}
\begin{proof}
  By \cref{11-nonterm2parity}, it suffices to show hardness for
  "non-termination games".  We reduce from the \lang{3SAT} problem:
  given a formula $\varphi=\bigwedge_{1\leq i\leq m}C_i$ where each
  clause $C_i$ is a disjonction of the form
  $\litt_{i,1}\vee\litt_{i,2}\vee\litt_{i,3}$ of literals taken from
  $X=\{x_1,\neg x_1,x_2,$ $\neg x_2,\dots,x_k,\neg x_k\}$, we construct
  an "asymmetric" "vector system" $\?V$ where \Eve wins the
  "non-termination game" with "existential initial credit" if and only
  if~$\varphi$ is not satisfiable; since the game is determined, we
  actually show that \Adam wins the game if and only if~$\varphi$ is
  satisfiable.

  Our "vector system" has dimension~$2k$, and for a literal
  $\litt\in X$, we define the vector
  \begin{equation*}
    \vec u_\litt\eqdef\begin{cases}
      \vec e_{2n-1}-\vec e_{2n}&\text{if }\litt=x_n\;,\\
      \vec e_{2n}-\vec e_{2n-1}&\text{if }\litt=\neg x_n\;.
    \end{cases}
  \end{equation*}
  We define $\?V\eqdef(\Loc,\Act,\Loc_\mEve,\Loc_\mAdam,2k)$ where
  \begin{align*}
    \Loc_\mEve&\eqdef\{\varphi\}\cup\{\litt_{i,j}\mid 1\leq i\leq m,1\leq j\leq
                3\}\;,\\
    \Loc_\mAdam&\eqdef\{C_i\mid 1\leq i\leq m\}\;,\\
    \Act&\eqdef\{\varphi\step{\vec 0}C_i\mid 1\leq i\leq m\}\cup\{C_i\step{\vec 0}\litt_{i,j},\;\;\litt_{i,j}\xrightarrow{\vec u_{\litt_{i,j}}}\varphi\mid 1\leq i\leq m,1\leq j\leq 3\}\;.
  \end{align*}
  \begin{scope}
    We use~$\varphi$ as our initial location.
    \knowledge{literal assignment}{notion}
    \knowledge{conflicting}{notion}
    %
    Let us call a map $v{:}\,X\to\{0,1\}$ a ""literal assignment""; we
    call it ""conflicting"" if there exists $1\leq n\leq k$ such that
    $v(x_n)=v(\neg x_n)$.

    Assume that~$\varphi$ is satisfiable.  Then there exists a
    non-"conflicting" "literal assignment"~$v$ that satisfies all the
    clauses: for each $1\leq i\leq m$, there exists $1\leq j\leq 3$
    such that $v(\litt_{i,j})=1$; this yields a "counterless" strategy
    for \Adam, which selects $(C_i,\litt_{i,j})$ for each
    $1\leq i\leq m$.  Consider any infinite "play" consistent with
    this strategy.  This "play" only visits literals $\litt$ where
    $v(\litt)=1$.  There exists a literal $\litt\in X$ that is visited
    infinitely often along the "play", say $\litt=x_n$.  Because~$v$ is
    non-"conflicting", $v(\neg x_n)=0$, thus the location $\neg x_n$
    is never visited.  Thus the play uses the action
    $\litt\step{\vec e_{2n-1}-\vec e_{2n}}\varphi$ infinitely often,
    and never uses any action with a positive effect on
    component~$2n$.  Hence the play is losing from any initial credit.

    Conversely, assume that~$\varphi$ is not satisfiable.  By
    contradiction, assume that \Adam wins the game for all initial
    credits.  By \cref{11-counterless}, he has a "counterless" winning
    strategy~$\tau$ that selects a literal in every clause.  Consider
    a "literal assignment" that maps each one of the selected literals
    to~$1$ and the remaining ones in a non-conflicting manner.  By
    definition, this "literal assignment" satisfies all the clauses,
    but because~$\varphi$ is not satisfiable, it is "conflicting":
    necessarily, there exist $1\leq n\leq k$ and $1\leq i,i'\leq m$,
    such that $\tau$ selects $x_n$ in $C_i$ and $\neg x_n$ in
    $C_{i'}$.  But this yields a winning strategy for \Eve, which
    alternates in the initial location $\varphi$ between $C_{i}$
    and $C_{i'}$, and for which an initial credit
    $\vec e_{2n-1}+\vec e_{2n}$ suffices: a contradiction.
  \end{scope}
  % We reduce from the \lang{Partition} problem: given a finite set
  % $S=\{w_0,\dots,w_{n-1}\}\subseteq\+N$ (encoded in binary), does
  % there exist a partition $S_1,S_2$ of~$S$ such that
  % $\sum_{w\in S_1}w=\sum_{w\in S_2}w$?  From such an instance, we
  % construct an "asymmetric vector system" where \Eve wins the
  % "non-termination game" with "existential initial credit" if and only
  % if there is no such partition.  By \cref{11-nonterm2parity}, this
  % also entails the \coNP-hardness of "parity@parity vector games"
  % "asymmetric vector games".
  % \begin{figure}[htbp]
  % \centering
  % \begin{tikzpicture}[auto,on grid,node distance=1.8cm]
  %   \node[adam,inner sep=2.1](0){$\loc_0$};
  %   \node[eve,above right=1 and 1 of 0,inner sep=2.2](01){$\loc_{0,1}$};
  %   \node[eve,below right=1 and 1 of 0,inner sep=2.2](02){$\loc_{0,2}$};
  %   \node[adam,below right=1 and 2 of 01,inner sep=2.1](1){$\loc_1$};
  %   \node[eve,above right=1 and 1 of 1,inner sep=2.2](11){$\loc_{1,1}$};
  %   \node[eve,below right=1 and 1 of 1,inner sep=2.2](12){$\loc_{1,2}$};
  %   \node[adam,below right=1 and 2 of 11,inner sep=2.1](2){$\loc_2$};
  %   \node[right=1 of 2](dots){$\Huge\cdots$};
  %   \node[adam,right=1 of dots,inner sep=-1.8](nm){$\loc_{n-1}$};
  %   \node[eve,above right=1 and 1 of nm,inner sep=0](nm1){$\loc_{n-1,1}$};
  %   \node[eve,below right=1 and 1 of nm,inner sep=0](nm2){$\loc_{n-1,2}$};
  %   \node[eve,below right=1 and 2 of nm1,inner sep=2.2](n){$\loc_n$};    
  %   \path[->,every node/.style={font=\footnotesize,inner sep=.1}]
  %   (0) edge node {$\vec 0$} (01)
  %   (0) edge[swap] node {$\vec 0$} (02)
  %   (01) edge[near start] node {$\!-w_0\cdot\vec e_1$} (1)
  %   (02) edge[swap,near start] node {$\!-w_0\cdot\vec e_2$} (1)
  %   (1) edge node {$\vec 0$} (11)
  %   (1) edge[swap] node {$\vec 0$} (12)
  %   (11) edge[near start] node {$\!-w_1\cdot\vec e_1$} (2)
  %   (12) edge[swap,near start] node {$\!-w_1\cdot\vec e_2$} (2)
  %   (nm) edge node {$\vec 0$} (nm1)
  %   (nm) edge[swap] node {$\vec 0$} (nm2)
  %   (nm1) edge[near start] node {$\!\!-w_{n-1}\cdot\vec e_1$} (n)
  %   (nm2) edge[swap,near start] node {$\!\!-w_{n-1}\cdot\vec e_2$} (n);
  %   \draw[->,rounded corners=20pt,>=stealth',shorten >=1pt] (n) -- (10.9,1.6) -- (0.1,1.6) -- (0);
  %   \draw[->,rounded corners=20pt,>=stealth',shorten >=1pt] (n) -- (10.9,-1.6) -- (0.1,-1.6) --
  %   (0);
  %   \node[font=\footnotesize] at (5.5,1.8) {$((W/2)-1)\cdot\vec e_1$};
  %   \node[font=\footnotesize] at (5.5,-1.85) {$((W/2)-1)\cdot\vec e_2$};
  % \end{tikzpicture}
  % \caption{\label{11-fig-part} The "vector system" built from a
  %   \lang{Partition} instance.}
  % \end{figure}
  % We may assume that~$\sum_{w\in S}w$ is even, as otherwise trivially
  % no partition exists; let $W\eqdef(\sum_{w\in S}w)/2$.  We define
  % $\?V\eqdef(\Loc,\Act,\Loc_\mEve,\Loc_\mAdam,2)$ where
  % \begin{align*}
  %   \Loc_\mEve&\eqdef\{\loc_{i,j}\mid 0\leq i<n,1\leq 2\leq
  %               j\}\cup\{\loc_n\}\;,\\
  %   \Loc_\mAdam&\eqdef\{\loc_i\mid 0\leq i<n\}\;,\\
  %   \Act&\eqdef\{\loc_i\step{\vec 0}\loc_{i,j}\mid 0\leq i<n,1\leq 2\leq
  %               j\}\\
  %   &\:\cup\:\{\loc_{i,j}\step{-w_i\cdot\vec e_j}\loc_{i+1}\mid 0\leq i<n,1\leq 2\leq
  %               j\}\\
  %   &\:\cup\:\{\loc_n\step{(W-1)\cdot\vec e_j+W\cdot\vec e_{3-j}}\loc_0\mid 1\leq 2\leq
  %               j\}\;;
  % \end{align*}
  % see \cref{11-fig-part} for a depiction of~$\?V$.  We use~$\loc_0$ as
  % our initial location.
  % Let us show that \Adam wins if and only if the \lang{Partition}
  % instance was positive.  If there exist $S_1,S_2\subseteq S$ with
  % $S_1\cap S_2$ and $\sum_{w\in S_1}w=\sum_{w\in S_2}w=W/2$, then
  % \Adam has a winning "counterless" strategy where, in
  % location~$\loc_i$ for $0\leq i<n$, he goes to $\loc_{i,1}$ if and
  % only if $w_i\in S_1$.  Then, if the game begins in
  % $\loc_0(c_1,c_2)$, it reaches $\loc_n(c_1-W/2,c_2-W/2)$ after
  % visiting each $\loc_i$~once, and 
\end{proof}

Note that \cref{11-exist-hard} does not apply to fixed dimensions
$\dd\geq 2$.  We know by \cref{11-exist-pseudop} that those games can
be solved in pseudo-polynomial time if the number of priorities is
fixed, and by \cref{11-exist-easy} that they are in \coNP.

\subsubsection{Given Initial Credit}
With "given initial credit", we have a lower bound matching the
\kEXP[2] upper bound of \cref{11-avag-easy}, already with a unary
encoding.  The proof itself is an adaptation of the proof by
\citem[Lipton]{Lipton:1976} of \EXPSPACE-hardness of "coverability" in
the one-player case.

\begin{theorem}\label{11-avag-hard}
  "Coverability", "non-termination", and "parity@parity vector game"
  "asymmetric" "vector games" with "given initial credit" are
  \kEXP[2]-hard.
\end{theorem}
\begin{proof}
  We reduce from the "halting problem" of an ""alternating Minsky
  machine"" $\?M=(\Loc,\Act,\Loc_\mEve,\Loc_\mAdam,\dd)$ with counters
  bounded by $B\eqdef 2^{2^n}$ for $n\eqdef|\?M|$.  Such a machine is
  similar to an "asymmetric" "vector system" with increments
  $\loc\step{\vec e_i}\loc'$, decrements $\loc\step{-\vec e_i}\loc'$,
  and "zero test" actions $\loc\step{i\eqby?0}\loc'$, all
  restricted to locations $\loc\in\Loc_\mEve$; the only actions
  available to \Adam are actions $\loc\step{\vec 0}\loc'$.  The
  set of locations contains a distinguished `halt' location
  $\loc_\mathtt{halt}\in\Loc$ with no outgoing action.  The
  machine comes with the promise that, along any "play", the norm of
  all the visited configurations $\loc(\vec v)$ satisfies
  $\|\vec v\|<B$.  The "halting problem" asks, given an initial
  location $\loc_0\in\Loc$, whether \Eve has a winning strategy to
  visit $\loc_\mathtt{halt}(\vec v)$ for some $\vec v\in\+N^\dd$ from
  the initial configuration $\loc_0(\vec 0)$.  This problem is
  \kEXP[2]-complete if $\dd\geq 3$ by standard
  arguments~\cite{Fischer&Meyer&Rosenberg:1968}.

  %%%%%\begin{scope}
    \knowledge{meta-increment}[meta-increments]{notion}
    \knowledge{meta-decrement}[meta-decrements]{notion} Let us start
    by a quick refresher on Lipton's construction~\cite{Lipton:1976};
    see also~\cite{Esparza:1996} for a nice exposition.  At the heart
    of the construction lies a collection of one-player gadgets
    implementing \emph{level~$j$} ""meta-increments""
    $\loc\mstep{2^{2^j}\cdot\vec c}\loc'$ and \emph{level~$j$}
    ""meta-decrements"" $\loc\mstep{-2^{2^j}\cdot\vec c}\loc'$ for
    some "unit vector"~$\vec c$ using $O(j)$ auxiliary counters and
    $\poly(j)$ actions, with precondition that the auxiliary counters
    are initially empty in~$\loc$ and postrelation that they are empty
    again in~$\loc'$.  The construction is by induction over~$j$; let
    us first see a naive implementation for "meta-increments".  For
    the base case~$j=0$, this is just a standard action
    $\loc\step{2\vec c}\loc'$.  For the induction step $j+1$, we use
    the gadget of \cref{11-fig-meta-incr} below, where
    $\vec x_{j},\bar{\vec x}_{j},\vec z_{j},\bar{\vec z}_{j}$ are
    distinct fresh "unit vectors": the gadget performs two nested
    loops, each of $2^{2^j}$ iterations, thus iterates the unit
    increment of~$\vec c$ a total of $\big(2^{2^j}\big)^2=2^{2^{j+1}}$
    times.  A "meta-decrement" is obtained similarly.

    \begin{figure}[htbp]
      \centering
      \begin{tikzpicture}[auto,on grid,node distance=1.55cm]
      \node[s-eve](0){$\loc$};
      \node[s-eve-small,right=of 0](1){};
      \node[s-eve-small,right=of 1](2){};
      \node[s-eve-small,right=of 2](3){};
      \node[s-eve-small,right=of 3](4){};
      \node[s-eve-small,right=of 4](5){};
      \node[s-eve-small,right=of 5](6){};
      \node[s-eve,right=of 6](7){$\loc'$};
      \path[arrow,every node/.style={font=\footnotesize,inner sep=2pt}]
      (0) edge node{$2^{2^j}\cdot\vec x_{j}$} (1)
      (1) edge node{$2^{2^j}\cdot\vec z_{j}$} (2)
      (2) edge node{$\bar{\vec x}_{j}-\vec x_{j}$} (3)
      (3) edge node{$\bar{\vec z}_{j}-\vec z_{j}$} (4)
      (4) edge node{$\vec c$} (5)
      (5) edge node{$-2^{2^j}\cdot\bar{\vec z}_{j}$} (6)
      (6) edge node{$-2^{2^j}\cdot\bar{\vec x}_{j}$} (7);
      \draw[->,rounded corners=10pt,>=stealth'] (5) --
      (7.4,.65) -- (5,.65) -- (3);
      \node[font=\footnotesize,inner sep=2pt] at (6.2,.75) {$\vec 0$};
      \draw[->,rounded corners=10pt,>=stealth'] (6) --
      (8.95,1.25) -- (1.9,1.25) -- (1);
      \node[font=\footnotesize,inner sep=2pt] at (5.43,1.35) {$\vec 0$};
    \end{tikzpicture}
    \caption{\label{11-fig-meta-incr} A naive implementation of the
      "meta-increment" $\loc\mstep{2^{2^{j+1}}\cdot\vec c}\loc'$.}
  \end{figure}
  
  Note that this level~$(j+1)$ gadget contains two copies of the
  level~$j$ "meta-increment" and two of the level~$j$
  "meta-decrement", hence this naive implementation has
  size~$\mathsf{exp}(j)$.  In order to obtain a polynomial size, we would like
  to use a single \emph{shared} level~$j$ gadget for each~$j$, instead
  of hard-wiring multiple copies.  The idea is to use a `dispatch
  mechanism,' using extra counters, to encode the choice of "unit
  vector"~$\vec c$ and of return location~$\loc'$.  Let us see how to
  do this in the case of the return location~$\loc'$; the mechanism
  for the vector~$\vec c$ is similar.  We enumerate the (finitely many)
  possible return locations~$\loc_0,\dots,\loc_{m-1}$ of the gadget
  implementing $\loc\mstep{2^{2^{j+1}}\cdot\vec c}\loc'$.  We use two
  auxiliary counters with "unit vectors" $\vec r_j$
  and~$\bar{\vec r}_j$ to encode the return location.  Assume $\loc'$
  is the $i$th possible return location, i.e., $\loc'=\loc_i$ in our
  enumeration: before entering the shared gadget implementation, we
  initialise~$\vec r_j$ and~$\bar{\vec r}_j$ by performing the action
  $\loc\step{i\cdot\vec r_j+(m-i)\cdot\bar{\vec r}_j}\cdots$.  Then,
  where we would simply go to~$\loc'$ in \cref{11-fig-meta-incr} at
  the end of the gadget, the shared gadget has a final action
  $\cdots\step{\vec 0}\loc_{\mathrm{return}_j}$ leading to a dispatch
  location for returns: for all $0\leq i<m$, we have an action
  $\loc_{\mathrm{return}_j}\step{-i\cdot\vec r_j-(m-i)\cdot\bar{\vec
      r}_j}\loc_i$
  that leads to the desired return location.\todoquestion{Is that
    clear enough?}
  

  \bigskip Let us return to the proof.  Consider an instance of the
  "halting problem".  We first exhibit a reduction to "coverability";
  by \cref{11-cov2parity}, this will also entail the \kEXP[2]-hardness
  of "parity@parity vector game" "asymmetric" "vector games".  We
  build an "asymmetric vector system"
  $\?V=(\Loc',\Act',\Loc'_\mEve,\Loc_\mAdam,\dd')$ with
  $\dd'=2\dd+O(n)$.  Each of the counters~$\mathtt{c}_i$ of $\?M$ is
  paired with a \emph{complementary} counter~$\bar{\mathtt{c}_i}$ such
  that their sum is~$B$ throughout the simulation of~$\?M$.  We
  denote by $\vec c_i$ and $\bar{\vec c}_i$ the corresponding "unit
  vectors" for $1\leq i\leq\dd$.  The "vector system"~$\?V$ starts by
  initialising the counters $\bar{\mathtt{c}}_i$ to~$B$ by a sequence
  of "meta-increments"
  $\loc'_{i-1}\mstep{2^{2^n}\cdot\bar{\vec c}_i}\loc'_i$ for
  $1\leq i\leq\dd$, before starting the simulation by an action
  $\loc'_\dd\step{\vec 0}\loc_0$.  The simulation of~$\?M$ uses the
  actions depicted in \cref{11-fig-lipton}.  Those maintain the
  invariant on the complement counters.  Regarding "zero tests", \Eve
  yields the control to \Adam, who has a choice between performing a
  "meta-decrement" that will fail if $\bar{\mathtt c}_i< 2^{2^n}$,
  which by the invariant is if and only if $\mathtt{c}_i>0$, or going
  to~$\loc'$.

  \begin{figure}[htbp]
    \centering
    \begin{tikzpicture}[auto,on grid,node distance=1.5cm]
      \node(to){$\mapsto$};
      \node[anchor=east,left=2.5cm of to](mm){"alternating Minsky machine"};
      \node[anchor=west,right=2.5cm of to](mwg){"asymmetric vector system"};
      % increment of Eve
      \node[below=.7cm of to](imap){$\rightsquigarrow$};
      \node[s-eve,left=2.75cm of imap](i0){$\loc$};
      \node[right=of i0](i1){$\loc'$};
      \node[right=1.25cm of imap,s-eve](i2){$\loc$};
      \node[right=1.8 of i2](i3){$\loc'$};      
      \path[arrow,every node/.style={font=\footnotesize}]
      (i0) edge node{$\vec e_i$} (i1)
      (i2) edge node{$\vec c_i-\bar{\vec c}_i$} (i3);
      % decrement of Eve
      \node[below=1cm of imap](dmap){$\rightsquigarrow$};
      \node[s-eve,left=2.75cm of dmap](d0){$\loc$};
      \node[right=of d0](d1){$\loc'$};
      \node[right=1.25cm of dmap,s-eve](d2){$\loc$};
      \node[right=1.8 of d2](d3){$\loc'$};
      \path[arrow,every node/.style={font=\footnotesize}]
      (d0) edge node{$-\vec e_i$} (d1)
      (d2) edge node{$-\vec c_i+\bar{\vec c}_i$} (d3);
      % zero test of Eve
      \node[below=1.5cm of dmap](zmap){$\rightsquigarrow$};
      \node[s-eve,left=2.75cm of zmap](z0){$\loc$};
      \node[right=of z0](z1){$\loc'$};
      \node[right=1.25cm of zmap,s-eve](z2){$\loc$};
      \node[right=of z2,s-adam-small](z3){};
      \node[above right=.8 and 1.1 of z3,s-eve-small](z4){};
      \node[below right=.8 and 1.1 of z3,inner sep=0pt](z5){$\loc'$};
      \node[right=1.8 of z4](z6){$\loc_\mathtt{halt}$};
      \path[arrow,every node/.style={font=\footnotesize}]
      (z0) edge node{$i\eqby?0$} (z1)
      (z2) edge node{$\vec 0$} (z3)
      (z3) edge node{$\vec 0$} (z4)
      (z3) edge[swap] node{$\vec 0$} (z5)
      (z4) edge node{$-2^{2^n}\cdot\bar{\vec c}_i$} (z6);
      % action of Adam
      \node[below=1.5cm of zmap](amap){$\rightsquigarrow$};
      \node[s-adam,left=2.75cm of amap](a0){$\loc$};
      \node[right=of a0](a1){$\loc'$};
      \node[right=1.25cm of amap,s-adam](a2){$\loc$};
      \node[right=of a2](a3){$\loc'$};
      \path[arrow,every node/.style={font=\footnotesize}]
      (a0) edge node{$\vec 0$} (a1)
      (a2) edge node{$\vec 0$} (a3);
    \end{tikzpicture}
    \caption{\label{11-fig-lipton}Schema of the reduction to
      "coverability" in the proof of \cref{11-avag-hard}.}
  \end{figure}
  
  It is hopefully clear that \Eve wins the "coverability game" played
  on~$\?V$ starting from $\loc'_0(\vec 0)$ and with target
  configuration $\loc_\mathtt{halt}(\vec 0)$ if and only if the
  "alternating Minsky machine" halts.

  \medskip Regarding "non-termination" games, we use essentially the
  same reduction.  First observe that, if \Eve can ensure reaching
  $\loc_\mathtt{halt}$ in the "alternating Minsky machine", then she
  can do so after at most $|\Loc|B^\dd$ steps.  We therefore use a
  `time budget': this is an additional component in $\?V$ with
  associated "unit vector"~$\vec t$.  This component is initialised to
  $|\Loc|B^\dd=|\Loc|2^{\dd 2^n}$ before the simulation, and decreases
  by~one at every step; see \cref{11-fig-lipton-nonterm}.  We also add
  a self loop $\loc_\mathtt{halt}\step{\vec 0}\loc_\mathtt{halt}$.
  Then the only way to avoid the "sink" and thus to win the
  "non-termination" game is to reach $\loc_\mathtt{halt}$.
  \begin{figure}[htbp]
    \centering
    \begin{tikzpicture}[auto,on grid,node distance=1.5cm]
      \node(to){$\mapsto$};
      \node[anchor=east,left=2.5cm of to](mm){"alternating Minsky machine"};
      \node[anchor=west,right=2.5cm of to](mwg){"asymmetric vector system"};
      % increment of Eve
      \node[below=.7cm of to](imap){$\rightsquigarrow$};
      \node[s-eve,left=2.75cm of imap](i0){$\loc$};
      \node[right=of i0](i1){$\loc'$};
      \node[right=1.25cm of imap,s-eve](i2){$\loc$};
      \node[right=1.8 of i2](i3){$\loc'$};      
      \path[arrow,every node/.style={font=\footnotesize}]
      (i0) edge node{$\vec e_i$} (i1)
      (i2) edge node{$\vec c_i-\bar{\vec c}_i-\vec t$} (i3);
      % decrement of Eve
      \node[below=1cm of imap](dmap){$\rightsquigarrow$};
      \node[s-eve,left=2.75cm of dmap](d0){$\loc$};
      \node[right=of d0](d1){$\loc'$};
      \node[right=1.25cm of dmap,s-eve](d2){$\loc$};
      \node[right=1.8 of d2](d3){$\loc'$};
      \path[arrow,every node/.style={font=\footnotesize}]
      (d0) edge node{$-\vec e_i$} (d1)
      (d2) edge node{$-\vec c_i+\bar{\vec c}_i-\vec t$} (d3);
      % zero test of Eve
      \node[below=1.5cm of dmap](zmap){$\rightsquigarrow$};
      \node[s-eve,left=2.75cm of zmap](z0){$\loc$};
      \node[right=of z0](z1){$\loc'$};
      \node[right=1.25cm of zmap,s-eve](z2){$\loc$};
      \node[right=of z2,s-adam-small](z3){};
      \node[above right=.8 and 1.1 of z3,s-eve-small](z4){};
      \node[below right=.8 and 1.1 of z3,inner sep=0pt](z5){$\loc'$};
      \node[right=1.8 of z4](z6){$\loc_\mathtt{halt}$};
      \path[arrow,every node/.style={font=\footnotesize}]
      (z0) edge node{$i\eqby?0$} (z1)
      (z2) edge node{$-\vec t$} (z3)
      (z3) edge node{$\vec 0$} (z4)
      (z3) edge[swap] node{$\vec 0$} (z5)
      (z4) edge node{$-2^{2^n}\cdot\bar{\vec c}_i$} (z6);
      % action of Adam
      \node[below=1.5cm of zmap](amap){$\rightsquigarrow$};
      \node[s-adam,left=2.75cm of amap](a0){$\loc$};
      \node[right=of a0](a1){$\loc'$};
      \node[right=1.25cm of amap,s-adam](a2){$\loc$};
      \node[right=of a2,s-eve-small](a3){};
      \node[right=of a3](a4){$\loc'$};
      \path[arrow,every node/.style={font=\footnotesize}]
      (a0) edge node{$\vec 0$} (a1)
      (a2) edge node{$\vec 0$} (a3)
      (a3) edge node{$-\vec t$} (a4);
    \end{tikzpicture}
    \caption{\label{11-fig-lipton-nonterm}Schema of the reduction to
      "non-termination" in the proof of \cref{11-avag-hard}.}
  \end{figure}

  We still need to extend our initialisation phase.  It suffices for
  this to implement a gadget for $\dd$-"meta-increments"
  $\loc\mstep{2^{\dd 2^j}\cdot\vec c}\loc'$ and $\dd$-"meta-decrements"
  $\loc\mstep{-2^{\dd 2^j}\cdot\vec c}\loc'$; this is the same argument as
  in Lipton's construction, with a base case $\loc\mstep{2^\dd}\loc'$
  for $j=0$.  Then we initialise our time budget through $|\Loc|$
  successive $\dd$-"meta-increments"
  $\loc\mstep{2^{\dd 2^n}\cdot\vec t}\loc'$.
  %%\end{scope}
\end{proof}

The proof of \cref{11-avag-hard} relies crucially on the fact that the
dimension is not fixed: although $\dd\geq 3$ suffices in the
"alternating Minsky machine", we need $O(|\?M|)$ additional counters
to carry out the reduction.  A separate argument is thus needed in
order to match the \EXP\ upper bound of \cref{11-avag-easy} in fixed
dimension.

\begin{theorem}\label{11-avag-two}
  "Coverability", "non-termination", and "parity@parity vector game"
  "asymmetric" "vector games" with "given initial credit" are
  \EXP-hard in dimension $\dd\geq 2$.
\end{theorem}
\begin{proof}
  We exhibit a reduction from "countdown games" with "given initial
  credit", which are \EXP-complete by \cref{11-countdown-given}.
  Consider an instance of a "configuration reachability" countdown
  game: a "countdown system"
  $\?V=(\Loc,\Act,\Loc_\mEve,\Loc_\mAdam,1)$ with initial
  configuration $\loc_0(n_0)$ and target
  configuration~$\smiley(0)$---as seen in the proof
  of \cref{11-countdown-given}, we can indeed assume that the target
  credit is zero; we will also assume that \Eve controls~$\smiley$ and
  that the only action available in~$\smiley$ is
  $\smiley\step{-1}\smiley$.  We construct an "asymmetric" "vector
  system" $\?V'$ of dimension~2 such that \Eve can ensure
  reaching~$\smiley(0,n_0)$ from $\loc_0(n_0,0)$ in~$\?V'$ if and only
  if she could ensure reaching $\smiley(0)$ from $\loc_0(n_0)$
  in~$\?V$.  The translation is depicted in \cref{11-fig-dim2}.
  
  \begin{figure}[htbp]
    \centering
    \begin{tikzpicture}[auto,on grid,node distance=1.5cm]
      \node(to){$\mapsto$};
      \node[anchor=east,left=2.5cm of to](mm){"countdown system"};
      \node[anchor=west,right=2.5cm of to](mwg){"asymmetric vector system"};
      % action of Eve
      \node[below=.7cm of to](imap){$\rightsquigarrow$};
      \node[s-eve,left=2.75cm of imap](i0){$\loc$};
      \node[right=of i0](i1){$\loc'$};
      \node[right=1.25cm of imap,s-eve](i2){$\loc$};
      \node[right=1.8 of i2](i3){$\loc'$};      
      \path[arrow,every node/.style={font=\footnotesize,inner sep=1pt}]
      (i0) edge node{$-n$} (i1)
      (i2) edge node{$-n,n$} (i3);
      % minimal action of Adam
      \node[below=1cm of imap](dmap){$\rightsquigarrow$};
      \node[s-adam,left=2.75cm of dmap](d0){$\loc$};
      \node[right=of d0](d1){$\loc'$};
      \node[below=.5 of d0]{$n=\min\{n'\mid\exists\loc''\in\Loc\mathbin.\loc\step{-n'}\loc''\in\Act\}$};
      \node[right=1.25cm of dmap,s-adam](d2){$\loc$};
      \node[right=1.8 of d2,s-eve-small](d3){};
      \node[right=1.8 of d3](d4){$\loc'$};
      \path[arrow,every node/.style={font=\footnotesize,inner sep=1pt}]
      (d0) edge node{$-n$} (d1)
      (d2) edge node{$0,0$} (d3)
      (d3) edge node{$-n,n$} (d4);
      % non-minimal action of Adam
      \node[below=1.5cm of dmap](zmap){$\rightsquigarrow$};
      \node[s-adam,left=2.75cm of zmap](z0){$\loc$};
      \node[right=of z0](z1){$\loc'$};
      \node[below=.5 of z0]{$n\neq\min\{n'\mid\exists\loc''\in\Loc\mathbin.\loc\step{-n'}\loc''\in\Act\}$};
      \node[right=1.25cm of zmap,s-adam](z2){$\loc$};
      \node[right=of z2,s-eve-small](z3){};
      \node[above right=.8 and 2.1 of z3](z4){$\loc'$};
      \node[below right=.8 and 2.1 of z3,s-eve](z5){$\smiley$};
      \path[arrow,every node/.style={font=\footnotesize,inner sep=1pt}]
      (z0) edge node{$-n$} (z1)
      (z2) edge node{$0,0$} (z3)
      (z3) edge[bend left=8] node{$-n,n$} (z4)
      (z3) edge[swap,bend right=8] node{$n_0-n+1,-n_0+n-1$} (z5)
      (z5) edge[loop above] node{$-1,1$} ()
      (z5) edge[loop right] node{$\,0,0$} ();
    \end{tikzpicture}
    \caption{\label{11-fig-dim2}Schema of the reduction in the proof
    of \cref{11-avag-two}.}
  \end{figure}
    
  The idea behind this translation is that a configuration $\loc(c)$
  of~$\?V$ is simulated by a configuration $\loc(c,n_0-c)$ in~$\?V'$.
  The crucial point is how to handle \Adam's moves.  In a
  configuration $\loc(c,n_0-c)$ with $\loc\in\Loc_\mAdam$, according
  to the "natural semantics" of $\?V$, \Adam should be able to
  simulate an action $\loc\step{-n}\loc'$ if and only if $c\geq n$.
  Observe that otherwise if $c<n$ and thus $n_0-c>n_0-n$, \Eve can
  play to reach~$\smiley$ and win immediately.  An exception to the
  above is if $n$ is minimal among the decrements in~$\loc$, because
  according to the "natural semantics" of~$\?V$, if $c<n$ there should
  be an edge to the "sink", and this is handled in the second line
  of \cref{11-fig-dim2}.

  Then \Eve can reach $\smiley(0,n_0)$ if and only if she can cover
  $\smiley(0,n_0)$, if and only if she can avoid the "sink" thanks to
  the self loop $\smiley\step{0,0}\smiley$.  This
  shows the \EXP-hardness of "coverability" and "non-termination"
  "asymmetric" "vector games" in dimension~two; the hardness of
  "parity@parity vector game" follows
  from \cref{11-cov2parity,12-nonterm2parity}.
\end{proof}



% Local IspellDict: british
