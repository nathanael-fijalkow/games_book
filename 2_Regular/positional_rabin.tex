\begin{theorem}\label{2-thm:Rabin_positional_determinacy}
Rabin games are uniformly "positionally determined".
\end{theorem}
\Cref{2-thm:Rabin_positional_determinacy} holds for infinite games.
However the proof we give here only applies to finite games and does not easily extend to infinite ones.
A different approach is required to obtain the positionality result for infinite games, see \cref{2-thm:characterisation_Zielonka_tree}.

\begin{proof}
We proceed by induction over the following quantity: total outdegree of vertices controlled by Eve minus number of vertices controller by Eve.
Since we assume that every vertex has an outgoing edge, the base case is when each vertex of Eve has only one successor. 
In that case Eve has only one strategy and it is positional, so the property holds.
    
In the inductive step, we consider a Rabin game $\game$ where Eve has a winning strategy $\sigma$.
Let $v \in \VE$ with at least two successors. 
We partition the outgoing edges of $v$ in two non-empty subsets which we call $E^v_1$ and $E^v_2$.
Let us define two games $\game_1$ and $\game_2$: the game $\game_1$ is obtained from $\game$ by removing the edges from $E^v_2$, and symmetrically for $\game_2$.

We claim that Eve has a winning strategy in either $\game_1$ or $\game_2$.
Let us assume towards contradiction that this is not the case: then there exist $\tau_1$ and $\tau_2$ two strategies for Adam
which are winning in $\game_1$ and $\game_2$ respectively.
We construct a strategy $\tau$ for Adam in $\game$ as follows: it has two modes, $1$ and $2$. The initial mode is $1$, and the strategy simulates $\tau_1$ from the mode $1$ and $\tau_2$ from the mode $2$. Whenever $v$ is visited, the mode is adjusted: if the outgoing edge is in $E^v_1$ then the new mode is $1$, otherwise it is $2$.
To be more specific: when simulating $\tau_1$ we play ignoring the parts of the play using mode $2$, so removing them yields a play consistent with $\tau_1$. The same goes for $\tau_2$.

Consider a play $\play$ consistent with $\sigma$ and $\tau$. 
Since $\sigma$ is winning, the play $\play$ is winning. It can be decomposed following which mode the play is in:
\[
\begin{array}{ccccccccc}
\text{mode } 1 & \overbrace{v_0 \cdots v}^{\play_1^0} & & 
\overbrace{v \cdots v}^{\play_1^1} & & \ \cdots \\
\text{mode } 2 && \underbrace{v \cdots v}_{\play_2^0} 
& & \underbrace{v \cdots v}_{\play_2^1} & \ \cdots
\end{array}
\]
where $\play_1 = \play_1^0 \play_1^1 \cdots$ is consistent with $\tau_1$ and $\play_2 = \play_2^0 \play_2^1 \cdots$ is consistent with $\tau_2$.
Since $\tau_1$ and $\tau_2$ are winning strategies for Adam, $\play_1$ and $\play_2$ do not satisfy $\Rabin$.

\vskip1em
There are two cases: the decomposition is either finite or infinite.
If it is finite we get a contradiction: since $\play$ is winning and $\Rabin$ is prefix independent any suffix of $\play$ is winning as well, contradicting that it is consistent with either $\tau_1$ or $\tau_2$ hence cannot be winning.

In the second case to get a contradiction we observe the following property of the Rabin objective:
let $(\rho_1^\ell)_{\ell \in \N}$ and $(\rho_2^\ell)_{\ell \in \N}$ such that:
\[
\begin{array}{lccccccccccc}
& \rho_1 & = & \rho_1^0 & & \rho_1^1 & & \cdots & \rho_1^\ell & & \cdots & \notin \Rabin \\
\text{and} & \rho_2 & = & & \rho_2^0 & & \rho_2^1 & \cdots & & \rho_2^\ell & \cdots & \notin \Rabin, \\ 
\text{then: } & \rho_1 \Join \rho_2 & = & \rho_1^0 & \rho_2^0 & \rho_1^1 & \rho_2^1 & \cdots & \rho_1^\ell & \rho_2^\ell & \cdots & \notin \Rabin.
\end{array}
\]
Indeed, since neither $\rho_1$ nor $\rho_2$ satisfy $\Rabin$, in both for all $i \in [1,d]$ if $R_i \in \Inf(\rho)$, then $G_i \in \Inf(\rho)$.
Since $\Inf(\rho_1~\Join~\rho_2) = \Inf(\rho_1) \cup \Inf(\rho_2)$, this implies that $\rho_1 \Join \rho_2$ does not satisfy $\Rabin$.

This directly yields a contradiction: neither $\play_1$ nor $\play_2$ satisfy $\Rabin$, yet their shuffle $\play$ does.
\end{proof}

The proof of \cref{2-thm:Rabin_positional_determinacy} yields a stronger result: every "prefix independent" submixing objective is "positionally determined" over finite arenas, where the ""submixing property"" is defined as follows for the objective $\Omega$:
\[
\begin{array}{lccccccccccc}
\text{if} & \rho_1 & = & \rho_1^0 & & \rho_1^1 & & \cdots & \rho_1^\ell & & \cdots & \notin \Omega \\
\text{and} & \rho_2 & = & & \rho_2^0 & & \rho_2^1 & \cdots & & \rho_2^\ell & \cdots & \notin \Omega, \\ 
\text{then: } & \rho_1 \Join \rho_2 & = & \rho_1^0 & \rho_2^0 & \rho_1^1 & \rho_2^1 & \cdots & \rho_1^\ell & \rho_2^\ell & \cdots & \notin \Omega.
\end{array}
\]

\begin{theorem}
\label{2-thm:submixing_positional}
Every "prefix independent" "submixing" objective is uniformly "positionally determined" over finite arenas.
\end{theorem}

