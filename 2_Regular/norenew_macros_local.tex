\newcommand{\F}{\mathcal{F}} 

\newcommand{\LAR}{\mathrm{LAR}}
\newcommand{\Zielonka}{\mathrm{Zielonka}}

\newcommand{\depth}{\mathrm{depth}}
\newcommand{\support}{\mathrm{supp}}
