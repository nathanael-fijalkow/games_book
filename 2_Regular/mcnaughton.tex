\begin{theorem}
\label{2-thm:muller}
Muller objectives are determined with finite memory strategies of size $d!$\footnote{See \cref{2-rmk:finite_infinite} for the case of infinite games.}.
There exists an algorithm for computing the winning regions of Muller games in exponential time,
and more specifically of complexity $O(m d (dn)^d)$, and in polynomial space, and more specifically $O(dm)$.
\end{theorem}
The presentation of the recursive algorithm for computing the winning regions of Muller games follows the exact same lines as for parity games:
indeed, the Muller objective extends the parity objective, and specialising the algorithm for Muller games to parity games
yields the algorithm we presented above.
%As for parity games, our proof is by induction over the number of colours but it uses the finiteness of the games.
%The finite memory determinacy actually holds for infinite games and the proof we present can be adapted to obtain this result.

The following lemma induces the recursive algorithm for computing the winning regions of Muller games.

\begin{lemma}
\label{2-lem:Muller_even}
Let $\Game$ be a Muller game such that $C \in \F$.
For each $c \in C$, let $\Game_c$ be the subgame of $\Game$ induced by $V \setminus \AttrE(c)$.
\begin{itemize}
	\item If for all $c \in C$, we have $\WA(\Game_c) = \emptyset$, then $\WE(\Game) = V$.
	\item If there exists $c \in C$ such that $\WA(\Game_c) \neq \emptyset$,
	let $\Game'$ be the subgame of $\Game$ induced by $V \setminus \AttrA( \WA(\Game_c) )$,
	then $\WE(\Game) = \WE(\Game')$.	
\end{itemize}
\end{lemma}

\begin{proof}
We prove the first item.

For each $c \in C$, let $\sigma_c$ be an attractor strategy ensuring to reach $c$ from $\AttrE(c)$,
and consider a winning strategy for Eve from $V \setminus \AttrE(c)$ in $\Game_c$, it induces a strategy $\sigma'_c$ in $\Game$.
We construct a strategy $\sigma$ in $\Game$ which will simulate the strategies above in turn; to do so it uses $C$ as top-level memory states.
(We note that the strategies $\sigma'_c$ may use memory as well, so $\sigma$ may actually use more memory than just $C$.)
The strategy $\sigma$ with memory $c$ simulates $\sigma_c$ from $\AttrE(c)$ and $\sigma'_c$ from $V \setminus \AttrE(c)$,
and if it ever reaches $c$ it updates its memory state to $c + 1$ and $1$ if $c = d$.
Any play consistent with $\sigma$ either updates its memory state infinitely many times, 
or eventually remains in $V \setminus \AttrE(c)$ and is eventually consistent with $\sigma'_c$.
In the first case it sees each colour infinitely many times, and since $C \in \F$ the play satisfies $\Muller(\F)$, 
and in the other case since $\sigma'_c$ is winning the play satisfies $\Muller(\F)$.
Thus $\sigma$ is winning from $V$.

We now look at the second item.

Let $\tau_a$ denote an attractor strategy from $\AttrA(\WA(\Game_c)) \setminus \WA(\Game_c)$.
Consider a winning strategy for Adam from $\WA(\Game_c)$ in $\Game_c$, it induces a strategy $\tau_c$ in $\Game$.
Since $V \setminus \AttrE(c)$ is a "trap" for Eve, this implies that $\tau_c$ is a winning strategy in $\Game$.
Consider now a winning strategy in the game $\Game'$ from $\WA(\Game')$, it induces a strategy $\tau'$ in $\Game$.
The set $V \setminus \AttrA( \WA(\Game_c) )$ may not be a trap for Eve, so we cannot conclude that $\tau'$ is a winning strategy in $\Game$,
and it indeed may not be.
We construct a strategy $\tau$ in $\Game$ as the (disjoint) union of the strategy $\tau_a$ on $\AttrA(\WA(\Game_c)) \setminus \WA(\Game_c)$,
the strategy $\tau_c$ on $\WA(\Game_c)$ and the strategy $\tau'$ on $\WA(\Game')$.
We argue that $\tau$ is winning from $\AttrA( \WA(\Game_c) ) \cup \WA(\Game')$ in $\Game$.
Indeed, any play consistent with this strategy in $\Game$ either stays forever in $\WA(\Game')$ hence is consistent with $\tau'$
or enters $\AttrA( \WA(\Game_c) )$, so it is eventually consistent with $\tau_c$.
In both cases this implies that the play is winning.
Thus we have proved that $\AttrA( \WA(\Game_c) ) \cup \WA(\Game') \subseteq \WA(\Game)$.

We now show that $\WE(\Game') \subseteq \WE(\Game)$, which implies the converse inclusion.
Consider a winning strategy from $\WE(\Game')$ in $\Game'$, it induces a strategy $\sigma$ in $\Game$.
Since $\Game'$ is a "trap" for Adam, any play consistent with $\sigma$ stays forever in $\WE(\Game')$, 
implying that $\sigma$ is winning from $\WE(\Game')$ in $\Game$.
\end{proof}

To get the full algorithm we need the analogous lemma for the case where $C \notin \F$.
We do not prove it as it is the exact dual of the previous lemma, and the proof is the same swapping the two players.

\begin{lemma}
\label{2-lem:Muller_odd}
Let $\Game$ be a Muller game such that $C \notin \F$.
For each $c \in C$, let $\Game_c$ be the subgame of $\Game$ induced by $V \setminus \AttrA(c)$.
\begin{itemize}
	\item If for all $c \in C$, we have $\WE(\Game_c) = \emptyset$, then $\WA(\Game) = V$.
	\item If there exists $c \in C$ such that $\WE(\Game_c) \neq \emptyset$,
	let $\Game'$ be the subgame of $\Game$ induced by $V \setminus \AttrE( \WE(\Game_c) )$,
	then $\WA(\Game) = \WA(\Game')$.	
\end{itemize}
\end{lemma}

The algorithm is presented in pseudocode in \Cref{2-algo:mcnaughton}.
We only give the case where $C \in \F$, the other case being symmetric.
The base case is when there is only one colour $c$, in wich case Eve wins everywhere if $\F = \set{c}$
and Adam wins everywhere if $\F = \emptyset$.

We now perform a complexity analysis of the algorithm.

Let us start with time complexity and write $f(n,d)$ for the number of recursive calls performed by the algorithm on Muller games with $n$ vertices and $d$ colours.
We have $f(n,1) = f(0,d) = 0$, with the general induction:
\[
f(n,d) \le d \cdot f(n,d-1) + f(n-1,d) + d + 1.
\]
The term $d \cdot f(n,d-1)$ corresponds to the recursive call to $\Game_c$ for each $c \in C$,
the term $f(n-1,d)$ to the recursive call to $\Game'$.
We obtain $f(n,d) \le d n \cdot f(n,d-1) + (d+1)n$,
so $f(n,d) \le (d+1)n (1 + dn + (dn)^2 + \dots + (dn)^{d-2}) = O((dn)^d)$.
In each recursive call we perform $d+1$ attractor computations so the number of operations in one recursive call is $O(dm)$.
Thus the overall time complexity is $O(m d (dn)^d)$.

%The space complexity is bounded by $O(nd)$.

The proofs of \cref{2-lem:Muller_even} and \cref{2-lem:Muller_odd} also imply that Muller games are determined with finite memory of size $d!$.
We do not make it more precise here because an improved analysis of the memory requirements will be conducted in \cref{2-sec:zielonka}
using a variant of this algorithm.

\begin{algorithm}
 \KwData{A Muller game $\Game$ over $C = [1,d]$}
 \SetKwFunction{FSolveIn}{SolveIn}
 \SetKwFunction{FSolveOut}{SolveOut}
 \SetKwProg{Fn}{Function}{:}{}

\Fn{\FSolveIn{$\Game$}}{
	\If{$d = 1$}{
		\If{$\F = \set{1}$}{
			\Return{$V$}
		}
		\Else{
			\Return{$\emptyset$}
		}
	}

	\For{$c \in C$}{
		Let $\Game_c$ the subgame of $\Game$ induced by $V \setminus \AttrE(c)$
	
		\If{$C \setminus \set{c} \in \F$}{
			$\WE(\Game_c) \leftarrow$ \FSolveIn($\Game_c$)
		}
		\Else{
			$\WA(\Game_c) \leftarrow$ \FSolveOut($\Game_c$)
		}
	}

	\If{$\forall c \in C,\ \WA(\Game_c) = \emptyset$}{
		\Return{$V$}
	}
	\Else{
		Let $c$ such that $\WA(\Game_c) \neq \emptyset$

		Let $\Game'$ the subgame of $\Game$ induced by $V \setminus \AttrA( \WA(\Game_c) )$

		$\WE(\Game') \leftarrow$ \FSolveIn($\Game'$)
					
		\Return{$\WE(\Game')$}
	}
}
\vskip1em
\Fn{\FSolveOut{$\Game$}}{
	Symmetric to \FSolveIn
}
\vskip1em
\If{$C \in \F$}{
	\FSolveIn{$\Game$}
}
\Else{
	\FSolveOut{$\Game$}
}
\caption{A recursive algorithm for computing the winning regions of Muller games.}
\label{2-algo:mcnaughton}
\end{algorithm}
