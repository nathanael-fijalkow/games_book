%\def\payoff{\textsf{payoff}}
\def\payoff{\ensuremath{f}}
%\def\Act{\textsf{Act}}
\def\Act{A}
\def\Agt{\mathcal{P}}
\def\move{\textsf{move}}
\def\Out{\textsf{Out}}
\def\Dev{\textsf{Dev}}
\def\maxinf{\text{\rm maxinf}}
\def\pes{\textsf{pes}}
\def\opt{\textsf{opt}}
\def\proj{\textsf{proj}}
\def\devg{\textsf{DevGame}}
% \section*{Outline}
% \begin{itemize}
% \item Introduction
% \item Nash Equilibria in Games in normal form
% \item Nash Equilibria in $\Omega$-regular games
% \item Admissible strategies
% \end{itemize}

% \begin{itemize}
% \item Check notations, outcome, play etc.
% \item Always specify start vertex in $\Out$
% \end{itemize}
\section{Introduction}
\label{14-section:introduction}

In two player games seen so far, players had objectives that are
opposite to each other's, so we were able to define them giving only
Eve's objective. Adam was seen as a purely adversarial player. Such games
are called \emph{zero-sum} games since, in a quantitative setting, the
sum of the payoffs of the two players would sum up to zero in any
outcome. However, the objectives of the players are not entirely
conflicting in all games.
In particular, in multiplayer games, that is,
games with more than two players, the binary view of zero-sum games
does not make sense;
%In particular, multiplayer games, that is,
%games with more than two players, cannot be zero-sum by definition;
but there are also interesting examples of non-zero sum games with only two
players (we will see one below). In this setting, winning strategies are
no longer suitable to describe rational behaviors since the opponents
should no longer be seen as purely adversarial. In fact, when the
objectives of the players are not opposite, some cooperation becomes
possible. Then, rather than assuming that opponents are purely
adversarial, it is interesting to study the possible outcomes when they
are simply \emph{rational}, that is, follow the best strategy for their
own objectives. The notion of equilibria we will study in this chapter
aims at describing such rational behaviors.

If one is expecting for sure some specific strategies to be played by the
opponents, then the most rational response is to choose the
\emph{best response}, that is, the strategy that is optimal for the
player against the given strategies of the other players. Thus, if we
assign strategies to players, and if the players are all aware of the
strategies of the other players, then each player will be willing to
change their strategy if theirs does not turn out to be a best response.
Such a situation is seen as unstable and is undesirable in many
applications of game theory. \emph{Nash equilibrium} is defined simply
as a stable situation in such a setting: a strategy profile in which the
strategy of each player is a best response to the rest of the
strategies. Thus, no player has any incentive to change their strategy.

We will see the formal definition of a Nash equilibrium in the next
section. Let us first consider the following example.

The following Hawk-Dove game was first presented by the biologists Smith
and Price, and shown in Table~\ref{14-tab:hawk-dove}.
Here, two animals are fighting for ressources and can choose to
either act as a hawk or as a dove.
If both player choose hawk they will have to fight for resources, and
thus only get payoff 0. If only one chooses hawk, they get a high payoff of
4, because they get all the valuable resources for themselves, while the dove
gets 1: they get plenty of resources but gets hunted. When they both choose
dove, they both get a payoff of 3: they have to share resources but do not get
hunted.

When a player chooses hawk then the best payoff for the opponent is
obtained by choosing dove, so as to avoid fighting for resources.
So, dove is the best response to hawk. Reciprocally, the best response to
dove is to play hawk. There are two ``equilibria'': (Hawk, Dove) and
(Dove, Hawk), where no player has an interest in changing their
strategy. Note that the highest payoff a player can ensure
(against all adversary strategies) is only~$1$.
%however that none of the players has a winning strategy.
%\todo{What does winning strategy mean??}
%\todo{Say rather that the best players can ensure is 1 by playing Dove}

Nash showed the existence of such equilibria in any normal-form game
\footnote{normal-form games are also called matrix games, and some
specific form was defined in \Cref{chap:concurrent}.},
which may require randomized strategies. This result
revolutionized the field of economics, where it is used to analyze
competitions between firms or government economic policies for example.
Game theory and the concept of Nash equilibrium are now applied to
diverse fields: in finance to analyse the evolution of market prices, in
biology to understand the evolution of some species, in political
sciences to explain public choices made by parties.

\begin{table}
  \caption{The Hawk-Dove game. Each column corresponds to a strategy of
    \(P_1\) and each line to a strategy of \(P_2\).}
  \label{14-tab:hawk-dove}
  \begin{center}
    \begin{tabular}[c]{|@{~}l@{~}|@{~}c@{~} @{~}c@{~}|}
      \hline
      & Hawk & Dove \\
      \hline
      Hawk & 0 , 0 & 1 , 4 \\
      Dove & 4 , 1 & 3 , 3 \\
      \hline
    \end{tabular}
  \end{center}
\end{table}

\medskip
In this chapter, we will first study the computation of Nash
equilibria in multiplayer concurrent games with $\omega$-regular
%\todo{Replace omega by $\omega$?}
objectives. The algorithms we present here differ from those that were
given for normal-form games since ours are infinite-duration with
omega-regular objectives. We will then present extensions of this notion
such as secure and
robust equilibria. The second result we develop is
the notion of admissibility: this is a different approach to the study
of rational behaviours and consist in eliminating for each player
irrational choices of strategies.

\section{Nash Equilibria in Games in Normal
Form}\label{14-section:nash-equilibria-in-games-in-normal-form}

% Let us introduce the notions we will study in normal-form games.

\fbox{Pas mal de d\'efinitions vont \^etre h\'erit\'ees du chapitre concurrent}

\subsection{Normal form games}\label{normal-form-games}

\textbf{Definition}: A normal-form game is given by a set of players
\(Players\), a set of strategies \(Strat_P\) for each player
\(P \in Players\) and a payoff function
\({\tt payoff}_P : \prod_{P \in Players} Strat_P \to \mathbb{R}\).

The goal of each player is to maximize their payoff, and the payoff of
each player can depend on other players' strategies. Note that we are in
a concurrent setting, so each player chooses their strategy
independently.

A \emph{strategy profile} is a map assigning a strategy to each player.
We simply denote it as a tuple
\((\sigma_1,\sigma_2, \ldots,\sigma_n) \in \prod_{p \in P} Strat_p\).
Such a tuple will often be written \(\sigma_P\), and for a subset of
players \(A \subseteq P\), \(\sigma_A\) denotes the strategy profile for
the players. The
pair\textasciitilde{}\((\sigma_A,\sigma_{P\setminus A})\) then refers to
the full strategy profile.

\textbf{Example}: In the Hawk-Dove game, there are two players \(P_1\)
and \(P_2\). Possible strategies are
\(Strat_{P_1} = Strat_{P_2} = \{ Hawk, Dove \}\). The payoffs
corresponding to different strategy profiles are given in Table
\emph{(reference to table)}, \({\tt payoff}_{P_1}(Hawk, Hawk) = 0\),
\({\tt payoff}_{P_1}(Hawk, Dove) = 4\), etc\ldots{}

A \emph{randomised} strategy is a probability distributing over the
actions, i.e. \(\forall a\in Act.\ \sigma_P(a)\in[0,1]\) and
\(\sum_{a \in Act} \sigma_P(a) = 1\). The resulting payoff of a strategy
profile is then
\(\sum_{a_{Agt} \in Act^{Agt}} {\tt payoff}(a_{Agt}) \cdot \prod_{A\in Agt} \sigma_A(a_A)\).
\fbox{probablement sera d\'ej\`a d\'efini}

A strategy is said to be \emph{pure} if the support of its distribution
is a singleton. A pure strategy profile is a profile consisting of pure
strategies. A randomised strategy profile is a general strategy profile
possibly containing randomised strategies.

A \emph{pure (resp. randomised) Nash equilibrium} is pure (resp.
randomised) strategy profile \(\sigma_P\) such that for all players
\(A_i\in P\), and all strategies \(\sigma'_{A_i}\),
\({\tt payoff}(\sigma'_{A_i}, \sigma_{P\setminus A_i}) \le {\tt payoff}(\sigma_{A_i}, \sigma_{P\setminus A_i})\).

The normal form games we consider differ from the matrix games of \Cref{chap:concurrent}, in that each player has their own payoff.
So for instance, when player 1 chooses column Hawk, and player 2 chooses
row Dove, the payoff for player 1 is $\payoff_{P_1}(\text{Hawk}, \text{Dove}) = 4$.
%\todo{The first chapter uses $f$ for quantitative objectes. Should we do the same?}
Let us call a vector of strategies specifying a strategy for each player a \emph{strategy profile}. In normal-form games, each cell of the table~$\Delta$ corresponds to a strategy profile.

\begin{definition}
  A \emph{Nash equilibrium} is a \emph{stable} strategy profile in which
  strategy is a best response against the other strategies.
\end{definition}
Thus a Nash equilibrium is a stable situation in the sense that
no player has an incentive in changing their strategy.
Nash proved that when
players are allowed to randomise among all their strategies, there always
exists a Nash equilibrium.

\begin{theorem}[Existence of Nash equilibria]
In every normal-form game with a finite number of
players, each having a finite number of pure strategies, there exists a
randomised Nash equilibrium.
\end{theorem}

Note that not all games contain pure Nash equilibria.
For example, in the rock-paper-scissors game, the best response to rock
is paper, to paper is scissors, and to scissors is rock, so none of these
pure strategies can be an equilibrium.

For finding a pure Nash equilibrium in a normal-form game, there is a simple
polynomial time algorithm.
For each strategy profile, we look for each player whether they have a better
response than their current strategy.
If no player has a better response, the strategy profile is a Nash equilibrium,
otherwise we move to the next one, and if none satisfies the condition then there is no equilibrium.

%\subsection{Multiplayer Games for Synthesis}\label{multiplayer-games-for-synthesis}

\begin{example}[Medium Access Control]
Consider a medium access control
problem, where several users share access to a wireless channel. A
communication over the channel is successful if there are no collisions,
that is, if a single user is transmitting their message only. During each
slot, each user chooses either to transmit or to idle. Intuitively, the
number of packets transmitted without collision decreases with
the number of users emitting in the same slot. Furthermore each attempt
at transmitting has a cost. An example payoff for two players,
is represented in Table~\ref{ex:medium-access}.

\begin{table}
  \caption{A game of medium access.}
  \label{ex:medium-access}
  \begin{center}
    \begin{tabular}[c]{|@{\hspace{1em}}l@{\hspace{1em}}|@{\hspace{1em}}c@{\hspace{1em}}c@{\hspace{1em}}|}
      \hline
      & Emit & Wait\\
      \hline
      Emit & -1, -1 & 2, 0\\
      Wait & 0 , 2 & 0, 0\\
      \hline
    \end{tabular}
  \end{center}
\end{table}
\end{example}
We encourage the reader to find the Nash equilibria of the above game.

The game described above corresponds to a single slot of this system. In
a practical scenario, there would be a succession of slots and the payoff would be
the sum of payoffs over all slots. Normal-form games are thus not
sufficient to represent games with repetitions and to study the
evolution of the behaviours as the game evolves.

One possibilty to model repetition is to use
\emph{games in extensive form} which are games played on finite trees.
However such games only model a fixed number of repetitions unlike
infinite or arbitrary duration games as studied in this book. We thus
study, in the rest of this chapter, algorithms for games played on
graphs.

\section{Nash Equilibria in Omega-Regular
Games}\label{nash-equilibria-in-omega-regular-games}

\subsection{Definitions}\label{definition}

\begin{definition}
  A multiplayer \emph{arena} \(\mathcal{A}\) with~$k$ players is a tuple
  % \(\langle V, Agt, Act,Tab,(c_A)_{A\in Agt_i} \rangle \), where:
  \(\langle V, \Act,\Delta,(c_P)_{P\in \Agt} \rangle \), where:
%  \todo{First chapter uses $A$ for actions}
%  \todo{$A$ for players shoould be replaced by $P_i$ and $c_A$ by $C_i$}

  \begin{itemize}

  \item
    \(V\) is a finite set of vertices;
  \item
    \(\Agt = \{1,2,\ldots,k\}\) is the set of players;
  \item
    \(\Act\) is a finite set of actions, a tuple \((a_P)_{P \in \Agt}\)
    containing one action \(a_P\) for each player $A$ is called a
    \emph{move}, thus \(\Act^k\) is the set of possible moves;
  \item
    \(\Delta : V \times \Act^k \to V\) is the transition function which
    associates to a pair of vertices and moves the resulting state;
  \item
    \((c_P)_{P \in \Agt}\) is a tuple of colouring functions
    with~$c_P : V \rightarrow C$ for each~$P \in \Agt$.
  \end{itemize}
\end{definition}

\begin{example}
A simple three-player concurrent game is represented in \Cref{14-fig:example1}.
Vertices are $v_0$, $v_1$, $v_2$, $v_3$ and $v_4$.
%\todo{replace $s_i$s by $v_i$s}
Players are named $P_1$, $P_2$, $P_3$.
The set of actions is $\Act = \{ a , b\}$.
The transition relation is given by the edges in the graph, for instance
$\Delta(s_0, (a, b, a))$ is $v_1$. In our figures, $\ast$ represents
an arbitrary action.
The colouring function is represented below vertices as tuples ranging over players.
For instance, a vertex labelled by $(1,1,0)$ assigns
the first two players the colour~$1$, and the third player the colour~$0$,
In particular, $c_{P_1}(v_2) = 1$, and $c_{P_3}(v_2)= 0$.
\end{example}

\begin{figure}
  \begin{center}
    \begin{tikzpicture}
      \draw (0,0) node[draw, inner sep=7pt] (I) {$v_0$};
      \draw (I.-90) node[below] {$1, 1, 1$};
      \draw (4,1) node[draw, inner sep=7pt] (C1) {$v_1$};
      \draw (C1.-100) node[below] {$1, 1, 0$};
      \draw (4,-1) node[draw, inner sep=7pt] (C2) {$v_2$};
      \draw (C2.-90) node[below] {$1, 1, 0$};
      \draw (1.5, 2) node[draw, inner sep=7pt] (S1) {$v_3$};
      \draw (S1.-90) node[below] {$0, 1, 1$};
      \draw (1.5,-2) node[draw, inner sep=7pt] (S2) {$v_4$};
      \draw (S2.-90) node[below] {$1, 0, 1$};
      \draw[-latex'] (-1, 0) -- (I);
      \draw[-latex'] (I) -- node[sloped, text width=1cm]{$a, b, \ast$ \\ $b, a, \ast$} (C1);
      \draw[-latex'] (I) -- node[sloped, text width=1cm]{$a, a, \ast$ \\ $b, b, \ast$} (C2);
      \draw[-latex'] (C1) -- node[left]{$\ast, \ast, a$} (C2);
      % \draw[-latex'] (C2) -- node[right]{$a, a$} (C1);
      \draw[-latex'] (C1) -- node[above]{$\ast, \ast, b$} (S1);
      \draw[-latex'] (C2) -- node[below]{$\ast, b, \ast$} (S2);
      \draw[-latex', rounded corners] (C2) -- +(1,1) -- node[right]{$\ast, a, \ast$} (C1);
      \draw[-latex'] (S1) -- node[above, sloped, pos=0.3]{$\ast, \ast, \ast$} (I);
      \draw[-latex'] (S2) -- node[above, sloped, pos=0.3]{$\ast, \ast, \ast$} (I);
    \end{tikzpicture}

    % \begin{tikzpicture}
    %   \draw (0,0) node[draw, inner sep=7pt] (I) {$s_0$};
    %   \draw (I.-90) node[below] {$1, 1, 1$};
    %   \draw (4,1) node[draw, inner sep=7pt] (C1) {$s_1$};
    %   \draw (C1.-100) node[below] {$1, 1, 0$};
    %   \draw (4,-1) node[draw, inner sep=7pt] (C2) {$s_2$};
    %   \draw (C2.-90) node[below] {$1, 1, 0$};
    %   \draw (1.5, 2) node[draw, inner sep=7pt] (S1) {$s_3$};
    %   \draw (S1.-90) node[below] {$0, 1, 1$};
    %   \draw (1.5,-2) node[draw, inner sep=7pt] (S2) {$s_4$};
    %   \draw (S2.-90) node[below] {$1, 0, 1$};
    %   \draw[-latex'] (-1, 0) -- (I);
    %   \draw[-latex'] (I) -- node[sloped, text width=1cm]{$a, b, \ast$ \\ $b, a, \ast$} (C1);
    %   \draw[-latex'] (I) -- node[sloped, text width=1cm]{$a, a, \ast$ \\ $b, b, \ast$} (C2);
    %   \draw[-latex'] (C1) -- node[left]{$\ast, \ast, a$} (C2);
    %   % \draw[-latex'] (C2) -- node[right]{$a, a$} (C1);
    %   \draw[-latex'] (C1) -- node[above]{$\ast, \ast, b$} (S1);
    %   \draw[-latex'] (C2) -- node[below]{$\ast, b, \ast$} (S2);
    %   \draw[-latex', rounded corners] (C2) -- +(1,1) -- node[right]{$\ast, a, \ast$} (C1);
    %   \draw[-latex'] (S1) -- node[above, sloped, pos=0.3]{$\ast, \ast, \ast$} (I);
    %   \draw[-latex'] (S2) -- node[above, sloped, pos=0.3]{$\ast, \ast, \ast$} (I);
    % \end{tikzpicture}
    \caption{Example of a three-player concurrent arena.
      The symbol $\ast$ on edges can be replaced by either $a$ or $b$.}
    \label{14-fig:example1}
  \end{center}
\end{figure}


% \begin{definition}[History and plays]
A \emph{history} of the multiplayer arena
% \todo{For the first chapter this is a finite play}
\({\mathcal A}\) is a finite sequence of states and moves ending with a
state, i.e.~a word in \((V \cdot \Act^\Agt)^* \cdot V\). Note that unlike
for two player games we include actions in the history, because knowing
the source and target vertices does not mean you know which player chose
which actions.
%\end{definition}

For a history~$\pi$, we write \(\pi_i\) the $i$-th vertex of $\pi$, starting from $0$, and
%\todo{$\pi_i$ corresponds to an edge rather than a vertex: Je ne trouve pas \c ca grave}
\(\move_i(\pi)\) its $i$-th move, thus
\(\pi = \pi_0 \cdot \move_0(\pi \cdot \pi_1 \cdots \move_{n-1}(\pi)\cdot \pi_n\), and
with this notation $\move_i(\pi)_P$ is the $i$-th action of player $P$ in $h$.
The length $|\pi|$ of such a history is $n + 1$. We write
$\last(\pi)$ the last vertex of h, i.e. \(\pi_{|\pi|-1}\).
A play \(\rho\) is an
infinite sequence of vertices and moves, i.e.~an element of
\((V \cdot \Act^{\Agt})^\omega\).

\def\Coalition{\ensuremath{\mathcal{C}}}
\begin{definition}[Strategy and coalition]
  A strategy is a function which associates an action to each history.
  We often write $\sigma_A$ for a strategy of player $P$.
  A coalition $\Coalition$ is a set of players in $\Agt$, and we write
  % \todo{$\Coalition$ could be confused with colours}
  $-\Coalition$ for the remaining players, that is $-\Coalition = Agt \setminus \Coalition$.
  Let \(\Coalition\) be a coalition, a strategy \(\sigma_\Coalition\) for \(\Coalition\) is a function
  which associates a strategy \(\sigma_P\) to each player \(P\in \Coalition\).
  Given a strategy $\sigma_\Coalition$, when it is clear from the context, we simply
  write \(\sigma_P\) for \(\sigma_\Coalition(P)\).
\end{definition}

\begin{definition}[Outcomes]
  A history \(\pi\) is compatible
  with the strategy \(\sigma_\Coalition\) for coalition~$\Coalition$ if, for all \(k < |\pi| - 1\) and all
  \(P \in \Coalition\), \((\move_k(\pi))_P = \sigma_P(\pi{\le k})\), and
  \(\Delta(\pi_k, \move_k(\pi)) = \pi_{k+1}\). A play \(\rho\) is compatible with
  the strategy \(\sigma_\Coalition\) if all its prefixes are. We write
  % \todo{The first chapter would rather write $\pi^{v_0}_{\sigma_\Coalition}$}
  \(\Out_{\mathcal{A}}(v_0, \sigma_\Coalition)\) for the set of plays in \(\mathcal{A}\) that
  are compatible with strategy \(\sigma_\Coalition\) and have initial vertex
  \(v_0\). Let \(\Out_{\mathcal{A}}(\sigma_\Coalition)\) denote the union
  of \(\Out_{\mathcal{A}}(v_0, \sigma_\Coalition)\) for all~$v_0$,
  and \(\Out_{\mathcal{A}}(v_0)\) the union of all \(\Out_{\mathcal{A}}(v_0, \sigma_\Coalition)\).
  The subscript~$\mathcal{A}$ can be omitted if it is clear from the context.
  These paths are called \emph{outcomes} of \(\sigma_\Coalition\) from
  \(v_0\). 
%We write \(\Out_{\mathcal A}\) the set of all plays that are
%compatible with some strategy \(\sigma_{\Agt}\) of \(\Agt\) regardless of the initial vertex. 
\end{definition}
Note that
when the coalition \(\Coalition\) is composed of all the players
the outcome from a given state is unique.

\begin{example}
  Consider in the example of \Cref{14-fig:example1}, the following
  strategies:
  \begin{itemize}
  \item $P_1$ always plays $a$, i.e. $\sigma_{P_1}(\pi) = a$ for all histories $\pi$;
  \item $P_2$ plays $a$ in $v_0$ if it is the first state and then always plays $b$, i.e. $\sigma_{P_2}(v_0) = a$ and $\sigma_{P_2}(\pi) = b$ for all $\pi \ne v_0$;
  \item $P_3$ always plays $b$, i.e. $\sigma_{P_3}(\pi) = b$.
  \end{itemize}
  The outcome from $v_0$ in that case is
  \begin{align*}
    \Out(v_0, \sigma_{\{P_1, P_2, P_3\}}) ~ = & ~ v_0 \cdot (a, a, b) \cdot v_2 \cdot (a, b, b)
                                                \cdot v_4 \cdot (a, b, b) \cdot\\
                                              & \left(v_0 \cdot (a, b, b)
                                                \cdot v_1 \cdot (a, b, b)
                                                \cdot v_3 \cdot (a, b, b)\right)^\omega
  \end{align*}
\end{example}


\begin{definition}[Multiplayer game]
  A \emph{payoff} function associates a real number to each outcome.
  We will be mostly be interested in solving games with qualitative
  objectives, that is payoffs that take values $0$ and $1$.
%  In that case we talk about \emph{qualitative objectives} 
%  \todo{we hould talk of qualitative objective, and maybe quantitative objective rather than payoff}
  A \emph{multiplayer game} \((\mathcal{A}, (\payoff_P)_{P \in \Agt})\) is given by a multiplayer arena $\mathcal{A}$, an initial
  vertex $v_0$ and payoff function $\payoff_P$ for each player~$P$.
  When $\payoff_P$ is qualitative we simply write $\Omega_P$
  for the corresponding objective.
\end{definition}


\subsection{The Nash equilibrium problem}
\label{14-subsection:algorithm-for-finding-nash-equilibria}

In this section we will present an algorithm to compute
Nash equilibria in
multiplayer games.
The problem we are interested in is to decide the existence of a Nash
equilibrium in which the objectives of a given set of players are
satisfied.

\begin{svgraybox}
\decisionproblem{
\parbox[t]{0.75\textwidth}{Multiplayer game $(\mathcal{A}, (\payoff_P)_{P \in \Agt})$, payoff
  bounds $(b_P)_{P\in\ \Agt}$, and initial vertex~$v_0$}
}{
  \parbox[t]{0.75\textwidth}{is there
a Nash equilibrium $\sigma_{\Agt}$ such that for all
$P \in \Agt, \payoff_P(\Out({v_0,\sigma_{\Agt}})) \ge b_P$?}}
\end{svgraybox}
\todo{OS: I don't know how to fix this alignment issue}

The algorithm is based on a reduction to zero-sum two-players games,
which allows us to use algorithms presented in the previous chapters of
this book. More precisely, we present the \emph{deviator game}, which is
a transformation of a concurrent multiplayer game into a turn-based
zero-sum game, such that there are strong links between equilibria in
the former and winning strategies in the latter. The proofs of
this section are independent of the type of objectives we consider.

\subsection{Deviators}\label{14-subsection:deviators}

A central notion we use is that of \emph{deviators}. These are the
players who have played different moves from those prescribed in a given
profile, thus causing a deviation from the expected outcome. Formally, a
deviator from move
\(a_{\Agt}\) to \(a'_{\Agt}\) is a player
\(D \in \Agt\) such that \(a_D \ne a'_D\) . We denote the set of
deviators by \[
        \Dev(a_{\Agt} , a'_{\Agt} ) = \{D \in \Agt \mid a_D \ne a'_D \}.
\] We extend the definition to pairs of histories and strategies by
taking the union of deviator sets of each step along the history.
Formally,
\[
\Dev(\pi, \sigma_{\Agt}) = \bigcup_{0\le i < |h|}~ \Dev(\move_i(\pi), \sigma_{\Agt}(\pi_{\le i})).
\]
For an infinite play \(\rho\), we define
%\todo{should use $\pi$ rather than $\rho$. I don't think it's a problem to use $\pi$ for histories and~$\rho$ for infinite plays}
\(\Dev(\rho, \sigma_{\Agt} ) = \bigcup_{i \in \mathbb{N}} \Dev(\move_i(\rho), \sigma_{\Agt}(\rho_{\le i} ))\).
Intuitively, having chosen a strategy profile \(\sigma_{\Agt}\) and
observed a play \(\rho\), deviators represent the players that must have
changed their strategies from \(\sigma_{\Agt}\) in order to generate
\(\rho\).

\begin{lemma}\label{14-lem:deviator}
Given a play \(\rho\), strategy profile~$\sigma_\Agt$, a coalition \(\Coalition\)
contains \(\Dev(\rho)\), if and only if, there exists a strategy
\(\sigma'_\Coalition\) such that \(\Out(\rho_1, \sigma_{-\Coalition}, \sigma'_\Coalition) = \rho\).
\end{lemma}
\begin{proof}
Assume that coalition \(\Coalition\) contains
\(\Dev(\rho, \sigma_{\Agt})\). We define the strategy \(\sigma_\Coalition\) to be
such that for all \(i\in \mathbb{N}\),
\(\sigma_\Coalition(\rho_{\le i} ) = (\move_i(\rho))_\Coalition\). By hypothesis, we have,
for all indices \(i\),
\(\Dev(\move_i(\rho), \sigma_{\Agt}(\rho_{\leq i})) \subseteq \Coalition\), so for
all players \(A\not\in \Coalition\),
\(\sigma_A(\rho_{\le i}) = (\move_i(\rho))_A\). Then
\(\Delta(\rho_i, \sigma'_\Coalition(\rho_{\le i}), \sigma_{-\Coalition}(\rho_{\le i})) = \rho_{i+1}\).
Hence \(\rho\) is the outcome of the profile
\((\sigma_{-\Coalition}, \sigma'_\Coalition)\).

For the other direction, let \(\sigma_{\Agt}\) be a strategy profile,
\(\sigma'_\Coalition\) a strategy for coalition \(\Coalition\), and
\(\rho \in Out_G(\rho_0 , \sigma_{-\Coalition}, \sigma'_\Coalition)\). We have for all
indices \(i\) that
\(\move_i(\rho) = (\sigma_{-\Coalition}(\rho_{\le i}), \sigma'_\Coalition(\rho_{\le i}))\).
Therefore for all players \(A \not\in \Coalition\),
\((\move_i(\rho))_A = \sigma_A(\rho_{\le i})\). Then
\(\Dev(\move_i(\rho), \sigma_{\Agt}(\rho_{\le i})) \subseteq \Coalition\). Hence
\(\Dev(\rho, \sigma_{\Agt}) \subseteq \Coalition\).
\end{proof}

\begin{example}
  In the example of \Cref{14-fig:example1}, we consider again the strategies,
  such that for all histories $\pi$, $\sigma_{P_1}(\pi) = a$,
  $\sigma_{P_2}(v_0) = a$ and if $\pi \ne v_0$, $\sigma_{P_2}(\pi) = b$,
  and $\sigma_{P_3}(\pi) = b$.
  Then $\Dev(v_0 \cdot (a, a, b) \cdot v_2 \cdot (a, a, b) \cdot v_1 \cdot
  (a, b, a) \cdot v_2, \sigma_{\Agt})$ is the union of:
  \begin{itemize}
  \item $\Dev(\sigma_{\Agt}(v_0), (a, a, b)) = \Dev((a, a, b), (a, a, b)) = \varnothing$
  \item $\Dev(\sigma_{\Agt}(v_0 \cdot (a, a, b) \cdot v_2), (a, a, b)) =
    \Dev( (a, b, b), (a, a, b)) = \{P_2\}$
  \item $\Dev(\sigma_{\Agt}(v_0 \cdot (a, a, b) \cdot v_2 \cdot (a, a, b) \cdot v_1), (a, b, a)) =
    \Dev( (a, b, b), (a, b, a)) = \{P_3\}$.
  \end{itemize}
  We obtain
  $\Dev(v_0 \cdot (a, a, b) \cdot v_2 \cdot (a, a, b) \cdot v_1 \cdot
  (a, b, a) \cdot v_2, \sigma_{\Agt}) = \{ P_2, P_3\}$.
  This means that both $P_2$ and $P_3$ need to change their strategies
  from $\sigma_{\Agt}$ to obtain the given history.
\end{example}

Note that Nash equilibria are defined only with respect to deviations by
single players, that is, we require all players to achieve worse or equal
payoffs than the prescribed profile when they single-handedly change
strategies. Thus, only the outcomes with singleton deviator sets
will be of interest for us in the next section where we present the algorithm.

% Note that Nash equilibria are defined only with respect to deviations by
% single players.
% In fact, the case of deviators exceeding size 2 (two players
% simultaneously changing their strategies) should be of no matter to us.
% Indeed, we will see that when this happens, the two player game of our
% reduction can be stopped.



\subsection{Deviator Game}\label{deviator-game}
We now present an algorithm to reduce multiplayer games to two-player games
using the notion of deviators we just defined.
%We now use the notion of deviators to draw a link between multiplayer
%games and a two-player game.
Given a \emph{game}
\(\mathcal{G} = (\mathcal{A}, (\payoff_A)_{A \in \Agt})\),
%\todo{use $\mathcal{G}$ rather than $G$}
we define the deviator game, denoted \(\devg(\mathcal{G})\).
%\todo{would $\textsf{DevGame}(\mathcal{G})$ be clearer?}
Intuitively, in this game, Eve needs to play
according to an equilibrium, while Adam tries to find a profitable
deviation for any player. The vertices are \(V' = V \times 2^{\Agt}\),
where the second component, a subset of \(\Agt\), records the deviators
of the current history.

At each step, Eve chooses an action profile, and Adam chooses the move
that will apply. Adam can either respect Eve's choice, or pick a
different action profile in which case the deviators will be added to
the second component of the vertex. The game begins in
\((v_0 , \varnothing)\) and then proceeds as follows: from a vertex
\((v, D)\), Eve chooses an action profile \(a_{\Agt}\), and Adam chooses
a possibly different one \(a'_{\Agt}\). The next vertex is
\((\Delta(v, a'_{\Agt} ), D \cup \Dev(a_{\Agt} , a'_{\Agt} ))\).

\begin{example}
  An example of a partial construction of the deviator game for the
  example of \Cref{14-fig:example1}, is given in figure~\Cref{14-fig:ex-dev}.
  We cannot represent the full construction here, as there are 40 vertices.
\end{example}
\begin{figure}
  \begin{center}
    \begin{tikzpicture}
      \draw (0,0) node[draw, inner sep=7pt] (I) {$v_0, \varnothing$};
      \draw (80:4) node[draw, inner sep=7pt] (S10) {$v_1, \varnothing$};
      \draw (62:4.5) node[draw, inner sep=7pt] (S11) {$v_1, \{P_1\}$};
      \draw (44:5) node[draw, inner sep=7pt] (S12) {$v_1, \{P_2\}$};
      \draw (26:6) node[draw, inner sep=7pt] (S13) {$v_1, \{P_3\}$};
      \draw (8:7) node[draw, inner sep=7pt] (S14) {$v_1, \{P_1, P_2\}$};
      \draw (-8:6) node[draw, inner sep=7pt] (S15) {$v_1, \{P_1, P_3\}$};
      \draw (-26:5) node[draw, inner sep=7pt] (S16) {$v_1, \{P_2, p_3\}$};
      \draw (-44:4.5) node[draw, inner sep=7pt] (S17) {$v_1, \{P_1, P_2, P_3\}$};
      \draw (-76:4) node[draw, inner sep=7pt] (C2) {$v_2, \varnothing$};
      \draw (-88:4) node[below] {$\dots$};
      \draw[-latex'] (-1, 0) -- (I);
      \draw[-latex'] (I) -- node[sloped, text width=1cm]{$(a, b, a), (a, b, a)$  \\ $(b, a, a), (b, a, a)$} (S10);
      \draw[-latex'] (I) -- node[sloped, text width=1cm]{$(a, a, a), (b, a, a)$ \\ $(a, b, a), (b, a, b)$} (S11);
      \draw[-latex'] (I) -- node[sloped, text width=1cm]{$(a, a, a), (a, b, a)$ \\ $(b, b, a), (b, a, a)$} (S12);
      \draw[-latex'] (I) -- node[sloped, text width=1cm]{$(a, b, a), (a, b, b)$ \\ $(b, a, a), (b, a, b)$} (S13);
      \draw[-latex'] (I) -- node[sloped, text width=1cm]{$(a, b, a), (b, a, a)$ \\ $(b, a, a), (a, b, a)$} (S14);
      \draw[-latex'] (I) -- node[sloped, text width=1cm]{$(a, a, a), (b, a, b)$ \\ $(b, b, a), (a, b, b)$} (S15);
      \draw[-latex'] (I) -- node[sloped, text width=1cm]{$(a, a, a), (a, b, b)$ \\ $(b, b, a), (b, a, b)$} (S16);
      \draw[-latex'] (I) -- node[sloped, text width=1cm]{$(a, b, a), (b, a, b)$ \\ $(b, a, a), (a, b, b)$} (S17);
      \draw[-latex'] (I) -- node[sloped, text width=1cm]{$(a, a, a), (a, a, a)$ \\ $(b, b, b), (b, b, b)$} (C2);
    \end{tikzpicture}
    \caption{Example a deviator game construction.}
    \label{14-fig:ex-dev}
  \end{center}
\end{figure}

We define projections \(\proj_{V}\) and \(\proj_{\Dev}\) from \(V'\) to
% \todo{$\pi$ can be confused with plays, should we use $\textsf{proj}_V$?}
\(V\) and from \(V'\) to \(2^{\Agt}\) respectively, as well as
\(\proj_{\Act}\) from \(Act^{\Agt} \times Act^{\Agt}\) to \(Act^{\Agt}\) which
maps to the second component of the product, that is, Adam's action.

For a history or play \(\rho\), define \(\pi_{\Out}(\rho)\) as the play
\(\rho'\) for which, \(\rho'_i = \proj_V(\rho_i)\) and
\(\move_i(\rho') = \proj_{\Act}(\move_i(\rho))\) for all~$i$. This is thus the play
induced by Adam's actions.
Let us also denote~$\Dev(\rho) = \proj_{\Dev}(\last(\rho))$.

We can associate a strategy of Eve to each strategy profile
\(\sigma_{\Agt}\) such that she chooses the moves prescribed by
\(\sigma_{\Agt}\) at each history of \(\devg(\mathcal{G})\). Formally, we write
\(\kappa(\sigma_{\Agt})\) for the strategy defined by
\(\kappa(\sigma_{\Agt})(\pi) = \sigma_{\Agt}(\proj_{\Out}(\pi))\) for all histories~$\pi$.

The following lemma states the correctness of the construction of the
deviator game \(\devg(\mathcal{G})\), in the sense that it records the set of
deviators in the strategy profile suggested by Adam with respect to the
strategy profile suggested by Eve.

\begin{lemma}\label{14-prop:correctness-deviator-game}
  Let $\mathcal{G}$ be a multiplayer game, $v$ a vertex, \(\sigma_{\Agt}\) a
  strategy profile, and \(\sigma_{\exists} = \kappa(\sigma_{\Agt})\) the
  associated strategy in the deviator game.

  \begin{enumerate}
    \def\labelenumi{\arabic{enumi}.}
  \item
    If \(\rho \in \Out_{\devg(\mathcal{G})}((v,\emptyset),\sigma_\exists)\), then
    \(\Dev(\proj_{\Out}(\rho), \sigma_{\Agt} ) = \Dev(\rho)\).
  \item
    If \(\rho \in \Out_G(v)\) and for all index \(i\),
    \(\rho'_i = (\rho_i , \Dev(\rho_{\le i} , \sigma_{\Agt}))\) and
    \(\move_i(\rho') = (\sigma_{\Agt} (\rho_{\le i} ), \move_i(\rho))\), then
    \(\rho' \in \Out_{\devg(\mathcal{G})}((v,\emptyset), \sigma_\exists)\).
  \end{enumerate}
\end{lemma}
\begin{proof}
  We prove that for all $i$,
  \(\Dev(\proj_{\Out}(\rho_{\le i} , \sigma_{\Agt}) = \proj_{\Dev} (\rho_{\le i} )\),
which implies the property. The property holds for i = 0, since
initially both sets are empty. Assume now that it holds for \(i \ge 0\).
Then:
\begin{align*}
  \Dev(\proj_{\Out}(\rho_{\le i+1}) , \sigma_{\Agt} ) = & \Dev(\proj_{\Out}(\rho_{\le i}), \sigma_{\Agt} ) \cup \Dev(\sigma_{\Agt} (\proj_{\Out}(\rho_{\le i})), \proj_{\Act} (\move_{i+1} (\rho))) \\
  & \text{(by definition of deviators)}\\
  =& \Dev (\rho_{\le i} ) \cup \Dev(\sigma_{\Agt} (\proj_{\Out} (\rho_{\le i}), \proj_{\Act} (\move_{i+1} (\rho))) \\
  & \text{(by induction hypothesis)} \\
  = & \Dev (\rho_{\le i} ) \cup \Dev(\sigma_\exists (\rho_{\le i} ), \proj_{\Act} (\move_{i+1}(\rho))) \\
  & \text{(by definition of \(\sigma_\exists\) )}\\
  = & \Dev (\rho_{\le i} ) \cup \Dev(\move_{i+1}(\rho)) \\
  & \text{(by assumption \(\rho \in \Out_{\devg(\mathcal{G})} ((v,\emptyset), \sigma_\exists)\))}\\
  = & \Dev(\rho_{\le i+1} ) \\
  & \text{(by construction of \(\devg(\mathcal{G})\))} \\
\end{align*}

Which concludes the induction.

We now prove the second part. The property is shown by induction. It
holds for \(v_0\). Assume it is true up to index \(i>0\), then
\begin{align*}
\Delta'(\rho'_i , \sigma_\exists(\rho'_{\le i}), \move_i(\rho)) = &
\Delta'((\rho_i , \Dev(\rho_{\le i} , \sigma_{\Agt})), \sigma_{\exists} (\rho'_{\le i} ), \move_i(\rho))\\
&  \text{(by definition of \(\rho'\))} \\
= & \Delta(\rho_i , \move_i(\rho)), \Dev(\rho_{\le i}, \sigma_{\Agt}) \cup \Dev(\sigma_{\exists}(\rho'_{\le i}), \rho_{i+1} )) \\
&  \text{(by construction of \(\Delta'\) )}\\
= & (\rho_{i+1} , \Dev(\rho_{\le i}, \sigma_{\Agt} ) \cup \Dev(\sigma_{\exists}(\rho'_{\le i}), \rho_{i+1} )) \\
& \text{(since \(\rho\) is an outcome of the game)} \\
= & (\rho_{i+1} , \Dev(\rho_{\le i} , \sigma_{\Agt} ) \cup \Dev(\sigma_{\Agt} (\rho_{\le i}), \rho_{i+1} )) \\
& \text{(by construction of \(\sigma_\exists\))} \\
= & (\rho_{i+1} , \Dev(\rho_{\le i+1} , \sigma_{\Agt} ))\\
& \text{(by definition of deviators)} \\
= & \rho'_{i+1}. \\
\end{align*}
\end{proof}

%\subsection{Objectives in the Deviator
%Game}\label{objectives-in-the-deviator-game}

The objective of Eve in the deviator game is defined so that winning
strategies correspond to equilibria of the original game. First, as an
intermediary step, given coalition \(\Coalition\), player \(P\) and
bound \(b\), we will construct an objective stating that we can ensure
that the payoff of $P$ will not exceed $b$ even if players in $\Coalition$ change
their strategies.
Consider the following objective in \(\devg(\mathcal{G})\):
\[
  \Omega(\Coalition, P, b) = \{\rho \in \Out_{\devg(\mathcal{G})} \mid \Dev(\rho) \subseteq \Coalition \Rightarrow \payoff_P(\proj_{\Out}(\rho)) \le b\}.
\]
Intuitively, this says that if only players
in $\Coalition$ deviated from the strategy suggested by Eve, then the payoff
of $P$ is smaller than $b$.
We now show that a strategy ensuring bound \(b\) for the payoff of
$P$ against coalition \(\Coalition\) corresponds to a winning strategy for
\(\Omega(\Coalition, P, b)\) in the deviator game.

\begin{lemma} \label{14-lem:omegaCAg}
  Let \(\Coalition \subseteq \Agt\) be a coalition,
  \(\sigma_{\Agt}\) be a strategy profile, \(b \in \mathbb{R}\) a bound,
  and \(P\) a player. For all strategies \(\sigma'_\Coalition\), vertex~$v_0$,
  and coalition \Coalition, \(\payoff_P(\Out_{\devg(\mathcal{G})}(v_0, \sigma_{-\Coalition}, \sigma'_\Coalition)) \le b\) if, and
  only if, \(\kappa(\sigma_{\Agt})\) is winning in \(\devg(\mathcal{G})\) for objective
  \(\Omega(\Coalition, P, b)\).
\end{lemma}
\begin{proof} Let \(\rho\) be an outcome of
  \(\sigma_\exists=\kappa(\sigma_{\Agt}) \in \devg(\mathcal{G})\). By Lemma
  \ref{14-prop:correctness-deviator-game}, we have that
  \(\Dev(\rho) = \Dev(\proj_V(\rho),\sigma_{\Agt})\). By Lemma
  \ref{14-lem:deviator}, \(\proj_V(\rho)\) is the outcome of
  \((\sigma_{-\Dev(\rho)},\sigma'_{\Dev(\rho)})\) for some
  \(\sigma'_{\Dev(\rho)}\). If \(\Dev(\rho) \subseteq \Coalition\), then
  \(\payoff_P(\proj_V(\rho)) = \payoff_P(\sigma_{-\Coalition},\sigma_{\Coalition\setminus \Dev(\rho)}, \sigma'_{\Dev(\rho)}) = \payoff_P(\sigma_{-\Coalition},\sigma''_{\Coalition})\)
  where \(\sigma''_P = \sigma'_P\) if \(P \in \Dev(\rho)\) and \(\sigma_P\)
  otherwise. By hypothesis, this payoff is smaller than or equal to \(b\).
  This holds for
  all outcomes \(\rho\) of \(\sigma_\exists\), thus \(\sigma_\exists\) is
  a winning strategy for \(\Omega(\Coalition,P,b)\).

  For the other direction, assume \(\sigma_\exists = \kappa(\sigma_{\Agt})\)
  is a winning strategy in \(\devg(\mathcal{G})\) for \(\Omega(\Coalition,P,b)\). Let
  \(\sigma'_\Coalition\) be a strategy for \(\Coalition\) and \(\rho\) the outcome of
  \((\sigma'_{\Coalition},\sigma_{-{\Coalition}})\). By
  Lem.~\ref{14-lem:deviator},
  \(\Dev(\rho,\sigma_{\Agt}) \subseteq \Coalition\). By
  Lem.~\ref{14-prop:correctness-deviator-game},
  \(\rho'= (\rho_j, \Dev(\rho_{\le j},\sigma_{\Agt}))_{j\in \mathbb{N}}\) is
  an outcome of \(\sigma_\exists\). We have that
  \(\Dev(\rho') = \Dev(\rho,\sigma_{\Agt}) \subseteq \Coalition\). Since
  \(\sigma_\exists\) is winning, \(\rho\) is such that
  \(\payoff_P(\proj_V(\rho)) \le b\). Since
  \(\payoff_{P}(\proj_V(\rho')) = \payoff_{P}(\rho)\),
  this shows that for all strategies \(\sigma'_\Coalition\),
  \(\payoff_P(\sigma_{-\Coalition},\sigma'_\Coalition) \le b\).
\end{proof}

Now, Eve can show that there is a Nash equilibrium in a given game
by proving that whenever there is a single deviator,
the deviating player does not gain more than without the deviation,
while she does not have to prove anything on plays involving several
deviators.

\begin{theorem}\label{14-thm:dev-nash}
  Let \(\mathcal{G} = (\mathcal{A}, (\payoff_P)_{P \in \Agt})\) be a game, \(\sigma_{\Agt}\) a strategy
  profile in \(\mathcal{G}\), vertex~$v_0$, and \(F = (\payoff_P(\Out_{\mathcal{A}}(v_0,\sigma_{\Agt})))_{P\in \Agt}\)
  the payoff
  profile of \(\sigma_{\Agt}\) from~$v_0$. The strategy profile \(\sigma_{\Agt}\) is a
  Nash equilibrium if, and only if, strategy \(\kappa(\sigma_{\Agt})\) is
  winning in \(\devg(\mathcal{A})\) for the objective \(N(F)\) defined by:
  \[N(F) = \{\rho \mid |\Dev(\rho)| \ne 1\}
    \cup \bigcup_{P\in \Agt} \{\rho \mid \Dev(\rho) = \{P\}
    \land \payoff_P(\proj_{\Out}(\rho)) \le F_P\}.\]
\end{theorem}
\begin{proof} By \Cref{14-lem:omegaCAg}, \(\sigma_{\Agt}\) is a Nash
  equilibrium if, and only if, for each player \(P\),
%  \todo{What is this $\pi$? Check this proof}
  \(\kappa(\sigma_{\Agt})\) is winning for
  % \(\Omega(\{A\}, A, \payoff_A(\sigma_{\Agt})\).
  \(\Omega(\{P\}, P, F_P)\).
  So it is enough to show that for each player \(P\),
  \(\kappa(\sigma_{\Agt})\) is winning for
  \(\Omega(\{P\},P, F_P)\) if,
  % \(\Omega(\{A\},A, \payoff_A(\sigma_{\Agt})\) if,
  and only if, \(\kappa(\sigma_{\Agt})\) is winning for \(N(F)\).

  \textbf{Implication} Let \(\rho\) be an outcome of
  \(\kappa(\sigma_{\Agt})\).

\begin{itemize}

\item
  If \(|\Dev(\rho)| \ne 1\), then \(\rho\) is in \(N(F)\) by
  definition.
\item
  If \(|\Dev(\rho)| = 1\), then for \(\{P\} = \Dev(\rho)\),
  \(\payoff_P(\proj_{\Out}(\rho)) \leq F_P\) because
  \(\kappa(\sigma_{\Agt})\) is winning for
  \(\Omega(\Dev(\rho), P, F_P)\). Therefore \(\rho\) is
  in \(N(F)\).
\end{itemize}

This holds for all outcomes \(\rho\) of \(\kappa(\sigma_{\Agt})\) and shows
that \(\kappa(\sigma_{\Agt})\) is winning for \(N(F)\).

\textbf{Reverse implication} %Let \(P\) be such that strategy
Assume that
\(\kappa(\sigma_{\Agt})\) is winning for \(N(F)\). We now show that
\(\kappa(\sigma_{\Agt})\) is winning for
\(\Omega(\{P\},P, F_P)\) for each player \(P\). Let
\(\rho\) be an outcome of \(\kappa(\sigma_{\Agt})\), we have
\(\rho \in N(F)\). We show that \(\rho\) belongs to
\(\Omega(\{P\}, P, F_P)\):

\begin{itemize}

\item
  If \(\Dev(\rho) = \varnothing\) then \(\rho = \Out(v_0, \sigma_{\Agt})\) and
  \(\payoff_P(\rho) = F_P\), so \(\rho\) is in
  \(\Omega(\{P\},P, F_P)\)
\item
  If \(\Dev(\rho) \not\subseteq \{ P \}\), then
  \(\rho \in \Omega(\Coalition,P,F_P)\) by definition.
\item
  Otherwise \(\Dev(\rho) = \{P\}\). Since \(\rho \in N(F)\),
  \(\payoff_P(\rho) \le F_P\). Hence
  \(\rho \in \Omega(\Coalition,P,F_P)\).
\end{itemize}

This holds for all outcomes \(\rho\) of \(\kappa(\sigma_{\Agt})\) and shows
it is winning for \(\Omega(\{P\},P,F_P)\) for each
player \(P\in \Agt\), which shows that \(\sigma_{\Agt}\) is a Nash
equilibrium.
\end{proof}

%\fbox{Give an example}

\subsubsection{Algorithm for Parity
Objectives}\label{14-subsection:algorithm-for-parity-objectives}
We now focus on the case of Parity objectives.
Recall that
Each player $P$ has
a colouring function $c_P : V \rightarrow \mathbb{N}$, inducing the parity objective $\Omega_A$.
%
%\todo{$c_A$ should be $C_i$ and already given in the definition of game}
%the set of paths for which the maximum of the colours seen
%infinitely often is even.
%\todo{The book's definition is that the largest must be even}
Thus the payoff $\payoff_Ps$ assigns 1 to paths belonging to $\Omega_A$ and 0
to the others.

We now give an algorithm for the Nash equilibrium problem with parity
objectives. Given a payoff for each player \((F_P)_{P\in \Agt} \),
we can deduce from the previous theorem an algorithm that
constructs a Nash equilibrium if there exists one. We construct the
deviator game and note that we can reduce the number of vertices as
follows: since  \(\Dev(\rho_{\le k})\) is nondecreasing,
we know that \Eve wins whenever this set has at least two elements.
In the construction, states with at least two deviators can be replaced by a
sink vertex that is winning for \Eve. This means that the constructed
game has at most \(n \times (|\Agt| + 1) + 1\) states.

The objective can be expressed as a Parity condition in the following
way:

\begin{itemize}
\item
  for each vertex \(v' = (v, \{ P \})\), \(c'(v') = c_P(v) + 1\) if
  \(F_P = 0\) and \(2 \cdot \max_v c_P(v) \) otherwise;
\item
  for each vertex \(v' = (v, D)\) with \(|D| \ne 1\), \(c'(v') = 2 \cdot \max_v c_P(v)\)
  i.e.~it is winning for Eve.
\end{itemize}

Notice that the colouring function $c'$ inverts the parity
in the case where there is a single deviator who is losing in the
prescribed strategy profile (that is, $F_P=0$). In fact,
when~$F_P=1$, the player cannot obtain more since they are already winning
so the colour is set to $2\cdot \max_v c_P(v) $ which is winning for \Eve.

\begin{lemma}
  We have \(\maxinf(c'(\rho_i)) \in 2 \mathbb{N}\) if, and
  only if, \(\rho\in N(F)\),
  where $N(F)$ is as defined in \Cref{14-thm:dev-nash}.
\end{lemma}
\begin{proof} For the implication, we will prove the contrapositive.
Let \(\rho\) be a play not in \(N(F)\), then since the deviators can only
increase along a play, we have that \(\Dev(\rho) = \{ P \}\) for some
player \(P\) and \(\payoff_P(\rho) > F_P\). This means
\(F_P = 0\) and \(\maxinf(c_P(\rho_i)) \in 2 \mathbb{N}\). By
definition of \(c'\) this implies that
\(\maxinf(c'(\rho_i)) \in 2 \mathbb{N} + 1\) which proves the
implication.

For the other implication, let \(\rho\) be such that
\(\maxinf(c'(\rho_i)) \in 2 \mathbb{N} + 1\). By definition of
\(c'\) this means \(\rho\) contains infinitely many states of the form
\((v, \{P\})\) with \(F_P = 0\). Since the deviators only increase
along the run, there is a player \(P\) such that \(\rho\) stays in the
component \(V \times \{P\}\) after some index \(k\). Then for \(i\geq k\),
\(c'(\rho_i) = c_P(\rho_i)+1\), hence
\(\maxinf(c'(\rho_i)) = \maxinf(c_P(\rho_i)) + 1\).
Therefore \(\maxinf(c_P(\rho_i)) \in 2 \mathbb{N}\), which means
\(\payoff_P(\rho) = 1 > F_P\). By definition of \(N(F)\),
\(\rho\not\in N(F)\).
\end{proof}

Given that the size of the game is polynomial and that parity games can
be decided in quasipolynomial time (see \Cref{3-corollary:quasipoly}), the preceeding lemma
implies the following theorem.

\begin{theorem}
For parity games, there is a quasipolynomial algorithm to decide whether there is a Nash
equilibrium with a given payoff.
\end{theorem}

\subsection{Extensions of Nash
Equilibria}\label{14-subsection:extensions-of-nash-equilibria}

\subsubsection{Subgame Perfect
Equilibria}\label{14-subsection:subgame-perfect-equilibria}

Nash equilibria present the disadvantage that once a player has deviated,
the others will try to punish him, forgetting everything about their own
objectives.
If we were to observe the game after this point of
deviation, it would not look like the players are playing rationally and
in fact it would not correspond to a Nash equilibrium. The concept of
\emph{subgame perfect equilibria} refines the concept of Nash
equilibrium by imposing that at each step of the history, the strategy
behaves like Nash equilibrium if we were to start the game now. Formally,
let us write \(\sigma_P \circ h\) the strategy which maps all histories
\(h'\) to \(\sigma_P(h \cdot h')\), that is the strategy that behave
like \(\sigma_P\) after \(h\). Then \((\sigma_P)_{P\in \Agt}\) is a
\emph{subgame perfect equilibrium} if for all histories \(h\),
\((\sigma_P \circ h)_{P \in \Agt}\) is a Nash equilibrium.

Imposing such a strong restriction is justified by the fact that subgame
perfect Nash equilibria exist for a large class of games. In particular
subgame perfect equilibria always exist in turn-based games with
reachability objectives.

\begin{example}
  Consider the example of \Cref{14-fig:ex-subgame}.
  There is a Nash equilibria whose outcome goes through states
  $v_0 \rightarrow v_1 \rightarrow \Omega_1$.
  In this equilibrium, Player $1$ should play $b$ in $v_2$,
  so that the best response of Player~$2$ is to play~$a$ at~$v_0$.
  Intuitively, player $1$ is threatening player $2$, to make them
  both lose from $v_2$, but this threat is not credible,
  and the profile is not a subgame perfect equilibrium.
  In fact, once $v_2$ is reached it is better for Player $1$
  to play $a$ so it is unlikely that the player will execute the said threat.
  The only subgame perfect equilibrium of this game ends in 
  the vertex satisfying both
  $\Omega_1$ and $\Omega_2$.
\end{example}
\begin{figure}
\begin{center}  
\begin{tikzpicture}
  \draw (0,0) node[draw, inner sep=7pt] (V0) {$v_0$};
  \draw (3,1) node[draw, inner sep=7pt] (V1) {$v_1$};
  \draw (3,-1) node[draw, inner sep=7pt] (V2) {$v_2$};
  \draw (6,2) node[draw, inner sep=7pt] (V3) {$\Omega_1$};
  \draw (6,0.6) node[draw, inner sep=7pt] (V4) {$\Omega_2$};
  \draw (6,-0.6) node[draw, inner sep=7pt] (V5) {$\Omega_1, \Omega_2$};
  \draw (6,-2) node[draw, inner sep=7pt] (V6) {$\varnothing$};

  \draw[-latex'] (-1, 0) -- (V0);
  \draw[-latex'] (V0) -- node[sloped, text width=1cm, above]{$(\ast, a)$} (V1);
  \draw[-latex'] (V0) -- node[sloped, text width=1cm, above]{$(\ast, b)$} (V2);
  \draw[-latex'] (V1) -- node[sloped, text width=1cm, above]{$(a, \ast)$} (V3);
  \draw[-latex'] (V1) -- node[sloped, text width=1cm, above]{$(b, \ast)$} (V4);
  \draw[-latex'] (V2) -- node[sloped, text width=1cm, above]{$(a, \ast)$} (V5);
  \draw[-latex'] (V2) -- node[sloped, text width=1cm, above]{$(b, \ast)$} (V6);
  \draw[-latex'] (V3) .. controls +(2,1) and +(2,-1) .. (V3);
  \draw[-latex'] (V4) .. controls +(2,1) and +(2,-1) .. (V4);
  \draw[-latex'] (V5) .. controls +(2,1) and +(2,-1) .. (V5);
  \draw[-latex'] (V6) .. controls +(2,1) and +(2,-1) .. (V6);
\end{tikzpicture}
\caption{Two-player game with reachability objectives. The goal of
  player 1 is to reach a state labeled with $\Omega_1$ and that of
  player 2 is to reach a state labeled with $\Omega_2$. }
\label{14-fig:ex-subgame}
\end{center}
\end{figure}

\subsubsection{Robust equilibria}\label{robust-equilibria.}
The notion of robust equilibria refines Nash equilibria in two ways:

\begin{itemize}

\item
  a robust equilibrium is \emph{resilient}, i.e.~when a small coalition
  change its strategy, none of the players of the coalition improves their
  payoff;
\item
  it is \emph{immune}, i.e.~when a small coalition changes its strategy,
  it does not decrease the payoffs of the non-deviating players.
\end{itemize}

The size of small coalitions is determined by parameter \(k\) for
resilience and \(t\) for immunity. When a strategy is both
\(k\)-resilient and \(t\)-immune, it is called a \((k,t)\)-robust
equilibrium.

The motivation behind this concept is to address these two weaknesses of
Nash equilibra:

\begin{itemize}

\item
  There is no guarantee on payoffs when two (or more) players deviate together.
  Such a situation can occur in networks if the same person controls several devices
  (a laptop and a phone for instance) and can then coordinate their
  behaviours. In this case, the devices would be considered as different
  players and Nash equilibria can offer no guarantee.
\item
  When a deviation occurs, the strategies of the equilibrium can punish
  the deviating user without any regard for the payoffs of the others. This
  can result in a situation where, because of a faulty device, nobody
  can use the protocol anymore.
\end{itemize}

By comparison, finding resilient equilibria with \(k>1\), 
ensures that clients have no interest in forming coalitions (up
to size \(k\)), and finding immune equilibria with \(t>0\)
ensures that other clients will not suffer from some players (up to
\(t\)) behaving differently from what was expected.

The deviator construction can be reused for finding such equilibria. We
only need to adapt the objectives. Given a game \(G=(\mathcal{A}, (\payoff_P)_{P \in \Agt})\), a
strategy profile \(\sigma_{\Agt}\), and parameters \(k\), \(t\), we have

\begin{itemize}

\item
  The strategy profile \(\sigma_{\Agt}\) is \(k\)-resilient if, and only
  if, strategy \(\kappa(\sigma_{\Agt})\) is winning in \(\devg(\mathcal{A})\) for the
  \emph{resilience objective} \(\mathcal{R}(k,F)\) where
  \(F = (\payoff_P(\Out_{\mathcal{A}}(v_0, \sigma_{\Agt})))_{P \in \Agt}\) is the payoff profile of
  \(\sigma_{\Agt}\) and \(\mathcal{R}(k,F)\) is defined by:
  \[
    \begin{array}{ll}
      \mathcal{R}(k,F) = & \{ \rho \in \Out_{\devg(\mathcal{A})}\mid ~ |\Dev(\rho)| > k \} \\
                          &\cup
                            \{ \rho  \in \Out_{\devg(\mathcal{A})} \mid ~ |\Dev(\rho)| = k \land \forall P \in \Dev(\rho).\ \payoff_{P}(\proj_{\Out}(\rho)) \le F_P\} \\
                          & \cup \{ \rho  \in \Out_{\devg(\mathcal{A})}\mid ~ |\Dev(\rho)| < k \land \forall P \in \Agt.\ \payoff_{P}(\proj_{\Out}(\rho)) \le F_P\}.
    \end{array}
  \]

\item
  The strategy profile \(\sigma_{\Agt}\) is \(t\)-immune if, and only if,
  strategy \(\kappa(\sigma_{\Agt})\) is winning for the \emph{immunity
  objective} \(\mathcal{I}(t,F)\) where
  \(F = (\payoff(\Out_{\mathcal{A}}(v_0, \sigma_{\Agt})))_{P \in \Agt}\) is the payoff profile of
  \(\sigma_{\Agt}\) and \(\mathcal{I}(t,F)\) is defined by:
  \[
    \begin{array}{ll}
    \mathcal{I}(t,F) = & \{ \rho  \in \Out_{\devg(\mathcal{A})} \mid |\Dev(\rho)| > t \}  \\
      & \cup \{ \rho  \in \Out_{\devg(\mathcal{A})} \mid ~ \forall P \in \Agt \setminus \Dev(\rho).\  F_P \le \payoff_{P}(\proj_{\Out}(\rho)) \}.
    \end{array}
  \]
\item
  The strategy profile \(\sigma_{\Agt}\) is a \((k,t)\)-robust profile in
  \(G\) if, and only if, \(\kappa(\sigma_{\Agt})\) is winning for the
  \emph{robustness objective}
  \[
        \mathcal{R}(k,t,F)= \mathcal{R}(k,F) \cap \mathcal{I}(t,F),
        \] where
        \(F = (\payoff_P(\Out_{\mathcal{A}}(v_0, \sigma_{\Agt})))_{P \in \Agt}\) is the payoff profile of
        \(\sigma_{\Agt}\).
\end{itemize}

We omit the proof and encourage the reader to do it by themselves.

\subsubsection{Extension to games with hidden
actions}\label{extension-to-games-with-hidden-actions}
In most practical cases, players only have a partial view of the
state of the system; so they may not be able, for instance, to
detect a deviating player immediately.
Studying equilibria in general imperfect information as in \Cref{chap:signal} 
would be well adapted in such situations.
Unfortunately, these games are too powerful in general since
the existence of Nash equilibria is undecidable in this case.

Nevertheless, the problem is decidable for a restricted form of imperfect
information where the players observe the visited states
but do not see the played actions; thus the actions are \emph{hidden}.
%In this restricted setting, one recovers decidability.

We will thus consider strategies defined as functions from \(V^*\) to \(Act\),
which represents the fact that players' decision can only depend on observed
sequence of states but not on other players' actions.

In this case, deviators cannot be defined as obviously as before, as it may
not always be possible to identify one unique deviator responsible for a
deviation. The construction will thus maintain a set of \emph{suspects}, those players that might have been
responsible for the observed deviation.
Formally, suspects for an edge \((v, v')\) with respect to a move
\((a_P)_{P\in \Agt}\) are players \(P\) such that there is \(a'_P\) and
\(\Delta(a'_P, a_{-P}) = (v, v')\). Rather than computing the
union of deviators along a history, we now consider the intersection of
suspects. That is, if at vertex~$v$, the suspect set is~$S$, and the strategy profile is~$\sigma_\Agt$,
and if the next vertex is~$v'$, then
the suspect set becomes~$S \cap \{ P \in \Agt \mid \exists a'_P, \Delta(a'_P, a_{-P}) = (v, v')\}$.

%Here, if there is no deviation at all, then all players are suspects, and we assume only one player
%will deviate (this is enough for Nash equilibria).

The \emph{suspect game} can be defined just like the deviator game by replacing the deviators component
by the suspects component.
The objective for Eve is that no suspect player improves their payoff. In fact, in
case of deviation, we know that the deviator belongs to the set of suspects although
we cannot know which one has deviated for sure so Eve must ensure this for all suspects.

\begin{example}
  Consider the example of figure~\ref{fig:hidden}.
  If actions were visible there would be an equilibrium ending in the
  state labeled with $\Omega_3$: player 3 simply has to punish the player
  who would deviate from this path.
  But if we now consider hidden actions, in case of a deviation, player 3
  would observe that the play went arrives in $v_1$ instead of $v_2$
  and both Player 1 and Player 2 are suspects.
  Since Player 3 cannot punish both players at the same time, there is no
  Nash equilibrium ending satisfying $\Omega_3$.
\end{example}
\begin{figure}
\begin{center}
\begin{tikzpicture}
  \draw (0,0) node[draw, inner sep=7pt] (V0) {$v_0$};
  \draw (3,1) node[draw, inner sep=7pt] (V1) {$v_1$};
  \draw (3,-1) node[draw, inner sep=7pt] (V2) {$v_2$};
  \draw (6,2) node[draw, inner sep=7pt] (V3) {$\Omega_1$};
  \draw (6,0.6) node[draw, inner sep=7pt] (V4) {$\Omega_2$};
  \draw (6,-0.6) node[draw, inner sep=7pt] (V5) {$\Omega_3$};
  \draw (6,-2) node[draw, inner sep=7pt] (V6) {$\varnothing$};

  \draw[-latex'] (-1, 0) -- (V0);
  \draw[-latex'] (V0) -- node[sloped, text width=1cm]{$(a, a, \ast)$\\$(b, b, \ast)$} (V1);
  \draw[-latex'] (V0) -- node[sloped, text width=1cm]{$(a, b, \ast)$ \\ $(b, a, \ast)$} (V2);
  \draw[-latex'] (V1) -- node[sloped, text width=1cm, above]{$(\ast, \ast, a)$} (V3);
  \draw[-latex'] (V1) -- node[sloped, text width=1cm, above]{$(\ast, \ast, b)$} (V4);
  \draw[-latex'] (V2) -- node[sloped, text width=1cm, above]{$(\ast, \ast, a)$} (V5);
  \draw[-latex'] (V2) -- node[sloped, text width=1cm, above]{$(\ast, \ast, b)$} (V6);
  \draw[-latex'] (V3) .. controls +(2,1) and +(2,-1) .. (V3);
  \draw[-latex'] (V4) .. controls +(2,1) and +(2,-1) .. (V4);
  \draw[-latex'] (V5) .. controls +(2,1) and +(2,-1) .. (V5);
  \draw[-latex'] (V6) .. controls +(2,1) and +(2,-1) .. (V6);
k\end{tikzpicture}
\caption{Three-player game with hidden actions. The goal of
  player $i$ is to reach a state labeled with $\Omega_i$.}
\label{fig:hidden}
\end{center}
\end{figure}

\section{Admissible strategies}\label{admissible-strategies}
Nash equilibria and their variants seen so far describe stable situations
from which players have no incentive to deviate. This however is of limited use
in some situations. First, the stability relies on the fact that all players are informed
of the strategy profile to be played; that is, some central authority needs to publicly
announce the strategies for all players. Second, each equilibrium describes a single possible
situation. If there are several equilibria, it is not clear which one is to be chosen.

Rather than concentrating on particular equilibria, game theorists have studied
reasonings players may follow in order to exclude some strategies that are necessarily worse than others.
These worse strategies are called \emph{dominated}.
By formalizing dominated and non-dominated strategies for a given player, one can then predict the behavior of rational players
since such a player would never use a dominated strategy but rather always pick the best one available.

In this section, we will formalize dominated strategies and show how these can be computed in games.
We then briefly show that this reasoning can be repeated, and present the iterated elimination of dominated strategies.

\subsection{Definition}\label{definition-1}
The notion of \emph{dominance} is used to compare strategies 
with respect to payoffs they yield against the rest of the players' strategies.
Consider the example of \ref{14-tab:normal-adm}.
Given a strategy of the second player, playing $B$ is always at least as good
as playing $A$ for the first player.
In fact, again~$C$, $B$ yields a payoff of~$4$ which is better than $3$, the payoff of~$A$;
and against $D$, both yield~$0$.
The strategy $B$ is said to dominate $A$. Intuitively, 
$B$ is either better or as good as~$A$ \emph{in all situations}, so 
playing~$B$ is the rational choice for Player~$1$.

Furthermore, by this analysis, Player~$2$ knows that Player~$1$ will player~$B$.
Given this information, the best response of Player~$1$ is to play~$C$.
By \emph{iterated elimination}, we established that
%We say that the game is solvable by \emph{iterated elimination},
$(B, C)$ should be the only strategy profile to be played by players following this reasoning.

\begin{table}
  \caption{A normal form game solvable by iterated elimination.}
  \label{14-tab:normal-adm}
  \begin{center}
    \begin{tabular}[c]{|@{\hspace{1em}}l@{\hspace{1em}}|@{\hspace{1em}}c@{\hspace{1em}}c@{\hspace{1em}}|}
      \hline
      & A & B \\
      \hline
      C & 3, 3 & 4, 1\\
      D & 0 , 4 & 0, 0\\
      \hline
    \end{tabular}
  \end{center}
\end{table}

Let us formalize this notion.

\begin{definition}[Dominance]
Let \(S \subseteq \mathcal{S}^{\Agt}\) be a set of the form
\(S = S_1 \times S_2 \times \cdots \times S_n\) which we will call a
rectangular set. Let \(\sigma,\sigma' \in S_i\). Strategy \(\sigma\)
\emph{very weakly dominates} strategy \(\sigma'\) with respect to \(S\),
written \(\sigma_i \ge_S \sigma'_i\), if from all vertices \(v_0\):

\[
  \forall \sigma_{-i} \in S_{-i}, \payoff_i(\Out(v_0,\sigma'_i,\sigma_{-i}))
  \ge
  \payoff_i(\Out(v_0, \sigma_i, \sigma_{-i})).
\]

Strategy \(\sigma_i\) \emph{weakly dominates} strategy \(\sigma'_i\)
in \(S\), written \(\sigma >_S \sigma'\), if
\(\sigma \ge_S \sigma'\) and \(\neg(\sigma' \le_S \sigma)\).
%A strategy
%\(\sigma \in S_i\) is weakly dominated in \(S\) if there exists
%\(\sigma' \in S_i\) such that \(\sigma' >_S \sigma\).
A strategy that is
not weakly dominated in \(S\) is \emph{admissible} in \(S\). The
subscripts on \(\ge_S\) and \(>_S\) are omitted when the sets of
strategies are clear from the context.
\end{definition}

Algorithms rely on the notion of \emph{optimistic} and
\emph{pessimistic value} of a history. The pessimistic value is the
maximum payoff that a player can ensure in the worst case within the strategy set~$S$.
The optimistic value is
the best the player can achieve with the help of other players, given the strategy set~$S$.


\begin{definition}[Values] The \emph{pessimistic value} of a strategy
\(\sigma_i\) for a history \(h\) with respect to a rectangular set of
strategies \(S\), is

\begin{itemize}
\item
  \(\pes_i(S,h,\sigma_i) = \inf_{\sigma_{-i} \in S_{-i}} \payoff_{i}(h \cdot \Out(\last(h), \sigma_i,\sigma_{-i})).\)
\end{itemize}

The \emph{pessimistic value of a history} \(h\) for \(A_i\) with respect
to a rectangular set of strategies \(S\) is given by:

\begin{itemize}

\item
  \(\pes_i(S,h) = \sup_{\sigma_i \in S_i} \pes_i(S,s,\sigma_i).\)
\end{itemize}

The \emph{optimistic value} of a strategy \(\sigma_i\) for a history
\(h\) with respect to a rectangular set of strategies \(S\) is given by:

\begin{itemize}

\item
  \(\opt_i(S,h,\sigma_i) = \sup_{\sigma_{-i} \in S_{-i}} \payoff_i (h_{\le |h|-2} \cdot \Out(\last(h), \sigma_i,\sigma_{-i})).\)
\end{itemize}

The \emph{optimistic value} of a history \(h\) for \(A_{i}\) with
respect to a rectangular set of strategies \(S\) is given by:

\begin{itemize}

\item
  \(\opt_i(S,h) = \sup_{\sigma_i \in S_i} \payoff_{i}(\opt_i(S,h,\sigma_i))\)
\end{itemize}
\end{definition}

We will first consider the case where~$S$ is the set of all strategies,
and omit~$S$ in the above notations.

\subsection{Simple Safety games}\label{simple-safety-games}

\begin{figure}
\begin{center}  
\begin{tikzpicture}
\draw (0,0) node[draw, inner sep=2pt, circle] (I) {$v_0$};
\draw (I.-90) node[below] {};
\draw (4,1) node[draw, circle, inner sep=2pt] (C1) {$v_1$};
\draw (C1.-100) node[below] {};
\draw (4,-1) node[draw, circle, inner sep=2pt, circle] (C2) {$v_2$};
%\draw (C2) node[draw,circle, inner sep=2pt] {$v_2$};
%\draw (C2.-90) node[below] {};
\draw (8, 1) node[draw, circle, inner sep=2pt] (S1) {$v_3$};
\draw (S1.-90) node[below] {$-1, -1, 0$};
\draw (8,-1) node[draw, circle, inner sep=2pt] (S2) {$v_4$};
\draw (S2.-90) node[below] {$-1, 0, -1$};
\draw[-latex'] (-1, 0) -- (I);
\draw[-latex'] (I) -- node[above]{$(*,a,*)$} (C1);
\draw[-latex'] (I) -- node[below]{$(*,b,*)$} (C2);
\draw[-latex'] (C1) --node[left]{$(a,*,*)$} (C2);
\draw[-latex'] (C1) -- node[above]{$(b,*,*)$} (S1);
\draw[-latex'] (C2) -- node[above]{$(*,*,a)$} (S2);
\draw[-latex', rounded corners] (C2) -- +(1,1) -- node[right]{$(*,*,b)$}(C1);
\end{tikzpicture}
\caption{Example of a three-player turn-based simple safety game.
%  Square represent vertices controlled by player 1, circle those of player 2, diamond those of player 3. 
Numbers below states describe the safety objective, for instance $-1, -1, 0$ is losing for player 1 and 2.
}
\label{14-fig:simple-safety}
\end{center}
\end{figure}

Simple safety games, are safety games in which there are no transitions
from losing vertices to non-losing ones. Restricting to this particular
class of game makes the problem simpler because the objective becomes
prefix independent.
Any safety game can be converted to an equivalent simple safety game
by encoding in the states which players have visited so far a losing
state. Note that this translation can be exponential in the number of
players.

For simple safety games, the pessimistic and optimistic values do not depend
on the full history but only on the last state: for all histories
\(\pes_i(h) = \pes_i(\last(h))\) and \(\opt_i(h) = \opt_i(\last(h))\).

Note that in safety games (and any qualitative games) values can be only
1 (for winning) and 0 (for losing) and since the pessimistic value is
always less than the optimistic one, the pair \((\pes_i, \opt_i)\) can
only take three values: \((0, 0)\), \((0, 1)\) and \((1, 1)\).

Intuitively, players should avoid decreasing this pair of values if they can.
In fact, the characterization of admissible strategies we give below will be based on this
simple observation.

\begin{example}
  An example of a simple safety game is given in \Cref{14-fig:simple-safety}.
  In this game, player 1 controls $v_1$ where its optimistic
  value is $0$, as it is possible that the outcome will never reach
  $v_3$ or $v_4$.
  However $v_3$ has optimistic value $-1$ for player 1, as it is a losing
  state for player 1.
  Going from $v_1$ to $v_3$ is a bad choice for player 1, and
  it is indeed dominated by the strategy that would always choose to go
  to $v_2$.
  In state $v_3$, player 3 has pessimistic value $0$ since it can ensure not
  visiting $v_4$.
  The winning strategy for player 3 which is to always go to $v_1$ is also
  the only non-dominated strategy.
  For player 2, from state $v_0$, both choices lead to a state with values
  $(-1, 0)$ so no choice is particularly better and both strategies are
  non-dominated.
\end{example}

\begin{definition} Let \(D_i\) be the set of edges
\((v,v') \in E\) such that \(v \in V_i\) %is controlled by player
%\(i\) and
\(\pes_i(v) > \pes_i(v')\) or \(\opt_i(v) > \opt_i(v')\). These
are called \emph{dominated edges}.
\end{definition}

\begin{theorem}[Characterisation of Admissible Strategies]
  \label{14-thm:adm}
Admissible
strategies for player \(A_i\) are the strategies that never take actions
in \(D_i\).
\end{theorem}
\begin{proof} We show that if \(A_i\) plays an admissible strategy
\(\sigma_i\) then the value cannot decrease on a transition controled by
\(A_i\). Let \(\rho \in \Out(\sigma_i,\sigma_{-i})\), and \(k\) an index
such that \(\rho_k \in V_i\). Let \((v, v') = \sigma_i(\rho_{\le k})\):

\begin{itemize}

\item
  If \(\pes_i(\rho_k)=1\), then \(\sigma_i\) has to be winning against
  all strategies \(\sigma_{-i}\) of \(A_{-i}\), otherwise it would be
  weakly dominated by such a strategy. Since there is no such strategy
  from a state with value \(\pes_i \leq 0\), we must have \(\pes_i(v')=1\).
\item
  If \(\opt_i(\rho_k) = 1\), then, by definition, there is a profile \(\sigma'\) such that
  \(\rho = \payoff(\Out(v, \sigma')) = 1\).
  Assume that $\opt_i(v')=0$. Then~$\sigma_i$ is dominated by the strategy
  $\sigma_i''$ obtained from~$\sigma_i$ by making it switch to~$\sigma'$ at~$v$.
  In fact, $\sigma_i$ is losing against all strategies of~$-i$, while~$\sigma_i''$
  is winning at least against~$\sigma_{-i}'$.
  %Note that
%  \(h \cdot v \cdot v'\) is a prefix of \(\rho\). If \(\opt_i(s')=0\),
%  there can be no such profile, thus \(\opt_i(s') = 1\).
\item
  If $\pes_i(v)=0$ or $\opt_i(s) = 0$, then the value cannot decrease further.
\end{itemize}

In the other direction, let \(\sigma_i,\sigma_i'\) be two strategies of
player \(A_i\) and assume \(\sigma_i' >_S \sigma_i\). We will prove
\(\sigma_i\) takes a transition in \(D_i\) at some point.

Let us fix some objects before developing the proof. There is a vertex
\(v\) and strategy profile \(\sigma_{-i} \in S_{-i}\) such that
\(\payoff(\Out(v,\sigma_i',\sigma_{-i}))=1 \wedge \payoff(\Out(v,\sigma_i,\sigma_{-i})=0\).
Let \(\rho = \Out(v,\sigma_i,\sigma_{-i})\) and
\(\rho' = \Out(v,\sigma_i',\sigma_{-i})\). Consider the first position
where these runs differ: write \(\rho = w \cdot s' \cdot s_2 \cdot w'\)
and \(\rho' = w \cdot s' \cdot s_1 \cdot w''\).

The following are simple facts that can be seen easily:

\begin{itemize}
\item
  \(s' \in V_i\), because the strategy of the other players
  are identical in the two runs.
\item
  \(\opt_i(s_1) = 1\) because
  \(\payoff(\Out(v,\sigma_i',\sigma_{-i}))=1\)
\item
  \(\pes_i(s_2) = 0\) because
  \(\payoff(\Out(v,\sigma_i,\sigma_{-i}))=0\)
\end{itemize}

If \(\opt_i(s_2) = 0\) or \(\pes_i(s_1) = 1\) then
\(s' \rightarrow s_2 \in D_i\) so \(\sigma_i\) takes a transition of
\(D_i\). The remaining case to complete the proof is \(\opt_i(s_2) = 1\)
and \(\pes_i(s_1) = 0\).
Let us assume that~$\sigma_i$ does not take any edges from~$D_i$.
We will show that there is
a strategy for~$-i$ against which $\sigma_i$ wins and~$\sigma_i'$ loses,
which contradicts the hypothesis that~$\sigma_i'$ weakly dominates~$\sigma_i$.

We first construct a profile \(\sigma_{-i}^2 \in S_{-i}\) such that
\(\payoff(\Out(s_2,\sigma_i,\sigma_{-i}^2))=1\).
%For any history  \(h\) with \(\last(h)\notin V_i\), 
%we must have 
%\(\sigma^2_{-i} \in S_{-i}\), 
%\(\opt_i(\sigma^2_{-i}(h)) = 0\) then
%\(\opt_i(last(h))=0\). Thus 
Strategy \(\sigma_{-i}^2\in S_{-i}\) never decreases
the optimistic value from \(1\) to \(0\) since the optimistic value is nonincreasing.
By assumption, \(\sigma_i\)
itself does not decrease the value of \(A_i\) because it does not take
transitions of~\(D_i\). So the outcome of \((\sigma_i,\sigma_{-i}^2)\)
never reaches a state of optimistic value \(0\). Hence it never reaches
a state in \(Bad_i\) and therefore it is winning for \(A_i\).

Let us now consider a profile \(\sigma_{-i}^1 \in S_{-i}\) such that
\(\payoff(\sigma_i',\sigma_{-i}^1)=0\) from \(s_1\). Such a strategy exists
because
\(\pes_i(s_1) = 0\), so~$\sigma_i'$ is not a winning strategy.
Then there exists a
strategy profile \(\sigma_{-i}^1\) such that \(\sigma_i'\) loses from
\(s_1\).

Now consider strategy profile \(\sigma_{-i}'\) that plays like
\(\sigma_{-i}\) if the play does not start with \(w\);
and otherwise switches to
\(\sigma_{-i}^1\) at history \(ws_1\) and to \(\sigma_{-i}^2\) at history \(ws_2\).
Formally, given a history \(h\), \(\sigma_{-i}'(h) =\)

\begin{itemize}
\item
  \(\sigma_{-i}^1( h')\) if \(w \cdot s_1\) is a prefix of \(h\) and
  \(w \cdot s_1 \cdot h' = h\)
\item
  \(\sigma_{-i}^2( h')\) if \(w \cdot s_2\) is a prefix of \(h\) and
  \(w \cdot s_2 \cdot h' = h\)
\item
  \(\sigma_{-i}(h)\) otherwise
\end{itemize}

Clearly we have
\(\payoff_i(\Out_s(\sigma_i,\sigma_{-i}'))= 1 \wedge \payoff_i(\Out_s(\sigma_i',\sigma_{-i}')) = 0\),
which contradicts \(\sigma_i' \ge_S \sigma_i\).
\end{proof}

\subsection{Parity games}\label{parity-games}

The caracterization given for simple safety game is not enough for
parity objectives, as we will see in the following example.

\begin{figure}
  \begin{center}
    \begin{tikzpicture}
      \draw (0,0) node[draw, circle, inner sep=4pt] (I) {$v_0$};
      \draw (I.-90) node[below] {};
      \draw (4,0) node[draw, inner sep=4pt, circle] (C1) {$v_1$};
      \draw (C1.-100) node[below] {};
      \draw (8,0) node[draw, circle, inner sep=4pt] (C2) {$\Omega_1$};
      \draw (I.-90) node[below] {$1$};
      \draw (C1.-90) node[below] {$1$};
      \draw (C2.-90) node[below] {$2$};
      \draw[-latex'] (-1, 0) -- (I);
      \draw[-latex'] (I) -- node[above]{$(a,*)$} (C1);
      \draw[-latex'] (I) ..controls +(1,2) and +(-1,2)  .. node[above]{$(b,*)$}  (I);
      \draw[-latex'] (C1) -- node[above]{$(*,a)$}(C2);
      \draw[-latex'] (C1) .. controls +(-2, -2) ..  node[above]{$(*,b)$} (I);
      \draw[-latex', rounded corners] (C2)..controls +(1,2) and +(-1,2) .. node[above]{$(a,*)$}(C2);
    \end{tikzpicture}
    \caption{Parity game where the objective for player 1 is to visit
      $\Omega_1$ infinitely often. Player 1 controls square vertices and player 2
      round vertices.}
    \label{fig:adm-parity}
  \end{center}
\end{figure}

\begin{example}
  Consider the example in figure~\ref{fig:adm-parity}.
  In this example, although the strategy that always stays in $v_0$
  does not decrease the value of player 1, it is dominated because
  it has no chance of winning.
  By contrast the strategy that always go to $v_1$ has a chance of
  being helped by player 2 and actualy reaching $\Omega_1$ it therefore
  dominates the first strategy.
\end{example}

However the fact that an admissible strategy should not decrease its own
value still holds. Assuming strategy \(\sigma_i\) of player \(P_i\) does
not decrease its own value, we can classify its outcome in three
categories according to their ultimate values.

\begin{itemize}
\item
  either ultimately \(\opt_i = 0\), in which case all strategies are
  losing, and thus any strategy is admissible
\item
  or ultimately \(\pes_i = 1\), in which case admissible strategies are
  exactly the winning ones
\item
  or ultimately \(\pes_i = 0\) and \(\opt_i = 1\). We will focus on this
case which is more involved.
\end{itemize}

From a state of value \(0\), an admissible strategy of \(P_i\) should
allow a winning play for \(P_i\) with the help of other players.

We write \(H_i\) for set of vertices \(v\) controlled by a player
\(P_j\ne P_i\) that have at least two successors of optimistic value
\(1\). Formally, the \emph{help-states} $H_i$ of player \(P_i\) are defined
as:
\[
 \bigcup_{P_j\in \Agt \setminus\{i\}} \left\{ s \in V_j \mid \exists s',s'',\ s' \neq s'' \wedge\ s\rightarrow s' \land s \rightarrow s'' \wedge\ \opt_i(s') = 1 \wedge\ \opt_i(s'') = 1 \right\}.
\]

Intuitively, admissible strategies in the case satisfying \(\pes_i = 0\) and \(\opt_i = 1\)
are those that visit infinitely often help states. In fact, letting other players make choices
means that the player is allowing the possibility of them helping to achieve the objective.
More precisely, we have the following property whose proof is omitted.
\begin{lemma}
Let \(v\in V\), \(P_i\in \Agt\) and \(\rho\) a play be
such that \(\exists^\infty k. \opt_{i}(\rho_k) = 1\). There exists
\(\sigma_i\) admissible such that \(\rho \in \Out(v, \sigma_i)\) if, and
only if, \(\payoff_i(\rho) = 1\) or
\(\exists^\infty k. \rho_k \in H_i\).
\end{lemma}

\subsection{Iterated elimination}\label{iterated-elimination}

Once each player is restricted to use admissible strategies, they can
further refine their choices knowing that other players will not be using dominated strategies.
We already saw this in the example of \Cref{14-tab:normal-adm}.
In fact, once player~$1$ has eliminated strategy~$A$ (which is dominated by~$B$),
player~$2$ can use this information since its best response against the remaining strategies is~$C$.
In more complex games, this reasoning can be repeated and take several steps before converging.
This repeated process is called \emph{iterated elimination of dominated strategies}.

We now define this process formally.
\begin{definition}[Iterated elimination]
The \emph{sequence of iterative elimination} is a sequence of rectangular strategy sets defined as follows.
$S^0 = (S_i^0)_{P_i \in \Agt}$ is the set of all strategies.
For $k~\geq 0$, if we write $S^k = (S_i)_{P_i \in \Agt}$,
then~$S^{k+1}_i$ is the set of strategies in~$S^{k}_i$ that are not dominated in~$S^k$.
\end{definition}

Thus, the step~$1$ of the sequence of iterative elimination corresponds
to admissible strategies defined above. Let us call them $1$-admissible.
In step 2, we again compute strategies that are dominated by only
considering $1$-admissible strategies for all players, and repeat.
Strategies that survive all step of elimination are said \emph{iteratively admissible}.
% \todo{What is an admissible strategy? Is it just the first step or the limit?}

\begin{theorem}
  In parity games, the sequence of iterative elimination converges, and it reaches a non-empty fixpoint.
\end{theorem}
We prove this result only for simple safety games since the case of parity conditions is too complex for the scope of this book.

Intuitively, given a game~$G$, \Cref{14-thm:adm} tells us that any strategy for player~$P_i$ that avoids using edges~$D_i$ is admissible. So if we remove all edges~$\cup_{P_i \in \Agt} D_i$ from the
game to obtain a new game called~$G_1$, then all strategies of~$G_1$ (for all players) are admissible in~$G$, and conversely.
We can then repeat this process to~$G_1$: we construct~$G_2$ by eliminating all dominated edges
in~$G_1$, and get that admissible strategies in~$G_1$ are exactly all strategies of~$G_2$,
which correspond to~$2$-admissible strategies, and so on.

Since the size of the games decrease at each step, this process necessarily stops.
It remains to show that the limit game~$G_\infty$ contains strategies. We will show that
all vertices have at least one outgoing edge in~$G_\infty$.
It suffices to show that the sets~$D_i$ never contains all edges leaving a vertex.
Let us consider any game~$G_j$.
For a vertex~$v \in V_i$ with~$\pes_i(v)=0$ and~$\opt_i(v)=0$, none of the edges are dominated.
For a vertex~$v \in V_i$ with~$\pes_i(v)=1$ (and necessarily~$\opt_i(v)=1$), there exists a winning strategy in~$G_j$ so
there must be a successor~$v'$ with~$\pes_i(v')=1$ which is not dominated.
Last, for a vertex~$v \in V_i$ with~$\pes_i(v)=0$ and~$\opt_i(v)=1$, there exists a winning play from~$v$ so for some successor~$v'$ we must have~$\opt_i(v')=1$ which is an edge that is not dominated.

\section*{Bibliographic references}\label{other-results}




Most results about equilibria fall into two-categories: they either prove
that equilibria always exist for some class of games, or they characterize
the complexity of finding a particular one.


\subsection*{Existence results}
Several authors have noticed that Nash equilibria always exist in turn-based
game for some classes of objectives, in particular this is true of
$\omega$-regular objectives.
The most general result of that kind, shows that this is true for all objectives
for which there exists finite memory optimal strategies~\cite{LeRoux&Pauly:2018}.

%\subsubsection*{Nash theorem}
The notion of equilibrium we now call Nash equilibrium was
defined in the article of Nash~\cite{Nash:1950} in which he proves the
existence for a class of normal form game.
The Hawk-dove game we presented as an example in the first part of
this chapter is also called game of chicken.
The first reference to this game was by Smith and Price \cite{Smith&Price:1973}.
The example of medium of access control we presented as a motivation was
studied from a game theoric point of view in \cite{MacKenzie&Wicker:2003}.

%\subsubsection*{Subgame perfect equilibria}
The notion of subgame perfect equilibria is interesting because in games on
trees (or extensive games), for which they were originaly introduced, they
always exist. This results can be extended to game played on graph.
In particular subgame perfect equilibria always exist in reachability
games~\cite{Brihaye&Bruyere&DePril&Gimbert:2012}.

\subsection*{Algorithms and complexity results}

\mynote{Work here}
existence of Nash equilibria by Nash~\cite{Nash:1950,Nash:1951},
and Kuhn's theorem by Kuhn~\cite{Kuhn:1953}.

%\subsubsection*{Link between determinacy and existence of Nash equilibria}
The deviator construction and the algorithm presented in this chapter are based on \cite{Bouyer&Brenguier&Markey&Ummels:2011,Brenguier:2012}.
Algorithms on admissible strategies on infinite games and the complexity of the related problems
were studied in
\cite{Berwanger:2007,Brenguier&Raskin&Sassolas:2014}.
\todo{Romain: please check the refs}


%\subsubsection*{Stochastic and imperfect information games}
Imperfect information games in the context of multiplayer games are difficult.
As soon as there are ``information forks'' interesting problems are indecidable.
Deciding whether two players can ensure an objective against a third player
is undecidable.
As a corollary the Nash equilibrium problem is also undecidable~\cite{Pnueli&Rosner:1990}.
The problem of Nash equilibrium is also undecidable in stochastic games even
with only three players~\cite{Bouyer&Markey&Stan:2014}.

%\subsubsection*{LP techniques for mixed equilibria}
In the first section we presented a polynomial algorithm for
finding pure Nash equilibria in normal form games.
It is actually also possible to find a mixed Nash equilibria in
polynomial time using linear programming.
The same extends to finding memoryless mixed Nash equilibria in concurrent
games, and even resilient equilbria~\cite{Brenguier:2016},.

%\subsubsection*{Action-graph games}\label{14-subsection:action-graph-games}
Action-graphs are succinct representation of matrix games. Indeed,
representing games with matrices can be costly when the number of
players increases. The size of the matrix is in fact exponential in the
number of players: when each player has two strategies there are
\(2^{\Agt}\) cells in the table.
The action-graph representation is more compact, and the representation can be
exponentially smaller.
Because of that, the algorithm is no longer polynomial.
If there are no constraints on the Nash equilibrium we are looking for, the
complexity of the problem cannot be characterized using classical classes
like \NP-completness because equilibria always
exists and thus the answer to the decision problem would always be true.
The characterization of the complexity was done using the PPAD class~\cite{Daskalakis&Goldberg&Papadimitriou:2009}.
% Computer scientist wanted to show that this was
%indeed a difficult problem. Usally we do this by showing that the
%corresponding decision problems are NP-hard. However existence of Nash
%equilibrium cannot be made into a decision problem, because they always
% exist (the answer is always true so the problem is trivial). To
%characterize the complexity of this problem, the PPAD complexity class
%was created, and it can be shown that finding a Nash equilibrium in an
% action-graph game is a PPAD complete problem.
% \fbox{J'ai l'impression que ce paragrahphe est un peu trop rapide et n'apporte pas grand chose au lecteur.
% Ca n'a pas d'importance pour le reste du chapitre non plus.}
% \fbox{We can shorten it, and point to the relevent paper}

%\subsubsection*{Strategy logic}

Nash equilibria with LTL objectives is expressible in logics such as
strategy logic \cite{Chatterjee&Henzinger&Piterman:2010} or ATL$^\ast$~\cite{Alur&Henzinger&Kupferman:2002}, as well as other extensions of this equilibria.
However, satisfiability in these logic is difficult: it is
2\EXP-complete for ATL$^\ast$ and undecidable for
strategy logic in general.
An decidable fragment of strategy logic has been identified \cite{Mogavero&Murano&Perelli&Vardi:2012},
but remains difficult; it is 2\EXP-complete.

%%% Local Variables:
%%% mode: latex
%%% TeX-master: "../13_standalone"
%%% End:
