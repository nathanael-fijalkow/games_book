

%*** General probabilistic notation ***

\newcommand{\expv}{\mathbf{E}} % EXP. VALUE
\newcommand{\discProbDist}{f} % Discrete prob distribution
\newcommand{\sampleSpace}{S} % Generic sample space
\newcommand{\sigmaAlg}{\mathcal{F}} % Generic sigma-algebra
\newcommand{\probm}{\mathbb{P}} % Generic probability measure, also prob. measure operator
\newcommand{\rvar}{X} % Generic random variable
%\newcommand{\dist}{\mathit{Dist}}

%*** MDP notation ***

\newcommand{\actions}{A} % The set of actions.
\newcommand{\colouring}{c} % the colouring function
\newcommand{\probTranFunc}{\Delta} % Transition function of an MDP
\newcommand{\edges}{E} % Set of edges in an MDP.
\newcommand{\colours}{C} % The set of colours in an MDP.
\newcommand{\mdp}{\mathcal{M}} % A generic MDP. 
\newcommand{\vinit}{v_0} % An initial vertex in an MDP.
\newcommand{\cylProb}{p} % Function assigning probabilities to cylinder sets in 
%the measure construction.
\newcommand{\emptyPlay}{\epsilon} %empty play
\newcommand{\objective}{\Omega} % Qualitative objective
\newcommand{\genColour}{\textsc{c}} % Generic colour
\newcommand{\quantObj}{f} % Generic quantitative objective
\newcommand{\indicator}[1]{\mathbf{1}_{#1}} % In.d RV
\newcommand{\eps}{\varepsilon} % Numerical epsilon
\newcommand{\maxc}{W} % Maximal abs. value of a colour

\newcommand{\winPos}{W_{>0}}
\newcommand{\winAS}{W_{=1}}
\newcommand{\cylinder}{\mathit{Cyl}}

\newcommand{\PrePos}{\text{Pre}_{>0}}
\newcommand{\PreAS}{\text{Pre}_{=1}}

\newcommand{\PreOPPos}{\mathcal{P}_{>0}}
\newcommand{\OPAS}{\mathcal{P}_{=1}}

\newcommand{\safeOP}{\mathit{Safe_{=1}}}
\newcommand{\closed}{\mathit{Cl}}

\newcommand{\reachOP}{\mathcal{V}}
\newcommand{\discOP}{\mathcal{D}}
\newcommand{\valsigma}{\vec{x}^{\sigma}}

\newcommand{\lp}{\mathcal{L}}
\newcommand{\lpdisc}{\lp_{\mathit{disc}}}
\newcommand{\lpmp}{\lp_{\mathit{mp}}}
\newcommand{\lpsol}[1]{\bar{#1}}
\newcommand{\lpmpdual}{\lpmp^{\mathit{dual}}}

\newcommand{\actevent}[3]{\actions^{#1}_{#2,#3}} % Returns #1-th action on the run 

\newcommand{\MeanPayoffSup}{\MeanPayoff^{+}}
\newcommand{\MeanPayoffInf}{\MeanPayoff^{-}}

\newcommand{\mcprob}{M}
\newcommand{\invdist}{\vec{z}}

\newcommand{\hittime}{T}



This chapter studies two-player games whose arena is defined by a pushdown system\footnote{We use here the term pushdown system rather than pushdown automaton to stress the fact that we are not considering these devices as language acceptors but rather focus on the transitions systems they define.}. The vertices of the arena are the configurations of the pushdown system (i.e., pairs composed of a control state and a word representing the content of the stack) and the edges of the arena are defined by the pushdown system's transitions. For simplicity, both the ownership of a configuration and the objective will only depend on the control state of the configuration. Hence the partition of all the configurations between the two players will simply be given by a partition the control states. We will mainly consider the parity objective. Via a standard reduction\AC{Insert a reference to a previous chapter.}, parity pushdown game can be used to solve any pushdown game with an $\omega$-regular objectives (which in our setting is simply an $\omega$-regular set of infinite words over  the alphabet of control states). 

The main conceptual novelty of this chapter is that the arena is no longer finite. However as these games are described by a finite amount of information: the pushdown system, the ownership partition and the $\omega$-regular objective, they are amenable to algorithmic treatment. The first natural problem in this line is to decide the winner of the game from a given configuration. We will also consider the computation of \emph{finite representations} for the winning regions and the winning strategies. 


\section{Notations}
\label{10-sec:notations}
The definition of arena $\arena$ in this chapter is $\arena=(G,\dest)$, where $G=(V,E)$ is a graph and $\dest:V\times A\times A\rightarrow \Dist(E)$. In particular, we are not using the sets $\VA$ and $\VE$.
The games are played similarly to before and formally as follows: 
There is a token, initially on the initial vertex. 
Whenever the token is on some vertex $v$, 
Eve selects an action $r$ in $A$ and Adam selects an action $c$ in $A$. The edge $e=(v,c,w)$ is then drawn from the distribution $\dest(v,r,c)$ and the token is pushed from $v$ to $w$.
 In general, the game continues like that forever.

We will use the following simplifying assumptions in this chapter:
\begin{enumerate}
\item We will assume that all colors are in $\{0,1\}$, except for the section on Matrix games where we additionally also allow $-1$ (to be able to easily illustrate the game rock-paper-scissors). This simplifies some expressions, but generally, the dependency on the number of colors is not too bad comparatively.
\item To make illustrations easier, we assume that for any pair of edges $e,e'$ in $\dest(v,a,a')$ for any $v,a,a'$, we have that $c(e)=c(e')$, i.e. the color does not depend on which edge is picked from $\dest(v,a,a')$, but only $v,a,a'$. This assumption does not matter for the types of games considered.
\end{enumerate}

We will overload the notation slightly for notational convenience, in that for any $v,a,a'$, we will write $c(v,a,a')$ for $c(e)$ where $e\in \dest(v,a,a')$ (note that the second assumption ensures that this is well-defined, i.e. there is only one such color).


A vertex $v$ is absorbing iff each player has only 1 action in $v$ and $\Delta(v,1,1)=v$.

To describe the complexity of good stationary strategies in concurrent games, we will use the notion of patience. Given a probability distribution $d\in \Dist$ the distribution has patience $p$ if $p=\min_{i\in \supp(d)} d(i)$ (i.e. the patience is the smallest, non-zero probability that an event may happen according to $d$).
In essence, if you have low enough patience you can typically guess the strategy and check whether it is a good strategy (when you fix a strategy, the game becomes a Markov decision process, which are relative easy to work with), the game can solved in $\NP\cup \coNP$. However, some times the patience is huge and writing down a good strategy, in binary, cannot be done in polynomial space (it is quite surprising in some sense that even with this property, finding the values in the games remain in $\PSPACE$).

We will illustrate a stochastic arena $\arena=(G,\dest)$ as follows:
For each non-absorbing vertex $v$, there is matrix.
 Entry $(i,j)$ of the matrix illustrating $v\in V$ describes what happens if, when the token is on $v$, Eve plays $i$ and Adam $j$. The entry contains a color $c$, which is $c(v,i,j)$, and 
there is an arrow from entry $(i,j)$ of $v$ to $w$ if there is an edge   
$e=(v,c,w)$ in $\dest(v,i,j)$. 
 The arrow corresponding to $e$ is denoted with the probability $\dest(v,i,j)(e)$. 
Especially, to simplify the illustrations we will do as follows: If $|\supp(\dest(v,i,j))|=1$, we do not include the probability (which is 1). Also, in that case, let $e=(v,c,w)=\dest(v,i,j)$ 
and 
if $v=w$, we omit the arrow and if $w$ is absorbing we write $c^*$ in entry $i,j$ of $v$, where $c$ is the color $c(w,1,1)$ (in this case, we omit the number $c(e)$ from the illustration, but in none of our illustrations does this number matter for what we try to illustrate). 


\section{Profiles and regularity of the winning regions}
\label{10-sec:profiles}
In this section, we consider a large class of objectives called \emph{prefix independent}. For these objectives, a pushdown game can be meaningfully decomposed by considering the part of the game between the moment a symbol is pushed onto the stack and stopping as soon as it is popped.  As a consequence, we will see that for prefix independent
objectives, the winning region can be described  using finite state automata.




%\subsection{Profiles}
\acchanged{To this extent, we introduce \emph{reduced games} which start with a stack containing only one symbol $\gamma$ and stop as soon as this symbol is popped from the stack. If the symbol is never popped, the objective is unchanged and otherwise the winner of the game is determined by the state reached when popping the symbol.}

Let $\game=(\arena,\Omega)$ be a pushdown game played on an arena $\arena = (G,\VE,\VA)$ generated by a pushdown system $\PDS = (Q,Q_{\mEve}, Q_{\mAdam}, \Gamma,\Delta)$. For any subset $R\subseteq Q$ of control states of $\PDS$, we define a new objective $\Omega(R)$ such that a play $\play$ belongs to $\Omega(R)$ if one of the following happens:
%\begin{itemize}
%\item  In $\play$ no configuration with an empty stack, {i.e.} of the form $(q,\bot)$, is visited, and $\play\in\Omega$.
%\item In $\play$ a configuration with an empty stack is visited and the control state in the first such configuration belongs to $R$.
%\end{itemize}
\acchanged{
\begin{itemize}
\item the play $\play$ belongs to $\Omega$ and does not contain
any configuration with an empty stack ( {i.e.}, of the form $(q,\bot)$ for some state $q \in Q$),
\item the play $\play$ contains a configuration with the empty stack and the first such configuration has a state in R.
\end{itemize}}


More formally, letting $V=V_\mEve\cup V_\mAdam$, $V_R = R \times \bot \Gamma^*$ and $V_\bot = Q \times \{\bot\}$ 
$$ \Omega(R) = (\Omega \setminus V^*V_\bot V^\omega)\cup (V \setminus V_\bot)^* V_R V^\omega$$
Finally, we let $\game(R)$ denote the game $(\arena,\Omega(R))$.

\acchanged{Remark that contrarily to the rest of the objectives considered in this chapter, $\Omega(R)$ does depend on the sequences of vertices visited by the play and not only on their colour. It would have been possible by a slight modification of the arena to only express the objective on their colours. However, our choice simplifies the presentation of the reduced games.}

For any state $q\in Q$ and any stack letter $\gamma\in\Gamma$, we denote by $\mathcal{R}(q,\gamma)$ the set of subsets $R\subseteq Q$ for which \Eve wins in $\game(R)$ from $(q,\bot\gamma)$:
$$\mathcal{R}(q,\gamma)=\{
R\subseteq Q\mid (q,\bot\gamma) \text{ is winning for \Eve in } \game(R)
\}$$ and we refer to $\mathcal{R}(q,\gamma)$ as the \emph{$(q,\gamma)$-profile} of $\game$.

An objective $\Omega\subseteq C^\omega$ is \emph{prefix independent} if the following holds: for every $u\in C^\omega$ and for every \acchanged{$v\in C^*$},  $u\in \Omega$ if and only if $vu\in \Omega$. The Büchi, co-Büchi and parity objectives are examples of prefix independent objectives and the reachability objective is not.


\begin{remark} 
\label{10-rem:stay-staying-alive}
For a prefix independent objective, a play respecting a winning strategy for $\Eve$ that starts in \Eve's winning region always stay in this region. Obviously this is no longer true if the objective is not prefix independent. For instance in a reachability game, a play following a winning strategy for $\Eve$ can leave the winning region of \Eve once the target has been reached. 	
\end{remark}


This simple property allows to use profiles to give an inductive characterization of the winning region for \Eve when the objectif is prefix independent.

\begin{proposition}\label{10-prop:returning} Assume that $\Omega\subseteq C^\omega$ is prefix independent. 
Let $s\in \Gamma^*$, $q\in Q$ and $\gamma\in\Gamma$. Then \Eve has a winning strategy in $\game$ from $(q,\bot s\gamma)$ if and only if there exists some $R\in\mathcal{R}(q,\gamma)$ such that $(r,\bot s)$ is winning for \Eve in $\game$ for every $r\in R$.
\end{proposition}

\begin{proof}
Assume \Eve has a winning strategy $\sigma$ from $(q,\bot s\gamma)$ in $\game$. Consider the set $\Pi_\sigma$ of all plays in $\game$ that starts from $(q,\bot s\gamma)$ and where \Eve respects $\sigma$. Define $R$ to be the (possibly empty) set that consists of all $r\in Q$ such that there is a play in $\Pi_\sigma$ of the form $v_0\cdots v_k (r,\bot s) v_{k+1}\cdots$ where each $v_i$ for $0\leq i\leq k$ is of the form $(p_i,\bot s t_i)$ for some non-empty $t_i$. In other words, $R$ consists of all states that can be reached on popping $\gamma$ for the first time in a play where \Eve respects $\sigma$. As seen in \cref{10-rem:stay-staying-alive}, \Eve is winning from $(r,\bot s)$ for all $r \in R$. It remains to show that $R\in\mathcal{R}(q,\gamma)$.

To this end, define a (partial) function $\xi$ as $\xi((p,\bot s t))=(p,\bot t)$ for every $p\in Q$ and set $\xi^{-1}((p,\bot t))=(p,\bot s t)$. Then $\xi^{-1}$ is extended as a morphism over $V^*$.  Now a winning strategy for \Eve in $\game(R)$ is defined as follows:
\begin{itemize}
\item if some empty stack configuration has already been visited play any valid move, 
\item otherwise go to $\xi(\sigma(\xi^{-1}(\pi))$, where $\pi$ is the current play.
\end{itemize}
By definition of $\Pi_\sigma$ and $R$, it easily follows that the previous strategy is winning for \Eve in $\game(R)$, and therefore $R\in\mathcal{R}(p,\gamma)$. 
%Finally, for every $r\in R$ there is, by definition of $\Pi_\sigma$ a finite play $\pi_r$ that starts from $(q,\bot s\gamma)$, where \Eve respects $\sigma$ and that ends in $(r,\bot s)$. A winning strategy for \Eve in $\game$ from $(r,\bot s)$ is given by $\sigma'(\pi)=\sigma(\pi'_r\pi)$, where $\pi'_r$ denotes the partial play obtained from $\pi_r$ by removing its last vertex $(r,\bot s)$.

Conversely, let us assume that there is some $R\in\mathcal{R}(q,\gamma)$ such that $(r,\bot s)$ is winning for \Eve in $\game$ for every $r\in R$. For every $r\in R$, let us denote by $\sigma_r$ a winning strategy for \Eve from $(r,\bot s)$ in $\game$. Let $\sigma_R$ be a winning strategy for \Eve in $\game(R)$ from $(q,\bot\gamma)$. Let us define $\xi$ and $\xi^{-1}$ as in the direct implication and extend them as (partial) morphism over $V^*$. Define the following strategy $\sigma$ for \Eve in $\game$ for plays starting from $(q,\bot s\gamma)$. For any such play $\pi$, 
\begin{itemize}
\item if $\pi$ does not contain a configuration of the form $(p,\bot s)$ then we take $\sigma(\pi)=\xi^{-1}(\sigma_R(\xi(\pi)))$;
\item otherwise let $\pi = \pi'\cdot(r,\bot s)\cdot \pi''$ where $\pi'$ does not contain any configuration of the form $(p,\bot s)$. If $r$ does not belong to $R$, $\sigma$ is undefined. Note that this situation will never be encountered in a play respecting $\sigma$ as $\sigma_R$ ensures that $r \in R$. If $r \in R$, one finally sets $\sigma(\pi)=\sigma_r((r,\bot s)\pi'')$.
\end{itemize} 
The strategy $\sigma$ is a winning strategy for \Eve in $\game$ from $(q,\bot s\gamma)$. To see this, consider a play $\pi$ starting from $(p,\bot s \gamma)$ and respecting $\sigma$.

 If the play $\pi$ does not contain configurations of the form $(r,\bot s)$ for some $r \in Q$, then the play $\xi(\pi)$ starting in $(p,\bot \gamma)$ respects $\sigma_R$ and is won by $\Eve$.
As $\xi(\pi)$ does not contain configurations with an empty stack, it must be the case that $\xi(\pi) \in \Omega$. As $\Omega$ only depends on the colours of the states, it is also the case that $\pi \in \Omega$ and hence, $\pi$ is winning for $\Eve$. 

If the play $\pi$ can be decomposed as $\pi' (r,\bot s) \pi''$ where $\pi'$ does not contain any configuration with the stack $\bot s$, the play $\xi(\pi' (r,\bot s))$ respects $\sigma_R$ in $\game(R)$. As $\sigma_R$ is winning, it follows that $r \in R$. By definition of $\sigma$, $(r,\bot s)\pi''$ respects $\sigma_r$ which being winning for \Eve implies that $(r,\bot s) \pi'' \in \Omega$. As $\Omega$ is prefix independent, it follows that $\pi \in \Omega$.
\end{proof}

%Consider a quantitative objective $\Omega\subseteq C^\omega$. 

\Cref{10-prop:returning} implies that the winning region of any pushdown game equipped with a prefix independent objectives can be described by regular languages.

\begin{theorem}\label{10-thm:regularity-wr}
Let $\game=(\arena,\Omega)$ be a pushdown game played on an arena $\arena = (G,\VE,\VA)$ generated by a pushdown system $\PDS = (Q,Q_{\mEve}, Q_{\mAdam}, \Gamma,\Delta)$, and such that $\Omega\subseteq C^\omega$ is prefix independent. Then for any state $q\in Q$, the set \[
L_q=\{u\in \Gamma^*\mid (q,\bot s)\in W_\mEve \}
\] is a regular language over the alphabet $\Gamma$.
\end{theorem}

\begin{proof}
Fix a control state $q \in Q$, we consider a deterministic finite state automaton defined as follows. Its set of control states consists of the subsets of $Q$ and the initial state is $S_{in}=\{p\mid (p,\bot)\in W_\mEve\}$. From the state $S$ upon reading the letter $\gamma$ the automaton goes to the state $\{p\mid S\in\mathcal{R}(p,\gamma)\}$. Finally a state $S$ is final if and only if $q\in S$. It is then an immediate consequence of \Cref{10-prop:returning} that this automaton accepts the language $L_q$.
\end{proof}


\begin{remark}\label{10-rk:automata-winning-region}
By a slight abuse, we can think of the $|Q|$ automata in \Cref{10-thm:regularity-wr}	as a single automaton that is design to first read the stack content and finally reads the control state $q$ (in this latter step, from state $S$ it either go to a final state if $q\in S$ or to a rejecting one otherwise).
\end{remark}


\begin{remark}
Note that the characterisation in \Cref{10-thm:regularity-wr} is \emph{a priori} not effective. Indeed, to construct automata for the languages $L_q$ one needs to be able to compute all the $(q,\gamma)$-profiles $\mathcal{R}(q,\gamma)$ of $\game$ and compute
the winner from configurations of the form $(q,\bot)$. 

\newcommand{\Omegapar}{\Omega_{\textrm{pal}}} 
Consider, for instance, the objective over the set of the colours $C=\{0,1,\#,\$\}$
\[
\begin{array}{l}
\Omegapar = \{ w \in C^\omega \mid w \;\textrm{contains infinitely many factors of the form}\\
\quad\quad\quad\quad\quad\quad\quad\;\;\textrm{$\#u\$\tilde{u}\#$ with $u \in \{0,1\}^*$} \} \\
\end{array}
\]
where $\tilde{u}$ denotes the mirror of the word $u$.

As a language of $\omega$-words, this objective is accepted by a deterministic $\omega$-pushdown automaton with a Büchi acceptance condition. Between two consecutive occurrences of the $\#$-symbol, the automaton checks that the word $w$ appearing in between these two occurrences is of the form
$u\$\tilde{u}$ for some word $u \in \{0,1\}^*$. This can be done in a deterministic maner as follows. First the automaton pushes onto the stack a symbol $\bot'$ (which will play the same role as the bottom of stack symbol) then it pushes onto the stack all symbols in $\{0,1\}$ that are read. Then when the first $\$$-symbol is read, it only allows to read the symbol that is on the top of the stack before popping it. Finally when a $\#$-symbol is read, if the top-most symbol of the stack is $\bot'$  the unique final state is visited. Hence ensuring that a final state is visited if the word $w$ is of the required form. If $w$ contains several $\$$ symbols or if the symbol read does not correspond to the top of the stack, the automaton enters a non-final state is in which it waits for the next $\#$-symbol.

For games with finite arenas and the $\Omegapar$ objective,  deciding the winner reduces to deciding the winner in a  pushdown game with the Büchi objective which is decidable as we will prove later in this chapter. The pushdown game is essentially a synchronized product between the finite arena and the $\omega$-pushdown automaton described previously.

However the problem of deciding the winner in a pushdown game with the $\Omegapar$ objective is undecidable even if all vertices belong to \Eve. The undecidability is proved by a reduction from  Post correspondance problem (PCP) which is a well-known to be undecidable. Recall that an instance of PCP is a finite sequence $(r_1,\ell_1),\ldots,(r_n,\ell_n)$ of pairs of words over $\{0,1\}$. Such an instance is said to admit a solution if there exists a sequence of indices $i_1\cdots i_k \in [1,n]^*$ such that:
\[
 r_{i_1} r_{i_2} \cdots r_{i_k} = \ell_{i_1} \ell_{i_2} \cdots \ell_{i_k}.
\]
The PCP problem is, given an instance, to decide if it admits a solution.
 
For an instance $I=(\ell_1,r_1),\ldots,(\ell_n,r_n)$ of PCP, we construct a pushdown game $G_I$  with the objective $\Omegapar$ such that \Eve wins $G_I$ from $(p_\star,\bot)$ if and only if $I$ admits a solution. In this game, \Eve plays alone and the play is decomposed in two phases that will repeat:
\begin{itemize}
\item in phase 1, \Eve can push any indice $i$ in $[1,n]$	while producing the sequence of colors $r_i$. As soon as at least one index has been pushed, she can also choose to move to the sequence phase while producing the colour $\$$.
\item in phase 2, \Eve must (until the bottom of stack symbol is reached) pop the top most element of the stack $i$ while producing the sequence of colours $\tilde{\ell_i}$. When the bottom of stack symbol is encountered, \Eve goes back to the first phase while producing the colour $\#$.
\end{itemize}

If $I$ has a solution $i_1 \cdots i_k$ then the strategy in which \Eve always pushes this sequence in phase 1 is winning for her. As $i_1 \cdots i_k$ is a solution of $I$,
we have $u=r_{i_1}\cdot r_{i_k}=\ell_{i_1}\cdots\ell_{i_k} $ and by construction of the game, the sequence of colors associated with the play is $(u\$\tilde{u})^\omega \in \Omegapar$. Conversely if \Eve has a winning strategy from $(q_\star,\bot)$ then the sequence of colours associated with the winning play belongs to $\Omegapar$. In particular, it must contain a factor of the form $\#u\$\tilde{u}\#$. By construction of the game the sequence of indices $i_1 \cdots i_k$ pushed while producing $u$ is a solution of $I$.
\end{remark}



\subsection{Reachability Pushdown Game}

We will see in the following section that for the parity condition and more generally for any $\omega$-regular winning conditions the $(q,\gamma)$-profiles can be computed for any $q \in Q$ and $\gamma \in \Gamma$. We start by the simpler case of the reachability objective.

At first sight, the reachability objective is not captured by \Cref{10-thm:regularity-wr} as it is not prefix independent. However with a slight adaptation of the reduced game $\game(R)$ \Cref{10-thm:regularity-wr}  for the reachability condition. Intuitively we ask that the play stops as soon as a target vertex is reached.

More formally, for a reachability objective $\Reach(F)$ and letting $V=V_\mEve\cup V_\mAdam$ and $V_F = \{ v \in V \mid \pdscol{v} \in F\}$:
\[ \overline{\Omega}(R) = [(V \setminus Q \times \{\bot\})^* \cdot V_F \cdot V^\omega]\cup [(V \setminus (Q\times\{\bot\} \cup V_F))^* \cdot (R\times \{\bot\}) \cdot V^\omega]
\]

It is easily shown that with this modification to the definition of profiles both \Cref{10-prop:returning} and \Cref{10-thm:regularity-wr} are true for the reachability objective.
\newcommand{\Profs}{\mathrm{Profs}}
In the special case of the reachability objective, the set $\Profs$ of triples $(p,\gamma,R)$ such that $R \in \mathcal{R}(p,\gamma)$ can be expressed as a smallest fix-point.

More precisely, $\Profs$ is the smallest subset of $Q\times \Gamma \times \mathcal{P}(Q))$ such that for $p \in Q$, $\gamma \in \Gamma$ and $R \subseteq Q$, $(p,\gamma,R)$ belongs $\Profs$ if either: 
\begin{enumerate}
\item $p \in Q_F=\{ q \in Q \mid \pdscol{q} \in F\}$,
\item or $p \in Q_{\mEve}$ and for some $q \in Q$,
\begin{itemize}
\item either $(q,\pop) \in \Delta(p,\gamma)$ and $q \in R$,
\item or $(q,\push{\gamma'}) \in \Delta(p,\gamma)$  and $(q,\gamma',R') \in \Profs$ for some $R' \subseteq Q$ such that for all $p' \in R'$, $(p',\gamma,R) \in \Profs$.
\end{itemize}
\item or $p \in Q_A$ and for all $q \in Q$ the following hold: 
\begin{itemize}
\item $(q,\pop) \in \Delta(p,\gamma)$ implies $q \in R$,
\item  $(q,\push{\gamma'}) \in \Delta(p,\gamma)$ implies that there exists $R' \subseteq Q$ such that $(q,\gamma',R') \in \Profs$ and for all $p' \in R'$, $(p',\gamma,R) \in \Profs$. 
\end{itemize}
\end{enumerate}

\AC{Do we include the proof of the correctness of the characterization ?}

Using this characterization, the set $\Profs$ can be computed using the standard method for computing small-fixed point of a monotonic function by computing the sequence of approximants $\Profs_0 = \emptyset \subseteq \Profs_1 \subseteq \Profs_2 \cdots$ until it stabilizes. More precisely, for all $i \geq 0$, $\Profs_{i+1}$ is obtained by adding to $\Profs_{i}$ all the tuples that can be inferred using the properties $(1)$, $(2)$ and $(3)$ above applied to $\Profs_i$. As at most $|Q|\cdot |\Gamma|\cdot 2^{|Q|}$ tuples can be added, the sequence must stabilize in at most $|Q| \cdot |\Gamma| \cdot 2^{|Q|}$ steps. As the computation of $\Profs_{i+1}$ from $\Profs_i$ can be performed in polynomial time, the profils in a reachability pushdown game can be computed in time  $p(|Q| \cdot |\Gamma| \cdot 2^{|Q|})$ for some polynomial $p$.

\begin{remark}
In the case where \Eve plays alone (i.e., $Q=Q_\mEve$), there is only on play respecting a fixed strategy for \Eve and as a result, one only need to compute profils of the form $(p,\gamma,R)$ with $|R|\leq 1$. In this setting, the fixed-point characterization yields a polynomial time algorithm to compute the set of profiles.
\end{remark}



%TODO If all vertices belong to \Eve, all triplets in $(p,\gamma,\{q}$
%For every prefix independent winning condition on the state sequences, the winning region is regular. The construction is not effective but serves to introduces reduced games and their profiles.

%A reduced game is simply the pushdown game with a special meaning for popping the bottom of the stack. 


%\AClong{Profiles of reduces games can be extended to parity.}


\section{Parity pushdown games}
\label{10-sec:parity}
Recall that the ""parity"" objective extends B{\"u}chi and coB{\"u}chi objectives:
\[
\Parity = \set{\rho \in [1,d]^\omega \mid \text{ the largest priority appearing infinitely often in } \rho \text{ is even}}.
\]

\begin{theorem}
\label{2-thm:parity}
Parity objectives are uniformly positionally determined for both players\footnote{See \cref{2-rmk:finite_infinite} for the case of infinite games.}.
There exists an algorithm for computing the winning regions of parity games in exponential time,
and more precisely of complexity $O(m n^d)$.
The space complexity of $O(nd)$.

Furthermore, solving parity games is in $\NP \cap \coNP$.
\end{theorem}

To prove \cref{2-thm:parity} we first construct a recursive algorithm for computing the winning regions of parity games.
The algorithm is often called Zielonka's algorithm, or more accurately McNaughton Zielonka's algorithm.
We refer to the reference section~\cref{2-sec:references} for a discussion on this nomenclature.
We will see that the positionaly determinacy result for both players will be a consequence of the analysis of the algorithm.
The $\NP \cap \coNP$ complexity bounds will be discussed at the end of this section.

The following lemma induces (half of) the recursive algorithm.
Identifying a colour and its set of vertices we write $d$ for the set of vertices of priority $d$.

\begin{lemma}
\label{2-lem:zielonka_even}
Let $\Game$ be a parity game with priorities in $[1,d]$, and $d$ even.
Let $\Game'$ be the subgame of $\Game$ induced by $V \setminus \AttrE(d)$.
\begin{itemize}
	\item If $\WA(\Game') = \emptyset$, then $\WE(\Game) = V$.
	\item If $\WA(\Game') \neq \emptyset$, 
	let $\Game''$ be the subgame of $\Game$ induced by $V \setminus \AttrA( \WA(\Game') )$,
	then $\WE(\Game) = \WE(\Game'')$.	
\end{itemize}
\end{lemma}

To see that this leads to a recursive algorithm, we note that $\Game'$ has priorities in $[1,d-1]$
and that if $\WA(\Game') \neq \emptyset$, then $\Game''$ has less vertices than $\Game$.

%An equivalent (yet a bit heavier) formulation of \cref{2-lem:zielonka_even} is to construct
%$\WE(\Game)$ as the least fixed point of a monotonic operator based on $\WE(\Game')$ (in line with \cref{2-lem:Buchi}, which is the special case for $d = 2$).
%This formulation would show that \cref{2-algo:zielonka} is an instantiation of Kleene's fixed point theorem stated in \cref{1-thm:kleene},
%and would make it easier to extend the positionality result for all games even infinite ones 
%(using transfinite induction as discussed before the proof of \cref{2-lem:reachability}).

%In the following proof, we will use the notion of "traps" introduced in \cref{1-sec:subgames}.

\begin{proof}
We prove the first item. 

Let $\sigma_d$ be an attractor strategy ensuring to reach $d$ from $\AttrE(d)$.
Consider a winning strategy for Eve from $V \setminus \AttrE(d)$ in $\Game'$, it induces a strategy $\sigma'$ in $\Game$.
We construct a strategy $\sigma$ in $\Game$ as the disjoint union of $\sigma_d$ on $\AttrE(d)$ and of $\sigma'$ on $V \setminus \AttrE(d)$.
Any play consistent with $\sigma$ either enters $\AttrE(d)$ infinitely many times, 
or eventually remains in $V \setminus \AttrE(d)$ and is eventually consistent with $\sigma'$.
In the first case it sees infinitely many times $d$, which is even and maximal, hence satisfies $\Parity$, 
and in the other case since $\sigma'$ is winning the play satisfies $\Parity$.
Thus $\sigma$ is winning from $V$.

We now look at the second item.

Let $\tau_a$ denote an attractor strategy %ensuring to reach $\WA(\Game')$ 
from $\AttrA(\WA(\Game')) \setminus \WA(\Game')$.
Consider a winning strategy for Adam from $\WA(\Game')$ in $\Game'$, it induces a strategy $\tau'$ in $\Game$.
Since $V \setminus \AttrE(d)$ is a "trap" for Eve, this implies that $\tau'$ is a winning strategy in $\Game$.
Consider now a winning strategy in the game $\Game''$ from $\WA(\Game'')$, it induces a strategy $\tau''$ in $\Game$.
The set $V \setminus \AttrA( \WA(\Game') )$ may not be a trap for Eve, so we cannot conclude that $\tau''$ is a winning strategy in $\Game$,
and it indeed may not be.
We construct a strategy $\tau$ in $\Game$ as the (disjoint) union of the strategy $\tau_a$ on $\AttrA(\WA(\Game')) \setminus \WA(\Game')$,
the strategy $\tau'$ on $\WA(\Game')$ and the strategy $\tau''$ on $\WA(\Game'')$.
We argue that $\tau$ is winning from $\AttrA( \WA(\Game') ) \cup \WA(\Game'')$ in $\Game$.
Indeed, any play consistent with this strategy in $\Game$ either stays forever in $\WA(\Game'')$ hence is consistent with $\tau''$
or enters $\AttrA( \WA(\Game') )$, hence is eventually consistent with $\tau'$.
In both cases this implies that the play is winning.
Thus we have proved that $\AttrA( \WA(\Game') ) \cup \WA(\Game'') \subseteq \WA(\Game)$.

We now show that $\WE(\Game'') \subseteq \WE(\Game)$, which implies the converse inclusion.
Consider a winning strategy from $\WE(\Game'')$ in $\Game''$, it induces a strategy $\sigma$ in $\Game$.
Since $\Game''$ is a "trap" for Adam, any play consistent with $\sigma$ stays forever in $\WE(\Game'')$, 
implying that $\sigma$ is winning from $\WE(\Game'')$ in $\Game$.
\end{proof}

To get the full algorithm we need the analogous lemma for the case where the maximal priority is odd.
We do not prove the following lemma as it is the exact dual of the previous lemma, and the proof is the same swapping the two players.

\begin{lemma}
\label{2-lem:zielonka_odd}
Let $\Game$ be a parity game with priorities in $[1,d]$, and $d$ odd.
Let $\Game'$ be the subgame of $\Game$ induced by $V \setminus \AttrA(d)$.
\begin{itemize}
	\item If $\WE(\Game') = \emptyset$, then $\WA(\Game) = V$.
	\item If $\WE(\Game') \neq \emptyset$, let $\Game''$ be the subgame of $\Game$ induced by $V \setminus \AttrE( \WE(\Game') )$,
	then $\WA(\Game) = \WA(\Game'')$.	
\end{itemize}
\end{lemma}

The algorithm is presented in pseudocode in \cref{2-algo:zielonka}.

The proofs of \cref{2-lem:zielonka_even} and \cref{2-lem:zielonka_odd} also imply that parity games are positionally determined for both players.
Indeed, winning strategies are defined as disjoint unions of strategies constructed inductively.

\vskip1em
We now perform a complexity analysis.
We write $f(n,d)$ for the number of recursive calls performed by the algorithm on parity games with $n$ vertices and priorities in $[1,d]$.
We have $f(n,1) = 0 = f(0,d) = 0$, with the general induction:
\[
f(n,d) \le f(n,d-1) + f(n-1,d) + 2.
\]
The term $f(n,d-1)$ corresponds to the recursive call to $\Game'$, 
the term $f(n-1,d)$ to the recursive call to $\Game''$.
We obtain $f(n,d) \le n \cdot f(n,d-1) + 2n$,
so $f(n,d) \le 2(n + n^2 + \dots + n^{d-1}) = O(n^d)$.
In each recursive call we perform two attractor computations so the number of operations
in one recursive call is $O(m)$.
Thus the overall time complexity is $O(m n^d)$.

\vskip1em
We finish the proof of \cref{2-thm:parity} by sketching the argument that solving parity games is in $\NP \cap \coNP$.
The first observation is that computing the winning regions of the one player variants of parity games can be done in polynomial time
through a simple graph analysis that we do not detail here.
The $\NP$ and $\coNP$ algorithms are the following: guess a winning positional strategy,
and check whether it is winning by computing the winning regions of the one player game induced by the strategy.
Guessing a strategy for Eve is a witness that the answer is yes so it yields an $\NP$ algorithm,
and guessing a strategy for Adam yields a $\coNP$ algorithm.

\Cref{3-chap:parity} is devoted to the study of advanced algorithms for parity games.

\begin{algorithm}
 \KwData{A parity game $\Game$ with priorities in $[1,d]$}
 \SetKwFunction{FSolveEven}{SolveEven}
 \SetKwFunction{FSolveOdd}{SolveOdd}
 \SetKwProg{Fn}{Function}{:}{}
 \DontPrintSemicolon

\Fn{\FSolveEven{$\Game$}}{
%	\If{$V = \emptyset$}{\Return{$\emptyset$}}

	Let $\Game'$ the subgame of $\Game$ induced by $V \setminus \AttrE(d)$
	
	$X \leftarrow$ \FSolveOdd{$\Game'$}

	\If{$X = \emptyset$}{\Return{$V$}}
	\Else{
		Let $\Game''$ the subgame of $\Game$ induced by $V \setminus \AttrA(X)$
		
		\FSolveEven{$\Game''$}
%		$\WE(\Game'') \leftarrow$ \FSolveEven{$\Game''$}
%
%		\Return{$\WE(\Game'')$}
	}
}
\vskip1em
\Fn{\FSolveOdd{$\Game$}}{
	\If{$d = 1$}{\Return{$V$}}

	Let $\Game'$ the subgame of $\Game$ induced by $V \setminus \AttrA(d)$
	
	$X \leftarrow$ \FSolveEven{$\Game'$}

	\If{$X = \emptyset$}{\Return{$V$}}
	\Else{
		Let $\Game''$ the subgame of $\Game$ induced by $V \setminus \AttrE(X)$
		
		\FSolveOdd{$\Game''$}
%		$\WA(\Game'') \leftarrow$ \FSolveOdd{$\Game''$}
%
%		\Return{$\WA(\Game'')$}
	}
}
\vskip1em
\If{$d$ is even}{
	\FSolveEven{$\Game$}
}
\Else{
	\FSolveOdd{$\Game$}
}
\caption{A recursive algorithm for computing the winning regions of parity games.}
\label{2-algo:zielonka}
\end{algorithm}


\section*{Bibliographic references}
\label{10-sec:references}
We refer to~\cref{2-sec:references} for the role of parity objectives and how they emerged in automata theory as a subclass of Muller objectives.
Another related motivation comes from the works of Emerson, Jutla, and Sistla~\cite{Emerson&Jutla&Sistla:1993},
who showed that solving parity games is linear-time equivalent to the model-checking problem for modal $\mu$-calculus.
This logical formalism is an established tool in program verification, and a common denominator to a wide range of modal, temporal and fixpoint logics used in various fields.

\vskip1em
Let us discuss the progress obtained over the years for each of the three families of algorithms.

\vskip1em
\textit{Value iteration algorithms and separating automata}.
The heart of value iteration algorithms is the value function, which in the context of parity games and related developments for automata
have been studied under the name progress measures or signatures.
They appear naturally in the context of fixed point computations so it is hard to determine who first introduced them.
Streett and Emerson~\cite{Streett&Emerson:1984,Streett&Emerson:1989} defined signatures for the study of the modal $\mu$-calculus,
and Stirling and Walker~\cite{Stirling&Walker:1989} later developped the notion.
Both the proofs of Emerson and Jutla~\cite{Emerson&Jutla:1991} and of Walukiewicz~\cite{Walukiewicz:1996} use signatures to show the positionality of parity games over infinite games.

Jurdzi{\'n}ski~\cite{Jurdzinski:2000} used this notion to give the first value iteration algorithm for parity games, 
with running time $O(m n^{d/2})$.
The algorithm is called ``small progress measures'' and is an instance of the class of value iteration algorithms we construct 
in~\cref{3-sec:value_iteration} by considering the universal tree of size $n^h$.
Bernet, Janin, and Walukiewicz~\cite{Bernet&Janin&Walukiewicz:2002} investigated reductions from parity games to safety games
through the notion of permissive strategies, and constructed a separating automaton\footnote{We note that the general framework of separating automata came later, introduced by Boja{\'n}czyk and Czerwi{\'n}ski~\cite{Bojanczyk&Czerwinski:2018}.} corresponding to the universal tree of size $n^h$.

The new era for parity games started in 2017 when Calude, Jain, Khoussainov, Li, and Stephan~\cite{Calude&Jain&al:2017} constructed a quasipolynomial time algorithm. 
Our presentation follows the technical developments of the subsequent paper by Fearnley, Jain, Schewe, Stephan, and Wojtczak~\cite{Fearnley&Jain&al:2017} which recasts the algorithm as a value iteration algorithm.
Boja{\'n}czyk and Czerwi{\'n}ski~\cite{Bojanczyk&Czerwinski:2018} introduce the separation framework to better understand the original algorithm.

Soon after two other quasipolynomial time algorithms emerged.
Jurdzi{\'n}ski and Lazi{\'c}~\cite{Jurdzinski&Lazic:2017} showed that the small progress measure algorithm can be adapted to a ``succinct progress measure'' algorithm, matching (and slightly improving) the quasipolynomial time complexity.
The presentation using universal tree that we follow in~\cref{3-sec:value_iteration} and an almost matching lower bound on their sizes is due to Fijalkow~\cite{Fijalkow:2018}.
The connection between separating automata and universal trees was shown by Czerwi{\'n}ski, Daviaud, Fijalkow, Jurdzi{\'n}ski, Lazi{\'c}, and Parys~\cite{Czerwinski&Daviaud&al:2018}. 

The third quasipolynomial time algorithm is due to Lehtinen~\cite{Lehtinen:2018}.
The original algorithm has a slightly worse complexity ($n^{O(\log(n))}$ instead of $n^{O(\log(d))}$),
but Parys~\cite{Parys:2020} later improved the construction to (essentially) match the complexity of the previous two algorithms.
Although not explicitly, the algorithm constructs an automaton with similar properties as a separating automaton,
but the automaton is non-deterministic.
Colcombet and Fijalkow~\cite{Colcombet&Fijalkow:2019} revisited the link between separating automata and universal trees
and proposed the notion of good for small games automata, capturing the automaton defined by Lehtinen's algorithm.
The equivalence result between separating automata, good for small games automata, and universal graphs, holds for any positionally determined objective, giving a strong theoretical foundation for the family of value iteration algorithms.

\vskip1em
\textit{Attractor decomposition algorithms}.
The McNaughton Zielonka's algorithm has complexity $O(m n^d)$.
Parys~\cite{Parys:2019} constructed the fourth quasipolynomial time algorithm as an improved take over McNaughton Zielonka's algorithm.
As for Lehtinen's algorithm, the original algorithm has a slightly worse complexity ($n^{O(\log(n))}$ instead of $n^{O(\log(d))}$).
Lehtinen, Schewe, and Wojtczak~\cite{Lehtinen&Schewe&Wojtczak:2019} later improved the construction.
As discussed in~\cref{3-sec:relationships} the complexity of this algorithm is quasipolynomial and of the form $n^{O(\log(d))}$,
but a bit worse than the three previous algorithms since the algorithm is symmetric and has a recursion depth of $d$,
while the value iteration algorithms only consider odd priorities hence replace $d$ by $d/2$.

Jurdzi{\'n}ski and Morvan~\cite{Jurdzinksi&Morvan:2020} constructed a generic McNaughton Zielonka's algorithm parameterised by the choice of two universal trees, one for each player.
\mynote{CONTINUE}


\vskip1em
\textit{Strategy improvement algorithms}.
As we will see in~\cref{4-chap:payoff}, parity games can be reduced to mean payoff games,
so any algorithm for solving mean payoff games can be used for solving parity games.
In particular, the existing strategy improvement algorithm for mean payoff games can be run on parity games. 
V{\"o}ge and Jurdzin{\'n}ski~\cite{Voge&Jurdzinski:2000} introduced the first discrete strategy improvement for parity games,
running in exponential time.
For some time there was some hope that the strategy improvement algorithm, for some well chosen policy on switching edges,
solves parity games in polynomial time.
Friedmann~\cite{Friedmann:2011} cast some serious doubts by constructing numerous exponential lower bounds applying to different variants of the algorithm.
Fearnley~\cite{Fearnley:2017} investigated efficient implementations of the algorithm, focussing on the cost of computing and updating the value function for a given strategy.
Our proof of correctness is original. \mynote{SAY MORE?}

The complexity was reduced to subexponential with randomised algorithms 
by Jurdzin{\'n}ski, Paterson, and Zwick~\cite{Jurdzinski&Paterson&Zwick:2008}.
A natural question is whether there exists a quasipolynomial strategy improvement algorithm; 
as discussed in~\cref{3-sec:relationships} the notion of universal trees cannot be used to achieve this,
and the question remains to this day open.


%\newpage \OSlong{A partir d'ici reste du draft de plan}
%
%
%\subsection{Lower Bound}
%
%Here we show that even for reachability condition the problem is hard for exponential time. We give the proof of Walukiewicz that reduces from the acceptance problem for linear-space alternating Turing machine.
%
% Hardness for PSPACE when using a counter instead of a general stack (proof ?)
%
%\subsection{Alternative Approaches}
%
%Connection the full binary tree. Again regularity of the winning region plus the existence of regular positional strategies with Rabin's lemma. 
%It is effective but the complexity is not tractable.
%
%A more tractable approach is  using \textbf{two-way alternating tree automata}. This also gives an regular positional defined by deterministic exponential automaton reading the stack from bottom to top.
%
%Also mention that for ""one-counter games"" it leads a \PSPACE\ algorithm.\label{10-one-counter}
%
%
%
%
%\section{Beyond Parity Pushdown Games}
%
%\subsection{Beyond Parity Condition}
%
%Here we present the unboundedness winning condition (mention topological complexity?) and we argue how it can be decided in the same way as parity. Discuss also existing extensions where it is mixed with parity.
%
%\subsection{Beyond Pushdown Games}
%
%Here we introduce/discuss (depending space) higher-order pushdown automata. We mainly describe the results and also their implications (wrt logic, program verification). Just a discussion prob	ably not the time.
%
%

