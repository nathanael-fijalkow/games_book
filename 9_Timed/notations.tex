We~fix a finite set~$\Clocks$ of ""clock"" variables to be used
in our timed games. Elements of~$\Realnn^\Clocks$, which assign a
value to each clock, are called \emph{valuations}.

Clocks will be used to define \emph{""clock constraints""}, which in turn
are used in timed automata to restrict the set of allowed behaviours:
edges are decorated with clock constraints defining conditions for
their availability.
%on edge~$e$ is a precondition on clock valuations
%under which this edge~$e$ is available.
%to activate a given edge that depend on the clock values.
An~\emph{atomic clock constraint} is a formula of the form $k \preceq
x \preceq' l$ or $k \preceq x - y \preceq' l$ where $x,y \in \Clocks$,
$k,l \in \mathbb{Z}\cup\{-\infty,\infty\}$ and
${\mathord\preceq,\mathord\preceq' \in
  \{\mathord<,\mathord\leq\}}$. A~\emph{clock constraint} is a
conjunction of atomic clock constraints.  A~valuation~$\nu\colon
\Clocks\to\Realnn$ satisfies a clock constraint~$g$, denoted $\nu \models g$,
if~all atomic clock constraints are satisfied when each $x\in \Clocks$
is replaced with its value~$\nu(x)$.  We~write $\Phi_\Clocks$ for the
set of clock constraints built on the clock set~$\Clocks$.



For a subset $R\subseteq \Clocks$ and a valuation~$\nu$, we~denote
with ${\nu[R \leftarrow 0]}$ the valuation defined by ${\nu[R
    \leftarrow 0](x) = 0}$ if $x \in R$ and ${\nu[R\leftarrow 0](x) =
  \nu(x)}$ otherwise. Given $d \in \mathbb{R}_{\geq 0}$ and a
valuation~$\nu$, the~valuation~$\nu+d$ is defined by $(\nu+d)(x) =
\nu(x)+d$ for all $x\in \Clocks$. We~extend these operations to sets
of valuations in the obvious~way.

%As we saw in~\cref{9-ex:intro}, timed arenas are defined on a finite
%graph just like regular arenas, but they additionally contain clock
%guards and resets.  A~\emph{configuration} of a timed game is thus
%defined as a~pair~$(\ell,\nu)$ of a \emph{location}~$\ell$ and a
%valuation~$\nu$.

%is called a \emph{configuration} and determines the precise state of the game.
%The configurations will become the vertices of the game 
%
% Hence, the game is played in an arena whose vertices are 
%pairs~$(\ell,\nu)$ where~$\ell$ is a \emph{location}, that is,
%a state of the underlying finite automaton, and~$\nu$ a clock valuation.

%\NM{definir ``configuration''?}
%\OS{j'ai d\'efini ``vertex'' comme une paire $(\ell,\nu)$ pour faire commes les jeux finis.
%`Configuration' est peut-etre mieux}
%Hence, the precise state of the arena is determined by the
%current vertex in the graph \emph{and} the clock valuation.  We~will
%thus call the vertices~$\ell$ of the underlying finite graph
%\emph{locations}, while pairs $(\ell,\nu)$ of location and valuations
%will be called \emph{vertices}.

We now formally define \emph{timed games}, which are finite
representations that define infinite-state, non-stochastic concurrent
games. 

\begin{definition}
  A~\emph{""timed arena""} $\calT$ is a tuple $(\Locs,
  \Clocks,\EE,\EA, c)$, where $\Locs$ is a finite set of locations,
  $\Clocks$~is a finite set of clocks, $\EE,\EA \subseteq \Locs \times
  \Phi_\Clocks \times 2^\Clocks \times \mathcal{L}$ are the sets of
  edges respectively controlled by Eve and~\Adam,
%controllable and uncontrollable \NM{Parler d'Adam et Eve ?} edges
%  respectively, 
  and $c\colon \EE\cup\EA \to C$ is the coloring function.
%  \NM{In Def. 1.1, colors are attached to edges... OS: Yes but
%the chapter also allows attaching them to states later.}
  A~\emph{""timed game""} is a pair~$(\calT,\Omega)$ where
  $\Omega \subseteq C^\omega$ a qualitative objective.
\end{definition}

%\paragraph{Concurrent Game Semantics}

A~configuration of such a timed automaton is a pair~$(\ell,\nu)$ where
$\ell\in\Locs$ and $\nu\colon \Clocks\to\Realnn$.  The~set of
configurations is the set of vertices of the infinite-state game
defined by~$\calT$.
%The coloring function is inherited from that of~$\calT$, that is,
%$c'((\ell,\nu)) = c(\ell)$.

The~actions of both players are pairs~$(d,e)$ where $d\in\Realnn$ is a
delay they want to wait before playing their controlled
action~$e$. Action~$(d,e)$ is available for Eve (resp. Adam) in
configuration~$(\ell,\nu)$ if $e\in\EE(\ell)$ (resp.~$e\in\EA(\ell)$)
and, writing $e=(\ell,g,R,\ell')$, if additionally $\nu+d\models g$;
in~other terms, the~clock constraint (called~\emph{""guard""}~hereafter)
on~$e$ must hold true under the clock valuation reached after elapsing
$d$ time units. We~add an extra dummy action~$\bot$ for the case where
some player has no available action (this~action is only available if
no other actions~are).

Once both players have selected an action available from
configuration~$(\ell,\nu)$, the action~$(d,e)$ with smallest delay is
performed (by breaking ties in favor of Adam), leading to configuration $(\ell',(\nu+d)[R\leftarrow 0])$:
this corresponds to letting $d$ time units elapse, thereby reaching
configuration~$(\ell,\nu+d)$, and to following edge~$e$ (which by
construction is available from~$(\ell,\nu+d)$).  We~define
$\step((\ell,\nu),(d,e))$ for the configuration
$(\ell',(\nu+d)[R\leftarrow 0])$ reached from~$(\ell,\nu)$ by applying
action~$(d,e)$.



%% In~any configuration~$(\ell,\nu)$, an edge~$e = (\ell,g,R,\ell')$ is
%% said to be \emph{enabled} if~$\nu \models g$.  From
%% configuration~$(\ell,\nu)$, a~single transition consists in waiting
%% for some delay~$d\geq 0$ and taking an edge~$e$, provided that~$e$ is
%% enabled at~$\nu+d$. This results in the configuration~$(\ell',\nu')$
%% where~$\nu' = (\nu+d)[R\leftarrow 0]$ and~$e=(\ell,g,R,\ell')$. Let us
%% denote this single step by $(\ell',\nu') = \step( (\ell,\nu), (d,e))$.

%% We are going to start by defining a \emph{concurrent game} induced by a timed game. 
%% The set of vertices is the set of configurations.
%% Intuitively, at each configuration, both players choose a delay and an edge (in their respective sets) that is enabled after that delay. The shortest delay and the corresponding edge determines the next vertex.

This definition captures the concurrent nature of the interaction
between a controller (Eve) and its environment (Adam) in a real-time
system, since none of the players knows in advance how long the
opponent will want to wait before performing her transition.
The~semantics of a timed arena $(\Locs, \Clocks,\EE,\EA)$ can then
formally be defined in terms of a concurrent arena (following the
definition of \cref{8-sec:notations}).  The~underlying
graph~$(V,E)$ is such that $V=\Locs\times \Realnn^{\Clocks}$, and $E=
V\times C\times V$; the~set of actions of Eve is $\Realnn\times\EE$, and that
of Adam is $\Realnn\times\EA$; finally, the transition function,
which is not stochastic in our case, maps any
configuration~$(\ell,\nu)$ and pair of actions~$(d_\Eve,e_\Eve)$ and
$(d_\Adam,e_\Adam)$ to the edge $((\ell,\nu),\gamma,
\step((\ell,\nu),(d,e)))$, where $(d,e)=(d_\Eve,e_\Eve)$ and $\gamma=c(e_\mEve)$
if
$d_\Eve<d_\Adam$,  and $(d,a)=(d_\Adam,e_\Adam)$ and $\gamma=c(e_\mAdam)$ otherwise.
%The~coloring function of a timed game is transferred to its underlying
%infinite-state concurrent game in the natural way:

A~path in a timed arena~$\calT$ then is a path in its
associated infinite-state concurrent arena. The~qualitative
objective~$\Omega$ can then be evaluated along runs of a timed arena in
the natural way.

%\NM{Example of non-determined game} 
%\NM{definition written to stick to that of \cref{chap:concurrent}.
%  Update accordingly if definition is modified there}

%% We define the concurrent game induced by a timed game
%% $\calT = (\Locs, \Clocks,\EE,\EA, c, \Omega)$.
%% We consider the concurrent arena~$\calC = (V,\Act,\delta, c')$
%% and the identical objective~$\Omega$
%% where~$V = \Locs\times \Realnn^\Clocks$.
%% Note that unlike games seen in previous chapters, $V$ is infinite.
%% % is the infinite state space.
%% %We~write~$\vec{0}$ for the valuation that assigns~$0$ to every clock;
%% %and let~$(\ell_0,\vec{0}) \in \calQ$ be the \emph{initial state}.

%% At any configuration~$(\ell,\nu)$, the actions of Eve are pairs~$(d,e) \in \Realnn \times \EE(\ell)$
%% such that~$e$ is enabled at~$\nu+d$. Similarly, the actions of Adam are pairs~$(d,e)\in \Realnn \times \EA(\ell)$ with~$e$ enabled at~$\nu+d$.

%% The next configuration is determined via the function
%% \[
%%   \delta( (\ell,\nu), (d,e), (d',e') ) =
%%   \left\{
%%   \begin{array}{ll}
%%     \step((\ell,\nu), (d,e)) & \text{if } d < d',\\
%%     \step((\ell,\nu), (d',e'))  & \text{if } d \geq d'.\\
%%   \end{array}
%% \right.
%% \]

%Notice that we break ties in favor of Adam since we will be interested winning strategies for Eve.
Contrary to~\cref{7-chap:concurrent}, in this chapter we only consider deterministic strategies\footnote{Adding randomization over the infinite sets of actions is beyond the scope of this chapter.}
%  , and to the best of our knowledge, stochastic strategies
% in timed games have never been studied.}.
%\todo{Benjamin mentionne le papier: Trading infinite memory for uniform randomness in timed games, de Krish, Tom et Vinayak.
%On enleve le footnote? Ou bien on dit ca a ete peu etudie avec la ref.}
%  . We~mention some references\fbox{TBD} to such extensions at
%the end of this chapter.}.
  As~a result, timed games are not
determined, as illustrated in the following example.

\begin{figure}[ht]
  \centering
  \begin{tikzpicture}[node distance=2cm,auto]
    \tikzstyle{every state}=[minimum size=17pt,inner sep=0pt]
    %\everymath{\scriptstyle}
    \node[state] at (1,0) (A) {$\ell_0$};
    \node[state,right of=A] (B) { $\ell_1$};
    \node[state,left of=A] (C) {$\ell_2$};
    \path[initial] (A.-135) edge +(-135:3mm);
    \path[arrow] 
    (A) edge node[above] {$0<x<1$} (B)
    (A) edge[dashed] node[above] {$0<x < 1$} (C);
  \end{tikzpicture}
  \caption{Timed arena~$\TA_2$. Solid arrow is Eve's, dashed one is Adam's.}
%    Eve's edges are shown with plain arrows, and Adam's with dashed ones.}
  \label{9-fig:ta2}
\end{figure}




\begin{example}[Timed Games are not Determined]
  In the timed arena~$\TA_2$ defined in~\cref{9-fig:ta2}, from configuration~$(\ell_0,\vec{0})$, Eve does not have a winning strategy to reach location~$\ell_1$, but Adam does not have a winning strategy either to avoid~$\ell_1$.
  In fact, available moves for both players consist in~$(d,(\ell_0,\ell_1))$ with~$0< d<1$ for Eve, and~$(d',(\ell_0,\ell_2))$ with~$0< d' < 1$ for~\Adam.
  Thus, for any particular delay~$0<d<1$ chosen by~\Eve, Adam has a possible delay $d<d'<1$ which leads to~$\ell_2$, which is losing for~\Eve.
  This shows that Eve does not have a winning strategy.
  The argument is however symmetric, and Adam also does not have a winning strategy to avoid~$\ell_1$.
  Timed games are thus non determined.
\end{example}




%\medskip
%Although the infinite-state concurrent games we defined above
%precisely capture the semantics of timed games we study,




%% \medskip
%% \NM{rewrite}
%% we consider the game~$(\TA,\Omega)$ in the rest of the chapter.
%% In~particular, we~define a~\emph{play} of~$\calT$ as a play of~$\TA$,
%% and a strategy for a player in~$\calT$ as a strategy for the same
%% player in~$\TA$.

%sequence $\rho = q_1q_2\ldots$ where $q_i \in
%V$ with~$q_1 = (\ell_0,\vec{0})$ and 
%such that for each~$i\geq 1$, 
%there exists~$e_i \in E$ and~$d\geq 0$ with $q_{i+1} = \step(q_i,(d,e_i))$.
%A run is \emph{maximal} if it is an infinite sequence or if it ends in
%$V_{\Eve}$ with no possible successor.
%Finite runs are called \emph{histories}.

%We define a~\emph{strategy} for Eve
%as a function~$f\colon V^* \rightarrow \Realnn \times \EE$
%such that for all histories~$q_1q_2\ldots q_k$, $f(q_1q_2\ldots q_k)$
%is a valid move at~$q_k$.

%The set of \emph{outcomes} of the timed game~$\calT$ \emph{from location~$\ell_0$ under
%strategy~$f$} is defined as the set of maximal runs~$\rho$ such that
%\begin{enumerate}
%  \item $\rho_1 = (\ell_0,\vec{0})$,
%  \item for all~$i\geq 1$, either~$q_{i+1} \in \post(q_i,f(q_1q_2\ldots q_i))$
%\end{enumerate}
%We only consider reachability objectives in this chapter. 
%Given a target
%location~$T \in \Locs$, we
%define~$\Omega(T)$ as the set of maximal runs that visit~$T$ at least once.
%A strategy~$f$ for Eve is \emph{winning} if all outcomes under~$f$ belong to~$\Omega(T)$.


\begin{example}[Winning strategy on running example]
Let us consider again the example of \cref{9-fig:ta1} and see whether Eve
has a winning strategy from the initial configuration.
At the initial configuration~$\ell_1,x=0$, Eve needs to make a move towards~$\ell_2$
while~$x\leq 1$ since whenever~$x>1$, Adam can move to~$\ell_5$ which guarantees Eve's lost.
Assume the game proceeds to~$\ell_2$ with any value~$x\leq 1$. Here, Eve can try to wait
until~$x\geq 2$ and go to~$G$. However, if~$x<1$, then Adam can move to~$\ell_3$.
From this location, Eve can guarantee to come back to~$\ell_2$ with~$x=1$, and then move to~$G$ and win the game.
Assume now that from~$\ell_1$, the game proceeds to~$\ell_3$ with~$x=0$ due to Adam's move.
Again, Eve can wait for 1 time unit, and go back to~$\ell_2$ with~$x=1$, and win the game.
%
Hence, Eve has a winning strategy from $\ell_1$ for all~$x\leq 1$, and from~$\ell_2$
for all values of~$x$.
\end{example}


In this chapter, the main problem we are interested in is determining
whether Eve has a strategy for her reachability objective.
Let~$\vec{0}$ denote the clock valuation assigning~$0$ to all clocks.

\decpb[Solving a timed reachability game]{A timed arena~$\calT$, initial location~$\ell_0$, and a reachability objective~$\Reach(\Win)$}{Does Eve has a winning strategy in~$(\calT,\Reach(\Win))$ from configuration~$(\ell_0,\vec{0})$.}

The difficulty of this problem is that the concurrent
game~$((V,E),\Delta,\Omega)$ has an infinite state-space, and players
have infinitely many actions.  We~thus start by studying a data
structure to represent sets of states and operations to compute
successors and predecessors on these sets.  We then give two
algorithms to solve the above problem.  We~also show how such a
strategy for Eve can be computed and finitely represented.
