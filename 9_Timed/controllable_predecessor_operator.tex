We present the controllable-predecessor operator which, given sets of
states $X$ and~$Y$, returns the set of states from which \Eve can
reach~$X$ in one step, while avoiding states of~$Y$ during \Eve's
delay. Intuitively, the states in~$Y$ are the states from which \Adam
may force an action leading outside of~$X$, which \Eve would better avoid.

%\OS{This operations make sense with the concurrent semantics...}
Recall that $\vertices = \Locs \times \Realnn^\Clocks$.
The set of \emph{""safe time-predecessors""} to reach~$X \subseteq \vertices$ while
avoiding~$Y \subseteq \vertices$
%\NM{$\VE$ or $V$?}
is defined as follows:
%\[
%  \begin{array}{ll}
\begin{multline*}
\predt(X,Y) = \{(\ell,\nu) \in \vertices \mid \exists d\geq
    0.\ (\ell,\nu+d) \in X \wedge{} \\
    \forall d' \in[0,d).\ (\ell,\nu+d') \not  \in Y\}.
\end{multline*}
%  \end{array}
%\]
In words, from any configuration of $\predt(X,Y)$,
the set~$X$ can be reached by
delaying while never crossing any configuration of~$Y$ on the way. 
For any $X \subseteq \vertices$, let us denote
\[
  \predc(X) = \{ (\ell,\nu) \in \vertices \mid \exists e \in \EE(\ell).\ 
\exists (\ell',\nu') \in X.\ (\ell,\nu) \xrightarrow{e} (\ell',\nu')\}.
\]
This is the set of immediate predecessors of~$X$ by edges in~$\EE$.
Symmetrically, we~define $\predu(X)$ using~$\EA$ instead of~$\EE$.
Our controllable-predecessor operator is then defined as 
%
%\NM{Replace $\predc(X)$ with $X\cup\predc(X)$? besoin dans une preuve, je crois. A verifier}
%\NM{C'est bien X cup predc(X) dans le papier [BCF+05]. J'ai corrig\'e}
\[
  \pi(X) = \predt(X\cup \predc(X), \predu(\vertices \setminus X)).
\]

%\NM{!!! Cette approche necessite la semantique ``jeux concurrents''.
%  Reecrire avec jeux concurrents + ajouter remarque ``jeux a tours''}
%\NM{$\calQ$ undefined}
Intuitively, the states in~$\pi(X)$ are those from which \Eve~can wait
until she
can take a controllable transition to~reach~$X$, and so that
%any uncontrollable transition that might happen while waiting remains
%inside~$X$.
no transitions that \Adam could take while \Eve is waiting may lead
outside of~$X$.

\begin{lemma}\label{9-lem:pimonotonic}
  The operator $\pi$ is %non-decreasing
  order-preserving: if~$X\subseteq Y$, then
  $\pi(X) \subseteq \pi(Y)$.  %\NM{useful later}
\end{lemma}

\begin{example}
Consider the timed game to the left of
\cref{9-fig:contpred}. In~that game, \Eve can reach her target when
clock~$x_2$ reaches value~$3$ while~$x_1$ is in~$[1;4]$. However,
\Adam can take the game to a bad state when simultaneously
$x_1\in[1;2]$ and $x_2\leq 2$. The~diagram to the right of
\cref{9-fig:contpred} shows the set of winning valuations for~\Eve:
it~is computed as $\predt(X,Y)$ where $X=\predc(\{W\}\times\Realnn)$
and $Y=\predu(\{\ell_0,\ell_1\}\times\Realnn)$.

\begin{figure}[h]
  \centering
  \begin{tikzpicture}
    \draw (0,1) node[state] (a) {$\ell_0$};
    \draw (0,-1) node[state] (b) {$\ell_1$};
    \draw (3,0) node[state,double] (W) {$W$};
    \draw[arrow] (a) -- (W) node[midway,above right] {$\genfrac{}{}{0pt}{0}{1\leq x_1\leq 4}{{}\wedge x_2=3}$};
    \draw (a) edge[arrow,dashed] (b) node[midway,left] {$\genfrac{}{}{0pt}{0}{1\leq x_1\leq 2}{{}\wedge x_2\leq 2}$};
    \draw[initial] (a.135) -- +(135:3mm);
    %
    %
    \begin{scope}[xshift=5cm,yshift=-1cm,scale=.55]
      \draw[latex'-latex'] (4.5,0) node[right] {$x_1$}
      -| (0,4.5) node[left] {$x_2$};
      \begin{scope}
        \path[clip] (0,0) -- (4.5,0) [rounded corners=10mm]
          -- (4.5,4.5) [rounded corners=0mm] -| (0,0);
        \fill[opacity=.6,dgrey,hatcharea] (1,0) -| (2,2) -| cycle;
        %\draw[line width=1pt,black,opacity=1] (1,0) -| (2,2) -| cycle;
        \fill[dgrey,fillarea] (0,1) -- (1,2) -| (2,1) -- (4,3) -- (1,3) -- (0,2) -- cycle;
        \draw[line width=3.2pt,dgrey] (1,3) -- (4,3);
        \begin{scope}[opacity=.3]
        \foreach \x in {-4,...,4}
                 {\draw (\x,0) -- +(9,9);
                   \draw (0,\x) -- +(9,0);
                   \draw (\x,0) -- +(0,9);}
        \end{scope}
      \end{scope}
%      \path (3,0.5) node (Z) {$Z$} edge[-latex',bend left=20] (1.6,1.4);
%      \path (4,1.5) node (Y) {$\posttime(Z)$} edge[-latex',bend right=20] (2.9,2.7);
      \path (3.5,4) node {$X$} edge[-latex',bend right=12] (2.8,3.2);
      \path (3.5,0.5) node {$Y$} edge[-latex',bend left=12] (2.2,0.6);
      \path (4.5,1.5) node {$\pi(W)$} edge[-latex',bend left=12] (2.6,1.8);
    \end{scope}
  \end{tikzpicture}
  \caption{Controllable predecessors}\label{9-fig:contpred}
\end{figure}
\end{example}
% \begin{lemma}
%   Consider a timed arena~$\calT$.
%   For any~$X \subseteq V$,  %%\mathbb{R}_{\geq 0}^\Clocks$,
%   %%\NM{il faut $X \subseteq \Locs\times \mathbb{R}_{\geq 0}^\Clocks$?}
%   and any configuration~${q \in \vertices}$, it~holds $q \in \pi(X)$ if, and only~if
%   $q \in \mathsf{Attr}(X)$.
% \end{lemma}
% \NM{!!  l'attracteur correspond au point fixe. + enoncer ``$\pi(X)=\mathsf{Attr}(X)$''}

% \begin{proof}
%   .... \NM{(re)definir $\textsf{Attr}_\TA(X)$? Mettre ref section 2.1 ?}
% \end{proof}

% The following theorem is the fixpoint characterization of the winning states
% for~$\Eve$. It is the timed automata analogue to Theorem ....
% \begin{theorem}
%   \label{9-thm:fp}
%   Given a timed game~$(\calT,\Omega)$ with a reachability objective~$\Omega$,
%   a configuration~$q$ is winning for~$\Eve$ if, and only if
%   there exists~$X \subseteq \mathbb{R}_{\geq 0}^\Clocks$ such that
%   $q \in X$ and~$X = \pi(X)$.
%   \NM{Il faut prendre le plus petit pt fixe (sinon $X=\Locs\times\Realnn^{\Clocks}$ convient)}
% \end{theorem}



\begin{remark}
  When considering zero-sum objectives (which we do in this chapter),
a~turn-based arena can be built to decide whether \Eve has a winning
strategy (and~possibly construct~one).  Intuitively, the turn-based
arena is obtained by letting \Eve first pick a pair~$(d,e)$ with~$e
\in \EE$, and then letting \Adam either respect this choice (i.e.,
play a larger delay) or preempt \Eve's action by choosing $(d',e')$
with~$d' \leq d$ and~$e' \in \EA$.
\Eve has a winning strategy in this turn-based arena if, and only if she has one in the original game.
However, this does not apply to \Adam; if he has a winning strategy, this only means that \Eve
does not have a winning strategy in the original game.

%\paragraph{Turn-Based Game Semantics}
Given~$\calT = (\Locs, \Clocks,\EE,\EA, c)$, we define the arena~$\TA
= (\vertices,\VE,\VA,E, c'')$ 
with objective~$\Omega$, where~$\VE = \Locs\times \Realnn^\Clocks$ and
$\VA=\Locs\times\Realnn^\Clocks\times(\Realnn \times \EE
\cup\{\bot\})$.  The~set of successors of configuration~$(\ell,\nu)
\in \VE$ is the set $\{(\ell,\nu,a)\mid a \text{ available at
}(\ell,\nu)\}$.
%  \[
%  \{ (\ell,\nu,(d,e)) \mid e \text{ valid at } \nu+d\},
%  \]
From a configuration~$((\ell,\nu),(d,e))\in\VA$, several cases may
occur:
%\todo{OS: Rephrased. Please check}
\begin{itemize}
\item if \Adam has no available action from~$(\ell,\nu)$, or if he can
  only play actions~$(d',e')$ with $d'>d$, then the only transition
  from~$((\ell,\nu),(d,e))$ goes to $\step((\ell,\nu),(d,e))$;
\item Otherwise, for all actions~$(d',e')$ for \Adam satisfying $d'\leq d$,
  there is a transition from~$((\ell,\nu),(d,e))$
  to $\step((\ell,\nu),(d',e'))$.
  Moreover, if \Adam has an available action~$(d',e')$ with~$d'\geq d$,
  then there is also a transition
  from~$((\ell,\nu),(d,e))$ to $\step((\ell,\nu),(d,e))$.
% \item if \Adam can play an action~$(d',e')$ with $d'\leq d$, then
%   there is a transition from~$((\ell,\nu),(d,e))$
%   to $\step((\ell,\nu),(d',e'))$.
\end{itemize}
%\[
%  \{\step((\ell,\nu),(d,e))\}
%  \cup
%  \{ (\step((\ell,\nu), (d',e')) \mid d' \leq d, e' \in \EA(\ell), e'+d' \models g_{e'}\}.
%\]
From a configuration~$((\ell,\nu),\bot)$, there is a transition
to~$\step((\ell,\nu),(d',e'))$ for each available action~$(d',e')$
of~\Adam.
The coloring function is defined as~$c''((\ell,\nu)) = c(\ell)$
and~$c''((\ell,\nu),(d,e)) = c(\ell)$.
%The function~$\Delta$ is defined so that it induces a bijection between actions
%and the above edges for each player. We do not need its precise definition
%since we will directly refer to edges.


\end{remark}
