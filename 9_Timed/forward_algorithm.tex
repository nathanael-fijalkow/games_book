The backward algorithm we just presented is conceptually simple, but
it~is often not very efficient in practice,
as federations tend to grow too much in size in each iteration of the
computation.
The forward algorithm we will now present is more efficient in practice.
It performs a forward exploration and only applies the controllable predecessor along branches
that actually reach the target state from the initial state. If the witness
trace is not excessively long, which is often the case in practice,
this limits the size of the federations.
%In finite-state systems, this algorithm is known to find small fixpoints

%When dealing with timed systems, backwards algorithms are easy to
%develop and implement, but they are usually not very efficient in
%practice.
%\NM{Give intuition why? Especially, why for timed games?? La
%  justification donnee dans [Cassez, FORMATS 2007] n'est pas hyper
%  convaincante...}
%As~for solving reachability in timed automata, forward algorithms have
%been designed to gain in efficiency.

We present below the algorithm proposed in~\cite{CDFLL05},
and as a first step, we explain the untimed version of that algorithm,
based on an algorithm of Liu and Smolka for computing
fixpoints~\cite{LS98}.


\subsection*{A Forward Algorithm for Finite-State Games}
%Give the full proof.

The original algorithm of Liu and Smolka is expressed in terms of
\emph{(pre-)fixpoints} in \emph{dependency graphs}: a~dependency graph
is a pair~$G=(V,E)$ in which $E \subseteq V \times 2^V$ relates
states with sets of states.
%
For any order-preserving 
function~$f\colon 2^V\to2^V$
(\emph{order-preserving} meaning non-decreasing for the $\subseteq$-relation),
%\todo{Define order-preserving}
a~\emph{pre-fixpoint} is a set~$X\subseteq V$ for which $f(X)\subseteq
X$; it~is a \emph{fixpoint} if $f(X)=X$. By~Knaster-Tarski theorem, such
functions always admit a least pre-fixpoint which is also the least fixpoint.
% Observe that the least
%pre-fixpoint and the least fixpoint coincide.
%

Fix a dependency graph~$G=(V,E)$. 
For~$W\subseteq V$,
we~define a mapping $f_W\colon 2^V\to 2^V$: for
each~$X\subseteq V$, we~let $f_W(X)=W\cup \{v\in V\mid \exists (v,Y)\in
E.\ Y\subseteq X\}$.
%
%By~Knaster-Tarski theorem, $f_W$~admits a least
%pre-fixpoint, which is also its least
%fixpoint.
Clearly, $X\subseteq X'$ implies $f_W(X)\subseteq f_W(X')$, so that $f_W$ admits
a least (pre-)fixpoint.
The~Liu-Smolka
algorithm aims at deciding whether a given vertex~$v_0\in V$ belongs
to the least fixpoint of~$f_W$.  Classical algorithms for computing
least fixpoints consist in iteratively
computing~$(f_W^i(\emptyset))_i$ until convergence (assuming~$V$~is
finite). Observe that this corresponds to the algorithm of \cref{9-thm:timed-control}
for timed games.
The~Liu-Smolka algorithm proceeds from~$v_0$, and explores
the dependency graph until it can claim that~$v_0$~is, or that it
is~not, in the least fixpoint of~$f_W$.


Before tackling the algorithm, let us link least fixpoints in
dependency graphs and winning sets in concurrent games (with
reachability objectives): with a concurrent arena
$\calC=(V,\Act,\delta,c')$ and a target set~$\Win$, we~associate
the dependency graph $G=(V,E)$, where $(v,T)\in E$ whenever $v\in V$
and $T\subseteq V$ is such that there exists an action~$a$ for which
$T=\{v' \mid \exists a'\in\Act.\ v'=\delta(v,a,a')\}$.  Then for any
set~$X\subseteq V$, the~set~$f_{\Win}(X)$ contains $\Win$ and all the
states from which Eve can force a visit to~$X$ in one step.
We~then~have:
\begin{proposition}\label{9-prop:fixp-game}
The least fixpoint of~$f_{\Win}$ in~$G$ corresponds to the set~$W$ of
winning states for~\Eve in~$\calC$.
\end{proposition}

\begin{proof}
  The winning states of~\Eve form a pre-fixpoint of~$f_{\Win}$
  containing~\Win: indeed, for any~$v\in f_{\Win}(W)$, either $v\in
  \Win$, or \Eve has an action to move from~$v$ to some state
  in~$W$. Hence $v$ is winning, i.e., $v\in W$.
  
%  indeed, in each winning state, Eve~has an action whose successors
%  are all winning. Hence all the states in the least fixpoint of~$f_{\Win}$ 
%  are winning for~Eve.

  Conversely, from any state~$v$ that is not in the least
  pre-fixpoint~$X$, for any~edge~$(v,T)$, there is a state~$v'\in T$
  that again is not in~$X$. This defines a strategy for \Adam to avoid
  reaching~$\Win$, so that \Eve does not have a winning strategy from~$v$.
\end{proof}



\begin{algorithm}
  \SetKwFunction{Pop}{pop}\SetAlgoNoEnd
  \KwData{A dependency graph~$G=(V,E)$, a~set~$W\subseteq V$, a~node $v_0\in V$}
  \KwResult{Is $v_0$ in the least fixpoint of~$f_W$?}

  %\For{$v\in V$}{$F(v):=\bot$}
  %$F(v_0):=0$;
  \For{$v\in V$}{\lIf{$v\in W$}{$F(v):=1$}
    \lElseIf{$v==v_0$}{$F(v):=0$}
            \lElse{$F(v):=\bot$}}
  
  $\Dep(v_0):=\emptyset$;

  $\Wait:=\{(v,T)\in E \mid v=v_0\}$;

  \While{($\Wait\not=\emptyset$ and $F(v_0)==0$)}
        {$(v,T):=\Pop{\Wait}$;\\
          \If(\hfill{// case 1}){$F(v)==0$ and $\forall v'\in T.\ F(v')==1$}
             {$F(v):=1$;\\
              $\Wait:=\Wait\cup\Dep(v)$;}
          \ElseIf(\hfill{// case 2}){$\exists v'\in T.\ F(v')==0$}
                {$\Dep(v'):=\Dep(v') \cup \{(v,T)\}$;}
                \ElseIf(\hfill{// case 3}){$\exists v'\in T.\ F(v')==\bot$}
                  {$F(v'):=0$; \\
                  $\Dep(v'):=\{(v,T)\}$;\\
                  $\Wait:=\Wait\cup\{(w,U)\in E \mid w=v'\}$;
                  }
                  }
                  \Return{$F(v_0)$}
  \caption{Liu-Smolka algorithm\protect\footnotemark for least fixpoint of~$f_W$}
  \label{9-algo:LS98}
\end{algorithm}
\footnotetext{Our version of the algorithm slightly differs from the
  original one~\cite{LS98}: we~let $\Dep(v'):=\{(v,T)\}$ at the penultimate line
  of the \textbf{while} loop (which otherwise results in a wrong
  result, as~already noticed in~\cite{JLSO13}), reinforce the
  condition for case~$1$ (which otherwise would not guarantee
  termination), and reinforce the condition of the \textbf{while} loop
  to get earlier termination.}
%% counter example for termination: V={1}, E={e0=(1,{1}),e1=(1,empty)}.
%% after init:       Wait={e0,e1}, F(1)=0, Dep(1)=empty
%% first run:  e=e0: Wait={e1}, F(1)=0, Dep(1)={e0}
%% second run: e=e1: F(1)=1, Wait={e0}
%% third run:  e=e0: F(1)=1, Wait={e0}...

\Cref{9-algo:LS98} can be seen as an alternation of forward
exploration and backward propagation. Intuitively, the algorithm first
explores the graph in a forward manner, remembering for each node~$v$
the set~$\Dep(v)$ of nodes that \emph{depend} on~$v$, and have to be
reexplored if the status of~$v$ is updated.
%\todo{Clear enough?}
%the parent of
%each explored vertex thanks to~$\Dep(\cdot)$.
For~each~$v$, the algorithm maintains a
value~$F(v)$, which is~$\bot$ if~$v$ has not been explored~yet,
$0$~if~$v$~has been explored but not yet been shown to be winning, %belong to~$S$,
and~$1$ if~$v$ is known to be winning. % During exploration of an
% edge~$e=(v,T)$, if some vertex~$v'\in T$ has~$F(v)=0$, then~$e$ is
% added to the set~$\Dep(v')$ of edges that will have to be re-explored
%The function~$F$ is used to mark whether a given vertex has not been explored yet ($\bot$),
%has been explored but not known to be winning~($0$), or known to be winning~($1$).
Whenever a vertex~$v$ whose all successors are winning is found, the value of~$v$ is set to~$1$, and its parents (in~$\Dep(v)$)
are scheduled to be visited again to check whether their statuses have to be changed.
This is how the backward propagation is triggered; in~fact, the~search will climb in the tree as long as the values of
vertices can be updated to~$1$.
%\todo{Benjamin: est-ce que vous pourriez expliquer l’utilisation du symbole Dep pour les parents des sommets explorés ?}
%Thus, the backward propagation is only triggered when the value of some node becomes~$1$ so as to update its parents in the exploration.


% \Cref{9-algo:LS98} decides if vertex~$v_0$ belongs to the least
% fixpoint of~$f_W$. For~each~$v$, the algorithm computes a
% value~$F(v)$, which is~$\bot$ if~$V$ has not yet been explored,
% $0$~if~$v$~has been explored but not yet been shown to belong to~$S$,
% and~$1$ if~$v$ belongs to~$S$. During exploration of an
% edge~$e=(v,T)$, if some vertex~$v'\in T$ has~$F(v)=0$, then~$e$ is
% added to the set~$\Dep(v')$ of edges that will have to be re-explored
% in case~$F(v')$ is eventually set to~$1$. Notice that for any~$v\in
% V$, the value of~$F(v)$ may not decrease (assuming $\bot < 0 < 1$).
%\NM{Should we say more? We are just rephrasing the algo...}

 %% \NMlong{Add intuition: when node is evaluated, value set to~$0$, and
 %%   outgoing hyper-edges added in $\Wait$. When an hyper-edge is
 %%   evaluated (taken from~$\Wait$), if some successor has value~$0$,
 %%   the hyper-edge is added to list of edges to be re-evaluated if the
 %%   value of that node changes. Reevaluation is enforced by adding
 %%   $\Dep(v)$ to $W$ when setting value of~$v$ to~$1$.}


The correctness of this algorithm relies on the following lemma:
\begin{lemma}[\cite{LS98}]\label{9-lemma:ls98}
  The following properties hold at the end of each run in the
  \textbf{while} loop of \Cref{9-algo:LS98}:
  \begin{itemize}
  \item for any $v\in V$, if $F(v)=1$, then $v$ belongs to the least
    fixpoint containing~$W$;
  \item for any $v\in V$ with $F(v)=0$, and any $(v,T)\in E$, either
    $(v,T)\in \Wait$ or $(v,T)\in\Dep(v')$ for some $v'\in T$ with
    $F(v')=0$.
  \end{itemize}
\end{lemma}

\begin{proof}
  We~fix an execution of \Cref{9-algo:LS98}, and prove that both
  claims are true at the beginning and end of each iteration through
  the \textbf{while} loop. To~clarify the presentation, we~use
  superscript~$i$ to indicate the value of variables at the end of the
  $i$-th run through the loop, so that $\Wait^3$ is the value of
  variable~$\Wait$ after three iterations (and $\Wait^0$ is its value
  after initialization). In~particular, $(v^i,T^i)$ is the
  symbolic transition popped from the $\Wait$ list during the $i$-th
  iteration (and is not defined for~$i=0$).

  Let us first prove the first property: the initialization phase
  clearly enforces that if $F^0(x)=1$, then $x\in W$, which is
  included in any fixpoint of~$f_W$. Now, assume that the property
  holds true at the beginning of the $i$-th run through the loop
  (i.e.,~if~$F^{i-1}(x)=1$, then $x$ is in the least fixpoint), and
  pick some~$x\in V$ such that $F^i(x)=1$. If~$F^{i-1}(x)=1$, our
  result follows; if~not, then the $i$-th iteration of the loop must
  have run via case~$1$, hence $F^{i-1}(x')=1$ for all~$x'\in
  T^i$. By~our induction hypothesis, this indicates that $T^i$ is part
  of the least fixpoint, and by definition of~$f_W$, $x$~must also
  belong to the least fixpoint.

  %% After initialization, the first statement clearly holds. Now,
  %% consider a run of the loop after popping hyper-edge~$(v,T)$
  %% from~$\Wait$. Pick any~$v\in V$ such that~$F(v)=1$: if it was
  %% already the case before running the loop, then by induction $v$
  %% belongs to the least post-fixpoint containing~$R$.  Otherwise,
  %% $F(v)$~has just been set to~$1$; this may only happen if $F(v')=1$
  %% for all~$v'\in T$ before running the loop. By~induction, this
  %% indicates that $T$ is included in the least post-fixpoint
  %% containing~$R$; by~definition of least fixpoints, $v$~also belongs
  %% to that least post-fixpoint.

  \smallskip
  The second statement also clearly holds after initialization:
  initially, $F^0(x)=0$ only for~$x=v_0$, and all transitions
  from~$v_0$ have been stored in~$\Wait^0$. We~now assume that the
  property holds when entering the \textbf{while} loop for the $i$-th
  time, and consider $x$ such that $F^i(x)=0$ at the end of that
  loop. We~pick $(x,Y)\in E$.
  \begin{itemize}
  \item if the $i$-th run through the loop visits case~$1$, then
    already $F^{i-1}(x)=0$ (hence $x\not=v^i$). Then either
    $(x,Y)\in\Wait^{i-1}$, or $(x,Y)\in\Dep^{i-1}(x')$ for some~$x'\in Y$
    with $F(x')=0$. In~the former case: since $x\not=v^i$,
    if~$(x,Y)\in\Wait^{i-1}$ then also $(x,Y)\in \Wait^i$;
%    %, since
%    %$\Wait^i=(\Wait^{i-1} \setminus\{(v^i,T^i)\})\cup
%    %\Dep^{i-1}(v^i)$.
    in the latter case: if $(x,Y)\in\Dep^{i-1}(x')$ for some~$x'\in Y$
    with~$F(x')=0$, then \emph{(i)}~either $x'=v^i$, and
    $(x,Y)\in\Wait^i$ because $\Dep^{i-1}(v^i)\subseteq \Wait^i$
    (last~line of case~$1$), \emph{(ii)}~or~${x'\not=v^i}$, and
    $(x,Y)\in \Dep^{i-1}(x')=\Dep^{i}(x')$ and $F^i(x')=F^{i-1}(x')=0$.

  \item if the $i$-th run goes to case~$2$, then again $F^{i-1}(x)=0$,
    and the induction hypothesis applies: either $(x,Y)$ is
    in~$\Wait^{i-1}$, or~it is in~$\Dep^{i-1}(x')$ for some~$x'\in Y$
    with $F^{i-1}(x')=0$. For the latter case, observe that
    $\Dep^{i-1}(x')\subseteq \Dep^i(x')$, and $F^{i}(x')=F^{i-1}(x')$
    when running case~$2$; for the former case, we~have
    $\Wait^i=\Wait^{i-1}\setminus\{(v^i,T^i)\}$, so that $(x,Y)$
    remain in~$\Wait^i$ if~$(x,Y)\not=(v^i,T^i)$; finally,
    if~$(x,Y)=(v^i,T^i)$, then case~$2$ precisely adds~$(v^i,T^i)$
    to~$\Dep^i(v')$ for some~$v'\in T^i$ with~$F^i(v')=0$, which concludes
    the proof for this case.

  \item for case~$3$, we~first consider the case where $x$ is the
    state~$v'$ selected at the beginning of case~$3$: in~that case,
    all transitions from~$x$ are added to~$\Wait$, so that
    $(x,Y)\in\Wait^i$. Now, if~$x$ is not the selected state~$v'$,
    then $F^{i-1}(x)=0$, and again either $x\in\Wait^{i-1}$ or
    $x\in\Dep^{i-1}(x')$ for some~$x'\in Y$ with
    $F^{i-1}(x')=0$. The~latter case is preserved when running
    case~$3$; if~$x\in\Wait^{i-1}$: if~$(x,Y)\not=(v^i,T^i)$, then
    $x\in\Wait^i$, while if~$(x,Y)=(v^i,T^i)$, then
    $(x,Y)\in\Dep^i(v')$ for the state~$v'\in T^i$ selected at the
    beginning of case~$3$ (and~for which $F^i(v')=0$).
  \end{itemize}
%  
  %% We~now prove the second statement. Again, it~is enforced by the
  %% initialization step, hence it holds when entering the \textbf{while}
  %% loop. Consider a run of the loop after popping~$(v,T)$ from~$\Wait$,
  %% and pick any~$w\in V$ with~$F(w)=0$, and an
  %% hyper-edge~$(w,U)$.
  %% \begin{itemize}
  %% \item if~the loop went through case~$1$, then already $F(w)=0$
  %%   before running the loop.  By~induction, before running the loop,
  %%   either $(w,U)\in \Wait$, or $(w,U)\in\Dep(w')$ for some~$w'$ with
  %%   $F(w')=0$. Since~$(v,T)\not=(w,U)$ (because $F(v)\not=F(w)$ after
  %%   running the loop), the former case is preserved. In~case
  %%   $w'\not=v$, the second case is preserved as~well; if~$w'=v$,
  %%   $\Dep(w')$ is added to~$\Wait$, so that $(w,U)\in\Wait$ at the end
  %%   of the loop.
%
  %% \item if the loop visited case~$2$, then again $F(w')=0$ must have
  %%   held before running through the loop. Thus at that time, either
  %%   $(w,U)\in\Wait$, or $(w,U)\in\Dep(w')$ for some~$w'$ with
  %%   $F(w')=0$. The~latter property is preserved when running through
  %%   case~$2$; the~former property may be lost in case $(w,U)=(v,T)$,
  %%   but then $(v,T)$ is added to some~$\Dep(v')$ for some~$v'$
  %%   with~$F(v')=0$, which restores the global property.
%
  %% \item if the loop goes to case~$3$, then $F(v')$ is set to~$0$ for
  %%   some~$v'$. If~$v'=w$, then all hyper-edges from~$w$ are added
  %%   to~$\Wait$, and the result follows.  If~$w\not=v'$, then $F(w)=0$
  %%   before running the loop, and at that time either $(w,U)\in\Wait$
  %%   or $(w,U)\in\Dep(w')$ for some~$w'$ with~$F(w')=0$. The~latter
  %%   case is preserved when running the loop; the~former case is
  %%   preserved except if $(w,U)=(v,T)$, but this is solved with the
  %%   newly created $\Dep(v')$.
  %% \end{itemize}
\end{proof}

As a corollary, if the algorithm terminates after~$n$ rounds of the
\textbf{while} loop, then either~$F^n(v_0)=1$, or
$F^n(v_0)=0$ and $\Wait^n=\emptyset$.
\begin{itemize}
\item From the first claim of the lemma above, the former case entails
  that $v_0$ belongs to the least fixpoint.
\item Now consider the second case, and let $B=\left\{v\in V\mid
  F^n(v)\in \{\bot,1\}\right\}$. We prove that $f_W(B)\subseteq
  B$. For this, we~pick $v\in f_W(B)$:
  \begin{itemize}
  \item if~$v\in W$, then $F(v)$~is set to~$1$ initially, and may
    never be changed, so that $v\in B$;
  \item otherwise, there is a transition $(v,T)$ such that $T\subseteq
    B$. If~$v\notin B$, then $(v,T)\in\Dep^n(v')$ for some~$v'\in T$
    with~$F^n(v')=0$, which contradicts the fact that $T\subseteq B$.
  \end{itemize}
  This proves that $f_W(B)\subseteq B$, so that $B$ is a pre-fixpoint
  of~$f_W$. It~thus contains the least (pre-)fixpoint of~$f_W$, so
  that any state~$v$ not in~$B$ (i.e., with $F^n(v)=0$) for sure does
  not belong to that least fixpoint. In~particular, $v_0$ is not in
  the fixpoint.
\end{itemize}

%% the set of
%% vertices~$v$ having~$F(v)\in\{\bot,1\}$ is a post-fixpoint
%% of~$G$. Indeed, assume that for some hyper-edge~$(v,S)$, it~holds
%% $0\notin F(S)$ and $F(v)=0$. Since $\Wait$ is empty, $(v,S)$ must
%% \OS{$\Wait$ is not necessarily empty when the alg. terminates.
%% I don't think the result is always a postfixpoint. It is if we continue until $\Wait$ is empty.}
%% belong to some~$\Dep(v')$ for some~$v'\in S$ with $F(v')=0$,
%% contradicting the fact that $0\notin F(S)$. 

%% It~follows that for any~$v$ with~$F(v)=0$, then $v$~cannot belong to
%% the least post-fixpoint of~$G$. Conversely, if~$F(v)=1$, then
%% $v$~belongs to the least post-fixpoint. Hence, if the algorithm
%% terminates, it~decides whether~$v_0$ belongs to the least
%% post-fixpoint.

\medskip
It~remains to prove termination. For this, we~first notice that, for
any hyper-edge~$(v,S)$, if $(v,S)\in\Dep^i(v')$ then $(v,S)\in\Wait^j$
for some~$j<i$, and if $(v,S)\in\Wait^j$ or $(v,S)\in\Dep^j(v')$ for
some~$v'$, then $F^j(v)\not=\bot$.

We~then set
\[
M= 
2\size{\Wait} + 
2\hskip-4pt\sum_{v\text{ s.t. }F(v)=\bot}\hskip-4pt \size{\{(v,S)\in E\}} 
+ \hskip-4pt\sum_{v\text{ s.t. }F(v)=0}\hskip-4pt \size{\Dep(v)}
- \hskip-4pt\sum_{v\text{ s.t. }F(v)=1}\hskip-4pt \size{\Dep(v)}
%\sum_{(v,S)\in\Wait}(|S|+1) + \sum_{v\text{ s.t. } F(v)=\bot} \sum_{(v,S)\in E} (|S|+1)
%+ \sum_{v\mid F(v)=0} \sum_{(v',S)\in \Dep(v)} (|S|+1).
\]
again writing~$M^i$ for the value of~$M$ after the $i$-th run through
the \textbf{while} loop.  This value is at most~$2\size{E}$ when
entering the \textbf{while} loop for the first time; clearly, it~can
never go below~$-\size{E}$. We~now prove that $M^{i}<M^{i-1}$,
%it decreases at each run through the \textbf{while} loop,
which implies termination of the algorithm.

Consider the $i$-th run through the loop, popping~$(v^i,T^i)$
from~$\Wait^{i-1}$ (which decreases~$M$ by~$2$). We~consider all three cases:
\begin{itemize}
\item if~case~$1$ applies, $F^i(v)$ is set to~$1$.
  The~set~$\Dep^{i-1}(v)$ is added to~$\Wait^i$. This globally
  leaves~$M$ unchanged, so that globally $M^i=M^{i-1}-2$;
\item if~case~$2$ applies, $\Dep^{i-1}(v')$~is augmented by (at~most) one edge,
  and since $F^i(v')=0$, this increases~$M$ by at most~$1$. Hence globally
  $M^i\leq M^{i-1}-1$;
\item finally, case~$3$ increases~$M$ by~$1$: the edges added
  to~$\Wait$ are compensated by the fact that~$F(v')$ no longer
  is~$\bot$, and only one extra edge has to be considered in the
  second sum. Hence again $M^i\leq M^{i-1}-1$.
\end{itemize}
This concludes the correctness proof of~\Cref{9-algo:LS98}.

\begin{theorem}
\Cref{9-algo:LS98} terminates, and returns~$1$ if, and only~if,
$v_0$~belongs to the least fixpoint of~$f_W$.
\end{theorem}

Using~\cref{9-prop:fixp-game}, we~get a \emph{forward} algorithm for
deciding if a given state of a concurrent game is winning
for~Eve. This corresponds to the OTFUR algorithm
of~\cite{CDFLL05}.


%% \NM{stopped here}
%% \medskip
%% This algorithm can be turned into an algorithm for deciding the winner
%% from a some state~$q_0$ in a two-player finite-state
%% \fbox{concurrent}\NM{Il faudra adapter les notations, ou se
%%   restreindre \`a des jeux \`a tours (mais les jeux temporis\'es sont
%%   concurrents...)}
%% \OS{En fait cet algo n'est pas appliqu\'e directement \`a la s\'emantique qu'on donne aux jeux temp.
%% mais \`a un jeu symbolique sur les zones...}
%% reachability game~$\game$: indeed, such a game can
%% be turned into a dependency graph~$G$ by creating, for each vertex~$v$ and
%% each possible move~$m$ of~Eve, an~hyper-edge~$(v,T)$ where~$T$ is the
%% set of possible successors of~$v$ when Eve plays~$m$. Notice that
%% there are no hyper-edges~$(v,\emptyset)$ in the resulting depencency
%% graph.

%% Using~\Cref{9-algo:LS98}, we~can compute the least post-fixpoint
%% of~$G$ containing all vertices coloured~\Win.  Clearly, the set of
%% vertices where Eve wins the game is a post-fixpoint containing~\Win,
%% so that all vertices in the least such post-fixpoint are
%% winning. Conversely, if a vertex does not belong to the least
%% post-fixpoint, we~can inductively build a run from that vertex that
%% never reaches a vertex coloured by~\Win, hence such a vertex is not
%% winning. Hence the least post-fixpoint containing~\Win exactly
%% corresponds to the set of winning vertices for~Eve.  Notice that this
%% approach is very similar to the OTFUR algorithm
%% of~\cite{CDFLL05}. 

%% We~end up with~\Cref{10:algo-otfur}, which corresponds\NM{again
%%   with a few minor corrections...} to the OTFUR algorithm
%% of~\cite{CDFLL05}.  \NM{adapter les notations pour jeux ?
%%   Rephraser un peu plus pr\'ecis\'ement ?}  \NM{explain why
%%   post-fixpoints corresponds to winning states?}

%% \begin{algorithm}
%%   \SetKwFunction{Pop}{pop}\SetAlgoNoEnd
%%   \KwData{A reachability game~$\game=(\arena, \Reach(\Win))$, a~node $v_0\in V$}
%%   \KwResult{Is $v_0$ winning for Eve?}

%%   %\For{$v\in V$}{$F(v):=\bot$}
%%   $\Passed=\{v_0\}$;  \hfill // $\Passed$ is intended to contain vertices~$v$ for which $F(v)\not=\bot$

%%   \For{$v\in V$}{\leIf{$c(v)==\Win$}{$F(v):=1$}{$F(v):=0$}}

%%   $\Dep(v_0):=\emptyset$;

%%   $\Wait:=\{(v_0,m,T)\in E \mid m\in A \fbox{NM: A=Actions}, T=\textsf{Next}(v,m)\}$;

%%   \While{($\Wait\not=\emptyset$ and $F(v_0)==0$)}
%%         {$(v,m,T):=\Pop{\Wait}$;\\
%%           \If(\hfill{// case 1}){$F(v)==0$ and $\forall v'\in T.\ (F(v')==1$}
%%              {$F(v):=1$;\\
%%               $\Wait:=\Wait\cup\Dep(v)$;}
%%           \ElseIf(\hfill{// case 2}){$\exists v'\in T.\ F(v')==0$}
%%                 {$\Dep(v'):=\Dep(v') \cup \{(v,T)\}$;}
%%                 \ElseIf(\hfill{// case 3}){$\exists v'\in T.\ v'\notin\Passed$}
%%                   {$\Passed:=\Passed \cup\{v'\}$; \\
%%                   $\Dep(v'):=\{(v,m,T)\}$;\\
%%                   $\Wait:=\Wait\cup\{(v',n,U)\in E \mid n\in A\fbox{Actions}, U=\textsf{Next}(v',n)\}$;
%%         }
%%         }  
%%   \caption{On-the-fly algorithm for untimed reachability games}
%%   \label{9-algo:otfur}
%% \end{algorithm}


\subsection*{Extension to Timed Games}
%% Presentation without proofs.
%% To simplify, we can avoid using extrapolation if we bound the clocks.
%% There are a few lemmas from CONCUR'05 which help to compute the cpre operator using
%% standard operations and complementation.

We now explain how to adapt the algorithm above to (infinite-state)
timed games. 
%
For~efficiency, the~algorithm relies on zones (and~DBMs); instead of
computing whether a given zone~$(\ell,Z)$ is winning, the~algorithm
maintains, for each~zone~$S=(\ell,Z)$ it~explores, a~subzone~$(\ell,Z')$ of
configurations that it knows are winning; this subzone is stored as
$F(S)$, and is updated during the execution.
As~in~\cref{9-algo:LS98}, a~waiting list keeps track of the
zones to be explored, and a dependency list stores the list of nodes
to be revisited upon update of the winning subzone of a zone.
The algorithm is given in~\cref{9-algo:sotftr}.
%by
%computing for each explored zone~$Z$ of the timed game,
%a~subzone~$W\subseteq Z$ of winning
%configurations. \Cref{9-algo:sotftr} follows the same ideas as
%\Cref{9-algo:LS98}, using forward exploration of zones and
%re-evaluating transitions upon update of the winning subzone.

%% The~procedure we present at~\cref{9-algo:sotftr} may in general not terminate,
%% since the number of zones is not finite. However, 
%% %Because there are infinitely many zones, the algorithm may not
%% %terminate, but
%% classical workarounds apply here, such as bounding all
%% clocks using invariants~\cite{ref-boundedclocks} or applying classical
%% extrapolation techniques~\cite{ref-extrapol}.


\begin{algorithm}
  \SetKwFunction{Pop}{pop}\SetAlgoNoEnd
  \KwData{A reachability timed game~$\game=(\arena, \Reach(\Win))$,
    a~location $\ell_0\in \Locs$}
  \KwResult{Is $(\ell_0,\mathbf{0})$ winning for Eve?}

  $S_0:=\posttime(\ell_0,\mathbf 0)$;
  
  
  %\For{$v\in V$}{\leIf{$c(v)==\Win$}{$F(v):=\Realnn$}{$F(v):=\emptyset$}}
  %// $F$ defined on $V$ and on $V x zones$.

  \leIf{$c(S_0)==\Win$}{$F(S_0):=S_0$}{$F(S_0):=\emptyset$}

  $\Passed:=\{S_0\}$;  \hfill
  // $\Passed$ stores all configurations for which $F$ is defined
    %not marked with~$\bot$

  
  $\Dep(S_0):=\emptyset$;

  $\Wait:=\{(S_0,\alpha,T) \mid 
    T=\posttime(\postta_{\alpha}(S_0))\not=\emptyset, \alpha\text{ transition of }\game\}$;

  \While{($\Wait\not=\emptyset$ and $(v_0,\mathbf 0)\notin F(S_0)$)}
        {$(S,\alpha,T)):=\Pop{\Wait}$;\\
          \If(\hfill{// case $A$}){$T\in\Passed$}
          {$\Dep(T):=\Dep(T) \cup \{(S,\alpha,T)\}$;\\
          $W:=\predt\left(F(S)\cup \bigcup_{\substack{S\xrightarrow{c} V\\ \!\!V\in\Passed\!\!}} \predc(F(V)),\ \bigcup_{\substack{S\xrightarrow{u} V\\ \!\!V\in\Passed\!\!}}\predu(V\setminus F(V))\right) \cap S$;\\
          \If{$F(S) \subsetneq W$}{$F(S):=W$;\\
              $\Wait:=\Wait\cup\Dep(S)$;}}
          \Else(\hfill{// case $B$})
                  {$\Passed:=\Passed \cup\{T\}$; \\
                    \If{$c(T)==\Win$}{$F(T):=T$\\ $\Wait:=\Wait\cup\{(S,\alpha,T)\}$}
                       \Else{$F(T):=\emptyset$}
                  $\Dep(T):=\{(S,\alpha,T)\}$;\\
                  $\Wait:=\Wait\cup\{(T,\alpha,U) \mid U=\posttime(\postta_{\alpha}(T))\not=\emptyset, \alpha\text{ transition of }\game\}$;
        }
        }
  \leIf{$(v_0,\mathbf 0)\in F(S_0)$}{\Return 1}{\Return 0}

        \caption{Symbolic on-the-fly algorithm for timed reachability}
  \label{9-algo:sotftr}
\end{algorithm}




The correctness of the algorithm can be proven using the following
lemma. We~omit the proof of this lemma, as it is tedious and and does
not contain any difficult argument.

\begin{lemma}\label{9-lem:sotftr}
  The following properties hold at the end of each run through
  the \textbf{while} loop of~\Cref{9-algo:sotftr}:
  \begin{itemize}
  \item for any $S\in\Passed$ and any transition~$\alpha$, if
    $T=\posttime(\postta_\alpha(S))\not=\emptyset$,
    then either $(S,\alpha,T)\in\Wait$, or
    $T\in\Passed$ and $(S,\alpha,T)\in\Dep(T)$;
  \item for any~$S\in\Passed$ and $q\in F(S)$,  $q$ is winning for~Eve;
  \item for any $S\in\Passed$ and $q\in S\setminus F(S)$,
    either
    %the $\Wait$ list is not empty, 
    $\Wait$ contains a symbolic transition~$(S,\alpha,S')$ from~$S$
    with~$S'\in\Passed$,
    %\todo{is it normal that this condition does not depend on~$q$?},
    %% normal : ca veut dire ``le calcul n'est pas fini''
    %\fbox{Condition removed; false initially: with $S'\in\Passed$},
    or
    \[
      q\notin \predt\Bigl(F(S)\cup \bigcup_{\substack{S\xrightarrow{c} V\\ V\in\Passed}} \predc(F(V)),\ \bigcup_{\substack{S\xrightarrow{u} V\\ V\in\Passed}} \predu(V\setminus F(V))\Bigr) \cap S.
    \]
  \end{itemize}
\end{lemma}
The proof is omitted but can be found in~\cite{CDFLL05}.
%\todo{Il faudrait citer la version HAL,
%je ne sais pas comment le citer correctement \href{}{https://hal.archives-ouvertes.fr/hal-00350475} sans dupliquer la version concur}
%% \NM{Preuve longue et ennuyeuse. Je propose de ne pas la mettre (mais je voulais l'\'ecrire plus en d\'etail que dans le papier original...).}
%% \begin{proof}
%%   We pick an execution of \Cref{9-algo:sotftr}, and prove all three
%%   properties on this execution. We~use the same notations as or the
%%   proof of \Cref{9-lemma:ls98}, superscripting variable names
%%   with~$i$ to represent the values of those variables after $i$ runs
%%   through the \textbf{while} loop.  We~also write
%%   $(S_i,\alpha_i,T_i)$ for the item popped from~$\Wait_{i-1}$ during
%%   the $i$-th run through the loop.

%%   We~first observe that our algorithm enforces that
%%   $\Passed^{i}=\Passed^{i-1}\cup\{T^i\}$, so that the sequence
%%   $(\Passed^i)_i$ is non-decreasing (for~inclusion), and so are the
%%   sequences $(\Dep^i(S))_i$ and $(F^i(S))_i$, as~soon
%%   as~$S\in\Passed^i$.
%% %$\Passed^{i}\subseteq \Passed^{i+1}$, for any~$S\in\Passed^i$,
%% %$\Dep_i(S)\subseteq \Dep_{i+1}(S)$, and $F_i(S)\subseteq F_{i+1}(S)$.
%% Notice also that $(S,\alpha,T)\in\Wait_i$ implies $S\in\Passed_i$.

%% \smallskip
%% We~begin with proving the first property: notice that it holds true
%% before entering the loop for the first time, since
%% $\Passed^0=\{S_0\}$, and by definition of~$\Wait^0$.  We~then proceed
%% by induction: assuming that the result holds for index~$i$,
%% we~prove~it for index~$i+1$: take~$S\in\Passed_{i+1}$,
%% and a~transition~$\alpha$ such that $T=\posttime(\postta_\alpha(S))$ is
%% non-empty.
%% \begin{itemize}
%%   \item If~$S\in\Passed_i$, by induction hypothesis we~have either
%%     $(S,\alpha,T)\in\Wait_i$, or $T\in\Passed_i$ and $(S,\alpha,T)\in
%%     \Dep_i(T)$. In~the latter case, we~also have $T\in\Passed_{i+1}$
%%     and $(S,\alpha,T)\in \Dep_{i+1}(T)$, as both sets may only
%%     grow. In~the former case, we~also have
%%     $(S,\alpha,T)\in\Wait_{i+1}$ if
%%     $(S,\alpha,T)\not=(S_i,\alpha_i,T_i)$; finally, if
%%     $(S,\alpha,T)=(S_i,\alpha_i,T_i)$, we~then have
%%     $T_i\in\Passed_{i+1}$, and $(S_i,\alpha_i,T_i)\in\Dep_{i+1}(T_i)$,
%%     whichever case ($A$~or~$B$) of the algorithm applies.
%%   \item If $S\notin\Passed_i$, then $S=T_{i+1}$, and case~$B$ of the loop
%%     applies. Then by the last line of the algorithm, we~have
%%     $(S,\alpha,T)\in\Wait_{i+1}$.
%% \end{itemize}

%% %% Consider $T\in\Passed$, a~transition~$\beta$, and
%% %% $T'=\posttime(\postta_\beta(T))$.  By~hypothesis, either
%% %% $(T,\beta,T')\in\Wait$, or $T'\in\Passed$ and $(T,\beta,T')\in
%% %% \Dep(T')$. Notice that $\Passed$ and $\Dep(T')$ may only grow during
%% %% the execution of the loop; hence the latter case is preserved. We~now
%% %% consider the former case, where
%% %% $(T,\beta,T')\in\Wait$. If~$(T,\beta,T')\not=(S,\alpha,S')$, then the
%% %% property is preserved. Otherwise, one easily checks that after both
%% %% cases~$A$ and~$B$, $T'\in\Passed$ and $(T,\beta,T')\in\Dep(T')$.

%% \smallskip
%% The second property also clearly holds for~$i=0$. Assume it~holds at
%% some step~$i$, and pick~$S\in\Passed_{i+1}$ and~$q\in F_{i+1}(S)$.
%% \begin{itemize}
%%   \item First assume $S\in\Passed_i$. Then either~$q\in F_i(S)$, and
%%     our result follows from the induction hypothesis, or $q\in
%%     W_{i+1}$. In~that case, again by induction hypothesis,
%%     the~set~$F_i(S_i)$, as~well as all the sets~$F_i(T)$ for which
%%     $S\to T$ and $T\in\Passed_i$, only contain
%%     configurations that are winning for~Eve, and so does~$W_{i+1}$.
%%   \item In case $S\notin\Passed_i$, then we~have $S=T_{i+1}$, and
%%     case~$B$ applies. The result is then immediate from the definition
%%     of~$F_{i+1}(T_{i+1})$ in that case.
%% \end{itemize}

%% %% We turn to the second property, and pick $q\in F(S)$ for
%% %% some~$S\in\Passed$ after having popped~$(S,\alpha,S')$ from~$\Wait$
%% %% and run through the loop. Assume that case~$A$ applies: by~hypothesis,
%% %% for any~$T\in\Passed$, $F(T)$~is defined, and any~$q\in F(T)$ is
%% %% winning for~Eve. Hence the set~$W$ computed in case~$A$ also only
%% %% contains configurations that are winning for~Eve\NM{should/do we prove
%% %%   this in previous sections?}. The~result follows for this case. Now,
%% %% assume case~$B$ applies, and pick~$T\in\Passed$; if $T\not=S'$, then
%% %% already $T$ was in $\Passed$ before running the loop, and $F(T)$ has
%% %% not been modified, so that the property is preserved. Finally, if
%% %% $T=S'$, then by definition of~$F(S')$ in case~$B$, the result is
%% %% immediate.

%% \smallskip
%% We~now turn to the third property.
%% For any index~$i$ and any $S\in\Passed^i$, we~let
%% \[
%% W_i(S) = \predt\Biggl(F^i(S)\cup \bigcup_{\substack{S\xrightarrow{c}
%%     V\\ V\in\Passed^i}} \predc(F^i(V)),\ \bigcup_{\substack{S\xrightarrow{u}
%%     V\\ V\in\Passed^i}} \predu(V\setminus F^i(V))\Biggr) \cap S.
%% \]
%% Then during the $i$-th run through the \textbf{while} loop, if
%% $T^i\in\Passed^{i-1}$, we~then have $W^i=W_{i-1}(S^i)$.
%% Notice also that $W_i(S) \subseteq W_{i+1}(S)$.

%% We~then prove the third property inductively, showing that for
%% any~$i$, any~$S\in\Passed^i$ and any~$q\in S\setminus F^i(S)$, then
%% either $\Wait^i$ contains a transition $(S,\alpha,S')$ with
%% $S'\in\Passed^i$, or $q\notin W_{i}(S)$.

%% When~$i=0$, $\Passed^0$ only contains~$S_0$; then either $F^0(S_0)=S_0$,
%% for which the property holds vacuously, or $F^0(S_0)=\emptyset$,
%% and then $W_0(S_0)=\emptyset$, which entails the property.

%% Assume that the property holds at some step~$i$, and
%% that $T^{i+1}\notin\Passed^i$, so that case~$B$ applies.
%% Take $S\in\Passed^{i+1}$ and $q\in S\setminus F^{i+1}(S)$.
%% We~again consider two cases:
%% \begin{itemize}
%% \item If $S\in\Passed^i$, then we~have $q\in S\setminus F^i(S)$
%%   (because $F^i(S)\subseteq F^{i+1}(S)$). The~induction hypothesis
%%   applies, and two cases may occur:
%%   \begin{itemize}
%%   \item if $\Wait^i$ contains a transition~$(S,\alpha,S')$ with
%%     $S'\in\Passed^i$: then
%%     $(S^{i+1},\alpha^{i+1},T^{i+1})\not=(S,\alpha,S')$, since we
%%     assumed $T^{i+1}\notin\Passed^i$; hence $\Wait^{i+1}$ still contains
%%     the transition~$(S,\alpha,S')$, with $S'\in\Passed^{i+1}$;
    
%%     %% then
%%     %% $\Wait^{i+1}$ still contains a transition from~$S$, and the result
%%     %% is preserved at step~$i+1$. On~the other hand, if
%%     %% $(S^{i+1},\alpha^{i+1},T^{i+1})=(S,\alpha,S')$,
%%   \item or $q\notin W_i(S)$: then $S\not=T^{i+1}$, since
%%     $S\in\Passed^i$ and $T^{i+1}\notin\Passed^{i}$. Hence $F^{i+1}(S)=F^i(S)$.
%% %
%%     If also $T^{n+1}\not=\posttime(\postta_{\beta}(S))$ for any
%%     transition~$\beta$, then $W_{i+1}(S)=W_i(S)$.
%% %
%%     Otherwise, if $T^{n+1}=\posttime(\postta_{\beta}(S))$, then by the
%%     first property applied at the $i$th iteration,
%%     $(S,\beta,T^{i+1})\in\Wait^i$ (since
%%     $T^{i+1}\notin\Passed^i$).
%% %
%%       If~$S\not=S^{i+1}$, then also $(S,\beta,T^{i+1})\in\Wait^{i+1}$,
%%       and the result follows; if $S=S^{i+1}$, then we~again have two
%%       cases: either $c(T^{i+1})\not=\Win$, so that
%%       $F^{i+1}(T^{i+1})=\emptyset$, and $W_{i+1}(S)=W_i(S)$; or
%%       $c(T^{i+1})=\Win$, but then
%%       $(S^{i+1},\alpha^{i+1},T^{i+1})\in\Wait^{i+1}$, with $S=S^{i+1}$
%%       and $T^{i+1}\in\Passed^{i+1}$, which concludes this case.
%%   \end{itemize}
%% \item If $S\notin\Passed^i$, then $S=T^{i+1}$ (because
%%   $S\in\Passed^{i+1}$). Moreover, $c(T^{i+1})\not=\Win$, as otherwise
%%   $S\setminus F^{i+1}(S)=\emptyset$. Hence
%%   $F^{i+1}(S)=\emptyset$. Then two cases may arise: either $S$~has
%%   non-empty symbolic successors in~$\Passed^{i+1}$, in which case
%%   those successors belong to~$\Wait^{i+1}$, or it has no such
%%   successors, and $W_{i+1}(S)=\emptyset$, so that $q\notin W_{i+1}(S)$.
%% \end{itemize}


%% We now assume that case~$A$ applies, i.e., that
%% $T^{i+1}\in\Passed^i$. We~again take $S\in\Passed^{i+1}$ and $q\in
%% S\setminus F^{i+1}(S)$. Since case~$A$ does not modify~$\Passed$,
%% we~must have $S\in\Passed^i$, and
%% $q\in S\setminus F^i(S)$
%%   (because $F^i(S)\subseteq F^{i+1}(S)$).
%%   %The~induction hypothesis
%%   %applies, and two cases may occur:
%%   We~again consider several cases:
%%   \begin{itemize}
%%   \item if $S=S^{i+1}$, then after running case~$A$, we~have
%%     $F^{i+1}(S)=W_i(S)$. If~$S\xrightarrow{u} S$ for some
%%     uncontrollable transition, then $q\in S\setminus F^{i+1}(S)$
%%     entails $q\notin W_{i+1}(S)$. Otherwise, $W_{i+1}(S)=W_i(S)$, and
%%     again $q\notin W_{i+1}(S)$.
%%   \item otherwise, $S\not=S^{i+1}$: from the induction hypothesis,
%%     we~have either $(S,\alpha,S')\in \Wait^i$, or $q\notin W_i(S)$.
%%     In~the former case, we~still have $(S,\alpha,S')\in \Wait^{i+1}$.
%%     In the latter case, we~again split our analysis into two cases: if
%%     $S^{i+1}$ is not a symbolic successor of~$S$, then
%%     $W_{i+1}(S)=W_i(S)$ and our the property follows. Finally, if
%%     $S\xrightarrow{\beta} S^{i+1}$, then the first property applied at
%%     step~$i$ indicates that either $(S,\beta,S^{i+1})\in\Wait^{i}$, or
%%     $(S,\beta,S^{i+1})\in\Dep^{i}(S^{i+1})$ and $S^{i+1}\in\Passed^i$.
%%     \begin{itemize}
%%     \item We~begin with the former case: it~implies
%%       $(S,\beta,S^{i+1})\in\Wait^{i+1}$ (because $S\not=S^{i+1}$), and
%%       moreover $(S^{i+1},\alpha^{i+1},T^{i+1})\in\Wait^i$
%%       implies\NM{Lemme ?}  $S^{i+1}\in\Passed^i$, thus also
%%       $S^{i+1}\in\Passed^{i+1}$.
%%     \item otherwise, $(S,\beta,S^{i+1})\in\Dep^{i}(S^{i+1})$ and
%%       $S^{i+1}\in\Passed^i$. In~case $F^i(S^i)\subsetneq
%%       F^{i+1}(S^{i+1})$, we~get that $\Wait^{i+1}$ contains
%%       $(S,\beta,S^{i+1})$; otherwise, from the induction hypothesis,
%%       we~have either $(S,\alpha,S')\in \Wait^i$, or $q\notin W_i(S)$;
%%       in~both cases, the property is preserved by this iteration of
%%       the loop, which concludes our proof.
%%     \end{itemize}
%%   \end{itemize}
%% %% We finally prove the third invariant. We~assume that the property
%% %% holds before running the loop, and that $(S,\alpha,S')$ is popped
%% %% from~$\Wait$.  First assume case~$A$ applies, i.e., $S'\in\Passed$,
%% %% and pick~$q\in T\setminus F(T)$ for some~$T\in\Passed$ after running
%% %% through the loop. 
%% \end{proof}



Using these invariants, we get the following result:
\begin{lemma}
  If \cref{9-algo:sotftr} terminates, all configurations in~$F(S)$ for
  any~$S\in\Passed$ are winning for~Eve;  if~additionally
  $\Wait=\emptyset$, then all configurations in~$S\setminus F(S)$, for
  any~$S\in\Passed$, are losing for~Eve.
\end{lemma}

\begin{proof}
The~first statement corresponds to the second statement of
\cref{9-lem:sotftr}. Now, assume $\Wait^n=\emptyset$ after
termination at the $n$-th step,
and
%pick $S\in\Passed^n$ and $q\in S\setminus F^{n}(S)$ (if~any). Let
let $L=\{q\in \Locs\times\mathbb{R}_{\geq 0}^{\Clocks} \mid \exists
S\in\Passed^n.\ q\in S\setminus F^n(S)\}$, and $M$ be the complement
of~$L$. For~any~$S\in\Passed^n$, we~then have $M\cap S\subseteq
F^n(S)$.
%\todo{Why? Is not possible to have $S_1\neq S_2$ with~$q \in F(S_1)$
%  and~$q \not \in F(S_2)$?}
%\NM{Since $q\in M$, then $q\notin S'\setminus F^n(S')$ for all $S'$.
%  Since $q\in S$, then $q \in F^n(S)$.}

Pick $q\in \pi(M)\cup \Win$, where we abusively write $\Win$ to denote
all configurations with location colored~$\Win$. We~prove that $q\in M$:
\begin{itemize}
\item if $q\in\Win$: assume $q\in L$, and pick $S\in\Passed^n$ such
  that $q\in S\setminus F^n(S)$. Since $q\in S$, it~holds $c(S)=\Win$;
  but then $F^n(S)$ is defined (since $S\in\Passed^n$), and it
  equals~$S$ by initialization of~$F$. This contradicts the fact that
  $q\in S\setminus F^n(S)$, hence $q\in M$.
\item if $q\in\pi(M)$: we again assume $q\in L$. Then for some
  $S\in\Passed^n$, $q\in S\setminus F^n(S)$. By~the third property of
  \cref{9-lem:sotftr}, $q\notin W^n$
%  \todo{Benjamin: preuve du lemme 9.4 : $W^n(S)$ devrait être $W^n$ j’imagine, et ce serait bien de préciser ici que c’est la nième valeur de la valeur W. Par contre, je n’ai pas compris comment vous concluez que $q\notin pi(M)$ et en quoi ça nous permet de conclure finalement. Désolé, j’étais peut-être pas très réveillé.}
  %\todo{What is $W_n$?} % it is $W^n$, value of W at n-th round
  (because $\Wait^n$ is
  empty).  Since $M\cap U \subseteq F^n(U)$ for all~$U\in\Passed^n$
  and $F^n(U)=\emptyset$ for all~$U\notin\Passed^n$,
  and by monotonicity, we~get
  \[
  q\notin\predt\Biggl((M\cap S)\cup \bigcup_{\substack{S\xrightarrow{c}
    V}} \predc(M\cap V),\ \bigcup_{\substack{S\xrightarrow{u}
    V}} \predu(V\setminus (M\cap V))\Biggr) \cap S.
  \]
  %hence $q\notin \pi(M)$.
  %\NM{ajouter un qed (a la main ?). Bug de svmono.cls...}
%
  Now, notice that any $S\in\Passed^n$ is closed under~$\posttime$. Then
  \begin{xalignat*}1
    \pi(M)\cap S &= \predt(M\cup\predc(M), \predu(\overline M)) \\
     &= \predt((M\cap S)\cup (\predc(M)\cap S), \predu(\overline M)).
  \end{xalignat*}
  Now, it~can be checked that $\predc(M)\cap S \subseteq
  \bigcup_{\substack{S\xrightarrow{c} V}} \predc(M\cap V)$ and
  $\predu(\overline M) \supseteq \bigcup_{\substack{S\xrightarrow{u}
      V}} \predu(V\setminus (M\cap V))$.  So the fact that $q\in
  \pi(M)$ and $q\in S$ leads to a contradiction. This entails that
  $q\notin L$, hence $q\in M$.
\end{itemize}
In the end, we~have proven that $\pi(M)\cup\Win \subseteq M$, so that
$M$ is a pre-fixpoint of $X\mapsto \pi(X)\cup\Win$, hence it~contains
all winning configurations of~\Eve and $L$ only contains losing
configurations.
\end{proof}

%It~remains to prove termination of~\Cref{9-algo:sotftr}. [...]
%\NM{termination relies on extrapolation... Explain?}

The procedure given in \Cref{9-algo:sotftr} will in general not
terminate: as in the case of timed automata, the number of zones
generated by the algorithm may be infinite. This is classically
avoided using \emph{extrapolation}: this consists in abstracting the
zones being considered by larger zones, defined by only using integer
constants less than the maximal constant appearing in the timed arena
(as we did for regions in the proof
of~\cref{9-thm:timed-control}). This can be proven to preserve
correctness, and makes the~number of zones finite. Termination follows
by noticing that any triple~$(S,\alpha,T)$ may be added to~$\Wait$ only
a finite number of times (bounded by the number of regions
in~$T$). 
%
We~can then conclude:
\begin{theorem}
\Cref{9-algo:sotftr} terminates when extrapolation is used, and returns~$1$ if, and only~if,
$(\ell_0,\mathbf{0})$~is winning for~\Eve.
\end{theorem}
