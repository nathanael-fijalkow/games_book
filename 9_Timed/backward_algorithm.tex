Given the DBM data structure we presented,
a backward fixpoint algorithm follows for computing the winning configurations in a timed game:

\begin{theorem}
   \label{9-thm:timed-control}
   For any timed game~$\TA$ and target set~$G \subseteq \VE$, 
   the~limit $\lim_{k \rightarrow \infty} \pi^k(G)$ exists, and is reached 
   in a finite number of steps. This limit is precisely the set of states from
   which the controller has a winning strategy for reaching~$G$.
\end{theorem}

\begin{proof}[Sketch]
%Not immediate: Why does the fixpoint terminate here?
  We~briefly explain why the fixpoint computation terminates and refer
  to~\cite{AMPS98,CDFLL05} for more details.  The~proof relies on the notion of
  \emph{regions}. Intuitively, regions are minimal zones, not
  containing any other zone. More precisely, writing~$K$ for the maximal
  (absolute) integer constant appearing in the timed game, the~set of regions is
  the set of zones associated with
  DBMs~$(\mathord\prec_{i,j},M_{i,j})$ such that
  %\NM{verifier que c'est    ok...}
  \begin{itemize}
  \item $M_{i,j}\in [-K; K]\cup\{-\infty;+\infty\}$ for all~$i$ and~$j$;
  \item for all~$i\not=j$,
     \begin{itemize}
    \item either $M_{i,j}=-M_{j,i}$ and
      $\mathord\prec_{i,j}=\mathord\prec_{j,i}=\mathord\leq$,
     \item or $|M_{i,j}+M_{j,i}|=1$ and 
      $\mathord\prec_{i,j}=\mathord\prec_{j,i}=\mathord<$,
    \item or $(\prec_{i,j},M_{i,j})=(<,+\infty)$ and $(\prec_{j,i},M_{j,i})=(<,-K)$.
     \end{itemize}
  %\todo{Il y a un QED en plein milieu de la preuve...}
  % probleme global avec les QED, on regardera plus tard
  \end{itemize}
  The set of regions form a finite partition of~$\Realnn^{\Clocks}$.
  The~main argument for the proof is that the
  image by~$\pi$ of any finite union of regions is a finite union of
  regions. Since~$\pi$ is non-decreasing, the~sequence~$\pi^k(G)$
  converges after finitely many steps.

  The~fact that $\pi^k(G)$ may
  only contain winning configurations follows from our construction,
  and can be proved easily by induction on~$k$.
  One can also show that for all configurations outside of~$\pi^k(G)$,
  there is no strategy that is winning within~$k$ steps. This follows since
  the definition of~$\pi(\cdot)$ gives the actions to be played by \Adam to
  prevent \Eve from winning.
  Note that this does not necessarily yield a winning strategy for \Adam,
  since the game is not determined.
\end{proof}

%Add more comments.
%Hence, this operator allows us to express the states that can be controlled
%within one step. The following theorem shows that the greatest fixpoint of this
%operator exists and can be computed. 
%This should follow from~\cite{AMPS98,AMP-concur98}.
%See also Tripakis et al. before 2005.
