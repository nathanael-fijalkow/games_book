

%*** General probabilistic notation ***

\newcommand{\expv}{\mathbf{E}} % EXP. VALUE
\newcommand{\discProbDist}{f} % Discrete prob distribution
\newcommand{\sampleSpace}{S} % Generic sample space
\newcommand{\sigmaAlg}{\mathcal{F}} % Generic sigma-algebra
\newcommand{\probm}{\mathbb{P}} % Generic probability measure, also prob. measure operator
\newcommand{\rvar}{X} % Generic random variable
%\newcommand{\dist}{\mathit{Dist}}

%*** MDP notation ***

\newcommand{\actions}{A} % The set of actions.
\newcommand{\colouring}{c} % the colouring function
\newcommand{\probTranFunc}{\Delta} % Transition function of an MDP
\newcommand{\edges}{E} % Set of edges in an MDP.
\newcommand{\colours}{C} % The set of colours in an MDP.
\newcommand{\mdp}{\mathcal{M}} % A generic MDP. 
\newcommand{\vinit}{v_0} % An initial vertex in an MDP.
\newcommand{\cylProb}{p} % Function assigning probabilities to cylinder sets in 
%the measure construction.
\newcommand{\emptyPlay}{\epsilon} %empty play
\newcommand{\objective}{\Omega} % Qualitative objective
\newcommand{\genColour}{\textsc{c}} % Generic colour
\newcommand{\quantObj}{f} % Generic quantitative objective
\newcommand{\indicator}[1]{\mathbf{1}_{#1}} % In.d RV
\newcommand{\eps}{\varepsilon} % Numerical epsilon
\newcommand{\maxc}{W} % Maximal abs. value of a colour

\newcommand{\winPos}{W_{>0}}
\newcommand{\winAS}{W_{=1}}
\newcommand{\cylinder}{\mathit{Cyl}}

\newcommand{\PrePos}{\text{Pre}_{>0}}
\newcommand{\PreAS}{\text{Pre}_{=1}}

\newcommand{\PreOPPos}{\mathcal{P}_{>0}}
\newcommand{\OPAS}{\mathcal{P}_{=1}}

\newcommand{\safeOP}{\mathit{Safe_{=1}}}
\newcommand{\closed}{\mathit{Cl}}

\newcommand{\reachOP}{\mathcal{V}}
\newcommand{\discOP}{\mathcal{D}}
\newcommand{\valsigma}{\vec{x}^{\sigma}}

\newcommand{\lp}{\mathcal{L}}
\newcommand{\lpdisc}{\lp_{\mathit{disc}}}
\newcommand{\lpmp}{\lp_{\mathit{mp}}}
\newcommand{\lpsol}[1]{\bar{#1}}
\newcommand{\lpmpdual}{\lpmp^{\mathit{dual}}}

\newcommand{\actevent}[3]{\actions^{#1}_{#2,#3}} % Returns #1-th action on the run 

\newcommand{\MeanPayoffSup}{\MeanPayoff^{+}}
\newcommand{\MeanPayoffInf}{\MeanPayoff^{-}}

\newcommand{\mcprob}{M}
\newcommand{\invdist}{\vec{z}}

\newcommand{\hittime}{T}



This chapter considers concurrent games. The concurrent games we consider are extensions of the games considered in \cref{2-chap:regular} and \cref{4-chap:payoffs}, but where the choice of which edge to choose in a round is determined not by the choice of the owner of the vertex (indeed the vertices in concurrent games have no owners), but by the outcome of a matrix game corresponding to the vertex and played in that round. 
A matrix game is in turn a generalization of rock-paper-scissors, where each player picks an action simultaneously and then their pair of actions determines the outcome.

We will consider concurrent discounted, reachability and mean-payoff games and the definitions of the different objectives is as in the introduction. 
The chapter is divided into four sections:
\begin{enumerate}
\item The first section considers matrix games
\item The second section focuses on concurrent discounted games
\item The third section considers concurrent reachability games
\item The fourth section is about concurrent mean-payoff games
\end{enumerate}
As we go through the sections in this chapter, the complexity of the strategies and the computational complexity of solving the games rises: Indeed, since the games are generalizations of rock-paper-scissors, the strategies used requires randomness, but towards the end, no optimal or finite-memory $\epsilon$-optimal strategies exists in general and even the principle of sunken cost does not apply! 
Even with all this, the related questions about values are solvable in polynomial space and thus also in exponential time even in the last section.
The results we will focus on characterizes the complexity of the both the strategies as well as the computational complexity.
In each section we first give some positive result and some number of negative results. Each negative result also applies to the classes of games considered in the latter sections and each positive result applies to the classes considered in earlier sections (however, the positive results of latter sections will have worse complexity than the positive results from earlier sections).
As mentioned, the strategies for this chapter requires randomness and not too surprisingly, this implies that there is little difference between having stochastic or deterministic transition functions.

\section{Notations}
\label{7-sec:notations}
We write vectors in boldface: $ \vec{x}, \vec{y}, $ etc. For a vector $ \vec{x} $ indexed by a set $ I $ (i.e. $ \vec{x}\in \mathbb{R}^I $) we denote by $ \vec{x}_i $ the value of the component whose index is  $i\in I  $. 

A (discrete) ""probability distribution"" over a finite or countably infinite set $A$ is a function $\discProbDist A \colon \rightarrow [0,1]$ such that $\sum_{a\in A}\discProbDist(a)=1$. The ""support"" of such a distribution $\discProbDist$ is the set of all $a\in A$ with $\discProbDist(a)>0$. A distribution $f$ is called ""Dirac"" if its support has size 1.
%, which means that $f$ assigns probability 1 to a single element of $A$ and $0$ to all other elements. 
We denote by $\dist(A)$ the set of all probability distributions over $A$.

\knowledge{probability distribution}{notion,index={probability distribution}}
\knowledge{support}{notion,index={support}}
\knowledge{Dirac}{notion,index={probability distribution!Dirac}}

We also deal with probabilities over uncountable sets of events. This is accomplished via the standard notion of a \emph{probability space.}

\begin{definition}[""Probability space""]
\label{5-def:probspace}
A probability space is a triple
$(\sampleSpace,\sigmaAlg,\probm)$ where
\begin{itemize}
\item $\sampleSpace$ is a non-empty set of \emph{events} (so called
\emph{sample space}). 

\item $\sigmaAlg$ is a ""sigma-algebra"" over $\sampleSpace$,
i.e. a collection of subsets of $\sampleSpace$ that contains the empty set
$\emptyset$ and that is closed under complementation and countable unions. The members of $\sigmaAlg$ are called ""$\sigmaAlg$-measurable 
sets"".

\item $\probm$ is a ""probability measure"" on $\sigmaAlg$, i.e. a function
$\probm\colon \sigmaAlg\rightarrow[0,1]$ such that:
\begin{enumerate}
\item $\probm(\emptyset)=0$;

\item for all $A\in \sigmaAlg$ it holds $\probm(\sampleSpace \setminus
A)=1-\probm(A)$; and

\item for all countable sequences of pairwise disjoint sets $A_1,A_2,\dots \in \sigmaAlg$ (i.e., $A_i \cap A_j = \emptyset$ for all $i\neq j$)
we have $\sum_{i=1}^{\infty}\probm(A_i)=\probm(\bigcup_{i=1}^{\infty} A_i)$.
\end{enumerate}
\end{itemize}
\end{definition}

\knowledge{probability space}[Probability space]{notion,index={probability space}}
\knowledge{sigma-algebra}{notion,index={sigma-algebra}}
\knowledge{measurable set}[$\sigmaAlg$-measurable sets]{notion,index={measurable set}}
\knowledge{probability measure}{notion,index={probability measure}}

A ""random variable"" in the probability space $(\sampleSpace,\sigmaAlg,\probm)$ is an $\sigmaAlg$-measurable function $\rvar\colon \Omega \rightarrow \R \cup
\{-\infty,\infty\}$, i.e.,
a function such that for every $a\in \R \cup \{ -\infty,\infty\}$ the set
$\{\omega\in \Omega\mid \rvar(\omega)\leq a\}$ belongs to $\mathcal{F}$. We denote by $\expv[\rvar]$ the ""expected value"" of a random variable $\rvar$~(see Chapter 5 in \cite{Bil:1995} for a formal definition).

\knowledge{random variable}{notion,index={random variable}}
\knowledge{expected value}{notion,index={expected value}}

We first give a syntactic notion of an MDP which is an analogue of the notion of an "arena" for games.

\begin{definition}[""MDP""]
\label{5-def:MDP}
A ""Markov decision process"" is a tuple $(\vertices,\edges,\probTranFunc,\colouring)$. The meaning of $\vertices$, $\edges$, and $\colouring$ is the same as for games, i.e. $\vertices$ is a finite set of vertices, $\edges\subseteq \vertices\times\vertices$ is a set of edges and $\colouring\colon \edges \rightarrow \colours$ a mapping of edges to a set of colours. However, the meaning of $\probTranFunc$ is now different: $\probTranFunc$ is a partial ""probabilistic transition function"" of type $\probTranFunc\colon \vertices \times \actions \rightarrow \dist(\edges)$, such that the support of $\probTranFunc(v,a)$ only contains edges outgoing from $v$.
% We say that action $a$ is \emph{enabled} in $v$ if $\probTranFunc(v,a)$ is defined, and we denote by $\actions(v)$ the set of actions enabled in $v$. We stipulate that $\actions(v)\neq \emptyset$ for each $v$. 
 We usually write $\probTranFunc(v'\mid v,a)$ as a shorthand for $\probTranFunc(v,a)((v,v'))$, i.e. the probability of transitioning from $v$ to $v'$ under action $a$.
\end{definition}

\knowledge{MDP}[Markov decision process]{notion,index={Markov decision process}}
\knowledge{probabilistic transition function}{notion,index={transition function!probabilistic}}

We also stipulate that for each edge $(v_1,v_2)$ there exists an action $a\in \actions$ such that $\probTranFunc(v_2\mid v_1,a)>0$. Edges not satisfying this can be always removed without changing the semantics of the MDP, which is defined below. We denote by $ p_{\min} $ the smallest non-zero edge probability in a given MDP, i.e. $ p_{\min} = \min\{x>0 \mid \exists u,v \in \vertices, a \in \actions \text{ s.t. } x = \probTranFunc(v\mid u,a)\}. $

We denote by $\edges_\genColour$ the set of edges coloured by $\genColour$. Also, for MDPs where $\colours$ is some set of numbers, we use $\maxc$ to denote the number $\max_{e\in 
	\edges}|\colouring(e)|$.
In the setting of MDPs it is technically convenient to encode regular objectives (Reachability, B{\"u}chi,\dots) by colours on \emph{vertices} as opposed to edges. Hence, when discussing these objectives, we assume that the colouring function $\colouring$ has the type $\vertices \rightarrow \colours$.
%The probabilistic transition function is typically a partial function. An action $a$ is said to be \emph{enabled} in a vertex $v$ if $\probTranFunc(v,a)$ is defined. 

% ALLOWED ACTIONS

%\begin{remark}
%	TODO: REMARK ABOUT POSSIBLE REPRESENTATION WITH STOCHASTIC VERTICES
%\end{remark}

\paragraph{Plays and strategies in MDPs}

%The notion of plays in MDPs is the same as in classical games. There is, 
%however, one new important notion, the one of cylinder sets. A ""basic 
%cylinder"" determined by a finite play $\play$ is a set of all infinite plays 
%having $\play$ as a prefix.

 The way in which a play is generated in an MDP is similar to games, but now encompasses a certain degree of randomness. There is a single player, say Eve, who controls all the vertices. Eve's interaction with the ``world'' described by an MDP is probabilistic. One reason is the stochasticity of the transition function, the other is the fact that in MDP settings, it is usually permitted for Eve to use ""randomised strategies"". Formally, a randomised strategy is a function $\sigma : E^* \to \dist(A)$, which to each finite play assigns a probability distribution over actions. 
% We stipulate that the support of $\sigma(\play)$ consists of actions that are enabled in $\last(\play)$. 
 We typically shorten $\sigma(\play)(a)$ to $\sigma(a\mid \play)$.
 
 \knowledge{randomised strategy}[randomised strategies]{notion,index={strategy!randomised}}
%
In this section, we will refer to randomised strategies simply as ``strategies.'' The strategies known from the game setting will be called  ""deterministic strategies"". Formally, a deterministic strategy can be viewed as a special type of a randomised strategy which always selects a Dirac distribution over the edges. We shorten ``memoryless randomised/deterministic'' to ""MR"" and ""MD"", respectively.
\knowledge{deterministic strategy}[deterministic strategies]{notion,index={strategy!deterministic}}
\knowledge{MR}{notion,index={strategy!MR}}
\knowledge{MD}{notion,index={strategy!MD}}

Now a play in an MDP is produced as follows: in each step, when the finite play produced so far (i.e. the history of the game token's movement) is $\play$, Eve chooses an action $a$ randomly according to the distribution $\sigma(\play)$. Then, an edge outgoing from $\last(\play)$ is chosen randomly according to $\probTranFunc(\last(\play),a)$ and the token is pushed along the selected edge. As shown below, this intuitive process can be formalized by constructing a special probability space whose sample space consists of infinite plays in the MDP. 



%For each strategy $\sigma$ and each $i\geq 0$ we introduce an event $\actevent$

\paragraph{Formal semantics of MDPs}

Formally, to each MDP $\mdp$, each (Eve's) strategy $\sigma$ in $\mdp$, and 
each initial vertex $\vinit$ we assign a probability space 
$(\sampleSpace_{\mdp},\sigmaAlg_{\mdp},\probm^{\sigma}_{\mdp,\vinit})$. To 
explain the individual components, we need the notion of a cylinder set. A 
""basic cylinder"" determined by a finite play $\play$ is the set of 
all infinite plays in $\mdp$ having $\play$ as a prefix. Now the above 
probability space consists of the following components:
\begin{itemize}
	\item $\sampleSpace_{\mdp}$ is the set of all infinite plays in $\mdp$;
	\item $\sigmaAlg_{\mdp}$ is the \emph{Borel} sigma-algebra over 
	$\Omega_{\mdp}$; this is the smallest sigma-algebra containing all the 
	basic cylinder sets determined by finite plays in $\mdp$. The sets in 
	$\sigmaAlg_{\mdp}$ are called ""events"". Note that the smallest sigma-algebra of the desired property is guaranteed to exist, since an intersection of an arbitrary number of sigma-algebras is again a sigma algebra.
	\item $\probm^{\sigma}_{\mdp,\vinit}$ is the unique probability measure 
	arising from the \emph{cylinder construction} detailed below. We use 
	$\expv^{\sigma}_{\mdp,\vinit}$ to denote the ""expectation"" operator 
	associated to the measure $\probm^{\sigma}_{\mdp,\vinit}$.
\end{itemize}
\knowledge{basic cylinder}{notion,index={basic cylinder}}
\knowledge{event}[events]{notion,index={event}}
\knowledge{expectation}[expected value]{notion,index={expected value}}





Since the sample space $\sampleSpace_{\mdp}$ is uncountable, we construct the 
probability measure by first specifying a probability of certain simple sets of 
runs and then using an appropriate \emph{measure-extension} theorem to extend 
the probability measure, in a unique way, to all sets in $\sigmaAlg_{\mdp}$.
The standard cylinder construction  
proceeds as follows: for each finite play $\play$ we define the probability 
$\cylProb(\play)$ such that

\begin{itemize}
\item for an empty play $\emptyPlay$ we put $\cylProb(\emptyPlay)=1$;
\item for a non-empty play $\play=\play_0\cdots \play_{k}$ initiated in 
$\vinit$ we put 
\[\cylProb(\play) = \cylProb(\play_{< k})\cdot \Big(\sum_{a \in \actions} 
\sigma(a\mid \play_{< k})\cdot \probTranFunc(\last(\play)\mid 
\last(\play_{< k}),a) 
\Big), \]
where we use the convention that $\last(\play_{< 0})=\vinit$;
\item for all other $\play$ we have $\cylProb(\play)=0$.
\end{itemize}

Now using an appropriate measure-extension theorem
(such as Hahn-Kol\-mo\-go\-rov theorem~\cite[Corollary 2.5.4 and Proposition  2.5.7]{Rosenthal:2006}, or Carath\'eodory theorem~\cite[Theorem 1.3.10]{Ash&Doleans-Dade:2000}) one can show that there is a 
unique probability 
measure $\probm^{\sigma}_{\mdp,\vinit} $ on $\sigmaAlg_{\mdp}$ such that for 
every cylinder set $\cylinder(\play)$ determined by some finite play $\play$ we have $\probm_{\vinit}^\sigma(\cylinder(\play))=\cylProb(\play)$. (Abusing the notation, we write $\probm^{\sigma}_{\mdp,\vinit}(\play)$ for the probability of this cylinder set). There 
are some intermediate steps to be performed before an extension theorem 
can be applied, and we omit these due to space constraints. Full details on the 
cylinder construction can be found, e.g. in~\cite{Ash&Doleans-Dade:2000,Novotny:2015}.

While the construction of the probability measure 
$\probm^{\sigma}_{\mdp,\vinit}$ might seem a bit esoteric, in the context of 
MDP verification we do not usually need to be concerned with all the delicacies 
behind the associated probability space. The sets of plays that we work with 
typically arise from the basic cylinder sets by means of countable unions, 
intersections, and simple combinations thereof; such sets by definition belong 
to the 
sigma-algebra $\sigmaAlg_{\mdp}$, and their probabilities can be inferred using 
basic probabilistic reasoning. Nevertheless, one should keep in mind that all the 
probabilistic argumentation rests on solid formal grounds. 

%A reader wishing to 
%practice the formal understanding of probability theory might find the 
%following exercise useful: whenever this chapter (or some of the following 
%chapters with stochastic models) introduces some set of plays or a random 
%variable, try to understand why the set belongs to the sigma-algebra 
%$\sigmaAlg_{\mdp}$, why is the random variable $\sigmaAlg_{\mdp}$-measurable, 
%etc. This will help to develop an intuition for manipulating probabilistic 
%objects.

In the standard MDP literature~\cite{Puterman:2005}, the plays are often defined as alternating sequence of vertices and actions. Here we stick to the edge-based definition inherited from deterministic games. Still, we would sometimes like to speak about quantities such as ``probability that action $a$ is taken in step $i$.'' To this end, we introduce, for each strategy $\sigma$, each action $a$,  and each $i\geq 0$,  a random variable $\actevent{\sigma}{a}{i}$ such that $\actevent{\sigma}{a}{i}(\play)=\sigma(\play_{< i})(a)$. It is easy to check that  $\expv^\sigma_v[\actevent{\sigma}{a}{i}]$ is the probability that action $a$ is played in step $i$ when using strategy $\sigma$.

\paragraph{Objectives in MDPs}

Similarly to plays, the notions of both qualitative and quantitative objectives 
are inherited from the non-stochastic world of games. However, since plays in 
MDPs are generated stochastically, even for a fixed strategy $\sigma$ there is 
typically no single infinite play that would constitute the outcome of 
$\sigma$. A concrete $\sigma$ might yield different outcomes, depending on the 
results of random events during the interaction with the MDP. Hence, we need a 
more general way of evaluating strategies in MDPs. 

In the game setting, a qualitative objective was given as a set $\objective
\subseteq \colours^{\omega}$. In the MDP setting, we require that such 
$\objective$ is measurable in the sense that the set $\colouring^{-1}(\objective) = \{\play \in \sampleSpace_{\mdp} \mdp \colouring(\play) \in \objective \}$ belongs to $\sigmaAlg_{\mdp}$. We can then talk about a 
probability that the produced play falls satisfies $\objective$. For instance, for a 
colour $\genColour$ the objective $ \Reach(\genColour) $ is indeed measurable, since $ \colouring^{-1}(\objective) $ can be written as a countable union of all basic cylinders that are determined by finite plays ending in a vertex coloured by $ \genColour $. Indeed, all the qualitative objectives studied in previous chapters can be shown measurable in a similar way, and we encourage the reader to prove this as an exercise.
Hence, the expression 
$\probm^{\sigma}_{\mdp,\vinit}(\Reach(\genColour))$ 
denotes the probability that a vertex of colour $\genColour$ is reached when 
using 
strategy $\sigma$ from vertex $\vinit$. 
In line with previous conventions, we 
stipulate that Eve aims to maximize this probability. 

The situation is more complex for quantitative objectives. As shown in the previous chapter, 
when working with quantitative objectives, the set of colours $\colours$ is typically the set of real numbers (or a subset thereof), and the quantitative objective is given by an ``aggregating function'' $\quantObj\colon \colours^\omega \rightarrow \R$, which can be extended into a function $\quantObjExt\colon \edges^\omega \rightarrow \R $ by putting $ \quantObjExt(\play) = \quantObj( \colouring(\play_0)\colouring(\play_1)\cdots) $.
%a quantitative 
%objective in the game setting was given by a function 
%$\quantObj\colon \colours^\omega \rightarrow \R$. 
In the MDP setting, we 
require that $\quantObjExt$ is $\sigmaAlg_{\mdp}$-measurable, which 
means that for each $x\in \R$ the set $\{\pi\in \edges^\omega\mid 
\quantObj(\colouring(\play_0)\colouring(\play_1)\cdots) \leq x\}$ belongs to 
$\sigmaAlg_{\mdp}$ (again this holds for all the objectives studied 
in the previous chapters). Then there are two ways in which we can define the expected payoff achieved by strategy $ \sigma $ from a vertex $ v $.
First, we can treat 
$\quantObjExt$ as a random variable 
in the probability space 
$(\sampleSpace_{\mdp},\sigmaAlg_{\mdp},\probm^{\sigma}_{\mdp,v})$. Then the ""play-based payoff"" of $\sigma$ from $ v $, which we denote by $ \playPay_\quantObj(v,\sigma) $, is the expected value of this random variable, i.e. $ \playPay_\quantObj(v,\sigma) = \expv_{v}^\sigma [\quantObjExt] $. That is, we compute the expected payoff over all plays. This approach subsumes also qualitative objectives: For such an objective $\objective$ we can consider an ""indicator"" random 
variable $\indicator{\objective}$, such that $\indicator{\objective}(\play)=1$ 
of 
$\play\in\Omega$ and $\indicator{\objective}(\play)=0$ otherwise. Then 
$\probm^{\sigma}_{\mdp,v}(\objective) = 
\expv^{\sigma}_{\mdp,v}[\indicator{\objective}] = \playPay_{\indicator{\objective}}(v,\sigma)$.

\knowledge{indicator}{notion,index={random variable!indicator}}

\knowledge{play-based payoff}{notion,index={payoff!play-based}}


The second approach to quantitative objectives in MDPs, common e.g. in the operations research literature, is step-based: for each time step $i$ we compute the expected one-step reward (i.e. colour) encountered in that step and then  aggregate  these one-step expectations. Formally, the ""step-based payoff"" of $ \sigma $ from $ v $ is $ \stepPay_f(v,\sigma) = \quantObj(\expv_{v}^\sigma[\colouring(\play_0)] \expv_{v}^\sigma[\colouring(\play_1)]\cdots] ) $, where for each $ i $ we treat the expression $ \colouring(\play_i) $ as a random variable returning the colour (i.e. a number) which labels the $ i $-th edge of the randomly produced play (recall here that we index edges from~$ 0 $). 

\knowledge{step-based payoff}{notion,index={payoff!step-based}}

Depending on the concrete quantitative objective and on the shape of $\sigma$, the path- and step-based payoffs from a given vertex might or might not be equal. Nevertheless, in this chapter we study only objectives for which these two semantics yield the \emph{same optimization criteria:} no matter which of the two semantics we use, the optimal values will be the same and strategy that is optimal w.r.t. one of the semantics is also optimal for the other one. Hence, we will fix the play-based approach as the default one, writing just $ \Pay_f(v,\sigma)$ instead of $ \playPay_f(v,\sigma) $. We will prove the equivalence with step-based payoff where necessary. Also, we will drop the subscript $ f $ when the payoff function is known from the context.

%As in the game setting, we can also consider 
%qualitative objectives arising from quantitative ones. In such a case we could 
%be interested, e.g. in maximizing the probability that the value of a random 
%variable $\quantObj$ surpasses a given threshold $t\in \R$, i.e. maximizing the 
%probability of an event $\{\quantObj\geq t\}$.

%As a matter of fact, both quantitative and qualitative objectives can be 
%conveniently described in terms of optimizing the expected value. For a 
%qualitative objective $\objective$ we can consider an ""indicator"" random 
%variable $\indicator{\objective}$, such that $\indicator{\objective}(\play)=1$ 
%of 
%$\play\in\Omega$ and $\indicator{\objective}(\play)=0$ otherwise. Then 
%$\probm^{\sigma}_{\mdp,\vinit}(\objective) = 
%\expv^{\sigma}_{\mdp,\vinit}[\indicator{\objective}]$. Hence, we can think of 
%objectives as of random variables.

\paragraph{Optimal strategies and decision problems}

Let us fix an MDP $\mdp$ and an objective given by a random variable 
$\quantObj$. The value of a vertex $v\in\vertices$ is the number 
$\Value(v)=\sup_{\sigma} \Pay_f(v,\sigma)$. We let $\Value(\mdp)$ denote the $|\vertices|$-dimensional vector whose component 
indexed by $v$ equals $\Value(v)$.

We say that a strategy $\sigma$ is $\eps$-optimal in $v$, for some $\eps\geq 0$, if $\Pay_f(v,\sigma) \geq \Value(v) - \eps$. A $0$-optimal strategy is simply called optimal. 

For qualitative objectives, there are additional modes of objective satisfaction. Given such an objective $\objective$, we say that a strategy $\sigma$ is ""almost-surely winning"" from $v$ if $\expv^{\sigma}_{\mdp,v}[\indicator{\objective}]=1$, i.e. if the run produced by $\sigma$ falls into $\objective$ with probability $1$. We also say that $\sigma$ is ""positively winning"" from $ v $ if $\expv^{\sigma}_{\mdp,v}[\indicator{\objective}]>0$. For strategies that are winning in the non-stochastic game sense, i.e. that \emph{cannot} produce a run not belonging to $\objective$, are usually called ""surely winning"" to distinguish them from the above concepts. We denote by $\winPos(\mdp,\objective)$ and $\winAS(\mdp,\objective)$ the sets of all vertices of $\mdp$ from which there exists a positively or almost-surely winning strategy for the objective $\objective$, respectively.

\knowledge{almost-surely winning}[a.s. winning]{notion,index={strategy!almost-surely winning}}
\knowledge{positively winning}[pos. winning]{notion,index={strategy!positively winning}}
\knowledge{surely winning}[pos. winning]{notion,index={strategy!surely winning}}


The problems pertaining to the existence of almost-surely or positively winning strategy are often called \emph{qualitative problems} in the MDP literature, while the notion \emph{quantitative problems} covers the general notion of optimizing the expectation of some random variable. We do not use such a nomenclature here so as to avoid confusion with qualitative vs. quantitative objectives as defined in \Cref{1-chap:introduction}. Instead, we will refer directly to, e.g. ``almost-sure reachability'' while using the term ``optimal reachability'' to refer to the expectation-maximization problem.

%To reason about the complexity of decision problems pertaining to MDPs we assume, as usual, that all numbers appearing in the MDP (numerical colours, probabilities\ldots) are rational. We will be sometimes interested, how does the bit-size of these numbers influence the runtime of our algorithms; in particular, whether the algorithms run in \emph{strongly polynomial time.}
%
%\begin{definition}
%An algorithm which takes MDPs as inputs is said to run in \emph{strongly polynomial time} if the number of arithmetic 
%	operations performed by the algorithm is bounded by a polynomial in the number 
%	of vertices, edges, and actions (but independent of the bit size of the 
%	probabilities and possible numerical colours). 
%\end{definition}


%\subsection*{Expectation vs. Probabilistic Semantics of Objectives}
%
%The above view, where objectives are random variables whose expected value is to be optimized by choosing an appropriate strategy, is sometimes called an \emph{expectation semantics} (meaning semantics of the objective, not of the MDP dynamics per se). The expectation semantics is \emph{de facto} the standard  An alternative view is


\section{Matrix games}
\label{7-sec:matrix_games}
A matrix game is a game defined from a $(R\times C)$-matrix $M$  of numbers for some $R,C$.
The game is played as follows: Eve picks a row $r$ and Adam picks a column $c$ simulations like in rock-paper-scissors. Adam then pays Eve $M[r,c]$, i.e. the content of the entry defined by being in row $r$ and column $c$.
A strategy in such a game for Eve (resp. Adam) consists of a distribution over the rows (resp. columns). 
There is an illustration of rock-paper-scissors as a matrix game in Figure~\cref\{7-fig:rps}.


\begin{figure}

\center
\begin{tikzpicture}[node distance=3cm,-{stealth},shorten >=2pt]
\ma{main}[]{3}{3};

\node at (main-1-1.center) {0};
\node at (main-2-2.center) {0};
\node at (main-3-3.center) {0};
\node at (main-1-2.center) {-1};
\node at (main-2-3.center) {-1};
\node at (main-3-1.center) {-1};
\node at (main-1-3.center) {1};
\node at (main-2-1.center) {1};
\node at (main-3-2.center) {1};


\end{tikzpicture}
\caption{Rock-paper-scissors. The color is 1 if Eve wins, 0 if they draw and -1 if Adam wins. Also, the actions are ordered as in the name of the game}\label{7-fig:rps}
\end{figure}

The following theorem lists some known results for matrix games:
\begin{theorem}\label{lem:mat}
Each $(m\times n)$-matrix game $M$ is determined and there exists optimal strategies for each player. 
\begin{itemize}
\item The value and an optimal strategy for each player can be found in polynomial time and the problem is equivalent to linear programming.

\item Let $c>0$ be some constant. Consider the matrix $cM$ where each entry of $M$ has been multiplied by $c$. Then, the value of $cM$ is $cv$.
\item Let $c$ be some constant. Consider the matrix $M+c$ where each entry of $M$ is $c$ larger (additively). Then, the value of $M+c$ is $v+c$.
\item The value of matrix games are monotone in the entries.
\end{itemize}
\end{theorem}
We will omit the proof of the existence of values, optimal strategies and the first bullet.
The second bullet can be viewed as changing currency and clearly, this does not affect the optimal strategy.
The third bullet can be viewed as getting a reward before playing the game, and again, clearly this does not affect how to play it.
The last bullet can be seen from that each pair of strategies must give a higher reward if the entries of the matrix is higher.
This is especially true if you consider the optimal strategy for Eve in $M$ together with an arbitrary strategy for Adam, which then shows that the value is higher.

%if space: add proof

Given a matrix $M$, we will by $\Value[M]$ denote the value of the matrix game $M$. 

Perhaps interestingly, an illustration of a matrix $M$ can be viewed as a game arena $\arena$ (for concurrent games) with only one non-absorbing vertex. In each type of games considered in this section (apart from concurrent reachability games, where no game can be illustrated as a matrix with non-star entries different from 0), the value of the game with that arena matches $\Value[M]$ and the optimal strategies for each player is to play an optimal strategy from $M$ in each round. One can also consider a game arena $\arena^*$ with an illustration similar to $M$, but where there is a star in each entry (and $c(v,i,j)=0$ for the unique non-absorbing state $v$ and any pair of actions $i,j$).
Again, the value is $\Value[M]$ (except for the case of discounted objectives, where the value is $(1-\delta)\Value[M]$) and the optimal strategies for each player is again to play an optimal strategy from $M$. 

One could easily be lead to believe that in games (called repeated games with absorbing states) that can be illustrated as a single matrix $M$ with some entries stared and others not, the value would be similar to $\Value[M]$ and the optimal strategy would again be to play the optimal strategy from $M$. 
However, this is very much not true and indeed, many of the games in this chapter, illustrating how complex concurrent games can be, are repeated games with absorbing states! In particular repeated games with absorbing states may (1)~have irrational values and probabilities in optimal strategies (with any objective), (2)~have no optimal strategies (for reachability and mean-payoff objectives) and (3)~have no $\epsilon$-optimal finite-memory or $\epsilon$-optimal Markov strategies (for mean-payoff objectives)!


\section{Concurrent discounted games}
\label{7-sec:discounted}
In this section, we consider MDPs with edges coloured by rational numbers 
and with the objective $\DiscountedPayoff$, defined in the same way as in 
\Cref{chap:payoffs}. We start the chapter by proving that using the play-based semantics for the discounted-payoff objective yields no loss of generality. 

\begin{lemma}
	\label{5-lem:disc-step-one}
In a discounted-payoff MDP, for each strategy $ \sigma $ and each vertex $ v $ it holds $ \playPay(v, \sigma) = \stepPay(v, \sigma) $.
\end{lemma}
\begin{proof}
We have 
\begin{align*} \playPay(v,\sigma) &= \expv^\sigma_v[(1-\lambda)\lim_{k \rightarrow \infty} \sum_{i=0}^{k-1}\lambda^i \colouring(\play_i) ] = (1-\lambda) \lim_{k \rightarrow \infty} \expv^\sigma_v[\sum_{i=0}^{k-1}\lambda^i \colouring(\play_i) ] 
\\
&= (1-\lambda)\cdot\lim_{k \rightarrow \infty} \sum_{i=0}^{k-1}\lambda^i\expv^\sigma_v[ \colouring(\play_i) ] = \stepPay(v, \sigma).
\end{align*}
%\noindent
%
Here, the last equality on the first line follows from the dominated convergence theorem~\cite[Theorem 1.6.9]{Ash&Doleans-Dade:2000} and the following equality comes from the linearity of expectation. (To apply the dom. convergence 
theorem, we use the fact that for each 
$k$ we have $\DiscountedPayoff^{k}(\play)\leq \maxc.
$)
\end{proof}

\subsection*{Optimal values and memoryless optimality}

 In this sub-section we give a 
characterization of the value vector $\Value(\mdp)$ and prove that there always exists a 
memoryless deterministic strategy that is optimal in every vertex. Our 
exposition follows (in a condensed form) the one in~\cite{Puterman:2005}, the techniques 
being somewhat similar to the ones in the previous chapter.

%\subsection*{Optimal Reachability}
%
%We now switch our attention to the general case, where we aim to compute a 
%reachability value $\Value(v)$ of a given vertex as well as optimal 
%strategies. 
%We start with a characterization of $\Value(v)$ in terms of a fixed point of a 
%certain operator, which also leads to a simple approximation algorithm. To 
%this 
%end, fix an arbitrary MDP $\mdp$ and a reachability objective 
%$\Reach(\genColour)$, 

We define an operator $\discOP\colon 
\R^{\vertices}\rightarrow \R^{\vertices}$ as follows: each vector 
$\vec{x}$%=(x_v)_{v\in \vertices}$ 
is mapped to a vector 
$\vec{y}
%=(y_v)_{v\in\vertices} 
= \discOP(\vec{x})$ such that:
$$
\vec{y}_v = \max_{a \in \actions} \sum_{u\in \vertices} \probTranFunc(u\mid 
v,a) 
\cdot\left((1-\lambda)\cdot\colouring(v,u) + \lambda\cdot \vec{x}_u \right).
%1 & \text{if $\colouring(v)=\genColour$} \\
%\max_{a \in \actions} \sum_{u\in\in \vertices} \probTranFunc(u\mid v,a) \cdot 
%x_u & \text{otherwise.}
%\end{cases}
$$

\begin{lemma}
\label{5-lem:fixpoint}
The operator $\discOP$ is a contraction mapping. Hence, $\discOP$ has a unique 
fixed point $\vec{x}^*$ in $\R^{\vertices}$, and $\vec{x}^* = 
\lim_{k\rightarrow \infty} \discOP^k(\vec{x})$, for any 
$\vec{x}\in\R^{\vertices}$.
\end{lemma}
\begin{proof}
The proof proceeds by a computation analogous to the one in the first half of 
the proof of~\Cref{4-thm:discounted}; we just need to reason about actions 
rather than edges (and of course, use the formula defining $\discOP$ instead of 
the one for games). The second part follows from the Banach fixed-point theorem.
\end{proof}

We aim to prove that $\Value(\mdp)$ is the unique fixed point $\vec{x}^*$ of 
$\discOP$. We start with an auxiliary definition.

\begin{definition}
\label{5-def:disc-safe-act}
Let $\vec{x}\in \R^{\vertices}$. We say that an action $a$ is ""$\vec{x}$-safe"" in 
a vertex $v$ if
\begin{equation}
\label{5-eq:disc-safe-act}
a= \underset{a' \in \actions}{\arg\max} \sum_{u\in \vertices} 
\probTranFunc(u\mid 
v,a') 
\cdot\left((1-\lambda)\cdot\colouring(v,u) + \lambda\cdot \vec{x}_u \right).
\end{equation}
\noindent
A strategy $\sigma$ is $\vec{x}$-safe, if for 
each play $ \play $ ending in a vertex $v$, all actions that are selected with a positive 
probability by $\sigma$ for $\play$ are $\vec{x}$-safe in $v$.
\end{definition}

\knowledge{$\vec{x}$-safe}{notion,index={action!$\vec{x}$-safe}}

Given a strategy $\sigma$ we define $\valsigma$ to be the vector such that $\vec{x}_{v}^\sigma = 
\playPay(v,\sigma)$. For memoryless strategies, $\valsigma$ can be computed efficiently as follows:
Each memoryless strategy $\sigma$ defines a \emph{linear} operator $\discOP_{\sigma}$ which maps each vector 
$\vec{x}\in \R^{\vertices}$ to a vector $\vec{y}=\discOP_{\sigma}(\vec{x})$ 
such that $$\vec{y}_v = \sum_{a\in \actions} \sigma(a\mid v) \cdot 
\left(\sum_{u \in \vertices} 
\probTranFunc(u \mid v,a) \cdot( (1-\lambda)\cdot \colouring(u,v) + 
\lambda \cdot \vec{x}_u )\right).$$  

\begin{lemma}
\label{5-lem:disc-val-sigma}
Let $\sigma$ be a memoryless strategy using rational probabilities. Then the operator $\discOP_{\sigma}$ has a unique fixed point, which is equal to $\valsigma$ and which can be computed in polynomial time.
\end{lemma}
\begin{proof}
The operator $\discOP_{\sigma}$ can be seen as an instantiation of $\discOP$ in an MDP where there is no choice, since the action probabilities are chosen according to $\sigma$. \Cref{5-lem:fixpoint} shows that 
$\vec{x}^\sigma$ is a fixed-point of $\discOP^\sigma$. Since $\discOP_{\sigma}$ is again a contraction, it has a unique fixed point; and since it is linear, the fixed point can be computed in polynomial time, e.g. by Gaussian elimination (in its polynomial bit-complexity variant known as Bareiss algorithm~\cite{Bareiss:1968}).
\end{proof}

%\noindent
%Note that for any $\vec{x}\in\R^{\vertices}$ and any $v\in \vertices$ there is 
%at least one action that is $\vec{x}$-safe.
We now prove that there is a memoryless strategy ensuring outcome given by the fixed point of $\discOP$. Hence, the fixed point gives a lower bound on the values of vertices.

\begin{lemma}
\label{5-lem:disc-val-lower}
Let $\vec{x}^*$ be the unique fixed point of $\discOP$. 
Then there exists an MD strategy that is $\vec{x}^*$-safe. Moreover, for each $\vec{x}^*$-safe memoryless strategy it holds that  
$\playPay(v,\sigma) =\vec{x}_v^*$ for each $v\in \vertices$.
\end{lemma}
\begin{proof}
 Note that for each $\vec{x}\in \R^{\vertices}$ and each $v\in 
\vertices$ there always exists at least one action that is $\vec{x}$-safe in 
$v$. Hence, there is a memoryless deterministic strategy which 
in each $v$ chooses an arbitrary (but fixed) action that is $\vec{x}^*$-safe in 
$v$. 

Now let $ \sigma $ be an arbitrary $\vec{x}^*$-safe memoryless strategy.
By \Cref{5-lem:disc-val-sigma}, the vector $\valsigma$ is the unique fixed point of $\discOP^\sigma$.
 But since $\sigma$ 
is $\vec{x}^*$-safe, $\vec{x^*}$ is also a fixed point of $\discOP^\sigma$. 
Hence, $\vec{x}^* = \vec{x}^\sigma$.
\end{proof}

It remains to prove that $\vec{x}^*$ gives, for each vertex, an upper 
bound on the expected discounted payoff achievable by any strategy from that 
vertex. We introduce some additional notation to be used in the proof of the 
next lemma, as well as in some later results: namely, we denote by 
$\dPayoffStep{k}(\play)$ the 
discounted 
payoff accumulated during the first $k$ steps of $\play$, i.e. the number 
$(1-\lambda)\sum_{i=0}^{k-1} \lambda^i
\, \colouring(\play_i)$. The following lemma can be proved by an easy induction.

\begin{lemma}
\label{5-lem:disc-iterates}
For each $k\geq 0$ and each vertex $v$ we have 
$$\sup_{\sigma}\expv^{\sigma}_{v}[\dPayoffStep{k}] = 
(\discOP^k(\vec{0}))_v$$ 
(here $\vec{0}$ is a $|\vertices|$-dimensional vector of zeroes).
\end{lemma}

The previous lemma is used to prove the required upper bound on $\Value(v)$.

\begin{lemma}
\label{5-lem:disc-val-upper}
For each vertex $v$ it holds 
$\Value(v)\leq \vec{x}^*_v$, where $\vec{x}^*$ is the 
unique fixed point of $\discOP$.
\end{lemma}
\begin{proof}
%An easy induction shows that 
%for each $k\geq 0$ and each vertex $v$ we have 
%$$\sup_{\sigma}\expv^{\sigma}_{v}[\DiscountedPayoff^{k}] = 
%(\discOP^k(\vec{0}))_v$$ 
%(here $\vec{0}$ is a $|\vertices|$-dimensional vector of zeroes).
%
We have $\DiscountedPayoff(\play) = \lim_{k\rightarrow 
\infty}\dPayoffStep{k}(\play)$ (for each $\play$) and hence, by 
dominated 
convergence theorem we have $\expv^\sigma_v[\DiscountedPayoff] = 
\lim_{k\rightarrow 
\infty}\expv^\sigma_v[\dPayoffStep{k}]$. 
%Now $\Reach(\genColour) = \bigcup_{k=1}^{\infty}\Reach^k(\genColour)$ and 
Hence,
%
\begin{align}
\Value(v) &= \sup_{\sigma}\expv^\sigma_v[\DiscountedPayoff] \nonumber\\
%\sup_{\sigma}\expv^\sigma_v[\lim_{k\rightarrow 
%\infty}\DiscountedPayoff^k]\nonumber \\
&= \sup_{\sigma}\lim_{k\rightarrow \infty}\expv^\sigma_v[\dPayoffStep{k}] 
\label{5-eq:disc-limit-transition}.
%&=\sup_{k\geq 1} \sup_{\sigma}\probm^\sigma_v(\Reach^k(\genColour)) 
%=\big(\sup_{k\geq 1} \reachOP^k(\vec{0})\big)_v = \big(\lim_{k\rightarrow 
%\infty} \reachOP^k(\vec{0})\big)_v.
\end{align}

\noindent
It remains to prove that the RHS of~\eqref{5-eq:disc-limit-transition} is not 
greater than $\vec{x}^*= \lim_{k\rightarrow 
\infty}\discOP^k(\vec{0})=\lim_{k\rightarrow \infty} 
\sup_{\sigma}\expv^\sigma_v[\dPayoffStep{k}]$ (the last inequality follows 
by~\Cref{5-lem:disc-iterates}). It suffices to 
show that for each $\sigma'$ we have $\lim_{k\rightarrow 
\infty}\expv^{\sigma'}_v\dPayoffStep{k}] \leq \lim_{k\rightarrow 
\infty}\sup_{\sigma}\expv^\sigma_v[\dPayoffStep{k}]$. But this immediately 
follows from the fact that the inequality holds for each concrete $k$.
\end{proof}

The following theorem summarizes the results.

\begin{theorem}
\label{5-thm:disc-val-char-mem}
The vector of values $\Value(\mdp)$ in a discounted sum MDP $\mdp$ is the 
unique fixed point $\vec{x}^*$ of the operator $\discOP$. Moreover, there 
exists a 
memoryless deterministic strategy that is optimal in every vertex.
\end{theorem}
\begin{proof}
The characterization of $\Value(\mdp)$ follows directly from 
Lemmas~\ref{5-lem:disc-val-lower} and~\ref{5-lem:disc-val-upper}. The MD 
optimality follows from~\Cref{5-lem:disc-val-lower}.
\end{proof}

In the rest of this section we discuss several algorithms for computing the 
values and optimal strategies in discounted-payoff MDPs.

\subsection*{Value iteration}

The value iteration algorithm works in the same way as in the case of 
discounted-payoff games: we simply iterate the operator $\discOP$ on the 
initial argument $\vec{0}$. We know that $\Value(\mdp)=\lim_{k\rightarrow 
\infty}\discOP^k(\vec{0})$, and hence, iterating $\discOP$ yields an 
approximation of $\Value(\mdp)$. The iteration might not reached the fixed 
point (i.e. $\Value(\mdp)$) in a finite number of steps, but we can provide 
simple bounds on the precision of the approximation.

\begin{lemma}
\label{5-lem:disc-val-it-convergence}
For each $k\in \N$, $||\Value(\mdp)-\discOP^k(\vec{0}) ||_{\infty} \leq 
\lambda^k \cdot \maxc$. (Recall that $\maxc=\max_{e\in 
\edges}|\colouring(e)|$).	
\end{lemma}
\begin{proof}
This follows immediately from~\Cref{5-lem:disc-iterates} and from the fact that 
for each play $\play$, $\sum_{i=k}^{\infty}\lambda^i\cdot 
\colouring(\play_i)\leq \frac{1}{1-\lambda}\cdot\lambda^k \cdot \maxc$.
\end{proof}

Similar analysis can be applied to strategies induced by the approximate 
vectors.

\begin{lemma}
\label{5-lem:disc-val-it-eps-strategies}
Let $\eps>0$ be arbitrary and let 
$$k(\eps)=\left\lceil\frac{\log_2\left(\frac{\eps(1-\lambda)}{4\maxc}\right)}{\log_2(\lambda)}
 \right\rceil .$$ Then, every 
$\discOP^{k(\eps)}(\vec{0})$-safe memoryless strategy is $\eps$-optimal in 
every vertex.
\end{lemma}
\begin{proof}
Let $\sigma$ be any $\discOP^{k(\eps)}(\vec{0})$-safe memoryless strategy and 
let $\discOP_{\sigma}$ be the corresponding operator. We have that
\begin{align}
||\Value(\mdp) - \valsigma ||_{\infty} &= ||\Value(\mdp) 
-\discOP^{k(\eps)}(\vec{0}) +\discOP^{k(\eps)}(\vec{0}) - \valsigma 
||_{\infty} \nonumber
\\
&\leq ||\Value(\mdp) -\discOP^{k(\eps)}(\vec{0}) 
||_{\infty} + || \discOP^{k(\eps)}(\vec{0}) - \valsigma
||_{\infty}. \label{5-eq:disc-val-it-bound}
\end{align}
\noindent
The first term in~\eqref{5-eq:disc-val-it-bound} is $\leq \eps/2$ 
by the choice of $k(\eps)$ and~\Cref{5-lem:disc-val-it-convergence}. We prove 
that the second term 
in~\eqref{5-eq:disc-val-it-bound} is also bounded by $\eps/2$. Note that we 
have $\valsigma=\discOP_{\sigma}(\valsigma)$ (as was already proved 
in~~\Cref{5-lem:disc-val-lower}) and $\discOP(\discOP^{k(\eps)}(\vec{0})) = 
\discOP_\sigma(\discOP^{k(\eps)}(\vec{0}))$ (since $\sigma$ is 
$\discOP^{k(\eps)}(\vec{0})$-safe). Using this we get
\begin{align*}
|| \discOP^{k(\eps)}(\vec{0}) - \valsigma
||_{\infty} & = || \discOP^{k(\eps)}(\vec{0}) -\discOP^{k(\eps)+1}(\vec{0}) + 
\discOP^{k(\eps)+1}(\vec{0}) - \valsigma
||_{\infty}\\
&\leq || \discOP^{k(\eps)}(\vec{0}) -\discOP^{k(\eps)+1}(\vec{0}) ||_{\infty} + 
||\discOP^{k(\eps)+1}(\vec{0}) - \valsigma
||_{\infty}
\\
&=|| \discOP^{k(\eps)}(\vec{0}) -\discOP^{k(\eps)+1}(\vec{0}) ||_{\infty} + 
||\discOP_\sigma(\discOP^{k(\eps)}(\vec{0})) - 
\discOP_\sigma(\valsigma)||_{\infty}\\
%&\leq 
%|| \discOP^{k(\eps)}(\vec{0}) -\discOP^{k(\eps)+1}(\vec{0}) ||_{\infty} + 
%||\discOP_\sigma(\discOP^{k(\eps)}(\vec{0})) - 
%\discOP_\sigma(\valsigma)||_{\infty}\\
&\leq || \discOP^{k(\eps)}(\vec{0}) -\discOP^{k(\eps)+1}(\vec{0}) ||_{\infty} + 
\lambda\cdot||(\discOP^{k(\eps)}(\vec{0})) - 
\valsigma||_{\infty}
\end{align*}

\noindent
Re-arranging yields $|| \discOP^{k(\eps)}(\vec{0}) - \valsigma
||_{\infty} \leq \frac{1}{1-\lambda}\cdot|| 
\discOP^{k(\eps)}(\vec{0}) - 
\discOP^{k(\eps)+1}
||_{\infty} $.
It follows from~\Cref{5-lem:disc-val-it-convergence}  that 
$||\discOP^{k(\eps)}(\vec{0}) -\discOP^{k(\eps)+1}(\vec{0}) ||_{\infty} 
\leq 2\cdot\lambda^{k(\eps)}\cdot \max|\colouring|\leq 
\frac{(1-\lambda)\eps}{2}$, the last 
inequality holding by the choice of $k(\eps)$. Plugging this into the 
formula above yields $|| \discOP^{k(\eps)}(\vec{0}) - \valsigma
||_{\infty} \leq\frac{\eps}{2}$, as required. 
%
\end{proof}


%TODO: vector $\valsigma$

%The previous lemma shows that for a fixed discount factor $\lambda$, an 
%$\eps$-optimal strategy can be computed in time polynomial in the size of the 
%input MDP and in the bit-size of $\eps$. However, 
Using a value-gap result 
(similar to the game case, but proved using a different technique), one can 
show that sufficiently precise iterates can be used to computate an \emph{optimal} strategy. 
This is summarized in the following lemma due to~\cite{Tseng:1990}.

\begin{lemma}
\label{5-lem:disc-vi-optimal-strategy}
Let $d$ be the least common multiple of denominators of the following numbers: $\lambda$, all   
transition probabilities, and all edge colourings in $\mdp$. Next, let $\eps^* = 
\frac{1}{d^{(3|\vertices|+3)}\cdot |\vertices|^{\frac{\vertices}{2}}}$.
Then, any $\discOP^{k(\eps^*)}(\vec{0})$-safe memoryless deterministic strategy 
is 
optimal in every 
vertex.
\end{lemma}
\begin{proof}
%Let $\sigma$ be any strategy satisfying the assumptions. 
%By~\Cref{5-lem:disc-val-it-eps-strategies}, $\sigma$ is $\eps^*$-optimal. 
%Hence, it suffices to show that any $\eps^*$-optimal memoryless deterministic strategy is 
%optimal.
 Let $\sigma^*$ be any MD optimal strategy (it is guaranteed 
to exist by~\Cref{5-thm:disc-val-char-mem}). By the same theorem, we have that 
$\Value(\mdp)=\discOP^{\sigma}(\Value(\mdp))$. By the definition of 
$\discOP^{\sigma}$, we can write the above equation as $\Value(\mdp)= 
(1-\lambda)\cdot \vec{c}+\lambda \cdot P\cdot \Value(\mdp)$, where $\vec{c}$ is 
a vector whose each 
component 
is a 
sum of several terms, each term being a product of an edge colour and of a 
transition probability; and $P$ is a matrix containing 
transition 
probabilities. Multiplying the equation by $d^3$ yields $d^3\Value(\mdp)= 
d^3(1-\lambda)\cdot \vec{c}+d^3\lambda \cdot P\cdot \Value(\mdp)$. Since this equation has a unique fixed point (due to 
$\discOP^\sigma$ being a contraction), the matrix $A = d^3(I - \lambda P)$ (where $ I $ is the unit matrix) is 
regular, and moreover, composed of integers (ue to the choice of $ d $). By Cramer's rule, each entry of $\Value(\mdp)$ is equal to 
$\det(B)/\det(A)$, where $B$ is a matrix obtained by replacing some column of 
$A$ with $d^3(1-\lambda)\vec{c}$ (which is again an integer vector, due to the multiplication by $ d^3 $). Hence, each entry of $\Value(\mdp)$ is a rational number with denominator $\det(A)$. Hadamard's inequality~\cite{Garling:07} implies $|\det(A)|\leq d^{3|\vertices|}{|\vertices|}^{\frac{|\vertices|}{2}}$.

Now let $\sigma$ be any $\discOP^{k(\eps^*)}(\vec{0})$-safe MD strategy. By~\Cref{5-lem:disc-val-it-eps-strategies}, $\sigma$ is $\eps^*$-optimal. We prove that all actions used by $\sigma$ are $\vec{x}^*$-safe, which means that $\sigma$ is optimal by~\Cref{5-lem:disc-val-lower}. Assume that in some vertex $v$ the strategy $\sigma$ uses action $a$ that is not $\vec{x}^*$-safe. Denote $\vec{y}=\discOP_\sigma(\vec{x}^*)$. We have $ |\vec{y}_v - \vec{x}^*_v| > 0 $, since otherwise $a$ would be $\vec{x}^*$-safe. But then we can obtain a lower bound on the difference by investigating the bitsize of the numbers involved:
\begin{align*}
|\vec{y}_v - \vec{x}^*_v| &= \left|\frac{d^3}{d^3}\vec{y}_v - \frac{d^3}{d^3}\vec{x}^*_v\right|
\\
&=\frac{1}{d^3}\left|\sum_{u \in \vertices} 
\underbrace{d\cdot\probTranFunc(u \mid v,a)}_{\text{integer}} \cdot( \underbrace{d^2(1-\lambda)\cdot \colouring(u,v)}_{\text{integer}} + 
\underbrace{d^2\cdot\lambda \cdot \vec{x}^*_u ) - d^3\vec{x}^*}_{\text{int. multiples of $1/\det(A)$}}\right| \\
&\geq \frac{1}{d^3\cdot \det(A)}\geq \frac{1}{d^{(3|\vertices|+3)}\cdot{|\vertices|}^{\frac{|\vertices|}{2}}}.
\end{align*}

\noindent
%(The difference cannot be equal to $0$, otherwise $a$ would be $\vec{x}^*$ safe.) 
Now put $\vec{z}=\discOP_\sigma(\discOP^{k(\eps)}(\vec{0}))$. We have the following (using, on the first line, the fact that $|a+b| \geq |a|-|b|$):
%\begin{align*}
%	|\vec{z}_v - \vec{x}^*_v| &= |\sum_{u \in \vertices} 
%	\probTranFunc(u | v,a) \cdot( (1-\lambda)\cdot \colouring(u,v) + 
%	\lambda \cdot \discOP^{k(\eps)}(\vec{0})_u) - \vec{x}^*_v| && \\
%%	
%& = |\sum_{u \in \vertices} 
%\probTranFunc(u | v,a) \cdot( (1-\lambda)\cdot \colouring(u,v) + 
%\lambda \cdot (\discOP^{k(\eps)}(\vec{0})_u \underbrace{-\vec{x}^*_u)   +\discOP_\sigma(\vec{x}^*)}_{\parbox{2.5cm}{\scriptsize\text{introduce  $-\discOP_\sigma(\vec{x}^*)_v+\discOP_\sigma(\vec{x}^*)_v$}} }-\vec{x}^*_v | && \\
%&\geq |\discOP_\sigma(\vec{x}^*)_v-\vec{x}^*_v | - |\sum_{u \in \vertices} 
%\probTranFunc(u | v,a) \cdot( (1-\lambda)\cdot \colouring(u,v) + 
%\lambda \cdot (\discOP^{k(\eps)}(\vec{0})_u-\vec{x}^*_u) | &&\\
%&\geq \frac{1}{d^{3|\vertices|+3}\cdot |\vertices|^{|\vertices|}} - 
%%
%\end{align*}
\begin{align*}
	|\vec{z}_v - \vec{x}^*_v| &=
 |\vec{z}_v-\discOP_\sigma(\vec{x}^*)_v+\discOP_\sigma(\vec{x}^*)_v-\vec{x}^*_v | 
\geq |\discOP_\sigma(\vec{x}^*)_v-\vec{x}^*_v | - |\vec{z}_v-\discOP_\sigma(\vec{x}^*)_v |  \\
	&\geq \frac{1}{d^{(3|\vertices|+3)}\cdot |\vertices|^{\frac{|\vertices|}{2}}} - |\discOP_\sigma(\discOP^{k(\eps)}(\vec{0}))_v-\discOP_\sigma(\vec{x}^*)_v |\quad (\text{as shown above})\\
	&\geq \frac{1}{d^{(3|\vertices|+3)}\cdot |\vertices|^{\frac{|\vertices|}{2}}} - |\discOP_{\sigma}(\discOP^{k(\eps^*)}(\vec{0}) - \vec{x}^*)_v|  \quad (\text{since $\discOP_{\sigma}$ is linear})\\
	&\geq \eps^* - \lambda\cdot||\discOP^{k(\eps^*)}(\vec{0}) - \vec{x}^* ||_{\infty}
	> \eps^* - \frac{\eps^*}{2}  \quad (\text{\Cref{5-lem:disc-val-it-convergence}})\\&=\frac{\eps^*}{2}.
%	
%	
%
	%
\end{align*}

\noindent
In particular, it must hold that $\vec{z}_v< \vec{x}^*_v$. Otherwise we would have $\discOP^{k(\eps^*)+1}(\vec{0})_v \geq \discOP_{\sigma}(\discOP^{k(\eps^*)}(\vec{0}))_v \geq \vec{x^*}_v + \frac{\eps^*}{2} $, a contradiction with $\discOP^{k(\eps^*)+1}(\vec{0})$ being an $\frac{\eps^*}{4}$-ap\-prox\-imation of $\vec{x}^*$ (by \Cref{5-lem:disc-val-it-convergence} and the choice of $k(\eps^*)$).

At the same time, $|\discOP(\discOP^{k(\eps^*)}(\vec{0}))_v - \vec{x}^*|\leq \frac{\eps^*}{4}$, due to the choice of $k(\eps^*)$. Since $\vec{z}_v \leq \vec{x}_v^*$, we get $\vec{z}_v < \vec{x}_v^* - \frac{\eps}{2} \leq \discOP(\discOP^{k(\eps^*)}(\vec{0}))_v$, a contradiction with $\sigma$ being $\discOP^{k(\eps^*)}(\vec{0})$-safe. 
\end{proof}


\begin{corollary}
\label{5-cor:VI-optimal-strategy-comp}
An optimal MD strategy in discounted-payoff MDPs with a fixed discount factor can be computed in polynomial time. 
\end{corollary}
\begin{proof}
The number $1/\eps^*$, where $\eps^*$ is from \Cref{5-lem:disc-vi-optimal-strategy}, has bitsize polynomial in the size of the MDP when the discount factor is fixed. Hence, the number $k(\eps^*)$ defined as in \Cref{5-lem:disc-val-it-eps-strategies} has a polynomial magnitude, so it suffices to perform only polynomially many iterations. Since each iteration requires polynomially many arithmetic operations that involve only summation and multiplication by a constant, the result follows.
\end{proof}

\subsection*{Strategy improvement, linear programming, and (strongly) 
polynomial time}

The strategy (or policy) improvement (also called strategy/policy iteration in the literature) for MDPs works similarly as for games, see \Cref{5-algo:disc-strategy-improvement}. In the algorithm, we use $\discOP_{a,v}(\vec{x})$ as a shortcut for $ \sum_{u \in \vertices}\probTranFunc(u\mid v,a)\left((1-\lambda)\cdot\colouring(v,u) + \lambda\cdot \vec{x}_u \right)$

%\begin{enumerate}
%\item Let $\sigma_0$ be any MD strategy. Put $i=0$.
%\item Compute $\vec{x}^{\sigma_i} = \left(\expv^{\sigma_i}_v[\DiscountedPayoff]\right)_{v\in \vertices}$ (using \Cref{5-lem:disc-val-sigma}).
%\item For each vertex $v$, check if there is an \emph{improving} action $a$, i.e. an action  such that $\sum_{u \in \vertices}\probTranFunc(u\mid v,a)\left((1-\lambda)\cdot\colouring(v,u) + \lambda\cdot \vec{x}_u \right) > \vec{x}^{\sigma_i}_v. $ If yes, then let $a_v$ be any action such that $a_v = \underset{a' \in \actions}{\arg\max} \sum_{u\in \vertices} 
%\probTranFunc(u\mid 
%v,a') 
%\cdot\left((1-\lambda)\cdot\colouring(v,u) + \lambda\cdot \vec{x}_u \right)$. If there is no improving action, put $a_v = \sigma_i(v)$.
%\item If no state admitted an improving action, return $\sigma_{i}$ as the optimal strategy. Otherwise, let $\sigma_{i+1}$ be the strategy such that $\sigma_{i+1}(v)=a_v$ for each $v\in V$; then put $i:=i+1$ and go to (2.).
%\end{enumerate}

\begin{algorithm}
	\KwData{A discounted-payoff MDP $ \mdp $}
%	\SetKwFunction{KwOPAS}{$\OPAS$}
%	\SetKwProg{Fn}{Function}{:}{}

	$i \leftarrow 0$\;
	$ \sigma_i \leftarrow \text{arbitrary MD strategy} $\;
	\Repeat{$ \sigma_{i} = \sigma_{i-1} $}{
	compute $ \vec{x}^{\sigma_i} = \left(\expv^{\sigma_i}_v[\DiscountedPayoff]\right)_{v\in \vertices} $ \tcp*{Using \Cref{5-lem:disc-val-sigma}}
	\ForEach{$ v \in \vertices $}{
		$ \mathit{Improve}(v) \leftarrow \sigma_{i}(v) $\;
		\ForEach{$ a \in \actions $}{
			\lIf{$\discOP_{a,v}(\vec{x}^{\sigma_i}) >\discOP_{a,\mathit{Improve}(v)}(\vec{x}^{\sigma_i})$}{
				$\mathit{Improve}(v) \leftarrow a$
				}
			}
		$\sigma_{i+1}(v) \leftarrow \mathit{Improve}(v)$
		}
		$ i \leftarrow i+1 $
	}
\Return{$ \sigma_i $}
	
%	\Fn{\KwOPAS{\genColour,X}}{
%		$Y \leftarrow \winPos(\mdp_X,\Reach(\genColour))$ \tcp*{Computed by \Cref{5-algo:reach-pos}}
%		\Repeat{$Y' \neq Y$}{
%			$Y' \leftarrow Y$\;
%			$Y \leftarrow \safeOP(\genColour,)y$
%		}
%		\Return{$Y$} \tcp*{$Y$ is now a closed set}
%	}
	
	\caption{An algorithm computing an optimal MD strategy in a discounted MDP}
	\label{5-algo:disc-strategy-improvement}	
\end{algorithm}

\begin{theorem}
\label{5-thm:disc-strat-it}
The strategy improvement algorithm for discounted MDPs terminates in a finite (and at most exponential) number of steps and returns an optimal MD strategy.
\end{theorem}
\begin{proof}
First we need to show that whenever $\sigma_{i+1}\neq \sigma_i$, then  $\vec{x}^{\sigma_{i+1}} \geq \vec{x}^{\sigma_i}$ and $\vec{x}^{\sigma_{i+1}} \neq \vec{x}^{\sigma_i}$ (we write this by $\vec{x}^{\sigma_{i+1}} \succ\vec{x}^{\sigma_i}$). So fix some $ i $ s.t. an improvement is performed in the $ i $-th iteration of the repeat-loop. We have $\discOP_{\sigma_{i+1}}(\vec{x}^{\sigma_i})\succ\discOP_{\sigma_{i}}(\vec{x}^{\sigma_i})= \vec{x}^{\sigma_i}$, i.e. $\discOP_{\sigma_{i+1}}(\vec{x}^{\sigma_i})-\vec{x}^{\sigma_i} \succ 0$. Let $P$, $\vec{c}$ be the matrix and vector such that the equation $\vec{x}=\discOP_{\sigma_{i+1}}(\vec{x})$ can be written as $\vec{x}= (1-\lambda)\cdot \vec{c}+\lambda \cdot P\cdot\vec{x}$. Since the equation $\vec{x}=\discOP_{\sigma_{i+1}}(\vec{x})$ has a unique fixed point (as $ \discOP_{\sigma_{i+1}} $ is a contraction), the matrix $ I-\lambda P $ is invertible. Then $\discOP_{\sigma_{i+1}}(\vec{x}^{\sigma_i})-\vec{x}^{\sigma_i} \succ 0$ can be written as  $(1-\lambda)\vec{c} + (\lambda P - I)\vec{x}^{\sigma_i} \succ 0 $, or equivalently $\vec{x}^{\sigma_i}\prec (1-\lambda)\vec{c}\cdot(I-\lambda P)^{-1}.$ But the RHS of this inequality is equal to the fixed point of $\discOP_{\sigma_{i+1}}$, i.e. to $\vec{x}^{\sigma_{i+1}} .$

Now there are only finitely (exponentially) many MD strategies and since$\vec{x}^{\sigma_{i+1}} \succ\vec{x}^{\sigma_i}$, we have that no strategy is considered twice. Hence, the algorithm eventually terminates. Upon termination, there is no improving action, so the algorithm has found a strategy $\sigma$ whose value vector $\valsigma$ is the fixed point of $\discOP$. Such a strategy is optimal by \Cref{5-thm:disc-val-char-mem}. 
\end{proof}


While each of the steps (1.)--(4.) can be performed in polynomial time, the 
worst-case number of iterations is exponential~\cite{Hollanders&Delvenne&Jungers:2012}. However, the 
algorithm has nice properties in the case when the discount factor is fixed, as we'll see below. It is also intimately connected to the linear programming approach.

We can indeed aim to directly solve 
the equation $\vec{x} = \discOP(\vec{x})$, thus obtaining the fixed point of 
$\discOP$, by using a suitable LP. While the operator $\discOP$ is not 
in itself linear, solving the equation can be encoded into a linear  program. 
The main idea can be described as follows: given some numbers $y,z$, the 
solution $\bar{x}$ to the equation $x=\max\{y,z\}$ is exactly the smallest 
solution to the set of inequalities $x\geq y$, $x\geq z$. Hence, to solve the 
equation  $\vec{x} = \discOP(\vec{x})$, we set up the following linear program 
$\lpdisc$:
\vspace{-1em}

\begin{figure}[h]
\begin{align*}
&\text{{minimize} $\sum_{v\in \vertices} x_v$, \textrm{ 
subject to }}&\\
&x_v \geq \sum_{u\in \vertices} \probTranFunc(u\mid 
v,a)\cdot\left((1-\lambda)\cdot \colouring(v,u) + \lambda\cdot x_u \right)
&
\text{for all $v\in \vertices$ and $a\in \actions$.}%\\
% z_q & \geq 0 & \text{ for all } q\in Q
\end{align*}
\caption{The linear program $\lpdisc$ with variables $x_v$, $v\in \vertices$.}
\label{5-fig:disc-lp}
\end{figure}

\begin{lemma}
\label{5-lem:disc-lp}
The linear program $\lpdisc$ has a unique optimal solution 
$\bar{\vec{x}}$ that satisfies $\bar{\vec{x}} = \Value(\mdp)$.
\end{lemma}
\begin{proof}
Let $\vec{x}^* = \Value(\mdp)$ be the unique fixed point of $\discOP$. Clearly 
setting $x_v = \vec{x}^*_v$ yields a feasible solution of $\lpdisc$. 
We show 
that $\vec{x}^*$ is actually an optimal solution, by proving that for each 
feasible solution $\vec{x}$ it holds $\vec{x} \geq \vec{x}^*$. (This also 
shows 
the uniqueness, since it says that an optimal solution is the infimum of the 
set of 
all feasible solutions.) First, note that for any feasible solution $\vec{x}$ 
it holds $\discOP(\vec{x})\leq \vec{x}$, by the construction of $\lpdisc$. 
Next, if $\vec{x}$ is a feasible solution, then $\discOP(\vec{x})$ is also a 
feasible solution; otherwise, there would be some $v$ and $a\in \actions$ 
such that 
\begin{align*}\discOP(\vec{x})_v &< \sum_{u\in \vertices} \probTranFunc(u\mid 
v,a)\cdot\left((1-\lambda)\cdot \colouring(v,u) + \lambda\cdot 
\discOP(\vec{x})_u \right) \\ &\leq \sum_{u\in \vertices} \probTranFunc(u\mid 
v,a)\cdot\left((1-\lambda)\cdot \colouring(v,u) + \lambda\cdot \vec{x}_u 
\right) \leq \discOP(\vec{x})_v.
\end{align*}
Here, the first inequality on the second line follows from 
$\discOP(\vec{x})\leq \vec{x}$, while the second inequality follows from the 
definition of $\discOP$. But $\discOP(\vec{x})_v < \discOP(\vec{x})_v$ is an 
obvious contradiction. So $\discOP(\vec{x})$ is indeed a feasible solution and 
by applying the argument above, we get $\discOP^2(\vec{x}) \leq 
\discOP(\vec{x})$. By a simple induction, $\discOP^{i+1}(\vec{x})\leq 
\discOP^{i}(\vec{x})\leq \vec{x}$ for each $i\geq 0.$ Hence, also $\vec{x}^* = 
\lim_{i\rightarrow \infty} \discOP^i(\vec{x}) \leq \vec{x}$ (the first equality 
comes from \Cref{5-lem:fixpoint}).
\end{proof}

It is known that linear programming can be solved in polynomial time by 
interior-point techniques~\cite{Kha:1979,Karmarkar:1984}. Hence, we get the following.

\begin{theorem}
\label{5-thm:disc-polytime-lp}
The following holds for discounted-payoff MDPs:
\begin{enumerate}
\item
The value of each vertex as well as an MD optimal 
strategy can be computed in polynomial time. 
\item 
The problem whether the value of a given vertex $v$ is at least a given constant 
(say~1) is \P-complete (under logspace reductions). The hardness result holds 
even for a fixed discount factor.
\end{enumerate}
\end{theorem}
\begin{proof}(1.)
The first part comes directly from~\Cref{5-lem:disc-lp}. Once the optimal value 
vector $\Value(\mdp)$ is computed, we can choose any $\Value(\mdp)$-safe MD 
strategy as the optimal one 
(\Cref{5-lem:disc-val-lower}).

(2.) Let $\lambda$ be the fixed discount factor. We show the lower 
bound, by extending the reduction 
from the CVP problem used for almost-sure reachability. First, we 
transform the input circuit into an MDP in the same way as in the reachability 
case, and we let $v$ be the vertex corresponding to a gate we wish to evaluate. 
Assume, for a while, that each path from $v$ to a target state has the same 
length $\ell$. Then we simply assign reward 
$\frac{1}{(1-\lambda)\cdot\lambda^{\ell -1}}$ to each edge 
entering a target state, and $0$ to all other edges. It is easy to check that 
the value of $v$ in the resulting discounted MDP is equal to the value of $v$ 
in the reachability MDP. If the reachability MDP $\mdp$ does not have the 
``uniform 
path length'' property, we modify it by producing $|\vertices|$ copies of 
itself so that each new vertex carries, apart from the original name, an index 
from $\{1,\dots,n\}$. The transition function in the new MDP mimics the 
original one, but from vertices with index $j<n$ we transition to the 
appropriate vertices of index $j+1$. The target vertices in the new MDP are 
those vertices of index $n$ that correspond to a target vertex of the 
original MDP (this does not break down the reduction, as target vertices in the original vertices can be assumed to have no outgoing edges other than self loops). This new MDP has the desired property and can be produced in a 
logarithmic space.
\end{proof}

The previous theorem shows that discounted-payoff MDPs can be solved in 
polynomial time even if the discount factor is not fixed (i.e., it is taken as 
a part of the input). This is an important difference from the 2-player 
setting. However, the proof, resting on polynomial-time solvability of linear 
programming, leaves opened a question whether the discounted-payoff 
MDPs be solved in strongly polynomial time.  An answer was provided by Ye~\cite{Ye:2011}: already the classic simplex 
algorithm of Dantzig solves $\lpdisc$ in a strongly 
polynomial time in MDPs with a fixed discount factor. Formally, Ye proved that 
the number of iterations taken by the simplex method is bounded by 
$\frac{|\vertices|^2\cdot (|\actions|-1)}{1-\lambda}\cdot 
\log(\frac{|\vertices|^2}{1-\lambda})$, with each iteration requiring  
$\mathcal{O}(|\vertices|^2\cdot |\actions|)$ arithmetic operations. This has 
also an impact on the strategy improvement method: it can be shown that strategy 
improvement in discounted MDPs is really just a ``re-implementation'' of the 
simplex algorithm using a different syntax. Hence, the strongly polynomial 
complexity bound for a fixed discount factor holds there as well.

\begin{theorem}
For MDPs with a fixed discount factor, the value of each vertex as well as an 
optimal MD strategy can be computed in a strongly polynomial time.
\end{theorem}


\section{Concurrent reachability games}
\label{7-sec:reachability}
In this section we consider concurrent reachability games. 
Intuitively, unlike concurrent discounted games, these games cares only about the final part of the play.
This, while perhaps not clear directly from the definitions, makes the games somewhat harder. For instance, the value iteration algorithm requires double-exponential time.
Note that like for concurrent discounted games, if we force Eve to follow some strategy, the game reduces to a MDP and there exists optimal positional strategies.

We will not prove the following known lemma. We will however, show the weaker statement that the decision problem for the value can be done in PSPACE.

\begin{lemma}\label{lemm:reach_determined}
Concurrent reachability games are determined. Also, finding the value of a concurrent reachability game can be done in TFNP[NP]
\end{lemma}


We will argue that there might not be optimal strategies for Eve in concurrent reachability games. 
The game we will use will later be a member of a family of games that requires high patience to play well.

The snowball game (or purgatory 1) is defined as follows:
There are 3 vertices, the goal vertex $\Win$, $\bot$ (an absorbing vertex) and a vertex 1, which has a 2x2 matrix, such that $\dest(x,r,c)$ is a dirac distribution over (1)
$\Win$ for $r=c$, (2) $1$ for $r<c$ and (3) $\bot$ for $r>c$.
When we illustrate the game, we write view the goal vertex $\Win$ as being an absorbing vertex with color 1.
There is an illustration of the snowball game in \cref\{7-fig:snowball}. 

\begin{figure}

\center
\begin{tikzpicture}[node distance=3cm,-{stealth},shorten >=2pt]
\ma{s}[$1:$]{2}{2};

\node at (s-1-1.center) {$1^*$};
\node at (s-2-1.center) {$0^*$};
\node at (s-1-2.center) {$0$};
\node at (s-2-2.center) {$1^*$};


\end{tikzpicture}
\caption{The snowball game or purgatory 1, in which no optimal strategy exists for Eve}\label{7-fig:snowball}
\end{figure}


For any $\epsilon>0$, consider the stationary strategy for Eve that plays the first action with probability $1-\epsilon$. 
If Adam plays the left column always, play will reach $\Win$ with pr. $1-\epsilon$. If Adam plays the right column always, play reaches $\Win$ with pr. 1. Hence, the strategy for Eve is $\epsilon$-optimal and the value of the vertex is 1.

On the other hand, if Adam plays the right column whenever Eve plays the first action with pr. 1, and otherwise plays the first action, 
then in the last round in the vertex there must be a positive pr. that play goes to vertex 0. Hence, Eve has no optimal strategy.


\begin{lemma}\label{lemm:no_opt_reach}
Eve need not have an optimal strategy in a concurrent reachability game
\end{lemma}

The following lemma states some classical results for concurrent reachability games that we will not prove.
\begin{lemma}\label{lemm:reach_class}
\begin{itemize}
\item For any $\epsilon>0$, there are always $\epsilon$-optimal stationary strategies for Eve and optimal stationary strategies for Adam. 
\item The value iteration algorithm converges to the optimal value and is defined exactly like for concurrent discounted games, except that $\gamma=0$ and the target vertex has value 1 in the first iteration
\item The values $v$ are the least fix-point (i.e., every other fix-point $v'$ is such that for all $i$ $v_i\leq v'$) of the value iteration operator $\ValueOp$

\end{itemize}
\end{lemma}
(Note that the games are not symmetric, in that Eve tries to reach a node and Adam tries to stay away from it, and in particular, even though Eve need not have an optimal strategy in a concurrent reachability game, Adam always has one)






The decision problem for the existential first order theory over the reals is the following decision problem:
Given a function $F:\R^n\rightarrow \{\text{true},\text{false}\}$, is there an vector $v$ such that $F(v)$ is true?
The function $F$ must be an well-formed (i.e. connected with logical ``and'', ``or'' and ``not'') quantifier-free formula over polynomial inequalities.
E.g. $x^2y+z\geq 5\wedge \neg (xz\leq 3)$ would be such a function.

\begin{lemma}
The decision problem for the existential theory over the reals is in PSPACE
\end{lemma} 

We will now, given a number $c$, encode the problem whether the value in a concurrent reachability game is  $<c$, starting from some vertex $x$.
The idea is that we can describe a fix-point of the value iteration operator, i.e.
we can describe that $v_i=\Value[M^i(v)]_i$. Since we know that the values are the least fix-point, we can then just add the condition that $v_x<c$. 
We can describe the value of a matrix game by guessing a strategy for each player, $\sigma$ for Eve and $\tau$ for Adam, and then checking that these strategies are optimal by showing that the outcome obtained by following Eve's strategy is equal to what is obtained by following Adam's.
I.e. we check that for Eve's strategy, the least outcome obtained when Adam plays any given column is $v_i$ and similar for Adam.
We describe that as $v_i=\min_{a\in [m]} (\sigma M^i_{*,a}(v))$ and $v_i=\max_{a\in [m]} (\tau M^i_{a,*}(v))$, where $M^i_{*,j}(v)$ is the $j$-th column of $M^i(v)$ and $M^i_{j,*}(v)$ is the $j$-th row for all $j$. 
We can express that $x=\min(a_1,a_2,\dots,a_n)$ for any number $n$ and any polynomials $a_1,a_2,a_n$, in the first order theory of the reals by stating that \[
x\leq a_1\wedge x\leq a_2\wedge \dots \wedge x\leq a_n \wedge (x=a_1\vee x=a_2\vee \dots \vee x=a_n) \enspace .\]
Similarly, one can also describe that $x=\max(a_1,a_2,\dots,a_n)$ for any number $n$ and any polynomials $a_1,a_2,a_n$.

We can similarly make a statement that $v_x\leq c$. Using that PSPACE is equal to co-PSPACE, we also get that we can find if $v_x\geq c$ and $v_x>c$ and therefore also $v_x=c$ and $v_x\neq c$, all in PSPACE.

\begin{lemma}
Decision problems for the values in a concurrent reachability game is in PSPACE
\end{lemma}

The set of vertices that have value 0 can be found in polynomial time. This is because, the set of vertices that have value 0 in the time limited game of length $n$ has also value 0 in all other time limited games. That this is so is easy to see by considering that Eve plays an $\epsilon$-optimal strategy for $v>\epsilon$, where $v$ is the value of the vertex with the lowest value. The game then devolves to a MDP for Adam, and the statement is true for such.

\begin{lemma}\label{lemm:find_0_reach}
The set of vertices of value 0 in a concurrent reachability game can be found in polynomial time
\end{lemma}

Next, we will also argue that we can find the set of vertices $S_1$ that have value 1 in polynomial time as well.
For notational convenience, for stationary strategies $\sigma,\tau$ we will, for a set $x$ and a set of vertices $S$ use \[
F^{\sigma,\tau}(x,S):=\sum_{r\in A_1}\sum_{c\in A_2}\sum_{x'\in S}\sigma(x)(r)\tau(x)(c)\delta(x,r,c)(x')\enspace ,\]
 i.e. the probability when the players follows $\sigma,\tau$ to go from $x$ directly to a vertex in $S$.

For a set $S$, containing $\Win$ and a non-empty subset $S'\subseteq(S\setminus \{\Win\})$ and a vertex $x\in S'$, let the {\em subset property} be the following:
For each $\epsilon>0$, there is a strategy $\sigma$ for Eve, such that for any strategy $\tau$ for Adam, $F^{\sigma,\tau}(x,S\setminus S')\cdot \epsilon >F^{\sigma,\tau}(x,V\setminus S)$.

For a set $S$, containing $\Win$, let the {\em value-1-property} be that the subset property is satisfied for each $S'\subseteq(S\setminus \{\Win\})$ (with some $x\in S'$).
For a set $S$ satisfying the value-1-property, we will define some subsets.
Let $S^0$ be the set consisting of $\Win$.
Let $S^i$, for each $i\geq 1$ be the set of vertices such that for each $x\in S^i$,
the vertex $x$ satisfies the subset property for $S'$ and $S=\bigcup_{j=0}^{i-1}S^j$.
Let $\ell$ be the largest number such that $S^\ell$ is non-empty (note that $S^i$, for $i>\ell$ must then be empty).


We will be using the following lemma.



\begin{lemma}
The set of vertices $S_1$ of value 1 satisfies the value-1-property
\end{lemma}
\begin{proof}
The proof is by contradiction. Thus, there is an $S$ (not containing $\Win$) such that $S_1$ and $S$ does not satisfy the subset property for any $x\in S$. I.e. for some constant $\epsilon>0$,
$F^{\sigma,\tau}(x,S_1\setminus S)\epsilon \leq F^{\sigma,\tau}(x,V\setminus S_1)$ for any strategy $\sigma$ for Eve and some strategy $\tau_{\sigma}$ for Adam and for every $x\in S$.
But then, all vertices in $S$ have value $\leq (1-\epsilon)+\epsilon V_{\max}$ where $v_{\max}<1$ is the largest value of a vertex in $V\setminus S_1$.
This is because to get to $\Win$ from $S$, it must leave $S$ and in that step, the probability to go to a vertex in  $V\setminus S_1$ (from which one cannot obtain more than $v_{\max}$) is at least the constant $\epsilon$.  
\end{proof}

We will argue that this is a precise characterization of $S_1$ next.

\begin{lemma}
Consider a set $S'$ satisfying the value-1-property.
Then each vertex of $S'$ has value 1.\label{lem:sufficent_for_value1}
\end{lemma}
\begin{proof}
We will for all $i$ and any $\epsilon>0$ construct a strategy $\sigma_{i,\epsilon}$ for Eve that starting from a vertex in $\bigcup_{j=i}^n S^j$ will eventually get to $S^{i-1}$ with probability at least $1-\epsilon$ (especially, the strategy $\sigma_{1,\epsilon}$ is $\epsilon$-optimal). We will do it using backwards induction in $i$ and thus start from $i=\ell$.

Note that the base case for $i=\ell$ follows directly from the condition.
We now for any $\epsilon>0$ will find a strategy $\sigma_{i,\epsilon}$, given that we have a strategy $\sigma_{i+1,\epsilon'}$ for any $\epsilon'>0$.
The idea is that the subset property gives a strategy $\sigma$ such that for all $\tau$, $F^{\sigma,\tau}(x,S'\setminus S)\epsilon/2 >F^{\sigma,\tau}(x,V\setminus S')$, for $S=\bigcup_{j=i}^{n} S^j$.
Note that in expectation, following this strategy, we go to some other vertex in $S$ with at least some fixed probability $p$ (that could be quite close to 1).
Hence, in expectation, we need to be in such a vertex $1/(1-p)$ times before entering either $S'\setminus S$ or $V\setminus S'$.
We therefore follow the strategy $\sigma_{i+1,\epsilon'}$ in $\bigcup_{j=i+1}^n S^j$, where $\epsilon'=\frac{\epsilon}{2(1-p)}$.
The inductive construction then follows by applying union bound over the $1/(1-p)$ times we are in $S^i$.
\end{proof}
Note that the lemmas together shows that $S_1$ is the largest set satisfying value-1-property.


Our algorithm, \crgLim, for finding $S_1$ is then as follows:
Assign to each vertex $x$ a rank $\rk_x$, a value in $0,1,\dots,n,\bot$, starting with $\rk_x=0$.
Let $\overline{S}^i$ be the set of vertices of rank $i$.
Let $\overline{S}_1=V\setminus S^{\bot}$.
We increment (where a rank is less than another if it is a smaller number. Also, all other ranks are below $\bot$) the rank of a non-goal vertex $x\in \overline{S}^i$, whenever it does not satisfy the subset property for $S=\overline{S}_1$ and $S'=\bigcup_{j=i}^n \overline{S}^j$.
Note that no vertex can satisfy the subset property for rank 0, since $S'$ is all vertices.
Whenever a stable configuration is reached, output $\overline{S}_1$.


\begin{lemma}
The output of the \crgLim\ algorithm is correct
\end{lemma}
\begin{proof}
The idea is that we want $S^i=\overline{S}^i$ at termination.
Note that the subset property is harder to satisfy for a vertex $x$ if we remove vertices from $S$ or from $(S\setminus S')$.
Thus, if at some time we have that $x$ does not satisfy the subset property for $S=\overline{S}_1$ and $S'=\bigcup_{j=i}^n \overline{S}^j$, then it does not do so for $S$ being any subset of $\overline{S}_1$ or $(S\setminus S')$ being any subset of $\bigcup_{j=0}^{i-1} \overline{S}^j$.
However, initially $S_1\supseteq\overline{S}_1$ (being all vertices) and $\bigcup_{j=i}^n S^j\supseteq \bigcup_{j=0}^{i-1} \overline{S}^j$ for all $i$. But we must have that $S_1\supseteq\overline{S}_1$ and $\bigcup_{j=i}^n S^j\supseteq \bigcup_{j=0}^{i-1} \overline{S}^j$ for all $i$, at all latter points as well, since in the last iteration it was satisfied, for all $i$ and all $x\in S^i$, we have that the subset property is satisfied for $S=\overline{S}_1$ and $S'=\bigcup_{j=i}^n \overline{S}^j$, because we have that $S\supseteq S_1$ and $(S\setminus S')=\bigcup_{j=1}^{i-1} \overline{S}^j\supseteq \bigcup_{j=1}^{i-1} S^j$.
On the other hand, eventually no vertex gets it rank incremented (since there are a finite number of ranks and vertices) and the algorithm terminates with a set $\overline{S}_1\supseteq S_1$ satisfying the value-1-property. Since $S_1$ is the largest such set, we have that $\overline{S}_1=S_1$.
\end{proof}


We will now consider the running time of the algorithm.
We will consider that we can decide whether a pair of sets $S$ and $S'$ satisfies the subset property for a vertex $x$ can be solved in $O(k)$ time.
Let $S_x$ be the set of vertices that can be visited in a play immediately after $x$. Observe that $|S_x|$ is a lower bound on the number of arrows from matrix $x$ in our illustration of the game.
Consider a vertex $x$ and a rank $i$, and we want to find an upper bound on the computation we do on $x$ while it has rank $i$.
Clearly, we only need to consider incrementing the rank of $x$ whenever a vertex in $S_x$ has changed its value and only if it is changed to either $i$ (from $i-1$) or to $\bot$ (from $n$), because otherwise, the sets $S$ and $S'$ have not changed. 
Thus, we only do at most $2|S_x|+1$ checks whether $x$ satisfies the subset property.
There are $n+1$ ranks, so in total for $x$, we use at most $(n+1)(2|S_x|+1)$ checks.
Hence, in total over all $x$, we do $O(n\sum_{x} |S_x|)$ checks.

\begin{lemma}
The run time of the \crgLim\ algorithm is $O(nk\sum_{x} |S_x|)$ times 
\end{lemma}

We will then finally consider how to check whether $x$ satisfies the subset property for a pair of sets $S$ and $S'$.
We will do so by constructing a strategy $\sigma=\sigma(\epsilon)$ for Eve satisfying the property for any fixed $\epsilon>0$.
We will construct the strategy  $\sigma$ from some sequence of pairs of sets (of rows and columns) $(R_1,C_1),(R_2,C_2),\dots,(R_{\ell},C_{\ell})$. We will let $C_i^*=\bigcup_{j=1}^i C_j$ and similar for $R^*_i$.
For convenience, we also define $C_0^*$ as the empty set of columns.
We will define $R_i$ from $C_{i-1}^*$ as each row $r\not\in R_{i-1}^*$ such that, for all $c\not\in C^*_{i-1}$, we have that $F^{r,c}(x,V\setminus S)=0$.
We will define $C_i\not\in C_{i=1}^*$ from $R_i$ as each column $c$ such that there is a $r\in R_i$ such that 
$F^{r,c}(x,S\setminus S')>0$.
The set $R^{*}_{\ell}$ is the first set such that $R^*_{\ell+1}$ is empty (clearly, by construction all sets $R_i,C_i$ for $i>\ell$ would also be empty).

\begin{lemma}
There is a strategy $\sigma(\epsilon)$ for all $\epsilon>0$ iff $C_{\ell}^*$ is the set of all columns
\end{lemma}
\begin{proof}
We will first argue that if $C_{\ell}^{*}$ is not all columns $C$, then the strategy $\tau$ that plays uniformly over $C'=(C\setminus C_{\ell}^*)$ shows that no strategy $\sigma(\epsilon)$ exists for small enough $\epsilon>0$. This is because any row $r$ such that $F^{r,c}(x,S\setminus S')>0$ for some $c\in C'$ is also such that $F^{r,c'}(x,V\setminus S)>0$ for some column $c'\in C'$. This is because otherwise $r$ would be in $R^i$ for some $i$ and then $c$ would be in $C_{\ell}^*$. Hence, the probability $F^{r,c'}(x,V\setminus S)$ cannot be more than a constant factor smaller than $F^{r,c}(x,S\setminus S')$.


Otherwise, if $C_{\ell}^*=C$, then let $\sigma(\epsilon)$ be the strategy that picks an $i$ from the distribution $\dist$ and then plays an action in $R^i$ uniformly at random. The distribution $\dist$ is such that for all $j\in \{1,\dots,\ell-1\}$ we have $\Pr^{\dist}[i=j]\epsilon\delta_{\min}/m=\Pr^{\dist}[i>j]$. 

To argue that $\sigma=\sigma(\epsilon)$ satisfies the subset property for $x,S,S'$
consider each column $c$. We have that $c\in C^j$ for some $j$. 
Let $p$ be the pr. with which a row in $R^j$ is played by $\sigma$ and thus, $F^{\sigma,c}(x,S\setminus S')\geq p\delta_{\min}>0$.
Any row $r$ such that $F^{r,c}(x,V\setminus S)>0$ must be outside $R^*_j$ by construction (and if such a row exists $j<\ell$). 
We play such rows with pr. $\leq \Pr^{\dist}[i>j]$ and thus $\Pr^{\dist}[i>j]\geq F^{\sigma,c}(x,V\setminus S)$.
We have that $pm>\Pr^{\dist}[i=j]$ (strict because $j<\ell$) and thus, \[
F^{\sigma,c}(x,S\setminus S')\epsilon\geq p\delta_{\min}\epsilon>\delta_{\min}\epsilon/m\Pr^{\dist}\nolimits[i=j]\geq 
\Pr^{\dist}\nolimits[i>j]\geq F^{\sigma,c}(x,V\setminus S) \enspace .\]
This completes the proof of the lemma.\end{proof}


Our algorithm for checking if a vertex $x$ satisfies the subset property for sets $S,S'$ is as follows:
We will construct the sequence of sets $(R_1,C_1),(R_2,C_2),\dots,(R_{\ell},C_{\ell})$.
To do so we will use a datastructure.
The datastructure has the following properties:
Initially, for each column $c$, we will make a list $L_c$ of the rows $r$ such that $F^{r,c}(x,V\setminus S)>0$.
We will also have a counter for each row $r$ that initially contains how many such columns there are.
Finally for each row $r$, there is a list $L_r$ of columns such that $F^{r,c}(x,S\setminus S')>0$.

The algorithm then uses the datastructure as follows:
Let $i\leftarrow 1$.
Add all the rows with the counter at 0 to $R^i$. 
If $R^i$ is the empty set, return whether $C_{i-1}^*$ is all columns.
For each row $r$ in $R^i$ go through $c\in L_r$ and subtract 1 from the counter of each row in $L_c$. If a counter reach 0, add it to $R^{i+1}$.
Increment $i$.
Go to line 3.

The total time is $O(\sum_{r,c}|\supp(\dest(x,r,c))|)$ for the algorithm.

\begin{lemma}
We can check whether a vertex $x$ satisfies the subset property for sets $S$ and $S'$ in time \[O(\sum_{r,c}|\supp(\dest(x,r,c))|)\]
\end{lemma}

We therefore get that \begin{lemma}\label{lem:val1}\label{lemm:find_1_reach}
We can find the set of vertices of value 1 in time $O(n\sum_{x} |S_x|\sum_{r,c}|\supp(\dest(x,r,c))|)$
\end{lemma}

Next, we will give a lower bound for patience, i.e. that in some games, the patience for every $\epsilon$-optimal stationary strategy must be high. 
For a number $k$, let purgatory $k$ be the following game:
There are $2+k$ vertices, $\Win$ (which is vertex 0), one vertex $\bot$ which is absorbing and each other vertex $i\in \{1,\dots, k\}$ has a 2x2 matrix, such that $\dest(x,r,c)$ is a dirac distribution over (1)
$i-1$ for $r=c$, (2) $k$ for $r<c$ and (3) $\bot$ for $r>c$.
There is an illustration of Purgatory $4$ in Figure \cref\{7-fig:purgatory}. 

\begin{figure}

\center
\begin{tikzpicture}[node distance=3cm,-{stealth},shorten >=2pt]

\ma{s2}[$3:$]{2}{2};
\ma[shift={($(s2)+(0,3cm)$)}]{s1}[$4:$]{2}{2};
\ma[shift={($(s2)+(3cm,0)$)}]{s3}[$2:$]{2}{2};
\ma[shift={($(s3)+(3cm,0)$)}]{s4}[$1:$]{2}{2};

\draw (s2-1-2.center) to (s1);
\draw (s3-1-2.center) to[bend right] (s1);
\draw (s4-1-2.center) to[bend right] (s1);

\foreach \x/\y in {2/3,3/4} {
\draw (s\x-2-2.center) to (s\y);
\draw (s\x-1-1.center) to[out=70] (s\y);
}
\foreach \x in {1,2,3,4} \node at (s\x-2-1.center) {$0^*$};
\draw (s1-2-2.center) to (s2-1-2);
\draw (s1-1-1.center) to[out=200] (s2-1-1);
%\foreach \x in {1,2,3} \node at ($(s\x-2-2.center)-(0.2cm,0)$) {$0$};
%\foreach \x in {2,3,4} \node at ($(s\x-1-2.center)-(0.2cm,0)$) {$0$};
%\foreach \x in {2,3} \node at ($(s\x-1-1.center)-(0.2cm,0)$) {$0$};

%\node at ($(s1-1-1.center)+(0.2cm,0)$) {$0$};
\niceloop{s1-1-2};

\node at (s4-1-1.center) {$1^*$};
\node at (s4-2-2.center) {$1^*$};

\end{tikzpicture}
\caption{Purgatory $4$. For clarity, the colors are omitted, except that $0^*$ corresponds to an edge to an absorbing vertex different from $\Win$ and $1^*$ corresponds to an edge to $\Win$}\label{7-fig:purgatory}
\end{figure}


It is easy to see that all vertices but $\bot$ is in $S_1$, using the value 1 property.

\begin{lemma}\label{lem:purgatory}
For any $0<\epsilon<1/2$ and any $k\geq 1$, there is a unique strategy for Eve with least patience which is $\epsilon$-optimal in purgatory $k$. That strategy has patience $\epsilon^{-2^{k-1}}$
\end{lemma}
\begin{proof}
We will find the best strategy with patience $1/p$ for any number $0<p<1/2$.
It is clear that the best strategy with patience $1/p$ is to play the strategy that maximizes the pr. of eventually reaching $i$ from vertex $j$ for all $j>i$ while having patience $1/p$.
Let $x_k=1-p$ and let $x_i=1-\sqrt{1-x_{i+1}}$. We will argue that $x_i$ is the probability of eventually reaching vertex $i-1$ from vertex $k$ for all $i\in \{1,\dots,k\}$ and that there is a unique best strategy with patience $1/p$.

The unique best strategy in vertex $k$ is to play the top action with pr. $1-p$ and the bottom action with pr. $p$. This ensures that the pr. $x_k$ of reaching $k-1$ from $k$ is $1-p$ if Adam plays the left column.

Consider now vertex $i\in \{1,\dots,k-1\}$.
For the purpose of finding good strategies in $i$, we can view vertex $i$, when Eve plays her best strategy in $j>i$ and Adam plays a best response, as a smaller reachability game with 3 vertices, i.e. $i-1$ (as $\Win$), $\bot$ and $i$, where $\dest(i,r,c)$ is (1) a dirac distribution over $i-1$ for $r=c$, (2) a dirac distribution over $\bot$ for $r>c$ and (3) a distribution that goes to $\bot$ with pr. $1-x_{i-1}$ and to $i$ with the remaining pr. for $r<c$. See Figure \cref\{7-fig:1purgatory}


Let $p_i$ be the probability with which a strategy $\sigma$ plays the top row in vertex $i$.
We can then consider the game as a MDP, since we have fixed a stationary strategy for one of the players.
It is clear that the pr. to reach $i-1$ from $i$ if Adam plays the right column is strictly increasing in $p_i$ and the pr. to reach $i-1$ from $i$ if Adam plays the left column is strictly decreasing in $p_i$. We will consider the strategy such that the pr. of reaching $i-1$ is equal no matter which column Adam plays (by the previous statement, this strategy must then be optimal).
Observe that the pr. of reaching $i-1$ if Adam always plays the left column is $p_i$ (which is then also the pr. to reach $i-1$ from $i$) and if he always plays the right column it is  $p_i x_i p_i+(1-p_i)$ (using that the pr. of reaching $i-1$ from $i$ is $p_i$).
Thus, we have that 
\[
p_i=p_i x_i p_i+(1-p_i)\Rightarrow 0=p_i^2/2 x_i-p_i+1/2 \Rightarrow p_i=\frac{1\pm \sqrt{1-x_i}}{x_i}\enspace .\] We see that $\frac{1+ \sqrt{1-x_i}}{x_i}>1$ and thus the solution is $p_i=\frac{1- \sqrt{1-x_i}}{x_i}$ or that $(x_{i-1}=)x_ip_i=1- \sqrt{1-x_i}$.

Consider now that the strategy is exactly $\epsilon$-optimal, implying that $x_1=1-\epsilon$. 
We will argue that $p=\epsilon^{2^{k-1}}$.
We will do so by arguing  using induction in $i$ that $x_i=1-\epsilon^{2^{k-1}}$, since $x_k=1-p$ (this also shows that the strategy is indeed using patience $1/p$ since the probabilities in the other vertices, which are $p_i=\frac{x_{i-1}}{x_i}$, are strictly above $1/p$).
We have already noted that  $x_1=1-\epsilon=1-\epsilon^{2^0}$.
We will next argue that $x_i=1-\epsilon^{2^{k-1}}$, for $i\geq 2$ using that $x_{i-1}=1-\epsilon^{2^{k-2}}$.
We have that \[
1-\epsilon^{2^{k-2}}=x_{i-1}=1-\sqrt{1-x_{i}}\Rightarrow \sqrt{1-x_{i}}= \epsilon^{2^{k-2}}\Rightarrow
1-\epsilon^{2^{k-1}}=x_i\enspace .
\]
This completes the proof of the lemma.
\end{proof}

Concurrent reachability games are not symmetric in the players. E.g. Adam always have an optimal strategy but Eve might not. We will next argue that Adam still requires double exponential patience to play well.

Consider the following game called purgatory duel $k$ which can be viewed as a symmetric version of purgatory $k$.
There are $3+2k$ vertices, $\Win$ (which is vertex 0), one vertex $\bot$ which is absorbing (and is also vertex $0'$), and the start vertex $s$ and each other vertex $\{1,\dots, k,1',\dots,k'\}$ has a 2x2 matrix. Each vertex $x\in \{1,\dots, k\}$ is such that $\dest(x,r,c)$ is a dirac distribution over (1) $x-1$ for $r=c$, (2) $s$ for $r<c$ and (3) $\bot$ for $r>c$.
 Each vertex $x'\in \{1',\dots, k'\}$ is such that $\dest(x',r,c)$ is a dirac distribution over (1) $x-1'$ for $r=c$, (2) $s$ for $r<c$ and (3) $\Win$ for $r>c$. The start vertex is 1x1 matrix and is such that $\dest(s,r,c)$ is a uniform distribution over $k$ and $k'$.
There is a illustration of Purgatory Duel $2$ in Figure \cref\{7-fig:purgatoryduel}.


\begin{figure}

\center
\begin{tikzpicture}[node distance=3cm,-{stealth},shorten >=2pt]

\ma[shift={($(4.5cm,4.5cm)$)}]{s}[$s:$]{1}{1};
\ma[shift={($(3cm,0)$)}]{s2'}[$2':$]{2}{2};
\ma[shift={($(s2')+(3cm,0)$)}]{s1'}[$1':$]{2}{2};
\ma[shift={($(0,3cm)$)}]{s2}[$2:$]{2}{2};
\ma[shift={($(s2)+(0,3cm)$)}]{s1}[$1':$]{2}{2};


\foreach \x in {1,2,1',2'} \draw (s\x-1-2.center) to (s);
\foreach \x/\y in {1/0,2/0,1'/1,2'/1} \node at (s\x-2-1.center) {$\y ^*$};
\foreach \x/\y in {1/1,1'/0} \foreach \z in {1,2} \node at (s\x-\z-\z.center) {$\y ^*$};
\draw (s2-1-1.center) to (s1);
\draw (s2-2-2.center) to[out=30,in=-30] (s1);
\draw (s2'-2-2.center) to (s1');
\draw (s2'-1-1.center) to[out=60,in=120] (s1');
\draw (s.center) to[out=225,in=100] node[pos=0.6,right] {$1/2$} (s2');
\draw (s.center) to[out=225,in=-10] node[pos=0.6,above] {$1/2$} (s2) ;


\end{tikzpicture}
\caption{Purgatory duel $2$. For clarity, the colors are omitted, except that $0^*$ corresponds to an edge to an absorbing vertex different from $\Win$ and $1^*$ corresponds to an edge to $\Win$}\label{7-fig:purgatoryduel}
\end{figure}

We will say that a strategy $\sigma$ for Eve mirrors a strategy $\tau$ for Adam, if $\sigma(i)=\tau(i')$ and $\sigma(i')=\tau(i)$ for all $i$.
Similarly, $\tau$ mirrors $\sigma$.

\begin{lemma}
The value of vertex $s$ is $1/2$. Also, for any $\epsilon>0$, any $(1/2-\epsilon)$-optimal strategy $\tau$ for Adam does not follow a dirac distribution in $i'$ for any $i'\in\{1',\dots,k'\}$. Finally, every $\epsilon$-optimal strategy is a mirror of an $\epsilon$-optimal strategy
\end{lemma}
\begin{proof}
First, the value of vertex $s$ is at most $1/2$. This is because Adam can mirror any strategy $\sigma$ for Eve. This ensures that any play reaching $\Win$ is mirrored by an equally likely play reaching $\bot$. Thus, then the players follows these strategies, the pr. to reach $\Win$ is equal to the pr. to reach $\bot$ (there might also be some positive pr. to not reach neither, but Adam also wins those plays). 

Fix $\epsilon>0$ and consider a strategy $\tau$ for Eve that plays a dirac distribution in $i'\in\{1',\dots,k'\}$. Then $\tau$ is not $(1/2-\epsilon)$-optimal. We can see that as follows: Let $\sigma$ be the strategy for Eve that plays $r=1$ when in $j'\in \{(i+1)',\dots,k'\}$ and the action which is not equal to $\tau(i')$ when in $i'$. This ensures that the play will always reach either $\Win$ or $s$ from $k'$. But then  Eve can play an $(\epsilon/2)$-optimal strategy for purgatory $k$ in $\{1,\dots,k\}$, ensuring that $\Win$ is reached with pr. at least $1-\epsilon/2$. But then $\tau$ is not $(1/2-\epsilon)$-optimal.

Consider that Adam is following a strategy $\tau$ that is not playing a dirac distribution in $i'\in\{1',\dots,k'\}$ and Eve is playing an arbitrary strategy $\sigma$. Then, eventually play reaches $\Win$ or $\bot$ with pr. 1,  because in every $k+1$ steps, either $s$ is visited or either $\Win$ or $\bot$ is reached, and after $s$ has been visited, $k'$ is next half the time and from $k'$ $\bot$ is reached with positive pr.

Consider an $\epsilon$-optimal strategy $\tau$ for Adam, for $\epsilon<1/2$. 
Then, let Eve's mirror strategy to $\tau$ be $\sigma_{\tau}$. Now, either $\Win$ or $\bot$ is reached and because the strategies mirrors each other, the pr. to reach $\Win$ is equal to that of reaching $\bot$. Thus, we see that the value is at most $1/2$, implying that it is exactly $1/2$.

It also follows that the strategies that are $\epsilon$-optimal mirrors each other.
\end{proof}



We will now argue that Eve's (and thus Adam's) $\frac{1}{4}$-optimal strategies requires high patience.

To do so we will use the following lemma, showing that you can sometimes modify a concurrent game (or any of its special cases) and get a game with less value. While the proof is explicitly for concurrent reachability games, the proof is basically identical for concurrent discounted and mean-payoff games.
In a game $G$, for a vertex $v$ and a duration $T$, let $v^T_G$ be the value of the time-limited game with duration $T$.

\begin{lemma}\label{lem:change_succ}
Consider a concurrent reachability game $G$ and a pair of vertices $u,v$, such that for all $T$, we have that $u^T_G\geq v^T_G$. Consider a vertex $w$ such that for a pair of actions, $(r,c)$ we have that $v\in \supp(\dest(w,r,c))$. Consider an alternate game $G'$ equal to $G$, except that some of the probability mass is moved from $v$ to $u$ when playing $(r,c)$ in $w$, i.e. $0<\dest(G,w,r,c)(v)-\dest(G',w,r,c)(v)=\dest(G',w,r,c)(u)-\dest(G,w,r,c)(u)$.
Then for all vertices $z$ we have that $z^T_G\leq z^T_{G'}$
\end{lemma}
\begin{proof}
The proof is by induction in $T$.
The proof is trivial for $T=0$, because $z^T_G=0= z^T_{G'}$ for all non-goal vertices (and the goal vertex $\Win$ has value 1).
For $T\geq 1$, we have that $z^{T-1}_G\leq z^{T-1}_{G'}$.
But, matrix games are monotone in their entries, so it follows directly that for $z\neq w$ we have that $z^{T}_G\leq z^{T}_{G'}$.
Consider the matrix for $w^T_G$ compared to $w^T_{G'}$. All entries but the one for $(r,c)$ are smaller directly by induction. We also have that $v^T_{G}\leq u^T_{G}\leq u^T_{G'}$, the first inequality by definition and the second by induction. We thus see that all entries in $w^T_{G}$ are smaller than in $w^T_{G'}$.
The lemma follows.
\end{proof}

We are now ready to find the patience in concurrent reachability games.

\begin{lemma}
Any $(1/4)$-optimal strategy, for either player, in purgatory duel $k$ has patience at least $(3/4)^{-2^{k-1}}$ for each $k$
\end{lemma}
\begin{proof}
We will show that the lemma is true for Eve's strategies and that it is true for Adam's follows from Lemma \cref\{lem:change_succ}.
Consider an $(1/4)$-optimal strategy $\sigma$ for Eve. Fixing this strategy for Eve, we get a MDP for Adam. Clearly, in this MDP $G'$, we have that $0=\bot^T_{G'}\leq s^T_{G'}$ for all $T$. We can thus apply Lemma \cref\{lem:change_succ} to changing $\dest(1',1,1)$ from $\bot$ to $s$. In the resulting game $G''$, 
we still have that $0=\bot^T_{G''}\leq s^T_{G''}$ and thus, we can change $\dest(1',2,2)$ from $\bot$ to $s$.
Let the next game be $G^*$.
Thus, for any $i'\in \{1',\dots,k'\}$, the plays from $i'$ to $\bot$ in $G^*$ all goes through $s$. Note that Adam can ensure that the play reaches $s$ from $i'$ and thus, when he plays optimally, do so.
Thus, whenever $s$ is entered and Adam plays optimally, $k$ is enter eventually with pr. 1.
For the purpose of the value, we can thus disregard $s$ and vertices in $\{1',\dots,k'\}$ and just view each edge going to $s$ as going to $k$ instead.
But the resulting game is purgatory $k$ (in which Eve has fixed his strategy) and Eve is playing a strategy that gives value at least $1/4$, which requires  at least $(3/4)^{-2^{k-1}}$ patience, by Lemma \cref\{lem:purgatory}.
\end{proof}


\section{Concurrent mean-payoff games}
\label{7-sec:mean_payoff}
In this section we consider concurrent mean-payoff games. 
We will show that in general, any  $\epsilon$-optimal strategy in some concurrent mean-payoff games are quite complex. 
We will first, however, show that finding the value of a concurrent mean-payoff game can be done in polynomial space.

\begin{lemma}\label{lemm:class_meanpayoff}
Concurrent mean-payoff games are determined and the value is the limit of the value of the corresponding time-limited game as well as the limit of the corresponding discounted game, for the discount factor going to 0 from above.
There is an polynomial time algorithm, ala Lemma~\ref{lem:val1}, for finding the set of vertices where a finite memory strategy suffice to ensure $1-\epsilon$ (recall that all rewards are in $\{0,1\}$).
For any fixed number $n$, there is a polynomial time algorithm for approximating the value in a concurrent mean-payoff game with $n$ vertices (i.e. the running time is polynomial in the number of actions)
\end{lemma}
We will not show this lemma, but simply note that the $\epsilon$-optimal strategies known for general concurrent mean-payoff games  can be viewed as playing the corresponding discounted game with a variable discount factor that depends on how ``nice'' the rewards has been up to now. Basically, in each round you play the optimal strategy in the corresponding discounted game with a discount factor $\gamma$. Whenever 
 your rewards are close to or better than the value, you decrease $\gamma$ towards 0 and in each round your rewards are much worse than the value you let $\gamma$ increase, except not bigger than the initial $\gamma$ in the first round. Much of this section will argue that many natural candidates for simpler types of strategies does not work.


We will show that approximating the value, however, can, as mentioned, be done in polynomial space. The proof relies on Proposition~22 from~\cite{HKLMT:2011}, stating the following:
\begin{proposition}
Let $\epsilon=2^{-j}$, where $j$ is some positive integer, and the probabilities be rational numbers where the nominator and denominator have bitsize at most $\tau$. Also, let $\lambda=\epsilon^{\tau m^{O(n^2)}}$. Consider some state $s$ and let the value of that state in the $\lambda$ discounted game be $v_{\lambda}$ and the value in mean-payoff game be $v$, then $|v-v_{\lambda}|<\epsilon$.
\end{proposition}

We will use that to again reduce to the existential theory over the reals. 
For a fixed discount factor $\gamma$, we can easily express the value of the corresponding discounted game, like we expressed the value of a concurrent reachability game.
We have that the value $v$ is then $v=\lim_{\gamma\rightarrow 0^+} f(\gamma)$, where $f$ is the found expression.
I.e. for any $\epsilon$, there is a $\gamma'$ such that for all $\gamma<\gamma$, we have that $|f(\gamma)-v|\leq \epsilon$.
Also, that $v>c$ means that there is $\epsilon$, such that $v-\epsilon>c$.

The problem is thus to come up with a polynomial sized formula to express that $\lambda$ is $\epsilon^{\tau m^{O(n^2)}}=2^{-j \tau m^{O(n^2)}}$.

That can be done as follows, using $\ell=O(n^2)\cdot \log(m)+\log(j\tau)$ many variables, $v_0,v_1,\dots v_{\ell-1}$:
\[
v_0=1/2
\]
and for all $0<i< \ell$, we have that
\[
v_i=v_{i-1}\cdot v_{i-1}.
\]
Using induction, we see that $v_i=2^{-2^{i}}$, i.e., $v_1=1/2=2^{-2^0}$ and \[
v_i=v_{i-1}\cdot v_{i-1}=2^{-2^{i-1}}\cdot 2^{-2^{i-1}}=2^{-2^{i-1}-2^{i-1}}=2^{-2^{i}}\]
In particular, \[
v_{\ell-1}=2^{-2^{\ell}}=2^{-2^{O(n^2)\cdot \log(m)+\log(j\tau)}}=2^{-j\tau m^{O(n^2)}}
\] is the value we wanted for $\lambda$.
Thus, for a given number $v$, we can test if the value of a concurrent  $\lambda$-discounted game is above $v+\epsilon$, which, using the proposition above, implies that $v$ is below the value of the corresponding concurrent mean-payoff game. On the other hand, the proposition also implies that if the value of the concurrent  $\lambda$-discounted game is below $v-\epsilon$, then the value of the concurrent mean-payoff game is below $v$. Being able to answer these questions lets you easily approximate the value of a concurrent mean-payoff using binary search. 

We get the following lemma.
\begin{lemma}
Approximating the value of a concurrent mean-payoff game can be in done in polynomial space
\end{lemma}



We will now consider a specific, well-studied example of a concurrent mean-payoff game, since it shows that many natural kinds of strategies do not suffice in general.
The game is called the big match and is defined as follows:
There are 3 vertices, $\{0,s,1\}$, where the vertices in $\{0,1\}$ are absorbing, and with value equal to their name.
The last vertex $s$ has a 2x2-matrix and for all $i,j$ for $i\neq j$, we have that 
$c(s,1,1)=1$, and for $i\neq 1\neq j$ we have that $c(s,1,1)=0$.
Also,  $\dest(s,1,i)=s$ for each $i$, $\dest(s,2,1)=0$ and $\dest(s,2,2)=1$. There is an illustration in Figure \ref{7-fig:bm}.
The value of the Big Match is $1/2$.

\begin{figure}

\center
\begin{tikzpicture}[node distance=3cm,-{stealth},shorten >=2pt]
\ma{s}[$s:$]{2}{2};

\node at (s-1-1.center) {$1$};
\node at (s-2-1.center) {$0^*$};
\node at (s-1-2.center) {$0$};
\node at (s-2-2.center) {$1^*$};


\end{tikzpicture}
\caption{The Big Match}\label{7-fig:bm}
\end{figure}

Consider a finite-memory strategy $\sigma$ for Eve. We will argue that $\sigma$ cannot guarantee $\epsilon$ (any strategy can guarantee $-1$, since the colors are between $0$ and $1$) for any $0<\epsilon$. Let $\tau$ be the stationary strategy for Adam that plays $1$ with pr. $\epsilon/2$.
Then playing $\sigma$ against $\tau$, we get an Markov chain, where the vertex space is pairs of memory states and game vertices. 
In Markov chains, eventually, with pr. 1, a set of vertices $S$ is reached such that the set of vertices visited infinitely often is $S$. Such a set is called ergodic.
The set $S$ can clearly only contain 1 game vertex, since whenever $s$ is left, it is never entered again.
Hence, if $S$ contains $s$, the pr. that play will ever reach $\{0,1\}$ is 0.
In the MC we get from the players playing $\sigma$ and $\tau$, let $T_{\epsilon/2}$ be such that with pr. $\epsilon/2$ some ergodic set has been reached. 
Let $\tau'$ be the strategy that plays $\tau$ for $T_{\epsilon/2}$ and afterwards plays $2$. 

When $\sigma$ is played against $\tau'$, either we reach $\{0,1\}$ and Adam plays 1 only finitely many times, while in $s$ (the latter because there are only finitely many numbers below $T_{\epsilon/2}$). Thus, for Eve to win a play, the play needs to reach vertex 1. There are two ways to do so, either Eve stops before $T_{\epsilon/2}$ or after. In the former case, the pr. to reach $1$ is only $\epsilon/2$ (because Adam needs to play $2$ at the time, which is only done with pr. $\epsilon/2$). The latter only happens with pr. $\epsilon/2$ by definition of $T_{\epsilon/2}$ (because, Adam could play $2$ for an arbitrary number of steps while following $\tau$ but $s$ would not be left anyway).

We get the following lemma.

\begin{lemma}\label{lemm:no_finite_meanpayoff}
No finite memory strategy can guarantee more than $0$ in the Big Match.
\end{lemma}

The principle of sunken cost states that, when acting rationally, one should disregard cost already paid. We will next argue that this does not apply (naively) to the Big Match.
A strategy following the principle of sunken cost would not depend on past cost paid and thus, in each step $T$, there is a pr. $p_T$ of stopping for Eve.
Such strategies are called Markov strategies in the Big Match.
Fix some Markov strategy $\sigma$ for Eve. We will argue, like before, that $\sigma$ cannot guarantee more than $\epsilon$ for any $\epsilon>0$.
Note that Eve does not depend on the choices of Adam and thus, either she stops with pr. 1 or she does not.
In the former case, Adam just plays $1$ forever. When Eve stops, the vertex reached is thus $-1$.
Alternately, if Eve does not stop with pr. 1, there must be a time $T$, such that she only stops with pr. $\epsilon$ after $T$ (this is actually also the case even if she stops with pr. 1). 
Adam's strategy is then to play $1$ for $T$ steps and $2$ thereafter. Observe that the pr. to reach $1$ is thus at most $\epsilon$, in that it must be that Eve stops after $T$. If she does not stop (or stops in $0$), there will be only finitely many 1s.

We see the following:
\begin{lemma}\label{lemm:no_markov_meanpayoff}
No Markov strategy can guarantee more than $0$ in the Big Match
\end{lemma}


\section*{Bibilographic references}
\label{7-sec:references}
As discussed in the introduction, the literature on multiobjective models is too vast to provide a full account here. We therefore focus on some directions particularly relevant to our focus.

\paragraph{Multidimension games.} Energy games and their related work were discussed in~\cref{chap:counters}. Our presentation of mean-payoff games is inspired by Velner et al.~\cite{Velner&al:2015}. Brenguier and Raskin studied the Pareto curves of these games in~\cite{Brenguier&Raskin:2015}. While we considered \textit{conjunctions} of mean-payoff objectives, Velner proved that Boolean combinations lead to undecidability~\cite{Velner:2015}.

The undecidability of total-payoff games was first established in~\cite{Chatterjee&al:2015} via reduction from the halting problem for two-counter machines: we provided here a new, simpler proof based on robot games~\cite{Niskanen&Potapov&Reichert:2016}. This undecidability result, along with the complexity barriers of mean-payoff and total-payoff games, motivated the introduction of (multidimension) ""\textit{window objectives}"": conservative variants of mean-payoff and total-payoff objectives that benefit from increased tractability and permit to reason about time bounds~\cite{Chatterjee&al:2015}. Window variants of "parity" objectives have been studied in~\cite{Bruyere&Hautem&Randour:2016}.

\paragraph{Combinations of different objectives.} We focused on multidimension games obtained by conjunction of \textit{identical} objectives. Conjunctions of \textit{heterogeneous} objectives have been studied in a variety of contexts including mean-payoff parity games~\cite{Chatterjee&Henzinger&Jurdzinski:2005,Daviaud&Jurdzinski&Lazic:2018}, energy parity games~\cite{Chatterjee&Doyen:2012,Chatterjee&Randour&Raskin:2014}, average-energy games with energy constraints~\cite{Bouyer&al:2018,Bouyer&al:2017}, simple quantitative objectives~\cite{Bruyere&Hautem&Raskin:2016}. Le Roux, Pauly and Randour studied general conditions under which finite-memory strategies suffice to play optimally, even in a broad multi-objective setting~\cite{LeRoux&Pauly&Randour:2018}.


\paragraph{Beyond worst-case synthesis.} Our presentation is mostly based on~\cite{Bruyere&al:2017}, where all technical details can be found. As noted in~\cite{Bruyere&al:2017}, allowing large inequalities in the BWC problem may require infinite-memory strategies. The case of infinite-memory strategies was studied in~\cite{Clemente&Raskin:2015} along with multidimension BWC mean-payoff problems. BWC problems were studied for other objectives, such as shortest path~\cite{Bruyere&al:2017} or parity~\cite{Berthon&Randour&Raskin:2017}; and on other related models (e.g.,~\cite{Brazdil&Kucera&Novotny:2016,Almagor&Kupferman&Velner:2016}). BWC principles have been implemented in the tool \textsc{Uppaal}~\cite{David&al:2014}.

Comparisons with other rich behavioural models can be found in~\cite{Randour&Raskin&Sankur:2015,Brenguier&al:2016}.








