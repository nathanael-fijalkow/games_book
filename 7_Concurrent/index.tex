

%*** General probabilistic notation ***

\newcommand{\expv}{\mathbf{E}} % EXP. VALUE
\newcommand{\discProbDist}{f} % Discrete prob distribution
\newcommand{\sampleSpace}{S} % Generic sample space
\newcommand{\sigmaAlg}{\mathcal{F}} % Generic sigma-algebra
\newcommand{\probm}{\mathbb{P}} % Generic probability measure, also prob. measure operator
\newcommand{\rvar}{X} % Generic random variable
%\newcommand{\dist}{\mathit{Dist}}

%*** MDP notation ***

\newcommand{\actions}{A} % The set of actions.
\newcommand{\colouring}{c} % the colouring function
\newcommand{\probTranFunc}{\Delta} % Transition function of an MDP
\newcommand{\edges}{E} % Set of edges in an MDP.
\newcommand{\colours}{C} % The set of colours in an MDP.
\newcommand{\mdp}{\mathcal{M}} % A generic MDP. 
\newcommand{\vinit}{v_0} % An initial vertex in an MDP.
\newcommand{\cylProb}{p} % Function assigning probabilities to cylinder sets in 
%the measure construction.
\newcommand{\emptyPlay}{\epsilon} %empty play
\newcommand{\objective}{\Omega} % Qualitative objective
\newcommand{\genColour}{\textsc{c}} % Generic colour
\newcommand{\quantObj}{f} % Generic quantitative objective
\newcommand{\indicator}[1]{\mathbf{1}_{#1}} % In.d RV
\newcommand{\eps}{\varepsilon} % Numerical epsilon
\newcommand{\maxc}{W} % Maximal abs. value of a colour

\newcommand{\winPos}{W_{>0}}
\newcommand{\winAS}{W_{=1}}
\newcommand{\cylinder}{\mathit{Cyl}}

\newcommand{\PrePos}{\text{Pre}_{>0}}
\newcommand{\PreAS}{\text{Pre}_{=1}}

\newcommand{\PreOPPos}{\mathcal{P}_{>0}}
\newcommand{\OPAS}{\mathcal{P}_{=1}}

\newcommand{\safeOP}{\mathit{Safe_{=1}}}
\newcommand{\closed}{\mathit{Cl}}

\newcommand{\reachOP}{\mathcal{V}}
\newcommand{\discOP}{\mathcal{D}}
\newcommand{\valsigma}{\vec{x}^{\sigma}}

\newcommand{\lp}{\mathcal{L}}
\newcommand{\lpdisc}{\lp_{\mathit{disc}}}
\newcommand{\lpmp}{\lp_{\mathit{mp}}}
\newcommand{\lpsol}[1]{\bar{#1}}
\newcommand{\lpmpdual}{\lpmp^{\mathit{dual}}}

\newcommand{\actevent}[3]{\actions^{#1}_{#2,#3}} % Returns #1-th action on the run 

\newcommand{\MeanPayoffSup}{\MeanPayoff^{+}}
\newcommand{\MeanPayoffInf}{\MeanPayoff^{-}}

\newcommand{\mcprob}{M}
\newcommand{\invdist}{\vec{z}}

\newcommand{\hittime}{T}



This chapter considers concurrent games. The concurrent games we consider are extensions of the games considered in \cref{2-chap:regular} and \cref{4-chap:payoffs}, but where the choice of which edge to choose in a round is determined not by the choice of the owner of the vertex (indeed the vertices in concurrent games have no owners), but by the outcome of a matrix game corresponding to the vertex and played in that round. 
A matrix game is in turn a generalization of rock-paper-scissors, where each player picks an action simultaneously and then their pair of actions determines the outcome.

We will consider concurrent discounted, reachability and mean-payoff games and the definitions of the different objectives is as in the introduction. 
The chapter is divided into four sections:
\begin{enumerate}
\item The first section considers matrix games
\item The second section focuses on concurrent discounted games
\item The third section considers concurrent reachability games
\item The fourth section is about concurrent mean-payoff games
\end{enumerate}
As we go through the sections in this chapter, the complexity of the strategies and the computational complexity of solving the games rises: Indeed, since the games are generalizations of rock-paper-scissors, the strategies used requires randomness, but towards the end, no optimal or finite-memory $\epsilon$-optimal strategies exists in general and even the principle of sunken cost does not apply! 
Even with all this, the related questions about values are solvable in polynomial space and thus also in exponential time even in the last section.
The results we will focus on characterizes the complexity of the both the strategies as well as the computational complexity.
In each section we first give some positive result and some number of negative results. Each negative result also applies to the classes of games considered in the latter sections and each positive result applies to the classes considered in earlier sections (however, the positive results of latter sections will have worse complexity than the positive results from earlier sections).
As mentioned, the strategies for this chapter requires randomness and not too surprisingly, this implies that there is little difference between having stochastic or deterministic transition functions.

\section{Notations}
\label{7-sec:notations}
The definition of arena $\arena$ in this chapter is $\arena=(G,\dest)$, where $G=(V,E)$ is a graph and $\dest:V\times A\times A\rightarrow \Dist(E)$. In particular, we are not using the sets $\VA$ and $\VE$.
The games are played similarly to before and formally as follows: 
There is a token, initially on the initial vertex. 
Whenever the token is on some vertex $v$, 
Eve selects an action $r$ in $A$ and Adam selects an action $c$ in $A$. The edge $e=(v,c,w)$ is then drawn from the distribution $\dest(v,r,c)$ and the token is pushed from $v$ to $w$.
 In general, the game continues like that forever.

We will use the following simplifying assumptions in this chapter:
\begin{enumerate}
\item We will assume that all colors are in $\{0,1\}$, except for the section on Matrix games where we additionally also allow $-1$ (to be able to easily illustrate the game rock-paper-scissors). This simplifies some expressions, but generally, the dependency on the number of colors is not too bad comparatively.
\item To make illustrations easier, we assume that for any pair of edges $e,e'$ in $\dest(v,a,a')$ for any $v,a,a'$, we have that $c(e)=c(e')$, i.e. the color does not depend on which edge is picked from $\dest(v,a,a')$, but only $v,a,a'$. This assumption does not matter for the types of games considered.
\end{enumerate}

We will overload the notation slightly for notational convenience, in that for any $v,a,a'$, we will write $c(v,a,a')$ for $c(e)$ where $e\in \dest(v,a,a')$ (note that the second assumption ensures that this is well-defined, i.e. there is only one such color).


A vertex $v$ is absorbing iff each player has only 1 action in $v$ and $\Delta(v,1,1)=v$.

To describe the complexity of good stationary strategies in concurrent games, we will use the notion of patience. Given a probability distribution $d\in \Dist$ the distribution has patience $p$ if $p=\min_{i\in \supp(d)} d(i)$ (i.e. the patience is the smallest, non-zero probability that an event may happen according to $d$).
In essence, if you have low enough patience you can typically guess the strategy and check whether it is a good strategy (when you fix a strategy, the game becomes a Markov decision process, which are relative easy to work with), the game can solved in $\NP\cup \coNP$. However, some times the patience is huge and writing down a good strategy, in binary, cannot be done in polynomial space (it is quite surprising in some sense that even with this property, finding the values in the games remain in $\PSPACE$).

We will illustrate a stochastic arena $\arena=(G,\dest)$ as follows:
For each non-absorbing vertex $v$, there is matrix.
 Entry $(i,j)$ of the matrix illustrating $v\in V$ describes what happens if, when the token is on $v$, Eve plays $i$ and Adam $j$. The entry contains a color $c$, which is $c(v,i,j)$, and 
there is an arrow from entry $(i,j)$ of $v$ to $w$ if there is an edge   
$e=(v,c,w)$ in $\dest(v,i,j)$. 
 The arrow corresponding to $e$ is denoted with the probability $\dest(v,i,j)(e)$. 
Especially, to simplify the illustrations we will do as follows: If $|\supp(\dest(v,i,j))|=1$, we do not include the probability (which is 1). Also, in that case, let $e=(v,c,w)=\dest(v,i,j)$ 
and 
if $v=w$, we omit the arrow and if $w$ is absorbing we write $c^*$ in entry $i,j$ of $v$, where $c$ is the color $c(w,1,1)$ (in this case, we omit the number $c(e)$ from the illustration, but in none of our illustrations does this number matter for what we try to illustrate). 


\section{Matrix games}
\label{7-sec:matrix_games}
A matrix game is a game defined from a $(R\times C)$-matrix $M$  of numbers for some $R,C$.
The game is played as follows: Eve picks a row $r$ and Adam picks a column $c$ simulations like in rock-paper-scissors. Adam then pays Eve $M[r,c]$, i.e. the content of the entry defined by being in row $r$ and column $c$.
A strategy in such a game for Eve (resp. Adam) consists of a distribution over the rows (resp. columns). 
There is an illustration of rock-paper-scissors as a matrix game in Figure~\cref{7-fig:rps}.


\begin{figure}

\center
\begin{tikzpicture}[node distance=3cm,-{stealth},shorten >=2pt]
\ma{main}[]{3}{3};

\node at (main-1-1.center) {0};
\node at (main-2-2.center) {0};
\node at (main-3-3.center) {0};
\node at (main-1-2.center) {-1};
\node at (main-2-3.center) {-1};
\node at (main-3-1.center) {-1};
\node at (main-1-3.center) {1};
\node at (main-2-1.center) {1};
\node at (main-3-2.center) {1};


\end{tikzpicture}
\caption{Rock-paper-scissors. The color is 1 if Eve wins, 0 if they draw and -1 if Adam wins. Also, the actions are ordered as in the name of the game}\label{7-fig:rps}
\end{figure}

The following theorem lists some known results for matrix games:
\begin{theorem}\label{lem:mat}
Each $(m\times n)$-matrix game $M$ is determined and there exists optimal strategies for each player. 
\begin{itemize}
\item The value and an optimal strategy for each player can be found in polynomial time and the problem is equivalent to linear programming.

\item Let $c>0$ be some constant. Consider the matrix $cM$ where each entry of $M$ has been multiplied by $c$. Then, the value of $cM$ is $cv$.
\item Let $c$ be some constant. Consider the matrix $M+c$ where each entry of $M$ is $c$ larger (additively). Then, the value of $M+c$ is $v+c$.
\item The value of matrix games are monotone in the entries.
\end{itemize}
\end{theorem}
We will omit the proof of the existence of values, optimal strategies and the first bullet.
The second bullet can be viewed as changing currency and clearly, this does not affect the optimal strategy.
The third bullet can be viewed as getting a reward before playing the game, and again, clearly this does not affect how to play it.
The last bullet can be seen from that each pair of strategies must give a higher reward if the entries of the matrix is higher.
This is especially true if you consider the optimal strategy for Eve in $M$ together with an arbitrary strategy for Adam, which then shows that the value is higher.

%if space: add proof

Given a matrix $M$, we will by $\Value[M]$ denote the value of the matrix game $M$. 

Perhaps interestingly, an illustration of a matrix $M$ can be viewed as a game arena $\arena$ (for concurrent games) with only one non-absorbing vertex. In each type of games considered in this section (apart from concurrent reachability games, where no game can be illustrated as a matrix with non-star entries different from 0), the value of the game with that arena matches $\Value[M]$ and the optimal strategies for each player is to play an optimal strategy from $M$ in each round. One can also consider a game arena $\arena^*$ with an illustration similar to $M$, but where there is a star in each entry (and $c(v,i,j)=0$ for the unique non-absorbing state $v$ and any pair of actions $i,j$).
Again, the value is $\Value[M]$ (except for the case of discounted objectives, where the value is $(1-\delta)\Value[M]$) and the optimal strategies for each player is again to play an optimal strategy from $M$. 

One could easily be lead to believe that in games (called repeated games with absorbing states) that can be illustrated as a single matrix $M$ with some entries stared and others not, the value would be similar to $\Value[M]$ and the optimal strategy would again be to play the optimal strategy from $M$. 
However, this is very much not true and indeed, many of the games in this chapter, illustrating how complex concurrent games can be, are repeated games with absorbing states! In particular repeated games with absorbing states may (1)~have irrational values and probabilities in optimal strategies (with any objective), (2)~have no optimal strategies (for reachability and mean-payoff objectives) and (3)~have no $\epsilon$-optimal finite-memory or $\epsilon$-optimal Markov strategies (for mean-payoff objectives)!


\section{Concurrent discounted games}
\label{7-sec:discounted}
In this section we focus on concurrent discounted games. 
The key property of these games is that to a high degree, only the relative early part of the play matters.
We will first argue that the value iteration algorithm works and especially converges to the value of the game and then that there are stationary optimal strategies in concurrent discounted games.
While the value iteration algorithm also works for the games considered in the latter sections, we will not explicitly show it there, since the proofs become much more complex. The argument here however will allow us to show quite a few more statements in essence as corollaries of the theorem that value iteration works.


The value iteration algorithm is based on the concept of finite-horizon (or time limited) games. It is also sometimes referred to as dynamic programming.
Specifically, apart from the usual definition of a game, there is an additional integer $T$, denoting how many rounds are remaining initially, and a vector $v$, assigning a reward to each node if the game ends in that node with 0 rounds remaining. After round $T$ the reward is 0. 
I.e. for $T=0$, the outcome reward from node $x$ is $v_x$ in the first round and 0 in each later round.
Let $\ValueOp^T(v)$ be the vector that assigns to each node its value in the game with time-limit $T$ with vector $v$.

In general this formulation leads to a simple dynamic algorithm that computes $\ValueOp^T(v)$ inductively in $T$. 
We have that $\ValueOp^0(v)=v$ and given $\ValueOp^{T-1}(v)$ it is easy to compute $\ValueOp^T(v)$ because, if Eve selects row $i$ and Adam column $j$ in node $x$ in the first round, the outcome is \[
\sum_{v\in V}\ValueOp^{T-1}(v)\dest(x,i,j)(v)
\]
and thus $(\ValueOp^T(v))_x$ is the value of the matrix $M^{T,x,v}$, where entry $i,j$ is \[
\sum_{v\in V}\ValueOp^{T-1}(v)\dest(x,i,j)(v)
\]

It is common to start with the all-0 vector for $v$ when using the value iteration algorithm.

The following lemma shows many interesting properties of concurrent discounted games.






\begin{lemma}\label{cor:long}
Concurrent discounted games have the following properties:
\begin{itemize}
\item The value iteration algorithm converges for any initial vector $v$
\item The value iteration algorithm has an unique fix-point, independent of the initial vector $v$.
\item There are optimal stationary strategies in concurrent discounted games and the unique fix-point of the value iteration algorithm is the value (thus, the games are determined)
\item The value of a concurrent discounted game can approximated in PPAD
\item There are $\epsilon$-optimal stationary strategies with patience below $\frac{m\log(\epsilon/2)}{\log(1-\gamma)\epsilon}$.
\end{itemize}
\end{lemma}
\begin{proof}
The first item comes from considering the vectors $v$ and $\ValueOp^1(v)$. We thus have that $\ValueOp^{T+1}(v)\in [\ValueOp^{T}(v)-(1-\gamma)^T,\ValueOp^{T}(v)+(1-\gamma)^T]$ for all $T$. The statement then comes from that $\sum_{i=1}^\infty (1-\gamma)^i$ is a converging sum.

The second item comes from considering two fix-points, $u,v$. I.e., $\ValueOp^1(v)=v$ and thus $\ValueOp^T(v)=v$ for all $T$. Similar for $u$.
But, $v=\ValueOp^T(v)\in [\ValueOp^T(u)-(1-\gamma)^T, \ValueOp^T(u)+(1-\gamma)^T]=[u-(1-\gamma)^T, u+(1-\gamma)^T]$. Since it is true for all $T$, we have that $u=v$.

\begin{claim}
 Consider some $T$ and the strategy for Eve that plays the first $T$ steps following an optimal strategy in the finite-horizon game of length $T$ with vector $v$, followed by playing arbitrarily. 
Then, the outcome is above $\ValueOp^T(v)- (1-\gamma)^T\max_{i} v_i$.
\end{claim}
\begin{proof}
For any strategy for Adam, the expected reward for the first $T$ rounds is at least the expected reward in the finite-horizon game. In each remaining round, the reward is at least $0$ in the real game, but $v_i$ in round $T$ for some $i$ followed by 0's in the finite-horizon game.
Since the outcome is $\ValueOp^T(v)$ in the finite-horizon game, the real outcome is then as described.
\end{proof}
One can show a similar statement for Adam.
For any $\epsilon>0$ one can pick a big enough $T$ such that $(1-\gamma)^T\max_{i} v_i\leq \epsilon$.


Let $v^*$ be the unique fix-point of the value iteration algorithm. 
Thus, $v^*=\ValueOp^T(v^*)$ for all $T$. Pick some optimal strategies $\sigma_x,\tau_x$ in $M^{T,x,v^*}$ for each $x$. Let $\sigma^*,\tau^*$ be the strategies that play $\sigma_x,\tau_x$ whenever in node $x$ in each round.
The strategy $\sigma,\tau$ are optimal in $\ValueOp^T(v^*)$ for each $T$, because $v^*$ is a fix-point. 
But, for each $\epsilon>0$,  the strategy $\sigma$ ensures outcome at least $v-\epsilon$ and $\tau$ ensures outcome at most $v+\epsilon$ using the claim. Hence, the third item follows.


The fourth item follows from that the value iteration algorithm is a contraction.

For the fifth item, consider the strategy used in the claim. 
Let $T$ be $\log(\epsilon/2)/\log(1-\gamma)$, i.e. $T$ is such that \[
\gamma \sum_{i=T}^{\infty}(1-\gamma)^i=\epsilon/2
\]
or in words, $T$ is such that the total outcome of each step after the $T$-th step is at most $\epsilon/2$.
Intuitively, if we modify the strategy very little, then the change is unlikely to come up in the first $T$ steps. More precisely, we will modify our strategy so that the probability that change will matter is less than $\epsilon/2$. That implies that the outcome differs by at most $\epsilon$ from the value.
We will use this intuition together with the argument for the third item to give a bound on the patience of $\epsilon$-optimal strategies. Fix some optimal stationary strategy $\sigma$ for Eve and an arbitrary stationary strategy $\tau$ for Adam. Let $\sigma'$ be a stationary strategy obtained from $\sigma$ rounded greedily  so that each probability is a rational number with denominator \[
q=mT/\epsilon=\frac{m\log(\epsilon/2)}{\log(1-\gamma)\epsilon}.
\] We will argue that $\sigma'$ is $\epsilon$-optimal.

The rounding proceeds inductively as follows for each node $x$:
The numbers $p_i$ are the original probability and the numbers $p_i'$ are the new probabilities.
For each $i$, the number $p_i'$ is defined as follows: If $\sum_{j=1}^{i-1}(p_i-p_i')>0$, then round up (i.e. $p_i'$ is the smallest rational with denominator $q$ so that $p_i<p_i'$) and otherwise round down, except the last number $p_\ell'$, which is such that $\sum_{j=1}^{\ell}p_i'=1$.
Note that this ensures that $-1/q<\sum_{j=1}^{i-1}(p_i-p_i')<1/q$. It also ensures that $|p_i-p_i'|<1/q$ for all $i$ (including for $i=\ell$).

For all nodes $x$ and rounds $T'\leq T$ we will define some random variables.
Specifically, the random variables denote what happen in round $T'$ if in node $x$.
The random variable $X_{x,T'}$ (resp. $Y_{x,T'}$) denotes the action picked by Eve if Eve follows $\sigma$ (resp. $\sigma'$).
The random variable $Z_{x,T'}$ denotes the action picked by Adam.
For each action pair $(i,j)$  the random variable $W_{x,i,j,T'}$ denotes the node entered in round $T'+1$, if Eve picks $i$ and Adam $j$. 
(As a side note: Each of the random variables are distributed the same way independent of $T'$).
Each of these random variables are independent of each other, except that (as we will define later) the random variables $X_{x,T'}$ and $Y_{x,T'}$ for each $x,T'$ are very much not independent of each other.

We see that we can view the first $T$ steps of the play when Eve follows $\sigma$ by only considering the outcome of $X_{x,T'}$, $Z_{x,T'}$ and $W_{x,i,j,T'}$ for all $T'$ and $x$ (even stronger: We only need to consider one $x,i,j$ for each $T'$, because the token is on only one node at a time). Similarly, for $\sigma'$, but using $Y_{x,T'}$ instead of $X_{x,T'}$.
For this to work, note that each random variable should be independent, except that the random variables $X_{x,T'}$ and $Y_{x',T''}$ need not be independent of each other for any $x',T''$. This is precisely the property we had for them!
For each $x,T'$,  we will then use a coupling $C_{x,T'}=(X'_{x,T'},Y'_{x,T'})$, a distribution over $[m]^2$, such that $X'_{x,T}$ is distributed as $X_{x,T'}$ and $Y'_{x,T'}$ is distributed as $Y_{x,T'}$. We will use a classic result for distributions, called the Coupling Lemma.

To introduce the Coupling Lemma, first, we need the notion of total variation distance. Given two distributions, $\Delta$ and $\Delta'$ over a set $S$, the total variation distance $t$ between $\Delta$ and $\Delta'$ is \[
t(\Delta,\Delta')=\frac{1}{2}\sum_{x\in S} |\Delta(x)-\Delta'(x)|
\] 

\begin{lemma}
For any distributions $\Delta$ and $\Delta'$ over a set $S$, we have 
\begin{itemize}
\item for all couplings $(X,Y)$ of $\Delta$ and $\Delta'$, that \[
t(\Delta,\Delta)\leq \Pr[X\neq Y]
\]
\item that there is a coupling $(X',Y')$ of $\Delta$ and $\Delta'$ satisfying that \[
t(\Delta,\Delta)= \Pr[X'\neq Y']
\]
\end{itemize}
\end{lemma}

Because of our rounding, we have that $t(X'_{x,T'},Y'_{x,T'})<\frac{m}{2q}$. 
Using that with the coupling lemma (the second part to be precise), lets us find a coupling $C_{x,T'}=(X'_{x,T'},Y'_{x,T'})$ 
such that $\Pr[X'_{x,T'}\neq Y'_{x,T'}]<\frac{m}{2q}$.

Consider the plays $\play_1,\play_2$ for when Eve follows $\sigma$ or $\sigma'$ respectively.
We can view the first $T$ steps of these plays by considering $X'_{x,T'}$ instead of $X_{x,T'}$ and similar when Eve follows $\sigma'$.
We can therefore see that the first $T$ steps two plays are different with probability 
$=p<\frac{mT}{2q}=\epsilon/2$
 using union bounds. 
 

We therefore see that the value for the path $\play_1$ cannot differ from the value of $\play_2$ with more than $p\gamma\sum_{i=1}^T(1-\gamma)^i=p$. I.e. in the worst case, if $\play_1$ and $\play_2$ differs, the reward is 1 in each step for $\play_1$ but 0 in each step for $\play_2$.
Also, the rewards in the steps after step $T$ can also differ by at most $1$ and by our choice of $T$, we have that outcome contributed from these remaining steps are worth less than $\epsilon/2$ as well.
Hence, we see that $\sigma'$ obtains the same as $\sigma$ except for $\epsilon$ against any strategy $\tau$ and is thus $\epsilon$-optimal.













\end{proof}

There is a classic problem in geometry called the sum-of-square-roots problem. The problem is defined as follows:
Let $a,b_1,b_2,\dots,b_n$ be natural numbers. Is $\sum_{i=1}^n\sqrt{b_i}>a$? 

The problem comes up for decision problems about distances in Euclidean space. It is not known to be in P or NP for that matter, but is in the fourth level of the countering hierarchy, slightly inside PSPACE. The issue is in essence that it is not known how good an approximation of $\sqrt{b_i}$ is necessary to decide the strict inequality. 

We will use the sum-of-square-roots problem to give an informal hardness argument, in that finding the exact value of a concurrent game is in general harder than solving the sum-of-square-roots problem. 

Consider the following game $G$:
There are three vertices, $\{0,1,s\}$ where $0$ and $1$ are absorbing, with color 0 and 1 respectively.
The vertex $s$ is such that (1)~$c(s,i,j)=0$, (2)~$\dest(s,i,i)=1$ (for $i\in \{1,2\}$), (3)~$\dest(s,2,1)=0$ and (4)~$\dest(s,1,i)$ is the uniform distribution over $s$ and $0$. The game is illustrated in Figure \cref\{7-fig:sqroot}.

Consider an optimal stationary strategy in $G$ for Eve. Let $p$ be the probability with which she plays the first action. If Adam knows that Eve will follow this strategy, the game devolves into a MDP. We know from that for such there exists optimal positional strategies and thus Adam is either going to play the left or right column always. Clearly, $0<p<1$ because $p=0$ or 1 means that either playing the left or right column with probability 1 would ensure that no positive reward ever happens.


\begin{figure}

\center
\begin{tikzpicture}[node distance=3cm,-{stealth},shorten >=2pt]
\ma{s}[$s:$]{2}{2};

\node at (s-1-1.center) {$1^*$};
\node at (s-2-2.center) {$1^*$};
\node at (s-2-1.center) {$0^*$};
\draw (s-1-2.center)  to ($(s-1-2.center)+(0.5cm,0)$)  arc (-90:180:0.5cm) node[pos=.5, right] {$1/2$};
\node at ($(s-1-2.center)-(0.2cm,0)$) {0};

\ma[shift={($(s)+(3 cm,0.5)$)}]{s0}[]{1}{1};
\node at ($(s0-1-1.east)+(0.5cm,0)$) {$:0$};
\node at (s0-1-1.center) {$0^*$};

\draw (s-1-2.center) -- (s0) node[below,pos=0.5] {$1/2$};


\end{tikzpicture}
\caption{Concurrent discounted game with value $v_s=-2+\sqrt{4+2(1-\gamma)}$}\label{7-fig:sqroot}
\end{figure}





Let $v_0=0,v_1=1,v_s$ be the values of the three vertices. If he plays the left column, the outcome is $p(1-\gamma)$.
If he plays the right column, the outcome is $p/2(1-\gamma)v_s+(1-p)(1-\gamma)$. Observe that the former is increasing in $p$ and the latter is decreasing (since clearly, $0<(1-\gamma)v_s<v_s$). Also, both are continues. Thus, the optimum is for $p(1-\gamma)$ to be equal to $p/2(1-\gamma)v_s+(1-p)(1-\gamma)$ and both equal to $v_s$.
We will first isolate $v_s$ in $v_s=p/2(1-\gamma)v_s+(1-p)(1-\gamma)$.
\begin{align*}
v_s&=p/2(1-\gamma)v_s+(1-p)(1-\gamma)\Rightarrow (1-p/2(1-\gamma))v_s=(1-p)(1-\gamma)\Rightarrow \\
v_s&=\frac{(1-p)(1-\gamma)}{1-p/2(1-\gamma)}\enspace .
\end{align*}

Note that $p,\gamma<1$ thus, $1-p/2(1-\gamma)\neq 0$.
We then have the equality 
\begin{align*}
\frac{(1-p)(1-\gamma)}{1-p/2(1-\gamma)}&=p(1-\gamma)\Rightarrow\\(1-p)(1-\gamma)&=p(1-\gamma)(1-p/2(1-\gamma))\Rightarrow\\
0&=\frac{1-\gamma}{2} p^2+2p-1\Rightarrow \\
p&=\frac{-2\pm\sqrt{4+2(1-\gamma)}}{1-\gamma} \enspace .
\end{align*}

We see that $\frac{-2-\sqrt{4+2(1-\gamma)}}{1-\gamma}<0$. Thus, $p=\frac{-2+\sqrt{4+2(1-\gamma)}}{1-\gamma}$.
Also, \[v_s=-2+\sqrt{4+2(1-\gamma)}\enspace .\] 
It is straight-forward to modify the construction to get any square-root desired for a fixed $\gamma$. 



By making such a construction for each number $\sqrt{b_i}$, we can make another vertex $s^*$ that has the value of $(1-\gamma)\frac{\sum_{i=1}^n\sqrt{b_i}}{n}$ with a single action for each player and that goes to a uniformly random vertex. 
Observe that decreasing each reward by $x$, reduces the value of each vertex by $x$. Reduce each reward by $\frac{an}{1-\gamma}$.
We can then decide the sum-of-square-roots problem by deciding whether the value of $s^*$ is strictly above $0$. 

We get the following lemma.

\begin{lemma}
The (exact) decision problem for the value is sum-of-square-root hard for concurrent discounted games
\end{lemma}

We will use this game  $G$ as an example to illustrate how to make the $\dest$-function deterministic for concurrent games while having the same value and a similar optimal strategy.


\begin{figure}

\center
\begin{tikzpicture}[node distance=3cm,-{stealth},shorten >=2pt]
\ma{s}[$s:$]{3}{3};

\node at (s-1-1.center) {$1^*$};
\node at (s-2-1.center) {$1^*$};
\node at (s-3-2.center) {$1^*$};
\node at (s-3-3.center) {$1^*$};
\node at (s-3-1.center) {$0^*$};
\node at (s-1-2.center) {0};
\node at (s-2-3.center) {0};
\node at (s-2-2.center) {$0^*$};
\node at (s-1-3.center) {$0^*$};

\end{tikzpicture}
\caption{Alternate concurrent discounted game with value $v_s=-2+\sqrt{4+2(1-\gamma)}$}\label{7-fig:sqroot2}
\end{figure}

Consider the following game $G'$:
There are three vertices, $\{0,1,s\}$ where $0$ and $1$ are absorbing, with color 0 and 1 respectively.
The vertex $s$ is such that (1)
$c(s,i,j)=0$ for all $i,j$, (2)~$\dest(s,i,j)=s$ for $i+1=j$ (i.e. for $(i,j)\in \{(1,2),(2,3)\}$), (3)~$\dest(s,i,j)=0$ for $i+j=4$ (i.e. the ``other'' diagonal, $(i,j)\in \{(3,1),(2,2),(1,3)\}$) and (4)~$\dest(s,i,j)=1$ otherwise (i.e. for $(i,j)\in \{(1,1),(2,1),(3,2),(3,3)\}$).
 The game is illustrated in Figure \cref\{7-fig:sqroot2}.
 
We will argue that the value of $G'$ is equal to that of $G$.
 We clearly have that the value of $s$ is in $(0,1)$.
Consider a stationary strategy $\sigma$ for Eve
such that $\sigma(1)\neq \sigma(2)$. Let $p_i=\sigma(i)$ for $i\in \{1,2,3\}$. 
Let $\sigma'$ be such that $\sigma'(3)=p_3$ and otherwise, $\sigma'(i)=\frac{p_1+p_2}{2}$ for $i\in\{1,2\}$. Let $p_i'=\sigma'(i)$ for $i\in \{1,2,3\}$.
\begin{claim}
The strategy $\sigma'$ is at least as good as $\sigma$
\end{claim}
\begin{proof}
If Adam plays 1, then the expected outcome is 
$p_1+p_2=p'_1+p'_2$ no matter if Eve plays $\sigma$ or $\sigma'$. 
If he plays $i$ for $i\in\{2,3\}$, the expected outcome is $
\frac{p_{4-i}}{1-p_{i-1}}$ if Eve plays $\sigma$ and otherwise, if she plays $\sigma'$, the expected outcome is   
$\frac{p'_1}{1-p'_2}=\frac{p'_2}{1-p'_1}$. 
Note that $\frac{p'_1}{1-p'_2}>\min_{i\in\{2,3\}\frac{p_{4-i}}{1-p_{i-1}}}$ and thus, $\sigma'$ is at least as good a strategy as $\sigma$.
\end{proof}

A similar argument shows that for any strategy $\tau$ for Adam the similar strategy $\tau'$ where $\tau'(1)=\tau(1)$ and $\tau'(i)=\frac{\tau(2)+\tau(3)}{2}$ for $i\in\{2,3\}$ is at least as good as $\tau$.
Consider that the players follows such stationary strategies $\sigma'$ and $\tau'$.
Let $\sigma$ be 
\[\sigma(i)=\begin{cases} \sigma'(1)+\sigma'(2)&\text{if }i=1\\
\sigma'(3)&\text{if }i=2\enspace .\end{cases}\]
Similarly, let 
 $\tau$ be 
\[\tau(i)=\begin{cases} \tau'(1)&\text{if }i=1\\
\tau'(2)+\tau'(3)&\text{if }i=2\enspace .\end{cases}\]
But playing $\sigma$ and $\tau$ in $G$ gives the same outcome as playing $\sigma'$ and $\tau'$ in $G'$ as can be seen as follows: In either game, with probability
\[
\sigma'(1)\tau'(2)+\sigma'(2)\tau'(3)=\frac{\sigma(1)\tau(2)}{2}\] we play again with a reward of 0, with probability 
\[\sigma'(1)\tau'(3)+\sigma'(2)\tau'(2)+\sigma'(3)\tau'(1)=
\frac{\sigma(1)\tau(2)}{2}
+\sigma(2)\tau(1)
\]
we get absorbed in 0 after a reward of 0 and with probability \[
(\sigma'(1)+\sigma'(2))\tau'(1)+(\tau'(2)+\tau'(3))\sigma'(3)
\] we get absorbed in 1 after a reward of 0.
But this is in particular the case if the players play optimally and thus, the value is the same in the two games.



Before, in Corollary \cref\{cor:long}, we argued that the patience of $\epsilon$-optimal stationary strategies was $q=\frac{m\log(\epsilon/2)}{\log(1-\gamma)\epsilon}$.
Giving a similar exponential bound for the optimal stationary strategies is harder than solving the sum-of-square-roots problem, as we will argue next.
Assume that we had an exponential bound for optimal stationary strategies.

\begin{figure}

\center
\begin{tikzpicture}[node distance=3cm,-{stealth},shorten >=2pt]
\ma{s}[$s':$]{2}{2};

\node at (s-1-1.center) {$1^*$};
\node at (s-2-1.center) {$0^*$};
\node at (s-1-2.center) {$0^*$};

\ma[shift={($(s)+(3 cm,-0.5)$)}]{ss}[]{1}{1};
\node at ($(ss-1-1.east)+(0.5cm,0)$) {$:s^*$};
\node at ($(s-2-2.center)-(0.2cm,0)$) {0};


\draw [yscale=-1] (s-2-2.center)  to ($(s-2-2.center)+(0.5cm,0)$)  arc (-90:180:0.5cm);
\draw (s-2-2.center) -- (ss);


\end{tikzpicture}
\caption{Concurrent discounted game that implies that if there is an exponential lower bound on patience, then the sum-of-square-roots problem is in P}
\label{7-fig:exact-hard}
\end{figure}

Consider an arbitrary yes-instance of the sum-of-square-roots problem, giving a vertex $s^*$. Reduce each reward by $a$ and in the new game let $s^*_a$ be the vertex corresponding to $s^*$. 
We will now create a game that uses the previous game as a sub-game.
The game has 1 additional vertex $s'$, which is a 2x2-matrix, such that $c(s',i,j)=0$ and $\dest(s,1,1)=1$ and $\dest(s,2,2)=s^*$ and $\dest(s,i,j)=0$ for $i\neq j$.
There is an illustration in Figure~\cref\{7-fig:exact-hard}, using the vertex $s^*$ as above. 
Using an argument like above, we see that the probability $p$ to play the top action in the vertex $s'$ is such that $p(1-\gamma)=(1-p)(1-\gamma)x$, where $x$ is the value of $s^*$. Thus, $x=\frac{p}{1-p}$. If $p$ only needs to be exponential small, then $x$ is exponentially small as well. This is true for any yes-instance of the sum-of-square-roots problem and thus, we only need polynomially many digits to decide the problem. We can find polynomially many digits of $\sqrt{b_i}$ for each $i$ in polynomial time. We get the following lemma.

\begin{lemma}
Giving an exponential lower-bound on patience for optimal stationary strategies in concurrent discounted games implies that the sum-of-square-roots problem is in $\PTIME$
\end{lemma}


\section{Concurrent reachability games}
\label{7-sec:reachability}
%We start our study of algorithmic problems for MDPs with the most basic 
%objective: reachability. 
% As we shall see later, the general problem of 
% computing optimal reachability values and strategies can be reduced to optimizing 
% expected mean-payoff, without an increase in the theoretical complexity (as 
% opposed to the game case, where reachability games can be solved in polynomial 
% time, while for mean-payoff games no polynomial-time algorithm is known). 
% Hence, in this section 
%We first focus on solving the positive and almost-sure 
% reachability problems. 
 % While these problems are also in \P{} (and, as we shall 
% see, actually \P-complete), they can be efficiently solved by specific 
% techniques that do not apply in the more general setting.

%\subsection*{Positive and Almost-Sure Reachability}

\paragraph{Positive reachability}  
Analogously to classical games (cf. \cref{2-chap:regular}), we define a one-step \emph{positive 
probability} predecessor 
operator, $\PrePos$,
as follows: for $U\subseteq \vertices$ we put

\begin{align*}
\PrePos(U) &= \{v \in \vertices \mid \exists a \in \actions, \exists u \in U: 
\probTranFunc(u\mid v,a)>0 \}.\\
%\PreAS(U) &= \{v \in \vertices \mid \exists a \in \actions, \forall t \in 
%\vertices: \probTranFunc(t\mid v,a)>0 \Rightarrow t \in U \}.
\end{align*}

%Note that the operator $\PrePos$ is monotonic, i.e. $\PrePos(X) \subseteq 
%\PrePos(Y)$ if $X\subseteq Y$, while $\PreAS$ is \emph{antitone,} i.e.  
%$\PreAS(X) \supseteq \PreAS(Y)$ whenever $X\subseteq Y$.
\noindent
We also define an operator $\PreOPPos$ s.t. for each $X\subseteq\vertices$ we have
$$\PreOPPos(X) = X\cup \PrePos(X).$$
It is easy to see that $\PreOPPos$ is a classical 
reachability operator in the underlying graph of the MDP, i.e. denoting $X_0 = 
X$ and $X_i = \PreOPPos(X_{i-1})$, we get that $X_i$ is exactly the set of 
vertices from which a vertex of $X$ is reachable via a finite play of length at 
most $i$. It follows that iterating $\PreOPPos$ on any initial set reaches a 
fixed point in at most $n-1$ steps, where $n=|\vertices|$.

We have the following simple characterization of the positively winning set:

\begin{theorem}[Characterisation of the positively winning set]
\label{5-thm:positive-char}
For each vertex $v$, the following conditions are equivalent:
\begin{enumerate}
\item The vertex $v$ belongs to $\winPos(\mdp,\Reach(\genColour))$.
\item There 
exists a (possibly empty) finite play from $v$ to a vertex of colour $\genColour$.
\item The vertex $v$ 
belongs to the fixed point of the iteration $\vertices_\genColour, 
\PreOPPos(\vertices_\genColour),\PreOPPos^2(\vertices_\genColour),\cdots$.
\end{enumerate}
Moreover, there exists a memoryless deterministic strategy that is positively 
winning from each vertex in $\winPos(\mdp,\Reach(\genColour))$.
\end{theorem}
\begin{proof}
$(1)\Rightarrow(2)$: We have that $\Reach(\genColour) = \cup_{\play \in X} 
\cylinder(\pi)$, where $X$ is the set of all finite plays ending in a vertex of 
colour $\genColour$ and $\cylinder(\pi)$ is the basic cylinder determined by 
$\pi$. Since $X$ is a countable set, from the property (3.) of a probability 
measure it follows that $\probm^\sigma_{\mdp,v}(\Reach(\genColour))>0$ if and 
only if there exists $\play\in X$ with 
$\probm^\sigma_{\mdp,v}(\cylinder(\play))>0$.\footnote{Arguments of this style are said to invoke a ""union bound"". } For the latter to hold, it must 
be that either $\play=\emptyPlay$, in which case $\colours(v)=\genColour$, or 
$\play$ is a non-empty play initiated in $v$ and reaching a colour 
$\genColour$, as required.

\knowledge{union bound}{notion,index={union bound}}

$(2)\Rightarrow(3)$:
This is straightforward.
	
$(3)\Rightarrow (1)$:	
For a vertex $v$, let $\rank(v)$ be the smallest $i$ such that $v \in 
\PreOPPos^i(\vertices_\genColour)$ (if no such $i$ exists, then 
$\rank(v)=\infty$). For each $v$ with a positive rank there exists an action 
$a_v$ and vertex $u_v$ such that $\probTranFunc(u_v\mid v,a_v)>0$ and 
$\rank(u_v)< \rank(v)$. Consider any MD strategy $  \sigma $ with the following property: 
in each vertex of 
$\winPos(\mdp,\Reach(\genColour))\setminus \vertices_{\genColour}$, $ \sigma $ selects the 
action $a_v$ defined above with probability 1. A straightforward induction on the rank shows that such a $ \sigma $ is positively winning from each vertex of 
$\winPos(\mdp,\Reach(\genColour))$. This also proves the last part 
of the lemma.
\end{proof}

%\begin{algorithm}
%	\KwData{An MDP $ \mdp $}
%	\SetKwFunction{FTreat}{Treat}
%	\SetKwProg{Fn}{Function}{:}{}
%	
%	$W \leftarrow \vertices_\genColour$ \;	
%	\Repeat{$W' \neq W$}{
%	$W' \leftarrow W$\;
%	$W \leftarrow \PreOPPos(W)$
%	}
%	
%	
%	
%	\Return{$W$}
%	\caption{An algorithm computing $\winPos(\mdp,\Reach(\genColour))$}
%	\label{5-algo:reach-pos}
%\end{algorithm}

\noindent
%We also define multi-step iterates of the operators using the usual notation; e.g. $\PrePos^0(U)=\emptyset$ and $\PrePos^{i+1}(U)=\PrePos(\PrePos^i(U))$, and similarly for $\PreAS(U)$. 
As for complexity, we can focus on the problem of determining whether a given 
vertex belongs to $\winPos(\mdp,\Reach(\genColour))$. 

\begin{corollary}[Complexity of deciding positive reachability]
\label{5-cor:pos-complexity}
The problem of deciding whether a given vertex of a given MDP belongs to 
$\winPos(\mdp,\Reach(\genColour))$ is \NL-complete. Moreover, the set $\winPos(\mdp,\Reach(\genColour))$ can be computed in linear time.
\end{corollary}
\begin{proof}
\Cref{5-thm:positive-char} 
gives a blueprint for a logspace reduction from this problem to the 
\emph{s-t-connectivity} problem for directed graphs, and vice versa. The latter 
problem is well known to be \NL-complete~\cite{Savitch:1970}. Moreover, the set of states from which a target colour is reachable can be computed by a simple graph search (e.g. by BFS), hence in linear time.
\end{proof}
% TODO: if space permits,{} FNL membership of strategy synthesis problem. Should 
%be straightforward, via the ``certificate'' definition of FNL. Read the paper 
%referenced in complexity zoo.



%\begin{corollary}
%Memoryless deterministic strategies are sufficient for positive reachability 
%in MDPs. Moreover, the set $\winPos(\mdp,\Reach(\genColour))$, as well as an 
%MD 
%strategy that is positively winning from each vertex of 
%$\winPos(\mdp,\Reach(\genColour))$, can be both computed in polynomial time.
%\end{corollary}

% INTRODUCE NOTATION FOR A SET OF COLOURED VERTICES

\paragraph{Almost-sure reachability and safety} While the reachability and safety objectives are seemingly dual, in MDPs there is an intimate connection between them.
% We start by defining  
%
%\noindent
Let's start with almost-sure reachability. Consider the  \emph{almost-sure predecessor operator $\PreAS$}, s.t. for each $U \subseteq \vertices$ we have $$\PreAS(U) = \{v \in \vertices \mid \exists a \in \actions, \forall t \in 
\vertices: \probTranFunc(t\mid v,a)>0 \Rightarrow t \in U \}.$$
%one could be again tempted to compute the 
%winning 
%set $\winAS(\mdp,\Reach(\genColour))$ by iterating the operator $X \mapsto X 
%\cup \PreAS(X)$, starting with the initial argument $\vertices_\genColour$. 
%However, 
One might be tempted to mimic the positive reachability case and perform the iteration $ X \leftarrow X \cup \PreAS(X) $ on the set $ \vertices_{\genColour} $ until a fixed point is reached.
But this 
is not correct: consider an MDP with two vertices, $u,v$, the latter one being coloured by $\Win$. We have only one action $a$: in $v$, the action self loops on $v$, while in $u$ playing the action either switches moves us to $v$ or leaves us in $u$, both options having probability $\frac{1}{2}$. The probability that we \emph{never} reach $v$ from $u$ is 
equal to $\lim_{n\rightarrow \infty} \left(\frac{1}{2}\right)^n = 0$, and hence 
$\winAS(\mdp,\Reach(\Win))=\{u,v\}$. However, $\{v\}\cup\PreAS(\{v\})=\{v\}$, so 
$v$ is not included into the outcome of the iteration. Note that there indeed exists an infinite play which \emph{can} be generated by the strategy and 
which 
never visits $v$, but the probability of generating such a play is $0$. 

Instead, we make a detour via safety. Consider the one-step \emph{almost-sure safety} operator $\safeOP$ acting on sets of vertices:
%
\begin{align*}
\safeOP(X)= X \cap \PreAS(X).
\end{align*}
%
This operator gives rise to a notion of a closed set, which is important for the study of safety objectives in MDPs.

%Safety objectives in MDPs are related to the notion of a \emph{closed} set.

\begin{definition}[Closed set in an MDP]
\label{5-def:closed_set_MDP}
A set $X$ of vertices is ""closed"" if $ \safeOP(X)=X$. A ""sub-MDP"" of an MDP $ \mdp $ defined by a closed subset $X$ of $ \mdp $'s states is the MDP $\mdp_X = (X,\edges_X,\probTranFunc_X,\colouring_X)$ defined as follows:
\begin{itemize}
	\item $\edges_X$ is obtained from $\edges$ by removing edges incident to a vertex from $\vertices\setminus X$;
	\item $\probTranFunc_X$ is obtained from $\probTranFunc\subseteq \vertices\times\actions\times\dist(\edges)$ by removing all triples $(v,a,f)$ where either $v\not \in X$ or where the support of $f$ is not contained in $X$;
	\item $\colouring_X$ is a restriction of $\colouring$ to $X$.
\end{itemize}
We denote by $\closed(\mdp)$ the set of all closed sets in $\mdp.$
\end{definition}

Intuitively, a set is closed if Eve has a strategy ensuring that she stays in the set forever.



Now consider the iteration of $ X, \safeOP(X),\safeOP^2(X),\ldots $ of the safety operator. Clearly $\safeOP(X)\subseteq X$. Hence, the iteration reaches a fixed point in at most $|\vertices|$ steps. We get the following:

\begin{lemma}[Inclusion of the least fixed of the safety operator]
\label{5-lem:safety-iteration}
The fixed point of iterating $ \safeOP $ on some initial set $ X $ of $ \mdp $'s vertices is the largest (w.r.t. inclusion) closed set contained in $ X $. In particular, a vertex of $ X $ which does not belong to the fixpoint cannot belong to  $\winAS(\mdp,\Safe(\vertices \setminus X))$.
\end{lemma}
\begin{proof}
A straightforward induction on the number of iteration shows that if a vertex $ v $ is removed in an $ i $-th iteration, then no matter what strategy Eve uses, she has to reach $ \vertices\setminus X $ in at most $ i $ steps with a positive probability. The lemma follows.
\end{proof}


%To get the correct characterization,
%
% Intuitively, the fixed point of iterating $\safeOP$ on some initial argument $U$, which we denote by $\OPS$,\label{5-page:safetyop} is the largest subset $Y$ of $U$ for which there exists a strategy ensuring that $Y$ is never escaped (once it is entered).
%%is parametrized by the target colour $\genColour$ and 
%Now getting the almost-sure winning set for reachability entails repeated computation of positive winning sets on smaller and smaller sub-MDPs of $\mdp$.



\knowledge{closed}{notion,index={closed set of vertices in MDP}}
\knowledge{sub-MDP}{notion,index={sub-MDP}}

%\noindent
%To get the almost-surely winning set, we define the operator $\OPAS(c,\cdot)\colon \closed(\mdp) \rightarrow \closed(\mdp)$, parametrized by a colour $\genColour$, whose computation is shown in \Cref{5-algo:reach-opas}. Note that the algorithm terminates 



%\begin{itemize}
%\item We set $Y=\winPos(\mdp_X,\Reach(\genColour))$.
%\item We set $\OPAS(\genColour,X) = \OPS(Y)$, i.e. $\OPAS(\genColour,X)$ is the fixed point of the iteration: $Y,\safeOP(Y),\safeOP^2(Y),$ $\safeOP^3(Y),\ldots$. (This ensures that $\OPAS(c,X)$ is a closed set.)
%%	\item Compute the fixed point $Y$ of the iteration $X,\safeOP(X),\safeOP^2(X),\cdots$. 
%%%	CORRECT THIS, NOT THE CRUDE OPERATOR
%%%	\item Set $Z = Y \cup \vertices_{\genColour}$.
%%	\item Construct a sub-MDP $\mdp_Y$ of $\mdp$ by prohibiting all actions that transition from $Y$ to $\vertices\setminus Y$ with a positive probability. Formally, viewing $\probTranFunc$ as a subset of $\vertices\times\actions\times\dist(\edges)$, we remove from $\probTranFunc$ all triples $(v,a,f)$ such that $v\in Y$ and $\probTranFunc(v'\mid v,a)>0$ for some $v'\not \in Y$. In this way we obtain a transition function $\probTranFunc_Y$ of $\mdp_Y$. To obtain the set of edges of $\mdp_Y$, we remove from $\edges$ all edges that cannot be transitioned with positive probability under $\probTranFunc_Y$. The fact that $\PreAS(Y)=Y$ is crucial for the construction of $\mdp_Y$, since it guarantees that each vertex in $\mdp_Y$ has at least one enabled action.
%%	\item Finally, we let $\OPAS(X) = \winPos(\mdp_Y,\Reach(\genColour))$.
%\end{itemize}



\begin{algorithm}
	\KwData{An MDP $ \mdp $, a colour $\genColour$}
	\SetKwFunction{KwOPAS}{$\OPAS$}
	\SetKwProg{Fn}{Function}{:}{}
	
%	\Fn{\KwOPAS{\genColour,X}}{
		$Y \leftarrow \vertices \setminus\vertices_{\genColour}$ \;
		\Repeat{$Y' \neq Y$}{
			$Y' \leftarrow Y$\;
			$Y \leftarrow \safeOP(Y)$
		}
		\Return{$Y$} \tcp*{$Y$ is now a closed set}
%	}

\caption{An algorithm computing $\winAS(\mdp,\Safe(\genColour))$}
\label{5-algo:safety}
\end{algorithm}
%\noindent

\begin{theorem}[Complexity of the almost-sure safety winning set]
\label{5-thm:safety-main}
\Cref{5-algo:safety} computes the set $\winAS(\mdp,\Safe(\genColour))$ in strongly polynomial time. Moreover, there exists a memoryless deterministic strategy, computable in polynomial time, that is almost-surely winning from every vertex of $\winAS(\mdp,\Safe(\genColour))$.
\end{theorem}
\begin{proof}
The correctness follows immediately from \Cref{5-lem:safety-iteration}. The algorithm makes at most linearly many iterations, each of which has at most linear complexity. Hence, the complexity is at most quadratic. The strong polynomiality is testified by the fact that the algorithm only tests whether a 
probability of a given transition is positive or not, for which the exact 
values of positive probabilities are irrelevant. Clearly the output of the algorithm is a closed set, and hence for each $ v\in  \winAS(\mdp,\Safe(\genColour))$ there is an action $ a_v $ such that all edges in the support of $ \probTranFunc(v,a_v)  $ lead back to $ \winAS(\mdp,\Safe(\genColour)) $. Any MD strategy which in each $ v\in  \winAS(\mdp,\Safe(\genColour)) $ chooses $ a_v $ with probability 1 is a.s. winning inside $ \winAS(\mdp,\Safe(\genColour)) $, which proves the last part of the lemma.
\end{proof}



%It is easy to check that $\OPAS$ itself has the property that $\OPAS(X)\subseteq X$, and hence iterating it yields a fixed point in at most $|\vertices|$ steps.

\begin{algorithm}
	\KwData{An MDP $ \mdp $, a colour $ \genColour $}
\SetKwFunction{FTreat}{Treat}
$W \leftarrow \vertices$ \;	
\Repeat{$W' \neq W$}{
	$W' \leftarrow W$\;
	$Z \leftarrow \winPos(\mdp_W,\Reach(\genColour))   $} %\tcp*{\Cref{5-cor:pos-complexity}}
%	colour each vertex in $ W \setminus Z $ by $ \Lose $\;
	$W \leftarrow \winAS(\mdp_W,\Safe(W \setminus Z))$} %\tcp*{\Cref{5-algo:safety}}
}

\Return{$W$} \tcp*{A positive winning strategy in $\mdp_W$ is now almost-surely winning from $W$ in $\mdp$.}
\caption{An algorithm computing $\winAS(\mdp,\Reach(\genColour))$}
\label{5-algo:reach-as}
\end{algorithm}

With a.s. safety solved, we go back to a.s. reachability, which is solved via \Cref{5-algo:reach-as}. Note that in the first iteration, the algorithm computes the set %$ \winAS(\mdp,\Reach(\genColour)) \neq 
$\winAS(\mdp,\Safe(Z))$ where $ Z =  \winPos(\mdp,\Reach(\genColour))$. We might be tempted to think that this set already equals $ \winAS(\mdp,\Reach(\genColour)) $, but this is not the case. To see this, consider an MDP with three states $ u,v,t $ and two actions $ a,b $ such that $ t $ is coloured by $ \Win $, both actions self loop in $ v $ and $ t $, and $ \probTranFunc(t \mid u,a) = \probTranFunc(v \mid u,a) = \frac{1}{2} $ while $ \probTranFunc(u \mid u,b) = 1 $. Then $ \winAS(\mdp,\Reach(\Win)) = \{t\} $ while at the same time $ \winAS(\mdp,\Safe( \winPos(\mdp,\Reach(\Win)))) = \{u,t\}$. However, iterating the computation works, as shown in the following theorem.

\begin{theorem}[Algorithm for the almost-sure reachability winning set]
\label{5-thm:as-char}
\Cref{5-algo:reach-as} computes $\winAS(\mdp,\Reach(\genColour))$ in strongly polynomial time. Moreover, there is an MD strategy, computable in strongly polynomial time, that is almost-surely winning from every vertex of $\winAS(\mdp,\Reach(\genColour))$.
\end{theorem}
\begin{proof}
Since the set $ W $ can only decrease in each iteration, the algorithm terminates.
We prove that upon termination, $W$ equals $\winAS(\mdp,\Reach(\genColour))$.
	
We start with the $ \subseteq $ direction. We have $W \subseteq \winPos(\mdp_W,\Reach(\genColour))$. By \Cref{5-thm:positive-char} there exists an MD strategy $\sigma$ in $\mdp_W$ which is positively winning from each vertex of $W$. We show that the same strategy is also almost-surely winning from each vertex of $W$ in $\mdp_W$ and thus also from each vertex of $W$ in $\mdp$, which also proves the second part of the theorem. 
%Since $\mdp_W$ is a sub-MDP of $\mdp$ obtained by removing some vertices and transitions, the strategy $\sigma$ can be also regarded as a strategy in $\mdp$ (the choices in the missing vertices being defined arbitrarily) and it is then easy to see that it is also almost-surely winning in $\mdp,$ from each state of $X$. Together with the ``only if'' direction, this will also prove that MD strategies are sufficient for almost-sure reachability. 
Let $v$ be any vertex of $W$ and denote $|W|$ by $\ell$. Since $\sigma$ is memoryless, it guarantees that a vertex of $\vertices_{\genColour}$ is reached with a positive probability in at most $\ell$ steps (see also the construction of $ \sigma $ in the proof of \Cref{5-thm:positive-char}), and since it is also deterministic, it guarantees that the probability $p$ of reaching $\vertices_{\genColour}$ in at most $\ell$ steps is at least $p_{\min}^{\ell}$, where  $p_{\min}$ is the smallest non-zero edge probability in $\mdp_W$. Now imagine that $\ell$ steps have elapsed and we have not yet reached $\vertices_{\genColour}$. This happens with a probability at most $(1-p_{\min}^\ell)$. However, even after these $\ell$ steps we are still in $W$, since $ \sigma  $ is a strategy in $ \mdp_w $. Hence, the probability that we do not reach $\vertices_\genColour$ within the first $2\ell$ steps is bounded by $(1-p_{\min}^\ell)^{2}$. To realize why this is the case, note that any finite play $\play$ of length $2\ell$ can be split into two halves, $\play',\play''$ of length $\ell$, and then $\probm^{\sigma}_{v}(\cylinder(\play))=\probm^{\sigma}_{v}(\cylinder(\play'))\cdot\probm^{\sigma}_{\last(\play')}(\cylinder(\play''))$ (here we use the fact that $\sigma$ is memoryless). Using this and some arithmetic, one can show that, denoting $\mathit{Avoid}_i$ the set of all plays that avoid the vertices of $\vertices_{\genColour}$ in steps $\ell\cdot(i-1)$ to $\ell\cdot(i)-1$, it holds $$\probm^{\sigma}_{v}(\mathit{Avoid}_1\cap \mathit{Avoid}_2) \leq \probm^{\sigma}_{v}(\mathit{Avoid}_1)\cdot \max_{u\in W\setminus \vertices_{\genColour}}\probm^{\sigma}_{u}(\mathit{Avoid}_1)\leq (1-p_{\min}^\ell)^{2}.$$

\noindent
One can then continue by induction to show that $\probm^{\sigma}_{v}(\bigcap_{i=1}^j \mathit{Avoid}_i)\leq (1-p_{\min}^\ell)^{j},$ and hence
$$\probm^\sigma_{v}(\Reach(\genColour))= 1-\probm^{\sigma}_{v}(\bigcap_{i=1}^\infty \mathit{Avoid}_i) \leq 1-\lim_{j\rightarrow \infty}(1-p_{\min}^\ell)^{j}= 1-0=1.$$

Now we prove the $ \supseteq $ direction. Denote $X=\winAS(\mdp,\Reach(\genColour))$. We prove that $ W \supseteq X $ is an invariant of the iteration. Initially this is clear. Now assume that this holds before an iteration takes place. It is easy to check that $X$ is closed, so $\mdp_{X}$ is well-defined. We prove that $ X \subseteq \winAS(\mdp_W,\Safe(W\setminus Z)) $, where $ Z $ is defined during the iteration. A strategy in $\mdp$ that reaches $\vertices_{\genColour}$ with probability 1 must never visit a vertex from $\vertices\setminus X$ with a positive probability. Hence, each such strategy can be viewed also as a strategy in $\mdp_{X}$. It follows that  $X=\winAS(\mdp_X,\Reach(\genColour)) = \winPos(\mdp_{X},\Reach(\genColour)) \subseteq \winPos(\mdp_{W},\Reach(\genColour)) = Z$, the middle inclusion following from induction hypothesis. Now by \Cref{5-lem:safety-iteration} and \Cref{5-thm:safety-main}, the set $ \winAS(\mdp_W,\Safe(W\setminus Z)) $ is the largest closed set contained in $ Z $. But $ X $ is also closed, and ashown above, it is contained in $ Z $. Hence,  $ X \subseteq \winAS(\mdp_W,\Safe(W \setminus Z)) $.

The complexity follows form \Cref{5-cor:pos-complexity} and \Cref{5-thm:safety-main}; and also from the fact that the main loop must terminate in $ \leq |\vertices| $ steps. The strong polynomiality again follows from the algorithm being oblivious to precise probabilities.
\end{proof}

%Note that in the first part of the above proof we proved the following useful claim:
%\begin{claim}
%\label{5-clm:asclaim}
%If there is an MD strategy positively winning for $ \Reach(\genColour) $ in every vertex of the MDP, then the same strategy a.s. winning for $ \Reach(\genColour) $ from every vertex.
%\end{claim}

%\Cref{5-thm:as-char} can be easily distilled into an algorithm for computing the set  $\winAS(\mdp,\Reach(\genColour))$: simply iterate $\OPAS(\genColour,\cdot)$ on $\vertices$ until a fixed point is reached. It is easy to check that each of the iterations can be performed in time polynomial in $\mdp$ (for the computation of $Y=\winPos(\mdp_X,\Reach(\genColour))$, this follows from \Cref{5-cor:pos-complexity}). Hence, computing the fixed point of $\OPAS(\vertices,\cdot)$, i.e. the set $W=\winAS(\mdp,\Reach(\genColour))$ can be also performed in polynomial time. Once $W$  is computed, we compute an MD strategy $\sigma$ which is positively winning for $\Reach(\genColour)$ in $\mdp_W$. As shown in the proof of \Cref{5-thm:as-char}, $\sigma$ can be seen as a strategy in $\mdp$ which is almost-surely winning in each state of $W$.

%\noindent
We also have a complementary hardness result. 
% In total, we have the 
%following:

\begin{theorem}[Complexity of the almost-sure reachability winning set]
\label{5-thm:as-complexity}
The problem of determining whether a given vertex of a given MDP belongs to 
$\winAS(\mdp,\Reach(\genColour))$ is \P-complete.
\end{theorem}
\begin{proof}
	We procced by a reduction 
	from the \emph{circuit value problem (CVP)}.
	An instance of \emph{CVP} is a directed acyclic graph $\mathcal{C}$, 
	representing a boolean circuit: each internal node represents either an OR gate 
	or an AND gate, while each leaf node is labelled by \emph{true} or 
	\emph{false}. Each internal node is guaranteed to have exactly two children. 
	Each node of $\mathcal{C}$ evaluates to a unique truth value: the value of a 
	leaf is given by its label and the value of an internal node $v$ is given by 
	applying the logical operator corresponding to the node to the truth values of 
	the two children of $v$, the evaluation proceeding in a backward topological order. The task is to decide whether a given node $w$ of 
	$\mathcal{C}$ evaluates to \emph{true}. \emph{CVP} was shown to be 
	\P-hard (under logspace reductions) in~\cite{Ladner:1975}. 
	In~\cite{Chatterjee&Doyen&Henzinger:2010}, the following logspace reduction 
	from CVP to 
	almost-sure reachability in MDPs is presented: given a boolean circuit 
	$\mathcal{C}$, construct an MDP $\mdp_{\mathcal{C}}$ whose vertices correspond 
	to the gates 
	of $\mathcal{C}$. There are two actions, call them $\mathit{left}$ and $\mathit{right}$. In each vertex corresponding to an OR gate $g$, the 
	$\mathit{left}$ action transitions with probability 1 to the vertex 
	representing the left child of $g$, and similarly for the action 
	$\mathit{right}$ 
	and the right child. In a vertex corresponding to an AND gate $g$, both actions behave the same: the transition into each of the two children 
	of $g$ with probability $\frac{1}{2}$. Vertices corresponding to leafs have self loop as the only outgoing edges, and 
	moreover, they are coloured with the respective labels in $\mathcal{C}$. It is 
	easy to check that a gate of $\mathcal{C}$ evaluates to $\mathit{true}$ if and 
	only if the corresponding vertex belongs to 
	$\winAS(\mdp_{\mathcal{C}},\Reach(\mathit{true}))$.
\end{proof}

\paragraph{Positive safety} We conclude this section by a discussion of positive safety. 
%This objective is somewhat less prominent in the MDP literature, and hence (and due to space constraints) we omit full proof of the main theorem: the proof rests on simple extension of ideas from the previous cases and we invite the reader to reconstruct the argument as an exercise. Alternatively, positive safety can be seen as a special case of the optimal parity problem, for which a polynomial algorithm and a proof of memoryless optimality are provided in the latter parts of this chapter.


\begin{theorem}[Algorithm for the positive safety winning set]
\label{5-thm:pos-safety-main}
Let $ \mdp_{\bar\genColour} $ be an MDP obtained from $ \mdp $ by changing all $ \genColour $-coloured vertices to sinks (i.e. all actions in these vertices just self loop on the vertex). Then
$ \winPos(\mdp,\Safe(\genColour)) = \winPos(\mdp_{\bar\genColour},\Reach(\winAS(\mdp_{\bar\genColour},\Safe(\genColour))) ) $. In particular, the set $ \winPos(\mdp,\Safe(\genColour)) $ can be computed in a strongly polynomial time and there exists a memoryless deterministic strategy, computable in strongly polynomial time, that is positively winning from every vertex of $\winPos(\mdp,\Safe(\genColour))$.
\end{theorem}
\begin{proof}
Clearly $ \winPos(\mdp,\Safe(\genColour)) = \winPos(\mdp_{\bar{\genColour}},\Safe(\genColour)) $ and $ \winAS(\mdp,\Safe(\genColour)) = \winAS(\mdp_{\bar{\genColour}},\Safe(\genColour)) $; and moreover the corresponding winning strategies easily transfer between the two MDPs (for a safety objective, the behaviour after visiting a $ \genColour $-coloured state is inconsequential).
Hence, putting $ Z = \winAS(\mdp_{\bar{\genColour}},\Safe(\genColour)) $, it is sufficient to show that  $ \winPos(\mdp_{\bar{\genColour}},\Safe(\genColour)) =\winPos(\mdp_{\bar\genColour},\Reach(Z))  $  

The $ \supseteq $  inclusion is clear as well as the construction of the witnessing MD strategy (in the vertices of that are outside of $ Z $, we behave as the positively winning MD strategy for reaching $ Z $, while inside $ Z $ we behave as the a.s. winning strategy for $ \Safe(\genColour) $). 

For the other inclusion, let $ X = \vertices \setminus \winPos(\mdp_{\bar{\genColour}},\Reach(Z)) $. We prove that $ X\subseteq \vertices \setminus \winPos(\mdp_{\bar{\genColour}},\Safe(\genColour)) $. Assign ranks to vertices inductively as follows: each vertex coloured by $ \genColour $ gets rank $ 0 $. Now if ranks $ \leq i $ have been already assigned, then a vertex  $ v $ is assigned rank $i+1  $ if it  does  not already have a lower rank but for all actions $ a\in\actions $ there exists a vertex $ u $ of rank $ \leq i $ s.t. $ \probTranFunc(u \mid v,a) >0$. Then each vertex in $ X $ is assigned a finite rank: indeed, the set of vertices without a rank is closed and does not contain $ \genColour $-coloured vertices, hence it is contained in $ Z $. 
Now fix any strategy $ \sigma $ starting in a vertex $v \in X $. By definition of $ X $, $ \sigma $ never reaches $ Z $ and hence never visits an unranked state. At the same time, whenever $ \sigma $ is in a ranked state, there is, by definition of ranks, a probability at least $p_{\min}  $ (the minimal edge probability in $ \mdp $) of transitioning to a lower-ranked state in the next step. Hence, in every moment, the probability of $ \sigma $ reaching a $ \genColour $-coloured state within the next $ |\vertices| $ steps is at least $ p_{\min}^{|\vertices|} $. By a straightforward adaptation of the second part of the proof of \Cref{5-thm:as-char}, $ \sigma $ eventually visits $ \vertices_{\genColour} $ with probability 1. Since $ \sigma $ was arbitrary, this shows that $ v\not \in\winPos(\mdp_{\bar{\genColour}},\Safe(\genColour)).  $

The complexity follows from the results on positive reachability and a.s. safety.
\end{proof}

%As we shall see, the \P-completeness is retained for many other MDP problems 
%studied in this chapter, and in particular, the almost-sure reachability 
%problem can be reduced to the problem of determining optimal 
%values in mean-payoff MDPs, for which we give a polynomial-time algorithm 
%in~\Cref{xxx}. This would be another way of 
%establishing~\Cref{5-thm:as-complexity}. However, the algorithm we presented in 
%this sub-section has a certain special property: it runs in \emph{strongly 
%polynomial time}. 
%Indeed the algorithm  only tests whether a 
%probability of a given transition is positive or not, for which the exact 
%values of positive probabilities are irrelevant.
%
%\begin{theorem}
%\label{5-thm:qual-reach-sp}
%The sets $\winPos(\mdp,\Reach(\genColour))$ and  
%$\winAS(\mdp,\Reach(\genColour))$, as well as the corresponding winning 
%strategies, can be computed in strongly polynomial time.
%\end{theorem}


\section{Concurrent mean-payoff games}
\label{7-sec:mean_payoff}
In this section we consider concurrent mean-payoff games. 
We will show that in general, any  $\epsilon$-optimal strategy in some concurrent mean-payoff games are quite complex. 
We will first, however, show that finding the value of a concurrent mean-payoff game can be done in polynomial space.

\begin{lemma}\label{lemm:class_meanpayoff}
Concurrent mean-payoff games are determined and the value is the limit of the value of the corresponding time-limited game as well as the limit of the corresponding discounted game, for the discount factor going to 0 from above.
There is an polynomial time algorithm, ala Lemma~\ref{lem:val1}, for finding the set of vertices where a finite memory strategy suffice to ensure $1-\epsilon$ (recall that all rewards are in $\{0,1\}$).
For any fixed number $n$, there is a polynomial time algorithm for approximating the value in a concurrent mean-payoff game with $n$ vertices (i.e. the running time is polynomial in the number of actions)
\end{lemma}
We will not show this lemma, but simply note that the $\epsilon$-optimal strategies known for general concurrent mean-payoff games  can be viewed as playing the corresponding discounted game with a variable discount factor that depends on how ``nice'' the rewards has been up to now. Basically, in each round you play the optimal strategy in the corresponding discounted game with a discount factor $\gamma$. Whenever 
 your rewards are close to or better than the value, you decrease $\gamma$ towards 0 and in each round your rewards are much worse than the value you let $\gamma$ increase, except not bigger than the initial $\gamma$ in the first round. Much of this section will argue that many natural candidates for simpler types of strategies does not work.


We will show that approximating the value, however, can, as mentioned, be done in polynomial space. The proof relies on Proposition~22 from~\cite{HKLMT:2011}, stating the following:
\begin{proposition}
Let $\epsilon=2^{-j}$, where $j$ is some positive integer, and the probabilities be rational numbers where the nominator and denominator have bitsize at most $\tau$. Also, let $\lambda=\epsilon^{\tau m^{O(n^2)}}$. Consider some state $s$ and let the value of that state in the $\lambda$ discounted game be $v_{\lambda}$ and the value in mean-payoff game be $v$, then $|v-v_{\lambda}|<\epsilon$.
\end{proposition}

We will use that to again reduce to the existential theory over the reals. 
For a fixed discount factor $\gamma$, we can easily express the value of the corresponding discounted game, like we expressed the value of a concurrent reachability game.
We have that the value $v$ is then $v=\lim_{\gamma\rightarrow 0^+} f(\gamma)$, where $f$ is the found expression.
I.e. for any $\epsilon$, there is a $\gamma'$ such that for all $\gamma<\gamma$, we have that $|f(\gamma)-v|\leq \epsilon$.
Also, that $v>c$ means that there is $\epsilon$, such that $v-\epsilon>c$.

The problem is thus to come up with a polynomial sized formula to express that $\lambda$ is $\epsilon^{\tau m^{O(n^2)}}=2^{-j \tau m^{O(n^2)}}$.

That can be done as follows, using $\ell=O(n^2)\cdot \log(m)+\log(j\tau)$ many variables, $v_0,v_1,\dots v_{\ell-1}$:
\[
v_0=1/2
\]
and for all $0<i< \ell$, we have that
\[
v_i=v_{i-1}\cdot v_{i-1}.
\]
Using induction, we see that $v_i=2^{-2^{i}}$, i.e., $v_1=1/2=2^{-2^0}$ and \[
v_i=v_{i-1}\cdot v_{i-1}=2^{-2^{i-1}}\cdot 2^{-2^{i-1}}=2^{-2^{i-1}-2^{i-1}}=2^{-2^{i}}\]
In particular, \[
v_{\ell-1}=2^{-2^{\ell}}=2^{-2^{O(n^2)\cdot \log(m)+\log(j\tau)}}=2^{-j\tau m^{O(n^2)}}
\] is the value we wanted for $\lambda$.
Thus, for a given number $v$, we can test if the value of a concurrent  $\lambda$-discounted game is above $v+\epsilon$, which, using the proposition above, implies that $v$ is below the value of the corresponding concurrent mean-payoff game. On the other hand, the proposition also implies that if the value of the concurrent  $\lambda$-discounted game is below $v-\epsilon$, then the value of the concurrent mean-payoff game is below $v$. Being able to answer these questions lets you easily approximate the value of a concurrent mean-payoff using binary search. 

We get the following lemma.
\begin{lemma}
Approximating the value of a concurrent mean-payoff game can be in done in polynomial space
\end{lemma}



We will now consider a specific, well-studied example of a concurrent mean-payoff game, since it shows that many natural kinds of strategies do not suffice in general.
The game is called the big match and is defined as follows:
There are 3 vertices, $\{0,s,1\}$, where the vertices in $\{0,1\}$ are absorbing, and with value equal to their name.
The last vertex $s$ has a 2x2-matrix and for all $i,j$ for $i\neq j$, we have that 
$c(s,1,1)=1$, and for $i\neq 1\neq j$ we have that $c(s,1,1)=0$.
Also,  $\dest(s,1,i)=s$ for each $i$, $\dest(s,2,1)=0$ and $\dest(s,2,2)=1$. There is an illustration in Figure \ref{7-fig:bm}.
The value of the Big Match is $1/2$.

\begin{figure}

\center
\begin{tikzpicture}[node distance=3cm,-{stealth},shorten >=2pt]
\ma{s}[$s:$]{2}{2};

\node at (s-1-1.center) {$1$};
\node at (s-2-1.center) {$0^*$};
\node at (s-1-2.center) {$0$};
\node at (s-2-2.center) {$1^*$};


\end{tikzpicture}
\caption{The Big Match}\label{7-fig:bm}
\end{figure}

Consider a finite-memory strategy $\sigma$ for Eve. We will argue that $\sigma$ cannot guarantee $\epsilon$ (any strategy can guarantee $-1$, since the colors are between $0$ and $1$) for any $0<\epsilon$. Let $\tau$ be the stationary strategy for Adam that plays $1$ with pr. $\epsilon/2$.
Then playing $\sigma$ against $\tau$, we get an Markov chain, where the vertex space is pairs of memory states and game vertices. 
In Markov chains, eventually, with pr. 1, a set of vertices $S$ is reached such that the set of vertices visited infinitely often is $S$. Such a set is called ergodic.
The set $S$ can clearly only contain 1 game vertex, since whenever $s$ is left, it is never entered again.
Hence, if $S$ contains $s$, the pr. that play will ever reach $\{0,1\}$ is 0.
In the MC we get from the players playing $\sigma$ and $\tau$, let $T_{\epsilon/2}$ be such that with pr. $\epsilon/2$ some ergodic set has been reached. 
Let $\tau'$ be the strategy that plays $\tau$ for $T_{\epsilon/2}$ and afterwards plays $2$. 

When $\sigma$ is played against $\tau'$, either we reach $\{0,1\}$ and Adam plays 1 only finitely many times, while in $s$ (the latter because there are only finitely many numbers below $T_{\epsilon/2}$). Thus, for Eve to win a play, the play needs to reach vertex 1. There are two ways to do so, either Eve stops before $T_{\epsilon/2}$ or after. In the former case, the pr. to reach $1$ is only $\epsilon/2$ (because Adam needs to play $2$ at the time, which is only done with pr. $\epsilon/2$). The latter only happens with pr. $\epsilon/2$ by definition of $T_{\epsilon/2}$ (because, Adam could play $2$ for an arbitrary number of steps while following $\tau$ but $s$ would not be left anyway).

We get the following lemma.

\begin{lemma}\label{lemm:no_finite_meanpayoff}
No finite memory strategy can guarantee more than $0$ in the Big Match.
\end{lemma}

The principle of sunken cost states that, when acting rationally, one should disregard cost already paid. We will next argue that this does not apply (naively) to the Big Match.
A strategy following the principle of sunken cost would not depend on past cost paid and thus, in each step $T$, there is a pr. $p_T$ of stopping for Eve.
Such strategies are called Markov strategies in the Big Match.
Fix some Markov strategy $\sigma$ for Eve. We will argue, like before, that $\sigma$ cannot guarantee more than $\epsilon$ for any $\epsilon>0$.
Note that Eve does not depend on the choices of Adam and thus, either she stops with pr. 1 or she does not.
In the former case, Adam just plays $1$ forever. When Eve stops, the vertex reached is thus $-1$.
Alternately, if Eve does not stop with pr. 1, there must be a time $T$, such that she only stops with pr. $\epsilon$ after $T$ (this is actually also the case even if she stops with pr. 1). 
Adam's strategy is then to play $1$ for $T$ steps and $2$ thereafter. Observe that the pr. to reach $1$ is thus at most $\epsilon$, in that it must be that Eve stops after $T$. If she does not stop (or stops in $0$), there will be only finitely many 1s.

We see the following:
\begin{lemma}\label{lemm:no_markov_meanpayoff}
No Markov strategy can guarantee more than $0$ in the Big Match
\end{lemma}


\section*{Bibilographic references}
\label{7-sec:references}
We refer to~\cref{2-sec:references} for the role of parity objectives and how they emerged in automata theory as a subclass of Muller objectives.
Another related motivation comes from the works of Emerson, Jutla, and Sistla~\cite{Emerson&Jutla&Sistla:1993},
who showed that solving parity games is linear-time equivalent to the model-checking problem for modal $\mu$-calculus.
This logical formalism is an established tool in program verification, and a common denominator to a wide range of modal, temporal and fixpoint logics used in various fields.

\vskip1em
Let us discuss the progress obtained over the years for each of the three families of algorithms.

\vskip1em
\textit{Value iteration algorithms and separating automata}.
The heart of value iteration algorithms is the value function, which in the context of parity games and related developments for automata
have been studied under the name progress measures or signatures.
They appear naturally in the context of fixed point computations so it is hard to determine who first introduced them.
Streett and Emerson~\cite{Streett&Emerson:1984,Streett&Emerson:1989} defined signatures for the study of the modal $\mu$-calculus,
and Stirling and Walker~\cite{Stirling&Walker:1989} later developped the notion.
Both the proofs of Emerson and Jutla~\cite{Emerson&Jutla:1991} and of Walukiewicz~\cite{Walukiewicz:1996} use signatures to show the positionality of parity games over infinite games.

Jurdzi{\'n}ski~\cite{Jurdzinski:2000} used this notion to give the first value iteration algorithm for parity games, 
with running time $O(m n^{d/2})$.
The algorithm is called ``small progress measures'' and is an instance of the class of value iteration algorithms we construct 
in~\cref{3-sec:value_iteration} by considering the universal tree of size $n^h$.
Bernet, Janin, and Walukiewicz~\cite{Bernet&Janin&Walukiewicz:2002} investigated reductions from parity games to safety games
through the notion of permissive strategies, and constructed a separating automaton\footnote{We note that the general framework of separating automata came later, introduced by Boja{\'n}czyk and Czerwi{\'n}ski~\cite{Bojanczyk&Czerwinski:2018}.} corresponding to the universal tree of size $n^h$.

The new era for parity games started in 2017 when Calude, Jain, Khoussainov, Li, and Stephan~\cite{Calude&Jain&al:2017} constructed a quasipolynomial time algorithm. 
Our presentation follows the technical developments of the subsequent paper by Fearnley, Jain, Schewe, Stephan, and Wojtczak~\cite{Fearnley&Jain&al:2017} which recasts the algorithm as a value iteration algorithm.
Boja{\'n}czyk and Czerwi{\'n}ski~\cite{Bojanczyk&Czerwinski:2018} introduce the separation framework to better understand the original algorithm.

Soon after two other quasipolynomial time algorithms emerged.
Jurdzi{\'n}ski and Lazi{\'c}~\cite{Jurdzinski&Lazic:2017} showed that the small progress measure algorithm can be adapted to a ``succinct progress measure'' algorithm, matching (and slightly improving) the quasipolynomial time complexity.
The presentation using universal tree that we follow in~\cref{3-sec:value_iteration} and an almost matching lower bound on their sizes is due to Fijalkow~\cite{Fijalkow:2018}.
The connection between separating automata and universal trees was shown by Czerwi{\'n}ski, Daviaud, Fijalkow, Jurdzi{\'n}ski, Lazi{\'c}, and Parys~\cite{Czerwinski&Daviaud&al:2018}. 

The third quasipolynomial time algorithm is due to Lehtinen~\cite{Lehtinen:2018}.
The original algorithm has a slightly worse complexity ($n^{O(\log(n))}$ instead of $n^{O(\log(d))}$),
but Parys~\cite{Parys:2020} later improved the construction to (essentially) match the complexity of the previous two algorithms.
Although not explicitly, the algorithm constructs an automaton with similar properties as a separating automaton,
but the automaton is non-deterministic.
Colcombet and Fijalkow~\cite{Colcombet&Fijalkow:2019} revisited the link between separating automata and universal trees
and proposed the notion of good for small games automata, capturing the automaton defined by Lehtinen's algorithm.
The equivalence result between separating automata, good for small games automata, and universal graphs, holds for any positionally determined objective, giving a strong theoretical foundation for the family of value iteration algorithms.

\vskip1em
\textit{Attractor decomposition algorithms}.
The McNaughton Zielonka's algorithm has complexity $O(m n^d)$.
Parys~\cite{Parys:2019} constructed the fourth quasipolynomial time algorithm as an improved take over McNaughton Zielonka's algorithm.
As for Lehtinen's algorithm, the original algorithm has a slightly worse complexity ($n^{O(\log(n))}$ instead of $n^{O(\log(d))}$).
Lehtinen, Schewe, and Wojtczak~\cite{Lehtinen&Schewe&Wojtczak:2019} later improved the construction.
As discussed in~\cref{3-sec:relationships} the complexity of this algorithm is quasipolynomial and of the form $n^{O(\log(d))}$,
but a bit worse than the three previous algorithms since the algorithm is symmetric and has a recursion depth of $d$,
while the value iteration algorithms only consider odd priorities hence replace $d$ by $d/2$.

Jurdzi{\'n}ski and Morvan~\cite{Jurdzinksi&Morvan:2020} constructed a generic McNaughton Zielonka's algorithm parameterised by the choice of two universal trees, one for each player.
\mynote{CONTINUE}


\vskip1em
\textit{Strategy improvement algorithms}.
As we will see in~\cref{4-chap:payoff}, parity games can be reduced to mean payoff games,
so any algorithm for solving mean payoff games can be used for solving parity games.
In particular, the existing strategy improvement algorithm for mean payoff games can be run on parity games. 
V{\"o}ge and Jurdzin{\'n}ski~\cite{Voge&Jurdzinski:2000} introduced the first discrete strategy improvement for parity games,
running in exponential time.
For some time there was some hope that the strategy improvement algorithm, for some well chosen policy on switching edges,
solves parity games in polynomial time.
Friedmann~\cite{Friedmann:2011} cast some serious doubts by constructing numerous exponential lower bounds applying to different variants of the algorithm.
Fearnley~\cite{Fearnley:2017} investigated efficient implementations of the algorithm, focussing on the cost of computing and updating the value function for a given strategy.
Our proof of correctness is original. \mynote{SAY MORE?}

The complexity was reduced to subexponential with randomised algorithms 
by Jurdzin{\'n}ski, Paterson, and Zwick~\cite{Jurdzinski&Paterson&Zwick:2008}.
A natural question is whether there exists a quasipolynomial strategy improvement algorithm; 
as discussed in~\cref{3-sec:relationships} the notion of universal trees cannot be used to achieve this,
and the question remains to this day open.









