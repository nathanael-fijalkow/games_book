\usepackage{xspace}

\newcommand{\Win}{\textsc{Win}\xspace} 
\newcommand{\Lose}{\textsc{Lose}\xspace} 

%%%%%%%%%%%%%%%%%%%%%% Colors
%% all colors
\colorlet{grey-10}{black!10!white}
\colorlet{grey-20}{black!20!white}
\colorlet{grey-30}{black!30!white}
\colorlet{grey-40}{black!40!white}
\colorlet{grey-50}{black!50!white}
\colorlet{grey-60}{black!60!white}

\colorlet{lgrey-back}{black!10!white}
\colorlet{lgrey-border}{black!40!white}
\colorlet{dgrey-back}{black!30!white}
\colorlet{dgrey-border}{black!80!white}

%% default
\colorlet{state-back}{lgrey-back}
\colorlet{state-border}{lgrey-border}

%%%%%%%%%%%%%%%%%%%%%% Tikzstyles
%% background+border
\tikzstyle{grey}=[fill=lgrey-back,draw=lgrey-border]
\tikzstyle{lgrey}=[grey]
\tikzstyle{dgrey}=[fill=dgrey-back,draw=dgrey-border]
\tikzstyle{white}=[fill=white,draw=black]
\tikzstyle{black}=[fill=black,draw=black]

%% states
\tikzstyle{@state}=[fill=state-back,draw=state-border,inner sep=0pt,line width=.8pt]
\tikzstyle{square16}=[@state,rectangle,minimum height=16mm,minimum width=16mm]
\tikzstyle{square10}=[@state,rectangle,minimum height=10mm,minimum width=10mm]
\tikzstyle{square5}=[@state,rectangle,minimum height=5mm,minimum width=5mm]
\tikzstyle{square4}=[@state,rectangle,minimum height=4mm,minimum width=4mm]
\tikzstyle{circle6}=[@state,circle,minimum size=6mm]
\tikzstyle{circle4}=[@state,circle,minimum size=4.3mm]
\tikzstyle{diamond7}=[@state,diamond,minimum height=7.5mm,minimum width=7.5mm]
\tikzstyle{diamond5}=[@state,diamond,minimum height=5mm,minimum width=5mm]
\tikzstyle{state}=[s-eve]
\tikzstyle{s-eve}=[circle6]
\tikzstyle{s-adam}=[square5]
\tikzstyle{s-random}=[diamond7]
\tikzstyle{s-eve-small}=[circle4]
\tikzstyle{s-adam-small}=[square4]
\tikzstyle{s-random-small}=[diamond5]
\tikzset{node distance=2.5cm}
\tikzset{every node/.style={anchor=base}}

%% edges
\tikzset{>=latex,bend angle=20}
\tikzstyle{line}=[line width=.6pt]
\tikzstyle{arrow}=[->,line]
\tikzstyle{dblarrow}=[<->,line]
\tikzstyle{invarrow}=[<-,line]
\tikzstyle{initial}=[invarrow]
\tikzset{selfloop/.style={arrow,out={#1-30},in={#1+30},looseness=6}}
    % we use tikzset here because tikzstyle does not allow arguments
    

%% diagrams
\tikzstyle{fillarea}=[line width=.6pt,line join=round]
% define new hatch pattern
\newlength{\hatchspread}
\newlength{\hatchthickness}
\tikzset{hatchspread/.code={\setlength{\hatchspread}{#1}},
  hatchthickness/.code={\setlength{\hatchthickness}{#1}}}
\tikzset{hatchspread=3pt,hatchthickness=0.4pt}
\pgfdeclarepatternformonly[\hatchspread]% variables
   {custom north west lines}% name
   {\pgfqpoint{\dimexpr-2\hatchthickness}{\dimexpr-2\hatchthickness}}% lower left corner
   {\pgfqpoint{\dimexpr\hatchspread+2\hatchthickness}{2\dimexpr\hatchspread+2\hatchthickness}}% upper right corner
   {\pgfqpoint{\dimexpr\hatchspread}{2\dimexpr\hatchspread}}% tile size
   {% shape description
    \pgfsetlinewidth{\hatchthickness}
    \pgfpathmoveto{\pgfqpoint{\dimexpr-\hatchthickness}{2\dimexpr\hatchspread+\hatchthickness}}
    \pgfpathlineto{\pgfqpoint{\dimexpr\hatchspread+\hatchthickness}{\dimexpr-\hatchthickness}}
    \pgfusepath{stroke}
   }
\tikzstyle{NWlinesarea}=[line width=.6pt,pattern=custom north west lines, hatchspread=4pt]
\tikzstyle{hatcharea}=[NWlinesarea]


%% \newdimen{\calc}
%% \pgfdeclarepatternformonly[/tikz/angle,/tikz/dist]{hlines}
%%      %% angle should be in [-45;45]
%%   {\pgfpointorigin}
%%   {\setlength\calc=10pt\relax
%%   \pgfqpoint{\calc}{\pgfkeysvalueof{/tikz/dist}/cos(\pgfkeysvalueof{/tikz/angle})}}
%%   {\pgfqpoint{\pgfkeysvalueof{/tikz/dist}/sin(\pgfkeysvalueof{/tikz/angle})}{\pgfkeysvalueof{/tikz/dist}/cos(\pgfkeysvalueof{/tikz/angle})}}
%%   {\pgfsetlinewidth{.4pt}
%%    \pgfpathmoveto{\pgfpointorigin}
%%    \pgfpathlineto{\pgfqpoint{\pgfkeysvalueof{/tikz/dist}/sin(\pgfkeysvalueof{/tikz/angle})}{\pgfkeysvalueof{/tikz/dist}/cos(\pgfkeysvalueof{/tikz/angle})}}
%%    \pgfusepath{stroke}}

%% \tikzset{%
%%   angle/.initial=20,
%%   dist/.initial=4pt}

%% \tikzstyle{hatcharea}=[draw,line width=.6pt,pattern=hlines,angle=20,dist=5pt]

%\tikzstyle{arrow}=[-latex']
%\tikzstyle{init}=[arrow-]


%% Commands for drawing matrices for concurrent games
\newcommand{\niceloop}[1]{ %makes a nice loop up and round to the right for matrices
\draw[-,shorten >=0,solid](#1.center) -- (#1.north) to[out=90,in=180] ($(#1.north east)+(0,0.5cm)$); 
\draw[solid] ($(#1.north east)+(0,0.5cm)$) to[out=0,in=0,min distance=0.7cm]  (#1);
}

\newcommand{\sco}{} %used for internal parameters
% first optional parameter giving position, then name of matrix, then optional parameter for shown name (if it should be different from name in code), then 2 parameters for size. i.e. \ma[xshift=-1cm]{s}[$t:$]{2}{3} gives a matrix 1cm to the left, with internal name s but labled t, with 2 rows and 3 columns.
\newcommand{\ma}[1][xshift=0]{ 
\renewcommand{\sco}{#1}
\maA
}
\newcommand{\name}{}
\newcommand{\maA}[1]{
\renewcommand{\name}{#1}
\maB
}

\newcommand{\maB}[3][-NoValue-]{
\expandafter\maC\expandafter{\sco}{\name}{#1}{#2}{#3}
}
\newcommand{\maC}[5]{
\begin{scope}[#1,solid,-,shorten >=0,shorten <=0]
\expandafter\draw (0,0) node[rectangle, minimum height=#4 cm,minimum width=#5 cm,draw] (\name) {};

\begin{scope}[shift={($(0,0)!0.5!(-#5,-#4)$)}]

\foreach\i in {0,...,#5}{
\draw (\i,0) -- (\i,#4);
}
\foreach \j in {0, ..., #4} {
\draw (0,\j) -- (#5,\j);
}
\foreach\i in {1,...,#5}{
\foreach\j in {1,...,#4}{
\draw ($(0,1+#4)+(\i,-\j)-(0.5,0.5)$) node[rectangle, minimum height=1 cm,minimum width=1 cm,draw] (\name -\j -\i) {};
}}



\ifstrequal{#3}{-NoValue-}{
\node (name #2) at ($(-0.5 , 0)!.5!(-0.5 , #4)$) {#2};
}{
\node (name #2) at ($(-0.5 , 0)!.5!(-0.5 , #4)$) {#3};
}

\end{scope}
\end{scope}

}
