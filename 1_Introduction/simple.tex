The first model we define is the common denominator of most models studied in this book:
\begin{itemize}
	\item $2$-player,
	\item zero sum,
	\item turn based,
	\item perfect information
\end{itemize}
game.

\subsection*{Players}
The term $2$-player means that there are two ""players"", Eve and Adam.
Many, many different names have been used: Player $0$ and Player $1$, 
Player I and Player II as in descriptive complexity,
{\'E}lo{\"i}se and Ab{\'e}lard, Circle and Square, corresponding to the graphical representation, 
Even and Odd, mostly for parity objectives, Player and Opponent, Pathfinder and Verifier in the context of automata,
Max and Min, which makes sense for quantitative objectives,
and this is only a very partial list of names they have been given.
In the names Eve and Adam, the first letters refer to $\exists$ and $\forall$ suggesting a duality between them.
We will make use of their gender to distinguish between them, so we speak of her or his strategy.

We speak of $1$-player games when there is only one player.
In the context of stochastic games, we refer to random as a third player, and more precisely as half a player.
Hence a $2\frac{1}{2}$-player game is a stochastic game with two players,
and a $1\frac{1}{2}$-player game is a stochastic game with one player.

The situation where there are more than two players is called multiplayer games.

\subsection*{Graphs}
Let us fix $C$ a set of colours. 
A (directed, coloured) ""graph"" is given by a set $V$ of vertices and a set $E \subseteq V \times C \times V$ of edges.
For an ""edge"" $e = (v,c,v')$ we write $\ing(e)$ for the ""incoming vertex"" $v$,
$\out(e)$ for the ""outgoing vertex"" $v'$, and $\col(e)$ for the colour $c$:
we say that $e$ is an ""outgoing edge"" of~$v$ and an ""incoming edge"" to~$v'$ labelled by $c$.
It is sometimes convenient to define directly the functions $\ing,\out : E \to V$, and $\col : E \to C$.

A ""path"" $\pi$ is a non empty finite or infinite sequence of the form:
\[
\pi = v_0 c_0 v_1 c_1 v_2 \dots
\] 
where for all $i$ we have $(v_i,c_i, v_{i+1}) \in E$.
If $\pi$ is finite the last element of the sequence is a vertex.
We let $\first(\pi)$ denote the first vertex occurring in $\pi$ and $\last(\pi)$ the last one if $\pi$ is finite.
We say that $\pi$ ""starts from"" $\first(\pi)$ and if $\pi$ is finite that $\pi$ ""ends in"" $\last(\pi)$.
We sometimes talk of a path and let the context determine whether it is finite or infinite.

We let $\Paths(G)$ denote the set of finite paths in the graph $G$, sometimes omitting $G$ when clear from the context.
To restrict to paths starting from $v$ we write $\Paths(G,v)$.
The set of infinite paths is $\Paths_\omega(G) \subseteq V^\omega$, and $\Paths_\omega(G,v)$ for those starting from $v$.

We use the standard terminology of graphs: 
for instance 
a vertex $v'$ is a ""successor"" of $v$ if $(v,v') \in E$,
and then $v$ is a ""predecessor"" of $v'$,
a vertex $v'$ is ""reachable"" from $v$ if there exists a path starting from $v$
and ending in $v'$,
the ""outdegree"" of a vertex is its number of outgoing edges,
the ""indegree"" is its number of incoming edges,
a ""simple path"" is a path with no repetitions of vertices,
a ""cycle"" is a path whose first and last vertex coincide,
it is a ""simple cycle"" if it does not strictly contain another cycle,
a ""self loop"" is an edge from a vertex to itself,
and a ""sink"" is a vertex with only a self loop as outgoing edge.

\subsection*{Arenas}
The ""arena"" is the place where the game is played, they have also been called game structures or game graphs.

In the turn based setting we define here, the set of vertices is divided into vertices controlled by each player.
Since we are for now interested in \textit{$2$-player} games, 
we have $V = \VE \uplus \VA$, where $\VE$ is the set of vertices controlled by Eve and $\VA$ the set of vertices controlled by Adam.
We represent vertices in $\VE$ by circles, and vertices in $\VA$ by squares, and also say that $v \in \VE$
belongs to Eve, or that Eve owns or controls $v$, and similarly for Adam.
An arena is given by a graph and the sets $\VE$ and $\VA$.
In the context of games, vertices are often referred to as positions.

\vskip1em
The adjective \textit{finite} means that the arena is finite, \textit{i.e.} there are finitely many vertices (hence finitely many edges).
We oppose \textit{deterministic} to \textit{stochastic}: in the first definition we are giving here, 
there is no stochastic aspect in the game.
An important assumption, called \textit{perfect information}, says that the players see everything about how the game
is played out, in particular they see the other player's moves.

\begin{figure}
\centering
  \begin{tikzpicture}[scale=1.3]
    % create (Eve's, Adam's, and stochastic) states at selected positions
    \node[s-eve] (init) at (0,0) {$v$};
    \node[s-eve] (v1) at (2,0) {$v_1$};
    \node[s-eve] (v2) at (4,0) {$v_2$};    
    \node[s-adam] (v3) at (6,0) {$v_3$};
    \node[s-adam] (v4) at (0,-1.5) {$v_4$};
    \node[s-eve] (v5) at (2,-1.5) {$v_5$};
    \node[s-adam] (v6) at (4,-1.5) {$v_6$};    
    \node[s-eve] (v7) at (6,-1.5) {$v_7$};
    % create edges
    \path[arrow]
      (init) edge (v1)
      (init) edge[bend left] (v4)
      (v4) edge[bend left] (init)
      (v4) edge node[above] {} (v5)
      (v5) edge[bend left] (v6)
      (v6) edge (v7)
      (v6) edge[bend left] (v5)
      (v5) edge[selfloop=90]  (v5)
      (v7) edge[selfloop]  (v7)
      (v1) edge (v2)
      (v1) edge (v6)
      (v2) edge (v6)
      (v2) edge[bend left] (v3)
      (v3) edge[bend left] (v2)
      (v3) edge (v7);
  \end{tikzpicture}
\caption{An example of an arena. Circles are controlled by Eve and squares by Adam.}
\label{1-fig:arena_example}
\end{figure}
Our definition of an arena does not include the initial vertex. %which is a design choice of minor importance.

We assume that all vertices have an outgoing edge.
This is for technical convenience, as it implies that we do not need to explain what happens when a play cannot be prolonged.

\subsection*{Playing}
The interaction between the two players consists in moving a token on the vertices of the arena.
The token is initially on some vertex.
When the token is in some vertex~$v$, the player who controls the vertex chooses an outgoing edge $e$ of $v$
and pushes the token along this edge to the next vertex $\out(e) = v'$.
The outcome of this interaction is the sequence of vertices traversed by the token: it is a path. 
In the context of games a path is also called a ""play"" and as for paths usually written~$\play$.
We note that plays can be finite (but non empty) or infinite.

\subsection*{Strategies}
The most important notion in this book is that of \textit{strategies} (sometimes called policies).
A strategy for a player is a full description of his or her moves in all situations.
Formally, a ""strategy"" is a function mapping finite plays to edges: 
\[
\sigma : \Paths \to E.
\]
We use $\sigma$ for strategies of Eve and $\tau$ for strategies of Adam so when considering a strategy $\sigma$ it is implicitly for Eve,
and similarly $\tau$ is implicitly a strategy for Adam.

We say that a play $\play = v_0 v_1 \dots$ is consistent with a strategy $\sigma$ of Eve if
for all $i$ such that $v_i \in \VE$ we have $\sigma(\play_{\le i}) = (v_i,v_{i+1})$.
The definition is easily adapted for strategies of Adam.

Once an initial vertex $v$ and two strategies $\sigma$ and $\tau$ have been fixed, 
there exists a unique infinite play starting from $v$ and consistent with both strategies written~$\pi^{v}_{\sigma,\tau}$.
Note that the fact that it is infinite follows from our assumption that all vertices have an outgoing edge.

\subsection*{Conditions}
The last ingredient to wrap up the definitions is (winning) ""conditions"", which is what Eve wants to achieve.
There are two types of conditions: the \emph{"qualitative"}, or Boolean ones, and the \emph{"quantitative"} ones.

A ""qualitative condition"" is $W \subseteq \Paths_\omega$: it separates winning from losing plays, in other words a play which belongs to $W$ is winning and otherwise it is losing. We also say that the play satisfies $W$.
In the zero sum context a play which is losing for Eve is winning for Adam, so Adam's condition is $\Paths_\omega \setminus W$.

A ""quantitative condition"" is $f : \Paths_\omega \to \Rinfty$: it assigns a real value (or plus or minus infinity) to a play, which can be thought of as a payoff or a score.
In the zero sum context Eve wants to maximise while Adam wants to minimise the outcome.

Often we define $W$ as a subset of $V^\omega$ and $f$ as $f : V^\omega \to \Rinfty$,
since $\Paths_\omega$ is included in $V^\omega$.

\subsection*{Objectives}
To reason about classes of games with the same conditions, we introduce the notions of ""objectives"" and colouring functions.
An objective and a colouring function together induce a condition.
The main point is that \textit{objectives are independent of the arenas}, so we can speak of the class of conditions induced by a given objective,
and by extension a class of games induced by a given objective, for instance parity games.

We fix a set $C$ of colours. 
A ""qualitative objective"" is $\Omega \subseteq C^\omega$, and a ""quantitative objective"" is a function $\Phi : C^\omega \to \Rinfty$. 

The link between an arena and an objective is given by a ""colouring function"" $\col : V \to C$ labelling vertices of the graph by colours.
We extend $\col$ componentwise to induce $\col : \Paths_\omega \to C^\omega$ mapping plays to sequences of colours:
\[
\col(v_0 v_1 \dots) = \col(v_0)\ \col(v_1) \dots
\]
A qualitative objective $\Omega$ and a colouring function $\col$ induce a qualitative condition $\Omega[\col]$ defined by:
\[
\Omega[\col] = \set{\play \in \Paths_\omega : \col(\play) \in \Omega}.
\] 
When $\col$ is clear from the context we sometimes say that a play $\play$ satisfies $\Omega$ but the intended meaning is that 
$\play$ satisfies $\Omega[\col]$, equivalently that $\col(\play) \in \Omega$.

Similarly, a quantitative objective $\Phi : C^\omega \to \Rinfty$ and a colouring function $\col$ induce 
a quantitative condition $\Phi[\col] : \Paths_\omega \to \Rinfty$ defined by:
\[
\Phi[\col](\play) = \Phi(\col(\play)).
\]

\begin{remark}
In our definition the colouring function labels vertices.
Another more general definition would label edges, and yet another relaxation would be to allow partial functions,
meaning that some vertices (or edges) are not labelled by a colour.
In most cases the variants are all (in some sense) equivalent; 
whenever we use a different definition we will make it explicit by referring for instance to edge colouring functions
or partial colouring functions.
\end{remark}

\subsection*{Games}
We can now give the following definitions.

\begin{definition}[""Games""]
\begin{itemize}
	\item A "graph" is a tuple $G = (V,E)$ where $V$ is a set of vertices
	and $E$ is a set of edges.
	
	\item An "arena" is a tuple $\arena = (G,\VE,\VA)$ where $G$ is a graph over the set of vertices $V$ and $V = \VE \uplus \VA$.

	\item A "colouring function" is a function $\col : V \to C$ where $C$ is a set of colours.

	\item A "qualitative condition" is $W \subseteq \Paths_\omega$.
	
	\item A "qualitative objective" is a subset $\Omega \subseteq C^\omega$.
	A colouring function $\col$ and a qualitative objective $\Omega$ induce 
	a qualitative condition $\Omega[\col]$.

	\item A ""qualitative game"" $\game$ is a tuple $(\arena,W)$ where $\arena$ is an arena and $W$ a qualitative condition.

	\item A "quantitative condition" is $f : \Paths_\omega \to \Rinfty$.

	\item A "quantitative objective" $\Phi$ is a function $\Phi : C^\omega \to \Rinfty$.
	A colouring function $\col$ and a quantitative objective $\Phi$ induce 
	a quantitative condition $\Phi[\col]$.
	
	\item A ""quantitative game"" $\game$ is a tuple $(\arena,f)$ where $\arena$ is an arena and $f$ a quantitative condition.	
\end{itemize}
\end{definition}
To be specific, the definition above is for $2$-player zero sum turn based perfect information games.
As a convention we use the condition to qualify games, so for instance parity games are games equipped with a parity condition.
This extends to graphs: we speak of a graph with condition $W$ for a graph equipped with a condition $W$,
and for instance a mean payoff graph if $W$ is a mean payoff condition.

We often introduce notations implicitly: for instance when we introduce a qualitative game $\Game$ without specifying the arena and the condition, it is understood that the arena is $\arena$ and the condition $W$.

We always implicitly take the point of view of Eve. 
Since we consider zero sum games we can easily reverse the point of view by considering the qualitative game 
$(\arena,\Paths_\omega \setminus W)$ and the qualitative game $(\arena,-f)$.
Indeed for the latter Adam wants to minimise $f$, which is equivalent to maximising $-f$.
The term zero sum comes from this: the total outcome for the two players is $f + (-f)$, meaning zero.

\begin{remark}
Unless otherwise stated we assume that graphs are finite, meaning that there are finitely many vertices (hence finitely many edges).
We equivalently say that the arena or the game is finite.
\Cref{part:infinite} will study games over infinite graphs.
\end{remark}

\subsection*{Winning in qualitative games}
Now that we have the definitions of a game we can ask the main question: 
given a game $\game$ and a vertex $v$, who wins $\game$ from $v$?

Let $\game$ be a qualitative game and $v$ a vertex.
A strategy $\sigma$ for Eve is called ""winning"" from $v$ if every play starting from $v$ consistent with $\sigma$ is winning, 
\textit{i.e.} satisfies $W$.
Another common terminology is that $\sigma$ ensures $W$.
In that case we say that Eve has a winning strategy from $v$ in $\game$, or equivalently that Eve wins $\game$ from $v$.
This vocabulary also applies to Adam: for instance 
a strategy $\tau$ for Adam is called ""winning"" from $v$ if every play starting from $v$ consistent with $\tau$ is losing, 
\textit{i.e.} does not satisfy $W$.

We let $\WE(\game)$ denote the set of vertices $v$ such that Eve wins $\game$ from $v$,
it is called ""winning region"", or sometimes winning set. 
A vertex in $\WE(\game)$ is said winning for Eve.
The analogous notation for Adam is $\WA(\game)$.

We say that a strategy is ""optimal"" if it is winning from all vertices in $\WE(\game)$.

\begin{fact}[Winning regions are disjoint]
\label{1-fact:winning_regions_disjoint}
For all qualitative games $\game$ we have $\WE(\game) \cap \WA(\game) = \emptyset$.
\end{fact}
\begin{proof}
Assume for the sake of contradiction that both players have a winning strategy from $v$, then $\pi^{v}_{\sigma,\tau}$
would both satisfy $W$ and not satisfy $W$, a contradiction.
\end{proof}

It is however not clear that for every vertex $v$, \textit{some} player has a winning strategy from $v$,
which symbolically reads $\WE(\game) \cup \WA(\game) = V$.
One might imagine that if Eve picks a strategy, then Adam can produce a counter strategy beating Eve's strategy, 
and vice versa, if Adam picks a strategy, then Eve can come up with a strategy winning against Adam's strategy.
A typical example would be rock-paper-scissors (note that this is a concurrent game, meaning the two players play simultanously,
hence it does not fit in the definitions given so far), where neither player has a winning strategy.

Whenever $\WE(\game) \cup \WA(\game) = V$, we say that the game is ""determined"".
Being determined can be understood as follows: the outcome can be determined before playing assuming both players play optimally since one of them can ensure to win whatever is the strategy of the opponent.

\begin{theorem}[Borel determinacy]
\label{1-thm:borel_determinacy}
Qualitative games with Borel conditions are determined.
\end{theorem}

The definition of Borel sets goes beyond the scope of this book. 
Suffice to say that all conditions studied in this book are (very simple) examples of Borel sets,
implying that our qualitative games are all determined 
(as long as we consider perfect infomation and turn based games, the situation will change with more general models of games).

\subsection*{Computational problems for qualitative games}
We identify three computational problems.
The first is that of ""solving a game"", which is the simplest one and since it induces a decision problem, allows us 
to make complexity theoretic statements.

\decpb[Solving the game]{A "qualitative game" $\game$ and a vertex $v$}{Does Eve win $\game$ from $v$?}

The second problem extends the previous one: most algorithms solve games for all vertices at once instead of only for the given initial vertex.
This is called ""computing the winning regions"".

\decpb[Computing the winning regions]{A "qualitative game" $\game$}{$\WE(\game)$ and $\WA(\game)$}

The third problem is ""constructing a winning strategy"".

\decpb[Constructing a winning strategy]{A "qualitative game" $\game$ and a vertex $v$}{A winning strategy for Eve from $v$}

We did not specify how the winning regions or the winning strategies are represented, this will depend on the types of games we consider.

\subsection*{Values in quantitative games}
Let $\game$ be a quantitative game and $v$ a vertex.
Given $x \in \R$ called a threshold, we say that a strategy $\sigma$ for Eve ""ensures"" $x$ from $v$ 
if every play $\pi$ starting from $v$ consistent with $\sigma$ has value at least $x$ under $f$,
\textit{i.e.} $f(\play) \ge x$.
In that case we say that Eve has a strategy ensuring $x$ in $\game$ from $v$.

Note that by doing so we are actually considering a qualitative game in disguise, 
where the qualitative condition is the set of plays having value at least $x$ under $f$.
Formally, a quantitative condition $f$ and a threshold $x$ induce a qualitative condition 
\[
f_{\ge x} = \set{\play \in \Paths_\omega \mid f(\play) \ge x}.
\]

Analogously, we say that a strategy $\tau$ for Adam ""ensures"" $x$ from $v$ 
if every play $\play$ starting from $v$ consistent with $\tau$ has value at most $x$ under $f$,
\textit{i.e.} $f(\play) \le x$.


We let $\ValueE^{\game}(v)$ denote the quantity
\[
\sup_{\sigma} \inf_{\tau} f(\play_{\sigma,\tau}^{v}),
\]
where $\sigma$ ranges over all strategies of Eve and $\tau$ over all strategies of Adam.
We also write $\ValueE^{\sigma}(v) = \inf_{\tau} f(\play_{\sigma,\tau}^{v})$
so that $\ValueE^{\game}(v) = \sup_{\sigma} \ValueE^\sigma(v)$.
This is called the value of Eve in the game $\game$ from $v$,
and represents the best outcome that she can ensure against any strategy of Adam.
Note that $\ValueE^{\game}(v)$ is either a real number, $\infty$, or $-\infty$.

A strategy $\sigma$ such that $\ValueE^\sigma(v) = \ValueE^{\game}(v)$ is called ""optimal"" from $v$,
and it is simply optimal if the equality holds for all vertices.
Equivalently, $\sigma$ is optimal from $v$ if for every play $\play$ consistent with $\sigma$ starting from $v$ 
we have $f(\play) \ge \ValueE^{\game}(v)$.

There may not exist optimal strategies which is why we introduce the following notion.
For $\varepsilon > 0$, a strategy $\sigma$ such that $\ValueE^\sigma(v) \ge \ValueE^{\game}(v) - \varepsilon$
is called ""$\varepsilon$-optimal"".
If $\ValueE^{\game}(v)$ is finite there exist $\varepsilon$-optimal strategies for any $\varepsilon > 0$.

Symmetrically, we let $\ValueA^{\game}(v)$ denote 
\[
\inf_{\tau} \sup_{\sigma} f(\play_{\sigma,\tau}^{v}).
\]
\begin{fact}[Comparison of values for Eve and Adam]
\label{1-fact:comparaison_values_eve_adam}
For all quantitative games $\game$ and vertex $v$ we have $\ValueE^{\game}(v) \le \ValueA^{\game}(v)$.
\end{fact}
\begin{proof}
For any function $F : X \times Y \to \Rinfty$, we have 
\[
\sup_{x \in X} \inf_{y \in Y} F(x,y) \le \inf_{y \in Y} \sup_{x \in X} F(x,y).
\]
\end{proof}

If this inequality is an equality, we say that the game $\game$ is \textit{determined} in $v$,
and let $\Value^{\game}(v)$ denote the value in the game $\game$ from $v$
and $\Value^\sigma(v)$ for $\inf_{\tau} f(\play_{\sigma,\tau}^{v})$.
Similarly as for the qualitative case, being determined can be understood as follows: the outcome can be determined before playing assuming both players play optimally, and in that case the outcome is the value.

We say that a quantitative objective $f : C^\omega \to \Rinfty$ is Borel if for all $x \in \R$,
the qualitative objective $f_{\ge x} \subseteq C^\omega$ is a Borel set.

\begin{corollary}[Borel determinacy for quantitative games]
\label{1-cor:borel_determinacy}
Quantitative games with Borel conditions are determined, meaning that
for all quantitative games $\game$ we have $\ValueE^{\game} = \ValueA^{\game}$.
\end{corollary}
\begin{proof}
If $\ValueE^{\game}(v) = \infty$ then thanks to the inequality above $\ValueA^{\game}(v) = \infty$ and the equality holds.
Assume $\ValueE^{\game}(v) = -\infty$ and let $r$ be a real number.
(The argument is actually the same as for the finite case but for the sake of clarity we treat them independently.)
We consider $f_{\ge r}$.
By definition, this a qualitative Borel condition, so \Cref{1-thm:borel_determinacy} implies that it is determined.
Since Eve cannot have a winning strategy for $f_{\ge r}$, as this would contradict the definition of $\ValueE^{\game}(v)$,
this implies that Adam has a winning strategy for $f_{\ge r}$, 
meaning a strategy $\tau$ such that every play $\play$ starting from $v$ consistent with $\tau$ satisfy $f(\play) < r$.
In other words, $\ValueA^{\tau}(v) = \sup_{\sigma} f(\play_{\sigma,\tau}^{v}) \le r$, which implies that $\ValueA^{\game}(v) \le r$.
Since this is true for any real number $r$, this implies $\ValueA^{\game}(v) = -\infty$.

Let us now assume that $x = \ValueE^{\game}(v)$ is finite and let $\varepsilon > 0$.
We consider $f_{\ge x + \varepsilon}$.
By definition, this a qualitative Borel condition, so \Cref{1-thm:borel_determinacy} implies that it is determined.
Since Eve cannot have a winning strategy for $f_{\ge x + \varepsilon}$, as this would contradict the definition of $\ValueE^{\game}(v)$,
this implies that Adam has a winning strategy for $f_{\ge x + \varepsilon}$, 
meaning a strategy $\tau$ such that every play $\play$ starting from $v$ consistent with $\tau$ satisfy $f(\play) < x + \varepsilon$.
In other words, $\ValueA^{\tau}(v) = \sup_{\sigma} f(\play_{\sigma,\tau}^{v}) \le x + \varepsilon$, which implies that $\ValueA^{\game}(v) \le x + \varepsilon$.
Since this is true for any $\varepsilon > 0$, this implies $\ValueA^{\game}(v) \le \ValueE^{\game}(v)$.
As we have seen the converse inequality holds, implying the equality.
\end{proof}

Note that this determinacy result does not imply the existence of optimal strategies.

\subsection*{Computational problems for quantitative games}
As for qualitative games, we identify different computational problems.
The first is solving the game.

\decpb[Solving the game]{A "quantitative game" $\game$, a vertex $v$, and a threshold $x \in \Qinfty$}{Does Eve have a strategy ensuring $x$ from $v$?}

A very close problem is the ""value problem"".

\decpb[Solving the value problem]{A "quantitative game" $\game$, a vertex $v$, and a threshold $x \in \Qinfty$}{Is it true that $\Value^{\game}(v) \ge x$?}

The two problems of "solving a game" and the "value problem" are not quite equivalent: 
they become equivalent if we assume the existence of optimal strategies.

The "value problem" is directly related to ""computing the value"".

\decpb[Computing the value]{A "quantitative game" $\game$ and a vertex $v$}{$\Value^{\game}(v)$}

What "computing the value" means may become unclear if the value is not a rational number, making its representation complicated.
Especially in this case, it may be enough to approximate the value, which is indeed what the "value problem" gives us: 
by repeatingly applying an algorithm solving the value problem one can approximate the value to any given precision,
using a binary search.

\begin{lemma}[Binary search for computing the value]
\label{1-lem:binary_search_computing_value}
If there exists an algorithm $A$ for solving the value problem of a class of games, 
then there exists an algorithm for approximating the value of games in this class within precision $\varepsilon$ 
using $\log(\frac{1}{\varepsilon})$ calls to the algorithm $A$.
\end{lemma}

The following problem is global, in the same way as "computing the winning regions".

\decpb[Computing the value function]{A "quantitative game" $\game$}{Output the value function $\Value^{\game} : V \to \Rinfty$}

Finally, we are sometimes interested in constructing optimal strategies provided they exist.

\decpb[Constructing an optimal strategy]{A "quantitative game" $\game$ and a vertex $v$}{Output an optimal strategy from $v$}

A close variant is to construct $\varepsilon$-optimal strategies, usually with $\varepsilon$ given as input.

\subsection*{Prefix independent objectives}
A qualitative objective $\Omega$ is:
\begin{itemize}
	\item ""closed under adding prefixes"" if for every finite sequence $\rho$ and for every infinite sequence $\rho'$,
if $\rho' \in \Omega$ then $\rho \rho' \in \Omega$;
	\item ""closed under removing prefixes"" if for every finite sequence $\rho$ and for every infinite sequence $\rho'$,
if $\rho \rho' \in \Omega$ then $\rho' \in \Omega$;
	\item ""prefix independent"" if it is closed under both adding and removing prefixes;
in other words whether a sequence satisfies $\Omega$ does not depend upon finite prefixes.
\end{itemize}

Let $\play$ be a finite play consistent with $\sigma$, we write $\sigma_{\mid \play}$ for the strategy defined by
\[
\sigma_{\mid \play}(\play') = \sigma(\play \play').
\]

\begin{fact}[Winning for prefix independent objectives]
\label{1-fact:winning_prefix_independent_qualitative}
Let $\Game$ be a qualitative game with objective $\Omega$ closed under removing prefixes,
$\sigma$ a winning strategy from $v$,
and $\play$ a finite play consistent with $\sigma$ starting from $v$.
Then $\sigma_{\mid \play}$ is winning from $v' = \last(\play)$.
\end{fact}
\begin{proof}
Let $\play'$ be an infinite play consistent with $\sigma_{\mid \play}$ from $v'$,
then $\play \play'$ is an infinite play consistent with $\sigma$ starting from $v$, 
implying that it is winning, and since $\Omega$ is closed under removing prefixes
the play $\play'$ is winning. Thus $\sigma_{\mid \play}$ is winning from~$v'$.
\end{proof}

\begin{corollary}[Reachable vertices of a winning strategy for prefix independent objectives]
\label{1-cor:reachable_vertices_prefix_independent}
Let $\Game$ be a qualitative game with objective $\Omega$ closed under removing prefixes and $\sigma$ a winning strategy from $v$.
Then all vertices reachable from $v$ by a play consistent with $\sigma$ are winning.
\end{corollary}
In other words, when playing a winning strategy the play does not leave the winning region.


Similarly, a quantitative objective $\Phi$ is:
\begin{itemize}
	\item ""monotonic under adding prefixes"" if for every finite sequence $\rho$ and for every infinite sequence $\rho'$
	we have $\Phi(\rho') \le \Phi(\rho \rho')$;
	\item ""monotonic under removing prefixes"" if for every finite sequence $\rho$ and for every infinite sequence $\rho'$
	we have $\Phi(\rho') \ge \Phi(\rho \rho')$;
	\item ""prefix independent"" if it is monotonic under both adding and removing prefixes.
\end{itemize}
The fact above extends to quantitative objectives with the same proof.
\begin{fact}[Winning for prefix independent objectives, quantitative case]
\label{1-fact:winning_prefix_independent_quantitative}
Let $\Game$ be a quantitative game with objective $\Phi$ monotonic under removing prefixes,
$\sigma$ a strategy ensuring $x$ from $v$, and $\play$ a finite play consistent with $\sigma$ starting from $v$.
Then $\sigma_{\mid \play}$ ensures $x$ from $v' = \last(\play)$.
\end{fact}
\begin{proof}
Let $\play'$ be an infinite play consistent with $\sigma_{\mid \play}$ from $v'$,
then $\play \play'$ is an infinite play consistent with $\sigma$ starting from $v$, 
implying that $\Phi(\play \play') \ge x$, and since $\Phi$ is monotonic under removing prefixes
this implies that $\Phi(\play') \ge x$. Thus $\sigma_{\mid \play}$ ensures $x$ from $v'$.
\end{proof}

\begin{corollary}[Comparison of values along a play]
\label{1-cor:comparison_values_along_play}
Let $\Game$ be a quantitative game with objective $\Phi$ monotonic under removing prefixes and $\sigma$ an optimal strategy from $v$.
Then for all vertices $v'$ reachable from $v$ by a play consistent with $\sigma$ we have $\val^{\game}(v) \le \val^{\game}(v')$.
\end{corollary}
In other words, when playing an optimal strategy the value is non-decreasing along the play.
