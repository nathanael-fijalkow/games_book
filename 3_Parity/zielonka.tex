\begin{theorem}
There exists a quasipolynomial time algorithm for solving parity games, and more specifically of complexity $n^{O(\log n)}$.
\end{theorem}

\begin{algorithm}
 \KwData{A parity game $\Game$ with priorities in $[1,d]$}
 \SetKwFunction{FSolveEven}{SolveEven}
 \SetKwFunction{FSolveOdd}{SolveOdd}
 \SetKwProg{Fn}{Function}{:}{}
 \DontPrintSemicolon

\Fn{\FSolveEven{$\Game$}}{

	$i \leftarrow -1$ 
	
	$V_0 \leftarrow V$

	\Repeat{$\WA(\Game_i) = \emptyset$}{
		$i \leftarrow i + 1$

		Let $\Game_i$ the subgame of $\Game$ induced by $V_i \setminus \AttrE^{\Game_i}(d)$

		$\WE(\Game_i) \leftarrow$ \FSolveOdd{$\Game_i$}

		$V_{i+1} \leftarrow V_i \setminus \AttrA^{\Game}( \WA(\Game_i) )$ 
		}
		
	\Return{$V_i$}
}
\vskip1em
\Fn{\FSolveOdd{$\Game$}}{
	\If{$d = 1$}{
		\Return{$V$}	
	}
	
	$i \leftarrow -1$ 
	
	$V_0 \leftarrow V$

	\Repeat{$\WE(\Game_i) = \emptyset$}{
		$i \leftarrow i + 1$

		Let $\Game_i$ the subgame of $\Game$ induced by $V_i \setminus \AttrA^{\Game_i}(d)$

		$\WA(\Game_i) \leftarrow$ \FSolveEven{$\Game_i$}

		$V_{i+1} \leftarrow V_i \setminus \AttrE^{\Game}( \WE(\Game_i) )$ 
		}
		
	\Return{$V_i$}
}
\vskip1em
\If{$d$ is even}{
	\Return{\FSolveEven{$\Game$}}
}
\Else{
	\Return{\FSolveOdd{$\Game$}}
}
\caption{The recursive algorithm for computing the winning region of parity games.}
\label{3-algo:zielonka_even}
\end{algorithm}

We revisit the exponential recursive algorithm presented in \cref{2-sec:parity}.
We refer to \cref{3-algo:zielonka_even} for an equivalent presentation of this algorithm, 
where we make explicit all recursive calls involving the maximal priority $d$.
The benefit of doing this is to make the following observation:
during the $i$\textsuperscript{th} recursive call for $d$, the algorithm removes from the game $\Game$ the subset 
$X_i = \AttrA^{\Game}( \WA(\Game_i) )$. 
Note that $X_i$ is a trap for Eve in $\Game$ and a subset of the winning region of Adam in $\Game$:
we say that $X_i$ is a dominion for Eve.
More generally, given a game $\Game$, a set $X$ of vertices is a ""dominion"" for Eve if
it is a trap for Adam and a subset of the winning region of Eve.

Let $i_{\infty}$ be the final value of $i$, the sets $X_0,\dots,X_{i_{\infty}-1}$ are disjoint.
The \textit{key idea} is that this implies that at most one of them can have size more than half the total size.

To take advantage of this, we change the specification of the algorithm: 
the new algorithm takes as input a parity game and two parameters $s_{\mEve}$ and $s_{\mAdam}$.
As before, they are two mutually recursive procedures, $\textsl{SolveE}$ and $\textsl{SolveA}$.
At an intuitive level, the objective of $\textsl{SolveE}(\Game,s_{\mEve},s_{\mAdam})$ 
is to return a (non-empty whenever possible) "dominion" for Eve of size at most $s_{\mEve}$.

We spell out the pseudocode of $\textsl{SolveE}$ in \cref{3-algo:quasipoly_zielonka_even}, leaving out the perfectly symmetric $\textsl{SolveA}$.
The base cases are when there is only one priority, in which case Eve wins everywhere if the priority is even, and Adam wins everywhere if the priority is odd.

\begin{algorithm}
 \KwData{A parity game $\Game$ with priorities in $[1,d]$, and $d$ even, two parameters $s_{\mEve}$ and $s_{\mAdam}$}
 \SetKwFunction{FTreat}{Treat}
 \SetKwProg{Fn}{Function}{:}{}

	Let $i = 0$ and $V_0 = V$
	
	Let $\H_0$ the subgame of $\Game$ induced by $V_0$

	Let $\Game_0$ the subgame of $\H_0$ induced by $V_0 \setminus \AttrE^{\H_0}(d)$

	\vskip1em
	\Fn{\FTreat{$X_i$}}{
		Let $V_{i+1} = V_i \setminus \AttrA^{\H_i}(X_i)$
	
		Let $\H_{i+1}$ the subgame of $\H_i$ induced by $V_{i+1}$

		Let $\Game_{i+1}$ the subgame of $\H_{i+1}$ induced by $V_{i+1} \setminus \AttrE^{\H_{i+1}}(d)$
				
		$i = i + 1$
}

	\vskip1em
	\tcp{recursive calls for small dominions}
	\While{$X_i = \textsl{SolveA}(\Game_i, s_{\mEve}, \lfloor s_{\mAdam} / 2 \rfloor) \neq \emptyset$}{
		\FTreat($X_i$)
	}
	
	\tcp{one recursive call for a large dominion}

	\If{$X_i = \textsl{SolveA}(\Game_i, s_{\mEve}, s_{\mAdam}) \neq \emptyset$}{
		\FTreat($X_i$) 
	}
	
	\tcp{recursive calls for small dominions}
	\While{$X_i = \textsl{SolveA}(\Game_i, s_{\mEve}, \lfloor s_{\mAdam} / 2 \rfloor) \neq \emptyset$}{
		\FTreat($X_i$) 
	}

	\Return{$V_i$}	
\caption{A recursive quasipolynomial algorithm for computing the winning regions of parity games -- the procedure $\textsl{SolveE}$.}
\label{3-algo:quasipoly_zielonka_even}
\end{algorithm}

We need three simple facts about "traps".

\begin{fact}\label{3-fact:traps}\hfill
\begin{itemize}
	\item Let $S$ be a trap for Eve in the game $\Game$ and $X$ a set of vertices, 
	then $S \setminus \AttrE(X)$ is a trapfor Eve in the subgame of $\Game$ induced by $V \setminus \AttrE(X)$.
	\item Let $S$ be a trap for Eve in the game $\Game$ and $X$ a set of vertices such that $S \cap X = \emptyset$, 
	then $S \subseteq V \setminus \AttrA(X)$ and $S$ is a trap for Eve in the subgame of $\Game$ induced by $V \setminus \AttrA(X)$.
	\item Let $S$ be a trap for Eve in the game $\Game$ and $Z$ a trap for Eve in the subgame of $\Game$ induced by $S$,
	then $Z$ is a trap for Eve in $\Game$.
\end{itemize}
\end{fact}

The following lemma implies the correctness of the algorithm.

\begin{lemma}\label{3-lem:correctness_quasipoly_zielonka}\hfill
\begin{itemize}
	\item For all dominions $S$ for Eve, if $|S| \le s_{\mEve}$, then $S \subseteq \textsl{SolveE}(\Game,s_{\mEve},s_{\mAdam})$.
	\item For all dominions $S$ for Adam, if $|S| \le s_{\mAdam}$, then $S \cap \textsl{SolveE}(\Game,s_{\mEve},s_{\mAdam}) = \emptyset$.
\end{itemize}
\end{lemma}

Indeed, $\WE(\Game)$ and $\WA(\Game)$ are dominions for Eve and Adam in $\Game$, 
so \cref{3-lem:correctness_quasipoly_zielonka} implies that $\textsl{SolveE}(\Game,n,n) = \WE(\Game)$.

\begin{proof}
The proof is by induction on the number of priorities: indeed all recursive calls to $\textsl{SolveA}$ are for games with one less priority.
It follows that by induction hypothesis the following two properties hold, for all $i$.
\begin{itemize}
	\item For all dominions $S$ in $\Game_i$ for Adam, if $|S| \le s_{\mAdam}$, 
	then \[ S \subseteq \textsl{SolveA}(\Game_i,s_{\mEve},s_{\mAdam}). \]
	\item For all dominions $S$ in $\Game_i$ for Eve, if $|S| \le s_{\mEve}$, 
	then \[ S \cap \textsl{SolveA}(\Game_i,s_{\mEve},s_{\mAdam}) = \emptyset. \]
\end{itemize}
Since there will be an induction inside this induction, we refer to the induction above as the external induction,
and the second one as the internal induction.

We write $i_{\infty}$ for the final value of $i$, \textit{i.e.} such that 
$\textsl{SolveE}(\Game,s_{\mEve},s_{\mAdam}) = V_{i_{\infty}}$.
Note that the first item reads $S \subseteq V_{i_{\infty}}$.

Let $S$ be a dominion for Eve in $\Game$ such that $|S| \le s_{\mEve}$.
We show by internal induction on $i$ that 
$S \subseteq V_i$.

For $i = 0$ this is by definition.
We now assume that $S \subseteq V_i$.
Recall that $\Game_i$ is the subgame of $\H_i$ induced by $V_i \setminus \AttrE^{\H_i}(d)$.
It follows from the first item of \cref{3-fact:traps} that $S \setminus \AttrE^{\H_i}(d)$
is a dominion for Eve in $\Game_i$.
Since $S \setminus \AttrE^{\H_i}(d)$ has size at most $s_{\mEve}$, 
the second item of the external induction hypothesis implies that 
$S \setminus \AttrE^{\Game_i}(d)$ has an empty intersection with $X_i = \textsl{SolveA}(\Game_i,s_{\mEve},s_{\mAdam})$,
implying that $S \cap X_i = \emptyset$.
It follows from the second item of \cref{3-fact:traps} that $S \subseteq V_i \setminus \AttrA^{\H_i}(X_i) = V_{i+1}$.
This finishes the internal induction, and implies the first item.

\vskip1em
We now prove the second item.

Let $S$ be a (non-empty) dominion for Adam in $\Game$ such that $|S| \le s_{\mAdam}$.
We write $S_i = S \cap V_i$.
Note that the second item reads $S_{i_{\infty}} = \emptyset$.

We first show by internal induction on $i$ that 
$S_i$ is a dominion for Adam in $\H_i$.
For $i = 0$ this is by definition.
We now assume that $S_i$ is a dominion for Adam in $\H_i$.
Recall that $\H_{i+1}$ is the subgame of $\H_i$ induced by $V_{i+1} = V_i \setminus \AttrA^{\H_i}(X_i)$.
It follows from the first item of \cref{3-fact:traps} applied to $S_i$ (swapping the roles of Eve and Adam) 
that $S_i \setminus \AttrA^{\H_i}(X_i) = S_{i+1}$ is a dominion for Adam in $\H_{i+1}$.
This finishes the internal induction.

We showed that $S_i$ is a dominion for Adam in $\H_i$ for each $i$.
To apply the external inductive hypothesis, we need to exhibit dominions for Adam in $\Game_i$ for each $i$.
We consider $\H'_i$ the subgame of $\H_i$ induced by $S_i$.
Let $Z_i = \WA^{\H'_i}(\Parity \cup \Reach(d))$: 
in words, $Z_i$ is the subset of vertices of $S_i$ from where Adam can ensure that the parity objective is not satisfied
and never to see priority $d$.
We prove two properties:
\begin{itemize}
	\item If $Z_i = \emptyset$, then $S_i = \emptyset$.

The fact that $Z_i$ is empty implies that Eve has a strategy $\sigma$ in $\Game_i$ which from $S_i$ ensures $\Parity$ or to see priority $d$.
Since $S_i$ is a dominion for Adam in $\H_i$, there exists a strategy $\tau$ for Adam which from $S_i$ ensures the complement of $\Parity$
and to stay in $S_i$. 
Playing $\sigma$ against $\tau$ yields a contradiction, since $\sigma$ ensures $\Parity$ if the play remains in $S_i$.

	\item $Z_i$ is a dominion for Adam in $\Game_i$.

That $Z_i$ is a subset of the winning region of Adam in $\Game_i$ is clear.
To see that $Z_i$ is a trap for Eve in $\Game_i$, we first note that since $S_i$ is a trap for Eve in $\H_i$
and $Z_i$ is a trap for Eve in $\H'_i$, the subgame of $\H_i$ induced by $S_i$, 
then $Z_i$ is a trap in $\H_i$ by the third item of \cref{3-fact:traps}.
Now, since $Z_i$ has an empty intersection with $d$, by the second item of \cref{3-fact:traps} 
this implies that $Z_i$ is a trap for Eve in the subgame of $\H_i$ induced by $V_i \setminus \AttrE^{\H_i}(d)$,
which is exactly $\Game_i$.
\end{itemize}

We are now fully equipped to prove that $S_{i_{\infty}} = \emptyset$.
Let $i_{\ell}$ be the value of $i$ for which the algorithm performs a recursive call for a large dominion.
Since the sequence $(S_i)_i$ is non-increasing, if $S_i$ is empty for some $i$, then $S_{i_{\infty}}$ as well.

The first while loop was exited for $i = i_{\ell}$, implying that 
\[
\textsl{SolveA}(\Game_{i_{\ell}},s_{\mEve},\lfloor s_{\mAdam} / 2 \rfloor) = \emptyset.
\]
We apply the external inductive hypothesis to the dominion $Z_{i_{\ell}}$ for Adam in $\Game_{i_{\ell}}$
for the parameter $\lfloor s_{\mAdam} / 2 \rfloor$: if $|Z_{i_{\ell}}| \le \lfloor s_{\mAdam} / 2 \rfloor$, then
$Z_{i_{\ell}} \subseteq \textsl{SolveA}(\Game_{i_{\ell}},s_{\mEve},\lfloor s_{\mAdam} / 2 \rfloor)$,
implying that $Z_{i_{\ell}}$ is empty, which by the first property above implies that $S_{i_{\ell}}$ is empty,
thus $S_{i_{\infty}}$ as well, proving the second item of \cref{3-lem:correctness_quasipoly_zielonka}.
Excluding this case, we then analyse the situation where 
$|Z_{i_{\ell}}| > \lfloor s_{\mAdam} / 2 \rfloor$. 

We apply again the external inductive hypothesis to $Z_{i_\ell}$, but this time 
for the parameter $s_{\mAdam}$: 
if $|Z_{i_\ell}| \le s_{\mAdam}$, then $Z_{i_\ell} \subseteq X_{i_\ell}$.
Since $Z_{i_\ell} \subseteq S_{i_\ell} \subseteq S$ and $|S| \le s_{\mAdam}$, the premise is satisfied,
so $Z_{i_\ell} \subseteq X_{i_\ell}$.
Since $Z_{i_\ell}$ is non-empty, so is $X_{i_\ell}$: the search for a large dominion was successful, and in particular
$i$ is incremented at this stage, implying that $i_\ell < i_\infty$.

Consider $S_{i_\ell + 1} = S_{i_\ell} \setminus \AttrA^{\H_{i_\ell}}(X_{i_\ell})$.
In particular $S_{i_\ell + 1} \subseteq S_{i_\ell} \setminus X_{i_\ell} \subseteq S_{i_\ell} \setminus Z_{i_\ell}$, so
\[
|S_{i_\ell + 1}| \le |S_{i_\ell}| - |Z_{i_\ell}| \le \lfloor s_{\mAdam} / 2 \rfloor.
\]

The second while loop was exited for $i = i_{\infty}$, implying that 
\[
\textsl{SolveA}(\Game_{i_{\infty}},s_{\mEve},\lfloor s_{\mAdam} / 2 \rfloor) = \emptyset.
\]		
We apply the external inductive hypothesis to the dominion $Z_{i_{\infty}}$ for Adam in $\Game_{i_{\infty}}$
for the parameter $\lfloor s_{\mAdam} / 2 \rfloor$: if $|Z_{i_{\infty}}| \le \lfloor s_{\mAdam} / 2 \rfloor$, then
$Z_{i_{\infty}} \subseteq \textsl{SolveA}(\Game_{i_{\infty}},s_{\mEve},\lfloor s_{\mAdam} / 2 \rfloor)$.
The premise holds because $|S_{i_\ell + 1}| \le \lfloor s_{\mAdam} / 2 \rfloor$, the sequence $(S_i)_i$ is non-increasing, 
$i_\ell < i_\infty$, and $Z_{i_{\infty}} \subseteq S_{i_\infty}$.
Hence $Z_{i_\infty}$ is empty. Thanks to the first property above for $Z_i$, this implies that $S_{i_\infty}$ is empty.
This finishes the proof of the second item of \cref{3-lem:correctness_quasipoly_zielonka}.
\end{proof}

We obtain an algorithm for computing the winning regions of parity games using $\textsl{SolveE}(\Game,n,n)$,
where $\Game$ is a parity game with $n$ vertices.

We now perform a complexity analysis.
Let us write $f(m,n,d,s_{\mEve},s_{\mAdam})$ for the complexity of the algorithm over parity games with $m$ edges, $n$ vertices, $d$ priorities,
and parameters $s_{\mEve}$ and $s_{\mAdam}$.
We have $f(m,n,1,s_{\mEve},s_{\mAdam}) = O(n)$ and $f(m,0,d,s_{\mEve},s_{\mAdam}) = f(m,n,d,0,s_{\mAdam}) = f(m,n,d,s_{\mEve},0) = O(1)$, 
with the general induction 
\[
\begin{array}{lll}
f(m,n,d,s_{\mEve},s_{\mAdam}) & \le & n \cdot f(m,n,d-1,s_{\mEve},\lfloor s_{\mAdam} / 2 \rfloor) \\
							  & & +\ f(m,n,d-1,s_{\mEve},s_{\mAdam}) \\
							  & & +\ O(nm).
\end{array}
\]
The term $n \cdot f(m,n,d-1,s_{\mEve},\lfloor s_{\mAdam} / 2 \rfloor)$ corresponds to the recursive calls for small dominions, 
the term $f(m,n,d-1,s_{\mEve},s_{\mAdam})$ to the recursive call for a large dominion,
and $O(nm)$ for the computation of the attractors.
We obtain
\[
f(m,n,d,n,n) \le m \cdot n^{1 + 2 \lfloor \log n \rfloor} \cdot \binom{d + 2 \lfloor \log n \rfloor}{2 \lfloor \log n \rfloor}.
\]
which implies a quasipolynomial upper bound of $n^{O(\log n)}$.
