As discussed in the introduction, the literature on multiobjective models is too vast to provide a full account here. We therefore focus on some directions particularly relevant to our focus.

\paragraph{Multidimension games.} Energy games and their related work were discussed in~\cref{chap:counters}. Our presentation of mean-payoff games is inspired by Velner et al.~\cite{Velner&al:2015}. Brenguier and Raskin studied the Pareto curves of these games in~\cite{Brenguier&Raskin:2015}. While we considered \textit{conjunctions} of mean-payoff objectives, Velner proved that Boolean combinations lead to undecidability~\cite{Velner:2015}.

The undecidability of total-payoff games was first established in~\cite{Chatterjee&al:2015} via reduction from the halting problem for two-counter machines: we provided here a new, simpler proof based on robot games~\cite{Niskanen&Potapov&Reichert:2016}. This undecidability result, along with the complexity barriers of mean-payoff and total-payoff games, motivated the introduction of (multidimension) ""\textit{window objectives}"": conservative variants of mean-payoff and total-payoff objectives that benefit from increased tractability and permit to reason about time bounds~\cite{Chatterjee&al:2015}. Window variants of "parity" objectives have been studied in~\cite{Bruyere&Hautem&Randour:2016}.

\paragraph{Combinations of different objectives.} We focused on multidimension games obtained by conjunction of \textit{identical} objectives. Conjunctions of \textit{heterogeneous} objectives have been studied in a variety of contexts including mean-payoff parity games~\cite{Chatterjee&Henzinger&Jurdzinski:2005,Daviaud&Jurdzinski&Lazic:2018}, energy parity games~\cite{Chatterjee&Doyen:2012,Chatterjee&Randour&Raskin:2014}, average-energy games with energy constraints~\cite{Bouyer&al:2018,Bouyer&al:2017}, simple quantitative objectives~\cite{Bruyere&Hautem&Raskin:2016}. Le Roux, Pauly and Randour studied general conditions under which finite-memory strategies suffice to play optimally, even in a broad multi-objective setting~\cite{LeRoux&Pauly&Randour:2018}.


\paragraph{Beyond worst-case synthesis.} Our presentation is mostly based on~\cite{Bruyere&al:2017}, where all technical details can be found. As noted in~\cite{Bruyere&al:2017}, allowing large inequalities in the BWC problem may require infinite-memory strategies. The case of infinite-memory strategies was studied in~\cite{Clemente&Raskin:2015} along with multidimension BWC mean-payoff problems. BWC problems were studied for other objectives, such as shortest path~\cite{Bruyere&al:2017} or parity~\cite{Berthon&Randour&Raskin:2017}; and on other related models (e.g.,~\cite{Brazdil&Kucera&Novotny:2016,Almagor&Kupferman&Velner:2016}). BWC principles have been implemented in the tool \textsc{Uppaal}~\cite{David&al:2014}.

Comparisons with other rich behavioural models can be found in~\cite{Randour&Raskin&Sankur:2015,Brenguier&al:2016}.