%\knowledge{vector addition system with states}
  [Vector addition systems with states|vector addition systems with states]
  {notion,index={vector!addition system with states}}
\knowledge{Minsky machine}[Minsky machines]
  {notion,index={Minsky machine}}
\knowledge{deterministic Minsky machine}[deterministic Minsky machines]
  {notion,index={Minsky machine!deterministic}}
\knowledge{alternating Minsky machine}[alternating Minsky machines]
  {notion,index={Minsky machine!alternating}}
\knowledge{halting problem}{notion,index={Minsky machine!halting problem}}
\knowledge{vector game}
  [Vector games|vector games]
  {notion,index={vector!game}}
\knowledge{Petri net}[Petri nets]{notion,index={Petri net}}
\knowledge{one-counter game}[one-counter games|One-counter games]
  {notion,index={one-counter game}}
\knowledge{succinct}[succinct one-counter game|succinct one-counter games|Succinct one-counter games]
  {notion,index={one-counter game!succinct}}
\knowledge{countdown game}[Countdown games|countdown games]
  {notion,index={countdown!game}}
\knowledge{zero reachability}[Zero reachability]
  {notion,index={countdown!game!zero reachability}}
\knowledge{countdown system}[Countdown systems|countdown systems]
  {notion,index={countdown!system}}
\knowledge{vector system}
  [vector systems|Vector systems]
  {notion,index={vector!system}}
\knowledge{natural semantics}
  {notion,index={natural semantics}}
\knowledge{energy semantics}[Energy semantics]
  {notion,index={energy!semantics}}
\knowledge{integer semantics}
  {notion}
\knowledge{bounded semantics}
  {notion,index={bounded semantics}}
\knowledge{capped semantics}
  {notion,index={capped semantics}}
\knowledge{bounding game}[bounding games|Bounding games]
  {notion,index={bounding game}}
\knowledge{capping game}[capping games|Capping games]
  {notion,index={capping game}}
\knowledge{sink}
  {notion}
\knowledge{total}{notion}
\knowledge{configuration colouring}[configuration
  colourings|Configuration
  colourings]{notion,index={colouring!configuration}}
\knowledge{location colouring}[location colourings|Location
  colourings]{notion,index={colouring!location}}
\knowledge{configuration reachability}[Configuration
reachability|configuration reachability vector game]
                    {notion,index={vector!game!configuration
  reachability}}
\index{configuration reachability|see{vector game}}
\knowledge{coverability}[Coverability|coverability game|coverability
  vector game]{notion,index={vector!game!coverability}}
\index{coverability|see{vector game}}
\knowledge{parity vector game}[Parity vector games|parity vector games]
                    {notion,index={vector!game!parity}}
\knowledge{non-termination}[Non-termination|non-termination
  game|non-termination games|non-termination vector game]
                    {notion,index={vector!game!non-termination}}
\index{non-termination|see{vector game}}
\knowledge{given initial credit}[given]
                    {notion,index={initial credit!given}}
\knowledge{existential initial credit}[existential]
                    {notion,index={initial credit!existential}}
\knowledge{zero vector}{notion}
\knowledge{unit vector}[unit vectors]{notion}
\knowledge{infinity norm}[norm|norms]{notion}
\knowledge{zero test}[zero tests]{notion}
\knowledge{Pareto limit}[Pareto limits]{notion,index={Pareto!limit}}
\knowledge{well-quasi-order}[wqo|wqos|well-quasi-orders|Well-quasi-orders|well-quasi-ordered]
  {notion,index={well-quasi-order}}
\knowledge{energy game}[energy games|Energy
  games]{notion,index={energy!game}}
\knowledge{asymmetry}[asymmetric|Asymmetry|Asymmetric|asymmetric
vector system|asymmetric vector systems|Asymmetric vector
  systems|asymmetric vector game|asymmetric vector games|Asymmetric
  vector games|asymmetric games]
  {notion,index={vector!system!asymmetric}}
\knowledge{monotonic objective}[monotonic|monotonic objectives|Monotone
  objectives|monotonic vector game|monotonic vector games|Monotone
  vector games]
  {notion,index={vector!game!monotonic}}
\knowledge{hit-or-run game}[hit-or-run games]
  {notion,index={hit-or-run game}}
\knowledge{robot game}[robot games]
  {notion,index={robot game}}
\knowledge{quasi-order}[quasi-orders]
  {notion}
\knowledge{upward closure}[upward closures]
  {notion}
\knowledge{upwards closed}{notion}
\knowledge{downward closure}[downward closures]
  {notion}
\knowledge{downwards closed}{notion}
\knowledge{principal filter}[principal filters|filters|filter]{notion}
\knowledge{principal ideal}[principal ideals]{notion}
\knowledge{well-founded}{notion}
\knowledge{finite antichain condition}{notion}
\knowledge{ascending chain condition}{notion}
\knowledge{finite basis property}{notion}
%\knowledge{Pareto bound}[Pareto bounds]{notion,index={Pareto!bound}}
%\knowledge{viable}{notion}
\knowledge{bounding game}[bounding|Bounding games|bounding
games]{notion,index={bounding game}}
\knowledge{good sequence}[good]{notion}
\knowledge{self-covering tree}[self-covering trees|Self-covering trees]{notion}
\knowledge{return node}{notion}
%\knowledge{correct}{notion}
\knowledge{counterless}[counterless strategy|counterless
  strategies|Counterless
  strategies]{notion,index={strategy!counterless}}
\knowledge{simulate}[simulating|simulates]
  {notion}



%*** General probabilistic notation ***

\newcommand{\expv}{\mathbf{E}} % EXP. VALUE
\newcommand{\discProbDist}{f} % Discrete prob distribution
\newcommand{\sampleSpace}{S} % Generic sample space
\newcommand{\sigmaAlg}{\mathcal{F}} % Generic sigma-algebra
\newcommand{\probm}{\mathbb{P}} % Generic probability measure, also prob. measure operator
\newcommand{\rvar}{X} % Generic random variable
%\newcommand{\dist}{\mathit{Dist}}

%*** MDP notation ***

\newcommand{\actions}{A} % The set of actions.
\newcommand{\colouring}{c} % the colouring function
\newcommand{\probTranFunc}{\Delta} % Transition function of an MDP
\newcommand{\edges}{E} % Set of edges in an MDP.
\newcommand{\colours}{C} % The set of colours in an MDP.
\newcommand{\mdp}{\mathcal{M}} % A generic MDP. 
\newcommand{\vinit}{v_0} % An initial vertex in an MDP.
\newcommand{\cylProb}{p} % Function assigning probabilities to cylinder sets in 
%the measure construction.
\newcommand{\emptyPlay}{\epsilon} %empty play
\newcommand{\objective}{\Omega} % Qualitative objective
\newcommand{\genColour}{\textsc{c}} % Generic colour
\newcommand{\quantObj}{f} % Generic quantitative objective
\newcommand{\indicator}[1]{\mathbf{1}_{#1}} % In.d RV
\newcommand{\eps}{\varepsilon} % Numerical epsilon
\newcommand{\maxc}{W} % Maximal abs. value of a colour

\newcommand{\winPos}{W_{>0}}
\newcommand{\winAS}{W_{=1}}
\newcommand{\cylinder}{\mathit{Cyl}}

\newcommand{\PrePos}{\text{Pre}_{>0}}
\newcommand{\PreAS}{\text{Pre}_{=1}}

\newcommand{\PreOPPos}{\mathcal{P}_{>0}}
\newcommand{\OPAS}{\mathcal{P}_{=1}}

\newcommand{\safeOP}{\mathit{Safe_{=1}}}
\newcommand{\closed}{\mathit{Cl}}

\newcommand{\reachOP}{\mathcal{V}}
\newcommand{\discOP}{\mathcal{D}}
\newcommand{\valsigma}{\vec{x}^{\sigma}}

\newcommand{\lp}{\mathcal{L}}
\newcommand{\lpdisc}{\lp_{\mathit{disc}}}
\newcommand{\lpmp}{\lp_{\mathit{mp}}}
\newcommand{\lpsol}[1]{\bar{#1}}
\newcommand{\lpmpdual}{\lpmp^{\mathit{dual}}}

\newcommand{\actevent}[3]{\actions^{#1}_{#2,#3}} % Returns #1-th action on the run 

\newcommand{\MeanPayoffSup}{\MeanPayoff^{+}}
\newcommand{\MeanPayoffInf}{\MeanPayoff^{-}}

\newcommand{\mcprob}{M}
\newcommand{\invdist}{\vec{z}}

\newcommand{\hittime}{T}



Up to this chapter, we have mostly been interested in finding strategies that achieve a \emph{single} objective or optimise a \emph{single} payoff function. Our goal here is to discuss what happens when one goes further and wants to build strategies that (i) ensure \emph{several objectives}, or (ii) provide \emph{richer guarantees} than the simple worst-case or expectation ones used respectively in zero-sum games and MDPs. 

Consider case (i). Such requirements arise naturally in applications: for instance, one may want to define a trade-off between the performance of a system and its energy consumption. A model of choice for this is the natural \textit{multidimension} extension of the games of~\cref{4-chap:payoffs}, where we consider weight vectors on edges and combinations of objectives.

In case (ii), we base our study on stochastic models such as MDPs (\cref{5-chap:mdp}). We will notably present how to devise controllers that provide strong guarantees in a worst-case scenario while behaving efficiently on average (based on a stochastic model of its environment built through statistical observations); effectively reconciling the rational antagonistic behaviour of Adam, used in games, with the stochastic interpretation of uncontrollable interaction at the core of MDPs.

Stepping into the ""multiobjective"" world is like entering a jungle: the sights are amazing but the wildlife is overwhelming. Providing an exhaustive account of existing multiobjective models and the latest developments in their study is a task doomed to fail: simply consider the combinatorial explosion of all the possible combinations based on the already non-exhaustive set of games studied in the previous chapters. Hence, our goal here is to guide the reader through his first steps in the jungle, highlighting the specific dangers and challenges of the multiobjective landscape, and displaying some techniques to deal with them. To that end, we focus on models studied in~\cref{2-chap:regular},~\cref{4-chap:payoffs},~\cref{5-chap:mdp} and~\cref{11-chap:counters}, and multiobjective settings that extend them. We favour simple, natural classes of problems, that already suffice to grasp the cornerstones of multiobjective reasoning. 

\paragraph{Chapter outline} In~\cref{12-sec:multiple_dimensions}, we illustrate the additional complexity of multiobjective games and how relations between different classes of games that hold in the single objective case often break as soon as we consider combinations of objectives. 

The next two sections are devoted to the \emph{simplest form} of multiobjective games: games with \emph{conjunctions} of classical objectives. In~\cref{12-sec:mean_payoff_energy}, we present the classical case of multidimension "mean-payoff" and "energy" games, which preserve relatively nice properties with regard to their single-objective counterparts. In~\cref{12-sec:total_payoff_shortest_path}, we discuss the opposite situation of "total-payoff" and "shortest path" games: their nice single-objective behaviour vanishes here.

In the last two sections, we explore a different meaning of \emph{multiobjective} through so-called ""rich behavioural models"". Our quest here is to find strategies that provide several types of guarantees, of different nature, for the same quantitative objective. In~\cref{12-sec:beyond_worst_case}, we address the problem of \emph{"beyond worst-case synthesis"}, which combines the rational antagonistic interpretation of "two-player zero-sum games" with the stochastic nature of "MDPs". We will study the mean-payoff setting and see how to construct strategies that ensure a strict worst-case constraint while providing the highest expected value possible.
In~\cref{12-sec:percentile}, we briefly present \emph{"percentile queries"}, which extend \textit{probability threshold problems} in MDPs to their multidimension counterparts. \todo{Check what I do exactly.} Interestingly, \emph{"randomized strategies"} become needed in this context, whereas up to~\cref{12-sec:percentile}, we only consider "deterministic" strategies as they suffice.

We close the chapter with the usual bibliographic discussion and pointers towards some of the many recent advances in multiobjective reasoning.

\section{From one to multiple dimensions}
\label{12-sec:multiple_dimensions}
\input{multiple_dimensions}

\section{Mean-payoff and energy}
\label{12-sec:mean_payoff_energy}
\input{mean_payoff_energy}

\section{Total-payoff and shortest path}
\label{12-sec:total_payoff_shortest_path}
In this section, we turn to two other objectives deeply studied in~\cref{chap:payoffs}: we study "total-payoff" and "shortest path" games. We will see that the multidimension setting has dire consequences for both.

\subsection{Total-payoff vs.~mean-payoff}

We start with total-payoff games. As for the mean-payoff, we explicitly consider the two variants, $\TotalPayoff^+$ and $\TotalPayoff^-$, for the lim-sup and lim-inf definitions respectively. While~\cref{chap:payoffs} was written using the lim-sup variant, all results are identical for the lim-inf one in one-dimension games~\cite{Gawlitza&Seidl:2009}.

Recall that one-dimension total-payoff games are memoryless determined and solving them is in $\NP \cap \coNP$ (even in $\UP \cap \coUP$~\cite{Gawlitza&Seidl:2009})\todo{Could be moved in~\cref{chap:payoffs}}. Furthermore, \cref{chap:payoffs} taught us that total-payoff can be seen as a \textit{refinement} of mean-payoff, as it permits to reason about low (using the lim-inf variant) and high (using the lim-sup one) points of partial sums along a play when the mean-payoff is zero. We formalize this relationship in the next lemma, and study what happens in multiple dimensions. 

\begin{lemma}
\label{12-lem:MPTP}
Fix an arena $\arena$ and an initial vertex $v_0 \in \vertices$. Let A, B, C and D denote the following assertions.
\begin{enumerate}
\item[A.] Eve has a winning strategy for $\MeanPayoff^{+}_{\geq \vec{0}}$.
\item[B.] Eve has a winning strategy for $\MeanPayoff^{-}_{\geq \vec{0}}$.
\item[C.] There exists $\vec{x} \in \Q^{k}$ such that Eve has a winning strategy for $\TotalPayoff^{-}_{\geq \vec{x}}$.
\item[D.] There exists $\vec{x} \in \Q^{k}$ such that Eve has a winning strategy for $\TotalPayoff^{+}_{\geq \vec{x}}$.
\end{enumerate}
In one-dimension games ($k = 1$), all four assertions are equivalent. In multidimension ones ($k > 1$), the only implications that hold are: $C \implies D \implies A$ and $C \implies B \implies A$. All other implications are false in general.
\end{lemma}

\cref{12-lem:MPTP} is depicted in~\cref{12-fig:MPTP}: the only implications that carry over to multiple dimensions are depicted by solid arrows.

\begin{figure}[thb]
\centering
\scalebox{1}{\begin{tikzpicture}[dash pattern=on 10pt off 5,->,>=stealth',double,double distance=2pt,shorten >=1pt,auto,node
    distance=2.5cm,bend angle=45,scale=0.6,font=\normalsize]
    \tikzstyle{p1}=[]
    \tikzstyle{p2}=[draw,rectangle,text centered,minimum size=7mm]
    \node[p1]  (A)  at (-0.5, 0) {$A\colon\:\exists\,\sigma_{A} \models \MeanPayoff^{+}_{\geq \vec{0}}$};
    \node[p1]  (D) at (12.5, 0) {$D\colon\:\exists\, \vec{x} \in \Q^{k},\, \exists\,\sigma_D \models \TotalPayoff^{+}_{\geq \vec{x}}$};
    \node[p1]  (B) at (-0.5, -4) {$B\colon\:\exists\,\sigma_{B} \models \MeanPayoff^{-}_{\geq \vec{0}}$};
    \node[p1]  (C) at (12.5, -4) {$C\colon\:\exists\, \vec{x} \in \Q^{k},\, \exists\,\sigma_C \models \TotalPayoff^{-}_{\geq \vec{x}}$};
    \path
    ;
	\draw[dashed,dash phase =4pt,->,>=stealth,thin,double,double distance=1.5pt] (5.5,0) to (7,0);
	\draw[<-,>=stealth,thin,double,double distance=1.5pt,solid] (4,0) to (5.5,0);
	\draw[dashed,dash phase =4pt,->,>=stealth,thin,double,double distance=1.5pt] (5.5,-4) to (7,-4);
	\draw[<-,>=stealth,thin,double,double distance=1.5pt,solid] (4,-4) to (5.5,-4);
	\draw[<-,>=stealth,thin,double,double distance=1.5pt,solid] (0,-1) to (0,-2);
	\draw[dashed,dash phase =4pt,->,>=stealth,thin,double,double distance=1.5pt] (0,-2) to (0,-3);
	\draw[<-,>=stealth,thin,double,double distance=1.5pt,solid] (12,-1) to (12,-2);
	\draw[dashed,dash phase =4pt,->,>=stealth,thin,double,double distance=1.5pt] (12,-2) to (12,-3);
	\draw[<-,>=stealth,thin,double,double distance=1.5pt,solid] (3,-1) to (5.5,-2);
	\draw[dashed,dash phase =4pt,->,>=stealth,thin,double,double distance=1.5pt] (5.5,-2) to (8,-3);
	\draw[dashed,dash phase =4pt,<-,>=stealth,thin,double,double distance=1.5pt] (3,-3) to (5.5,-2);
	\draw[dashed,dash phase =4pt,->,>=stealth,thin,double,double distance=1.5pt] (5.5,-2) to (8,-1);
\end{tikzpicture}}
\caption{Equivalence between mean-payoff and total-payoff games. Dashed im\-pli\-ca\-tions are only valid in one-dimension games. We use $\sigma \models \Omega$ as a shortcut for ``$\sigma$ is winning from $v_0$ for $\Omega$''.}
\label{12-fig:MPTP}
\end{figure}

\begin{proof}
The implications that remain true in multiple dimensions are the trivial ones. First, satisfaction of the lim-inf version of a given objective clearly implies satisfaction of its lim-sup version by definition. Hence, $B \implies A$ and $C \implies D$. Second, consider a play $\pi \in \TotalPayoff^{-}_{\geq \vec{x}}$ (resp.~$\TotalPayoff^{+}_{\geq \vec{x}}$) for some $\vec{x} \in \Q^{k}$. For all dimension $i \in \{1, \ldots{}, k\}$, the corresponding sequence of mean-payoff infima (resp.~suprema) over prefixes can be \textit{lower-bounded} by a sequence of elements of the form $\frac{\vec{x}_i}{\ell}$ with $\ell$ the length of the prefix. We can do this because the sequence of total-payoffs over prefixes is a sequence of integers: it always achieves the value of its limit $\vec{x}_i$ instead of only tending to it asymptotically as could a sequence of rationals (such as the mean-payoffs). Since $\frac{\vec{x}_i}{\ell}$ tends to zero over an infinite play, we do have that $\pi \in \MeanPayoff^{-}_{\geq \vec{x}}$ (resp.~$\MeanPayoff^{+}_{\geq \vec{x}}$). Thus, $C \implies B$ and $D \implies A$. Along with the transitive closure $C \implies A$, these are all the implications preserved in multidimension games.


In one-dimension games, all assertions are equivalent. As seen before, we have that lim-inf and lim-sup mean-payoff games coincide as memoryless strategies suffice for both players. Thus, we add $A \implies B$, and $D \implies B$ by transitivity. Second, consider a memoryless (w.l.o.g.) strategy $\sigma_B$ for Eve for $\MeanPayoff^{-}_{\geq \vec{0}}$. Let $\play$ be any consistent play. Then all cycles in $\pi$ are non-negative, otherwise Eve cannot ensure winning with $\sigma_B$ (because Adam could pump the negative cycle). Thus, the sum of weights along $\play$ is at all times bounded from below by $-(\vert V\vert-1)\cdot W$ (for the longest acyclic prefix), with $W$ the largest absolute weight, as usual. Therefore, we have $B \implies C$, and we obtain all other implications by transitive closure.

For multidimension games, all dashed implications are false. We only need to consider two of them.
\begin{enumerate}
\item\label{12-lem:MPTP_proof1} To show that implication $D \implies B$ does not hold, consider the Eve-owned one-player game where $V = \{v\}$ and the only edges are two self loops of weights $(1, -2)$ and $(-2, 1)$. Clearly, any finite vector $\vec{x} \in \Q^{2}$ for $\TotalPayoff^{+}_{\geq \vec{x}}$ can be achieved by an infinite-memory strategy consisting in playing both loops successively for longer and longer periods, each time switching after getting back above threshold $\vec{x}$ in the considered dimension. However, it is impossible to build any strategy, even with infinite memory, that satisfies $\MeanPayoff^{-}_{\geq \vec{0}}$ as the lim-inf mean-payoff would be at best a linear combination of the two cycle values, i.e., strictly less than zero in at least one dimension in any case.
\item Lastly, consider the game in~\cref{12-fig:MultiMP} where we modify the weights to add a third dimension with value $0$ on the self loops and $-1$ on the other edges. As already proved, the strategy that plays for $\ell$ steps in the left cycle, then goes for $\ell$ steps in the right one, then repeats for $\ell' > \ell$ and so on, is a winning strategy for $\MeanPayoff^{-}_{\geq \vec{0}}$. Nevertheless, for any strategy of Eve, the play is such that either (i) it only switches between $v_0$ and $v_1$ a finite number of times, in which case the sum in dimension $1$ or $2$ decreases to infinity from some point on; or (ii) it switches infinitely often and the sum in dimension $3$ decreases to infinity. In both cases, objective $\TotalPayoff^{+}_{\geq \vec{x}}$ is not satisfied for any finite vector $\vec{x} \in  \Q^{3}$. Hence, $B \implies D$ is falsified.
\end{enumerate}
All other implications are deduced false as they would otherwise contradict the last two cases by transitivity.
\end{proof}

We see that the relationship between mean-payoff and total-payoff games breaks in multiple dimensions. Nonetheless, one may still hope for good properties for the latter, as one-dimension total-payoff games are in $\NP \cap \coNP$ (\cref{chap:payoffs}). \todo{I need label for Subsect. 4.4.3.} This hope, however, will not last long.

\subsection{Undecidability}

In contrast to mean-payoff games, total-payoff ones become undecidable in multiple dimensions.

\begin{theorem}
\label{12-thm:TPundec}
Total-payoff games are undecidable in any dimension $k \geq 5$.
\end{theorem}

\begin{proof}
We use a reduction from two-dimensional robot games~\cite{Niskanen&Potapov&Reichert:2016}, which were mentioned in~\cref{chap:counters}. They are a restricted case of "configuration reachability" "vector games", recently proved to be already undecidable. They are expressible as follows: $\mathcal{V} = (\mathcal{L} = \{\ell_0, \ell_1\}, T, \mathcal{L}_{\text{Eve}} = \{\ell_0\}, \mathcal{L}_{\text{Adam}} = \{\ell_1\})$ and $T \subseteq \mathcal{L} \times [-M, M]^2\times \mathcal{L}$ for some $M \in \N$. The game starts in configuration $\ell_0(x_0, y_0)$ for some $x_0, y_0 \in \Z$ and the goal of Eve is to reach configuration $\ell_0(0, 0)$.

The reduction is as follows. Given a robot game $\mathcal{V}$, we build a five-dimension total-payoff game $\game$ such that Eve wins in $\game$ if and only if she wins in $\mathcal{V}$. Let $\game = (\arena, \TotalPayoff^{+}_{\geq \vec{0}})$ (we will discuss the lim-inf case later), where arena $\arena$ has vertices $V = V_{\text{Eve}} \uplus V_{\text{Adam}}$ with $V_{\text{Eve}} = \{v_{\text{init}}, v_0, v_{\text{stop}}\}$ and $V_{\text{Adam}} = \{v_1\}$, and $E$ is built as follows:
\begin{itemize}
\item if $(\ell_i, (a,b), \ell_j) \in T$, then $(v_i, (a, -a, b, -b, 0), v_j) \in E$,
\item $(v_0, (0, 0, 0, 0, 1), v_{\text{stop}}) \in E$ and $(v_{\text{stop}}, (0, 0, 0, 0, 0), v_{\text{stop}}) \in E$,
\item $(v_{\text{init}}, (x_0, -x_0, y_0, -y_0, -1), v_0) \in E$ (where $(x_0, y_0)$ is the initial credit in $\mathcal{V}$).
\end{itemize}
The initial vertex is $v_{\text{init}}$. Intuitively, dimensions $1$ and $2$ (resp.~$3$ and $4$) encode the value of the first counter (resp.~second counter) and its opposite at all times. The initial credit is encoded thanks to the initial edge, afterwards the game is played as in the vector game, with the exception that Eve may branch from $v_0$ to the absorbing vertex $v_{\text{stop}}$, which has a zero self loop. The role of the last dimension is to force Eve to branch eventually.

We proceed to prove the correctness of the reduction. First, let $\sigma_{\game}$ be a winning strategy of Eve in $\game$. We claim that Eve can also win in $\mathcal{V}$. Any play $\pi$ consistent with $\sigma_{\game}$ necessarily ends in $v_{\text{stop}}$: otherwise its lim-sup total-payoff on the last dimension would be $-1$ (as the sum always stays at $-1$). Due to the branching edge and the self loop having weight zero in all first four dimensions, we also have that the current sum on these dimensions must be non-negative when branching, otherwise the objective would be falsified. By construction of $\arena$, the only way to achieve this is to have a sum exactly equal to zero in all first four dimensions (as dimensions $1$ and $2$ are opposite at all times and so are $3$ and $4$). Finally, observe that obtaining a partial sum of $(0, 0, 0, 0, -1)$ in $v_0$ is equivalent to reaching configuration $\ell_0(0, 0)$ in $\mathcal{V}$. Hence, we can easily build a strategy $\sigma_{\mathcal{V}}$ in $\mathcal{V}$ that mimics $\sigma_{\game}$ in order to win the robot game. This strategy $\sigma_{\mathcal{V}}$ could in general use arbitrary memory (since we start with an arbitrary winning strategy $\sigma_{\game}$) while formally robot games as defined in~\cite{Niskanen&Potapov&Reichert:2016} only allow strategies to look at the current configuration. Still, from $\sigma_{\mathcal{V}}$, one can easily build a corresponding strategy that meets this restriction ($\mathcal{V}$ being a configuration reachability game, there is no reason to choose different actions in two visits of the same configuration). Hence, if Eve wins in $\game$, she also wins in $\mathcal{V}$.

For the other direction, from a winning strategy $\sigma_{\mathcal{V}}$ in $\mathcal{V}$, we can similarly define a strategy $\sigma_{\game}$ that mimics it in $\game$ to reach $v_0$ with partial sum $(0, 0, 0, 0, -1)$, and at that point, branches to $v_{\text{stop}}$. Such a strategy ensures reaching the absorbing vertex with a total-payoff of zero in all dimensions, hence is winning in $\game$.

Thus, the reduction holds for lim-sup total-payoff. Observe that the exact same reasoning holds for the lim-inf variant. Indeed, the last dimension is always $-1$ outside of $v_{\text{stop}}$, hence any play not entering $v_{\text{stop}}$ also has its lim-inf below zero in this dimension. Furthermore, once $v_{\text{stop}}$ is entered, the sum in all dimensions stays constant, hence the limit exists and both variants coincide.
\end{proof}

An almost identical reduction can be used for "\textit{shortest path}" games.

\begin{theorem}
\label{12-thm:SPundec}
Shortest path games are undecidable in any dimension $k \geq 4$.
\end{theorem}

\begin{proof}
The proof is almost identical to the last one. We use objective $\ShortestPath_{\leq \vec{0}}$ with target edge $(v_{\text{stop}}, (0, 0, 0, 0), v_{\text{stop}})$ and drop the last dimension in arena $\arena$: it is now unnecessary as the shortest path objective by definition will force Eve to branch to $v_{\text{stop}}$, as otherwise the value of the play would be infinite in all dimensions. The rest of the reasoning is the same as before.
\end{proof}

\begin{remark}
The decidability of total-payoff games with $k \in \{2, 3, 4\}$ dimensions and shortest path games with $k \in \{2, 3\}$ dimensions remains an open question. Furthermore, our undecidability results crucially rely on weights being in $\Z$: they do not hold when we restrict weights to $\N$. A similar situation will be presented in~\cref{12-sec:percentile}, in the context of percentile queries.\todo{Check if enough place to do it.}
\end{remark}

\subsubsection*{Memory} Let us go back to the game used in \cref{12-lem:MPTP_proof1} in the proof of~\cref{12-lem:MPTP}: we have seen that for any threshold $\vec{x} \in \Q^{2}$, Eve has an infinite-memory winning strategy for $\TotalPayoff^{+}_{\geq \vec{x}}$. In other words, she can ensure an \textit{arbitrarily high} total-payoff with infinite memory. Yet, it is easy to check that there exists no finite-memory strategy of Eve that can achieve a finite threshold vector in the very same game: alternating would still be needed, but the negative amount to compensate grows boundlessly with each alternation, thus no amount of finite memory can ensure to go above the threshold infinitely often. Hence, this simple game highlights a huge gap between finite and infinite memory: with finite memory, the total-payoff on at least one dimension is $-\infty$; with infinite memory, the total-payoff in both dimensions may be as high as Eve wants. This further highlights the untameable behaviour of multidimension total-payoff games.

\subsubsection*{Wrap-up} Multiple dimensions are a curse for total-payoff and shortest path games as both become undecidable. This is in stark contrast to mean-payoff and energy games, which remain tractable, as seen in~\cref{12-sec:mean_payoff_energy}. The bottom line is that most of the equivalences, relationships, and well-known behaviours of one-dimension games simply fall apart when lifting them to multiple dimensions.


\section{Beyond worst-case synthesis}
\label{12-sec:beyond_worst_case}
\input{beyond_worst_case}

\section{Percentile queries}
\label{12-sec:percentile}
TODO discuss:
\begin{itemize}
\item decidable (compare with situation in $\Z$)
\item need for randomness
\item Discounted?
\item example of infinite Pareto frontier (simply two reachability sets in two absorbing vertices)
\end{itemize}


\section*{Bibliographic references}
\label{12-sec:references}
As discussed in the introduction, the literature on multiobjective models is too vast to provide a full account here. We therefore focus on some directions particularly relevant to our focus.

\paragraph{Multidimension games.} Energy games and their related work were discussed in~\cref{chap:counters}. Our presentation of mean-payoff games is inspired by Velner et al.~\cite{Velner&al:2015}. Brenguier and Raskin studied the Pareto curves of these games in~\cite{Brenguier&Raskin:2015}. While we considered \textit{conjunctions} of mean-payoff objectives, Velner proved that Boolean combinations lead to undecidability~\cite{Velner:2015}.

The undecidability of total-payoff games was first established in~\cite{Chatterjee&al:2015} via reduction from the halting problem for two-counter machines: we provided here a new, simpler proof based on robot games~\cite{Niskanen&Potapov&Reichert:2016}. This undecidability result, along with the complexity barriers of mean-payoff and total-payoff games, motivated the introduction of (multidimension) ""\textit{window objectives}"": conservative variants of mean-payoff and total-payoff objectives that benefit from increased tractability and permit to reason about time bounds~\cite{Chatterjee&al:2015}. Window variants of "parity" objectives have been studied in~\cite{Bruyere&Hautem&Randour:2016}.

\paragraph{Combinations of different objectives.} We focused on multidimension games obtained by conjunction of \textit{identical} objectives. Conjunctions of \textit{heterogeneous} objectives have been studied in a variety of contexts including mean-payoff parity games~\cite{Chatterjee&Henzinger&Jurdzinski:2005,Daviaud&Jurdzinski&Lazic:2018}, energy parity games~\cite{Chatterjee&Doyen:2012,Chatterjee&Randour&Raskin:2014}, average-energy games with energy constraints~\cite{Bouyer&al:2018,Bouyer&al:2017}, simple quantitative objectives~\cite{Bruyere&Hautem&Raskin:2016}. Le Roux, Pauly and Randour studied general conditions under which finite-memory strategies suffice to play optimally, even in a broad multi-objective setting~\cite{LeRoux&Pauly&Randour:2018}.


\paragraph{Beyond worst-case synthesis.} Our presentation is mostly based on~\cite{Bruyere&al:2017}, where all technical details can be found. As noted in~\cite{Bruyere&al:2017}, allowing large inequalities in the BWC problem may require infinite-memory strategies. The case of infinite-memory strategies was studied in~\cite{Clemente&Raskin:2015} along with multidimension BWC mean-payoff problems. BWC problems were studied for other objectives, such as shortest path~\cite{Bruyere&al:2017} or parity~\cite{Berthon&Randour&Raskin:2017}; and on other related models (e.g.,~\cite{Brazdil&Kucera&Novotny:2016,Almagor&Kupferman&Velner:2016}). BWC principles have been implemented in the tool \textsc{Uppaal}~\cite{David&al:2014}.

Comparisons with other rich behavioural models can be found in~\cite{Randour&Raskin&Sankur:2015,Brenguier&al:2016}.
