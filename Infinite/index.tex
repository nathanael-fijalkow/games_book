

%*** General probabilistic notation ***

\newcommand{\expv}{\mathbf{E}} % EXP. VALUE
\newcommand{\discProbDist}{f} % Discrete prob distribution
\newcommand{\sampleSpace}{S} % Generic sample space
\newcommand{\sigmaAlg}{\mathcal{F}} % Generic sigma-algebra
\newcommand{\probm}{\mathbb{P}} % Generic probability measure, also prob. measure operator
\newcommand{\rvar}{X} % Generic random variable
%\newcommand{\dist}{\mathit{Dist}}

%*** MDP notation ***

\newcommand{\actions}{A} % The set of actions.
\newcommand{\colouring}{c} % the colouring function
\newcommand{\probTranFunc}{\Delta} % Transition function of an MDP
\newcommand{\edges}{E} % Set of edges in an MDP.
\newcommand{\colours}{C} % The set of colours in an MDP.
\newcommand{\mdp}{\mathcal{M}} % A generic MDP. 
\newcommand{\vinit}{v_0} % An initial vertex in an MDP.
\newcommand{\cylProb}{p} % Function assigning probabilities to cylinder sets in 
%the measure construction.
\newcommand{\emptyPlay}{\epsilon} %empty play
\newcommand{\objective}{\Omega} % Qualitative objective
\newcommand{\genColour}{\textsc{c}} % Generic colour
\newcommand{\quantObj}{f} % Generic quantitative objective
\newcommand{\indicator}[1]{\mathbf{1}_{#1}} % In.d RV
\newcommand{\eps}{\varepsilon} % Numerical epsilon
\newcommand{\maxc}{W} % Maximal abs. value of a colour

\newcommand{\winPos}{W_{>0}}
\newcommand{\winAS}{W_{=1}}
\newcommand{\cylinder}{\mathit{Cyl}}

\newcommand{\PrePos}{\text{Pre}_{>0}}
\newcommand{\PreAS}{\text{Pre}_{=1}}

\newcommand{\PreOPPos}{\mathcal{P}_{>0}}
\newcommand{\OPAS}{\mathcal{P}_{=1}}

\newcommand{\safeOP}{\mathit{Safe_{=1}}}
\newcommand{\closed}{\mathit{Cl}}

\newcommand{\reachOP}{\mathcal{V}}
\newcommand{\discOP}{\mathcal{D}}
\newcommand{\valsigma}{\vec{x}^{\sigma}}

\newcommand{\lp}{\mathcal{L}}
\newcommand{\lpdisc}{\lp_{\mathit{disc}}}
\newcommand{\lpmp}{\lp_{\mathit{mp}}}
\newcommand{\lpsol}[1]{\bar{#1}}
\newcommand{\lpmpdual}{\lpmp^{\mathit{dual}}}

\newcommand{\actevent}[3]{\actions^{#1}_{#2,#3}} % Returns #1-th action on the run 

\newcommand{\MeanPayoffSup}{\MeanPayoff^{+}}
\newcommand{\MeanPayoffInf}{\MeanPayoff^{-}}

\newcommand{\mcprob}{M}
\newcommand{\invdist}{\vec{z}}

\newcommand{\hittime}{T}



\Cref{part:infinite} extends the basic setting to infinite arenas.
Since the end goal is to construct algorithms, we restrict our attention to infinite arenas
which can be described by finite means.
More specifically, we consider infinite arenas which arise from finite system with additional features.
\Cref{chap:timed} introduces clocks to measure time, \Cref{chap:pushdown} enriches the arena with a pushdown stack,
and \Cref{chap:counters} extends arenas with a set of counters.

\subsubsection*{Timed games}

\begin{definition}
A timed system is a tuple $(\Locations,T_{\text{Eve}},T_{\text{Adam}})$ where 
$\Locations$ is a set of locations
and 
\[
T_{\text{Eve}},T_{\text{Adam}} \subseteq \Locations \times \Phi_{\Clocks} \times 2^{\Clocks} \times C \times \Locations
\]
are legal transitions for Eve and Adam.
The interpretation of a transition $(\ell,\phi,R,c,\ell')$ is to go from location $\ell$
to location $\ell'$ if the clocks satisfy the guard condition $\phi$,
having as effect to see the colour $c$ and to reset the clocks in $R$.

A timed system induces a concurrent arena $\arena = (G,\Delta)$ called a timed arena where 
\begin{itemize}
	\item the set of vertices is $V = \Locations \times \R^{\Clocks}$,
	\item the set of coloured edges is 
	\[
	E = \set{(\ell,v),c, \step((\ell,v), d, \delta)) : \delta \in T_{\text{Eve}} \cup T_{\text{Adam}}, d \in \Rp},
	\]
	\item the set of actions is $A = \Rp \times (T_{\text{Eve}} \cup T_{\text{Adam}})$,
	\item the transition function $\Delta : V \times A \times A \to E$ is defined as follows.
	$\Delta((\ell,v), (d,\delta) \in \Rp \times T_{\text{Eve}}, (d',\delta') \in \Rp \times T_{\text{Adam}}$ is
	\[
	\begin{array}{cc}
	\left((\ell,v),c, \step((\ell,v), d, \delta) \right)	& \text{ if } d < d', \\
	\left((\ell,v),c, \step((\ell,v), d', \delta') \right)	& \text{ if } d \ge d'. \\
	\end{array}
	\]
\end{itemize}
\end{definition}

\subsubsection*{Pushdown games}

\begin{definition}
A pushdown system is a tuple $(Q,Q_{\text{Eve}}, Q_{\text{Adam}}, \Gamma,\Delta)$
where $Q$ is a set of control states with $Q = Q_{\text{Eve}} \uplus Q_{\text{Adam}}$, 
$\Gamma$ is the stack alphabet and $\Delta$ is the transition relation.
%where 
%\[
%\Delta \subseteq (Q \times \Gamma^*) \times C \times (Q \times \Gamma^*).
%\]
There is a special stack symbol denoted $\bot$ which does not belong to $\Gamma$;
we let $\Gamma_\bot$ denote the alphabet $\Gamma \cup \set{\bot}$.
There are three kinds of transitions in $\Delta$.
\begin{itemize}
	\item \textbf{push}: $(p,a,c,b,q)$: allowed if the top stack element is $a \in \Gamma_\bot$, 
	the symbol $b \in \Gamma$ is pushed onto the stack.
	\item \textbf{pop}: $(p,c,a,q)$: allowed if the top stack element is $a \in \Gamma$,
	the top stack symbol $a$ is popped from the stack.
	\item \textbf{skip}: $(p,a,c,q)$: allowed if the top stack element is $a \in \Gamma_\bot$, the stack remains unchanged.
\end{itemize}
The symbol $\bot$ is never pushed onto, nor popped from the stack.

A pushdown system induces a pushdown arena $\arena = (G,\VE,\VA)$ called a pushdown arena where
\begin{itemize}
	\item the set of vertices is $V = Q \times \Gamma^* \bot$ with $\VE = Q_{\text{Eve}} \times \Gamma^* \bot$ 
	and $\VA = Q_{\text{Adam}} \times \Gamma^* \bot$,
	\item the set of edges is $E$ induced by $\Delta$:
	for instance for a push transition $(p,a,c,b,q) \in \Delta$, 
	we have $\left((p,a w \bot),c,(q,baw \bot)\right) \in E$ for all words $w$ in $\Gamma^*$.	
\end{itemize}
\end{definition}

\subsubsection*{Counter games}

