\subsubsection{Existential Initial Credit}
\label{12-sub-up-exist}

\paragraph{Counterless Strategies}
Consider a "strategy"~$\tau$ of \Adam in a "vector game".  In all the
games we consider, "uniform" "positional" strategies suffice over the
infinite "arena" $\natural(\?V)=(V,E,\VE,\VA)$: $\tau$ maps vertices
in~$V$ to edges in~$E$.  We call~$\tau$ ""counterless"" if, for all
locations $\loc\in\Loc_\mAdam$ and all vectors
$\vec v,\vec v'\in\+N^\dd$, $\tau(\loc(\vec v))=\tau(\loc(\vec v'))$.
A "counterless" strategy thus only considers the current location of
the play.
\begin{lemma}\label{12-counterless}
  Let $\?V=(\Loc,\Act,\Loc_\mEve,\Loc_\mAdam,\dd)$ be an "asymmetric
  vector system", $\loc_0\in\Loc$ be a location, and
  $\lcol{:}\,\Loc\to\{1,\dots,d\}$ be a location colouring.  If \Adam
  wins from $\loc_0(\vec v)$ for every initial credit~$\vec v$ in the
  "parity@parity vector game" game played over $\?V$ with~$\lcol$, then
  he has a single "counterless strategy" such that he wins from
  $\loc_0(\vec v)$ for every initial credit~$\vec v$.
\end{lemma}
\begin{proof}
  Let $\Act_\mAdam\eqdef\{(\loc\step{\vec
    u}\loc')\in\Act\mid\loc\in\Loc_\mAdam\}$ be the set of actions
  controlled by \Adam.  We assume without loss of generality that
  every location $\loc\in\Loc_\mAdam$ has either one or two outgoing
  actions, thus $|\Loc_\mAdam|\leq|\Act_\mAdam|\leq
  2|\Loc_\mAdam|$.  We proceed by induction over $|\Act_\mAdam|$.  For
  the base case, if $|\Act_\mAdam|=|\Loc_\mAdam|$ then every location
  controlled by \Adam has a single outgoing action, thus any
  strategy for \Adam is trivially "counterless".

  For the induction step, consider some location
  $\hat\loc\in\Loc_\mAdam$ with two outgoing actions
  $a_l\eqdef\hat\loc\step{\vec 0}\loc_l$ and
  $a_r\eqdef\hat\loc\step{\vec 0}\loc_r$.  Let $\?V_l$ and $\?V_r$ be
  the "vector systems" obtained from~$\?V$ by removing
  respectively~$a_r$ and~$a_l$ from~$\Act$, i.e., by using
  $\Act_l\eqdef\Act\setminus\{a_r\}$ and
  $\Act_r\eqdef\Act\setminus\{a_l\}$.  If $\Adam$ wins the
  "parity@parity vector game" game from $\loc(\vec v)$ for every
  initial credit~$\vec v$ in either $\?V_l$ or $\?V_r$, then by
  induction hypothesis he has a "counterless" winning strategy winning
  from $\loc(\vec v)$ for every initial credit~$\vec v$, and the same
  strategy is winning in~$\?V$ from $\loc(\vec v)$ for every initial
  credit~$\vec v$.

  In order to conclude the proof, we show that, if \Adam loses in
  $\?V_l$ from $\loc_0(\vec v_l)$ for some $\vec v_l\in\+N^\dd$ and in
  $\?V_r$ from $\loc_0(\vec v_r)$ for some $\vec v_r\in\+N^\dd$, then
  there exists $\vec v_0\in\+N^\dd$ such that \Eve wins from
  $\loc_0(\vec v_0)$ in~$\?V$.  Let $\sigma_l$ and $\sigma_r$ denote
  \Eve's winning strategies in the two games.  By a slight abuse of
  notations (justified by the fact that we are only interested in a
  few initial vertices), we see plays as sequences of actions and
  strategies as maps $\Act^\ast\to\Act$.\todoquestion{I hope this is not too messy}  Consider the set of
  plays consistent with~$\sigma_r$ starting from $\loc_0(\vec v_r)$.
  If none of those plays visits $\hat\loc$, then $\Eve$ wins in $\?V$
  from $\loc_0(\vec v_r)$ and we conclude.  Otherwise, there is some
  finite prefix~$\hat\pi$ of a play that
  visits~$\hat\loc(\hat{\vec v})$ for some vector
  $\hat{\vec v}=\vec v_r+\weight(\hat\pi)$.  We let
  $\vec v_0\eqdef\vec v_l+\hat{\vec v}$ and show that \Eve wins from
  $\loc_0(\vec v_0)$.

  \begin{scope}\knowledge{mode}{notion}
    We define now a strategy $\sigma$ for $\Eve$ over~$\?V$ that
    switches between applying~$\sigma_l$ and~$\sigma_r$ each time
    $a_r$ is used and switches back each time~$a_l$ is used.  More
    precisely, given a finite or infinite sequence~$\pi$ of actions,
    we decompose $\pi$ as $\pi_1 a_1 \pi_2 a_2 \pi_3\cdots$ where each
    segment $\pi_j\in(\Act\setminus\{a_l,a_r\})^\ast$ does not use
    either~$a_l$ nor~$a_r$ and each $a_j\in\{a_l,a_r\}$.  The
    associated ""mode"" $m(j)\in\{l,r\}$ of a segment~$\pi_j$
    is~$m(1)\eqdef l$ for the initial segment and otherwise
    $m(j)\eqdef l$ if $e_{j-1}=a_l$ and $m(j)\eqdef r$ otherwise.  The
    $l$-subsequence associated with $\pi$ is the sequence of segments
    $\pi(l)\eqdef\pi_{l_1}a_{l_2-1}\pi_{l_2}a_{l_3-1}\pi_{l_3}\cdots$
    with "mode"~$m(l_i)=l$, while the $r$-subsequence is the sequence
    $\pi(r)\eqdef\hat\pi a_{r_1-1}\pi_{r_1}a_{r_2-1}\pi_{r_2}\cdots$
    with "mode"~$m(r_i)=r$ prefixed by~$\hat\pi$.  Then we let
    $\sigma(\pi)\eqdef\sigma_{m}(\pi(m))$ where $m\in\{l,r\}$ is the
    "mode" of the last segment of~$\pi$.

    Consider an infinite play $\pi$ consistent with~$\sigma$ starting
    from~$\loc_0(\vec v_0)$.  Since $\vec v_0\geq\vec v_l$ and
    $\vec v_0\geq \vec v_r+\weight(\hat\pi)$, $\pi(l)$ and $\pi(r)$
    starting from~$\loc_0(\vec v_0)$ are consistent with
    "simulating"---in the sense of \cref{12-fact-mono}---$\sigma_l$
    from $\loc_0(\vec v_l)$ and $\sigma_r$ from $\loc_0(\vec v_r)$.
    Let $\pi'$ be a finite prefix of~$\pi$.  Then
    $\weight(\pi')=\weight(\pi'(l))+\weight(\pi'(r))$ where $\pi'(l)$
    is a prefix of~$\pi(l)$ and $\pi'(r)$ of~$\pi(r)$, thus
    $\weight(\pi'(l))\leq\vec v_l$ and
    $\weight(\pi'(r))\leq\vec v_r+\weight(\hat\pi)$, thus
    $\weight(\pi')\leq\vec v_0$: the play~$\pi$ avoids the "sink".
    Furthermore, the maximal priority seen infinitely often along
    $\pi(l)$ and $\pi(r)$ is even (note that one of~$\pi(l)$
    and~$\pi(r)$ might not be infinite), thus the maximal priority
    seen infinitely often along~$\pi$ is also even.  This shows
    that~$\sigma$ is winning for \Eve from $\loc_0(\vec v_0)$.\todoquestion{Is
    that clear?}
  \end{scope}
\end{proof}

We are going to exploit \cref{12-counterless} in \cref{12-exist-easy}
in order to prove a~\coNP\ upper bound for "asymmetric games" with
"existential initial credit": it suffices in order to decide those
games to guess a "counterless" winning strategy~$\tau$ for \Adam and
check that it is indeed winning by checking that \Eve loses the
one-player game arising from~$\tau$.  This last step requires an
algorithmic result of independent interest.

\paragraph{One-player Case}
Let $\?V=(\Loc,\Act,\dd)$ be a "vector addition system with states",
$\lcol{:}\,\Loc\to\{1,\dots,d\}$ a location colouring, and
$\loc_0\in\Loc$ an initial location.  Then \Eve wins the
"parity@parity vector game" one-player game from~$\loc_0(\vec v_0)$
for some initial credit~$\vec v_0$ if and only if there exists some
location such that
\begin{itemize}
\item $\loc$ is reachable from~$\loc_0$ in the directed graph
  underlying~$\?V$ and
\item there is a cycle~$\pi\in\Act^\ast$ from $\loc$ to itself such
  that $\weight(\pi)\geq 0$ and the maximal priority occurring
  along~$\pi$ is even.
\end{itemize}
Indeed, assume we can find such a location~$\loc$.  Let
$\hat\pi\in\Act^\ast$ be a path from~$\loc_0$ to~$\loc$ and $\vec
v_0(i)\eqdef\max\{\|\weight(\pi')\|\mid\pi'\text{ is a prefix of
}\hat\pi\pi\}$ for all $1\leq i\leq\dd$.  Then $\loc_0(\vec v_0)$ can
reach $\loc(\vec v_0+\weight(\hat\pi))$ in the "natural semantics"
of~$\?V$ by following~$\hat\pi$, and then $\loc(\vec v_0+\vec
W(\hat\pi)+n\weight(\pi))\geq \loc(\vec v_0+\weight(\hat\pi))$ after
$n$~repetitions of the cycle~$\pi$.  The infinite play arising from
this strategy has an even maximal priority.

Conversely, if \Eve wins, then there is a winning play
$\pi\in\Act^\omega$ from $\loc_0(\vec v_0)$ for some $\vec v_0$.
Recall that $(V,{\leq})$ is a "wqo", and we argue as in
\cref{12-fact-finmem} that there is indeed such a location~$\loc$.

\medskip
Therefore, solving one-player "parity vector games" boils down to
determining the existence of a cycle with non-negative effect and even
maximal priority.  We shall use linear programming techniques in order
to check the existence of such a cycle in polynomial
time~\cite{Kosaraju&Sullivan:1988}.

\medskip
\begin{scope}
\knowledge{non-negative}{notion}
\knowledge{multi-cycle}[multi-cycles]{notion}
\knowledge{suitable}{notion}
Let us start with a relaxed problem: we call a
""multi-cycle"" a non-empty finite set of cycles~$\Pi$ and let
$\weight(\Pi)\eqdef\sum_{\pi\in\Pi}\weight(\pi)$ be its weight; we write
$t\in\Pi$ if~$t\in\pi$ for some $\pi\in\Pi$.
Let $M\in 2^{\Act}$ be a set of `mandatory' subsets of actions and
$F\subseteq\Act$ a set of `forbidden' actions.  Then we say that
$\Pi$ is ""non-negative"" if $\weight(\Pi)\geq\vec 0$, and that it is
""suitable"" for~$(M,F)$ if for all $\Act'\in M$ there exists
$t\in\Act'$ such that $t\in\Pi$, and if for all $t\in F$,
$t\not\in\Pi$.  We use the same terminology for a single cycle~$\pi$.

\begin{lemma}\label{12-lem-zmulticycle}
  Let $\?V$ be a "vector addition system with states",
  $M\in 2^{\Act}$, and $F\subseteq\Act$.  We can check in polynomial
  time whether~$\?V$ contains a "non-negative" "multi-cycle"~$\Pi$
  "suitable" for~$(M,F)$.
\end{lemma}
\begin{proof}
  We reduce the problem to solving a linear program.  For a
  location~$\loc$, let
  $\mathrm{in}(\loc)\eqdef\{(\loc'\step{\vec u}\loc)\in\Act\mid
  \loc'\in\Loc\}$
  and
  $\mathrm{out}(\loc)\eqdef\{(\loc\step{\vec u}\loc')\in\Act\mid
  \loc'\in\Loc\}$ be its sets of incoming and outgoing actions.  The
  linear program has a variable $x_a$ for each action $a\in\Act$,
  which represents the number of times the action~$a$ occurs in
  the "multi-cycle".  It consists of the following contraints:
  \begin{align*}
    \forall\loc&\in\Loc,&\sum_{a\in\mathrm{in}(\loc)}x_a&=\sum_{a\in\mathrm{out}(\loc)}x_a\;,\tag{"multi-cycle"}\\
    \forall a&\in\Act,&x_a&\geq 0\;,\tag{non-negative uses}\\
    \forall i&\in\{1,\dots,\dd\},&\sum_{a\in\Act} x_a\cdot\weight(t)(i)&\geq
                                            0\;,\tag{"non-negative" weight}\\
    &&\sum_{a\in\Act}x_a&\geq 0\tag{non empty}\\
    \forall \Act'&\in M,&\sum_{a\in\Act'}x_a&\geq 0\;,\tag{every subset
                                               in~$M$ is used}\\
    \forall a&\in F,&x_a&= 0\;.\tag{no forbidden actions}
  \end{align*}
  As solving a linear program is in polynomial time~\cite{}\todoquestion{agree
  on a ref with \cref{chap:signal}?}, the result follows.
\end{proof}

Of course, what we are aiming for is finding a "non-negative"
\emph{cycle} "suitable" for $(M,F)$ rather than a "multi-cycle".
Let us define for this the relation $\loc\sim\loc'$ over~$\Loc$ if
$\loc=\loc'$ or if there exists a "non-negative" "multi-cycle"~$\Pi$
"suitable" for~$(M,F)$ such that~$\loc$ and~$\loc'$ belong to some
cycle~$\pi\in\Pi$.
\begin{claim}\label{12-cl-sim} The relation~$\sim$ is an equivalence
  relation.\end{claim}
\begin{proof}
  Symmetry and reflexivity are trivial, and if $\loc\sim\loc'$ and
  $\loc'\sim\loc''$ because~$\loc$ and~$\loc'$ appear in some cycle
  $\pi\in\Pi$ and $\loc'$ and~$\loc''$ in some cycle $\pi'\in\Pi'$ for
  two "non-negative" "multi-cycles"~$\Pi$ and~$\Pi'$ "suitable"
  for~$(M,F)$, then up to a circular shift $\pi$ and~$\pi'$ can be
  assumed to start and end with $\loc'$, and then
  $(\Pi\setminus\{\pi\})\cup(\Pi'\setminus\{\pi'\})\cup\{\pi\pi'\}$ is
  also a "non-negative" "multi-cycle" "suitable" for~$(M,F)$.
\end{proof}

Thus~$\sim$ defines a partition~$\Loc/{\sim}$ of~$\Loc$.
In order to find a "non-negative" cycle~$\pi$ "suitable" for~$(M,F)$,
we are going to compute the partition~$\Loc/{\sim}$ of~$\Loc$
according to~$\sim$.  If we obtain a partition with a single
equivalence class, we are done: there exists such a cycle.  Otherwise,
such a cycle if it exists must be included in one of the subsystems
$(P,\Act\cap(P\times\+Z^\dd\times P),\dd)$ induced by the equivalence
classes $P\in\Loc/{\sim}$.  This yields \cref{12-algo-zcycle}, which
assumes that we know how to compute the partition~$\Loc/{\sim}$.  Note
that the depth of the recursion in \cref{12-algo-zcycle} is bounded
by~$|\Loc|$ and that recursive calls operate over disjoint subsets
of~$\Loc$, thus assuming that we can compute the partition in
polynomial time, then \cref{12-algo-zcycle} also works in polynomial
time.

\begin{algorithm}
 \KwData{A "vector addition system with states"
   $\?V=(\Loc,\Act,\dd)$, $M\in 2^\Act$, $F\subseteq\Act$}

\If{$|\Loc|=1$}
  {\If{$\?V$ has a "non-negative" "multi-cycle" "suitable" for~$(M,F)$}
    {\Return{true}}}

$\Loc/{\sim} \leftarrow \mathrm{partition}(\?V,M,F)$ ;

\If{$|\Loc/{\sim}|=1$}{\Return{true}}

\ForEach{$P\in\Loc/{\sim}$}{\If{$\mathrm{cycle}((P,\Act\cap(P\times\+Z^\dd\times
    P),\dd),M,F)$}{\Return{true}}}

\Return{false}
\caption{$\text{cycle}(\?V,M,F)$}
\label{12-algo-zcycle}
\end{algorithm}

It remains to see how to compute the partition $\Loc/{\sim}$. Consider
for this the set of actions
$\Act'\eqdef\{a\mid\exists\Pi\text{ a "non-negative" "multi-cycle"
  "suitable" for $(M,F)$ with $a\in\Pi$}\}$ and
$\?V'=(\Loc',\Act',\dd)$ the subsystem induced by $\Act'$.
\begin{claim}\label{12-cl-part}
  There exists a path from~$\loc$ to~$\loc'$ in $\?V'$
  if and only if $\loc\sim\loc'$.
\end{claim}
\begin{proof}
  If $\loc\sim\loc'$, then either $\loc=\loc'$ and there is an empty
  path, or there exist~$\Pi$ and~$\pi\in\Pi$ such that $\loc$
  and~$\loc'$ belong to~$\pi$ and $\Pi$ is a "non-negative"
  "multi-cycle" "suitable" for $(M,F)$, thus every action of~$\pi$ is
  in~$\Act'$ and there is a path in~$\?V'$.  

  Conversely, if there is a path $\pi\in{\Act'}^\ast$ from~$\loc$
  to~$\loc'$, then $\loc\sim\loc'$ by induction on~$\pi$.  Indeed, if
  $|\pi|=0$ then $\loc=\loc'$.  For the induction step, $\pi=\pi' a$
  with $\pi'\in{\Act'}^\ast$ a path from $\loc$ to $\loc''$ and
  $a=(\loc''\step{\vec u}\loc')\in\Act'$ for some~$\vec u$.  By
  induction hypothesis, $\loc\sim\loc''$ and since $a\in\Act'$,
  $\loc''\sim\loc'$, thus $\loc\sim\loc'$ by transitivity shown
  in~\cref{12-cl-sim}. 
\end{proof}

By \cref{12-cl-part}, the equivalence classes of~$\sim$ are the
strongly connected components of~$\?V'$.  This yields the following
polynomial time algorithm for computing~$\Loc/{\sim}$.

\begin{algorithm}
 \KwData{A "vector addition system with states"
   $\?V=(\Loc,\Act,\dd)$, $M\in 2^\Act$, $F\subseteq\Act$}

$\Act'\leftarrow\emptyset$;

\ForEach{$a\in\Act$}{\If{$\?V$ has a "non-negative" "multi-cycle"
    "suitable"
    for~$(M\cup\{\{a\}\},F)$}{$\Act'\leftarrow\Act'\cup\{a\}$}}

$\?V'\leftarrow \text{subsystem induced by~$\Act'$}$ ;

\Return{$\mathrm{SCC}(\?V')$}
\caption{$\text{partition}(\?V,M,F)$}
\label{12-algo-part}
\end{algorithm}

Together, \cref{12-lem-zmulticycle}
and \cref{12-algo-part,12-algo-zcycle} yield the following.

\begin{lemma}\label{12-lem-zcycle}
  Let $\?V$ be a "vector addition system with states",
  $M\in 2^{\Act}$, and $F\subseteq\Act$.  We can check in polynomial
  time whether~$\?V$ contains a "non-negative" cycle~$\pi$
  "suitable" for~$(M,F)$.
\end{lemma}

Finally, we obtain the desired polynomial time upper bound for
"parity@parity vector games" in "vector addition systems with states".
\begin{theorem}\label{12-thm-zcycle}
  Whether \Eve wins a one-player "parity vector game" with
  "existential initial credit" is in~\P.
\end{theorem}
\begin{proof}
  Let $\?V=(\Loc,\Act,\dd)$ be a "vector addition system with states",
  $\lcol{:}\,\Loc\to\{1,\dots,$ $d\}$ a location colouring, and
  $\loc_0\in\Loc$ an initial location.  We start by trimming~$\?V$ to
  only keep the locations reachable from~$\loc_0$ in the underlying
  directed graph.  Then, for every even priority $p\in\{1,\dots,d\}$,
  we use \cref{12-lem-zcycle} to check for the existence of a
  "non-negative" cycle with maximal priority~$p$: it suffices for this
  to set $M\eqdef\{\lcol^{-1}(p)\}$ and
  $F\eqdef\lcol^{-1}(\{p+1,\dots,d\})$.
\end{proof}
\end{scope}

\paragraph{Upper Bounds}
We are now equipped to prove our upper bounds.  We begin with a nearly
trivial case.  In a "coverability" "asymmetric vector game" with
"existential initial credit", the counters play no role at all: \Eve
has a winning strategy for some initial credit in the "vector game" if
and only if she has one to reach the target location~$\loc_f$ in the
finite game played over~$\Loc$ and edges~$(\loc,\loc')$ whenever
$\loc\step{\vec u}\loc'\in\Act$ for some~$\vec u$.  This entails that
"coverability" "asymmetric vector games" are quite easy to solve.

\begin{theorem}\label{12-cov-exist-P}
  "Coverability" "asymmetric" "vector games" with "existential initial
  credit" are \P-complete.
\end{theorem}

Regarding "non-termination" and "parity@parity vector game", we
exploit \cref{12-counterless,12-thm-zcycle}.

\begin{theorem}\label{12-exist-easy}
  "Non-termination" and "parity@parity vector game" "asymmetric"
  "vector games" with "existential initial credit" are in~\coNP.
\end{theorem}
\begin{proof}
  By \cref{12-nonterm2parity}, it suffices to prove the statement for
  "parity@parity vector games" games.  By \cref{12-counterless},
  if \Adam wins the game, we can guess a "counterless" winning
  strategy~$\tau$ telling which action to choose for every location.
  This strategy yields a one-player game, and by \cref{12-thm-zcycle}
  we can check in polynomial time that~$\tau$ was indeed winning
  for~\Adam.
\end{proof}

Finally, in fixed dimension and with a fixed number of priorities, we
can simply apply the results of \cref{12-bounding}.
\begin{corollary}\label{12-exist-pseudop}
  "Parity@parity vector game" "asymmetric" "vector games" with
  "existential initial credit" are in pseudo-polynomial time if the
  dimension and the number of priorities are fixed.
\end{corollary}
\begin{proof}
  Consider an "asymmetric vector system"
  $\?V=(\Loc,\Act,\Loc_\mEve,\Loc_\mAdam,\dd)$ and a location
  colouring $\lcol{:}\,\Loc\to\{1,\dots,2d\}$.
  By \cref{12-parity2bounding}, the "parity vector game" with
  "existential initial credit" over~$\?V$ problem reduces to a
  "bounding game" with "existential initial credit" over a "vector
  system"~$\?V'=(\Loc',\Act',\Loc'_\mEve,\Loc'_\mAdam,\dd+d)$ where
  $\Loc'\in O(|\Loc|)$ and $\|\Act'\|=\|\Act\|$.
  By \cref{12-th-bounding}, it suffices to consider the case of a
  "non-termination" game with "existential initial credit" played over
  the "bounded semantics" $\bounded(\?V')$ where $B$ is in
  $(|\Loc'|\cdot\|\Act'\|)^{O(\dd+d)^3}$.  Such a game can be solved in
  linear time in the size of the bounded arena using attractor
  techniques, thus in $O(|\Loc|\cdot B)^{\dd+d}$, which is in
  $(|\Loc|\cdot\|\Act\|)^{O(\dd+d)^4}$ in terms of the original instance.
\end{proof}

\subsubsection{Given Initial Credit}
\label{12-sub-up-given}
\TODO{\Cref{12-sub-up-given}}

\begin{theorem}\label{12-avag-easy}
  "Coverability", "non-termination", and "parity@parity vector game"
  "asymmetric" "vector games" with "given initial credit" are in
  \kEXP[2].  If the dimension is fixed, they are in \EXP, and if the
  number of priorities is also fixed, they are in pseudo-polynomial
  time.
\end{theorem}

% Local IspellDict: british
