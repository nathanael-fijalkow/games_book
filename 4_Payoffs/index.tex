

%*** General probabilistic notation ***

\newcommand{\expv}{\mathbf{E}} % EXP. VALUE
\newcommand{\discProbDist}{f} % Discrete prob distribution
\newcommand{\sampleSpace}{S} % Generic sample space
\newcommand{\sigmaAlg}{\mathcal{F}} % Generic sigma-algebra
\newcommand{\probm}{\mathbb{P}} % Generic probability measure, also prob. measure operator
\newcommand{\rvar}{X} % Generic random variable
%\newcommand{\dist}{\mathit{Dist}}

%*** MDP notation ***

\newcommand{\actions}{A} % The set of actions.
\newcommand{\colouring}{c} % the colouring function
\newcommand{\probTranFunc}{\Delta} % Transition function of an MDP
\newcommand{\edges}{E} % Set of edges in an MDP.
\newcommand{\colours}{C} % The set of colours in an MDP.
\newcommand{\mdp}{\mathcal{M}} % A generic MDP. 
\newcommand{\vinit}{v_0} % An initial vertex in an MDP.
\newcommand{\cylProb}{p} % Function assigning probabilities to cylinder sets in 
%the measure construction.
\newcommand{\emptyPlay}{\epsilon} %empty play
\newcommand{\objective}{\Omega} % Qualitative objective
\newcommand{\genColour}{\textsc{c}} % Generic colour
\newcommand{\quantObj}{f} % Generic quantitative objective
\newcommand{\indicator}[1]{\mathbf{1}_{#1}} % In.d RV
\newcommand{\eps}{\varepsilon} % Numerical epsilon
\newcommand{\maxc}{W} % Maximal abs. value of a colour

\newcommand{\winPos}{W_{>0}}
\newcommand{\winAS}{W_{=1}}
\newcommand{\cylinder}{\mathit{Cyl}}

\newcommand{\PrePos}{\text{Pre}_{>0}}
\newcommand{\PreAS}{\text{Pre}_{=1}}

\newcommand{\PreOPPos}{\mathcal{P}_{>0}}
\newcommand{\OPAS}{\mathcal{P}_{=1}}

\newcommand{\safeOP}{\mathit{Safe_{=1}}}
\newcommand{\closed}{\mathit{Cl}}

\newcommand{\reachOP}{\mathcal{V}}
\newcommand{\discOP}{\mathcal{D}}
\newcommand{\valsigma}{\vec{x}^{\sigma}}

\newcommand{\lp}{\mathcal{L}}
\newcommand{\lpdisc}{\lp_{\mathit{disc}}}
\newcommand{\lpmp}{\lp_{\mathit{mp}}}
\newcommand{\lpsol}[1]{\bar{#1}}
\newcommand{\lpmpdual}{\lpmp^{\mathit{dual}}}

\newcommand{\actevent}[3]{\actions^{#1}_{#2,#3}} % Returns #1-th action on the run 

\newcommand{\MeanPayoffSup}{\MeanPayoff^{+}}
\newcommand{\MeanPayoffInf}{\MeanPayoff^{-}}

\newcommand{\mcprob}{M}
\newcommand{\invdist}{\vec{z}}

\newcommand{\hittime}{T}



This chapter considers quantitative objectives defined using payoffs. 
Adding quantities can serve two goals:
the first is for refining qualitative objectives by quantifying how well, how fast, or at what cost a qualitative objective is satisfied,
and the second is to define richer specifications and preferences over outcomes.
%This chapter considers quantitative games. To model quantitative
%objectives, the set of colours on edges is $\R$. The goal of adding
%quantities is to help choosing among winning strategies that arise in
%classical games. We may thus use quantities in order to refine
%qualitative objectives, i.e.~to quantify how well / how fast / how
%costly a qualitative objective is satisfied. Quantitative objectives
%may also be used to design richer specifications that are not
%$\omega$-regular (like mean payoff for instance) or other quantitative
%specifications.
\begin{itemize}
	\item We start in~\cref{4-sec:qualitative} by studying extensions of the classical qualitative objectives. Among two strategies in a reachability game that guarantee to reach a target in ten steps or in a billion steps, we would certainly prefer the first one from a pragmatic point of view.

	\item We study \emph{mean payoff games} in~\cref{4-sec:mean_payoff}. 
	We present two algorithms for solving them, the first based on \emph{strategy improvement} and the second on a \emph{value iteration} for the related class of energy games.
	Along the way we show that parity games reduce to mean payoff games.
%	An algorithm for solving mean payoff games induces an algorithm for computing the value function using binary search.

	\item We study \emph{discounted payoff games} in~\cref{4-sec:discounted_payoff}.
	We construct a strategy improvement algorithm for computing the value function.
	We also show that mean payoff games reduce to discounted payoff games, so the previous algorithm yields an algorithm for computing the value function of a mean payoff game.

	\item We study \emph{shortest path games} in~\cref{4-sec:shortest_path}.
	They extend reachability games by requiring that Eve reaches her target with minimal cost, 
	which if the weights are all equal means \emph{as soon as possible}.
	
	\item We study \emph{total payoff games} in~\cref{4-sec:total_payoff}.
\end{itemize}
%\begin{itemize}
%\item We start in~\Cref{4-sec:shortest_path} by studying extensions of the
%  classical qualitative objectives for arenas with real numbers as set
%  of colours. Among two strategies in a reachability game that
%  guarantee to reach a winning vertex in ten steps or in a billion
%  steps, we would certainly prefer the first one in a pragmatic point
%  of view. Therefore, reachability games will turn into
%  \emph{shortest-path games} where a player wants to reach a winning
%  vertex \emph{as soon as possible}.
%\item A way to solve shortest-path games in full generality is to use
%  the famous \emph{mean-payoff games}, that have their own practical
%  motivation. We introduce and study them in~\Cref{4-sec:mean_payoff}. We solve
%  them with two algorithms, based on \emph{strategy improvement} and
%  \emph{value iteration}. We can then use a binary search algorithm to
%  compute the value of each player.
%\item Another very natural payoff consists in discounting the future,
%  to make the near-future more important. We define and solve
%  \emph{discounted-payoff games} in~\Cref{4-sec:discounted_payoff}. We obtain as a
%  corollary a direct (without the need of binary search for the value)
%  algorithm to compute the values of a mean-payoff game.
%\item With all those results, we finally come back to the
%  shortest-path objective in~\Cref{4-sec:shortest_path-bis}, and use the obtained
%  result to solve \emph{total-payoff games}.
%\end{itemize}

\section*{Notations}
\label{4-sec:notations}
The definition of arena $\arena$ in this chapter is $\arena=(G,\dest)$, where $G=(V,E)$ is a graph and $\dest:V\times A\times A\rightarrow \Dist(E)$. In particular, we are not using the sets $\VA$ and $\VE$.
The games are played similarly to before and formally as follows: 
There is a token, initially on the initial vertex. 
Whenever the token is on some vertex $v$, 
Eve selects an action $r$ in $A$ and Adam selects an action $c$ in $A$. The edge $e=(v,c,w)$ is then drawn from the distribution $\dest(v,r,c)$ and the token is pushed from $v$ to $w$.
 In general, the game continues like that forever.

We will use the following simplifying assumptions in this chapter:
\begin{enumerate}
\item We will assume that all colors are in $\{0,1\}$, except for the section on Matrix games where we additionally also allow $-1$ (to be able to easily illustrate the game rock-paper-scissors). This simplifies some expressions, but generally, the dependency on the number of colors is not too bad comparatively.
\item To make illustrations easier, we assume that for any pair of edges $e,e'$ in $\dest(v,a,a')$ for any $v,a,a'$, we have that $c(e)=c(e')$, i.e. the color does not depend on which edge is picked from $\dest(v,a,a')$, but only $v,a,a'$. This assumption does not matter for the types of games considered.
\end{enumerate}

We will overload the notation slightly for notational convenience, in that for any $v,a,a'$, we will write $c(v,a,a')$ for $c(e)$ where $e\in \dest(v,a,a')$ (note that the second assumption ensures that this is well-defined, i.e. there is only one such color).


A vertex $v$ is absorbing iff each player has only 1 action in $v$ and $\Delta(v,1,1)=v$.

To describe the complexity of good stationary strategies in concurrent games, we will use the notion of patience. Given a probability distribution $d\in \Dist$ the distribution has patience $p$ if $p=\min_{i\in \supp(d)} d(i)$ (i.e. the patience is the smallest, non-zero probability that an event may happen according to $d$).
In essence, if you have low enough patience you can typically guess the strategy and check whether it is a good strategy (when you fix a strategy, the game becomes a Markov decision process, which are relative easy to work with), the game can solved in $\NP\cup \coNP$. However, some times the patience is huge and writing down a good strategy, in binary, cannot be done in polynomial space (it is quite surprising in some sense that even with this property, finding the values in the games remain in $\PSPACE$).

We will illustrate a stochastic arena $\arena=(G,\dest)$ as follows:
For each non-absorbing vertex $v$, there is matrix.
 Entry $(i,j)$ of the matrix illustrating $v\in V$ describes what happens if, when the token is on $v$, Eve plays $i$ and Adam $j$. The entry contains a color $c$, which is $c(v,i,j)$, and 
there is an arrow from entry $(i,j)$ of $v$ to $w$ if there is an edge   
$e=(v,c,w)$ in $\dest(v,i,j)$. 
 The arrow corresponding to $e$ is denoted with the probability $\dest(v,i,j)(e)$. 
Especially, to simplify the illustrations we will do as follows: If $|\supp(\dest(v,i,j))|=1$, we do not include the probability (which is 1). Also, in that case, let $e=(v,c,w)=\dest(v,i,j)$ 
and 
if $v=w$, we omit the arrow and if $w$ is absorbing we write $c^*$ in entry $i,j$ of $v$, where $c$ is the color $c(w,1,1)$ (in this case, we omit the number $c(e)$ from the illustration, but in none of our illustrations does this number matter for what we try to illustrate). 


\section{Refining qualitative objectives with quantities}
\label{4-sec:qualitative}
In this section we define quantitative objectives extending the qualitative objectives $\Safe$, $\Reach$, $\Buchi$, and $\CoBuchi$.
%for instance we want to model preferences among the fact that Eve has a preference among its winning vertices: she prefers to reach (or
%visit infinitely often) $v_f^{(1)}$ if possible, or $v_f^{(2)}$
%otherwise, or $v_f^{(3)}$, etc. We may choose to encode this sequence
%of preferences with a parity condition by mapping $v_f^{(1)}$ to the
%largest even priority, then $v_f^{(2)}$ with a smallest even priority,
%etc. A more natural and versatile way to model this preference though
%is via some quantities on edges of the arena. This is what we study in
%this section.

The four quantitative objectives we will define in this section return as outcome some weight in the sequence (for instance, the maximum weight).
This is in contrast with the $\MeanPayoff$ and $\DiscountedPayoff$ objectives that we will study later,
which perform \emph{arithmetic operations} on the sequence of weights.

A first way to compute a payoff from a sequence of weights $\rho \in \Z^\omega$ is to consider the maximum weight in the sequence:
\[
\Sup(\rho) = \sup_i \rho_i.
\]
This extends the qualitative objective $\Reach[\Win]$ in the following sense: 
the objective $\Reach[\Win]$ corresponds to the quantitative objective $\Sup$ using two weights: $0$ for $\Lose$ and $1$ for $\Win$.
The outcome of a sequence is $1$ if and only if the sequence contains $\Win$.
It refines $\Reach[\Win]$ by specifying (numerical) preferences.

The dual objective is to consider the smallest weight:
\[
\Inf(\rho) = \inf_i \rho_i.
\]
The qualitative objective $\Safe[\Win]$ corresponds to the quantitative objective $\Inf$
using two weights: $0$ for $\Win$ and $1$ for $\Lose$.
The outcome of a sequence is $0$ if and only if the sequence does not contain $\Lose$.

Similarly the following quantitative objectives refine $\Buchi$ and $\CoBuchi$:
\[
  \LimSup(\rho) = \limsup_i \rho_i,\qquad 
  \LimInf(\rho) = \liminf_i \rho_i.
\]
%Using the same encoding as before changing the colouring of edges
%using weights $0$ and $1$, we have that the payoff of a play $\play$
%is $1$ if and only if $\play$ visits infinitely often a winning colour
%$\Win$ (respectively, visits finitely often a losing colour $\Lose$).

The analyses and algorithms for solving games with $\Reach$, $\Safe$, $\Buchi$, and $\CoBuchi$ objectives extend to these four quantitative objectives.

\begin{theorem}[Sup, Inf, LimSup, LimInf objectives]
\label{4-thm:sup-inf-limsup-liminf}
Games with objectives $\Sup$, $\Inf$, $\LimSup$, and $\LimInf$ are uniformly positionally determined for both players.
There exists an algorithm for computing the value function of those games in polynomial time and space.
More precisely, let $k$ be the number of different weights in the game,
the time complexity is $O(m)$ for objectives $\Sup$ and $\Inf$, and
$O(knm)$ for objectives $\LimSup$ and $\LimInf$,
and for all algorithms the space complexity is $O(m)$.
\end{theorem}

\begin{proof}
We sketch the algorithm for the objective $\Sup$, the other cases are similar.
Let $c_1 < \dots < c_k$ be the ordered enumeration of all weights in the game.
The set of vertices of value $c_k$ is $\AttrE(c_k)$, which can be computed in linear time.
We then construct the subgame $\Game'$ of $\Game$ induced by $V \setminus \AttrE(c_k)$,
and continue recursively: $\Game'$ has one less weight.

A naive complexity analysis yields a time complexity $O(km)$, but it is easily refined to $O(m)$ 
by revisiting the attractor computation and showing that each edge in the whole game is treated at most once
throughout the recursive attractor computations.
This complexity analysis does not extend to $\LimSup$ and $\LimInf$ objectives, where the complexity is multiplied by $k$.
\end{proof}


\section{Mean payoff games}
\label{4-sec:mean_payoff}
In this section we consider concurrent mean-payoff games. 
We will show that in general, any  $\epsilon$-optimal strategy in some concurrent mean-payoff games are quite complex. 
We will first, however, show that finding the value of a concurrent mean-payoff game can be done in polynomial space.

\begin{lemma}\label{lemm:class_meanpayoff}
Concurrent mean-payoff games are determined and the value is the limit of the value of the corresponding time-limited game as well as the limit of the corresponding discounted game, for the discount factor going to 0 from above.
There is an polynomial time algorithm, ala Lemma~\ref{lem:val1}, for finding the set of vertices where a finite memory strategy suffice to ensure $1-\epsilon$ (recall that all rewards are in $\{0,1\}$).
For any fixed number $n$, there is a polynomial time algorithm for approximating the value in a concurrent mean-payoff game with $n$ vertices (i.e. the running time is polynomial in the number of actions)
\end{lemma}
We will not show this lemma, but simply note that the $\epsilon$-optimal strategies known for general concurrent mean-payoff games  can be viewed as playing the corresponding discounted game with a variable discount factor that depends on how ``nice'' the rewards has been up to now. Basically, in each round you play the optimal strategy in the corresponding discounted game with a discount factor $\gamma$. Whenever 
 your rewards are close to or better than the value, you decrease $\gamma$ towards 0 and in each round your rewards are much worse than the value you let $\gamma$ increase, except not bigger than the initial $\gamma$ in the first round. Much of this section will argue that many natural candidates for simpler types of strategies does not work.


We will show that approximating the value, however, can, as mentioned, be done in polynomial space. The proof relies on Proposition~22 from~\cite{HKLMT:2011}, stating the following:
\begin{proposition}
Let $\epsilon=2^{-j}$, where $j$ is some positive integer, and the probabilities be rational numbers where the nominator and denominator have bitsize at most $\tau$. Also, let $\lambda=\epsilon^{\tau m^{O(n^2)}}$. Consider some state $s$ and let the value of that state in the $\lambda$ discounted game be $v_{\lambda}$ and the value in mean-payoff game be $v$, then $|v-v_{\lambda}|<\epsilon$.
\end{proposition}

We will use that to again reduce to the existential theory over the reals. 
For a fixed discount factor $\gamma$, we can easily express the value of the corresponding discounted game, like we expressed the value of a concurrent reachability game.
We have that the value $v$ is then $v=\lim_{\gamma\rightarrow 0^+} f(\gamma)$, where $f$ is the found expression.
I.e. for any $\epsilon$, there is a $\gamma'$ such that for all $\gamma<\gamma$, we have that $|f(\gamma)-v|\leq \epsilon$.
Also, that $v>c$ means that there is $\epsilon$, such that $v-\epsilon>c$.

The problem is thus to come up with a polynomial sized formula to express that $\lambda$ is $\epsilon^{\tau m^{O(n^2)}}=2^{-j \tau m^{O(n^2)}}$.

That can be done as follows, using $\ell=O(n^2)\cdot \log(m)+\log(j\tau)$ many variables, $v_0,v_1,\dots v_{\ell-1}$:
\[
v_0=1/2
\]
and for all $0<i< \ell$, we have that
\[
v_i=v_{i-1}\cdot v_{i-1}.
\]
Using induction, we see that $v_i=2^{-2^{i}}$, i.e., $v_1=1/2=2^{-2^0}$ and \[
v_i=v_{i-1}\cdot v_{i-1}=2^{-2^{i-1}}\cdot 2^{-2^{i-1}}=2^{-2^{i-1}-2^{i-1}}=2^{-2^{i}}\]
In particular, \[
v_{\ell-1}=2^{-2^{\ell}}=2^{-2^{O(n^2)\cdot \log(m)+\log(j\tau)}}=2^{-j\tau m^{O(n^2)}}
\] is the value we wanted for $\lambda$.
Thus, for a given number $v$, we can test if the value of a concurrent  $\lambda$-discounted game is above $v+\epsilon$, which, using the proposition above, implies that $v$ is below the value of the corresponding concurrent mean-payoff game. On the other hand, the proposition also implies that if the value of the concurrent  $\lambda$-discounted game is below $v-\epsilon$, then the value of the concurrent mean-payoff game is below $v$. Being able to answer these questions lets you easily approximate the value of a concurrent mean-payoff using binary search. 

We get the following lemma.
\begin{lemma}
Approximating the value of a concurrent mean-payoff game can be in done in polynomial space
\end{lemma}



We will now consider a specific, well-studied example of a concurrent mean-payoff game, since it shows that many natural kinds of strategies do not suffice in general.
The game is called the big match and is defined as follows:
There are 3 vertices, $\{0,s,1\}$, where the vertices in $\{0,1\}$ are absorbing, and with value equal to their name.
The last vertex $s$ has a 2x2-matrix and for all $i,j$ for $i\neq j$, we have that 
$c(s,1,1)=1$, and for $i\neq 1\neq j$ we have that $c(s,1,1)=0$.
Also,  $\dest(s,1,i)=s$ for each $i$, $\dest(s,2,1)=0$ and $\dest(s,2,2)=1$. There is an illustration in Figure \ref{7-fig:bm}.
The value of the Big Match is $1/2$.

\begin{figure}

\center
\begin{tikzpicture}[node distance=3cm,-{stealth},shorten >=2pt]
\ma{s}[$s:$]{2}{2};

\node at (s-1-1.center) {$1$};
\node at (s-2-1.center) {$0^*$};
\node at (s-1-2.center) {$0$};
\node at (s-2-2.center) {$1^*$};


\end{tikzpicture}
\caption{The Big Match}\label{7-fig:bm}
\end{figure}

Consider a finite-memory strategy $\sigma$ for Eve. We will argue that $\sigma$ cannot guarantee $\epsilon$ (any strategy can guarantee $-1$, since the colors are between $0$ and $1$) for any $0<\epsilon$. Let $\tau$ be the stationary strategy for Adam that plays $1$ with pr. $\epsilon/2$.
Then playing $\sigma$ against $\tau$, we get an Markov chain, where the vertex space is pairs of memory states and game vertices. 
In Markov chains, eventually, with pr. 1, a set of vertices $S$ is reached such that the set of vertices visited infinitely often is $S$. Such a set is called ergodic.
The set $S$ can clearly only contain 1 game vertex, since whenever $s$ is left, it is never entered again.
Hence, if $S$ contains $s$, the pr. that play will ever reach $\{0,1\}$ is 0.
In the MC we get from the players playing $\sigma$ and $\tau$, let $T_{\epsilon/2}$ be such that with pr. $\epsilon/2$ some ergodic set has been reached. 
Let $\tau'$ be the strategy that plays $\tau$ for $T_{\epsilon/2}$ and afterwards plays $2$. 

When $\sigma$ is played against $\tau'$, either we reach $\{0,1\}$ and Adam plays 1 only finitely many times, while in $s$ (the latter because there are only finitely many numbers below $T_{\epsilon/2}$). Thus, for Eve to win a play, the play needs to reach vertex 1. There are two ways to do so, either Eve stops before $T_{\epsilon/2}$ or after. In the former case, the pr. to reach $1$ is only $\epsilon/2$ (because Adam needs to play $2$ at the time, which is only done with pr. $\epsilon/2$). The latter only happens with pr. $\epsilon/2$ by definition of $T_{\epsilon/2}$ (because, Adam could play $2$ for an arbitrary number of steps while following $\tau$ but $s$ would not be left anyway).

We get the following lemma.

\begin{lemma}\label{lemm:no_finite_meanpayoff}
No finite memory strategy can guarantee more than $0$ in the Big Match.
\end{lemma}

The principle of sunken cost states that, when acting rationally, one should disregard cost already paid. We will next argue that this does not apply (naively) to the Big Match.
A strategy following the principle of sunken cost would not depend on past cost paid and thus, in each step $T$, there is a pr. $p_T$ of stopping for Eve.
Such strategies are called Markov strategies in the Big Match.
Fix some Markov strategy $\sigma$ for Eve. We will argue, like before, that $\sigma$ cannot guarantee more than $\epsilon$ for any $\epsilon>0$.
Note that Eve does not depend on the choices of Adam and thus, either she stops with pr. 1 or she does not.
In the former case, Adam just plays $1$ forever. When Eve stops, the vertex reached is thus $-1$.
Alternately, if Eve does not stop with pr. 1, there must be a time $T$, such that she only stops with pr. $\epsilon$ after $T$ (this is actually also the case even if she stops with pr. 1). 
Adam's strategy is then to play $1$ for $T$ steps and $2$ thereafter. Observe that the pr. to reach $1$ is thus at most $\epsilon$, in that it must be that Eve stops after $T$. If she does not stop (or stops in $0$), there will be only finitely many 1s.

We see the following:
\begin{lemma}\label{lemm:no_markov_meanpayoff}
No Markov strategy can guarantee more than $0$ in the Big Match
\end{lemma}


\section{Discounted payoff games}
\label{4-sec:discounted_payoff}
From a practical point of view, the modelling of a real-world
situation via mean payoff games requires that only the long-term
behaviour is important. Since mean payoff only depends on the limit of
the play, it cannot be used to model the beginning of the execution:
the mean payoff is said to be \emph{prefix independent}. In economical
studies, there is a tendency to make the prefixes count more, since
they represent short-term implications of the actions taken, even if
long-term behaviours also matter. The common payoff used to model this
preference to prefixes is the discounted payoff that associates to a
play $\play$ the weight
\[\DiscountedPayoff_\lambda(\play) = (1-\lambda)\sum_{i=0}^{\infty} \lambda^i
  \, c(\play_i)\] where $\lambda$ is a parameter in the interval
$(0,1)$, ensuring the convergence of the infinite series (since
weights $c(\play_i)$ are bounded). The coefficient $1-\lambda$ before
the series is just to counterbalance the fact that if all weights in
the game are $1$, we would like the payoff to be 1 too, which then
holds since $\sum_{i=0}^\infty \lambda^i = \frac 1{1-\lambda}$. When
$\lambda$ tends to $0$, only the prefixes (and even the first weight)
matters. On the contrary, when $\lambda$ tends to $1$, the
discounted payoff looks more and more like the mean payoff. To grasp
an intuition why this holds, consider a play that results from
positional strategies in a mean payoff game. The weights encountered
during the play then ultimately follow a periodic sequence
$w_0,w_1,\ldots,w_{r-1},w_0,w_1,\ldots,w_{r-1},w_0,\ldots$ with
average-payoff $\frac 1 r\sum_{i=0}^{r-1}w_i$. Grouping the terms of
the series $(1-\lambda)\sum_{i=0}^{\infty} \lambda^i \, w_i$ by
batches of $r$ terms, we then obtain
\[(1-\lambda)\sum_{i=0}^{\infty} \lambda^{ri} \sum_{j=0}^{r-1}
  \lambda^jw_j=\frac{1-\lambda}{1-\lambda^r}\sum_{j=0}^{r-1}
  \lambda^jw_j= \frac
  1{1+\lambda+\cdots+\lambda^{r-1}}\sum_{j=0}^{r-1} \lambda^jw_j\]
that tends towards the average-payoff $\frac 1 r\sum_{j=0}^{r-1} w_j$
when $\lambda$ tends to $1$.

The weighted game of~\cref{4-fig:MP} can also be equipped with a
discounted payoff. If $\lambda$ is close to $1$, for instance
$\lambda=0.9$, then optimal strategies are the same as for the
mean payoff objective: $\sigma^*(0)=1$, $\sigma^*(2)=3$,
$\tau^*(1)=2$, $\tau^*(3)=1$, and $\tau^*(4)=0$. In that case, the
discounted payoff of the unique lasso play
$(0,1)\big((1,2)(2,3)(3,1)\big)^\omega$ starting in $0$ and following
this profile of strategies is
$(1-\lambda)\left(4+
  \frac{2\lambda+4\lambda^2-\lambda^3}{1-\lambda^3}\right)$, which is
around $1.7$ when $\lambda=0.99$, while the mean payoff optimal value
of vertex 0 was $5/3\approx 1.67$. However, the situation completely
changes when $\lambda$ decreases. When $\lambda=0.5$ for instance,
Adam changes his decision in vertex $1$ and decides to go to vertex
$0$: this is because the cycle $(1,0)$ has now a discounted payoff
that becomes low enough, the first weight $0$ being much lower than
$2$ (if he decides to go to vertex~$2$). For a really low value of
$\lambda$, for instance $\lambda=0.1$, the decisions again change
drastically for both players: now the optimal (positional) strategy
become $\sigma^*(0)=4$, $\sigma^*(2)=3$, $\tau^*(1)=0$, $\tau^*(3)=0$,
and $\tau^*(4)=0$, i.e.~Adam changed his decision in vertex $3$, and
Eve in vertex $0$. However, we note that for each value of $\lambda$
considered so far, positional optimal strategies are described. This
is no accident, as we will see in the remainder of the chapter.

We now study how to solve this class of games, and how it is used to
obtain a theoretically more efficient solution to mean payoff
games. Indeed, we will obtain a pseudopolynomial time complexity for
both objectives.

\subsection*{Positional determinacy via a contraction mapping}

First, it is easier (than for mean payoff games) to prove that
discounted payoff games are positionally determined, by using a
description of the values as a fixed point of a contracting operator
$F\colon \R^V\to \R^V$ that maps every vector $\vec x=(x_v)_{v\in V}$
to a new vector $(y_v)_{v\in V}$ corresponding to the best that both
players can hope for in one transition if they are rewarded by vector
$\vec x$ afterwards: for all $v\in V$, we thus let
\begin{equation}y_v =
  \begin{cases}
    \max_{(v,v')\in E} [(1-\lambda) c(v,v') + \lambda x_{v'}] &
    \text{ if } v\in \VE\\
    \min_{(v,v')\in E} [(1-\lambda) c(v,v') + \lambda x_{v'}] &
    \text{ if } v\in \VA\\
  \end{cases}\label{4-eq:F-contraction}
\end{equation}

\begin{theorem}[Positional determinacy]\label{4-thm:discounted}
  Discounted payoff games are positionally determined.
\end{theorem}
\begin{proof}
  We prove that $F$ is indeed a contracting operator.
  For all vectors $\vec x=(x_v)_{v\in V}$ and
  $\vec y = (y_v)_{v\in V}$, we have, by definition,
  $\|F(\vec x)-F(\vec y)\|_\infty = \max_{v\in V}|F(\vec x)_v-F(\vec
  y)_v|$. Let $v\in \VE$. Consider a vertex $v'$ such that
  $F(\vec x)_v = (1-\lambda) c(v,v') + \lambda x_{v'}$. Then,
  $F(\vec y)_v\geq (1-\lambda) c(v,v') + \lambda y_{v'}$ so that
  $F(\vec x)_v-F(\vec y)_v\leq \lambda(x_{v'}-y_{v'})$. By now
  considering $v''$ such that
  $F(\vec y)_v = (1-\lambda) c(v,v'') + \lambda y_{v''}$, we obtain
  also that $F(\vec x)_v-F(\vec y)_v\geq \lambda(x_{v''}-y_{v''})$. In
  the overall, we thus have
  $|F(\vec x)_v-F(\vec y)_v|\leq \lambda\|\vec x-\vec y\|_\infty$. The
  same reasoning also applies to a vertex $v$ of $\VA$. Therefore, we
  obtain
  $\|F(\vec x)-F(\vec y)\|_\infty\leq \lambda\|\vec x-\vec y\|_\infty$
  which means that $F$ is a contraction mapping (since $0<\lambda<1$).

  By Banach fixed-point theorem, $F$ admits a unique fixed-point
  vector $\vec{x^*}$, such that~$F(\vec{x^*})=\vec {x^*}$. This
  fixed-point is moreover the limit of the sequence of vectors
  $(F^n(\vec 0))_{n\in \N}$ with $\vec 0$ being the null vector, by
  Kleene fixed-point theorem (since $F$ is continuous, by composition
  of continuous functions). Imagine that Eve plays a positional
  strategy dictated by the equality $F(\vec{x^*})=\vec {x^*}$,
  i.e.~when in vertex $v\in \VE$, she chooses to go to some vertex
  $v'$ such that $x^*_v = (1-\lambda) c(v,v') + \lambda
  x^*_{v'}$. Then, she guarantees, by an easy induction proof, a value
  at least $x^*_v$ when starting in vertex~$v$ (for all $v\in V$). A
  symmetrical argument for Adam allows one to obtain that, for all
  $v\in V$, \[\ValueA(v)\leq x^*_v \leq \ValueE(v)\,.\] Since we
  always have $\ValueE(v)\leq \ValueA(v)$, we finally obtain that the
  game is determined, and that $x^*$ is equal to the optimal value
  vector. Moreover, the two above positional strategies for Eve and
  Adam are optimal.
\end{proof}

As for mean payoff (or parity) games, the existence of positional
optimal (or winning) strategies for both players, and the ability to
solve in polynomial time the one-player version of these games, allows
us to obtain easily an $\NP\cap\coNP$ complexity to solve
discounted payoff games in the case of integer costs and a rational
discount factor. The use of the above contracting operator even
ensures that the Turing machines guessing and checking the optimal
strategies may indeed be designed as unambiguous (instead of just
non-deterministic). Calling $\UP$ the class of problems that can be
solved by an unambiguous Turing machine running in polynomial time,
and $\coUP$ the class of problems whose complement are in $\UP$, we
then obtain the theorem:
\begin{theorem}[Complexity]\label{4-thm:disc-up}
  Discounted payoff games with integer costs and rational discount
  factor $\lambda$ can be solved in $\UP\cap\coUP$.
\end{theorem}
\begin{proof}
  Using the previous result, we know that the value of a discounted
  payoff game is the \emph{unique} solution of the fixed point
  equation $\vec x = F(\vec x)$, with $F$ a contraction
  mapping. Therefore, guessing vector $\vec x$ and checking it is
  indeed a fixed point of $F$ can be performed in an unambiguous
  Turing machine. To conclude, it only remains to show that this check
  can be done in polynomial time, in particular we must show that the
  values $\vec x$ are \emph{short} in terms of their binary
  representation. The equality $\vec x = F(\vec x)$ induces optimal
  decisions in each vertex, thus leading to a profile of strategies
  for both players. We summarise this profile
  in: \begin{itemize} \item a square Boolean matrix
  $Q\in \{0,1\}^{V\times V}$, whose entry $Q_{v,v'}$ is $1$ if
  $(v,v')$ is the chosen edge in $v$ by the profile, and $0$
  otherwise; \item a vector $\vec c\in \Z^{V}$, whose entry $c_v$ is
  the weight of the edge $(v,v')$ chosen in $v$ by the
  profile.  \end{itemize} We can then write the fixed point equation
  as \[\vec x = (1-\lambda) \vec c + \lambda Q \vec x\,.\] Letting
  $\lambda = a/b$ the rational discount factor, the above equation
  rewrites into \begin{equation} A\vec x = (b-a)\vec
  c\label{4-eq:1} \end{equation} with $A= b I- a Q$ ($I$ being the
  identity matrix). Therefore, $A$ is a matrix that has at most two
  non-zero elements in each row: each of these non-zero elements can
  be written using at most $N=\max(\log_2 a,\log_2 b)$ bits (therefore
  polynomial in the representation of the game), and are therefore
  bounded in absolute value by $2^N$. By induction on the size of the
  matrix, we can then show that the determinant of $A$ is at most
  $4^{Nn}$. \cref{4-eq:1}\todo{Ne s'affiche pas dans la version html} then resolves, using Cramer's formula, by
  $x_v = \det (A_v) / \det (A)$, with $A_v$ the matrix obtained from
  $A$ by replacing the $v$-th column with vector $(b-a)\vec
  c$. Therefore, all components of $\vec x$ can be written with only a
  polynomial of bits with respect to the size of the costs in the
  arena and $N$.

  The $\coUP$ membership follows, as in \cref{4-thm:MP-NPcoNP}, from a
  dual reasoning for Adam, using the above determinacy result for
  discounted payoff games. 
\end{proof}

For the discounted payoff game of~\cref{4-fig:MP}, the contracting
operator is:
\[F\left(
  \begin{array}{c}
    x_{v_0}\\x_{v_1}\\x_{v_2}\\x_{v_3}\\x_{v_4}
  \end{array}\right) =
  \left(
    \begin{array}{c}
      \max\big(4(1-\lambda)+\lambda x_{v_1}, (1-\lambda)5+\lambda x_{v_4}\big)\\
      \min\big(\lambda x_{v_0}, 2(1-\lambda)+\lambda x_{v_2}\big)\\
      \max\big((1-\lambda)+\lambda x_{v_2}, 4(1-\lambda)+\lambda x_{v_3}\big)\\
      \min\big(-2(1-\lambda)+\lambda x_{v_0}, -(1-\lambda)+\lambda x_{v_1}\big)\\
      \min\big(-2(1-\lambda)+\lambda x_{v_0}, 2(1-\lambda)+\lambda x_{v_4}\big)
    \end{array}
  \right)\,.\]

A careful analysis gives the fixed points for all values of
$\lambda\in (0,1)$, which in turn allows us to find the associated
positional optimal strategies $\sigma^*$ and~$\tau^*$ on the various
intervals of values for $\lambda$, summarised in the following table:
\[
  \begin{array}{|c|c|c|c|c|}\hline
    \lambda & (0, \lambda_1]
    & (\lambda_1,\lambda_2] & (\lambda_2,\lambda_3]
    & (\lambda_3,1) \\\hline
    \sigma^*(v_0) & v_4 & v_4 &  v_1   & v_1  \\\hline
    \tau^*(v_1) & v_0 &  v_0 &   v_0 &   v_2  \\\hline
    \sigma^*(v_2) & v_3  & v_3   & v_3 &   v_3 \\\hline
    \tau^*(v_3) &v_0 &  v_1  & v_1 &   v_1  \\\hline
    \tau^*(v_4) & v_0 &  v_0  & v_0 &  v_0\\\hline
  \end{array}\]
where frontiers are at $\lambda_1 = 1-\sqrt 2/2 \approx 0.293$,
$\lambda_2 = 1/2$, and $\lambda_3 \approx 0.841$. For instance, on
interval $(0,\lambda_1]$, Adam gets discounted payoff
$\frac{5\lambda-2}{1+\lambda}$ when starting 
in vertex $v_3$, while switching his decision in interval
$(\lambda_1,\lambda_2]$ allows him to secure
$\frac{-2\lambda^3+6\lambda^2-1}{1+\lambda}$: this gives the
explanation for the value of $\lambda_1$ which allows one to equal the
two values. A similar reasoning provides the values of
$\lambda_2$ and $\lambda_3$.

\subsection*{Strategy improvement algorithm}

We apply the strategy improvement paradigm already used for
mean payoff games. However, in the context of discounted payoff games,
this algorithm will directly run on the game itself to compute optimal
values and an optimal strategy for Eve (instead of only deciding if
Eve can guarantee a positive mean payoff).

The principle is identical though: it relies on an amelioration of a
strategy based on a local switching policy. Starting from an arbitrary
positional strategy of Eve, we apply local improvements by switching
some decisions to obtain a strictly better positional strategy. Since
there are only a finite (but exponential) number of positional
strategies, this strategy improvement algorithm will terminate (with
at most exponential time worst-case complexity). Moreover, it is
correct, as we will see, meaning that the local optimum we find when
no more switching is applicable is indeed a global optimum for Eve,
meaning that we have computed a (positional) optimal strategy for her.

To describe the switching policy used to solve discounted payoff
games, consider a strategy $\sigma$ of Eve, and let $\Value^\sigma(v)$
be the best possible value Adam can get against $\sigma$, when the
play starts from vertex $v$:
\[\Value^\sigma(v) = \inf_\tau
  \DiscountedPayoff_\lambda(v,\sigma,\tau).\] Two strategies $\sigma$
and $\sigma'$ are compared with respect to the vectors
$\vec x=\Value^\sigma$ and $\vec{x'}=\Value^{\sigma'}$: we denote by
$\vec x\leq \vec{x'}$ the fact that $x_v\leq x'_v$ for all vertices
$v\in V$. Computing the vector $\Value^\sigma$ amounts to solving a
one-player discounted payoff game where Eve's vertices are now
constrained to follow $\sigma$, and thus can be replaced by Adam's
vertices. By~\cref{4-thm:discounted}, the solution of such a
one-player game is still the unique fixed point of the contraction
mapping $F_\sigma$ defined for all $\vec x\in \R^V$ and $v\in V$ by
\[F_\sigma(\vec x)_v = \min_{(v,v')\in E}[(1-\lambda)c(v,v')+\lambda x_{v'}].\]

\begin{proposition}[One player]\label{4-lem:one-player-DP}
  We can compute in polynomial time the optimal value of a one-player
  discounted payoff game, by finding the unique fixed point
  $\vec{x^*}$ of the previous contraction mapping $F_\sigma$.
\end{proposition}
\begin{proof}
  We first show that if a vector $\vec x$ satisfies
  $\vec x\leq F_\sigma(\vec x)$, then $\vec x\leq \vec{x^*}$. Indeed,
  consider any positional strategy $\tau$ of the unique player Adam,
  and a vertex $v\in V$. Then,
  $x_v\leq F_\sigma(\vec x)_v\leq (1-\lambda)c(v,v')+\lambda x_{v'}$ with
  $(v,v')=\tau(v)$. We let $\play$ be the play starting in $v$, following
  $\tau$, and denote by $v=v_0,v_1,\ldots$ the sequence of vertices
  visited by $\play$. By induction,
  $x_v \leq (1-\lambda)\sum_{i=0}^{n-1}\lambda^i c(\play_i) +
  \lambda^nx_{v_n}$. Letting $n$ go to $+\infty$, we get that
  $x_v\leq \DiscountedPayoff_\lambda(\play)$. Since this holds for all
  strategies of Adam, and $\vec{x^*}$ is the optimal value vector of the
  game by~\cref{4-thm:discounted}, we obtain $x_v\leq x^*_v$.

  Therefore, it follows that $x^*$ is the unique solution of the
  linear program
  \[
    \begin{cases}
      \max\sum_{v\in V}x_v \\
      \text{under the constraints }\quad x_v\leq
      (1-\lambda)c(v,v')+\lambda x_{v'}\qquad \forall (v,v')\in E
    \end{cases}\]
  Such a linear program can be solved in polynomial time. 
\end{proof}

On top of computing the value $\Value^\sigma$ of a strategy $\sigma$
of Eve, we also compute the best response of Adam, that is the best
(positional) strategy $\tau$ he must play to achieve the lowest payoff
possible for Eve.

For the discounted payoff game of~\cref{4-fig:MP} with $\lambda=0.5$,
if we start from Eve's strategy $\sigma(0)=4$ and $\sigma(2)=2$, the
simplified contraction mapping in the one-player remaining game is
given by:
\[F_\sigma\left(
  \begin{array}{c}
    x_0\\x_1\\x_2\\x_3\\x_4
  \end{array}\right) =
  \left(
    \begin{array}{c}
      (1-\lambda)5+\lambda x_4\\
      \min\big(\lambda x_0, 2(1-\lambda)+\lambda x_2\big)\\
      (1-\lambda)+\lambda x_2\\
      \min\big(-2(1-\lambda)+\lambda x_0, -(1-\lambda)+\lambda x_1\big)\\
      \min\big(-2(1-\lambda)+\lambda x_0, 2(1-\lambda)+\lambda x_4\big)
    \end{array}
  \right)\,.\] We compute the best response of Adam by solving the
linear program:
\[\begin{cases}
    \max x_0+x_1+x_2+x_3+x_4 \\
    \text{under the constraints } & x_0=(1-\lambda)5+\lambda
    x_4 \\
    & x_1\leq \lambda x_0 \\
    & x_1\leq 2(1-\lambda)+\lambda x_2\\
    & x_2 = (1-\lambda)+\lambda x_2 \qquad (\text{thus } x_2=1) \\
    & x_3\leq -2(1-\lambda)+\lambda x_0\\
    & x_3 \leq -(1-\lambda)+\lambda x_1 \\
    & x_4\leq -2(1-\lambda)+\lambda x_0\\
    & x_4\leq 2(1-\lambda)+\lambda x_4
  \end{cases}\] Feeding this linear program to a solver gives back the
solution $\vec x=(8/3,4/3,1,1/6,\allowbreak 1/3)$, which is associated
with the best response of Adam defined as $\tau(1)=0$, $\tau(3)=1$,
and $\tau(4)=0$. We can check if this vector $\vec x$ is a fixed point
of $F$ (and not only $F_\sigma$): we indeed have
$F(\vec x)_0=\max\big(4(1-\lambda)+\lambda x_1, (1-\lambda)5+\lambda
x_4\big)= \max(8/3,8/3)= 8/3$, but
$F(\vec x)_2=\max\big((1-\lambda)+\lambda x_2, 4(1-\lambda)+\lambda
x_3\big) = \max(1,25/12)=25/12\neq 1$. This suggests modifying
strategy~$\sigma$ to a new strategy~$\sigma'$ with $\sigma'(0)=4$ and
$\sigma'(2)=3$. The new vector of values found by solving the new
linear program is $\vec{x'}=(8/3,4/3,25/12,1/6,1/3)$, that is the
(unique) fixed point of~$F$. Notice in particular that
$\vec{x}\leq \vec{x'}$ with a strict inequality on the component of
vertex~$2$.


\medskip

We now come back to the general case by describing more precisely the
algorithm in~\cref{4-algo:DP-strategy-improvement}, and prove its
correctness.  For a particular strategy $\sigma$ for Eve, once
$\Value^\sigma$ computed, we can check whether
$F(\Value^\sigma)=\Value^\sigma$ (with $F$ the more general operator
defined in \cref{4-eq:F-contraction}). If it is the case, then we
know that the optimal value vector of the game is indeed
$\Value^\sigma$: thus, $\sigma$ is a positional optimal strategy for
Eve, and the best response of Adam is a positional optimal strategy
for him. In case $F(\Value^\sigma)\neq\Value^\sigma$, we consider for
every vertex $v\in \VE$, the decision $v'$ such that
$F(\Value^\sigma)_v=(1-\lambda)c(v,v')+\lambda \Value^\sigma(v')$. We
gather all these decisions in a new strategy $\sigma'$ for Eve (only
modifying $\sigma$ over vertices for which it allows for a strictly
better value, to ensure the termination of the algorithm).

\begin{algorithm}
 \KwData{A discounted payoff game $\game$ with discount factor
 $\lambda\in(0,1)$, and $F$ the contracting operator}
 %\SetKwFunction{FTreat}{Treat}
 %\SetKwProg{Fn}{Function}{:}{}

 $\sigma \leftarrow$ arbitrary positional strategy for Eve ;

 $\Value^\sigma \leftarrow $ fixed-point of $F_\sigma$ ;

 \While{$F(\Value^\sigma) \neq \Value^\sigma$}
 {   
 \For{$v\in \VE$}
 {
   \If{$F(\Value^\sigma)_v \neq \Value^\sigma(v)$}{
     $v' \leftarrow $ a vertex such that $F(\Value^\sigma)_v =
     (1-\lambda)c(v,v')+\lambda \Value^\sigma(v')$ ;
     
     $\sigma(v) \leftarrow v'$
     }
}

  $\Value^\sigma \leftarrow $ fixed-point of $F_\sigma$ ;

}

\Return{$(\Value^\sigma,\sigma)$}
\caption{The strategy improvement algorithm for discounted payoff games.}
\label{4-algo:DP-strategy-improvement}
\end{algorithm}

\begin{theorem}[Strategy improvement]\label{4-thm:DP-strategy-improvement-correctness}
  Strategy improvement algorithm computes the value of the game, as
  well as a positional optimal strategy for Eve, in exponential time. 
\end{theorem}
\begin{proof}
  When the algorithm returns a strategy $\sigma$, it fulfils
  $F(\Value^\sigma)\neq \Value^\sigma$, and thus,
  by~\cref{4-thm:disc-up}, $\Value^\sigma$ is the value of the game,
  and $\sigma$ an optimal strategy of Eve.

  We thus only have to prove the termination of the algorithm, as well
  as its complexity. Consider the positional strategy $\sigma$ in the
  beginning of an iteration such that
  $F(\Value^\sigma)\neq \Value^\sigma$, as well as the strategy
  $\sigma'$ updated at the end of the same iteration. We see why
  $\Value^\sigma\leq \Value^{\sigma'}$, with a strict inequality for
  at least one of the coefficients. If it is correct, then it shows
  that the \textbf{while} loop terminates after a finite number of
  iterations, since there are only a finite number of positional
  strategies and that we cannot visit twice the same one.  More
  precisely, the algorithm has an exponential worst-case complexity,
  since it may have to go through all (or at least a large fraction
  of) the positional strategies.

  We thus prove that $\Value^\sigma\leq \Value^{\sigma'}$. Consider
  for that the two best responses of Adam, $\tau$ and $\tau'$
  respectively. As in the proof of~\cref{4-thm:disc-up}, letting $Q$,
  $\vec c$, $Q'$, and $\vec {c'}$ the respective matrices and cost
  vectors described by the profiles of strategies $(\sigma,\tau)$ and
  $(\sigma',\tau')$, we have
  \[\Value^\sigma = (1-\lambda) \vec c + \lambda Q \Value^\sigma \qquad
    \text{ and } \qquad \Value^{\sigma'} = (1-\lambda) \vec {c'} +
    \lambda Q' \Value^{\sigma'}\,.\]
  Therefore, by adding and subtracting $\lambda Q'
  \Value^\sigma$ we obtain:
  \[\Value^{\sigma'}-\Value^\sigma=\lambda Q'
    (\Value^{\sigma'}-\Value^{\sigma}) + \underbrace{\lambda
      (Q'-Q)\Value^\sigma + (1-\lambda)(\vec{c'}-\vec c)}_{=\vec
      \delta}\,.\] Since $Q'$ is a positive matrix with coefficients in
  $\{0,1\}$, the series $\sum_i \lambda^iQ'^i$ converges, which shows
  that $I-\lambda Q'$ is invertible of inverse
  $\sum_{i=0}^\infty \lambda^i Q'^i$. In particular, the inverse
  $(I-\lambda Q')^{-1}$ has only non negative coefficients, and its
  diagonal coefficients are positive. Therefore, to show that
  $\Value^{\sigma'}-\Value^\sigma=(I-\lambda Q')^{-1} \vec\delta$ is
  non-negative with at least one positive coefficient, it suffices to
  show that $\vec\delta$ is non-negative with at least one positive
  coefficient. Consider thus $v\in V$:
  \begin{itemize}
  \item If $v\in \VE$, we have
    $\delta_v=\lambda(\Value^{\sigma}(v'_1)-\Value^\sigma(v_1)) +
    (1-\lambda)(c(v,v'_1)-c(v,v_1))$, with
    $\sigma(v)=(v,v_1)$ and $\sigma'(v)=(v,v'_1)$. If $v_1=v'_1$, then
    $\delta_v=0$. Otherwise, since $\sigma'$ is obtained by switching
    the decisions according to $F$, we have $\delta_v>0$: notice that
    there exists at least one such vertex $v$, since otherwise,
    $\sigma'=\sigma$ and the algorithm has already terminated.
  \item If $v\in \VA$, we have the same formula for $\delta_v$ with
    $\tau(v)=(v,v_1)$ and $\tau'(v)=(v,v'_1)$. Once again, $\delta_v=0$
    if $v_1=v'_1$. Otherwise, since $\tau$ is the best response of Adam
    to the strategy $\sigma$ of Eve, $\tau$ is at least as good as
    $\tau'$, which means that $\delta_v\geq 0$ too.
  \end{itemize}
\end{proof}

We now study another algorithm to compute the values with a possibly
better worst-case complexity, trading an exponential (with respect to
the number of vertices) complexity (but polynomial with respect to the
binary encoding of $\lambda$ and the weights of the arena), for a
pseudopolynomial time complexity (polynomial with respect to the
number of vertices and the binary encoding of the weights of the
arena, but exponential with respect to the binary encoding of
$\lambda$).

\subsection*{Value iteration algorithm}

Another way to make use of the contraction mapping $F$
of~\cref{4-eq:F-contraction} is to compute the sequence
$\big(F^n(\vec 0)\big)_{n\in \N}$ that converges towards the value
vector. However, the sequence is not stationary in general, and thus,
to obtain an exact value we find a index~$K$ for which $F^K(\vec 0)$
is close to $\Value$, as well as a rounding procedure to get the exact
value $\Value$ from its approximation $F^K(\vec 0)$. It is mainly
based on the following technical lemma stating that $\Value(v)$ is a
rational number with a denominator that we can bound, in a similar
manner as for~\cref{4-cor:rational-MP} in the mean payoff setting:
\begin{lemma}[Upper bound on rational values in mean payoff games]
\label{4-lem:rational-discounted}
  If the arena has integer costs and $\lambda=\frac a b\in (0,1)$,
  then for all vertices $v\in V$, $D\times \Value(v)\in \Z$, with
  $D= b^{n-1}\prod_{j=1}^{n}(b^j-a^j)$.
\end{lemma}
\begin{proof}
  By picking any two optimal positional strategies for Eve and Adam
  (by~\cref{4-thm:discounted}), we obtain a play $\pi$ starting in
  vertex $v\in V$ that has a discounted payoff
  $\DiscountedPayoff(\pi) = \Value(v)$. Since both strategies are
  positional, the play $\pi$ is a lasso: thus, the sequence of costs
  encountered through the play is of the form
  $w_0,w_1,\ldots,w_{k-1},(w_{k},\ldots,w_\ell)^\omega$, with
  $0\leq k\leq\ell< n$ and $w_i\in \Z$. We can thus compute the
  optimal value exactly:\todo{Ça ne passe pas dans la version
  html... pourquoi ?}
  \begin{align*}
    \Value(v) &= (1-\lambda) \left[\sum_{i=0}^{k-1} \lambda^i w_i +
               \lambda^{k}\sum_{m=0}^\infty
               \lambda^{(\ell-k+1)m}\sum_{i=0}^{\ell-k}
               \lambda^iw_{k+i}\right]\\
             &= \frac{b-a}b \left[\sum_{i=0}^{k-1} \frac{b^{k-1-i}a^i w_i}{b^{k-1}} +
               \frac{\lambda^{k}}{1-\lambda^{\ell-k+1}}\sum_{i=0}^{\ell-k}
               \frac{b^{\ell-k-i}a^iw_{k+i}}{b^{\ell-k}}\right]\\
             &= \frac {N_1}{b^{k}} + 
               \frac{a^{k}b^{\ell-k+1}}{b^{k+1}(b^{\ell-k+1}-a^{\ell-k+1})}
               \frac{N_2}{b^{\ell-k}} \qquad \text{(with $N_1,N_2\in \Z$)}\\
             &= \frac{N_3}{b^{k}(b^{\ell-k+1}-a^{\ell-k+1})} \qquad \text{(with
               $N_3\in \Z$)}\\
             &= \frac{N}{b^{n-1}\prod_{j=1}^{n}(b^j-a^j)} \qquad \text{(with
               $N\in \Z$)} 
  \end{align*}
  which proves that $\Value(v)\times D = N\in \Z$.
\end{proof}

Therefore, $\Value(v)$ is a rational number with a denominator bounded
by $D$. In particular, if we have an approximation $\eta$ of
$\Value(v)$ such that $|\Value(v)-\eta|<\frac 1 {2D}$, we get that
$\Value(v) = \frac{\lfloor D\eta+1/2\rfloor}D$. Using the fact that
operator $F$ is contracting, we can find an index $K$ after which this
rounding leads to the correct optimal value vector. In the following,
we let again $W = \max_{(v,c,v')\in E} |c|$ the maximal weight on edges of the arena, in
absolute values.

\begin{lemma}[Number of steps of value iteration]\label{4-lem:number-steps-VI-discounted}
  Let $K\in \N$ at most
  $\frac{1}{-\log_2\lambda} \left(\frac{n(n+3)}{2}\log_2b +
    \log_2 W+2\right)$. Then,
  $\|F^K(\vec 0)-\Value\|_\infty < \frac 1 {2D}$.
\end{lemma}
\begin{proof}
  First, we bound $D$ by $b^{n+\frac{n(n+1)}2}$, so that
  $\frac{n(n+3)}{2}\log_2b\geq \log_2D$. Therefore
  $K\geq \frac{1}{-\log_2\lambda} (\log_2D + \log_2 W+\log_24) =
  \log_{1/\lambda}(4DW)$. This implies that
  $\lambda^KW\leq \frac{1}{4D}< \frac 1 {2D}$. But $F$ is
  $\lambda$-contracting, so that
  $\|F^K(\vec 0)-\Value\|_\infty\leq
  \lambda^K\|\Value\|_\infty$. Since $\Value(v)$ is the discounted sum
  of weights all bounded in absolute value by $W$, we also know that
  $\|\Value\|_\infty\leq W$ which allows us to conclude.
\end{proof}

Therefore, the value iteration algoritm consists at iterating the
contracting mapping $F$ for a certain number $K$ of steps which is
polynomial with respect to the arena, each of the steps being
performed in time $O(m)$, and then finish the computation by a
rounding procedure (see~\cref{4-algo:DP-value-iteration}). In the
overall, it thus has complexity $O(K m)$.

\begin{algorithm}
 \KwData{A discounted payoff game $\game$ with discount factor
 $\lambda= a/b\in(0,1)$, and $F$ the contracting operator}
 %\SetKwFunction{FTreat}{Treat}
 %\SetKwProg{Fn}{Function}{:}{}

 $\vec x \leftarrow \vec 0$ ;

 $K \leftarrow \left\lceil \frac{1}{-\log_2\lambda} \left(\frac{n(n+3)}{2}\log_2b +
     \log_2 W+2\right) \right\rceil$ ;

 \For{$i = 1$ to $K$}
 {
   $\vec x \leftarrow F(\vec x)$
 }

 \Return{$\vec x$}
\caption{The value iteration algorithm for discounted payoff games.}
\label{4-algo:DP-value-iteration}
\end{algorithm}

\begin{theorem}[Correctness and complexity of value iteration]\label{4-thm:DP-value-iteration}
  Value iteration algorithm computes in pseudopolynomial time the
  value vector of a given discounted payoff game with only rational
  weights and a rational discount factor $\lambda\in (0,1)$. 
\end{theorem}
\begin{proof}
  The correctness of the algorithm follows
  from~\cref{4-lem:number-steps-VI-discounted}.  This algorithm runs
  in pseudopolynomial time (and not polynomial-time) because of the
  dependence in the discount factor $\lambda$. Indeed, consider that
  $\lambda = 1-\frac 1 b$, with $b\in \N\setminus \{0\}$. Then, we may
  store $\lambda$ with $\log_2 b$ bits, yet
  $\frac 1{-\log_2\lambda} \sim_{b\to \infty} b\ln 2$ is exponential
  in $\log_2b$.
\end{proof}


Once the optimal values are known, finding some positional optimal
strategies for both players still requires to work, as we have already
seen in~\cref{1-sec:memory}:
% %\todo{I prefer to detail that here, since it is the first time it appears in the quantitative setting where it is not at all trivial...} 
% It can be done by a recursive search (as explained in~\cite{Zwick&Paterson:1996} for
% mean payoff games). For each vertex $v$ with outdegree $d(v)>1$, we
% can recompute the value of the discounted payoff game over the arena
% where a subset of $\lceil d/2\rceil$ outgoing edges of $v$ is
% removed. If the new value is identical, there is a positional optimal
% strategy that picks one of the remaining outgoing edges of~$v$,
% otherwise we need to choose one of the removed edges. By a binary
% search algorithm, we find in $\bigO(\log d)$ the positional
% strategy in a given vertex~$v$. Doing that for all vertices requires
% $\bigO(\sum_{v} \log d(v))$ calls to the above value iteration
% algorithm. Since, $\sum_v\log d(v) \leq n \log (m/n)$, we get
% that
\begin{theorem}[Optimal strategies]\label{4-thm:DP-strategies}
  In a discounted payoff game with integer costs and rational discount
  factor $\lambda = a/b$, optimal strategies for both players can be
  found in
  $\bigO\big((n^3b\log_2b + \log_2W)\allowbreak
  \log(m/n) m\big)$ time.
\end{theorem}

\subsection*{Polynomial reduction from mean payoff games to discounted payoff games}\label{4-sec:mean_payoff-values}

Building upon the pseudopolynomial time algorithm for
discounted payoff, we now describe a classical encoding of mean payoff
games into discounted payoff games to obtain another algorithm with
pseudopolynomial time complexity for mean payoff games. Compared to
the algorithms studied before that were computing the values of a
mean payoff game by a binary search, this other algorithm (indeed the
oldest one) is more direct even if it does not obtain a better
complexity.

Recall that~\cref{4-cor:rational-MP} states that, in an arena where
costs are integers, the mean payoff value $\Value(v)$ is a rational
number with denominator in $\{1,\ldots,n\}$. The minimal distance
between two rational numbers $\frac \alpha k$ and $\frac{\alpha'}{k'}$
with $k,k'\in \{1,\ldots,n\}$ is
$\frac 1{n-1}-\frac 1{n}=\frac{1}{n(n-1)}$. Thus, a
$\frac{1}{2n(n-1)}$ approximation $\beta$ of $\Value(v)$ is enough
to apply a rounding procedure finding the only rational
$\frac \alpha k$ with $k\in \{1,\ldots,n\}$ in interval
$[\beta-\frac{1}{2n(n-1)}, \beta+\frac{1}{2n(n-1)}]$. By
interpreting the mean payoff game as a discounted payoff game with a
nicely chosen $\lambda$, we are able to find such a good
approximation:
\begin{theorem}[Discounted payoff approximation]\label{4-thm:MP-Zwick-Paterson}
  Let $\arena$ be an arena with integer costs. Let
  $\lambda\in(0,1)$. We let $\Value(v)$ be the value of vertex $v$ in
  the mean payoff game on $\arena$, and $\Value_\lambda(v)$ be the
  value of vertex $v$ in the discounted payoff game on $\arena$, with
  $\lambda$ as discount factor. Then
  $\|\Value-\Value_\lambda\|_\infty\leq 2n(1-\lambda)W$.
\end{theorem}
\begin{proof}
  Let $v\in V$. We prove the inequality
  $\Value_\lambda(v)-\Value(v)\geq -2n(1-\lambda)W$ by reasoning on
  Eve's strategies: a similar reasoning on Adam's strategies allows
  one to obtain the other inequality
  $\Value_\lambda(v)-\Value(v)\leq 2n(1-\lambda)W$.

  By~\cref{4-thm:mean_payoff_positional}, we may select positional optimal
  strategies for Eve and Adam in the mean payoff game, denoted by
  $\sigma^*$ and $\tau^*$ respectively. As we have already seen, the play
  $\play$ starting in $v$ following $\sigma^*$ and $\tau^*$ is then a
  lasso, with a sequence of costs encountered of the form
  $w_0,w_1,\ldots,w_{k-1},(w_k,\ldots,w_\ell)^\omega$ with
  $0\leq k\leq \ell<n$. Then,
  $\Value(v)=\frac 1 {\ell-k+1}\sum_{i=k}^\ell w_i$. By choosing the
  same strategy $\sigma$ for Eve in the discounted payoff game, we
  know that Eve's value is at least
  $\DiscountedPayoff_\lambda(\play)$: therefore, by the determinacy
  result of~\cref{4-thm:discounted},
  $\Value_\lambda(v)\geq \DiscountedPayoff_\lambda(\play)$. We now
  compute precisely $\DiscountedPayoff_\lambda(\play)$:\todo{Ne
  fonctionne pas en HTML}
  \begin{align*}
    \DiscountedPayoff_\lambda(\play)
    &=
      (1-\lambda)\sum_{i=0}^{k-1}\lambda^i
      \underbrace{w_i}_{\makebox[0pt][c]{\scriptsize$\geq -W$}}
      +
      (1-\lambda)\lambda^k\sum_{m=0}^\infty
      \lambda^{(\ell-k+1)m} \sum_{i=0}^{\ell-k}\lambda^iw_{k+i}\\
    & \geq -(1-\lambda^k)W +
      \frac{(1-\lambda) \lambda^k}{1-\lambda^{\ell-k+1}}
      \sum_{i=0}^{\ell-k}\lambda^iw_{k+i} \tag*{\label{4-eq:1}}
  \end{align*}
  By shifting all weights by $W$, we can rewrite the remaining
  sum as:
  \[\sum_{i=0}^{\ell-k}\lambda^iw_{k+i} =
    \sum_{i=0}^{\ell-k}\lambda^i(w_{k+i}+W) -
    W\sum_{i=0}^{\ell-k}\lambda^i\]
  By using the fact that $w_{k+i}+W$ is non-negative and
  $\lambda^i\geq \lambda^{\ell-k}$, we obtain
  \begin{align*}
    \sum_{i=0}^{\ell-k}\lambda^iw_{k+i}
    &\geq \lambda^{\ell-k}
      \sum_{i=0}^{\ell-k}(w_{k+i}+W) -W
      \frac{1-\lambda^{\ell-k+1}}{1-\lambda} \\
    &= \lambda^{\ell-k}
      \sum_{i=0}^{\ell-k}w_{k+i} + (\ell-k+1)\lambda^{\ell-k}W
      -W \frac{1-\lambda^{\ell-k+1}}{1-\lambda}
  \end{align*}
  Therefore, since $\Value(v)=\frac 1 {\ell-k+1}\sum_{i=k}^\ell w_i$,
  we obtain
  \[\sum_{i=0}^{\ell-k}\lambda^iw_{k+i}\geq
    \lambda^{\ell-k}(\ell-k+1)(\Value(v)+W)-
    W\frac{1-\lambda^{\ell-k+1}}{1-\lambda} \,.\]
  Simplifying~\cref{4-eq:1} gives:
  \[\DiscountedPayoff_\lambda(\play)
    \geq -W+ \frac{(1-\lambda)(\ell-k+1)}{1-\lambda^{\ell-k+1}}
    \lambda^\ell (\Value(v) + W)\] Since
  $\frac {1-\lambda^{\ell-k+1}}{1-\lambda} =
  \sum_{i=0}^{\ell-k}\lambda^i < \ell-k+1$, and $\Value(v) + W\geq 0$
  (the mean payoff is an average of costs of the arena, so it is in
  the interval $[-W,W]$), we have
  \[\DiscountedPayoff_\lambda(\play)
    \geq -W + \lambda^\ell (\Value(v) + W)\] Finally, notice that
  $\ell\leq n$, so that
  $\lambda^\ell\geq \lambda^{n}>(1-n(1-\lambda))$, using the fact
  that
  $\frac{1-\lambda^{n}}{1-\lambda} = \sum_{i=0}^{n-1}\lambda^i <
  n$. Therefore,
  \begin{align*}
    \DiscountedPayoff_\lambda(\play)
    &\geq -W + (1-n(1-\lambda)) (\Value(v) + W)\\
    &= -n(1-\lambda)(W+\Value(v)) + \Value(v)\\
    &\geq -2n(1-\lambda)W + \Value(v)
  \end{align*}
  by using again $\Value(v)\leq W$. We obtain
  \[\Value_\lambda(v)-\Value(v)\geq -2n(1-\lambda)W\]
  as wanted, which allows us to conclude. 
\end{proof}

Therefore, by picking $\lambda = 1-\frac 1{4n^2(n-1)W}$, we may
obtain a good enough approximation of the mean payoff optimal value,
by solving the associated discounted payoff game. From a complexity
point of view, this value iteration algorithm runs in polynomial time
with respect to the arena, but exponential with respect to the
representation of $\lambda$: here, it is therefore polynomial in
$4n^2(n-1)W$ which leads to a pseudopolynomial complexity to solve
mean payoff games. More precisely,

\begin{theorem}[Direct value iteration]\label{4-thm:MP-direct-value-iteration}
  The direct value iteration algorithm computes the values of a
  mean payoff game in complexity $O(mn^3W)$.
\end{theorem}

As for discounted payoff game, a binary search also
permits to obtain optimal positional strategies for both players in
$\bigO\big(n^4m\log(m/n)W\big)$ time.

Notice that the previous encoding implies a better theoretical
complexity for mean payoff and parity games, that what we obtained
before:
\begin{corollary}[Complexity]\label{4-col:UP}
  Deciding the winner (with respect to a threshold) for mean payoff
  games, and deciding the winner for parity games, can be done in
  $\UP\cap\coUP$.
\end{corollary}
\begin{proof}
  The previous polynomial-time reduction from mean payoff to
  discounted payoff games allows to lift the $\UP\cap\coUP$ complexity
  of~\cref{4-thm:disc-up}. Moreover, the polynomial-time reduction
  of~\cref{4-thm:parity2MP} allows one to obtain the same complexity for
  parity games. 
\end{proof}


\section{Shortest path games}
\label{4-sec:shortest_path}
The quantitative objective $\Sup$ generalises the qualitative
objective $\Reach$ by stating numerical preferences on the target. 
Another quantitative extension of the reachability objective is to quantify the cost of a path towards the target.
We define the quantitative objective $\ShortestPath$ over the set of colours $C = \Z \cup \set{\Win}$ by
\[
\ShortestPath(\rho) =
  \begin{cases}
    - \infty & \text{if } \rho_k \neq \Win \text{ for all } k \\
    - \sum_{i = 0}^{k-1} \rho_i & \text{for $k$ the first index such that } \rho_k = \Win.
  \end{cases}
\]
Two remarks are in order.
\begin{itemize}
	\item Recall that in our definition of quantitative objectives Eve wants to maximise the outcome,
which is why we introduce $\ShortestPath$ with a minus sign:
we interpret the weights as costs and Eve is trying to reach the target with the smallest possible cost.
	\item We use the same abusive terminology as for the shortest path graph problem: the ""cost"" of a path is the sum of the weights along it
(until the first occurence of $\Win$) and we are looking for a path of minimal cost, hence not necessarily the shortest in number of edges.
\end{itemize}

Solving shortest path games in full generality is not easy; we will come back to it at the end of this chapter using some results obtained along the way.
Let us first illustrate the difficulties, and then solve the special case where the weights are non-negative.
We fix a shortest path game $\Game$.
Recall that by definition:
\[
\val(v) = \sup_{\sigma} \inf_{\tau} \ShortestPath(\pi^v_{\sigma,\tau}).
%\val(v) = \sup_{\sigma} \val^\sigma(v) = \inf_{\tau} \val^\tau(v) = \sup_{\sigma} \inf_{\tau} \ShortestPath(\pi^v_{\sigma,\tau}).
\]
Hence for a vertex $v$ there are three possibilities: 
\begin{itemize}
	\item $val(v) = -\infty$, meaning that Eve cannot ensure to reach $\Win$ with a finite cost,
	\item $\val(v) \in \Z$, meaning that Eve can ensure to reach $\Win$ with a finite cost,
	\item $\val(v) = \infty$, meaning that Eve can ensure to reach $\Win$ with arbitrarily negative cost.
\end{itemize}
As we will see, detecting whether $\val(v) = -\infty$ is easy: this is equivalent to asking whether 
$v \notin \AttrE(\Win)$ as stated in \cref{4-lem:detecting_minus_infinity}.
The second case where $\val(v) \in \Z$ will be an occasion to revisit the attractor computation from \cref{2-sec:reachability}
in a quantitative setting.
Most of the difficulty lies in the third case, where 

%Unfortunately, Floyd-Warshall's or Bellman-Ford's algorithms for
%finding shortest paths in a weighted graph (with arbitrary weights) do
%not generalise well to two-player games. Indeed the common difficulty
%with these algorithms is the treatment of negative cycles: in a
%weighted graph, such a negative cycle results in the absence of
%shortest path, while in a weighted game, a negative cycle is only
%profitable for Adam if he is in position to \emph{control} it, i.e.~to
%force Eve to stay in it as long as he wants, before reaching a
%target. Apart from the reachability objective, the search for control
%of negative cycles naturally leads us to the study of mean payoff
%games. We will afterwards return to the study of the $\ShortestPath$
%objective with negative weights in~\cref{4-sec:shortestpath}.

\begin{lemma}
\label{4-lem:detecting_minus_infinity}
Let $\Game$ a shortest path game and $v$ a vertex.
Then $\val(v) = -\infty$ if and only if $v \notin \AttrE(\Win)$.
\end{lemma}
\begin{proof}
\mynote{TO DO}
\end{proof}

\begin{figure}
\centering
  \begin{tikzpicture}[scale=1.3]
    \node[s-eve] (v0) at (0,0) {$\begin{array}{c} v_0 \\ -1 \end{array}$};
    \node[s-eve] (v1) at (2,0) {$\begin{array}{c} v_1 \\ \Win \end{array}$};
    % create edges
    \path[arrow]
      (v0) edge[selfloop=180] (v0)
      (v0) edge (v1)
      (v1) edge[selfloop=0] (v1);
  \end{tikzpicture}
\caption{An example of a shortest path game with negative weights Eve does not have an optimal strategy.
Indeed $\val(v_0) = \infty$ since for any $k$, Eve has a strategy ensuring that $\ShortestPath$ is $k$
by using $k$ times the self loop $-1$ before reaching $\Win$.
However, if she never reaches $\Win$ the outcome is $-\infty$.}
\label{1-fig:optimal_strategies_shortest_path_game}
\end{figure}

\subsection*{Shortest path games with non-negative weights}
\begin{theorem}
\label{4-thm:shortest path-positive}
Shortest path games with non-negative weights are uniformly positionally determined for both players\footnote{This positionality result does not extend to infinite games.}.
There exists a value iteration algorithm for computing the value function of these games in polynomial time and space.
\mynote{More precisely}, 
%let $k$ be the number of different weights in the game,
%the time complexity is $O(m)$ for objective $\Sup$ and $O(knm)$ for objective $\LimSup$,
%and for both algorithms the space complexity is $O(m)$.
\end{theorem}

We rely on the high-level presentation of value iteration algorithms given in \cref{1-sec:value_iteration}.
%Our first lemma shows the existence of optimal strategies.
\begin{lemma}
\label{4-lem:optimal_strategies_shortest_path_games}
Let $\Game$ be a shortest path game with non-negative weights, then there exists an optimal strategy $\sigma$ for Eve.
\end{lemma}

\begin{proof}
Thanks to the assumption that the weights are positive,
$\ShortestPath$ takes value in the non-positive integers, in particular is a set of integers bounded from above.
This implies that the supremum is indeed a maximum.
\end{proof}

\Cref{1-fig:optimal_strategies_shortest_path_game} shows that the assumption that all weights are non-negative 
in \cref{4-lem:optimal_strategies_shortest_path_games} is necessary.

We consider the complete lattice $Y = -\N \cup \set{-\infty}$ equipped with the natural order and the function $\delta : Y \times C \to Y$ defined by
\[
\delta(x, w) = 
\begin{cases}
0 & \text{ if } w = \Win \\
x - w & \text{ if } w \in \N.
\end{cases}
\]
We let $F_V$ be the lattice of functions $V \to Y$ equipped with the componentwise order induced by $Y$.
Note that $\delta$ is monotonic, it induces the monotonic operator $\Op : F_V \to F_V$ defined by:
\[
\Op(f)(v) = 
\begin{cases}
\max \set{\delta( f(v'), w) : (v,w,v') \in E} & \text{ if } v \in \VE, \\
\min \set{\delta( f(v'), w) : (v,w,v') \in E} & \text{ if } v \in \VA.
\end{cases}
\]
Thanks to \cref{1-thm:kleene}, the operator $\Op$ has a greatest fixed point which is also the greatest post-fixed point of $\Op$.
The latter are functions $f \in F_V$ such that $f \le \Op(f)$ and called progress measures.

\begin{lemma}
Let $\Game$ be a shortest path game with non-negative weights, then $\val$ is the greatest fixed point of $\Op$.
\end{lemma}

\begin{proof}
We show the following two properties:
\begin{itemize}
	\item $\val$ is a progress measure;
	\item for every progress measure $f$ we have $f \le \val$.
	
\end{itemize}
Since the greatest fixed point of $\Op$ is also the greatest progress measure, this implies the result.

We show the first item.
Thanks to \cref{4-lem:optimal_strategies_shortest_path_games} there exists $\sigma$ an optimal strategy for Eve.
%and $\tau$ an optimal strategy for Adam.
Consider a vertex $v$.
If $v \in \VE$ we need to show that 
\[
\val(v) \le \max \set{\delta(\val(v'), w) : (v,w,v') \in E},
\]
and if $v \in \VA$ that 
\[
\val(v) \le \min \set{\delta(\val(v'), w) : (v,w,v') \in E}.
\]
Let $(v,w,v') \in E$ a move consistent with $\sigma$: if $v \in \VE$ then 
$\sigma(v) = (v,w,v')$, if $v \in \VA$ this is any outgoing edge of $v$,
we show that $\val(v) \le \delta(\val(v'), w)$ which implies both inequalities.

If $\val(v) = -\infty$ then the inequality holds, so we can assume that $\val(v) \neq -\infty$.

We distinguish two cases.
If $w = \Win$ then $\delta(\val(v'), \Win) = 0$ so the inequality holds.
Otherwise $w \in \N$. 
Let us assume towards contradiction that $\val(v) > \delta(\val(v'), w) = \val(v') - w$.
Let $\sigma' = \sigma_{\mid (v,w,v')}$ the strategy induced by $\sigma$ after playing $(v,w,v')$.
Let $\play'$ a play consistent with $\sigma'$ from $v'$.
The play $\play = (v,w,v') \play'$ is consistent with $\sigma$ from $v$ and since $\sigma$ is optimal 
this implies that $\ShortestPath(\play) \ge \val(v)$.
Hence every play consistent with $\sigma'$ from $v'$ ensure $\val(v) + w$, which is strictly greater than $\val(v')$
contradicting the definition of $\val(v')$.
Thus the inequality $\val(v) \le \delta(\val(v'), w)$ holds.

\vskip1em
We now show the second item.
Let $f$ be a progress measure, we define a positional strategy $\sigma$
by $\sigma(v) = (v,w,v')$ such that $f(v) \le \delta(f(v'), w)$, and show that $f \le \val$.
We consider a vertex $v$ and show that for every play $\play$ consistent with $\sigma$ from $v$ we have 
$f(v) \le \ShortestPath(\play)$. 
This is easily shown for finite plays by induction on the length and then for infinite plays by taking the limit.
This implies that $f(v) \le \val^\sigma(v) = \inf_\tau \ShortestPath(\play^v_{\sigma,\tau})$,
and then $f(v) \le \sup_{\sigma} \val^\sigma(v) = \val(v)$.
\end{proof}

Thanks to \cref{1-thm:kleene} $\val$ can be computed by a greatest fixed point algorithm.
To obtain the announced complexity we carefully define the data structure.

The pseudocode is given in \cref{4-algo:value_iteration_shortest_path_non_negative}.

\begin{algorithm}
 \KwData{A shortest path game with non-negative weights.}
 \SetKwFunction{FTreat}{Treat}
 \SetKwFunction{FInit}{Initialisation}
 \SetKwFunction{FVI}{ValueIteration}
 \SetKwProg{Fn}{Function}{:}{}
 \DontPrintSemicolon
 
\Fn{\FVI{}}{
	\FInit{$\ell_V, \ell_E$}
	
	$S \leftarrow V$
	
	\vskip1em
	\Repeat{$S$ empty}{
		Choose $v$ in $S$ minimal with respect to $\ell_V$ and remove it from $S$

		\FTreat($v$) 
	}

	\Return{$\ell_V$}
}

\vskip1em
\Fn{\FTreat{$v$}}{
	\For{$e = (u,c,v) \in E$}{
	
		\If{$u \in \VE$}{
			$\ell_E(e) \leftarrow \max(c + \ell_V(v),\ell_E(e))$ 

			$\ell_V(u) \leftarrow \max(\ell_E(e),\ell_V(u))$ 
		}
		
		\If{$v \in \VE$}{
			$\ell_E(e) \leftarrow \min(c + \ell_V(v),\ell_E(e))$ 

			$\ell_V(u) \leftarrow \min(\ell_E(e),\ell_V(u))$ 
		}
	}
}
\caption{The value iteration algorithm for shortest path games with non-negative weights.}
\label{4-algo:value_iteration_shortest_path_non_negative}
\end{algorithm}

\begin{proof}
  We follow a similar approach as in Dijkstra's algorithm, keeping an
  estimation $\ell(v)\in\R$ of the shortest paths towards the target
  $\Win$ from vertices $v$ in a set $S$, still to be considered; these
  estimations are refined greedily along the computation. We
  initialise $S$ to $V$, and all values $\ell(v)$ to $+\infty$, except
  for vertices just after an edge of the target $\Win$ that we put at
  $0$. For vertices $u$ of Eve, we need more information and thus keep
  track of a mapping from successors of $u$ towards the value
  $\ell(u,v)\in\R$ of the current shortest path from $u$ to $\Win$,
  going through the edge $(u,v)$. We have that $\ell(u)$ is the
  maximal value of $\ell(u,v)$ for all successor vertices $v$: in
  particular, as long as one of the successors $v$ still has value
  $\ell(v)=+\infty$, the value $\ell(u)$ remains $+\infty$.

  The algorithm consists in iteratively picking a minimal element $v$
  of $S$, with respect to $\ell(v)$, remove it from $S$, and update
  $\ell(u)$ for all vertices $u$ such that $(u,v)\in E$:
  \begin{itemize}
  \item if $u\in \VA$, $\ell(u)$ is updated to $c(u,v)+\ell(v)$
    whenever the latter value is smaller than $\ell(u)$;
  \item if $u\in \VE$, $\ell(u,v)$ is updated to $c(u,v)+\ell(v)$, and
    the value of $\ell(u)$ is updated accordingly.
  \end{itemize}

  This continues until there are no more vertices in $S$ after which
  we return the values $\ell(u)$ that we can prove to be the actual
  values $\Value(u)$ of each vertex $u$. Let us denote by $S_i$ and
  $\ell_i(u)$ the values of $S$ and $\ell(u)$ in iteration $i$. We
  prove the following invariants:
  \begin{enumerate}
  \item for all iterations $i$, $\ell_i(u)$ is equal to the value
    $\Value_{\game_i}(u)$ in the shortest path game~$\game_i$ obtained
    from $i$ by replacing the cost $c(v,w)$ of each edge with
    $+\infty$ if both endpoints $v$ and $w$ are still in $S_i$;
  \item moreover,
    $\min\{\ell_i(v)\mid v\in S_i\}\geq \max\{\Value_{\game}(v)\mid
    v\notin S_i\}$ which generalises the greedy property crucial to
    the correctness of Dijkstra's algorithm.
  \end{enumerate}
  Since the cost of each edge in $\game_i$ only decreases along the
  various iterations, it is also the case for the values
  $\Value_{\game_i}(u)$. More precisely, the invariant shows that
  $\ell_i(u)=\Value_{\game_i}(u)$ for all $u\in S_i$, and
  $\ell_i(u)=\Value_{\game}(u)$ for all $u\notin S_i$. The proofs are
  very similar to the usual proofs and can be found in great details
  in~\cite{Khachiyan&al:2008}.

  From a complexity point of view, there is no doubt that the various
  computations can be performed in polynomial time. A careful
  analysis, using (minimum) Fibonacci heaps, as in Dijkstra's
  algorithm, allows one to obtain an overall complexity
  $\bigO(m+n\log(n))$. 
\end{proof}

\subsection*{Detection of $-\infty$ vertices with mean payoff games}
However, contrary to the previous payoffs, the \emph{positional}
determinacy result no longer holds as shows the example
in~\cref{4-fig:memory}. In this game, there are two positional
strategies for Adam: $\tau_1(v_1)=(v_1,v_0)$ and
$\tau_2(v_1)=(v_1,v_2)$. Strategy $\tau_1$ does not guarantee Adam to
reach the target, since Eve can play the cycle $(v_1,v_0)$ forever and
obtain payoff $+\infty$. Strategy $\tau_2$ guarantees payoff
$0$. However, Adam can play smarter by scaring Eve. Imagine that Adam
plays once to $v_0$, and then switches to $v_2$: he then guarantees a
value $-1$. Doing so one more time, he can guarantee value $-2$. And
so on, until he decides to do it $50$ times, showing to Eve that he is
able to get $-50$. Then, the optimal decision for Eve is to go
directly to the target. Therefore, Adam needs memory to play optimally
in this example: his optimal strategy is to go to $v_0$ the first 50
times the play visits $v_1$, and switch to $v_2$ after the 50th
time. However Eve still has a positional optimal strategy that
consists in always going to $v_2$. This is always the case as we
discuss later. Notice that the possible absence of optimal positional
strategies for Adam makes non-trivial an upper bound of the form
$\NP\cap\coNP$ in the complexity of solving shortest path games.

\begin{figure}[tbp]
  \centering
    \begin{tikzpicture}
    \node[s-eve,accepting](2){$v_2$};%
    \node[s-eve,above left of=2,xshift=5mm](0){$v_0$};%
    \node[s-adam,above right of=2,xshift=-5mm](1){$v_1$};%

    \path[->] (0) edge[bend left] node[above]{$-1$} (1)%
    (1) edge[bend left] node[below]{0} (0)%
    (0) edge node[below left]{$-50$} (2)%
    (1) edge node[below right]{$0$} (2)%
    (2) edge[selfloop=0] node[right]{0}(2);%
    
  \end{tikzpicture}
\caption{A shortest path game, with $v_2$ being the target vertex,
    where Adam needs memory to play optimally}
  \label{4-fig:memory}
\end{figure}

Apart from the complication on the memory requirement for Adam to play
optimally, one other technical difficulty arises from the presence of
vertices with optimal value $-\infty$: this is the case when Adam may
reach a target of $\Win$ while controlling a negative cycle along the
way. As previously announced, this is closely related with mean payoff
games: \cite{Brihaye&Geeraerts&HaddadA&Monmege:2017}
\begin{theorem}\label{4-thm:-infty-MP}
  Let $\arena$ be an arena.
  \begin{itemize}
  \item If the game $\game=(\arena,\ShortestPath(\Win))$ has no
    vertices $v$ of value $\Value^\game(v)=+\infty$, and the only
    outgoing edges of vertices in $\Win$ are self loops of weight $0$,
    then for all vertices $v$, $\Value^\game(v)=-\infty$ if and only
    if $\Value^{\game'}(v)<0$ if $\game'=(\arena,\MeanPayoff)$ is the
    associated mean payoff game on the same arena.
  \item Reciprocally, if $\game=(\arena,\MeanPayoff)$, we can build in
    polynomial time a game $\game'=(\arena',\ShortestPath(\Win))$ such
    that $\Value^\game(v)<0$ if and only if
    $\Value^{\game'}(v) =-\infty$.
  \end{itemize}
\end{theorem}
\begin{proof}
  For the first item, if $\Value^{\game'}(v)<0$, there exists a
  profile of optimal positional strategies $(\sigma^*,\tau^*)$: the
  outcome starting in $v$ and following this profile ends up in a loop
  with total cost $<0$. For every $M>0$, we can construct a strategy
  $\tau^M$ for Adam that ensures in $\game$ a payoff at most $-M$,
  which then proves that $\Value^\game(v)=-\infty$: strategy $\tau^M$
  is obtained by playing strategy $\tau^*$ until the accumulated cost
  is less than $-M-nW$, with $W=\max_{(v,c,v')\in E} |c|$, after
  which he switches to an attractor strategy towards $\Win$ (which
  exists from every vertex of the game, since no vertices have value
  $+\infty$). Since the attractor strategy reaches the target in at
  most $n$ steps, the value of $\tau^M$ from $v$ is at most $-M$.

  Reciprocally, if $\Value^\game(v)=-\infty$, for $M=nW$,
  consider a strategy $\tau^M$ of Adam guaranteeing a payoff less than
  $-M$. Consider, towards a contradiction, a positional strategy
  $\sigma$ of Eve that secures a non-negative mean payoff. The play of
  $\game$ following the profile $(\sigma,\tau^M)$ necessarily leads to
  $\Win$, while visiting at least one negative cycle, around a given
  vertex $v'$. If $v'\in \VE$, Eve is not the one choosing to exit the
  cycle (since she is following a positional strategy), so Adam can
  modify its strategy to stay forever in the negative cycle, which
  contradicts the fact that $\sigma$ secures a non-negative
  mean payoff. If $v'\in \VA$, Adam can also choose to stay forever in
  the negative cycle by modifying his strategy. Therefore, Eve cannot
  have a positional strategy securing a non-negative mean payoff: by
  positional determinacy of mean payoff games
  (\cref{4-thm:mean_payoff_positional}), Eve cannot have any strategy securing
  a non-negative mean payoff.

  \medskip For the second item, without loss of generality, suppose
  that $\arena$ is a bipartite arena,
  i.e.~$E\subseteq \VA\times \VE\cup\VE\times \VA$. The new arena
  $\arena'$ is obtained by adding a fresh target $v_t$ with edges
  $(v,v_t)$ for all $v\in \VA$, as well as an edge $(v_t,v_t)$, all of
  cost 0. Consider the game $\game'=(\arena',\ShortestPath(\Win))$
  with $\Win = \{v_t\}$. By construction and using the bipartite
  hypothesis, Adam always has a strategy to reach $\Win$, so that no
  vertices $v$ have a value $\Value^{\game'}(v)=+\infty$. By letting
  $\game''=(\arena',\MeanPayoff)$, the previous item shows that
  $\Value^{\game'}(v)=-\infty$ if and only if
  $\Value^{\game''}(v)<0$. To conclude, it only remains to show that
  $\Value^{\game}(v)<0$ if and only if $\Value^{\game''}(v)<0$. By
  mapping positional strategies from $\game$ to $\game''$, we easily
  obtain $\Value^{\game''}(v)\leq \Value^\game(v)$, so the direct
  implication holds. For the converse, if $\Value^{\game''}(v)<0$, the
  target $v_t$ cannot be visited by a profile of positional optimal
  strategies (otherwise the mean payoff would be 0): projecting the
  play from $\game''$ on $\game$ therefore shows that
  $\Value^{\game}(v)\leq \Value^{\game''}(v) <0$. 
\end{proof}

\subsection*{A pseudopolynomial time value iteration algorithm}
As shown above, we can detect vertices of value $+\infty$ and
$-\infty$ if needed. We now explain how to compute the exact optimal
value of other vertices, by a value iteration algorithm. Similarly to
the case of mean payoff or discounted games, the algorithm consists in
an iterative search of a fixed point of the operator
$F\colon \R^V\to \R^V$ mapping every vector $\vec x=(x_v)_{v\in V}$ to
the new vector $(y_v)_{v\in V}$, defined, for all $v\in V$, by:
\[y_v =
  \begin{cases}
    0 & \text{ if } v\in \Win\\
    \max_{(v,v')\in E} [c(v,v') + x_{v'}] &
    \text{ if } v\in \VE\setminus \Win\\
    \min_{(v,v')\in E} [c(v,v') + x_{v'}] & \text{ if } v\in
    \VA\setminus \Win
  \end{cases}\]

Notice the similarity with respect to the $\Lift$ operator used in the
value iteration algorithm for mean payoff
(\cref{4-algo:value_iteration_MP}): the following arguments are thus
very resembling to the ones already presented in the case of
mean payoff. Here, we obtain a more precise information though, since
it directly gives us the values for the shortest path objective
towards a fixed target.

In the presence of vertices of value $-\infty$, the iterative fixed
point computation would not terminate. However, thanks to the
following lemma (where we again let $W=\max_{(v,c,v')\in E} |c|$), we
know that finite values are bounded below, so that an intermediate
step of speed-up can detect the vertices of value $-\infty$.

\begin{lemma}\label{4-lem:-infty}
  In a shortest path game $\game$, all vertices $v$ with a value
  $\Value(v)<-(n-1) W$ have value $\Value(v)=-\infty$.
\end{lemma}
\begin{proof}
  Consider a strategy $\tau$ of Adam securing a value $<-(n-1) W$
  from a given vertex $v$. We show that in the mean payoff game
  $\game'$ described in~\cref{4-thm:-infty-MP}, vertex $v$ has value
  $<0$, which allows us to conclude. Let $\sigma$ be a positional
  strategy of Eve. By hypothesis, the play $\pi$ starting in $v$ and
  following the profile $(\sigma,\tau)$ has a payoff
  $\ShortestPath(\pi)<-(n-1) W$: therefore, it contains a negative
  cycle. As before, Adam can therefore modify its strategy so that he
  sticks to the choices he does in the first cycle he visits: this new
  strategy is indeed independent of $\sigma$, so that the according
  strategy indeed secures a negative mean payoff.
\end{proof}

Thus, we let $G\colon\overline\R^V\to\overline\R^V$ mapping each
vertex $\vec x = (x_v)_{v\in V}$ to the mapping $(y_v)_{v\in V}$
defined by $y_v = x_v$ if $x_v\geq -(n-1)W$ and $y_v=-\infty$
otherwise. We consider then the sequence
$(\vec x^n = (GF)^n(\top))_{n\in \N}$, with $\top$ the vector having
all components equal to $+\infty$. Since it generalises the attractor
computation, after $n$ steps, vertices with a value
$\Value(v)<+\infty$ have been discovered: they are the only ones such
that $x^{n}_v < +\infty$. Moreover, these vertices satisfy
$x^{n}_v\leq n W$ since a given path towards the target has been
discovered along the first $n$ iterations. Since the mapping $GF$ is
monotonous, the sequence $(\vec x^n)_{n\in \N}$ is
non-increasing. Since it can only take values in the finite set
$\{-\infty\}\cup \{-(n-1) W+1, -(n-1) W+2,\ldots,nW\}
\cup\{+\infty\}$, it is stabilising: there exists a step $N$ such that
$\vec x^N=\vec x^{N+1}$. Notice that an a priori bound on $N$ is
$(2n-1)W n + n$. A careful analysis allows one to show that
$\vec x^N=\Value^\game$: the main argument is the fact that the
stabilisation of the sequence at index $N$ allows one to show by
induction that $N$ steps suffice for Adam to guarantee that he has
reached a target vertex while getting the optimal value.

\cite{Brihaye&Geeraerts&HaddadA&Monmege:2017}
\begin{theorem}\label{4-thm:SP-pseudopoly-algo}
  We can compute in pseudopolynomial time the values of a
  shortest path game. 
\end{theorem}

% This computation also permits to compute optimal strategies for both
% players: while Eve always has optimal positional strategies, Adam
% may require pseudopolynomial memory to play optimally.



\section{Total payoff games}
\label{4-sec:total_payoff}
Yet another interesting quantitative objective---that is closely
related with shortest path objective---is the total payoff defined by
\[\TotalPayoff(\pi)=\limsup_{n} \sum_{i=0}^{n-1} c(\pi_i)\]
Contrary to the shortest path objective, total payoff games have no
reachability objective intertwined with the quantitative objective. In
particular, all plays will be infinite (if there are no deadlocks in
the arena, which we suppose without loss of generality) and their
payoff is the superior limit of the partial sums: we need this
superior limit since partial sums might not have a limit (consider for
instance the sequence of costs $1,-1,1,-1,1,\ldots$ whose partial sums
alternate between $1$ and $0$).

This objective is also closely related to the mean payoff, in the
sense that it refines the mean payoff objective. Indeed, notice that
the total payoff of a play is finite if and only if the mean payoff of
this play is null. Then, the total payoff enables a finer, two-stage
analysis of a game: first, compute the mean payoff of a given start
vertex $v$; then subtract this value from all edge weights and scale
the resulting weights if necessary to obtain integers (the resulting
game has now mean payoff 0 starting from $v$); finally, compute the
(finite) total payoff of the new game to quantify how the partial sums
are fluctuating around the mean payoff of the original game. For
instance, this allows one to distinguish situation
$1,-1,1,-1,1,\ldots$ where the total payoff is $1$ from a close
situation $-1,1,-1,1,\ldots$ with total payoff $0$: this could be seen
as a budget constraint that players must satisfy in order to be able
to survive the game. Total payoff is therefore also closely related to
\emph{energy games} where Eve tries to optimise its energy consumption
(again computed with partial sums) while keeping at all time the
energy level above $0$. However, no trivial reduction exist to encode
total payoff games into energy games (that will be solved
in~\cref{chap:counters})
%\todo{More precise?}).

It should not be surprising that total payoff games are determined
but, contrary to shortest path games, positional strategies are now
sufficient for both players to play optimally in total payoff
games. Instead of giving yet another distinct proof, like for
mean payoff or discounted payoff games, we give a general recipe
defined by Gimbert and Zielonka \cite{Gimbert&Zielonka:2004}. They define
sufficient conditions for a quantitative objective to fulfil the
positional determinacy.

\begin{definition}[Fairly mixing]\label{4-def:fairly-mixing}
  A payoff $\mathsf{P}\colon C^\omega\to \overline R$ is \emph{fairly
    mixing} if:
  \begin{enumerate}
  \item for all $x\in C^+$, $y_0,y_1\in C^\omega$, if
    $\mathsf{P}(y_0)\leq \mathsf{P}(y_1)$ then
    $\mathsf{P}(xy_0)\leq \mathsf{P}(xy_1)$;
  \item for all $x\in C^+$, $y\in C^\omega$,
    $\min(\mathsf{P}(x^\omega),\mathsf{P}(y)) \leq \mathsf{P}(xy)\leq
    \max(\mathsf{P}(x^\omega),\mathsf{P}(y))$;
  \item if $(x_i)_{i\in \N}$ is a sequence of non-empty words
    $x_i\in C^+$ such that $x_0x_1x_2\cdots$ contains only a finite
    number of colours, and $I\uplus J=\N$ is a partition of $\N$ into two
    infinite sets, then
    \[\inf(\mathcal{P}_I\cup \mathcal{P}_J) \leq
      \mathsf{P}(x_0x_1x_2\cdots) \leq \sup(\mathcal{P}_I\cup
      \mathcal{P}_J)\] with
    $\mathcal P_I=\{\mathsf{P}(x_{i_0}x_{i_1}x_{i_2}\cdots)\mid i_k\in
    I\}$ and
    $\mathcal{P}_J=\{\mathsf{P}(x_{j_0}x_{j_1}x_{j_2}\cdots)\mid
    j_k\in J\}$.
  \end{enumerate}
\end{definition}

It is not difficult to convince oneself that
\begin{proposition}[Previous objectives are fairly mixing]\label{4-prop:objectives-fairly}
  Quantitative objectives $\Inf$, $\Sup$, $\LimInf$, $\LimSup$,
  $\Parity$ (mapping $1$ to sequences whose the greatest colour seen
  infinitely often is even), $\MeanPayoff$,
  $\DiscountedPayoff_\lambda$, $\TotalPayoff$ are fairly mixing
  payoffs.
\end{proposition}

Then, a rather technical proof by induction on the number of vertices
in the arena allows one to get the following strong result \cite{Gimbert&Zielonka:2004}:

\begin{theorem}[Fairly mixing induces positional determinacy]\label{4-thm:fairly-mixing}
  If $\mathsf P$ is a fairly mixing payoff function, then all finite
  games $(\arena,\mathsf P)$ are positionally determined.
\end{theorem}

\begin{corollary}[Total payoff games]\label{4-cor:TP-determinacy}
  Total payoff games are positionally determined.
\end{corollary}

In particular, it gives $\NP\cap\coNP$ complexity to solve
total payoff games, since one-player total payoff games can be solved
in polynomial time using Floyd-Warshall algorithm to compute all-pairs
shortest path in a weighted graph. Obtaining a deterministic algorithm
is nonetheless not as immediate as before, using the operator
$F\colon \R^V\to \R^V$ mapping every vector $\vec x=(x_v)_{v\in V}$
towards the new vector $(y_v)_{v\in V}$, defined, for all $v\in V$,
by:
\[y_v =
  \begin{cases}
    \max_{(v,v')\in E} [c(v,v') + x_{v'}] &
    \text{ if } v\in \VE\\
    \min_{(v,v')\in E} [c(v,v') + x_{v'}] & \text{ if } v\in \VA
  \end{cases}\]

Indeed, this operator may have several fixed points, and even more
problematic is the fact that the value of the game may be different
from the greatest and lowest fixed points, forbidding to find the
correct fixed point easily. Consider for example the total payoff game
of~\cref{4-fig:totalpayoff}. Operator $F$ is then
$F(x_0,x_1,x_2) = \big(\max(x_1-1,x_2-2),x_0+1,x_0+2\big)$. Its fixed
points are all vectors $(a,a+1,a+2)$ with $a\in \overline R$: in
particular, its greatest fixed point is $(+\infty, +\infty, +\infty)$
and its lowest fixed point is $(-\infty, -\infty, -\infty)$. However,
the optimal value vector is $(0,1,2)$.

Instead, to compute the value, a computation based on two nested fixed
points exists, relying upon the encoding of a total payoff game into a
pseudopolynomial size shortest path game---still with the same
costs---resulting in a pseudopolynomial time algorithm \cite{Brihaye&Geeraerts&HaddadA&Monmege:2017}.

\begin{theorem}[Solution of total payoff games]\label{4-thm:TP-optimal-strategies}
  We can compute the optimal values of total payoff games, as well as
  positional optimal strategies for both players, with a
  pseudopolynomial time complexity.
\end{theorem}
\begin{proof}%[Sketch of proof]
  Let $\arena$ be the arena of the total payoff game. Consider an
  arena $\arena^k$ consisting of $k$ modified copies of $\arena$ as
  well as a fresh target vertex $v_f$: in the $j$-th copy of $\arena$
  (with $1\leq j\leq k$), every time an edge is taken, Adam has the
  opportunity (encoded via a copy of each vertex, owned by Adam) to
  exit the $j$-th copy when he wants, then giving the token to Eve
  that has the choice between continuing the game in the $(j-1)$-th
  copy (unless $j=1$) or stopping the game by jumping into the target
  $v_f$. In the first copy of the game, Eve has no choice but to allow
  Adam to stop whenever he decides. However, when $k$ increases, the
  fact that Eve can delay the stopping of the game for a great number
  of times precludes Adam to cheat by asking to stop while obtaining a
  too low value. We can prove, by a careful analysis, that the
  shortest path game played on arena $\arena^K$ with
  $K= n\big((2n-1)W +1\big)$ has a value equal to the
  total payoff game on arena $\arena$.

  By using the previous value iteration techniques, we obtain a
  pseudopolynomial time algorithm to compute the optimal values in
  the total payoff game. Instead of building the entire arena
  $\arena^k$, we can benefit from the multiple copies of the same
  sub-arena to emulate the construction with an operator $H$ mapping
  the values of each vertex in the $k$-th copy, to the values of each
  vertex in the $k+1$-th copy (operator $H$ does not depend on
  $k$). Because Eve (the maximiser player) is the one that is
  responsible for allowing to exit the game or not, the actual value
  of the $k$-th copy is obtained by $H^k(\bot)$, with $\bot$ the
  vector with all components being $-\infty$. The computation of $H$,
  that is a shortest path game whose output values depend on a given
  vector, depends on another fixed-point computation, as described
  before. This way, we improve the spacial complexity, without
  deteriorating the time complexity.
\end{proof}



\begin{figure}[tbp]
  \centering
  \begin{tikzpicture}
    \node[s-eve](0){$v_0$};%
    \node[s-adam,left of=0](1) {$v_1$};%
    \node[s-adam,right of=0] (2) {$v_2$};%
    
    \path[->] (0) edge[bend right] node[above]{$-1$} (1)%
    (1) edge[bend right] node[below]{$1$} (0)%
    (0) edge[bend left] node[above]{$-2$} (2)%
    (2) edge[bend left] node[below]{$2$} (0);%
  \end{tikzpicture}

  \caption{A total payoff game}
  \label{4-fig:totalpayoff}
\end{figure}




\section*{Bibliographic references}
\label{4-sec:references}
We refer to~\cref{2-sec:references} for the role of parity objectives and how they emerged in automata theory as a subclass of Muller objectives.
Another related motivation comes from the works of Emerson, Jutla, and Sistla~\cite{Emerson&Jutla&Sistla:1993},
who showed that solving parity games is linear-time equivalent to the model-checking problem for modal $\mu$-calculus.
This logical formalism is an established tool in program verification, and a common denominator to a wide range of modal, temporal and fixpoint logics used in various fields.

\vskip1em
Let us discuss the progress obtained over the years for each of the three families of algorithms.

\vskip1em
\textit{Value iteration algorithms and separating automata}.
The heart of value iteration algorithms is the value function, which in the context of parity games and related developments for automata
have been studied under the name progress measures or signatures.
They appear naturally in the context of fixed point computations so it is hard to determine who first introduced them.
Streett and Emerson~\cite{Streett&Emerson:1984,Streett&Emerson:1989} defined signatures for the study of the modal $\mu$-calculus,
and Stirling and Walker~\cite{Stirling&Walker:1989} later developped the notion.
Both the proofs of Emerson and Jutla~\cite{Emerson&Jutla:1991} and of Walukiewicz~\cite{Walukiewicz:1996} use signatures to show the positionality of parity games over infinite games.

Jurdzi{\'n}ski~\cite{Jurdzinski:2000} used this notion to give the first value iteration algorithm for parity games, 
with running time $O(m n^{d/2})$.
The algorithm is called ``small progress measures'' and is an instance of the class of value iteration algorithms we construct 
in~\cref{3-sec:value_iteration} by considering the universal tree of size $n^h$.
Bernet, Janin, and Walukiewicz~\cite{Bernet&Janin&Walukiewicz:2002} investigated reductions from parity games to safety games
through the notion of permissive strategies, and constructed a separating automaton\footnote{We note that the general framework of separating automata came later, introduced by Boja{\'n}czyk and Czerwi{\'n}ski~\cite{Bojanczyk&Czerwinski:2018}.} corresponding to the universal tree of size $n^h$.

The new era for parity games started in 2017 when Calude, Jain, Khoussainov, Li, and Stephan~\cite{Calude&Jain&al:2017} constructed a quasipolynomial time algorithm. 
Our presentation follows the technical developments of the subsequent paper by Fearnley, Jain, Schewe, Stephan, and Wojtczak~\cite{Fearnley&Jain&al:2017} which recasts the algorithm as a value iteration algorithm.
Boja{\'n}czyk and Czerwi{\'n}ski~\cite{Bojanczyk&Czerwinski:2018} introduce the separation framework to better understand the original algorithm.

Soon after two other quasipolynomial time algorithms emerged.
Jurdzi{\'n}ski and Lazi{\'c}~\cite{Jurdzinski&Lazic:2017} showed that the small progress measure algorithm can be adapted to a ``succinct progress measure'' algorithm, matching (and slightly improving) the quasipolynomial time complexity.
The presentation using universal tree that we follow in~\cref{3-sec:value_iteration} and an almost matching lower bound on their sizes is due to Fijalkow~\cite{Fijalkow:2018}.
The connection between separating automata and universal trees was shown by Czerwi{\'n}ski, Daviaud, Fijalkow, Jurdzi{\'n}ski, Lazi{\'c}, and Parys~\cite{Czerwinski&Daviaud&al:2018}. 

The third quasipolynomial time algorithm is due to Lehtinen~\cite{Lehtinen:2018}.
The original algorithm has a slightly worse complexity ($n^{O(\log(n))}$ instead of $n^{O(\log(d))}$),
but Parys~\cite{Parys:2020} later improved the construction to (essentially) match the complexity of the previous two algorithms.
Although not explicitly, the algorithm constructs an automaton with similar properties as a separating automaton,
but the automaton is non-deterministic.
Colcombet and Fijalkow~\cite{Colcombet&Fijalkow:2019} revisited the link between separating automata and universal trees
and proposed the notion of good for small games automata, capturing the automaton defined by Lehtinen's algorithm.
The equivalence result between separating automata, good for small games automata, and universal graphs, holds for any positionally determined objective, giving a strong theoretical foundation for the family of value iteration algorithms.

\vskip1em
\textit{Attractor decomposition algorithms}.
The McNaughton Zielonka's algorithm has complexity $O(m n^d)$.
Parys~\cite{Parys:2019} constructed the fourth quasipolynomial time algorithm as an improved take over McNaughton Zielonka's algorithm.
As for Lehtinen's algorithm, the original algorithm has a slightly worse complexity ($n^{O(\log(n))}$ instead of $n^{O(\log(d))}$).
Lehtinen, Schewe, and Wojtczak~\cite{Lehtinen&Schewe&Wojtczak:2019} later improved the construction.
As discussed in~\cref{3-sec:relationships} the complexity of this algorithm is quasipolynomial and of the form $n^{O(\log(d))}$,
but a bit worse than the three previous algorithms since the algorithm is symmetric and has a recursion depth of $d$,
while the value iteration algorithms only consider odd priorities hence replace $d$ by $d/2$.

Jurdzi{\'n}ski and Morvan~\cite{Jurdzinksi&Morvan:2020} constructed a generic McNaughton Zielonka's algorithm parameterised by the choice of two universal trees, one for each player.
\mynote{CONTINUE}


\vskip1em
\textit{Strategy improvement algorithms}.
As we will see in~\cref{4-chap:payoff}, parity games can be reduced to mean payoff games,
so any algorithm for solving mean payoff games can be used for solving parity games.
In particular, the existing strategy improvement algorithm for mean payoff games can be run on parity games. 
V{\"o}ge and Jurdzin{\'n}ski~\cite{Voge&Jurdzinski:2000} introduced the first discrete strategy improvement for parity games,
running in exponential time.
For some time there was some hope that the strategy improvement algorithm, for some well chosen policy on switching edges,
solves parity games in polynomial time.
Friedmann~\cite{Friedmann:2011} cast some serious doubts by constructing numerous exponential lower bounds applying to different variants of the algorithm.
Fearnley~\cite{Fearnley:2017} investigated efficient implementations of the algorithm, focussing on the cost of computing and updating the value function for a given strategy.
Our proof of correctness is original. \mynote{SAY MORE?}

The complexity was reduced to subexponential with randomised algorithms 
by Jurdzin{\'n}ski, Paterson, and Zwick~\cite{Jurdzinski&Paterson&Zwick:2008}.
A natural question is whether there exists a quasipolynomial strategy improvement algorithm; 
as discussed in~\cref{3-sec:relationships} the notion of universal trees cannot be used to achieve this,
and the question remains to this day open.


%%% Local Variables:
%%% mode: latex
%%% TeX-master: "../4_standalone"
%%% End:
