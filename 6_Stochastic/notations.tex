Let us first define the arenas stochastic games will be played on:
\begin{definition}
  A ""stochastic arena"" is a tuple $\arena = (\vertices,E,\delta)$ where
  \begin{itemize}
  \item $\vertices = \VA \sqcup \VE \sqcup \Randomvertices$ is a
    finite set of vertices, partitionned into vertices of Adam, Eve,
    and ""random vertices"";
  \item $E \subseteq \vertices \times \vertices$ is the set of
    edges;
  \item $\delta : \Randomvertices \to \dist(\vertices)$ is the
    probabilistic transition function, which satisfies:
    $\forall v \in \Randomvertices$, $\delta(v)(w)>0$ iff
    $(v,w) \in E$.
  \end{itemize}
\end{definition}

\begin{figure}
\centering
    \begin{tikzpicture}[scale=1.3]
    \node[s-eve] (init) at (0,0) {$v_0$};
    \node[s-random] (v1) at (2,0) {$v_1$};
    \node[s-eve] (v2) at (4,0) {$v_2$};    
    \node[s-random] (v3) at (6,0) {$v_3$};
    \node[s-random] (v4) at (0,-1.5) {$v_4$};
    \node[s-eve] (v5) at (2,-1.5) {$v_5$};
    \node[s-adam] (v6) at (4,-1.5) {$v_6$};    
    \node[s-eve] (v7) at (6,-1.5) {$v_7$};
    
    \path[arrow] (init) edge (v1)
    (init) edge[bend left] (v4)
    (v4) edge[bend left] node[left] {$\frac 3 4$} (init)
    (v4) edge node[above] {$\frac 1 4$} (v5)
    (v5) edge[bend left] (v6)
    (v6) edge (v7)
    (v6) edge[bend left] (v5)
    (v5) edge[selfloop=90]  (v5)
    (v7) edge[selfloop]  (v7)
    (v1) edge node[above] {$\frac 2 3$} (v2)
    (v1) edge node[above right] {$\frac 1 3$} (v6)
    (v2) edge (v6)
    (v2) edge[bend left] (v3)
    (v3) edge[bend left] node[below] {$\frac 1 2$} (v2)
    (v3) edge node[right] {$\frac 1 2$} (v7)
%    (init) edge node[left] {$\frac 1 {10}$} (av)
%    (init) edge node[left] {$\frac 3 {10}$} (rv)
%    (init) edge node[right] {$\frac 2 {5}$} (ev)
    ;    
% \node[s-random] (root) at (0,0) {$v$};
% \node[s-random] (left) at (-.75,-1) {};
% \node[s-random] (right) at (.75,-1) {};
% \node[s-adam] (lleft) at (-1.5,-2) {$v_1$};
% \node[s-random] (rleft) at (0,-2) {$v_2$};
% \node[s-eve] (rright) at (.75,-2) {$v_3$};

% \path[arrow] (root) edge node[left] {$\frac 2 5$}  (left)
% (root) edge node[right] {$\frac 3 5$}  (right)
% (left) edge node[left] {$\frac 1 4$} (lleft)
% (left) edge node[right] {$\frac 3 4$} (rleft)
% (right) edge node[right] {$\frac 2 3$} (rright)
% (right) edge[out=70,in=0] node[right] {$\frac 1 3$} (root)
% % (ab) edge [loop] (ab)
% % (notb) edge  node [above] {} (notab)
% % (notab) edge   (ab)
% % (ab) edge[out=195,in=-15] (notab)
% % (ab) edge[out=205,in=-25] (notb)
% ;
    \end{tikzpicture}
  \caption{Example of a stochastic arena: circle nodes belong to Eve,
    square nodes to Adam, and diamond nodes are random.}
  \label{6-fig:ex-stoch-arena}
\end{figure}


Similarly to non-stochastic arenas, one can equip a stochastic arena
with a winning objectives to define a stochastic game.
\begin{definition}
  A ""stochastic game"" is a tuple $\game = (\arena,\Omega)$ where
  \begin{itemize}
  \item $\arena$ is a stochastic arena;
  \item $\Omega \subseteq \vertices^\omega$ is the (qualitative)
    winning objective.
  \end{itemize}
\end{definition}

For $\Win \subseteq \vertices$, letting $\Omega = \Reach(\Win)$ gives
rise to a stochastic reachability game.
Of course, one may consider more general $\omega$-regular
properties. We write $\mathds{1}_\Omega$ for the characteristic
function of the winning objective $\Omega$.


In this game, Eve aims at maximizing the probability that a play
belongs to $\Omega$, while Adam has the opposite objective of
minimizing that probability. In that sense, we study quantitative
games, although the winning objective is qualitative.

A \emph{strategy} for Eve is a function
$\sigma: V^* \VE \to \dist(\vertices)$ such that whenever
$\sigma(h v)(v') >0$ then $(v,v') \in E$. Similarly, one can define
strategies for Adam. When a strategy profile $(\sigma,\tau)$ is
fixed, and assuming the arena is clear from context, we write
$\probm_{\sigma,\tau}^v$ for the probability measure on infinite plays
when the initial vertex is $v$. In particular,
$\probm_{\sigma,\tau}^v(\mathds{1}_\Omega)$ is the outcome of the profile
$(\sigma,\tau)$.


% We note $\probm_{\sigma,\tau}^v$ without the name of the game when it
% is evident.
% We will also write $\sigma(h)(\adist,v')$ 

% \textit{Strategies}
% $\sigma$: strategies of Eve; $\tau$: strategies of Adam

%\textit{Notions  of inf value, sup value}

The \emph{"value" for Eve} in $\game$ from $v$ is defined as
$\ValueE^\game(v) = \sup_{\sigma} \inf_{\tau}
\probm_{\sigma,\tau}^v(\mathds{1}_\Omega)$, whereas symmetrically the
\emph{supremum value} is
$\ValueA^\game(v) = \inf_{\tau}\sup_{\sigma}
\probm_{\sigma,\tau}^v(\mathds{1}_\Omega)$. Clearly enough,
$\ValueE^\game(v) \leq \ValueA^\game(v)$.  When the converse
inequality holds, the game is \emph{determined} and the \emph{value}
of $v$ in $\game$ is denoted $\Value^\game(v)$. Moreover, if the value
is attained by positional strategies $\sigma$ and $\tau$, $\game$ is
said to be \emph{positionally determined}. \emph{Strong determinacy}
means that for every threshold $c$, either Eve has a strategy to
ensure that the probability that the play belongs to $\Omega$ is at
least $c$, or Adam has a strategy to ensure probability $< c$ to for
the set of plays satisfying $\Omega$.

Back to the example of Figure~\ref{6-fig:ex-stoch-arena}, assume that
Eve and Adam play the following pure positional strategies:
$\sigma(v_0) = v_1$, $\sigma(v_2) = v_3$, $\sigma(v_5) = v_5$ and
$\tau(v_6) = v_5$. Under such a strategy profile, starting in $v_0$,
the probability to reach $v_7$ is
$\probm_{\sigma,\tau}^{v_0}(\mathds{1}_{\Reach(\{v_7\})}) = \frac 2
3$. In fact, strategy $\sigma$ for Eve is optimal for the
reachability objective $\Reach(\{v_7\})$, and we will justify that
$\ValueE^\game(v_0) = \ValueA^\game(v_0) = \frac 2 3$.

