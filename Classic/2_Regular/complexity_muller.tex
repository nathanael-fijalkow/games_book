\begin{theorem}
\label{2-thm:complexity_Muller}
Solving Muller games is $\PSPACE$-complete.
\end{theorem}

As for the previous reduction, in the Muller game constructed in the reduction below we label edges rather than vertices,
and some edges have more than one colour.
As for Rabin games this can be reduced to the original definition of colouring functions (labelling vertices with exactly one colour each) with a polynomial increase in size.
%However having uncoloured edges in the context of Muller objectives is a priori more general.

\begin{proof}
The $\PSPACE$ algorithm was constructed in \cref{2-thm:muller}.

To prove the $\PSPACE$-hardness we reduce the evaluation of quantified boolean formulas in disjunctive normal form ($\QBF$) to solving Muller games. 

Let $\Psi$ be a quantified boolean formula in disjunctive normal form with $n$ variables $x_1 \ldots x_n$ and $m$ clauses $C_1 \dots C_m$, where each $C_j$ is of the form $\ell_{j_1} \wedge \ell_{j_2} \wedge \ell_{j_3}$:
\[
\Psi = \exists x_1,\forall x_2,\ldots,\exists x_n,\ \Phi(x_1,\dots,x_n) \text{ and } 
\Phi(x_1,\dots,x_n) = \bigvee_{j=1}^m \ell_{j_1} \wedge \ell_{j_2} \wedge \ell_{j_3}.
\]

We construct a Muller game $\Game$ with $m+1$ vertices (one per clause, all controlled by Adam, plus a unique vertex controlled by Eve), $4m$ edges ($4$ per clause), and $2n$ colours (one per literal).
We will show that the formula $\Psi$ evaluates to true if and only if Eve has a winning strategy in the Muller game $\Game$.

We first describe the Muller condition. 
The set of colours is the set of literals.
We let $x$ denote the lowest quantified variable such that $x$ or $\bar{x}$ is visited infinitely many times. 
The Muller condition requires that:
\begin{itemize}
	\item either $x$ is existential and only one of $x$ and $\bar{x}$ is visited infinitely many times,
	\item or $x$ is universal and both $x$ and $\bar{x}$ are visited infinitely many times,
\end{itemize}
and for all variables $y$ quantified after $x$, both $y$ and $\bar{y}$ are visited infinitely many times.
Formally, let $S_{> i} = \set{x_q, \bar{x_q} : q > p}$ and:
\[
\F = \set{ S_{> p},\ \set{x_p} \cup S_{> p},\ \set{\bar{x_p}} \cup S_{> p} : x_p \text{ existential}} \cup \set{ S_{\ge p} : x_p \text{ universal}}.
\]
Note that $\F$ contains $O(n)$ elements.

Let us now describe the arena. A play consists in an infinite sequence of rounds, where in each round first Eve chooses
a clause and second Adam chooses a literal $\ell$ in this clause corresponding to some variable $x_p$, 
and visits the colour $\ell$ as well as each colour in $S_{> p}$.

The reduction is illustrated in~\cref{2-fig:hardness_Muller}.
Note that the edges from the vertex controlled by Eve to the other ones do not have a colour,
which does not fit our definitions.
For this reason we introduce a new colour $c$ and colour all these edges by $c$.
We define a new Muller objective by adding $c$ to each set in $\F$: 
since every play in the game visit $c$ infinitely many times, the two games are equivalent.
We note that this construction works for this particular game but not in general.

\vskip1em
For a valuation $\mathbf{v} : \set{x_1,\dots,x_n} \to \set{0,1}$ and $p \in [1,n]$,
we write $\Psi_{\mathbf{v},p}$ for the formula obtained from $\Psi$ by fixing the variables $x_1,\dots,x_{p-1}$
to $\mathbf{v}(x_1),\dots,\mathbf{v}(x_{p-1})$ and quantifying only over the remaining variables.
Let us say that a valuation $\mathbf{v}$ is \textit{positive} if for every $p \in [1,n]$, the formula $\Psi_{\mathbf{v},p}$ evaluates to true,
and similarly a valuation is \textit{negative} if for every $p \in [1,n]$, the formula $\Psi_{\mathbf{v},p}$ evaluates to false.

\vskip1em
Let us first assume that $\Psi$ evaluates to true. 
We construct a winning strategy $\sigma$ for Eve.
It uses \textit{positive} valuations over the variables $x_1,\ldots,x_n$ as memory states. 
Note that the fact that $\Psi$ evaluates to true implies that there exists a positive valuation. 
Let us choose an arbitrary positive valuation as initial valuation.
We first explain what the strategy $\sigma$ does and then how to update its memory.

Assume that the current valuation is $\mathbf{v}$, since it is positive there exists a clause satisfying $\mathbf{v}$, the strategy $\sigma$ chooses such a clause. 
Therefore, any literal that Adam chooses is necessarily true under $\mathbf{v}$.

The memory is updated as follows: 
assume that the current valuation is $\mathbf{v}$ and that Adam chose a literal corresponding to the variable $x_p$. 
If $x_p$ is existential the valuation is unchanged.
If $x_p$ is universal, we construct a new positive valuation as follows. 
We swap the value of $x_p$ in $\mathbf{v}$ and write $\mathbf{v}[x_p]$ for this new valuation. 
Since $\mathbf{v}$ is positive and $x_p$ is universally quantified, the formula $\Psi_{\mathbf{v}[x_p],p+1}$ evaluates to true,
so there exists a positive valuation $\mathbf{v}_{p+1} : \set{x_{p+1},\dots,x_n} \to \set{0,1}$ for this formula.
The new valuation is defined as follows:
\[
\mathbf{v}'(x_q) = 
\begin{cases}
\mathbf{v}(x_q) & \text{ if } q < p, \\[.5em]
\overline{\mathbf{v}(x_q)} & \text{ if } q = p, \\
\mathbf{v}_{p+1}(x_q) & \text{ if } q > p,
\end{cases}
\]
it is positive by construction.

Let $\play$ be a play consistent with $\sigma$ and $x_p$ be the lowest quantified variable chosen infinitely many times by Adam. 
First, all colours in $S_{> p}$ are visited infinitely many times (when visiting $x$ or $\bar{x}$).
Let us look at the sequence $(\mathbf{v}_i(x_p))_{i \in \N}$ where $\mathbf{v}_i$ is the valuation in the $i$\textsuperscript{th} round.
If $x_p$ is existential, the sequence is ultimately constant as it can only change when a lower quantified variable is visited.
If $x_p$ is universal, the value changes each time the variable $x_p$ is chosen.
Since any literal that Adam chooses is necessarily true under the current valuation, 
this implies that in both cases $\play$ satisfies $\Muller(\F)$.

\vskip1em
For the converse implication we show that if $\Psi$ evaluates to false, then there exists a winning strategy $\tau$ for Adam.
The construction is similar but using \textit{negative} valuations.
The memory states are negative valuations. The initial valuation is any negative valuation.
If the current valuation is $\mathbf{v}$ and Eve chose the clause $C$, since the valuation is negative $\mathbf{v}$ does not satisfy $C$,
the strategy $\tau$ chooses a literal in $C$ witnessing this failure. 
The memory is updated as follows: assume that the current valuation is $\mathbf{v}$ and that the strategy $\tau$ chose a literal corresponding to the variable $x_p$. 
If $x_p$ is universal the valuation is unchanged.
If $x_p$ is existential, we proceed as above to construct another negative valuation where the value of $x_p$ is swapped.

Let $\play$ be a play consistent with $\tau$ and $x$ be the lowest quantified variable chosen infinitely many times by Adam. 
As before, we look at the sequence $(\mathbf{v}_i(x))_{i \in \N}$ where $\mathbf{v}_i$ is the valuation in the $i$\textsuperscript{th} round.
If $x$ is existential, the value changes each time the variable $x$ is chosen.
If $x$ is universal, the sequence is ultimately constant.
Since any literal that Adam chooses is necessarily false under the current valuation, 
this implies that in both cases $\play$ does not satisfy $\Muller(\F)$.
\end{proof}

\begin{figure}
\centering
  \begin{tikzpicture}[scale=1.3]
\node[s-eve] (v0) at (2,2.5) {};
\node[s-adam] (v1) at (0,3.6) {$\ x \vee y \vee z\ $};
\node[s-adam] (v2) at (4.3,2.7) {$\ x \vee \bar{y} \vee \bar{z}\ $};
\node[s-adam] (v3) at (1.7,0) {$\ \bar{x} \vee y \vee \bar{z}\ $};

% create edges
\path[arrow]
  (v0) edge[bend right] (v1)
  (v0) edge[bend right] (v2)
  (v0) edge[bend right] (v3)

  (v1) edge[bend left = 45] node[above, pos = 0.35] {$x$} node[above right, pos = 0.55] {$S_{> x}$} (v0)
  (v1) edge[bend right] node[above, pos = 0.35] {$y$} node[above, pos = 0.65] {$S_{> y}$} (v0)
  (v1) edge[bend right = 45] node[below, pos = 0.25] {$z$} node[below, pos = 0.55] {$S_{> z}$} (v0)

  (v2) edge[bend left = 45] node[below, pos = 0.25] {$x$} node[below, pos = 0.55] {$S_{> x}$}(v0)
  (v2) edge[bend right] node[below, pos = 0.35] {$\bar{y}$} node[below, pos = 0.65] {$S_{> y}$} (v0)
  (v2) edge[bend right = 45] node[above, pos = 0.35] {$\bar{z}$} node[above left, pos = 0.65] {$S_{> z}$} (v0)

  (v3) edge[bend left = 45] node[left, pos = 0.35] {$\bar{x}$} node[left, pos = 0.65] {$S_{> x}$} (v0)
  (v3) edge[bend right] node[left, pos = 0.35] {$y$} node[left, pos = 0.65] {$S_{> y}$} (v0)
  (v3) edge[bend right = 45] node[right, pos = 0.35] {$\bar{z}$} node[right, pos = 0.65] {$S_{> z}$} (v0);
  \end{tikzpicture}
\caption{The Muller game for $\Psi = \exists x, \forall y, \exists z, (x \wedge y \wedge z) \bigvee (x \wedge \bar{y} \wedge \bar{z}) \bigvee (\bar{x} \wedge y \wedge \bar{z})$. For a variable $v$ we write $S_{> v}$ for the set of literals corresponding to variables quantified after $v$, so for instance $S_{> x} = \set{y,\bar{y},z,\bar{z}}$.}
\label{2-fig:hardness_Muller}
\end{figure}
