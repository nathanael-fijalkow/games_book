This chapter considers quantitative objectives defined using payoffs. 
Adding quantities can serve two goals:
the first is for refining qualitative objectives by quantifying how well, how fast, or at what cost a qualitative objective is satisfied,
and the second is to define richer specifications and preferences over outcomes.
%This chapter considers quantitative games. To model quantitative
%objectives, the set of colours on edges is $\R$. The goal of adding
%quantities is to help choosing among winning strategies that arise in
%classical games. We may thus use quantities in order to refine
%qualitative objectives, i.e.~to quantify how well / how fast / how
%costly a qualitative objective is satisfied. Quantitative objectives
%may also be used to design richer specifications that are not
%$\omega$-regular (like mean payoff for instance) or other quantitative
%specifications.
\begin{itemize}
	\item We start in~\cref{4-sec:qualitative} by studying extensions of the classical qualitative objectives. Among two strategies in a reachability game that guarantee to reach a target in ten steps or in a billion steps, we would certainly prefer the first one from a pragmatic point of view.

	\item We study \emph{mean payoff games} in~\cref{4-sec:mean_payoff}. 
	We present two algorithms for solving them, the first based on \emph{strategy improvement} and the second on a \emph{value iteration} for the related class of energy games.
	Along the way we show that parity games reduce to mean payoff games.
%	An algorithm for solving mean payoff games induces an algorithm for computing the value function using binary search.

	\item We study \emph{discounted payoff games} in~\cref{4-sec:discounted_payoff}.
	We construct a strategy improvement algorithm for computing the value function.
	We also show that mean payoff games reduce to discounted payoff games, so the previous algorithm yields an algorithm for computing the value function of a mean payoff game.

	\item We study \emph{shortest path games} in~\cref{4-sec:shortest_path}.
	They extend reachability games by requiring that Eve reaches her target with minimal cost, 
	which if the weights are all equal means \emph{as soon as possible}.
	
	\item We study \emph{total payoff games} in~\cref{4-sec:total_payoff}.
\end{itemize}
%\begin{itemize}
%\item We start in~\Cref{4-sec:shortest_path} by studying extensions of the
%  classical qualitative objectives for arenas with real numbers as set
%  of colours. Among two strategies in a reachability game that
%  guarantee to reach a winning vertex in ten steps or in a billion
%  steps, we would certainly prefer the first one in a pragmatic point
%  of view. Therefore, reachability games will turn into
%  \emph{shortest-path games} where a player wants to reach a winning
%  vertex \emph{as soon as possible}.
%\item A way to solve shortest-path games in full generality is to use
%  the famous \emph{mean-payoff games}, that have their own practical
%  motivation. We introduce and study them in~\Cref{4-sec:mean_payoff}. We solve
%  them with two algorithms, based on \emph{strategy improvement} and
%  \emph{value iteration}. We can then use a binary search algorithm to
%  compute the value of each player.
%\item Another very natural payoff consists in discounting the future,
%  to make the near-future more important. We define and solve
%  \emph{discounted-payoff games} in~\Cref{4-sec:discounted_payoff}. We obtain as a
%  corollary a direct (without the need of binary search for the value)
%  algorithm to compute the values of a mean-payoff game.
%\item With all those results, we finally come back to the
%  shortest-path objective in~\Cref{4-sec:shortest_path-bis}, and use the obtained
%  result to solve \emph{total-payoff games}.
%\end{itemize}
