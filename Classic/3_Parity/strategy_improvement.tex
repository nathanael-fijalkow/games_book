\begin{theorem}
\label{3-thm:strategy_improvement}
There exists a strategy improvement algorithm for solving parity games in exponential time.
\end{theorem}

We rely on the high-level presentation of strategy improvement algorithms given in \cref{1-sec:strategy_improvement}.
In a nutshell: the algorithm constructs a sequence of strategies, the next one being an improvement over the current one,
until reaching an optimal strategy.

\paragraph{\bf Adding the option of stopping the game.}
Let $\game$ a parity game with priorities in $[1,d]$.
Let us give Eve an extra move $\siblank$ that indicates that the game should stop and that she can play from any vertex of hers.
So a strategy for Eve is now a function $\sigma : \VE \rightarrow E \cup \set{\siblank}$ 
where $\sigma(v) = \siblank$ indicates that Eve has chosen to stop the game, and $\sigma(v) \ne \siblank$ should be interpreted as normal.
Adam is not allowed to stop the game, so strategies for Adam remain unchanged.
We say that a play ending with $\siblank$ is stopped.

For reasoning it will be useful to consider the parity graph $\Game[\sigma]$ obtained from $\Game$ by restricting the outgoing edges from $\VE$
to those prescribed by $\sigma$. 
Recall that we say that a parity graph (without stopping option) ""satisfies"" parity from $v$ if all infinite paths from $v$ satisfy parity.
Then a strategy $\sigma$ is winning from $v$ if and only if the parity graph $\Game[\sigma]$ satisfies parity from $v$.

Since we added the option for Eve to stop the game we introduce a new terminology: 
we say that a strategy $\sigma$ ""respects"" parity if all infinite plays consistent with $\sigma$ satisfy parity,
equivalently all infinite paths in $\Game[\sigma]$ satisfy parity, not requiring anything of stopped plays.

We say that a cycle is even if the maximal priority along the cycle is even, and it is odd otherwise. 
Respecting parity is characterised using cycles:
\begin{fact}
A strategy $\sigma$ ""respects"" parity if and only if all cycles in $\Game[\sigma]$ are even.
\end{fact}
The algorithm will only manipulate strategies respecting parity.

\paragraph{\bf Evaluating a strategy.}
The first question is: given a strategy $\sigma$, how to evaluate it (in order to later improve it)?
As explained in \cref{1-sec:strategy_improvement} towards this goal we define a value function $\val^{\sigma} : V \to Y$.

We let $\val^{\sigma}(v) = \min_\tau \val(\play_{\sigma,\tau}^v)$ where $\tau$ ranges over (general) strategies for Adam, so we first need to define the value of the play $\play = \play_{\sigma,\tau}^v$.
If $\play$ is stopped, then $\val(\play)$ is the tuple $(p_1, p_2, \dots, p_d)$ where $p_i$ is the number of times that priority $p$ appears on the play.
Otherwise $\val(\play)$ is $\top$ if $\play$ satisfies parity, and $\bot$ if $\play$ does not satisfy parity.
So the value of a play is either $\top$, $\bot$, or a multiset of priorities;
we let $Y$ denote the (infinite) set consisting of these elements.

We define a total order $\le$ on~$Y$ making it a complete lattice.
First, $\top$ is the greatest element and $\bot$ the least element.
For $t = (a_1, a_2, \dots, a_d)$ and $t' = (b_1, b_2, \dots, b_d)$,
we have that $t < t'$ if and only if $p$ is the largest index such that $t_p \ne t'_p$ and 
\begin{itemize}
	\item either $p$ is even and $t_p < t'_p$,
	\item or $p$ is odd and $t_p > t'_p$.
\end{itemize}
Then $\le$ is the reflexive closure of~$<$: we have $t \le t'$ if $t < t'$ or $t = t'$.

These conditions specify which priorities Eve likes to see along a play.
Given a choice, Eve would prefer to see even priorities and to avoid odd priorities, but these preferences are hierarchical: 
assuming that $d$ is even, Eve would like to see as many edges of priority $d$ as possible. 
If two plays see $d$ the same number of times, Eve prefers the play that visits the fewest vertices of
priority $d-1$, and if two plays see $d$ and $d-1$ the same number of times,
then Eve would like to maximise the number of times that $d-2$ is visited, and so on recursively.

\paragraph{\bf The value function as a fixed point.}
We define $\delta : Y \times [1,d] \to Y$ by 
\[
\delta(t,p) = 
\begin{cases}
(a_1, a_2, \dots, a_p + 1, a_{p+1}, \dots, a_d) & \text{ if } t = (a_1, a_2, \dots, a_d), \\
\top & \text{ if } t = \top, \\
\bot & \text{ if } t = \bot.
\end{cases}
\]
We note that $\delta$ is monotonic: for all $p \in [1,d]$,
if $t \le t'$ then $\delta(t,p) \le \delta(t',p)$. 
We extend $\delta$ to $\delta : Y \times [1,d]^* \to Y$.

We let $F^\sigma_V$ denote the set of functions $\mu : V \to Y$ such that $\mu(v) = \emptyset$ if $\sigma(v) = \siblank$,
it is a lattice when equipped with the componentwise (partial) order induced by $Y$:
we say that $\mu \le \mu'$ if for all vertices $v$ we have $\mu(v) \le \mu'(v)$.
We then define an operator $\Op : F^\sigma_V \to F^\sigma_V$ by
\[
\Op(\mu)(v) = 
\begin{cases}
\min \set{\delta( \mu(v'), \col(v)) : (v,v') \in E} & \text{if } \sigma(v) \neq \siblank \\
\emptyset 											& \text{if } \sigma(v) = \siblank.
\end{cases}
\]
Since $\delta$ is monotonic so is $\Op$.

\begin{fact}
The function $\val^\sigma$ is a fixed point of $\Op$ in $F^\sigma_V$.
\end{fact}
Unfortunately, $\val^{\sigma}$ is not in general the greatest fixed point of $\Op$ in $F^\sigma_V$;
let us analyse this in more details.
Let $\mu$ a fixed point of $\Op$ in $F^\sigma_V$, there are two cases. 
For a vertex $v$ such that there exists a stopped play $\pi$ starting from $v$, we have $\mu(v) \le \val(\pi)$, and more generally
$\mu(v) \le \inf_{\pi} \val(\pi)$ where $\pi$ ranges over all stopped plays starting from $v$.
The problem is for a vertex $v$ such that no plays starting from $v$ are stopped: 
we can have either $\mu(v) = \top$ or $\mu(v) = \bot$, irrespective of whether the play satisfies parity or not.
From this discussion we obtain the following result.

\begin{lemma}
\label{3-lem:greatest_fixed_point}
If $\sigma$ "respects" parity, then $\val^{\sigma}$ is the greatest fixed point of $\Op$ in $F^\sigma_V$.
\end{lemma}

\paragraph{\bf Improving a strategy.}
We reach the last item in the construction of the algorithm: the notion of switchable edges.
Let $\sigma$ a strategy. We say that an edge $e = (v,v')$ is switchable if 
\[
\delta(\val^{\sigma}(v'),\col(v)) > \delta(\val^{\sigma}(u),\col(v)) \text{ where } \sigma(v) = (v,u).
\]
Intuitively: according to $\val^{\sigma}$, playing $e$ is better than playing $\sigma(v)$.

Given a strategy $\sigma$ and an edge $e = (v,v')$ we use $\sigma[v \to e]$ to denote the strategy playing $e$ from $v$ 
and follow $\sigma$ from all other vertices.
Let us write $\sigma \le \sigma'$ if for all vertices~$v$ we have $\val^{\sigma}(v) \le \val^{\sigma'}(v)$,
and $\sigma < \sigma'$ if additionally $\neg (\sigma' \le \sigma)$.

\paragraph{\bf The algorithm.}
The algorithm starts with a specified initial strategy, which is the strategy
$\sigma_0$ where $\sigma_0(v) = \siblank$ for all vertices $v \in \VE$. 
It may not hold that $\sigma_0$ respects parity since $\game$ may contain odd cycles fully controlled by Adam.
This can be checked in linear time and the attractor to the corresponding vertices removed from the game.
After this preprocessing $\sigma_0$ indeed respects parity.

The pseudocode of the algorithm is given in \cref{3-algo:strategy_improvement}.
%The algorithm is the following: in each round $i$ compute $\val^{\sigma_i}$ and look for a switchable edge.
%If there exists a switchable edge $e_i = (v_i,p_i,v'_i)$, let $\sigma_{i+1} = \sigma_i[v_i \to e_i]$ and iterate to the next round.
%Otherwise, return the strategy $\sigma_i$.

\begin{algorithm}
 \KwData{A parity game $\game$}
 \SetKwBlock{Repeat}{repeat}{}
 \DontPrintSemicolon
 
 $i \leftarrow 0$
 
 \For{$v\in \VE$}{
   $\sigma_0(v) \leftarrow \siblank$
 }

 $C \leftarrow \set{v \in V : v \text{ contained in an odd cycle in } \game[\sigma_0]}$

 $\Game \leftarrow \Game \setminus \AttrA(C)$
 
 \Repeat{
   Compute $\val^{\sigma_i}$ 

	\If{$\exists e_i = (v_i,v'_i) \in E \text{ switchable}$}{
		$\sigma_{i+1} \leftarrow \sigma_i[v_i \to e_i]$
		
		$i \leftarrow i + 1$

	} 
	\Else{
		\Return{$\set{v \in V : \val^{\sigma_i}(v) = \top}$}
	}
 }

 \caption{The strategy improvement algorithm for parity games.}
\label{3-algo:strategy_improvement}
\end{algorithm}

%Note that one cannot start the algorithm with an arbitrary strategy, since if we happen to pick a
%strategy $\sigma$ that allows Adam to win everywhere, then all vertices will
%have value $\bot$, and there will be no switchable edges. However, if we
%start with a strategy where all vertices have a value that is not $\bot$,
%then we can never reach a strategy where any vertex has value $\bot$.

\paragraph{\bf Proof of correctness.}
We start by stating a very simple property of $\delta$, which is key in the arguments below.

\begin{fact}
Let $t \in Y$ and $p_1,\dots,p_k \in [1,d]$ such that $t$ and $\delta(t,p_1 \dots p_k)$ are neither $\top$ nor $\bot$.
Then $t \le \delta(t,p_1 \dots p_k)$ if and only if $\max \set{p_1,\dots,p_k}$ is even.
\end{fact}

The following lemma states the two important properties of $(Y,\le)$ and $\delta$.

\begin{lemma}
\label{3-lem:key_property}
Let $G$ a parity graph (with no stopping option).
\begin{itemize}
	\item If there exists $\mu : V \to Y$ such that for all vertices $v$ we have $\mu(v) \neq \top,\bot$
	and for all edges $(v,u) \in E$ we have $\mu(v) \le \delta(\mu(u),\col(v))$,
	then $G$ satisfies parity.
	\item If there exists $\mu : V \to Y$ such that for all vertices $v$ we have $\mu(v) \neq \top,\bot$
	and for all edges $(v,u) \in E$ we have $\mu(v) \ge \delta(\mu(u),\col(v))$,
	then $G$ satisfies the complement of parity.
\end{itemize}
\end{lemma}
\begin{proof}
We prove the first property, the second is proved in exactly the same way.
Thanks to the characterisation using cycles it is enough to show that all cycles in $G$ are even.
Let us consider a cycle in $G$:
\[
\pi = (v_0,v_1) (v_1,v_2) \cdots (v_{k-1},v_0).
\]
For all $i \in [0,k-1]$ we have $\mu(v_i) \le \delta(\mu(v_{i+1 \mod k}),\col(v_i))$.
By monotonicity of $\delta$ this implies $\mu(v_1) \le \delta(\mu(v_1),\col(v_{k-1}) \cdots \col(v_0))$.
Thanks to \cref{3-lem:key_property} this implies that the maximum priority in $\set{\col(v_0),\dots,\col(v_{k-1})}$ is even.
\end{proof}

Let $\sigma$ a strategy respecting parity. 
A progress measure for $\Game[\sigma]$ is a post-fixed point of $\Op$ in $F^\sigma_V$:
it is a function $\mu : V \to Y$ such that $\mu(v) = \emptyset$ if $\sigma(v) = \siblank$ and $\mu \le \Op(\mu)$,
which means that $\mu(v) \le \min \set{ \delta(\mu(v'),\col(v)) : (v,v') \in E}$.

We now rely on \cref{3-lem:greatest_fixed_point} and \cref{3-lem:key_property} to prove the two principles: progress and optimality.

\begin{lemma}[Progress]
Let $\sigma$ a strategy respecting parity and $e = (v,v')$ a switchable edge.
We let $\sigma'$ denote $\sigma[v \to e]$.
Then $\sigma'$ respects parity and $\sigma < \sigma'$.
\end{lemma}

\begin{proof}
We first argue that $\sigma'$ respects parity.
The fact that $e = (v,v')$ is switchable reads
\[
\delta(\val^\sigma(v'),\col(v)) > \delta(\val^\sigma(u),\col(v)),
\]
and by definition of $\val^\sigma$ we have $\val^\sigma(v) = \delta(\val^\sigma(u),\col(v))$,
which implies $\val^\sigma(v) < \delta(\val^\sigma(v'),\col(v))$, and in particular $\val^\sigma(v) \neq \top$.

Let us consider the parity graph $\Game[\sigma']$ and note that for all edges $e' = (s,t)$ 
we have $\val^\sigma(s) \le \delta(\val^\sigma(t),\col(s))$:
indeed either $e'$ is an edge in $\Game[\sigma]$ and this is by definition of $\val^\sigma$,
or $e' = e$ and the inequality was proved just above.

Since $\sigma$ respects parity $\val^\sigma$ does not take the value $\bot$.
But we cannot apply (the first item of) \cref{3-lem:key_property} yet because $\val^\sigma$ may have value $\top$.
However by definition of $\val^\sigma$ for all vertices $s$ such that $\val^\sigma(s) = \top$ all paths from $s$ satisfy parity,
so it is enough to consider the parity graph obtained from $\Game[\sigma']$ by removing all such vertices.
The first item of \cref{3-lem:key_property} implies that it satisfies parity, hence $\Game[\sigma']$ as well.

\vskip1em
At this point we know that $\sigma'$ respects parity, which thanks to \cref{3-lem:greatest_fixed_point}
implies that $\val^{\sigma'}$ is the greatest fixed point of $\Op$ in $F^{\sigma'}_V$.

We now argue that $\val^\sigma$ is a progress measure for $\game[\sigma']$.
For all vertices but $v$ this is clear because the outgoing edges are the same in $\game[\sigma]$ and in $\game[\sigma']$.
For $v$ as argued above we have $\val^\sigma(v) < \delta(\val^\sigma(v'),\col(v))$.
It follows that $\val^\sigma$ is indeed a progress measure for $\game[\sigma']$.
Since $\val^{\sigma'}$ is the greatest fixed point of $\Op$ in $F^{\sigma'}_V$, this implies that 
$\val^{\sigma} \le \val^{\sigma'}$.

We now show that $\val^{\sigma} < \val^{\sigma'}$. 
Using $\val^{\sigma}(v') \le \val^{\sigma'}(v')$ and the monotonicity of $\delta$ we obtain that
$\delta(\val^\sigma(v'),\col(v)) \le \delta(\val^{\sigma'}(v'),\col(v))$.
By definition of $\val^{\sigma'}$ we have $\val^{\sigma'}(v) = \delta(\val^{\sigma'}(v'),\col(v))$
and together with $\val^\sigma(v) < \delta(\val^\sigma(v'),\col(v))$ this implies that
$\val^\sigma(v) < \val^{\sigma'}(v)$.
\end{proof}

\begin{lemma}[Optimality]
Let $\sigma$ a strategy respecting parity that has no switchable edges, then 
$\sigma$ is winning from all vertices of $\WE(\Game)$.
\end{lemma}

\begin{proof}
The fact that $\sigma$ respects parity means that it is a winning strategy
from all vertices $v$ such that $\val^\sigma(v) = \top$.
It also implies that for all vertices $v$ we have $\val^{\sigma}(v) \neq \bot$.
We now prove that Adam has a winning strategy from all vertices $v$ such that $\val^{\sigma}(v) \neq \top$.
We construct a strategy of Adam by
\[
\forall v \in \VA,\ \tau(v) = \argmin \set{ \delta(\val^{\sigma}(u),\col(v)) : (v,u) \in E }.
\]
We argue that $\tau$ ensures the complement of parity from all vertices $v$ such that $\val^{\sigma}(v) \neq \top$.
Let us consider $\Game[\tau]$ the parity graph obtained from $\Game$ by restricting the outgoing edges from $\VA$
to those prescribed by $\tau$.
We argue that for all edges $(v,,v')$ in $\game[\tau]$, we have 
$\val^{\sigma}(v) \ge \delta(\val^{\sigma}(v'),\col(v))$.
Once this is proved we conclude using the second item of \cref{3-lem:key_property} implying that $\Game[\tau]$ satisfies the complement of parity.

The first case is when $v \in \VE$. 
Let $\sigma(v) = (v,u)$.
Since the edge $e = (v,v')$ is not switchable we have 
$\delta(\val^{\sigma}(v'),\col(v)) \le \delta(\val^{\sigma}(u),\col(v))$.
By definition of $\val^\sigma$ we have $\val^\sigma(v) = \delta(\val^{\sigma}(u),\col(v))$,
implying the desired inequality.

The second case is when $v \in \VA$, it holds by definition of $\tau$.
\end{proof}

\paragraph{\bf Complexity analysis.}
The computation of $\val^\sigma$ for a strategy $\sigma$ can be seen to be a shortest path problem where distances are measured using the operator $\le$. 
Thus, any algorithm for the shortest path problem can be applied, such as the Bellman-Ford algorithm.
In particular computing $\val^\sigma$ can be done in polynomial time, and even more efficiently through a refined analysis.

An aspect of the algorithm we did not develop is choosing the switchable edge.
It is possible to switch not only one edge but a set of switchable edges at each iteration, making this question worse: 
which subset of the switchable edges should be chosen?
Many possible rules for choosing this set have been studied, as for instance the \emph{greedy all-switches} rule. 

The next question is the number of iterations, meaning the length of the sequence
$\sigma_0,\sigma_1,\dots$. It is at most exponential since it is bounded by the number of strategies (which is bounded aggressively by $m^n$).
There are lower bounds showing that the sequence can be of exponential length, which apply to different rules for choosing switchable edges.
Hence the overall complexity is exponential; we do not elaborate further here. 
%We refer to \cref{3-sec:references} for bibliographic references and a discussion on the family of strategy improvement algorithms.
