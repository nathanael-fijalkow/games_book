As shown in the previous section, the optimal solution of any of the programs $\lpmp$, $\lpmpdual$ gives us an upper bound on the optimal value. In this sub-section we show that in strongly connected MDPs: a) a value of every vertex is the same; b) from a solution of $\lpmp$ one can extract a memoryless deterministic strategy $\sigma$ whose expected mean-payoff is well defined (i.e., the preconditions of \Cref{5-lem:limit-defined} are satisfied)) and equal to the objective value of the solution. Moreover, if the solution in question is optimal, then $ \sigma $ is optimal for both $\playPay$- and $\stepPay$-semantics.

\begin{definition}
\label{5-def:scc-mdp}
An MDP is \emph{strongly connected} if for each pair of vertices $u,v$ there exists a strategy which, when starting in $u$, reaches $v$ with a positive probability. 
\end{definition}

For the rest of this section we fix an optimal solution $\lpsol{x}_{v,a}$ of $\lpmp$. We denote by $\solvset$ the set of all vertices for which there exists action $a$ s.t. $\lpsol{x}_{v,a}>0.$ From the shape of $\lpmp$ it follows that $\solvset$ is non-empty and closed, and hence we can consider a sub-MDP $\mdp_{\solvset}$ induced by $\solvset$. In $\mdp_{\solvset}$ we then define a memoryless randomized strategy $\sigma$ by putting $$\sigma(a\mid v)=\frac{\lpsol{x}_{(v,a)}}{\sum_{b\in \actions}\lpsol{x}_{(v,b)}}.$$

Fixing a strategy $\sigma$ yields a \emph{Markov chain} $\mdp_\solvset^{\sigma}$. Markov chain can be viewed as an MDP with a single action (and hence, with no non-determinism). $\mdp_{\solvset}^\sigma$ in particular can be viewed an MDP with the same vertices, edges, and colouring as $\mdp_\solvset$, but with a single action (as non-determinism was already resolved by $\sigma$). The probability of transitioning from a vertex $u$ to a vertex $v$ in a Markov chain is denoted by $\mcprob_{u,v}$. In $\mdp_{\solvset}^{\sigma}$ we have $\mcprob_{u,v}=\sum_{a\in \actions} \probTranFunc(v\mid u,a)\cdot\sigma(a\mid u)$, the right-hand side being computed in the original MDP $\mdp$. Both $\mdp_\solvset$ and $\mdp_{\solvset}^{\sigma}$ have the same sets of plays and for each initial vertex, the probability measure induced by $\sigma$ in $\mdp$ equals the probability measure arising (under the unique policy) in $\mdp_{\solvset}^{\sigma}$. Hence, to prove anything about $\sigma$ it suffices to analyse $\mdp_{\solvset}^{\sigma}$.

\paragraph{A refresher on Markov chains.} We review some fundamental notions of Markov chain theory~\cite{Norris:1998}. A Markov chain that is strongly connected is called \emph{irreducible}. The one-step transition probabilities in a Markov chain can be arranged into a square matrix $\mcprob$, which has one row and one column for each vertex. The cell in the row corresponding to a vertex $u$ and in the column corresponding to a vertex $v$ bears the value $\mcprob_{u,v}$ defined above. An easy induction shows that the matrix $\mcprob^k$ contains $k$-step transition probabilities. That is, the probability of being in $v$ after $k$ steps from vertex $u$ is equal to the $(u,v)$-cell of $\mcprob^k$, which we denote by $\mcprob^{(k)}_{u,v}$.

A vertex $u$ of a Markov chain is \emph{recurrent} if, when starting from $u$, it is revisited infinitely often with probability $1$. On the other hand, if the probability that $u$ is re-visited only finitely often is one, then the vertex is \emph{transient}. It is known~\cite[Theorem 1.5.3]{Norris:1998} that each vertex of a finite Markov chain is either recurrent or transient, and that these two properties can be equivalently characterized as follows: vertex $u$ is recurrent if and only if  $\sum_{k=0}^{\infty}\mcprob^{(k)}_{u,u}=\infty$, otherwise it is transient.

An \emph{invariant distribution} in a Markov chain with a vertex set $\vertices$ is a $|\vertices|$-dimensional non-negative row vector $\invdist$ which adds up to $1$ and satisfies $ \invdist\cdot \mcprob = \invdist$.

The following lemma holds for arbitrary finite Markov chains.

\begin{lemma}
	\label{5-lem:MC-inv-rec}
Let $\invdist$ be an invariant distribution and $v$ a vertex such that $\invdist_v > 0$. Then $v$ is recurrent.	
\end{lemma}
\begin{proof}
Let $n$ be the number of vertices in the chain and $p_{\min}$ the minimum non-zero entry of $\mcprob$.
Assume, for the sake of contradiction, that $v$ is transient. We show that in such a case, for each vertex $u$ it holds $\lim_{k\rightarrow\infty} \mcprob^{(k)}_{u,v} = 0$. For $u=v$ this is immediate, since the sum $\sum_{k=0}^{\infty}\mcprob^{(k)}_{v,v}$ converges for  transient $v$. Otherwise, let $f_{u,v,i}$ be the probability that a play starting in $u$ visits $v$ for the \emph{first time} in exactly $i$ steps. Then $\mcprob^{(k)}_{u,v}=\sum_{i=0}^k f_{u,v,i}\cdot \mcprob^{(k-i)}_{v,v}$. Now when starting in a vertex from which $v$ is reachable with a positive probability, at least one of the following events happens with probability $\geq p_{\min}^n$ in the first $n$ steps: either we reach a vertex from which $v$ is not reachable with positive probability, or we reach $v$. If neither of the events happens, we are, after $n$ steps, still in a vertex from which $v$ can be reached with a positive probability. In such a case, the argument can be inductively repeated (analogously to the proof of \Cref{5-thm:as-char}) to show that $f_{u,v,i}\leq (1-p_{\min}^n)^{\lfloor\frac{i}{n}\rfloor}\leq (1-p_{\min}^n)^{\frac{i-n}{n}}$.

Since $\sum_{k=0}^{\infty}\mcprob^{(k)}_{v,v}$ converges, for each $\eps>0$ there exists $j_\eps$ such that $\sum_{i=j_{\eps}}^{\infty}\mcprob^{(i)}_{v,v} < \frac{\eps}{2}$. Similarly, there exists $\ell_\eps$ such that $$\sum_{i=\ell_{\eps}}^{\infty}{(1-p_{\min}^n)^{\frac{i-n}{n}}} = \frac{(1-p_{\min}^n)^{\frac{\ell_\eps}{n}}}{\left(1-(1-p_{\min}^n)^{\frac{1}{n}}\right)\cdot(1-p_{\min}^n)}< \frac{\eps}{2},$$ and hence $\sum_{i=\ell_{\eps}}^{\infty} f_{u,v,i}< \frac{\eps}{2}.$

Now we put $m_{\eps}=\max\{j_\eps,\ell_\eps\}$. For any $k\geq 2m_{\eps}$ we have $\mcprob^{(k)}_{u,v}=\sum_{i=0}^k f_{u,v,i}\cdot \mcprob^{(k-i)}_{v,v} \leq \sum_{i=m_{\eps}}^{k}f_{u,v,i} + \sum_{i=0}^{m_{\eps}}\mcprob^{(k-i)}_{v,v}\leq\sum_{i=m_{\eps}}^{k}f_{u,v,i} + \sum_{i=m_{\eps}}^{k}\mcprob^{(i)}_{v,v}<\eps$ (note that all the series involved are non-negative). This proves that $\mcprob^{(k)}_{u,v}$ vanishes in the limit.

Finally, we derive the contradiction. Since $\invdist$ satisfies $\invdist\cdot \mcprob = \invdist$, we also have $\invdist\cdot \mcprob^k = \invdist$ for all $k$. Hence, the $v$-component of $\invdist\cdot \mcprob^k$ is equal to $\invdist_v>0$. But as shown above, the $v$-column of $\mcprob^k$ converges to the all-zero vector as $k\rightarrow \infty$, so also $(\invdist\cdot \mcprob^k)_v$ vanishes in the limit, a contradiction.
\end{proof}

\noindent
\paragraph{Towards the optimality of $ \sigma $.} We now turn back to the chain $\mdp_{\solvset}^{\sigma}$, where the memoryless strategy $ \sigma $ is obtained from the optimal solution of $ \lpmp $. In general, $ \mdp_{\solvset}^{\sigma} $ does not have to be irreducible. Hence, we use the following lemma and its corollary to extract an irreducible sub-chain, to which we can apply known results of Markov chain theory.

\begin{lemma}
\label{5-lem:mc-rec}
Let $\bar{\invdist}$ be a vector such that for each $v\in \solvset$ it holds $\bar{\invdist}_v=\sum_{a\in \actions} \lpsol{x}_{v,a}$. Then $\bar{\invdist}$ is an invariant distribution of $\mdp_{\solvset}^{\sigma}$. Consequently, all vertices of $\mdp_{\solvset}^{\sigma}$ are recurrent.
\end{lemma}
\begin{proof}
The first part follows directly from the fact that $\lpsol{x}_{v,a}$ is a feasible solution of $\lpmp$. The second part follows from~\Cref{5-lem:MC-inv-rec} and from the fact that $\bar\invdist$ is positive (by the definition of $\solvset$).
\end{proof}

\begin{corollary}
\label{5-cor:mp-scc-extraction}
The set $\solvset$ can be partitioned into subsets $\solvset_1,\solvset_2,\dots,\solvset_m$ such that each $\solvset_i$ induces a strongly connected sub-chain of $\mdp_{\solvset}^{\sigma}$.
\end{corollary}
\begin{proof}
Let $v\in\solvset$ be arbitrary and let $U_v\subseteq \solvset$ be the set of all vertices reachable with positive probability from $v$ in $\mdp_{\solvset}^{\sigma}$. Then $v$ is reachable (in $\mdp_{\solvset}^{\sigma}$) with positive probability from each $u\in U_v$: otherwise, there would be a positive probability of never revisiting $v$, a contradiction with each vertex being recurrent in $\mdp_{\solvset}^{\sigma}$ (\Cref{5-lem:mc-rec}). Hence, $U_v$ induces a strongly connected `sub-MDP' (or sub-chain) of $\mdp_{\solvset}^{\sigma}$. It is easy to show that if $U_v \neq U_w$ for some $v\neq w $, then the two sets must be disjoint.
\end{proof}

Hence, we can extract from $\solvset$ a set $Q$ inducing a strongly-connected sub-chain of $\mdp_{\solvset}^{\sigma}$, which we denote $\mdp^{\sigma}_{Q}$. The set $Q$ also induces a strongly connected sub-MDP of $\mdp$ denoted by $\mdp_Q$. The chain $\mdp^{\sigma}_{Q}$ arises by fixing, in $\mdp_Q$, a strategy formed by a restriction of $\sigma$ to $Q$. We use the following powerful theorem to analyse $\mdp^{\sigma}_{Q}$.

\begin{theorem}[""Ergodic theorem""; see Theorem~1.10.2 in~\cite{Norris:1998}]
\label{5-thm:ergodic} In a strongly connected Markov chain (with a finite set of vertices $\vertices$) there exists a unique invariant distribution $\invdist$. Moreover, for every vector $\vec{h}\in \R^{\vertices}$ the following equation holds with probability 1:
\[
\lim_{n \rightarrow \infty} \frac{1}{n} \sum_{i=0}^{n-1}\vec{h}_{\ing(\pi_i)} = \sum_{v\in\vertices} \invdist_v\cdot \vec{h}_v.
\]
(In particular, the limit is well-defined with probability 1).
\end{theorem}

\knowledge{Ergodic theorem}{notion,index={Ergodic theorem}}

We can use the Ergodic theorem to shows that the expected mean-payoff achieved by $\sigma$ in $\mdp_{Q}$ matches the optimal value of $ \lpmp $, in a very strong sense: the probability of a play having a mean-payoff equal to this optimal value is 1 under $ \sigma $.

\begin{theorem}
\label{5-cor:mp-scc-optimality}
Let $\sigma_Q$ be the restriction of $\sigma$ to $Q$. Then for every $v\in Q$ it holds that $\probm^{\sigma_Q}_{\mdp_Q,v}(\MeanPayoffInf = r^*)=1$, where $r^*$ is the is the optimal value of $\lpmp$. 
%As a consequence, $\expv_{\mdp_W,v}^{\sigma_W}[\MeanPayoffInf]=\expv_{\mdp_W,v}^{\sigma_W}[\MeanPayoffInf]=r^*$. 
\end{theorem}
\begin{proof}
Let $ \vec{w}\in\R^{\vertices \times\actions} $ be a vector sych that $\vec{w}_{(v,a)}=\lpsol{x}_{(v,a)}/\sum_{(q,a)\in Q\times\actions} \lpsol{x}_{(q,a)}$ for every $(v,a)\in Q\times \actions$, and $\vec{w}_{(v,a)}=0$ for all other $(v,a)$. We claim that $ \vec{w} $ is also an optimal solution of $\lpmp$. 

To prove feasibility, note that setting $\vec{w}_{(v,a)}=0$ for each $v\in \vertices\setminus Q$ does not break the constraints~\cref{5-eq:mdp-flow}. This is because $Q$ induces a strongly connected sub-chain of $\mdp_{\solvset}^{\sigma}$, and hence there are no $v\in \vertices$, $u\in \vertices\setminus Q$ such that $\lpsol{x}_{(u,a)}\cdot \probTranFunc(v\mid u,a)>0$. Next,~\cref{5-eq:mdp-flow} is invariant w.r.t. multiplication of variables by a constant, so normalizing the remaining values preserves~\cref{5-eq:mdp-flow} and ensures that~\cref{5-eq:mdp-freq-1} holds. 

To prove optimality, assume that the objective value of $\vec{w}$ is smaller than $r^*$. Then we can mirror the construction from the previous paragraph and produce a feasible solution ${\hat{\vec{w}}_{(v,a)}}$ whose $(Q\times\actions)$-indexed components are zero and the rest are normalized components of $\lpsol{x}$. Then $r^*$ is a convex combination of the objective values of $\vec{w}$ and $\hat{\vec{w}}$, so $\hat{\vec{w}}$ must have a strictly larger value than $r^*$, a contradiction with the latter's optimality.

We now plug $ \vec{w} $ into the ergodic theorem as follows: As in~\Cref{5-lem:mc-rec}, it easy to prove that setting $\invdist_v=\sum_{a\in\actions}\vec{w}_{(v,a)}$ yields an invariant distribution. Now put $\vec{h}_v=\sum_{a\in\actions}\sigma(a\mid v)\cdot \colouring(v,a)$ ($ =  \sum_{w \in \vertices} \mcprob_{v,w}\cdot \colouring(v,w)$). From the Ergodic theorem we get that $\lim_{n \rightarrow \infty} \frac{1}{n} \sum_{i=0}^{n-1}\vec{h}_{\ing(\pi_i)}$ almost-surely exists and equals 
\begin{align}
\sum_{v\in Q} \invdist_v\cdot \vec{h}_v &= \sum_{v\in \vertices} \left(\big(\sum_{d\in\actions}\vec{w}_{(v,d)}\big)\cdot \big(\sum_{a\in\actions}\sigma(a\mid v)\cdot \colouring(v,a) \big)\right) \nonumber\\
&= \sum_{v\in Q} \left(\Bigg( \frac{\sum_{d\in\actions}\lpsol{x}_{(v,d)}}{\sum\limits_{\substack{q\in Q\\ b\in \actions}} \lpsol{x}_{(q,b)}} \Bigg)\cdot \Bigg( \frac{\sum_{a\in\actions}\lpsol{x}_{(v,a)}\cdot\colouring(v,a)}{\sum\limits_{d\in  \actions} \lpsol{x}_{(v,d)}} \Bigg) \right) \nonumber\\
&= \frac{1}{\sum\limits_{\substack{q\in Q\\ b\in \actions}} \lpsol{x}_{(q,b)}}\cdot\sum\limits_{\substack{v\in Q\\ a\in\actions}} \lpsol{x}_{(v,a)}\cdot \colouring(v,a) = \sum\limits_{\substack{v\in Q\\ a\in\actions}} \vec{w}_{(v,a)}\cdot\colouring(v,a) =r^*.\label{5-eq:ergodic-use}
\end{align}

%\noindent
It remains to take a step from the left-hand side of~\cref{5-eq:ergodic-use} towards the mean payoff. To this end, we construct a new Markov chain $\mdp_Q'$ from $\mdp_Q$ by `splitting' every edge $(u,v)$ with a new dummy vertex $d_{u,v}$ (i.e., $d_{u,v}$ has one edge incoming from $u$ with probability $\mcprob_{u,v}$ and one edge outgoing to $v$ with probability $1$). In $\mdp_Q'$ we define a vector $\vec{h}'$ s.t. for each vertex $d_{u,v}$ the vector $ \vec{h}' $ has the $ d_{u,v} $-component equal to $\colouring(u,v)$, while the components corresponding to the original vertices are zero. It is easy to check that $\mdp_Q'$  is strongly connected and that it has an invariant distribution $\invdist'$ defined by $\invdist'_v=\invdist_v/2$ for $v$ in $Q$ and $\invdist'_{d_{u,v}}=\frac{\invdist_u\cdot\mcprob_{u,v}}{2}$ for $(u,v)$ an edge of $\mdp_Q$.
Also, by easy induction, for each play $\play$ of length $n$ in $\mdp_Q$ it holds $\frac{1}{n}\sum_{i=0}^{n-1}\colouring(\play_i) = \frac{1}{n}\sum_{i=0}^{2n-1}\vec{h}'_{\ing(\play_i')}$, where $\play'$ is the unique play in $\mdp_Q'$ obtained from $\play$ by splitting edges with appropriate dummy vertices. Hence, 
\begin{equation}
\label{5-eq:mc-opt-limit}
\lim_{n\rightarrow \infty}\frac{1}{n}\sum_{i=0}^{n-1}\colouring(\play_i) = 2\cdot \lim_{n\rightarrow \infty}\frac{1}{n}\sum_{i=0}^{n-1}\vec{h}'_{\ing(\play_i')},\end{equation} provided that both limits exist. By the ergodic theorem  applied to $\mdp_Q'$, we have that the RHS  limit in~\cref{5-eq:mc-opt-limit} is defined with probability 1 and equal to
\begin{align*}
\underbrace{\sum_{v\in Q} \invdist'_v \cdot \vec{h}'_{v}}_{=0} + \sum_{u,v\in Q} \invdist'_{d_{u,v}}\cdot \vec{h}'_{d_{u,v}} = \frac{1}{2}\sum_{u\in Q}\invdist_u\cdot\left( \sum_{v\in Q}\mcprob_{u,v}\cdot \colouring(u,v)\right)\\ =\frac{1}{2}\sum_{u\in Q} \invdist_u\cdot \vec{h}_u=\frac{r^*}{2},
\end{align*}
\noindent
the last equality being shown above. Plugging this into~\cref{5-eq:mc-opt-limit} yields that if a limit on the LHS (i.e., the mean payoff of a play) is well-defined with probability 1, then it is equal to $r^*$ also with probability 1. But if there was a set $L$ of positive probability in $\mdp_Q$ with $\lim_{n \rightarrow \infty}\frac{1}{n}\sum_{i=0}^{n-1}\colouring(\play_i)$ undefined for each $\play\in L$, by splitting the plays in $L$ we would obtain a positive-probability set of plays in $\mdp_Q'$ in which $\lim_{n \rightarrow \infty}\frac{1}{n}\sum_{i=0}^{n-1}\vec{h}'_{\ing(\play_i')}$ is also undefined, a contradiction with the ergodic theorem. 
%
%%
%
%\begin{align*}
%\expv^{\bar\sigma_W}_{\mdp_{W},v}[\lim_{n \rightarrow \infty} \frac{1}{n} \sum_{i=0}^{n-1}h(\ing(\pi_i))] &= \lim_{n \rightarrow \infty} \frac{1}{n}\sum_{i=0}^{n-1}\expv^{\bar\sigma_W}_{\mdp_{W},v}[h(\ing(\pi_i))]\\
%&= \lim_{n \rightarrow \infty} \frac{1}{n}\sum_{i=0}^{n-1}\sum_{w\in W}\left(\probm^{\bar\sigma_W}_{\mdp_{W},v}(\ing(\play_i)=w)\cdot \sum_{a\in\actions}\bar\sigma_W(a\mid v)\cdot \colouring(v,a)\right)\\
%&= \lim_{n \rightarrow \infty} \frac{1}{n}\sum_{i=0}^{n-1}\expv^{\bar\sigma_W}_{\mdp_{W},v}[\colouring(\play_i) ] = \expv^{\bar\sigma_W}_{\mdp_{W},v}[ \lim_{n \rightarrow \infty} \frac{1}{n}\sum_{i=0}^{n-1} \colouring(\play_i)].
%\end{align*}
%
%\noindent
%The first equality follows from the dominated convergence theorem and from the fact that the limit-average of $h$ is a.s. defined. For the last equality, we need to show that also $\lim_{n \rightarrow \infty} \frac{1}{n}\sum_{i=0}^{n-1} \colouring(\play_i)$ is a.s. defined. To this end, we construct a new Markov chain $\mdp_W'$ from $\mdp_W$ by `splitting' every edge $(u,v)$ with a new dummy vertex $d_{u,v}$ (i.e., $d_{u,v}$ has one edge incoming from $u$ with probability $\mcprob_{u,v}$ and one edge outgoing to $v$ with probability $1$). In $\mdp_W'$ we define a function $h'$ mapping each $d_{u,v}$ to $\colouring(u,v)$, and each other vertex to $0$. It is straightforward to check that $\mdp_W'$ is strongly connected and that for each play $\play$ of length $n$ in $\mdp_W$ it holds $\sum_{i=0}^{n-1}\colouring(\play_i) = \sum_{i=0}^{2n-1}h'(\ing(\play_i'))$, where $\play'$ is the unique play in $\mdp_W'$ obtained from $\play$ by splitting edges with dummy vertices. Hence, we have $ $
%
% Hence, if there was a set $L$ of positive probability in $\mdp_W$ with $\lim_{n \rightarrow \infty}\frac{1}{n}\sum_{i=0}^{n-1}\colouring(\play_i)$ undefined for each $\play\in L$, by splitting the plays in $L$ we would obtain a positive-probability set of plays in $\mdp_W'$ in which $\lim_{n \rightarrow \infty}\frac{1}{n}\sum_{i=0}^{n-1}h'(\ing(\play_i'))$ is also undefined, a contradiction with the ergodic theorem.
\end{proof}

So far, we have constructed an optimal strategy $\sigma_Q$ but only on the part $Q$ of the original MDP $\mdp$. To conclude the construction, we define a memoryless strategy $\sigma^*$ in $\mdp$ as follows: we fix a memoryless deterministic strategy $\sigma_{=1}$ that is winning, from each vertex of $\mdp,$ for the objective of almost-sure reaching of $Q$ (such a strategy exists since $\mdp$ is strongly connected, see also \Cref{5-thm:as-char}. Then we put $\sigma^*(v)=\sigma_{=1}(v)$ if $v\not\in Q$ and $\sigma^*(v)=\sigma_Q(v)$ otherwise. Hence, starting in any vertex, $\sigma^*$ eventually reaches $Q$ with probability 1 and then it starts behaving as $\sigma_Q$. The optimality of such a strategy follows from the prefix independence of mean payoff, as argued in the next theorem.

\begin{theorem}
\label{5-thm:mp-valcomp} For any sequence of numbers $c_0,c_1,\dots$ and any $k\in\N$ it holds $\liminf_{n\rightarrow \infty}\frac{1}{n}\sum_{i=0}^{n-1}c_i = \liminf_{m\rightarrow \infty}\frac{1}{m}\sum_{i=0}^{m-1}c_{k+i}$. As a consequence, 
for every vertex $v$ in $\mdp$ it holds $\probm^{\sigma_Q}_{\mdp_Q,v}(\MeanPayoffInf=r^*)=1,$ where $r^*$ is the optimal value of $\lpmp$. Hence, $\expv^{\sigma^*}_v[\MeanPayoffInf]= r^*$.
\end{theorem}
\begin{proof}
%For any sequence of numbers $c_0,c_1,\dots$ and any $k\in\N$ it holds $\liminf_{n\rightarrow \infty}\frac{1}{n}\sum_{i=0}^{n-1}c_i = \liminf_{m\rightarrow \infty}\frac{1}{m}\sum_{i=0}^{m-1}c_{k+i}$. This is because
We have
\begin{align*}
\liminf_{n\rightarrow \infty}\frac{c_0 + \cdots c_{n-1}}{n} &= \liminf_{n\rightarrow \infty}\left(\underbrace{\frac{k}{n}}_{\mathrlap{\text{vanishes for } n\rightarrow \infty}}\cdot\frac{c_0 + \cdots + c_{k-1}}{k} + \underbrace{\frac{n-k}{n}}_{\mathrlap{\rightarrow 1 \text{ for } n\rightarrow \infty}}\cdot\frac{c_k+\cdot+c_{n-1}}{n-k} \right)\\
&=\liminf_{m\rightarrow\infty} \frac{c_k+\dots+c_{k+m-1}}{m}.
\end{align*} A similar argument holds for $\limsup.$

With probability 1, a play has an infinite suffix consisting of plays from $\mdp_Q^{\sigma}$, and thus also $\MeanPayoffInf$ and $\MeanPayoffSup$ determined by this suffix. By \Cref{5-cor:mp-scc-optimality}, these quantities are equal to $r^*$ with probability 1.
\end{proof}

\noindent
The following theorem summarizes the computational aspects.

\begin{theorem}
\label{5-thm:mp-rand-opt-main}
In a strongly connected mean-payoff MDP, one can compute, in polynomial time, a memoryless randomized strategy which is optimal from every vertex, as well as the (single) optimal value of every vertex.
\end{theorem}
\begin{proof}
We obtain, in polynomial time, an optimal solution of $\lpmp$, with the optimal objective value being the optimal value of every vertex (\Cref{5-thm:mp-valcomp}). We then use this optimal solution $\lpsol{x}$ to construct the strategy $\sigma$ and the  Markov chain $\mdp_{\solvset}^{\sigma}$. From this chain we extract a strongly connected subset of vertices $Q$ (in polynomial time, by a simple graph reachability analysis). With the subset in hand, we can construct strategies $\sigma_Q$ and $\sigma_{=1}$, all polynomial-time computations (see \Cref{5-thm:as-char}). These two strategies are then combined to produce the optimal strategy $\sigma^*$.
\end{proof}

\subsection*{Deterministic optimality in strongly connected MDPs}

It remains to prove that we can actually compute a memoryless~\emph{deterministic} strategy that is optimal in every vertex. Looking back at the construction that resulted in \Cref{5-thm:mp-rand-opt-main}, we see that the optimal strategy $\sigma^*$ might be randomized because the computed optimal solution $\lpsol{x}$ of $\lpmp$ can contain two components $(v,a)$, $(v,b)$ with $a\neq b$ and both $\lpsol{x}_{(v,a)}$ and $\lpsol{x}_{(v,b)}$ being positive. To prove memoryless deterministic optimality, we will show that there is always an optimal solution which yields a deterministic strategy, and that such a solution can be obtained in polynomial time.

The previous section implicitly defined two mappings: First, a mapping $\Psi$, which maps every solution $ \vec{x} $ of $\lpmp$ to a memoryless strategy in some sub-MDP of $\mdp$, by putting $\Psi(\vec{x}) = \sigma$ where $\sigma(a\mid v) = \vec{x}_{(v,a)}/\sum_{b\in \actions}\vec{x}_{(v,b)}$. Second, mapping $\Xi$, which maps each memoryless strategy $\sigma$ that induces a strongly connected Markov chain to a solution $\Xi(\sigma)$ of $\lpmp$ such that $\Xi(\sigma)_{(v,a)}=\invdist_v\cdot \sigma(a\mid v)$, where $\invdist$ is the unique invariant distribution of the chain induced by $\sigma$.
\begin{lemma}
\label{5-lem:sol-strat-correspondence}
Let $X$ be the set containing exactly those solutions $\vec{x}$ of $\lpmp$ for which the strategy  $\Psi(\vec{x})$ induces a strongly connected Markov chain. Then the mappings $\Psi$ and $\Xi$ are bijections between $X$ and the set of all memoryless strategies in some sub-MDP of $\mdp$ that induce a strongly connected Markov chain.
\end{lemma}
\begin{proof}
A straightforward computation shows that $\Xi\circ\Psi$ and $\Psi\circ\Xi$ are identity functions on the respective sets.
\end{proof}

\begin{definition}
	\label{5-def:pure-lp}
A solution $\vec{x}$ of $\lpmp$	is \emph{pure} if for every vertex $v$ there is at most one action $a$ such that $\vec{x}_{(v,a)}>0$.
\end{definition}

\noindent
The following lemma follows from the way in which strategies $\sigma$ and $\sigma^*$ were constructed in the previous sub-section.

\begin{lemma}
	\label{5-lem:pure-lpsol}
Let $\lpsol{x}$ be a pure optimal solution of $\lpmp$ and denote $ S = \{v \in \vertices\mid \exists a \text{ s.t. }\lpsol{x}_{(v,a)}>0\} $. Then the strategy $\sigma=\Psi(\lpsol{x})$ is an MD strategy in $\mdp_{\solvset}$. Hence, in such a case, the strategy $\sigma^*$ constructed from $ \sigma $ as in \Cref{5-thm:mp-rand-opt-main} is an optimal MD strategy in $\mdp$.
\end{lemma}

It remains to show how to find a pure optimal solution of $\lpmp$. To this end we exploit some fundamental properties of linear programs.

A linear program is in the \emph{standard} (or equational) form if its set of constraints can be expressed as $A\cdot \vec{x} = \vec{b}$, $\vec{x}\geq 0$, where $\vec{x}$ is a vector of variables, $\vec{b}$ is a non-negative vector, and $A$ is a matrix of an appropriate dimension. In this notation, all the vectors are column vectors, i.e. $A$ has one column per each variable. Note that $\lpmp$ is a program in the standard form. A feasible solution $\vec{x}$ of such a program is \emph{basic} if the columns of $A$ corresponding to variables whose value is positive in $\vec{x}$ form a linearly independent set of vectors. Since the maximal number of linearly independent columns equals the maximal number of linearly independent rows (a number called a \emph{rank} of $A$), we know that each basic feasible solution has at most as many positive entries as there are rows of $A$. 

The next two lemmas prove some fundamental properties of basic feasible solutions.

\begin{lemma}
\label{5-lem:basic-cond-unique}
Assume that a linear program in a standard form has two basic feasible solutions $\vec{x},\vec{x}'$ such that both solutions have the same set of non-zero components, and the cardinality of this set equals the number of equality constraints in the program. Then $\vec{x}=\vec{x}'$.
\end{lemma}
\begin{proof}
Write $A\cdot \vec{x} = \vec{b}$ the equational constraints of the LP.
If $\vec{x}$ is a basic feasible solution, then it solves the equation $A_{N} \cdot \vec{x}_N = \vec{b}$, where $A_N$ ($  N$ stands for `non-zero') is obtained from $A$ by removing all columns corresponding to zero components of $\vec{x}$, and   $\vec{x}_N$ is obtained from $\vec{x}$ by removing all zero components. 

Since $\vec{x}$ has as many non-zero components as there are rows of $A$, it follows that $A_N$ is a square matrix. Since $\vec{x}$ is a basic solution, $A_N$ is regular (its columns are linearly independent) and $\vec{x}=A_{N}^{-1}\cdot \vec{b}$ is uniquely determined by $A_N$. Repeating the same argument for $\vec{x}'$ yields $\vec{x}'=A_{N}^{-1}\cdot \vec{b}= \vec{x}$.
\end{proof}

\begin{lemma}
\label{5-lem:basic-sol}
If a linear program in a standard form has an optimal solution, then it has also a basic optimal solution. Moreover, a basic optimal solution can be found in polynomial time.
\end{lemma}
\begin{proof}[Sketch]
The existence of a basic optimal solution is a well-known linear programming fact, e.g. the standard simplex algorithm works by traversing the set of basic feasible solutions until it finds an optimal one~\cite{Matousek:2007}. For computing an optimal basic solution, we can use one of the polynomial-time interior-point methods for linear programming, such as the path-following method~\cite{Karmarkar:1984,Gonzaga:1992}. While these methods work by traversing the interior of the polyhedron of feasible solutions, they converge, in polynomial time, to a point that is closer to the optimal basic solution than to all the other basic solutions. By a process called \emph{purification,} such a point can be then converted to the closest basic solution, i.e. to the optimal one~\cite{Gonzaga:1992}.
\end{proof}






%\noindent
%Now the linear program $\lpmp$ has $|\vertices|+1$ constraints, so by the above argument its basic solutions have at most $|\vertices|+1$ non-zero entries. To obtain a deterministic strategy, we need to push this bound further.

\begin{theorem}
\label{5-thm:lpmp-basic-dim}
One can find, in polynomial time, an optimal deterministic strategy in a given strongly connected  mean-payoff MDP.
\end{theorem}
\begin{proof}
First, we use~\Cref{5-lem:basic-sol} to find a basic optimal solution $\lpsol{x}$ of $\lpmp$.
We check if it is pure. If yes, we are done. Otherwise,
%
%Assume that there is a basic feasible solution $\lpsol{x}$ of $\lpmp$ that is not pure. Then 
there is $v\in \vertices$ and two distinct actions $a,b$ such that $\lpsol{x}_{(v,a)}>0$ and $\lpsol{x}_{(v,b)}>0.$ Let $ S = \{v \in \vertices\mid \exists a \text{ s.t. }\lpsol{x}_{(v,a)}>0\} $. By~\Cref{5-cor:mp-scc-extraction}, we can partition $\solvset$ into several subsets, each of which induces a strongly connected sub-MDP of $\mdp$. Let $Q$ be a class of this partition containing $v$. We have that the optimal mean-payoff value of every vertex in $\mdp_Q$ is the same as in $\mdp$. This is because, 
as in the beginning of the proof of~\Cref{5-cor:mp-scc-optimality}, we can transform $\lpsol{x}$ into another optimal solution of the same value as $\lpsol{x}$ which has non-zero entries only for components indexed by $(q,a)$ with $q\in Q$. All these computations can be easily implemented in polynomial time. 
%The solution $\tilde{x}$ is still non-pure (since positivity of entries within $W$ has not changed) and basic (since a subset of a linearly independent set of columns is still linearly independent).

We argue that $Q$ is a strict subset of $\vertices$. Indeed, assume that $Q=\vertices$. Then $\lpsol{x}$ induces a randomized strategy $\sigma$ in $\mdp$. Moreover, since $\lpsol{x}$ is a basic solution, it has at most $|\vertices|+1$ positive entries, and since it is non-pure, it must have exactly $n+1$ positive entries, i.e. \Cref{5-lem:basic-cond-unique} is applicable to $\lpsol{x}$, since $\lpmp$ has exactly $|\vertices|+1$ constraints. Now we define a new strategy $\sigma'$ in $\mdp$ by slightly changing the behaviour in $v$. To this end, choose some $\eps>0$ and put $\sigma'(a\mid v)=\sigma(a\mid v)-\eps$ and $\sigma'(b\mid v)=\sigma(b\mid v)+\eps$; we choose $\eps$ small enough so that both quantities are still non-zero. The chain $\mdp^{\sigma'}$ is still strongly connected. Now let $\vec{x}' = \Xi(\sigma')$. Then $\vec{x}'$ is a solution of $\lpmp$ which is still basic, with a set of non-zero components being the same as in $\lpsol{x}$. At the same time, $\vec{x}'\neq \lpsol{x}$, since $\sigma\neq {\sigma'}$ and $\Xi$ is a bijection (\Cref{5-lem:sol-strat-correspondence}). But this is a contradiction with \Cref{5-lem:basic-cond-unique}.

Hence, $\mdp_Q$ is a strict sub-MDP of $\mdp$ in which the value of every vertex is the same as in the original MDP. We can perform a recursive call of the aforementioned computation on $\mdp_Q$ (compute basic optimal solution of $\lpmp$, check purity, possibly extract and recurse on a sub-MDP).
The depth of recursion is bounded by $|\vertices|$, so the running time is polynomial. Since each sub-MDP obtained during the recursion is non-empty, and the size of the MDPs decreases, the recursion must eventually terminate with a basic optimal solution (in some sub-MDP $\mdp'$) that is pure. This yields a memoryless deterministic strategy in $\mdp'$ whose value is equal to the optimal value in $\mdp.$ Such a strategy can be extended to whole $\mdp$ by solving almost sure reachability to $ \mdp' $, as described in the previous sub-section.
%
%As such,  In particular, $\tilde{x}$ has exactly $|W|+1$ positive entries (due to its non-purity), which is the number of constraints in $\lpmp^{W}$. Hence,~\Cref{5-lem:basic-unique} applies to $\tilde{x}$.
%
%Since the solution $\tilde{x}$ is not pure, it induces a randomized strategy $\tilde\sigma$ in $\mdp_W$ (a restriction of $\bar\sigma$ to $\mdp_W$). 
%
%Now we define a new strategy $\sigma$ in $\mdp_W$ by slightly changing the behaviour in $v$. To this end, choose some $\eps>0$ and put $\sigma(a\mid v)=\tilde\sigma(a\mid v)-\eps$ and $\sigma(a\mid v)=\tilde\sigma(a\mid v)+\eps$; we choose $\eps$ small enough so that both quantities are still non-zero. The chain $\mdp^\sigma$ (whose transition matrix we denote $\mcprob$) is still strongly connected. Now let $x = \Xi(\sigma)$. Then $x$ is a solution of $\lpmp$ which is still basic, with a set of non-zero components being the same as in $\lpsol{x}$. At the same time, $x\neq \tilde{x}$, since $\sigma\neq \tilde{\sigma}$ and $\Xi$ is a bijection (\Cref{5-lem:sol-strat-correspondence}). But this is a contradiction, since there cannot be two basic feasible solutions with the same set of non-zero components. 
%Indeed, write $A\cdot x = b$ the equational constraints in $\lpmp$ (i.e., $b$ is a vector with all but last component being zero, the last component equals 1). 
%   
\end{proof}

%\textbf{PETR: Say that $x$ and $\tilde{x}$ have a full column rank in $\mdp_W$.}





	
	






%
%\begin{lemma}
%\label{5-lem:strat-to-lpsol}
%Each strategy $\sigma$ induces a solution $\lpsol{x}$ of $\lpmp$ whose objective value is at least $\expv^\sigma_v[\MeanPayoff].$
%\end{lemma}
%\begin{proof}
%
%\end{proof}



%
%\noindent
%To formally state the connection between the operator $\reachOP$ and the 
%reachability values, we denote by $\DiscountedPayoff^{k}(\play)$ the 
%discounted 
%payoff accumulated during the first $k$ steps of $\play$, i.e. the number 
%$(1-\lambda)\sum_{i=0}^{k-1} \lambda^i
%\, \colouring(\play_i)$. 
%The following 
%lemma can be proved by an easy induction.
%%
%\begin{lemma}
%\label{5-lem:disc-iterates}
%For each $k\geq 0$ and each vertex $v$ it holds that 
%$$\sup_{\sigma}\expv^{\sigma}_{v}[\DiscountedPayoff^{k}] = 
%(\discOP^k(\vec{0}))_v$$ 
%(here $\vec{0}$ is a $|\vertices|$-dimensional vector of zeroes).
%\end{lemma}
%%
%%
%%\noindent
%Now $\DiscountedPayoff(\play) = \lim_{k\rightarrow 
%\infty}\DiscountedPayoff^{k}(\play)$ (for each $\play$) and hence, by 
%dominated 
%convergence theorem it holds that $\expv^\sigma_v[\DiscountedPayoff] = 
%\lim_{k\rightarrow 
%\infty}\expv^\sigma_v[\DiscountedPayoff^{k}]$. (We use the fact that for each 
%$k$ we have $\DiscountedPayoff^{k}(\play)\leq \max_{e\in 
%\edges}\colouring(e).
%$)
%%Now $\Reach(\genColour) = \bigcup_{k=1}^{\infty}\Reach^k(\genColour)$ and 
%Hence 
%%
%\begin{align*}
%\Value(v) &= \sup_{\sigma}\expv^\sigma_v[\DiscountedPayoff] = 
%\sup_{\sigma}\expv^\sigma_v[\lim_{k\rightarrow \infty}\DiscountedPayoff^k]\\
%&= \sup_{\sigma}\lim_{k\rightarrow \infty}\expv^\sigma_v[\DiscountedPayoff^k]
%\\
%&=\sup_{k\geq 1} \sup_{\sigma}\probm^\sigma_v(\Reach^k(\genColour)) 
%=\big(\sup_{k\geq 1} \reachOP^k(\vec{0})\big)_v = \big(\lim_{k\rightarrow 
%\infty} \reachOP^k(\vec{0})\big)_v.
%\end{align*}
%



%\noindent
%Note that the last equality in the first line follows from (XXX - SET UP SOME 
%PROB MEASURE PROPERTIES IN PRELIMS), while the last equality on the second 
%line 
%holds since the sequence 
%$\big(\reachOP(\vec{0})\big)_v,\big(\reachOP^2(\vec{0})\big)_v,\dots$ is 
%non-decreasing. Denoting by $\Value(\mdp)$ the $|\vertices|$-dimensional 
%vector 
%s.t. $\Value(\mdp)_v=\Value(v)$, we get the following.
%
%\begin{theorem}
%\label{5-lem:reach-value-limit}
%We have $\Value(\mdp)=\lim_{k\rightarrow \infty} \reachOP^k(\vec{0})$.
%\end{theorem}
%
%The previous theorem yields a simple algorithm for approximation of 
%reachability values, so called \emph{value iteration.} The algorithm simply 
%consists of iterating $\reachOP$ with initial seed $\vec{0}$, which yields 
%better and better approximation of optimal values of all vertices. A 
%disadvantage of value iteration is that it might not converge to the true 
%reachability values in a finite number of steps, which is the case, e.g. for 
%the MDP in Figure~\cref{xxx}. However,...
%
%% NOTE ON A MORE EFFICIENT COMPUTATION, MAYBE ALSO FOR POS



%enColour)) $.
%\end{theorem}

% U POS TROCHU PREORGANIZOVAT, AT JE TAM HEZKY VIDET NOTION TOHO RANKU










%=\play_1\play_2\cdots \play_k = 
%(v_0,v_1)(v_1,v_2)\cdots(v_{k-1},v_k)$ we put
%\begin{equation}
%\label{5-eq:cylinder-probs}
%\cylProb(\play) = 
%\begin{cases}
%0 & \text{if } v_0 \neq \vinit\\
%\prod_{i=1}^{k} \sum_{a \in \actions}\sigma(a\mid )
%\end{cases}
%\end{equation}
